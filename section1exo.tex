

\subsection{Exercices}

Cette série d'exercices vise à renforcer votre compréhension des concepts fondamentaux du dénombrement et de la probabilité naïve. La difficulté augmente progressivement.

%  Concepts de Base et Probabilité Naïve 

\begin{exercicebox}[Exercice 1 : Univers et Événements]
On lance deux dés à 6 faces, un rouge et un bleu.
\begin{enumerate}
    \item Décrivez l'univers $S$ de cette expérience. Quelle est sa taille $|S|$ ?
    \item Soit $A$ l'événement "la somme des dés est égale à 7". Listez les issues appartenant à $A$. Calculez $P(A)$.
    \item Soit $B$ l'événement "le dé rouge montre un 3". Listez les issues appartenant à $B$. Calculez $P(B)$.
    \item Décrivez l'événement $A \cap B$ et calculez sa probabilité.
\end{enumerate}
\end{exercicebox}

\begin{exercicebox}[Exercice 2 : Tirage de Cartes (Prob. Naïve)]
On tire une carte au hasard d'un jeu standard de 52 cartes.
\begin{enumerate}
    \item Quelle est la probabilité de tirer un Roi ?
    \item Quelle est la probabilité de tirer une carte rouge (Cœur ou Carreau) ?
    \item Quelle est la probabilité de tirer une figure (Valet, Dame, Roi) ?
    \item Quelle est la probabilité de tirer un As rouge ?
\end{enumerate}
\end{exercicebox}

\begin{exercicebox}[Exercice 3 : Urne Simple (Prob. Naïve)]
Une urne contient 5 boules rouges, 3 boules bleues et 2 boules vertes. On tire une boule au hasard.
\begin{enumerate}
    \item Quelle est la probabilité qu'elle soit bleue ?
    \item Quelle est la probabilité qu'elle ne soit pas verte ?
\end{enumerate}
\end{exercicebox}

%  Permutations 

\begin{exercicebox}[Exercice 4 : Anagrammes (Permutation Simple)]
Combien d'anagrammes distinctes peut-on former avec les lettres du mot "MATHS" ?
\end{exercicebox}

\begin{exercicebox}[Exercice 5 : Course (Arrangement)]
Dix athlètes participent à une course. Combien y a-t-il de classements possibles pour les 3 premières places (médaille d'or, d'argent, de bronze) ?
\end{exercicebox}

\begin{exercicebox}[Exercice 6 : Anagrammes (Permutation avec Répétition)]
Combien d'anagrammes distinctes peut-on former avec les lettres du mot "PROBABILITE" ?
\end{exercicebox}

%  Combinaisons 

\begin{exercicebox}[Exercice 7 : Choix d'un Comité (Combinaison)]
Une classe compte 15 étudiants. De combien de manières peut-on choisir un comité de 4 étudiants ?
\end{exercicebox}

\begin{exercicebox}[Exercice 8 : Mains de Poker (Combinaison)]
Dans un jeu de 52 cartes, combien de "mains" de 5 cartes différentes peut-on former ?
\end{exercicebox}

\begin{exercicebox}[Exercice 9 : Comité Mixte (Combinaison)]
À partir d'un groupe de 6 hommes et 4 femmes, combien de comités de 3 personnes peut-on former contenant exactement 2 hommes et 1 femme ?
\end{exercicebox}

\begin{exercicebox}[Exercice 10 : Probabilité avec Combinaisons]
On tire simultanément 3 cartes d'un jeu de 52 cartes. Quelle est la probabilité d'obtenir exactement 2 Rois ?
\end{exercicebox}

%  Combinaisons avec Répétition (Étoiles et Bâtons) 

\begin{exercicebox}[Exercice 11 : Distribution de Bonbons (Étoiles et Bâtons)]
De combien de manières peut-on distribuer 8 bonbons identiques à 3 enfants ? (Certains enfants peuvent ne rien recevoir).
\end{exercicebox}

\begin{exercicebox}[Exercice 12 : Solutions d'Équation (Étoiles et Bâtons)]
Combien y a-t-il de solutions entières non négatives ($x_i \ge 0$) à l'équation $x_1 + x_2 + x_3 + x_4 = 10$ ?
\end{exercicebox}

\begin{exercicebox}[Exercice 13 : Distribution avec Minimum (Étoiles et Bâtons avec Contrainte)]
De combien de manières peut-on distribuer 12 pommes identiques à 4 enfants, si chaque enfant doit recevoir au moins une pomme ?
\end{exercicebox}

%  Principe d'Inclusion-Exclusion 

\begin{exercicebox}[Exercice 14 : Divisibilité (Inclusion-Exclusion 2 Ensembles)]
Parmi les entiers de 1 à 100, combien sont divisibles par 2 OU par 3 ?
\end{exercicebox}

\begin{exercicebox}[Exercice 15 : Langues (Inclusion-Exclusion 2 Ensembles)]
Dans un groupe de 50 étudiants, 30 étudient l'anglais, 25 étudient l'espagnol et 10 étudient les deux langues. Combien d'étudiants étudient au moins une de ces deux langues ? Combien n'en étudient aucune ?
\end{exercicebox}

\begin{exercicebox}[Exercice 16 : Divisibilité (Inclusion-Exclusion 3 Ensembles)]
Parmi les entiers de 1 à 100, combien sont divisibles par 2, 3 OU 5 ?
\end{exercicebox}

%  Problèmes Combinés et Plus Difficiles 

\begin{exercicebox}[Exercice 17 : Chemins sur un Grillage (Combinaison)]
Sur un grillage, combien y a-t-il de chemins pour aller du point (0,0) au point (4,3) en se déplaçant uniquement vers la droite (D) ou vers le haut (H) ?
\end{exercicebox}

\begin{exercicebox}[Exercice 18 : Probabilité Hypergéométrique]
Une urne contient 7 boules blanches et 5 boules noires. On tire successivement et sans remise 4 boules. Quelle est la probabilité d'obtenir 2 blanches et 2 noires ?
\end{exercicebox}

\begin{exercicebox}[Exercice 19 : Arrangement Circulaire]
De combien de manières 6 personnes peuvent-elles s'asseoir autour d'une table ronde ? (Deux arrangements sont considérés identiques si chaque personne a les mêmes voisins).
\end{exercicebox}

\begin{exercicebox}[Exercice 20 : Problème des Dérangements (Inclusion-Exclusion)]
Quatre lettres sont adressées à quatre personnes différentes, avec les enveloppes correspondantes. On met chaque lettre au hasard dans une enveloppe. Quelle est la probabilité qu'\textit{aucune} lettre ne soit dans la bonne enveloppe ?
\end{exercicebox}



\subsection{Corrections des Exercices}

%  Corrections : Concepts de Base et Probabilité Naïve 

\begin{correctionbox}[Correction Exercice 1 : Univers et Événements]
1) L'univers $S$ est l'ensemble de toutes les paires $(r, b)$ où $r$ est le résultat du dé rouge et $b$ celui du dé bleu. $S = \{ (1,1), (1,2), \dots, (1,6), (2,1), \dots, (6,6) \}$. La taille de l'univers est $|S| = 6 \times 6 = 36$.

2) L'événement $A$ (somme égale à 7) est $A = \{ (1,6), (2,5), (3,4), (4,3), (5,2), (6,1) \}$. Il y a $|A|=6$ issues favorables. La probabilité est $P(A) = |A|/|S| = 6/36 = 1/6$.

3) L'événement $B$ (dé rouge montre 3) est $B = \{ (3,1), (3,2), (3,3), (3,4), (3,5), (3,6) \}$. Il y a $|B|=6$ issues favorables. La probabilité est $P(B) = |B|/|S| = 6/36 = 1/6$.

4) L'événement $A \cap B$ est l'ensemble des issues où la somme est 7 ET le dé rouge est 3. La seule issue possible est $(3,4)$. Donc $A \cap B = \{ (3,4) \}$. La probabilité est $P(A \cap B) = |A \cap B|/|S| = 1/36$.
\end{correctionbox}

\begin{correctionbox}[Correction Exercice 2 : Tirage de Cartes (Prob. Naïve)]
Le nombre total d'issues est $|S| = 52$.

a) Il y a 4 Rois. $P(\text{Roi}) = 4/52 = 1/13$.

b) Il y a 26 cartes rouges (13 Cœurs + 13 Carreaux). $P(\text{Rouge}) = 26/52 = 1/2$.

c) Il y a 12 figures (4 Valets + 4 Dames + 4 Rois). $P(\text{Figure}) = 12/52 = 3/13$.

d) Il y a 2 As rouges (As de Cœur, As de Carreau). $P(\text{As Rouge}) = 2/52 = 1/26$.
\end{correctionbox}

\begin{correctionbox}[Correction Exercice 3 : Urne Simple (Prob. Naïve)]
Le nombre total de boules est $5+3+2 = 10$.

a) Il y a 3 boules bleues. $P(\text{Bleue}) = 3/10$.

b) L'événement "ne pas être verte" est le complémentaire de "être verte". Il y a 2 boules vertes, donc $P(\text{Verte}) = 2/10$. La probabilité cherchée est $P(\text{Non Verte}) = 1 - P(\text{Verte}) = 1 - 2/10 = 8/10 = 4/5$. (Alternativement, il y a $5+3=8$ boules non vertes, donc $P=8/10$).
\end{correctionbox}

%  Corrections : Permutations 

\begin{correctionbox}[Correction Exercice 4 : Anagrammes (Permutation Simple)]
Le mot "MATHS" a 5 lettres distinctes. Le nombre d'anagrammes est le nombre de permutations de ces 5 lettres, soit $5! = 5 \times 4 \times 3 \times 2 \times 1 = 120$.
\end{correctionbox}

\begin{correctionbox}[Correction Exercice 5 : Course (Arrangement)]
On cherche le nombre de façons d'ordonner 3 athlètes parmi 10. C'est un arrangement (permutation de $k$ parmi $n$) :
$P(10, 3) = \frac{10!}{(10-3)!} = \frac{10!}{7!} = 10 \times 9 \times 8 = 720$.
Il y a 720 podiums possibles.
\end{correctionbox}

\begin{correctionbox}[Correction Exercice 6 : Anagrammes (Permutation avec Répétition)]
Le mot "PROBABILITE" a 11 lettres. Les répétitions sont : B (2 fois), I (2 fois). Les autres lettres (P, R, O, A, L, T, E) apparaissent une fois.
Le nombre d'anagrammes distinctes est :
$$ \frac{11!}{2! \times 2!} = \frac{39,916,800}{2 \times 2} = \frac{39,916,800}{4} = 9,979,200 $$
\end{correctionbox}

%  Corrections : Combinaisons 

\begin{correctionbox}[Correction Exercice 7 : Choix d'un Comité (Combinaison)]
L'ordre ne compte pas, c'est donc une combinaison de 4 étudiants parmi 15 :
$$ \binom{15}{4} = \frac{15!}{4!(15-4)!} = \frac{15!}{4!11!} = \frac{15 \times 14 \times 13 \times 12}{4 \times 3 \times 2 \times 1} = 15 \times 7 \times 13 \times 1 = 1365 $$
Il y a 1365 comités possibles.
\end{correctionbox}

\begin{correctionbox}[Correction Exercice 8 : Mains de Poker (Combinaison)]
On choisit 5 cartes parmi 52, sans ordre. C'est une combinaison :
$$ \binom{52}{5} = \frac{52!}{5!(52-5)!} = \frac{52!}{5!47!} = \frac{52 \times 51 \times 50 \times 49 \times 48}{5 \times 4 \times 3 \times 2 \times 1} = 2,598,960 $$
Il y a 2,598,960 mains de poker possibles.
\end{correctionbox}

\begin{correctionbox}[Correction Exercice 9 : Comité Mixte (Combinaison)]
Il faut choisir 2 hommes parmi 6 ET 1 femme parmi 4. On multiplie les possibilités pour chaque choix :
Nombre de façons = (choix des hommes) $\times$ (choix des femmes)
$$ = \binom{6}{2} \times \binom{4}{1} = \frac{6 \times 5}{2 \times 1} \times \frac{4}{1} = 15 \times 4 = 60 $$
Il y a 60 comités possibles.
\end{correctionbox}

\begin{correctionbox}[Correction Exercice 10 : Probabilité avec Combinaisons]
L'univers $S$ est l'ensemble de toutes les mains de 3 cartes. $|S| = \binom{52}{3}$.
L'événement $A$ est "obtenir exactement 2 Rois". Pour cela, il faut choisir 2 Rois parmi les 4 Rois ET 1 carte qui n'est pas un Roi parmi les 48 autres cartes.
$|A| = \binom{4}{2} \times \binom{48}{1}$.
La probabilité est $P(A) = \frac{|A|}{|S|} = \frac{\binom{4}{2} \binom{48}{1}}{\binom{52}{3}}$.
$$ P(A) = \frac{\frac{4 \times 3}{2 \times 1} \times 48}{\frac{52 \times 51 \times 50}{3 \times 2 \times 1}} = \frac{6 \times 48}{22100} = \frac{288}{22100} \approx 0.013 $$
\end{correctionbox}

%  Corrections : Combinaisons avec Répétition (Étoiles et Bâtons) 

\begin{correctionbox}[Correction Exercice 11 : Distribution de Bonbons (Étoiles et Bâtons)]
C'est un problème de distribution de $k=8$ objets identiques (bonbons) dans $n=3$ boîtes distinctes (enfants). On utilise la formule $\binom{n+k-1}{k}$.
Nombre de manières = $\binom{3+8-1}{8} = \binom{10}{8} = \binom{10}{2} = \frac{10 \times 9}{2 \times 1} = 45$.
\end{correctionbox}

\begin{correctionbox}[Correction Exercice 12 : Solutions d'Équation (Étoiles et Bâtons)]
Cela revient à distribuer $k=10$ unités identiques dans $n=4$ variables distinctes.
Nombre de solutions = $\binom{n+k-1}{k} = \binom{4+10-1}{10} = \binom{13}{10} = \binom{13}{3} = \frac{13 \times 12 \times 11}{3 \times 2 \times 1} = 286$.
\end{correctionbox}

\begin{correctionbox}[Correction Exercice 13 : Distribution avec Minimum (Étoiles et Bâtons avec Contrainte)]
On doit distribuer $k=12$ pommes à $n=4$ enfants, avec $x_i \ge 1$.
On commence par donner une pomme à chaque enfant. Il reste $12 - 4 = 8$ pommes à distribuer sans contrainte (les $x'_i$ peuvent être nuls).
Le problème devient : distribuer $k'=8$ pommes à $n=4$ enfants.
Nombre de manières = $\binom{n+k'-1}{k'} = \binom{4+8-1}{8} = \binom{11}{8} = \binom{11}{3} = \frac{11 \times 10 \times 9}{3 \times 2 \times 1} = 165$.
\end{correctionbox}

%  Corrections : Principe d'Inclusion-Exclusion 

\begin{correctionbox}[Correction Exercice 14 : Divisibilité (Inclusion-Exclusion 2 Ensembles)]
Soit $A$ l'ensemble des entiers $\le 100$ divisibles par 2, et $B$ l'ensemble des entiers $\le 100$ divisibles par 3. On cherche $|A \cup B|$.
$|A| = \lfloor 100/2 \rfloor = 50$.
$|B| = \lfloor 100/3 \rfloor = 33$.
$|A \cap B|$ = ensemble des entiers divisibles par $2 \times 3 = 6$. $|A \cap B| = \lfloor 100/6 \rfloor = 16$.
Par inclusion-exclusion : $|A \cup B| = |A| + |B| - |A \cap B| = 50 + 33 - 16 = 67$.
\end{correctionbox}

\begin{correctionbox}[Correction Exercice 15 : Langues (Inclusion-Exclusion 2 Ensembles)]
Soit $E$ l'ensemble des étudiants étudiant l'anglais, $S$ l'ensemble de ceux étudiant l'espagnol.
$|E| = 30$, $|S| = 25$, $|E \cap S| = 10$.
Nombre d'étudiants étudiant au moins une langue : $|E \cup S| = |E| + |S| - |E \cap S| = 30 + 25 - 10 = 45$.
Nombre total d'étudiants = 50.
Nombre d'étudiants n'étudiant aucune de ces langues = Total - $|E \cup S| = 50 - 45 = 5$.
\end{correctionbox}

\begin{correctionbox}[Correction Exercice 16 : Divisibilité (Inclusion-Exclusion 3 Ensembles)]
Soit $A_2, A_3, A_5$ les ensembles des entiers $\le 100$ divisibles respectivement par 2, 3, 5. On cherche $|A_2 \cup A_3 \cup A_5|$.
$|A_2|=50$, $|A_3|=33$, $|A_5|=20$.
$|A_2 \cap A_3| = |A_6| = \lfloor 100/6 \rfloor = 16$.
$|A_2 \cap A_5| = |A_{10}| = \lfloor 100/10 \rfloor = 10$.
$|A_3 \cap A_5| = |A_{15}| = \lfloor 100/15 \rfloor = 6$.
$|A_2 \cap A_3 \cap A_5| = |A_{30}| = \lfloor 100/30 \rfloor = 3$.
Par inclusion-exclusion :
$|A_2 \cup A_3 \cup A_5| = (|A_2|+|A_3|+|A_5|) - (|A_2 \cap A_3|+|A_2 \cap A_5|+|A_3 \cap A_5|) + |A_2 \cap A_3 \cap A_5|$
$= (50+33+20) - (16+10+6) + 3 = 103 - 32 + 3 = 74$.
\end{correctionbox}

%  Corrections : Problèmes Combinés et Plus Difficiles 

\begin{correctionbox}[Correction Exercice 17 : Chemins sur un Grillage (Combinaison)]
Pour aller de (0,0) à (4,3), il faut faire un total de $4+3=7$ déplacements. Parmi ces 7 déplacements, il faut choisir les 4 moments où l'on va à droite (les 3 autres seront obligatoirement vers le haut), ou choisir les 3 moments où l'on va vers le haut.
Le nombre de chemins est $\binom{7}{4} = \binom{7}{3} = \frac{7 \times 6 \times 5}{3 \times 2 \times 1} = 35$.
\end{correctionbox}

\begin{correctionbox}[Correction Exercice 18 : Probabilité Hypergéométrique]
C'est un tirage sans remise. On peut utiliser la loi hypergéométrique ou le dénombrement.
Population totale = $7+5=12$ boules. On en tire $m=4$.
On veut $k=2$ blanches (parmi $w=7$) et $m-k=2$ noires (parmi $b=5$).
Probabilité = $\frac{\binom{w}{k} \binom{b}{m-k}}{\binom{w+b}{m}} = \frac{\binom{7}{2} \binom{5}{2}}{\binom{12}{4}}$.
$$ P = \frac{(\frac{7 \times 6}{2}) \times (\frac{5 \times 4}{2})}{(\frac{12 \times 11 \times 10 \times 9}{4 \times 3 \times 2 \times 1})} = \frac{21 \times 10}{495} = \frac{210}{495} = \frac{14}{33} \approx 0.424 $$
\end{correctionbox}

\begin{correctionbox}[Correction Exercice 19 : Arrangement Circulaire]
Pour $n$ objets distincts, le nombre d'arrangements circulaires est $(n-1)!$.
Ici, $n=6$. Le nombre de manières est $(6-1)! = 5! = 120$.
L'idée est de fixer une personne, puis d'arranger les 5 autres par rapport à elle.
\end{correctionbox}

\begin{correctionbox}[Correction Exercice 20 : Problème des Dérangements (Inclusion-Exclusion)]
On cherche le nombre de dérangements de 4 éléments, noté $D_4$ ou $!4$. La probabilité sera $D_4 / 4!$.
La formule générale des dérangements (obtenue par inclusion-exclusion) est $D_n = n! \sum_{i=0}^n \frac{(-1)^i}{i!}$.
Pour $n=4$:
$D_4 = 4! (1/0! - 1/1! + 1/2! - 1/3! + 1/4!)$
$D_4 = 24 (1 - 1 + 1/2 - 1/6 + 1/24)$
$D_4 = 24 (1/2 - 1/6 + 1/24) = 24 (12/24 - 4/24 + 1/24) = 24 (9/24) = 9$.
Il y a 9 dérangements possibles sur un total de $4! = 24$ permutations.
La probabilité est $P(\text{aucun match}) = D_4 / 4! = 9/24 = 3/8 = 0.375$.
\end{correctionbox}

\subsection{Exercices Python}

Les exercices suivants appliquent les concepts de dénombrement et de probabilité au célèbre jeu de données "Titanic". Ce dataset, chargé via la bibliothèque \texttt{seaborn}, contient des informations démographiques et de voyage sur les passagers du navire.

Le bloc de code ci-dessous initialise notre environnement en chargeant les données dans un DataFrame Pandas \texttt{df}. Pour garantir la consistance de nos calculs, nous définirons notre univers $S$ comme l'ensemble des passagers pour lesquels l'âge est connu (en supprimant les lignes avec un âge manquant).

\begin{codecell}
import pandas as pd
import seaborn as sns
import math

# Charger le dataset Titanic
df = sns.load_dataset("titanic")

# On retire les lignes ou l'age est inconnu pour simplifier les calculs
# C'est notre Univers S.
df = df.dropna(subset=["age"]) 

\end{codecell}

\begin{exercicebox}[Exercice 1 : Probabilité Naïve (Filtre multiple)]
Quelle est la probabilité qu'un passager, sélectionné au hasard dans l'univers \texttt{df}, soit un homme de plus de 40 ans ET voyageant en troisième classe ?

\textbf{Votre tâche :}
\begin{enumerate}
    \item Trouver $|S|$, la taille totale de l'univers \texttt{df}.
    \item Trouver $|A|$, le nombre de passagers remplissant les trois conditions (\texttt{sex == 'male'}, \texttt{age > 40}, \texttt{pclass == 3}).
    \item Calculer $P(A) = |A| / |S|$.
\end{enumerate}
\end{exercicebox}

\begin{exercicebox}[Exercice 2 : Dénombrement par Combinaisons ($\binom{n}{k}$)]
Pour une enquête de satisfaction, on veut créer un groupe de discussion (un "comité") composé de 5 personnes. Ces 5 personnes doivent être choisies exclusivement parmi les passagers ayant embarqué à Southampton (\texttt{embark\_town == 'Southampton'}).

Combien de comités uniques de 5 personnes est-il possible de former ?

\textbf{Votre tâche :}
\begin{enumerate}
    \item Trouver $n$, le nombre de passagers ayant embarqué à Southampton.
    \item Définir $k=5$.
    \item Calculer $\binom{n}{k}$ (par ex., avec \texttt{math.comb}).
\end{enumerate}
\end{exercicebox}

\begin{exercicebox}[Exercice 3 : Dénombrement par Permutations]
Lors d'un exercice de sécurité, on demande à 4 enfants (passagers de 12 ans ou moins) de s'aligner pour une photo de communication.

En supposant que l'on choisisse 4 enfants au hasard parmi tous les enfants du navire, et que l'ordre dans lequel ils sont alignés pour la photo est important, combien d'alignements différents sont possibles ?

\textbf{Votre tâche :}
\begin{enumerate}
    \item Trouver $n$, le nombre total d'enfants (âge $\le 12$) à bord.
    \item Définir $k=4$.
    \item Calculer $P(n, k)$ (par ex., avec \texttt{math.perm}).
\end{enumerate}
\end{exercicebox}

\begin{exercicebox}[Exercice 4 : Principe d'Inclusion-Exclusion]
Quelle est la probabilité qu'un passager sélectionné au hasard soit \textbf{soit un survivant} (ensemble $A$), \textbf{soit un passager de première classe} (ensemble $B$) (ou les deux) ?

\textbf{Votre tâche :}
\begin{enumerate}
    \item Trouver $|S|$.
    \item Trouver $|A|$ (nombre de survivants).
    \item Trouver $|B|$ (nombre de passagers en 1ère classe).
    \item Trouver $|A \cap B|$ (survivants de 1ère classe).
    \item Appliquer la formule : $P(A \cup B) = P(A) + P(B) - P(A \cap B)$.
\end{enumerate}
\end{exercicebox}

\begin{exercicebox}[Exercice 5 : Probabilité (Tirage sans remise)]
On sélectionne au hasard un échantillon de 10 passagers de l'univers \texttt{df}.

Quelle est la probabilité que cet échantillon contienne exactement \textbf{4 survivants} et \textbf{6 non-survivants} ?

\textbf{Votre tâche :}
\begin{enumerate}
    \item Trouver $N = |S|$, le nombre total de passagers.
    \item Trouver $m$, le nombre total de survivants dans $S$.
    \item Trouver $p$, le nombre total de non-survivants dans $S$.
    \item Calculer le dénominateur : $\binom{N}{10}$ (façons de choisir 10 passagers).
    \item Calculer le numérateur : $\binom{m}{4} \times \binom{p}{6}$ (façons de choisir 4 survivants ET 6 non-survivants).
    \item Calculer la probabilité (numérateur / dénominateur).
\end{enumerate}
\end{exercicebox}

\begin{exercicebox}[Exercice 6 : Probabilité Conditionnelle]
Calculez la probabilité qu'un passager ait survécu, \textbf{sachant que} ce passager était un homme adulte (\texttt{adult\_male == True}).

\textbf{Votre tâche :}
\begin{enumerate}
    \item Soit $A$ = "a survécu" et $B$ = "est un homme adulte".
    \item Trouver $|B|$, le nombre total d'hommes adultes.
    \item Trouver $|A \cap B|$, le nombre d'hommes adultes qui ont survécu.
    \item Calculer $P(A|B) = \frac{|A \cap B|}{|B|}$.
\end{enumerate}
\end{exercicebox}

\begin{exercicebox}[Exercice 7 : Probabilité Complémentaire]
On sélectionne au hasard un groupe de 5 passagers. Quelle est la probabilité que ce groupe contienne \textbf{au moins un} passager voyageant seul (\texttt{alone == True}) ?

\textbf{Votre tâche :}
\begin{enumerate}
    \item Calculer $P(E^c)$, la probabilité de l'événement complémentaire "aucun passager ne voyage seul".
    \item Trouver $N = |S|$, le nombre total de passagers.
    \item Trouver $n_{\text{non-seul}}$, le nombre de passagers qui ne voyagent *pas* seuls.
    \item Dénominateur $D = \binom{N}{5}$ (façons de choisir 5 passagers).
    \item Numérateur $N_c = \binom{n_{\text{non-seul}}}{5}$ (façons de choisir 5 passagers non-seuls).
    \item $P(E^c) = N_c / D$.
    \item Calculer le résultat final : $P(E) = 1 - P(E^c)$.
\end{enumerate}
\end{exercicebox}

\begin{exercicebox}[Exercice 8 : Probabilité et Analyse de Données]
Calculez la probabilité qu'un passager, choisi au hasard, ait payé un tarif (\texttt{fare}) supérieur au \textbf{tarif moyen} de l'ensemble du navire (\texttt{df}).

\textbf{Votre tâche :}
\begin{enumerate}
    \item Calculer le tarif moyen de tous les passagers dans \texttt{df}.
    \item Trouver $|S|$, la taille totale de l'univers \texttt{df}.
    \item Trouver $|A|$, le nombre de passagers dont le \texttt{fare} est strictement supérieur à ce tarif moyen.
    \item Calculer $P(A) = |A| / |S|$.
\end{enumerate}
\end{exercicebox}

\begin{exercicebox}[Exercice 9 : Probabilité (Tirage multi-groupes)]
On forme un comité spécial de 4 personnes en les tirant au hasard parmi tous les passagers.

Quelle est la probabilité que ce comité soit composé d'exactement \textbf{2 femmes de première classe} et \textbf{2 hommes de troisième classe} ?

\textbf{Votre tâche :}
\begin{enumerate}
    \item Trouver $N = |S|$, le nombre total de passagers.
    \item Trouver $n_1$, le nombre de femmes de 1ère classe.
    \item Trouver $n_2$, le nombre d'hommes de 3ème classe.
    \item Trouver $n_3$, le nombre de "autres" passagers (tous sauf $n_1$ et $n_2$).
    \item Calculer le dénominateur : $\binom{N}{4}$ (façons de choisir 4 passagers).
    \item Calculer le numérateur : $\binom{n_1}{2} \times \binom{n_2}{2} \times \binom{n_3}{0}$.
    \item Calculer la probabilité finale (numérateur / dénominateur).
\end{enumerate}
\end{exercicebox}