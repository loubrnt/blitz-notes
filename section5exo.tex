\subsection{Exercices}

% --- Lois Jointes et Marginales (Discret) ---

\begin{exercicebox}[Exercice 1 : Loi Jointe et Marginales]
Soit le tableau suivant représentant la loi de probabilité jointe $P(X=x, Y=y)$ d'un couple de variables aléatoires $(X, Y)$.

\begin{center}
\begin{tabular}{|c|ccc|}
\hline
\diagbox{$X$}{$Y$} & 0 & 1 & 2 \\ \hline
0 & 0.1 & 0.2 & 0.1 \\
1 & 0.3 & 0.1 & 0.2 \\ \hline
\end{tabular}
\end{center}

\begin{enumerate}
    \item Vérifiez qu'il s'agit bien d'une loi de probabilité.
    \item Calculez la loi marginale de $X$, $P(X=x)$.
    \item Calculez la loi marginale de $Y$, $P(Y=y)$.
\end{enumerate}
\end{exercicebox}

\begin{exercicebox}[Exercice 2 : Calcul de Probabilité Jointe]
En utilisant la loi jointe de l'exercice 1 :
\begin{enumerate}
    \item Calculez $P(X=0, Y \le 1)$.
    \item Calculez $P(X=Y)$.
    \item Calculez $P(X > Y)$.
\end{enumerate}
\end{exercicebox}

\begin{exercicebox}[Exercice 3 : Indépendance (Loi Jointe)]
En utilisant la loi jointe de l'exercice 1 :
\begin{enumerate}
    \item Calculez $P(X=0) \times P(Y=0)$.
    \item Comparez ce résultat à $P(X=0, Y=0)$.
    \item Les variables $X$ et $Y$ sont-elles indépendantes ? Justifiez.
\end{enumerate}
\end{exercicebox}

% --- Espérance, Covariance et Corrélation ---

\begin{exercicebox}[Exercice 4 : Espérances Marginales]
En utilisant les lois marginales calculées à l'exercice 1 :
\begin{enumerate}
    \item Calculez l'espérance $E[X]$.
    \item Calculez l'espérance $E[Y]$.
\end{enumerate}
\end{exercicebox}

\begin{exercicebox}[Exercice 5 : Espérance d'une Fonction (LOTUS)]
En utilisant la loi jointe de l'exercice 1, calculez $E[XY]$.
(Indice : $E[XY] = \sum_x \sum_y (xy) P(X=x, Y=y)$).
\end{exercicebox}

\begin{exercicebox}[Exercice 6 : Covariance (Calcul)]
En utilisant les résultats des exercices 4 et 5, calculez la covariance $\text{Cov}(X,Y)$.
\end{exercicebox}

\begin{exercicebox}[Exercice 7 : Variances Marginales]
En utilisant les lois marginales de l'exercice 1 et les espérances de l'exercice 4 :
\begin{enumerate}
    \item Calculez $E[X^2]$ et $\text{Var}(X)$.
    \item Calculez $E[Y^2]$ et $\text{Var}(Y)$.
\end{enumerate}
\end{exercicebox}

\begin{exercicebox}[Exercice 8 : Corrélation (Calcul)]
En utilisant les résultats des exercices 6 et 7, calculez le coefficient de corrélation $\text{Corr}(X,Y)$.
\end{exercicebox}

% --- Propriétés de la Variance et de la Covariance ---

\begin{exercicebox}[Exercice 9 : Variance d'une Somme (Non Indépendant)]
Soient $X$ et $Y$ deux variables aléatoires telles que $\text{Var}(X) = 10$, $\text{Var}(Y) = 5$ et $\text{Cov}(X,Y) = 2$.
Calculez $\text{Var}(X+Y)$.
\end{exercicebox}

\begin{exercicebox}[Exercice 10 : Variance d'une Différence (Indépendant)]
Soient $X$ et $Y$ deux variables aléatoires \textbf{indépendantes} telles que $\text{Var}(X) = 16$ et $\text{Var}(Y) = 9$.
\begin{enumerate}
    \item Que vaut $\text{Cov}(X,Y)$ ?
    \item Calculez $\text{Var}(X-Y)$. (Rappel : $\text{Var}(X-Y) = \text{Var}(X) + \text{Var}(Y) - 2\text{Cov}(X,Y)$).
\end{enumerate}
\end{exercicebox}

\begin{exercicebox}[Exercice 11 : Bilinéarité de la Covariance]
Soient $X, Y, Z$ trois variables aléatoires. Exprimez $\text{Cov}(X+Y, Z)$ en fonction des covariances des variables individuelles.
\end{exercicebox}

\begin{exercicebox}[Exercice 12 : Variance d'une Combinaison Linéaire]
Soient $X$ et $Y$ deux variables aléatoires indépendantes avec $\text{Var}(X) = 4$ et $\text{Var}(Y) = 2$.
Calculez $\text{Var}(3X - 5Y + 1)$.
\end{exercicebox}

\begin{exercicebox}[Exercice 13 : Variance d'une Somme (Dés)]
On lance deux dés équilibrés $D_1$ et $D_2$. Soit $S = D_1 + D_2$.
On rappelle que pour un dé, $\text{Var}(D_i) = 35/12$.
\begin{enumerate}
    \item Les variables $D_1$ et $D_2$ sont-elles indépendantes ?
    \item Calculez $\text{Var}(S)$.
\end{enumerate}
\end{exercicebox}

\begin{exercicebox}[Exercice 14 : Covariance et Variance]
Soit $X$ une variable aléatoire. En utilisant la bilinéarité de la covariance, montrez que $\text{Cov}(X, X) = \text{Var}(X)$.
\end{exercicebox}

\begin{exercicebox}[Exercice 15 : Covariance avec une Constante]
Soit $X$ une variable aléatoire et $c$ une constante.
Montrez que $\text{Cov}(X, c) = 0$. (Indice : $E[c]=c$ et $E[Xc] = cE[X]$).
\end{exercicebox}

% --- Standardisation et Somme de Poissons ---

\begin{exercicebox}[Exercice 16 : Standardisation (Centrer-Réduire)]
Soit $X$ une variable aléatoire avec $E[X] = 10$ et $\text{Var}(X) = 4$.
Soit $Z = \frac{X - E[X]}{\sqrt{\text{Var}(X)}} = \frac{X - 10}{2}$ la variable standardisée.
\begin{enumerate}
    \item Calculez $E[Z]$.
    \item Calculez $\text{Var}(Z)$.
\end{enumerate}
\end{exercicebox}

\begin{exercicebox}[Exercice 17 : Corrélation et Standardisation]
Soit $\text{Corr}(X,Y) = 0.5$. Soient $Z_X$ et $Z_Y$ les versions standardisées de $X$ et $Y$.
Que vaut $\text{Cov}(Z_X, Z_Y)$ ? (Indice : regardez l'intuition de la corrélation).
\end{exercicebox}

\begin{exercicebox}[Exercice 18 : Somme de Lois de Poisson]
Un magasin reçoit des clients au comptoir A selon $X \sim \text{Poisson}(\lambda_1=5 \text{ clients/heure})$ et au comptoir B selon $Y \sim \text{Poisson}(\lambda_2=3 \text{ clients/heure})$. On suppose que $X$ et $Y$ sont indépendantes.
Soit $S = X+Y$ le nombre total de clients arrivant au magasin en une heure.
\begin{enumerate}
    \item Quelle est la loi de $S$ ? Donnez son nom et son paramètre.
    \item Quelle est la probabilité qu'exactement 6 clients au total arrivent en une heure, $P(S=6)$ ?
\end{enumerate}
\end{exercicebox}

\begin{exercicebox}[Exercice 19 : Corrélation Nulle mais Dépendance]
Soit $X$ une variable aléatoire $X \in \{-1, 0, 1\}$, avec $P(X=-1)=1/3$, $P(X=0)=1/3$, $P(X=1)=1/3$.
Soit $Y = X^2$.
\begin{enumerate}
    \item Calculez $E[X]$.
    \item Calculez $E[XY]$. (Indice : $E[XY] = E[X^3]$).
    \item Calculez $\text{Cov}(X,Y)$.
    \item Les variables $X$ et $Y$ sont-elles indépendantes ?
\end{enumerate}
\end{exercicebox}

\begin{exercicebox}[Exercice 20 : Bornes de la Corrélation]
Soit $X$ une variable aléatoire et $Y = -3X + 5$.
Sans faire de calcul, que vaut $\text{Corr}(X,Y)$ ? Justifiez.
\end{exercicebox}

\subsection{Corrections des Exercices}

% --- Corrections : Lois Jointes et Marginales (Discret) ---

\begin{correctionbox}[Correction Exercice 1 : Loi Jointe et Marginales]
1.  On somme toutes les probabilités du tableau :
    $0.1 + 0.2 + 0.1 + 0.3 + 0.1 + 0.2 = 1.0$.
    Puisque la somme est 1 et toutes les probabilités sont non négatives, c'est une loi valide.

2.  Loi marginale de $X$ (somme des lignes) :
    $P(X=0) = P(X=0, Y=0) + P(X=0, Y=1) + P(X=0, Y=2) = 0.1 + 0.2 + 0.1 = 0.4$.
    $P(X=1) = P(X=1, Y=0) + P(X=1, Y=1) + P(X=1, Y=2) = 0.3 + 0.1 + 0.2 = 0.6$.

3.  Loi marginale de $Y$ (somme des colonnes) :
    $P(Y=0) = P(X=0, Y=0) + P(X=1, Y=0) = 0.1 + 0.3 = 0.4$.
    $P(Y=1) = P(X=0, Y=1) + P(X=1, Y=1) = 0.2 + 0.1 = 0.3$.
    $P(Y=2) = P(X=0, Y=2) + P(X=1, Y=2) = 0.1 + 0.2 = 0.3$.
\end{correctionbox}

\begin{correctionbox}[Correction Exercice 2 : Calcul de Probabilité Jointe]
1.  $P(X=0, Y \le 1) = P(X=0, Y=0) + P(X=0, Y=1) = 0.1 + 0.2 = 0.3$.
2.  $P(X=Y) = P(X=0, Y=0) + P(X=1, Y=1) = 0.1 + 0.1 = 0.2$.
3.  $P(X > Y) = P(X=1, Y=0) = 0.3$. (C'est la seule case où $x > y$).
\end{correctionbox}

\begin{correctionbox}[Correction Exercice 3 : Indépendance (Loi Jointe)]
On utilise les lois marginales de l'exercice 1 : $P(X=0)=0.4$ et $P(Y=0)=0.4$.
1.  $P(X=0) \times P(Y=0) = 0.4 \times 0.4 = 0.16$.
2.  Dans le tableau joint, $P(X=0, Y=0) = 0.1$.
3.  Puisque $P(X=0, Y=0) \neq P(X=0) \times P(Y=0)$ (car $0.1 \neq 0.16$), les variables $X$ et $Y$ \textbf{ne sont pas indépendantes}. (Un seul contre-exemple suffit).
\end{correctionbox}

% --- Corrections : Espérance, Covariance et Corrélation ---

\begin{correctionbox}[Correction Exercice 4 : Espérances Marginales]
1.  $E[X] = \sum_x x P(X=x) = (0)(P(X=0)) + (1)(P(X=1))$
    $E[X] = (0)(0.4) + (1)(0.6) = 0.6$.
2.  $E[Y] = \sum_y y P(Y=y) = (0)(P(Y=0)) + (1)(P(Y=1)) + (2)(P(Y=2))$
    $E[Y] = (0)(0.4) + (1)(0.3) + (2)(0.3) = 0 + 0.3 + 0.6 = 0.9$.
\end{correctionbox}

\begin{correctionbox}[Correction Exercice 5 : Espérance d'une Fonction (LOTUS)]
On somme $(xy)P(X=x, Y=y)$ sur les 6 cases. Les termes où $x=0$ ou $y=0$ sont nuls.
$E[XY] = (0 \cdot 0)(0.1) + (0 \cdot 1)(0.2) + (0 \cdot 2)(0.1) + (1 \cdot 0)(0.3) + (1 \cdot 1)(0.1) + (1 \cdot 2)(0.2)$
$E[XY] = 0 + 0 + 0 + 0 + (1)(0.1) + (2)(0.2) = 0.1 + 0.4 = 0.5$.
\end{correctionbox}

\begin{correctionbox}[Correction Exercice 6 : Covariance (Calcul)]
On utilise la formule $\text{Cov}(X,Y) = E[XY] - E[X]E[Y]$.
$$ \text{Cov}(X,Y) = 0.5 - (0.6)(0.9) = 0.5 - 0.54 = -0.04 $$
\end{correctionbox}

\begin{correctionbox}[Correction Exercice 7 : Variances Marginales]
1.  Pour $X$:
    $E[X^2] = (0^2)(0.4) + (1^2)(0.6) = 0.6$.
    $\text{Var}(X) = E[X^2] - (E[X])^2 = 0.6 - (0.6)^2 = 0.6 - 0.36 = 0.24$.
2.  Pour $Y$:
    $E[Y^2] = (0^2)(0.4) + (1^2)(0.3) + (2^2)(0.3) = 0 + 0.3 + (4)(0.3) = 0.3 + 1.2 = 1.5$.
    $\text{Var}(Y) = E[Y^2] - (E[Y])^2 = 1.5 - (0.9)^2 = 1.5 - 0.81 = 0.69$.
\end{correctionbox}

\begin{correctionbox}[Correction Exercice 8 : Corrélation (Calcul)]
On utilise la formule $\text{Corr}(X,Y) = \frac{\text{Cov}(X,Y)}{\sqrt{\text{Var}(X)\text{Var}(Y)}}$.
$$ \text{Corr}(X,Y) = \frac{-0.04}{\sqrt{0.24 \times 0.69}} = \frac{-0.04}{\sqrt{0.1656}} \approx \frac{-0.04}{0.4069} \approx -0.098 $$
La corrélation est très faible et négative.
\end{correctionbox}

% --- Corrections : Propriétés de la Variance et de la Covariance ---

\begin{correctionbox}[Correction Exercice 9 : Variance d'une Somme (Non Indépendant)]
On utilise la formule générale :
$$ \text{Var}(X+Y) = \text{Var}(X) + \text{Var}(Y) + 2\text{Cov}(X,Y) $$
$$ \text{Var}(X+Y) = 10 + 5 + 2(2) = 15 + 4 = 19 $$
\end{correctionbox}

\begin{correctionbox}[Correction Exercice 10 : Variance d'une Différence (Indépendant)]
1.  Puisque $X$ et $Y$ sont indépendantes, leur covariance est nulle : $\text{Cov}(X,Y) = 0$.
2.  On utilise la formule générale :
    $$ \text{Var}(X-Y) = \text{Var}(X + (-1)Y) = \text{Var}(X) + \text{Var}(-1 \cdot Y) + 2\text{Cov}(X, -Y) $$
    $$ = \text{Var}(X) + (-1)^2 \text{Var}(Y) - 2\text{Cov}(X, Y) $$
    $$ \text{Var}(X-Y) = \text{Var}(X) + \text{Var}(Y) - 2(0) $$
    $$ \text{Var}(X-Y) = 16 + 9 = 25 $$
\end{correctionbox}

\begin{correctionbox}[Correction Exercice 11 : Bilinéarité de la Covariance]
La covariance est linéaire sur son premier argument :
$$ \text{Cov}(X+Y, Z) = \text{Cov}(X, Z) + \text{Cov}(Y, Z) $$
\end{correctionbox}

\begin{correctionbox}[Correction Exercice 12 : Variance d'une Combinaison Linéaire]
On utilise $\text{Var}(aX + bY + c) = a^2 \text{Var}(X) + b^2 \text{Var}(Y) + 2ab\text{Cov}(X,Y)$.
Ici $a=3$, $b=-5$, $c=1$. $X$ et $Y$ sont indépendantes, donc $\text{Cov}(X,Y)=0$.
$$ \text{Var}(3X - 5Y + 1) = (3)^2 \text{Var}(X) + (-5)^2 \text{Var}(Y) + 0 $$
$$ = 9 \times (4) + 25 \times (2) = 36 + 50 = 86 $$
(Note : la constante additive $c=1$ ne change pas la variance).
\end{correctionbox}

\begin{correctionbox}[Correction Exercice 13 : Variance d'une Somme (Dés)]
1.  Oui, les lancers de deux dés standards sont des événements physiquement indépendants.
2.  Puisqu'ils sont indépendants, $\text{Cov}(D_1, D_2) = 0$.
    $$ \text{Var}(S) = \text{Var}(D_1 + D_2) = \text{Var}(D_1) + \text{Var}(D_2) $$
    $$ \text{Var}(S) = \frac{35}{12} + \frac{35}{12} = \frac{70}{12} = \frac{35}{6} $$
\end{correctionbox}

\begin{correctionbox}[Correction Exercice 14 : Covariance et Variance]
Par définition, $\text{Cov}(A, B) = E[(A-\mu_A)(B-\mu_B)]$.
Posons $A=X$ et $B=X$. Alors $\mu_A = \mu_X$ et $\mu_B = \mu_X$.
$$ \text{Cov}(X, X) = E[(X-\mu_X)(X-\mu_X)] = E[(X-\mu_X)^2] $$
C'est la définition de $\text{Var}(X)$.
\end{correctionbox}

\begin{correctionbox}[Correction Exercice 15 : Covariance avec une Constante]
On utilise la formule de calcul $\text{Cov}(X,c) = E[Xc] - E[X]E[c]$.
Par linéarité, $E[Xc] = cE[X]$.
L'espérance d'une constante est la constante elle-même : $E[c] = c$.
$$ \text{Cov}(X, c) = cE[X] - E[X]c = 0 $$
\end{correctionbox}

% --- Corrections : Standardisation et Somme de Poissons ---

\begin{correctionbox}[Correction Exercice 16 : Standardisation (Centrer-Réduire)]
$Z = \frac{X - 10}{2} = \frac{1}{2}X - 5$.
1.  Calcul de $E[Z]$ par linéarité :
    $$ E[Z] = E\left[ \frac{1}{2}X - 5 \right] = \frac{1}{2}E[X] - 5 = \frac{1}{2}(10) - 5 = 5 - 5 = 0 $$
2.  Calcul de $\text{Var}(Z)$ par les propriétés de la variance :
    $$ \text{Var}(Z) = \text{Var}\left( \frac{1}{2}X - 5 \right) = \left(\frac{1}{2}\right)^2 \text{Var}(X) = \frac{1}{4} \text{Var}(X) $$
    $$ \text{Var}(Z) = \frac{1}{4}(4) = 1 $$
    Par définition, une variable standardisée a une moyenne de 0 et une variance de 1.
\end{correctionbox}

\begin{correctionbox}[Correction Exercice 17 : Corrélation et Standardisation]
La corrélation $\text{Corr}(X,Y)$ EST, par définition, la covariance des versions standardisées $Z_X$ et $Z_Y$.
$$ \text{Cov}(Z_X, Z_Y) = \text{Corr}(X,Y) = 0.5 $$
\end{correctionbox}

\begin{correctionbox}[Correction Exercice 18 : Somme de Lois de Poisson]
1.  Puisque $X$ et $Y$ sont des v.a. de Poisson \textbf{indépendantes}, leur somme $S=X+Y$ suit aussi une \textbf{loi de Poisson}.
    Le nouveau paramètre est la somme des paramètres : $\lambda_S = \lambda_1 + \lambda_2 = 5 + 3 = 8$.
    Donc, $S \sim \text{Poisson}(\lambda=8)$.
2.  On cherche $P(S=6)$ pour $S \sim \text{Poisson}(8)$.
    $$ P(S=6) = \frac{e^{-8} 8^6}{6!} = \frac{e^{-8} \times 262144}{720} = 364.08 \times e^{-8} \approx 0.122 $$
\end{correctionbox}

\begin{correctionbox}[Correction Exercice 19 : Corrélation Nulle mais Dépendance]
1.  $E[X] = (-1)(1/3) + (0)(1/3) + (1)(1/3) = -1/3 + 0 + 1/3 = 0$.
2.  $E[XY] = E[X(X^2)] = E[X^3]$.
    $E[X^3] = (-1)^3(1/3) + (0)^3(1/3) + (1)^3(1/3) = (-1)(1/3) + 0 + (1)(1/3) = 0$.
3.  $\text{Cov}(X,Y) = E[XY] - E[X]E[Y] = 0 - (0)E[Y] = 0$.
    Les variables sont \textbf{non corrélées}.
4.  Les variables $X$ et $Y$ sont-elles indépendantes ? Non.
    Test : $P(X=1, Y=0) \stackrel{?}{=} P(X=1)P(Y=0)$.
    - $P(X=1, Y=0) = P(X=1, X^2=0) = 0$.
    - $P(X=1) = 1/3$.
    - $P(Y=0) = P(X^2=0) = P(X=0) = 1/3$.
    - $P(X=1)P(Y=0) = (1/3)(1/3) = 1/9$.
    Puisque $0 \neq 1/9$, elles \textbf{ne sont pas indépendantes}.
    C'est un exemple classique de dépendance non linéaire avec covariance nulle.
\end{correctionbox}

\begin{correctionbox}[Correction Exercice 20 : Bornes de la Corrélation]
$Y$ est une fonction linéaire parfaite de $X$ : $Y = aX + b$ avec $a=-3$ et $b=5$.
La corrélation $\text{Corr}(X,Y)$ mesure la force de la relation \textit{linéaire}. Puisqu'elle est parfaite, la corrélation doit être $\pm 1$.
Le coefficient $a = -3$ est négatif, donc la relation est décroissante.
Par conséquent, $\text{Corr}(X,Y) = -1$.
\end{correctionbox}

\subsection{Exercices Python}

Ces exercices appliquent les concepts de distributions multivariées (covariance, corrélation, variance d'une somme) à des données financières réelles. Nous allons analyser la relation entre les rendements boursiers de deux entreprises technologiques : Google (GOOG) et Microsoft (MSFT).

Nous travaillerons avec les \textbf{rendements journaliers} (variation en pourcentage), qui sont des variables aléatoires continues. L'espérance $E[X]$ sera estimée par la moyenne empirique (\texttt{.mean()}) et la variance $\text{Var}(X)$ par la variance empirique (\texttt{.var()}).

\begin{codecell}
!pip install yfinance
import yfinance as yf
import pandas as pd
import numpy as np

# Definir les tickers et la periode
tickers = ["GOOG", "MSFT"]
start_date = "2020-01-01"
end_date = "2024-12-31"

# Telecharger les prix de cloture ajustes
data = yf.download(tickers, start=start_date, end=end_date)["Adj Close"]

# Calculer les rendements journaliers en pourcentage
returns = data.pct_change().dropna()

# Renommer les colonnes pour plus de clarte
returns.columns = ["GOOG_Return", "MSFT_Return"]

# "returns" est notre DataFrame principal.
# X = returns["GOOG_Return"]
# Y = returns["MSFT_Return"]
\end{codecell}

\begin{exercicebox}[Exercice 1 : Espérances et Variances Marginales]
Soit $X$ la v.a. "Rendement journalier de GOOG" et $Y$ la v.a. "Rendement journalier de MSFT".

\textbf{Votre tâche :}
\begin{enumerate}
    \item Calculer l'espérance empirique $E[X]$ et $E[Y]$. (Que remarquez-vous sur leur ordre de grandeur ?)
    \item Calculer la variance empirique $\text{Var}(X)$ et $\text{Var}(Y)$.
    \item Calculer l'écart-type empirique $\sigma_X$ et $\sigma_Y$. Laquelle des deux actions est la plus "volatile" ?
\end{enumerate}
\end{exercicebox}

\begin{exercicebox}[Exercice 2 : Standardisation (Centrer-Réduire)]
Le concept de variable centrée réduite $Z = \frac{X - \mu_X}{\sigma_X}$ est très utilisé en finance (par ex: "Z-score").

\textbf{Votre tâche :}
\begin{enumerate}
    \item Récupérer $E[X]$ (la moyenne) et $\sigma_X$ (l'écart-type) des rendements de GOOG de l'exercice 1.
    \item Prendre le \textbf{dernier} rendement journalier de GOOG dans le jeu de données.
    \item Calculer le "Z-score" de ce dernier rendement.
    \item Interpréter ce score (par ex: "Le dernier jour, GOOG a performé à X écarts-types de sa moyenne...").
\end{enumerate}
\end{exercicebox}

\begin{exercicebox}[Exercice 3 : Covariance (Calcul via LOTUS)]
Calculez la covariance entre les rendements de GOOG ($X$) et de MSFT ($Y$) en utilisant la formule $\text{Cov}(X,Y) = E[XY] - E[X]E[Y]$.

\textbf{Votre tâche :}
\begin{enumerate}
    \item Récupérer $E[X]$ et $E[Y]$ de l'exercice 1.
    \item Calculer $E[XY]$ en utilisant le "Théorème de Transfert" (LOTUS) sur les données empiriques (Indice : calculez la moyenne de la série $X \times Y$).
    \item Appliquer la formule pour trouver $\text{Cov}(X,Y)$.
    \item Le signe est-il positif ou négatif ? Qu'est-ce que cela implique intuitivement ?
\end{enumerate}
\end{exercicebox}

\begin{exercicebox}[Exercice 4 : Corrélation (Calcul)]
La covariance de l'exercice 3 dépend des unités (rendement au carré). Nous allons la normaliser pour obtenir la corrélation $r \in [-1, 1]$.

\textbf{Votre tâche :}
\begin{enumerate}
    \item Récupérer $\text{Cov}(X,Y)$ (Exercice 3) et $\sigma_X, \sigma_Y$ (Exercice 1).
    \item Appliquer la formule : $\text{Corr}(X,Y) = \frac{\text{Cov}(X,Y)}{\sigma_X \sigma_Y}$.
    \item Interpréter ce coefficient. La relation linéaire entre GOOG et MSFT est-elle forte ou faible ?
\end{enumerate}
\end{exercicebox}

\begin{exercicebox}[Exercice 5 : Linéarité de l'Espérance (Portefeuille)]
Soit un portefeuille $P$ composé à 60\% de GOOG ($X$) et 40\% de MSFT ($Y$).
Le rendement du portefeuille est $P = 0.6X + 0.4Y$.
La théorie dit : $E[P] = E[0.6X + 0.4Y] = 0.6E[X] + 0.4E[Y]$.

\textbf{Votre tâche :}
\begin{enumerate}
    \item En utilisant $E[X]$ et $E[Y]$ de l'exercice 1, calculer l'espérance \textbf{théorique} $E[P]$.
    \item \textbf{Vérification empirique} : 
        \begin{itemize}
            \item Créer la série de données $P_{series} = 0.6 \times X + 0.4 \times Y$.
            \item Calculer l'espérance empirique de $P_{series}$ (\texttt{.mean()}).
        \end{itemize}
    \item Comparer votre résultat théorique (1) et empirique (2).
\end{enumerate}
\end{exercicebox}

\begin{exercicebox}[Exercice 6 : Variance d'un Portefeuille (Variance d'une Somme)]
Continuons avec le portefeuille $P = 0.6X + 0.4Y$.
La variance \textbf{théorique} est : $\text{Var}(P) = a^2\text{Var}(X) + b^2\text{Var}(Y) + 2ab\text{Cov}(X,Y)$.

\textbf{Votre tâche :}
\begin{enumerate}
    \item En utilisant $\text{Var}(X)$, $\text{Var}(Y)$ (Ex 1) et $\text{Cov}(X,Y)$ (Ex 3), calculer $\text{Var}(P)$ en appliquant la formule ci-dessus.
    \item \textbf{Vérification empirique} : 
        \begin{itemize}
            \item Utiliser la série $P_{series}$ de l'exercice 5.
            \item Calculer la variance empirique de $P_{series}$ (\texttt{.var()}).
        \end{itemize}
    \item Comparer votre résultat théorique (1) et empirique (2).
\end{enumerate}
\end{exercicebox}

\begin{exercicebox}[Exercice 7 : Le Bénéfice de la Diversification]
La diversification (le fait que $\text{Corr}(X,Y) \ne 1$) réduit le risque. Nous allons le prouver.
Le risque (l'écart-type) d'un portefeuille \textit{n'est pas} la moyenne pondérée des risques.

\textbf{Votre tâche :}
\begin{enumerate}
    \item Calculer le risque du portefeuille $\sigma_P$ (l'écart-type) en prenant la racine carrée de $\text{Var}(P)$ (calculée à l'exercice 6).
    \item Calculer la "moyenne pondérée des risques" : $\sigma_{moy} = 0.6 \sigma_X + 0.4 \sigma_Y$ (en utilisant $\sigma_X, \sigma_Y$ de l'Ex 1).
    \item Comparer $\sigma_P$ et $\sigma_{moy}$. Lequel est le plus petit ?
    \item (Conclusion) Pourquoi $\sigma_P < \sigma_{moy}$ ? (Indice : $\text{Corr}(X,Y)$).
\end{enumerate}
\end{exercicebox}

\begin{exercicebox}[Exercice 8 : Vérification des Bornes de Corrélation]
Le théorème stipule que si $Y = aX + b$, alors $\text{Corr}(X,Y) = \pm 1$. Vérifions cela.

\textbf{Votre tâche :}
\begin{enumerate}
    \item Soit $X$ la série des rendements de GOOG.
    \item Créer une nouvelle variable $Z = -3X + 0.005$ (une relation linéaire négative parfaite).
    \item Calculer la corrélation empirique entre $X$ et $Z$. (Vous pouvez utiliser \texttt{X.corr(Z)}).
    \item Le résultat est-il conforme au théorème ?
\end{enumerate}
\end{exercicebox}

\begin{exercicebox}[Exercice 9 : Loi Jointe (Discrétisation)]
Transformons nos variables continues $X$ et $Y$ en variables de Bernoulli discrètes.
Soit $X_{bern} = 1$ si le rendement de GOOG est positif ($> 0$), et $0$ sinon.
Soit $Y_{bern} = 1$ si le rendement de MSFT est positif ($> 0$), et $0$ sinon.
Nous voulons trouver la PMF jointe $P(X_{bern}=x, Y_{bern}=y)$.

\textbf{Votre tâche :}
\begin{enumerate}
    \item Créer les deux séries discrètes $X_{bern}$ et $Y_{bern}$.
    \item Utiliser \texttt{pandas.crosstab} pour créer un tableau de contingence (les effectifs).
    \item Normaliser ce tableau par l'effectif total pour obtenir la \textbf{loi jointe} (PMF jointe).
    \item Quelle est la probabilité que les deux actions aient un rendement positif le même jour, $P(X_{bern}=1, Y_{bern}=1)$ ?
\end{enumerate}
\end{exercicebox}

\begin{exercicebox}[Exercice 10 : Lois Marginales et Indépendance (Discret)]
En utilisant la loi jointe $P(X_{bern}, Y_{bern})$ de l'exercice 9.

\textbf{Votre tâche :}
\begin{enumerate}
    \item Calculer la loi marginale $P(X_{bern}=x)$ (somme des lignes).
    \item Calculer la loi marginale $P(Y_{bern}=y)$ (somme des colonnes).
    \item Les variables $X_{bern}$ et $Y_{bern}$ sont-elles indépendantes ? 
    \item Justifiez en comparant $P(X_{bern}=1, Y_{bern}=1)$ au produit $P(X_{bern}=1) \times P(Y_{bern}=1)$.
\end{enumerate}
\end{exercicebox}