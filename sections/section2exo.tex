\subsection{Exercices Python}

Ces exercices appliquent les concepts de probabilité conditionnelle, de la règle de Bayes et de la formule des probabilités totales au jeu de données "Titanic".

\begin{codecell}
import pandas as pd
import seaborn as sns
import math

# Charger le dataset Titanic
df = sns.load_dataset("titanic")

# On retire les lignes ou l'age est inconnu pour simplifier les calculs
# C'est notre Univers S.
df = df.dropna(subset=["age"]) 

\end{codecell}

\begin{exercicebox}[Exercice 1 : Définition de $P(A|B)$]
Calculez la probabilité qu'un passager ait survécu ($A$), \textbf{sachant que} ce passager était en première classe ($B$).

\textbf{Votre tâche :}
\begin{enumerate}
    \item Soit $A$ = "le passager a survécu" (\texttt{survived == 1}).
    \item Soit $B$ = "le passager est en première classe" (\texttt{pclass == 1}).
    \item Trouver $|B|$ (le nombre de passagers en 1ère classe).
    \item Trouver $|A \cap B|$ (le nombre de survivants de 1ère classe).
    \item Calculer $P(A|B) = \frac{|A \cap B|}{|B|}$.
\end{enumerate}
\end{exercicebox}

\begin{exercicebox}[Exercice 2 : Règle du Produit (Tirage sans remise)]
On tire au hasard 2 passagers de l'univers \texttt{df} sans remise. Calculez la probabilité que le premier passager soit un survivant ($A_1$) ET que le second passager soit aussi un survivant ($A_2$).

Utilisez la Règle du Produit : $P(A_1 \cap A_2) = P(A_1)P(A_2|A_1)$.

\textbf{Votre tâche :}
\begin{enumerate}
    \item Trouver $|S|$ (total passagers) et $|A_1|$ (total survivants).
    \item Calculer $P(A_1) = |A_1| / |S|$.
    \item Calculer $P(A_2|A_1)$. (Indice : après avoir tiré un survivant, combien de passagers restent ? Combien de survivants restent ?).
    \item Calculer le produit $P(A_1) \times P(A_2|A_1)$.
\end{enumerate}
\end{exercicebox}

\begin{exercicebox}[Exercice 3 : Formule des Probabilités Totales]
Calculez la probabilité totale qu'un passager ait survécu ($B$) en utilisant la formule des probabilités totales. Utilisez la partition des trois classes de passagers ($A_1$=1ère, $A_2$=2e, $A_3$=3e classe).

La formule est : $P(B) = \sum_{i=1}^{3} P(B|A_i)P(A_i)$.

\textbf{Votre tâche :}
\begin{enumerate}
    \item Pour $i=1, 2, 3$, calculer $P(A_i)$, la probabilité d'appartenir à chaque classe (ex: $P(A_1) = |\text{pclass 1}| / |S|$).
    \item Pour $i=1, 2, 3$, calculer $P(B|A_i)$, la probabilité de survie sachant la classe (cf. Exercice 1).
    \item Appliquer la formule : $P(B) = P(B|A_1)P(A_1) + P(B|A_2)P(A_2) + P(B|A_3)P(A_3)$.
    \item (Vérification) Comparez votre résultat au calcul direct $P(B) = |\text{survivants}| / |S|$.
\end{enumerate}
\end{exercicebox}

\begin{exercicebox}[Exercice 4 : Règle de Bayes]
En utilisant les résultats de l'exercice précédent, appliquez la Règle de Bayes.

On observe qu'un passager a survécu ($B$). Quelle est la probabilité qu'il s'agisse d'un passager de première classe ($A_1$) ?

\textbf{Votre tâche :}
\begin{enumerate}
    \item On cherche $P(A_1|B) = \frac{P(B|A_1)P(A_1)}{P(B)}$.
    \item Récupérer $P(B|A_1)$ (la probabilité de survie en 1ère classe) de l'exercice 3.
    \item Récupérer $P(A_1)$ (la probabilité d'être en 1ère classe) de l'exercice 3.
    \item Récupérer $P(B)$ (la probabilité totale de survie) de l'exercice 3.
    \item Effectuer le calcul.
\end{enumerate}
\end{exercicebox}

\begin{exercicebox}[Exercice 5 : Indépendance de deux événements]
Les événements $A$ = "être une femme" (\texttt{sex == 'female'}) et $B$ = "avoir survécu" (\texttt{survived == 1}) sont-ils indépendants ?

\textbf{Votre tâche :}
\begin{enumerate}
    \item Prouver ou réfuter l'indépendance en vérifiant si $P(A \cap B) = P(A)P(B)$.
    \item Calculer $P(A) = |\text{femmes}| / |S|$.
    \item Calculer $P(B) = |\text{survivants}| / |S|$.
    \item Calculer $P(A \cap B) = |\text{femmes survivantes}| / |S|$.
    \item Comparer $P(A \cap B)$ au produit $P(A) \times P(B)$.
    \item (Alternative) Comparer $P(B|A)$ à $P(B)$. L'information "être une femme" change-t-elle la probabilité de survie ?
\end{enumerate}
\end{exercicebox}