\subsection{Exercices}

\textit{Pour tous les exercices de calcul, vous pouvez utiliser les valeurs suivantes pour la fonction de répartition de la loi normale standard $\Phi(z) = P(Z \le z)$ :}
\begin{itemize}
    \item $\Phi(0) = 0.5$
    \item $\Phi(0.675) \approx 0.75$
    \item $\Phi(1) \approx 0.8413$
    \item $\Phi(1.28) \approx 0.90$
    \item $\Phi(1.5) \approx 0.9332$
    \item $\Phi(1.96) \approx 0.975$
    \item $\Phi(2) \approx 0.9772$
    \item $\Phi(2.5) \approx 0.9938$
    \item $\Phi(3) \approx 0.9987$
\end{itemize}
\textit{Et rappelez-vous la propriété de symétrie : $\Phi(-z) = 1 - \Phi(z)$.}

% --- Section 1 : Concepts, Z-Scores & Règle Empirique ---

\begin{exercicebox}[Exercice 1 : Concepts (PDF)]
Soit deux lois normales $A \sim \mathcal{N}(10, 4)$ et $B \sim \mathcal{N}(10, 9)$.
\begin{enumerate}
    \item Laquelle des deux distributions a le pic le plus élevé ?
    \item Laquelle des deux distributions est la plus "aplatie" ?
\end{enumerate}
\end{exercicebox}

\begin{exercicebox}[Exercice 2 : Z-Score (Calcul)]
La taille des étudiants suit $X \sim \mathcal{N}(175, 6^2)$, où les unités sont en cm. Un étudiant mesure 184 cm. Calculez son Z-score.
\end{exercicebox}

\begin{exercicebox}[Exercice 3 : Z-Score (Interprétation)]
Un score de $Z = -2.5$ est obtenu pour une variable $X$. Qu'est-ce que cela signifie en termes de moyenne ($\mu$) et d'écart-type ($\sigma$) ?
\end{exercicebox}

\begin{exercicebox}[Exercice 4 : Z-Score (Calcul Inverse)]
Pour une distribution $\mathcal{N}(50, 100)$, quelle valeur $x$ correspond à un Z-score de $z = 1.5$ ?
\end{exercicebox}

\begin{exercicebox}[Exercice 5 : Z-Score (Comparaison)]
Alice obtient 115 à un test de QI $\mathcal{N}(100, 15^2)$. Bob obtient 24 à un test d'aptitude $\mathcal{N}(20, 2^2)$. Qui a le mieux réussi par rapport à son groupe ?
\end{exercicebox}

\begin{exercicebox}[Exercice 6 : Règle Empirique (68-95-99.7)]
Le poids de paquets de café suit $\mathcal{N}(500g, 10^2)$.
En utilisant la règle empirique, quel pourcentage approximatif de paquets pèse entre 490g et 510g ?
\end{exercicebox}

\begin{exercicebox}[Exercice 7 : Règle Empirique (Queue)]
En utilisant la règle empirique pour $\mathcal{N}(\mu, \sigma^2)$, quelle est la probabilité approximative $P(X > \mu + 2\sigma)$ ?
\end{exercicebox}

% --- Section 2 : Calculs de Probabilités (Standardisation) ---

\begin{exercicebox}[Exercice 8 : Loi Standard (Lecture Directe)]
Soit $Z \sim \mathcal{N}(0, 1)$. Calculez $P(Z \le 1.5)$.
\end{exercicebox}

\begin{exercicebox}[Exercice 9 : Loi Standard (Queue Droite)]
Soit $Z \sim \mathcal{N}(0, 1)$. Calculez $P(Z > 1)$.
\end{exercicebox}

\begin{exercicebox}[Exercice 10 : Loi Standard (Queue Gauche)]
Soit $Z \sim \mathcal{N}(0, 1)$. Calculez $P(Z \le -2)$.
\end{exercicebox}

\begin{exercicebox}[Exercice 11 : Loi Standard (Intervalle)]
Soit $Z \sim \mathcal{N}(0, 1)$. Calculez $P(-1 \le Z \le 2)$.
\end{exercicebox}

\begin{exercicebox}[Exercice 12 : Calcul (Standardisation)]
Soit $X \sim \mathcal{N}(50, 25)$. (Attention : $\sigma^2=25$).
Calculez $P(X \le 60)$.
\end{exercicebox}

\begin{exercicebox}[Exercice 13 : Calcul (Standardisation)]
Soit $X \sim \mathcal{N}(100, 225)$. (Attention : $\sigma^2=225$).
Calculez $P(X > 77.5)$.
\end{exercicebox}

\begin{exercicebox}[Exercice 14 : Calcul (Intervalle)]
La durée de vie d'une batterie suit $X \sim \mathcal{N}(40 \text{ heures}, 16)$.
Calculez $P(38 \le X \le 42)$.
\end{exercicebox}

\begin{exercicebox}[Exercice 15 : Propriété de la CDF]
En utilisant les définitions, montrez que $P(Z > z) = P(Z \le -z)$.
\end{exercicebox}

% --- Section 3 : Problèmes Inverses ---

\begin{exercicebox}[Exercice 16 : Inverse (Z-Score)]
Soit $Z \sim \mathcal{N}(0, 1)$. Trouvez la valeur $z$ telle que $P(Z \le z) = 0.975$.
\end{exercicebox}

\begin{exercicebox}[Exercice 17 : Inverse (Z-Score Queue Droite)]
Soit $Z \sim \mathcal{N}(0, 1)$. Trouvez la valeur $z$ telle que $P(Z > z) = 0.1587$.
(Indice : $1 - 0.1587 = 0.8413$).
\end{exercicebox}

\begin{exercicebox}[Exercice 18 : Inverse (Z-Score Intervalle Central)]
Soit $Z \sim \mathcal{N}(0, 1)$. Trouvez la valeur $z$ (positive) telle que $P(-z \le Z \le z) = 0.95$.
(Indice : si l'aire centrale est 0.95, combien vaut l'aire à gauche de $z$ ?)
\end{exercicebox}

\begin{exercicebox}[Exercice 19 : Problème Inverse (Application)]
Le score à un examen suit $\mathcal{N}(100, 15^2)$. Pour obtenir la note "A", un étudiant doit être dans le top 2.5\% (c'est-à-dire $P(X > x) = 0.025$).
Quel score $x$ minimum faut-il obtenir ?
\end{exercicebox}

\begin{exercicebox}[Exercice 20 : Problème Inverse (Application)]
Une machine remplit des sacs de farine $\mathcal{N}(1000g, 20^2)$. On veut garantir que 99.87\% des sacs pèsent *plus* qu'un certain poids $x$.
Quelle est la valeur de $x$ ?
\end{exercicebox}

% --- Section 4 : Propriétés (Stabilité, Sommes) ---

\begin{exercicebox}[Exercice 21 : Stabilité par Transformation Linéaire]
Si $X \sim \mathcal{N}(10, 4)$ (donc $\sigma=2$), quelle est la loi de la variable $Y = 3X + 5$ ?
\end{exercicebox}

\begin{exercicebox}[Exercice 22 : Stabilité (Changement d'Unités)]
La taille $T_{cm}$ d'une population suit $\mathcal{N}(170, 100)$. Quelle est la loi de la taille $T_{m}$ en mètres ? (Rappel : $T_{m} = T_{cm} / 100$).
\end{exercicebox}

\begin{exercicebox}[Exercice 23 : Stabilité par Addition (Indépendante)]
Soit $X \sim \mathcal{N}(50, 10)$ et $Y \sim \mathcal{N}(30, 6)$ deux variables indépendantes.
Quelle est la loi de la somme $S = X + Y$ ?
\end{exercicebox}

\begin{exercicebox}[Exercice 24 : Stabilité par Différence (Indépendante)]
En utilisant les variables $X$ et $Y$ de l'exercice 23, quelle est la loi de la différence $D = X - Y$ ?
(Indice : $D = X + (-1)Y$).
\end{exercicebox}

\begin{exercicebox}[Exercice 25 : Somme de $n$ variables i.i.d.]
On prélève 10 pommes d'un lot où le poids d'une pomme suit $\mathcal{N}(150g, 10^2)$. Soit $P_{total}$ le poids total des 10 pommes (supposées indépendantes).
Quelle est la loi de $P_{total}$ ?
\end{exercicebox}

\subsection{Corrections des Exercices}

\begin{correctionbox}[Correction Exercice 1 : Concepts (PDF)]
1.  La variance de A ($\sigma_A^2=4$) est plus petite que celle de B ($\sigma_B^2=9$). Une variance plus petite signifie une distribution plus concentrée, donc un pic plus élevé (l'aire totale devant rester 1). C'est la distribution A.
2.  La distribution B ($\sigma_B=3$) a un écart-type plus grand que A ($\sigma_A=2$), elle est donc plus dispersée, c'est-à-dire plus "large et aplatie".
\end{correctionbox}

\begin{correctionbox}[Correction Exercice 2 : Z-Score (Calcul)]
On a $x=184$, $\mu=175$, $\sigma=6$.
$$Z = \frac{x - \mu}{\sigma} = \frac{184 - 175}{6} = \frac{9}{6} = 1.5$$
L'étudiant se situe à 1.5 écarts-types au-dessus de la moyenne.
\end{correctionbox}

\begin{correctionbox}[Correction Exercice 3 : Z-Score (Interprétation)]
Cela signifie que la valeur $X$ observée est $2.5$ écarts-types \textbf{en dessous} de la moyenne $\mu$. (c-à-d, $x = \mu - 2.5\sigma$).
\end{correctionbox}

\begin{correctionbox}[Correction Exercice 4 : Z-Score (Calcul Inverse)]
On a $\mu=50$ et $\sigma^2=100 \implies \sigma=10$. On cherche $x$.
$z = \frac{x - \mu}{\sigma} \implies x = \mu + z\sigma$
$x = 50 + (1.5)(10) = 50 + 15 = 65$.
\end{correctionbox}

\begin{correctionbox}[Correction Exercice 5 : Z-Score (Comparaison)]
On compare les Z-scores :
$Z_{Alice} = \frac{115 - 100}{15} = \frac{15}{15} = 1.0$.
$Z_{Bob} = \frac{24 - 20}{2} = \frac{4}{2} = 2.0$.
Bob a un Z-score plus élevé (2.0 contre 1.0), il a donc mieux réussi par rapport à son groupe.
\end{correctionbox}

\begin{correctionbox}[Correction Exercice 6 : Règle Empirique (68-95-99.7)]
On a $\mu=500$ et $\sigma=10$. L'intervalle [490, 510] correspond à $[\mu - 1\sigma, \mu + 1\sigma]$.
Selon la règle empirique, environ \textbf{68\%} des paquets se trouvent dans cet intervalle.
\end{correctionbox}

\begin{correctionbox}[Correction Exercice 7 : Règle Empirique (Queue)]
La règle dit que $P(\mu - 2\sigma \le X \le \mu + 2\sigma) \approx 0.95$.
L'aire totale est 1. L'aire en dehors de cet intervalle est $1 - 0.95 = 0.05$.
En raison de la symétrie, cette aire de 0.05 est répartie également entre les deux queues (gauche et droite).
L'aire de la queue droite $P(X > \mu + 2\sigma)$ est donc $0.05 / 2 = 0.025$ (soit 2.5\%).
\end{correctionbox}

\begin{correctionbox}[Correction Exercice 8 : Loi Standard (Lecture Directe)]
$P(Z \le 1.5) = \Phi(1.5)$. En utilisant la table fournie, $P(Z \le 1.5) \approx 0.9332$.
\end{correctionbox}

\begin{correctionbox}[Correction Exercice 9 : Loi Standard (Queue Droite)]
$P(Z > 1) = 1 - P(Z \le 1) = 1 - \Phi(1)$.
$P(Z > 1) \approx 1 - 0.8413 = 0.1587$.
\end{correctionbox}

\begin{correctionbox}[Correction Exercice 10 : Loi Standard (Queue Gauche)]
$P(Z \le -2) = \Phi(-2)$. Par symétrie, $\Phi(-2) = 1 - \Phi(2)$.
$P(Z \le -2) \approx 1 - 0.9772 = 0.0228$.
\end{correctionbox}

\begin{correctionbox}[Correction Exercice 11 : Loi Standard (Intervalle)]
$P(-1 \le Z \le 2) = P(Z \le 2) - P(Z \le -1) = \Phi(2) - \Phi(-1)$.
$\Phi(-1) = 1 - \Phi(1) \approx 1 - 0.8413 = 0.1587$.
$P(-1 \le Z \le 2) \approx 0.9772 - 0.1587 = 0.8185$.
\end{correctionbox}

\begin{correctionbox}[Correction Exercice 12 : Calcul (Standardisation)]
$X \sim \mathcal{N}(50, 25) \implies \mu=50, \sigma=5$.
On cherche $P(X \le 60)$.
$Z = \frac{60 - 50}{5} = \frac{10}{5} = 2$.
$P(X \le 60) = P(Z \le 2) = \Phi(2) \approx 0.9772$.
\end{correctionbox}

\begin{correctionbox}[Correction Exercice 13 : Calcul (Standardisation)]
$X \sim \mathcal{N}(100, 225) \implies \mu=100, \sigma=15$.
On cherche $P(X > 77.5)$.
$Z = \frac{77.5 - 100}{15} = \frac{-22.5}{15} = -1.5$.
$P(X > 77.5) = P(Z > -1.5) = 1 - P(Z \le -1.5) = 1 - \Phi(-1.5)$.
$\Phi(-1.5) = 1 - \Phi(1.5) \approx 1 - 0.9332 = 0.0668$.
$P(Z > -1.5) = 1 - 0.0668 = 0.9332$. (Logique : $P(Z > -1.5) = P(Z \le 1.5)$ par symétrie).
\end{correctionbox}

\begin{correctionbox}[Correction Exercice 14 : Calcul (Intervalle)]
$X \sim \mathcal{N}(40, 16) \implies \mu=40, \sigma=4$.
On cherche $P(38 \le X \le 42)$.
$z_1 = \frac{38 - 40}{4} = -0.5$. $z_2 = \frac{42 - 40}{4} = 0.5$.
$P(38 \le X \le 42) = P(-0.5 \le Z \le 0.5) = \Phi(0.5) - \Phi(-0.5)$.
$\Phi(-0.5) = 1 - \Phi(0.5)$.
$P = \Phi(0.5) - (1 - \Phi(0.5)) = 2\Phi(0.5) - 1$.
(La valeur $\Phi(0.5) \approx 0.6915$ n'est pas fournie, mais $P(-1 \le Z \le 1) \approx 0.68$, donc on s'attend à une valeur plus petite).
\end{correctionbox}

\begin{correctionbox}[Correction Exercice 15 : Propriété de la CDF]
$P(Z > z) = 1 - P(Z \le z) = 1 - \Phi(z)$.
$P(Z \le -z) = \Phi(-z)$.
Par symétrie de la PDF $\phi(z) = \phi(-z)$, l'aire à droite de $z$ est égale à l'aire à gauche de $-z$.
Donc $P(Z > z) = P(Z \le -z)$, ce qui implique $1 - \Phi(z) = \Phi(-z)$.
\end{correctionbox}

\begin{correctionbox}[Correction Exercice 16 : Inverse (Z-Score)]
On cherche $z$ tel que $P(Z \le z) = 0.975$.
C'est $\Phi(z) = 0.975$. D'après la table, $z = 1.96$.
\end{correctionbox}

\begin{correctionbox}[Correction Exercice 17 : Inverse (Z-Score Queue Droite)]
On cherche $z$ tel que $P(Z > z) = 0.1587$.
Cela signifie $P(Z \le z) = 1 - 0.1587 = 0.8413$.
$\Phi(z) = 0.8413$. D'après la table, $z = 1$.
\end{correctionbox}

\begin{correctionbox}[Correction Exercice 18 : Inverse (Z-Score Intervalle Central)]
Si $P(-z \le Z \le z) = 0.95$, l'aire restante dans les deux queues est $1 - 0.95 = 0.05$.
Par symétrie, l'aire dans la queue gauche $P(Z < -z)$ est $0.05 / 2 = 0.025$.
L'aire totale à gauche de $z$ est $P(Z \le z) = 0.95 + 0.025 = 0.975$.
On cherche $z$ tel que $\Phi(z) = 0.975$.
D'après la table, $z = 1.96$. (On retrouve $\mu \pm 1.96\sigma$ comme l'intervalle à 95\% exact).
\end{correctionbox}

\begin{correctionbox}[Correction Exercice 19 : Problème Inverse (Application)]
$X \sim \mathcal{N}(100, 15^2)$. On cherche $x$ tel que $P(X > x) = 0.025$.
Standardisation : $P(Z > z) = 0.025$, où $z = (x-100)/15$.
$P(Z \le z) = 1 - 0.025 = 0.975$.
D'après la table, $z$ tel que $\Phi(z)=0.975$ est $z=1.96$.
On résout : $1.96 = \frac{x - 100}{15} \implies x = 100 + 1.96(15) = 100 + 29.4 = 129.4$.
Il faut un score minimum de 129.4.
\end{correctionbox}

\begin{correctionbox}[Correction Exercice 20 : Problème Inverse (Application)]
$X \sim \mathcal{N}(1000, 20^2)$. On cherche $x$ tel que $P(X > x) = 0.9987$.
Standardisation : $P(Z > z) = 0.9987$, où $z = (x-1000)/20$.
$P(Z \le z) = 1 - 0.9987 = 0.0013$.
C'est une valeur très faible. Utilisons la symétrie : $\Phi(-z) = 1 - \Phi(z)$.
$\Phi(z) = 0.0013$. On cherche $z$ dans la table.
On voit que $\Phi(3) = 0.9987$, donc $\Phi(-3) = 1 - 0.9987 = 0.0013$.
Le Z-score est $z = -3$.
On résout : $-3 = \frac{x - 1000}{20} \implies x = 1000 - 3(20) = 1000 - 60 = 940$.
Le poids garanti est 940g.
\end{correctionbox}

\begin{correctionbox}[Correction Exercice 21 : Stabilité par Transformation Linéaire]
$X \sim \mathcal{N}(10, 4) \implies \mu_X=10, \sigma_X^2=4$. $Y = aX + b$ avec $a=3, b=5$.
$Y$ suit une loi normale.
$E[Y] = a\mu_X + b = 3(10) + 5 = 35$.
$\text{Var}(Y) = a^2 \text{Var}(X) = 3^2 \times 4 = 9 \times 4 = 36$.
Donc, $Y \sim \mathcal{N}(35, 36)$.
\end{correctionbox}

\begin{correctionbox}[Correction Exercice 22 : Stabilité (Changement d'Unités)]
$T_{cm} \sim \mathcal{N}(170, 100) \implies \mu=170, \sigma^2=100$.
$T_m = a T_{cm} + b$ avec $a=1/100 = 0.01$ et $b=0$.
$E[T_m] = a\mu + b = 0.01(170) + 0 = 1.7$.
$\text{Var}(T_m) = a^2 \text{Var}(T_{cm}) = (0.01)^2 \times 100 = 0.0001 \times 100 = 0.01$.
Donc, $T_m \sim \mathcal{N}(1.7, 0.01)$. (L'écart-type est $\sigma_m = \sqrt{0.01} = 0.1$ m, ce qui est logique : 10cm = 0.1m).
\end{correctionbox}

\begin{correctionbox}[Correction Exercice 23 : Stabilité par Addition (Indépendante)]
$X \sim \mathcal{N}(50, 10)$, $Y \sim \mathcal{N}(30, 6)$. $X, Y$ indépendantes.
$S = X + Y$ suit une loi normale.
$E[S] = E[X] + E[Y] = 50 + 30 = 80$.
$\text{Var}(S) = \text{Var}(X) + \text{Var}(Y) = 10 + 6 = 16$.
Donc, $S \sim \mathcal{N}(80, 16)$.
\end{correctionbox}

\begin{correctionbox}[Correction Exercice 24 : Stabilité par Différence (Indépendante)]
$D = X - Y = X + (-1)Y$. C'est une somme de variables normales indépendantes (si $X, Y$ le sont, $X, -Y$ le sont aussi). $D$ suit une loi normale.
$E[D] = E[X] + E[-Y] = E[X] - E[Y] = 50 - 30 = 20$.
$\text{Var}(D) = \text{Var}(X) + \text{Var}(-1 \cdot Y) = \text{Var}(X) + (-1)^2 \text{Var}(Y)$
$\text{Var}(D) = \text{Var}(X) + \text{Var}(Y) = 10 + 6 = 16$.
Donc, $D \sim \mathcal{N}(20, 16)$. (Note : les variances s'ajoutent toujours !)
\end{correctionbox}

\begin{correctionbox}[Correction Exercice 25 : Somme de $n$ variables i.i.d.]
Soit $P_i \sim \mathcal{N}(150, 100)$ le poids de la $i$-ème pomme.
$P_{total} = P_1 + \dots + P_{10}$. C'est une somme de 10 v.a. normales indépendantes.
$P_{total}$ suit une loi normale.
$E[P_{total}] = E[P_1] + \dots + E[P_{10}] = 10 \times E[P_1] = 10 \times 150 = 1500g$.
$\text{Var}(P_{total}) = \text{Var}(P_1) + \dots + \text{Var}(P_{10}) = 10 \times \text{Var}(P_1) = 10 \times 100 = 1000$.
Donc, $P_{total} \sim \mathcal{N}(1500, 1000)$.
\end{correctionbox}

\subsection{Exercices Python}

Ces exercices appliquent les concepts de la loi normale au jeu de données "Yahoo Finance". Nous allons modéliser les \textbf{rendements journaliers} (variation en pourcentage) des actions, qui sont souvent (par approximation) considérés comme suivant une loi normale.

Nous allons travailler avec les rendements de Google ($X$) et de Microsoft ($Y$).

\begin{codecell}
!pip install yfinance
import yfinance as yf
import pandas as pd
import numpy as np
from scipy.stats import norm # Moteur pour les calculs de CDF/PDF

# Definir les tickers et la periode
tickers = ["GOOG", "MSFT"]
start_date = "2020-01-01"
end_date = "2024-12-31"

# Telecharger les prix de cloture ajustes
data = yf.download(tickers, start=start_date, end=end_date)["Adj Close"]

# Calculer les rendements journaliers en pourcentage
returns = data.pct_change().dropna()

# Renommer les colonnes pour plus de clarte
returns.columns = ["GOOG_Return", "MSFT_Return"]

# 'returns' est notre DataFrame principal.
# X = returns["GOOG_Return"]
# Y = returns["MSFT_Return"]
\end{codecell}

\begin{exercicebox}[Exercice 1 : Estimation des Paramètres $\mu$ et $\sigma^2$]
Soit $X$ la v.a. "Rendement journalier de GOOG" et $Y$ la v.a. "Rendement journalier de MSFT".
Nous supposons $X \sim \mathcal{N}(\mu_X, \sigma_X^2)$ et $Y \sim \mathcal{N}(\mu_Y, \sigma_Y^2)$.

\textbf{Votre tâche :}
\begin{enumerate}
    \item Estimer $\mu_X$ et $\mu_Y$ (les espérances) en calculant la moyenne empirique (\texttt{.mean()}) des deux séries.
    \item Estimer $\sigma_X^2$ et $\sigma_Y^2$ (les variances) en calculant la variance empirique (\texttt{.var()}).
    \item Estimer $\sigma_X$ et $\sigma_Y$ (les écarts-types) en prenant la racine carrée des variances (ou \texttt{.std()}).
\end{enumerate}
\end{exercicebox}

\begin{exercicebox}[Exercice 2 : Standardisation (Score Z)]
Le Z-score nous dit à combien d'écarts-types une observation se situe de la moyenne.

\textbf{Votre tâche :}
\begin{enumerate}
    \item Utiliser les $\mu_X$ et $\sigma_X$ (pour GOOG) estimés à l'exercice 1.
    \item Trouver le rendement du dernier jour disponible dans vos données pour GOOG.
    \item Calculer le Z-score de ce rendement : $Z = (x - \mu_X) / \sigma_X$.
    \item Interpréter ce score (ex: "Le rendement de ce jour était à [Z] écarts-types de la moyenne...").
\end{enumerate}
\end{exercicebox}

\begin{exercicebox}[Exercice 3 : Calcul de Probabilité (Utilisation de $\Phi$)]
En utilisant les paramètres $\mu_X$ et $\sigma_X$ pour GOOG (Ex 1) et en supposant la distribution normale :

\textbf{Votre tâche :}
\begin{enumerate}
    \item Calculer la probabilité qu'un jour donné, le rendement de GOOG soit "calme", c'est-à-dire $P(-0.01 \le X \le 0.01)$.
    \item (Indice : Standardisez $a=-0.01$ et $b=0.01$ en $z_a, z_b$, puis calculez $\Phi(z_b) - \Phi(z_a)$. Vous aurez besoin de \texttt{scipy.stats.norm.cdf()}).
\end{enumerate}
\end{exercicebox}

\begin{exercicebox}[Exercice 4 : Problème Inverse (Value at Risk)]
Le "Value at Risk" (VaR) est une valeur $x$ telle qu'il y a une probabilité $p$ de perdre plus que $x$.

\textbf{Votre tâche :}
\begin{enumerate}
    \item Toujours avec les paramètres de GOOG, trouvez la valeur $x$ (le rendement) qui correspond au "top 5\%" des pires jours.
    \item Autrement dit, trouvez $x$ tel que $P(X \le x) = 0.05$.
    \item (Indice : Trouvez le Z-score $z$ tel que $\Phi(z) = 0.05$ (vous aurez besoin de \texttt{scipy.stats.norm.ppf()}), puis "dé-standardisez" : $x = \mu_X + z \sigma_X$).
\end{enumerate}
\end{exercicebox}

\begin{exercicebox}[Exercice 5 : PDF (Calcul de la Densité)]
La PDF $f(x)$ n'est pas une probabilité, mais une "densité". La valeur $f(\mu)$ est le point le plus haut de la cloche.

\textbf{Votre tâche :}
\begin{enumerate}
    \item Utiliser les $\mu_X$ et $\sigma_X$ de GOOG (Ex 1).
    \item Calculer la valeur de la densité de probabilité au point $x=\mu_X$ (le pic de la cloche).
    \item Calculer la valeur de la densité au point $x=0.0$.
    \item (Indice : Vous aurez besoin de \texttt{scipy.stats.norm.pdf(x, loc=mu, scale=sigma)}).
\end{enumerate}
\end{exercicebox}

\begin{exercicebox}[Exercice 6 : Vérification de la Règle Empirique (68-95-99.7)]
La théorie dit que $\approx 68\%$ des données devraient être dans $\mu \pm \sigma$. Nous allons vérifier si les rendements boursiers respectent cette règle.

\textbf{Votre tâche :}
\begin{enumerate}
    \item Utiliser les $\mu_X$ et $\sigma_X$ de GOOG (Ex 1).
    \item Calculer la proportion \textbf{réelle} des rendements qui tombent dans l'intervalle $[\mu_X - \sigma_X, \mu_X + \sigma_X]$.
    \item (Bonus) Faire de même pour $[\mu_X - 2\sigma_X, \mu_X + 2\sigma_X]$ (théorie: 95\%).
    \item (Conclusion) Les rendements de GOOG semblent-ils suivre parfaitement la règle ?
\end{enumerate}
\end{exercicebox}

\begin{exercicebox}[Exercice 7 : Calcul des Quartiles]
L'écart interquartile (IQR) est une autre mesure de dispersion, $IQR = Q_3 - Q_1$.
$Q_1$ est la valeur $x$ telle que $P(X \le x) = 0.25$.
$Q_3$ est la valeur $x$ telle que $P(X \le x) = 0.75$.

\textbf{Votre tâche :}
\begin{enumerate}
    \item En utilisant le modèle normal pour GOOG ($\mu_X, \sigma_X$), trouver les Z-scores $z_1$ et $z_3$ pour $p=0.25$ et $p=0.75$ (\texttt{norm.ppf}).
    \item "Dé-standardiser" ces Z-scores pour trouver $Q_1$ et $Q_3$.
    \item Calculer l'IQR \textbf{théorique} ($Q_3 - Q_1$).
    \item Comparer cet IQR théorique à l'IQR \textbf{empirique} de la série $X$ (Indice : \texttt{X.quantile(0.75) - X.quantile(0.25)}).
\end{enumerate}
\end{exercicebox}

\begin{exercicebox}[Exercice 8 : Stabilité par Transformation Linéaire (Y=aX+b)]
Soit un produit financier (ETF) qui amplifie par 2 les mouvements de GOOG, $Y = 2X$.
La théorie prédit $E[Y] = 2E[X]$ et $\text{Var}(Y) = 2^2 \text{Var}(X) = 4\text{Var}(X)$.

\textbf{Votre tâche :}
\begin{enumerate}
    \item Calculer $E[Y]$ et $\text{Var}(Y)$ \textbf{théoriquement} en utilisant $E[X]$ et $\text{Var}(X)$ de l'Ex 1.
    \item \textbf{Vérification empirique} :
        \begin{itemize}
            \item Créer la série de données $Y_{series} = 2 \times X$.
            \item Calculer la moyenne (\texttt{.mean()}) et la variance (\texttt{.var()}) de $Y_{series}$.
        \end{itemize}
    \item Comparer les résultats théoriques et empiriques.
\end{enumerate}
\end{exercicebox}

\begin{exercicebox}[Exercice 9 : Stabilité par Addition (Portefeuille S=X+Y)]
Théorie : $E[X+Y] = E[X] + E[Y]$ et $\text{Var}(X+Y) = \text{Var}(X) + \text{Var}(Y) + 2\text{Cov}(X,Y)$.
*Note : $X$ et $Y$ (GOOG, MSFT) ne sont PAS indépendantes.*

\textbf{Votre tâche :}
\begin{enumerate}
    \item Créer la série $S = X + Y$.
    \item Calculer $E[S]$ (la moyenne de $S$).
    \item Vérifier que $E[S] = E[X] + E[Y]$ (en utilisant les valeurs de l'Ex 1).
    \item Calculer $\text{Var}(S)$ (la variance de $S$).
    \item Est-ce que $\text{Var}(S) = \text{Var}(X) + \text{Var}(Y)$ ? Pourquoi y a-t-il une différence ?
\end{enumerate}
\end{exercicebox}

\begin{exercicebox}[Exercice 10 : Covariance et Variance du Portefeuille (Corrigé)]
Reprenons l'exercice 9. La différence est due à la covariance.

\textbf{Votre tâche :}
\begin{enumerate}
    \item Calculer $\text{Cov}(X,Y)$ (vous pouvez utiliser \texttt{returns.cov()}).
    \item Calculer la variance \textbf{théorique} de $S = X+Y$ avec la formule complète : $\text{Var}(S_{theo}) = \text{Var}(X) + \text{Var}(Y) + 2\text{Cov}(X,Y)$.
    \item Comparer $\text{Var}(S_{theo})$ à la variance empirique $\text{Var}(S)$ calculée à l'exercice 9.
\end{enumerate}
\end{exercicebox}

\begin{exercicebox}[Exercice 11 : Distribution d'un Portefeuille Pondéré]
Un portefeuille $P$ est composé de 60\% de GOOG ($X$) et 40\% de MSFT ($Y$). Donc $P = 0.6X + 0.4Y$.
Si $X$ et $Y$ sont (conjointement) normales, $P$ est aussi normale.

\textbf{Votre tâche :}
\begin{enumerate}
    \item Calculer l'espérance $E[P]$ (en utilisant la linéarité : $0.6E[X] + 0.4E[Y]$).
    \item Calculer la variance $\text{Var}(P)$ (en utilisant la formule complète : $\text{Var}(0.6X + 0.4Y) = 0.6^2\text{Var}(X) + 0.4^2\text{Var}(Y) + 2(0.6)(0.4)\text{Cov}(X,Y)$).
    \item Énoncer la loi de probabilité approximative du portefeuille : $P \sim \mathcal{N}(\mu_P, \sigma_P^2)$.
\end{enumerate}
\end{exercicebox}

\begin{exercicebox}[Exercice 12 : Distribution de la Différence (Pairs Trading)]
Un "spread" $D$ est la différence de rendement $D = X - Y$.
Théorie : $E[D] = E[X] - E[Y]$ et $\text{Var}(D) = \text{Var}(X) + \text{Var}(Y) - 2\text{Cov}(X,Y)$.

\textbf{Votre tâche :}
\begin{enumerate}
    \item Calculer $E[D]$ et $\text{Var}(D)$ \textbf{théoriquement} en utilisant les valeurs des exercices précédents.
    \item \textbf{Vérification empirique} :
        \begin{itemize}
            \item Créer la série de données $D_{series} = X - Y$.
            \item Calculer la moyenne (\texttt{.mean()}) et la variance (\texttt{.var()}) de $D_{series}$.
        \end{itemize}
    \item Comparer les résultats théoriques et empiriques.
\end{enumerate}
\end{exercicebox}