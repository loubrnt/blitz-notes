\subsection{Exercices Python}

Ces exercices appliquent les concepts de la loi normale au jeu de données "Yahoo Finance". Nous allons modéliser les \textbf{rendements journaliers} (variation en pourcentage) des actions, qui sont souvent (par approximation) considérés comme suivant une loi normale.

Nous allons travailler avec les rendements de Google ($X$) et de Microsoft ($Y$).

\begin{codecell}
!pip install yfinance
import yfinance as yf
import pandas as pd
import numpy as np
from scipy.stats import norm # Moteur pour les calculs de CDF/PDF

# Definir les tickers et la periode
tickers = ["GOOG", "MSFT"]
start_date = "2020-01-01"
end_date = "2024-12-31"

# Telecharger les prix de cloture ajustes
data = yf.download(tickers, start=start_date, end=end_date)["Adj Close"]

# Calculer les rendements journaliers en pourcentage
returns = data.pct_change().dropna()

# Renommer les colonnes pour plus de clarte
returns.columns = ["GOOG_Return", "MSFT_Return"]

# 'returns' est notre DataFrame principal.
# X = returns["GOOG_Return"]
# Y = returns["MSFT_Return"]
\end{codecell}

\begin{exercicebox}[Exercice 1 : Estimation des Paramètres $\mu$ et $\sigma^2$]
Soit $X$ la v.a. "Rendement journalier de GOOG" et $Y$ la v.a. "Rendement journalier de MSFT".
Nous supposons $X \sim \mathcal{N}(\mu_X, \sigma_X^2)$ et $Y \sim \mathcal{N}(\mu_Y, \sigma_Y^2)$.

\textbf{Votre tâche :}
\begin{enumerate}
    \item Estimer $\mu_X$ et $\mu_Y$ (les espérances) en calculant la moyenne empirique (\texttt{.mean()}) des deux séries.
    \item Estimer $\sigma_X^2$ et $\sigma_Y^2$ (les variances) en calculant la variance empirique (\texttt{.var()}).
    \item Estimer $\sigma_X$ et $\sigma_Y$ (les écarts-types) en prenant la racine carrée des variances (ou \texttt{.std()}).
\end{enumerate}
\end{exercicebox}

\begin{exercicebox}[Exercice 2 : Standardisation (Score Z)]
Le Z-score nous dit à combien d'écarts-types une observation se situe de la moyenne.

\textbf{Votre tâche :}
\begin{enumerate}
    \item Utiliser les $\mu_X$ et $\sigma_X$ (pour GOOG) estimés à l'exercice 1.
    \item Trouver le rendement du dernier jour disponible dans vos données pour GOOG.
    \item Calculer le Z-score de ce rendement : $Z = (x - \mu_X) / \sigma_X$.
    \item Interpréter ce score (ex: "Le rendement de ce jour était à [Z] écarts-types de la moyenne...").
\end{enumerate}
\end{exercicebox}

\begin{exercicebox}[Exercice 3 : Calcul de Probabilité (Utilisation de $\Phi$)]
En utilisant les paramètres $\mu_X$ et $\sigma_X$ pour GOOG (Ex 1) et en supposant la distribution normale :

\textbf{Votre tâche :}
\begin{enumerate}
    \item Calculer la probabilité qu'un jour donné, le rendement de GOOG soit "calme", c'est-à-dire $P(-0.01 \le X \le 0.01)$.
    \item (Indice : Standardisez $a=-0.01$ et $b=0.01$ en $z_a, z_b$, puis calculez $\Phi(z_b) - \Phi(z_a)$. Vous aurez besoin de \texttt{scipy.stats.norm.cdf()}).
\end{enumerate}
\end{exercicebox}

\begin{exercicebox}[Exercice 4 : Problème Inverse (Value at Risk)]
Le "Value at Risk" (VaR) est une valeur $x$ telle qu'il y a une probabilité $p$ de perdre plus que $x$.

\textbf{Votre tâche :}
\begin{enumerate}
    \item Toujours avec les paramètres de GOOG, trouvez la valeur $x$ (le rendement) qui correspond au "top 5\%" des pires jours.
    \item Autrement dit, trouvez $x$ tel que $P(X \le x) = 0.05$.
    \item (Indice : Trouvez le Z-score $z$ tel que $\Phi(z) = 0.05$ (vous aurez besoin de \texttt{scipy.stats.norm.ppf()}), puis "dé-standardisez" : $x = \mu_X + z \sigma_X$).
\end{enumerate}
\end{exercicebox}

\begin{exercicebox}[Exercice 5 : PDF (Calcul de la Densité)]
La PDF $f(x)$ n'est pas une probabilité, mais une "densité". La valeur $f(\mu)$ est le point le plus haut de la cloche.

\textbf{Votre tâche :}
\begin{enumerate}
    \item Utiliser les $\mu_X$ et $\sigma_X$ de GOOG (Ex 1).
    \item Calculer la valeur de la densité de probabilité au point $x=\mu_X$ (le pic de la cloche).
    \item Calculer la valeur de la densité au point $x=0.0$.
    \item (Indice : Vous aurez besoin de \texttt{scipy.stats.norm.pdf(x, loc=mu, scale=sigma)}).
\end{enumerate}
\end{exercicebox}

\begin{exercicebox}[Exercice 6 : Vérification de la Règle Empirique (68-95-99.7)]
La théorie dit que $\approx 68\%$ des données devraient être dans $\mu \pm \sigma$. Nous allons vérifier si les rendements boursiers respectent cette règle.

\textbf{Votre tâche :}
\begin{enumerate}
    \item Utiliser les $\mu_X$ et $\sigma_X$ de GOOG (Ex 1).
    \item Calculer la proportion \textbf{réelle} des rendements qui tombent dans l'intervalle $[\mu_X - \sigma_X, \mu_X + \sigma_X]$.
    \item (Bonus) Faire de même pour $[\mu_X - 2\sigma_X, \mu_X + 2\sigma_X]$ (théorie: 95\%).
    \item (Conclusion) Les rendements de GOOG semblent-ils suivre parfaitement la règle ?
\end{enumerate}
\end{exercicebox}

\begin{exercicebox}[Exercice 7 : Calcul des Quartiles]
L'écart interquartile (IQR) est une autre mesure de dispersion, $IQR = Q_3 - Q_1$.
$Q_1$ est la valeur $x$ telle que $P(X \le x) = 0.25$.
$Q_3$ est la valeur $x$ telle que $P(X \le x) = 0.75$.

\textbf{Votre tâche :}
\begin{enumerate}
    \item En utilisant le modèle normal pour GOOG ($\mu_X, \sigma_X$), trouver les Z-scores $z_1$ et $z_3$ pour $p=0.25$ et $p=0.75$ (\texttt{norm.ppf}).
    \item "Dé-standardiser" ces Z-scores pour trouver $Q_1$ et $Q_3$.
    \item Calculer l'IQR \textbf{théorique} ($Q_3 - Q_1$).
    \item Comparer cet IQR théorique à l'IQR \textbf{empirique} de la série $X$ (Indice : \texttt{X.quantile(0.75) - X.quantile(0.25)}).
\end{enumerate}
\end{exercicebox}

\begin{exercicebox}[Exercice 8 : Stabilité par Transformation Linéaire (Y=aX+b)]
Soit un produit financier (ETF) qui amplifie par 2 les mouvements de GOOG, $Y = 2X$.
La théorie prédit $E[Y] = 2E[X]$ et $\text{Var}(Y) = 2^2 \text{Var}(X) = 4\text{Var}(X)$.

\textbf{Votre tâche :}
\begin{enumerate}
    \item Calculer $E[Y]$ et $\text{Var}(Y)$ \textbf{théoriquement} en utilisant $E[X]$ et $\text{Var}(X)$ de l'Ex 1.
    \item \textbf{Vérification empirique} :
        \begin{itemize}
            \item Créer la série de données $Y_{series} = 2 \times X$.
            \item Calculer la moyenne (\texttt{.mean()}) et la variance (\texttt{.var()}) de $Y_{series}$.
        \end{itemize}
    \item Comparer les résultats théoriques et empiriques.
\end{enumerate}
\end{exercicebox}

\begin{exercicebox}[Exercice 9 : Stabilité par Addition (Portefeuille S=X+Y)]
Théorie : $E[X+Y] = E[X] + E[Y]$ et $\text{Var}(X+Y) = \text{Var}(X) + \text{Var}(Y) + 2\text{Cov}(X,Y)$.
*Note : $X$ et $Y$ (GOOG, MSFT) ne sont PAS indépendantes.*

\textbf{Votre tâche :}
\begin{enumerate}
    \item Créer la série $S = X + Y$.
    \item Calculer $E[S]$ (la moyenne de $S$).
    \item Vérifier que $E[S] = E[X] + E[Y]$ (en utilisant les valeurs de l'Ex 1).
    \item Calculer $\text{Var}(S)$ (la variance de $S$).
    \item Est-ce que $\text{Var}(S) = \text{Var}(X) + \text{Var}(Y)$ ? Pourquoi y a-t-il une différence ?
\end{enumerate}
\end{exercicebox}

\begin{exercicebox}[Exercice 10 : Covariance et Variance du Portefeuille (Corrigé)]
Reprenons l'exercice 9. La différence est due à la covariance.

\textbf{Votre tâche :}
\begin{enumerate}
    \item Calculer $\text{Cov}(X,Y)$ (vous pouvez utiliser \texttt{returns.cov()}).
    \item Calculer la variance \textbf{théorique} de $S = X+Y$ avec la formule complète : $\text{Var}(S_{theo}) = \text{Var}(X) + \text{Var}(Y) + 2\text{Cov}(X,Y)$.
    \item Comparer $\text{Var}(S_{theo})$ à la variance empirique $\text{Var}(S)$ calculée à l'exercice 9.
\end{enumerate}
\end{exercicebox}

\begin{exercicebox}[Exercice 11 : Distribution d'un Portefeuille Pondéré]
Un portefeuille $P$ est composé de 60\% de GOOG ($X$) et 40\% de MSFT ($Y$). Donc $P = 0.6X + 0.4Y$.
Si $X$ et $Y$ sont (conjointement) normales, $P$ est aussi normale.

\textbf{Votre tâche :}
\begin{enumerate}
    \item Calculer l'espérance $E[P]$ (en utilisant la linéarité : $0.6E[X] + 0.4E[Y]$).
    \item Calculer la variance $\text{Var}(P)$ (en utilisant la formule complète : $\text{Var}(0.6X + 0.4Y) = 0.6^2\text{Var}(X) + 0.4^2\text{Var}(Y) + 2(0.6)(0.4)\text{Cov}(X,Y)$).
    \item Énoncer la loi de probabilité approximative du portefeuille : $P \sim \mathcal{N}(\mu_P, \sigma_P^2)$.
\end{enumerate}
\end{exercicebox}

\begin{exercicebox}[Exercice 12 : Distribution de la Différence (Pairs Trading)]
Un "spread" $D$ est la différence de rendement $D = X - Y$.
Théorie : $E[D] = E[X] - E[Y]$ et $\text{Var}(D) = \text{Var}(X) + \text{Var}(Y) - 2\text{Cov}(X,Y)$.

\textbf{Votre tâche :}
\begin{enumerate}
    \item Calculer $E[D]$ et $\text{Var}(D)$ \textbf{théoriquement} en utilisant les valeurs des exercices précédents.
    \item \textbf{Vérification empirique} :
        \begin{itemize}
            \item Créer la série de données $D_{series} = X - Y$.
            \item Calculer la moyenne (\texttt{.mean()}) et la variance (\texttt{.var()}) de $D_{series}$.
        \end{itemize}
    \item Comparer les résultats théoriques et empiriques.
\end{enumerate}
\end{exercicebox}