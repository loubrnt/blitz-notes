\subsection{Exercices}

\textit{Pour tous les exercices de calcul, vous pouvez utiliser les valeurs (arrondies) suivantes pour la fonction de répartition de la loi normale standard $\Phi(z) = P(Z \le z)$ :}
\begin{itemize}
    \item $\Phi(0) = 0.5$
    \item $\Phi(1.0) \approx 0.8413$
    \item $\Phi(1.5) \approx 0.9332$
    \item $\Phi(1.96) \approx 0.975$
    \item $\Phi(2.0) \approx 0.9772$
    \item $\Phi(2.5) \approx 0.9938$
    \item $\Phi(3.0) \approx 0.9987$
\end{itemize}
\textit{Et rappelez-vous la propriété de symétrie : $\Phi(-z) = 1 - \Phi(z)$.}

% --- Section 1 : Paramètres de S_n et X_n barre ---

\begin{exercicebox}[Exercice 1 : Paramètres de $\bar{X}_n$]
Soit $X_i$ une suite de v.a. i.i.d. avec $\mu=50$ et $\sigma^2=100$. Soit $n=25$.
Calculez $E[\bar{X}_{25}]$ et $\text{Var}(\bar{X}_{25})$.
\end{exercicebox}

\begin{exercicebox}[Exercice 2 : Paramètres de $S_n$]
Soit $X_i$ une suite de v.a. i.i.d. avec $\mu=10$ et $\sigma=3$. Soit $n=16$.
Calculez $E[S_{16}]$ et l'écart-type $\sigma_{S_{16}}$.
\end{exercicebox}

\begin{exercicebox}[Exercice 3 : Paramètres de $\bar{X}_n$ (bis)]
Soit $X_i$ une suite de v.a. i.i.d. avec $\mu=70$ et $\sigma=10$. Soit $n=400$.
Calculez $E[\bar{X}_{400}]$ et l'écart-type $\sigma_{\bar{X}_{400}}$.
\end{exercicebox}

\begin{exercicebox}[Exercice 4 : Paramètres de $S_n$ (bis)]
Soit $X_i$ une suite de v.a. i.i.d. avec $\mu=0.5$ et $\sigma^2=0.01$. Soit $n=64$.
Calculez $E[S_{64}]$ et $\text{Var}(S_{64})$.
\end{exercicebox}

\begin{exercicebox}[Exercice 5 : Retrouver $\sigma^2$ (via $\bar{X}_n$)]
La moyenne d'échantillon $\bar{X}_n$ de $n=49$ observations a une variance $\text{Var}(\bar{X}_{49}) = 2$.
Quelle est la variance $\sigma^2$ de la population d'origine ?
\end{exercicebox}

\begin{exercicebox}[Exercice 6 : Retrouver $n$ (via $S_n$)]
La somme $S_n$ d'observations i.i.d. a une variance $\text{Var}(S_n) = 300$. La variance de la population est $\sigma^2 = 12$.
Quelle est la taille de l'échantillon $n$ ?
\end{exercicebox}

% --- Section 2 : Calculs de probabilité (Forme $\bar{X}_n$) ---

\begin{exercicebox}[Exercice 7 : CLT pour $\bar{X}_n$ (Queue Droite)]
Soit $X_i$ i.i.d. avec $\mu=100$ et $\sigma=15$. Soit $n=36$.
Calculez $P(\bar{X}_{36} > 105)$.
\end{exercicebox}

\begin{exercicebox}[Exercice 8 : CLT pour $\bar{X}_n$ (Queue Gauche)]
Soit $X_i$ i.i.d. avec $\mu=50$ et $\sigma=8$. Soit $n=64$.
Calculez $P(\bar{X}_{64} < 48)$.
\end{exercicebox}

\begin{exercicebox}[Exercice 9 : CLT pour $\bar{X}_n$ (Intervalle)]
Soit $X_i$ i.i.d. avec $\mu=20$ et $\sigma=5$. Soit $n=100$.
Calculez $P(19 \le \bar{X}_{100} \le 21.25)$.
\end{exercicebox}

\begin{exercicebox}[Exercice 10 : Application $\bar{X}_n$ (Bouteilles)]
Une machine remplit des bouteilles avec $\mu=500$ ml et $\sigma=6$ ml. On prend $n=36$ bouteilles.
Calculez $P(\bar{X}_{36} > 501.5)$.
\end{exercicebox}

\begin{exercicebox}[Exercice 11 : Application $\bar{X}_n$ (Tailles)]
La taille des individus a $\mu=175$ cm et $\sigma=8$ cm. On prend $n=64$ individus.
Calculez $P(\bar{X}_{64} < 173)$.
\end{exercicebox}

\begin{exercicebox}[Exercice 12 : Application $\bar{X}_n$ (Notes)]
Les notes à un examen ont $\mu=70$ et $\sigma=12$. Une classe de $n=36$ étudiants est un échantillon.
Calculez $P(\bar{X}_{36} < 67)$.
\end{exercicebox}

% --- Section 3 : Calculs de probabilité (Forme $S_n$) ---

\begin{exercicebox}[Exercice 13 : CLT pour $S_n$ (Queue Droite)]
Soit $X_i$ i.i.d. avec $\mu=10$ et $\sigma=2$. Soit $n=100$.
Calculez $P(S_{100} > 1020)$.
\end{exercicebox}

\begin{exercicebox}[Exercice 14 : CLT pour $S_n$ (Queue Gauche)]
Soit $X_i$ i.i.d. avec $\mu=5$ et $\sigma=4$. Soit $n=64$.
Calculez $P(S_{64} \le 304)$.
\end{exercicebox}

\begin{exercicebox}[Exercice 15 : CLT pour $S_n$ (Intervalle)]
Soit $X_i$ i.i.d. avec $\mu=2$ et $\sigma=3$. Soit $n=36$.
Calculez $P(S_{36} > 90)$.
\end{exercicebox}

\begin{exercicebox}[Exercice 16 : Application $S_n$ (Ascenseur)]
Un ascenseur transporte $n=49$ personnes. Poids : $\mu=70$kg, $\sigma=14$kg.
Calculez $P(S_{49} > 3528)$.
\end{exercicebox}

\begin{exercicebox}[Exercice 17 : Application $S_n$ (Rendement)]
Le rendement quotidien $X_i$ a $\mu=0.001$ et $\sigma=0.01$. Soit $n=100$.
Calculez $P(S_{100} > 0.2)$.
\end{exercicebox}

\begin{exercicebox}[Exercice 18 : Application $S_n$ (Rendement)]
En utilisant les données de l'exercice 17, calculez $P(S_{100} < 0)$.
\end{exercicebox}

% --- Section 4 : Problèmes Inverses (CLT) ---

\begin{exercicebox}[Exercice 19 : Inverse (Trouver $c$ pour $\bar{X}_n$)]
Soit $X_i$ i.i.d. avec $\mu=50$ et $\sigma=10$. Soit $n=100$.
Trouvez la valeur $c$ telle que $P(\bar{X}_{100} \le c) \approx 0.8413$.
\end{exercicebox}

\begin{exercicebox}[Exercice 20 : Inverse (Trouver $c$ pour $S_n$)]
Soit $X_i$ i.i.d. avec $\mu=10$ et $\sigma=3$. Soit $n=36$.
Trouvez la valeur $c$ telle que $P(S_{36} \le c) \approx 0.0013$.
\end{exercicebox}

\begin{exercicebox}[Exercice 21 : Inverse (Taille d'échantillon $n$)]
Une population a $\mu=0$ et $\sigma=10$.
Quelle taille d'échantillon $n$ faut-il pour que $P(|\bar{X}_n| \le 1) \ge 0.95$ ?
(Indice : $P(-1.96 \le Z \le 1.96) = 0.95$).
\end{exercicebox}

\begin{exercicebox}[Exercice 22 : Inverse (Taille d'échantillon $n$)]
Une population a $\mu=100$ et $\sigma=20$.
Quelle taille d'échantillon $n$ faut-il pour que $P(\bar{X}_n \ge 102) \le 0.0228$ ?
\end{exercicebox}

% --- Section 5 : Proportions (Application de Bernoulli) ---

\begin{exercicebox}[Exercice 23 : Paramètres (Proportion)]
On sonde $n=400$ personnes. La vraie proportion $p$ est $0.25$. On modélise $X_i \sim \text{Bern}(p)$.
\begin{enumerate}
    \item Calculez $\mu = E[X_i]$ et $\sigma^2 = \text{Var}(X_i)$.
    \item Calculez $E[\hat{p}]$ et $\text{Var}(\hat{p})$ (où $\hat{p} = \bar{X}_n$).
\end{enumerate}
\end{exercicebox}

\begin{exercicebox}[Exercice 24 : Calcul (Proportion)]
On sonde $n=100$ personnes. La vraie proportion est $p=0.5$.
Calculez la probabilité que la proportion observée $\hat{p}$ soit supérieure à 0.6, $P(\hat{p} > 0.6)$.
\end{exercicebox}

\begin{exercicebox}[Exercice 25 : Marge d'Erreur (Proportion)]
Un sondage sur $n=1000$ personnes donne un résultat $\hat{p} = 0.54$.
Calculez la marge d'erreur à 95\% (c'est-à-dire $1.96 \times SE_{\hat{p}}$).
\end{exercicebox}


\subsection{Corrections des Exercices}

\begin{correctionbox}[Correction Exercice 1 : Paramètres de $\bar{X}_n$]
$E[\bar{X}_{25}] = \mu = 50$.
$\text{Var}(\bar{X}_{25}) = \frac{\sigma^2}{n} = \frac{100}{25} = 4$.
\end{correctionbox}

\begin{correctionbox}[Correction Exercice 2 : Paramètres de $S_n$]
$E[S_{16}] = n\mu = 16 \times 10 = 160$.
$\text{Var}(S_{16}) = n\sigma^2 = 16 \times 3^2 = 16 \times 9 = 144$.
$\sigma_{S_{16}} = \sqrt{\text{Var}(S_{16})} = \sqrt{144} = 12$.
\end{correctionbox}

\begin{correctionbox}[Correction Exercice 3 : Paramètres de $\bar{X}_n$ (bis)]
$E[\bar{X}_{400}] = \mu = 70$.
$\sigma_{\bar{X}_{400}} = \frac{\sigma}{\sqrt{n}} = \frac{10}{\sqrt{400}} = \frac{10}{20} = 0.5$.
\end{correctionbox}

\begin{correctionbox}[Correction Exercice 4 : Paramètres de $S_n$ (bis)]
$E[S_{64}] = n\mu = 64 \times 0.5 = 32$.
$\text{Var}(S_{64}) = n\sigma^2 = 64 \times 0.01 = 0.64$.
\end{correctionbox}

\begin{correctionbox}[Correction Exercice 5 : Retrouver $\sigma^2$ (via $\bar{X}_n$)]
$\text{Var}(\bar{X}_n) = \frac{\sigma^2}{n} \implies \sigma^2 = n \cdot \text{Var}(\bar{X}_n)$.
$\sigma^2 = 49 \times 2 = 98$.
\end{correctionbox}

\begin{correctionbox}[Correction Exercice 6 : Retrouver $n$ (via $S_n$)]
$\text{Var}(S_n) = n\sigma^2 \implies n = \frac{\text{Var}(S_n)}{\sigma^2}$.
$n = \frac{300}{12} = 25$.
\end{correctionbox}

\begin{correctionbox}[Correction Exercice 7 : CLT pour $\bar{X}_n$ (Queue Droite)]
$\mu=100, \sigma=15, n=36$. $\bar{X}_{36} \approx \mathcal{N}(\mu, \sigma^2/n)$.
$\mu_{\bar{X}} = 100$. $\sigma_{\bar{X}} = \frac{15}{\sqrt{36}} = \frac{15}{6} = 2.5$.
On cherche $P(\bar{X}_{36} > 105)$.
$Z = \frac{105 - 100}{2.5} = \frac{5}{2.5} = 2$.
$P(Z > 2) = 1 - \Phi(2) \approx 1 - 0.9772 = 0.0228$.
\end{correctionbox}

\begin{correctionbox}[Correction Exercice 8 : CLT pour $\bar{X}_n$ (Queue Gauche)]
$\mu=50, \sigma=8, n=64$. $\bar{X}_{64} \approx \mathcal{N}(\mu, \sigma^2/n)$.
$\mu_{\bar{X}} = 50$. $\sigma_{\bar{X}} = \frac{8}{\sqrt{64}} = \frac{8}{8} = 1$.
On cherche $P(\bar{X}_{64} < 48)$.
$Z = \frac{48 - 50}{1} = -2$.
$P(Z < -2) = \Phi(-2) = 1 - \Phi(2) \approx 1 - 0.9772 = 0.0228$.
\end{correctionbox}

\begin{correctionbox}[Correction Exercice 9 : CLT pour $\bar{X}_n$ (Intervalle)]
$\mu=20, \sigma=5, n=100$. $\bar{X}_{100} \approx \mathcal{N}(\mu, \sigma^2/n)$.
$\mu_{\bar{X}} = 20$. $\sigma_{\bar{X}} = \frac{5}{\sqrt{100}} = \frac{5}{10} = 0.5$.
On cherche $P(19 \le \bar{X}_{100} \le 21.25)$.
$Z_1 = \frac{19 - 20}{0.5} = -2$.
$Z_2 = \frac{21.25 - 20}{0.5} = \frac{1.25}{0.5} = 2.5$.
$P(-2 \le Z \le 2.5) = \Phi(2.5) - \Phi(-2) = \Phi(2.5) - (1 - \Phi(2))$.
$P \approx 0.9938 - (1 - 0.9772) = 0.9938 - 0.0228 = 0.9710$.
\end{correctionbox}

\begin{correctionbox}[Correction Exercice 10 : Application $\bar{X}_n$ (Bouteilles)]
$\mu=500, \sigma=6, n=36$. $\bar{X}_{36} \approx \mathcal{N}(\mu, \sigma^2/n)$.
$\mu_{\bar{X}} = 500$. $\sigma_{\bar{X}} = \frac{6}{\sqrt{36}} = \frac{6}{6} = 1$.
On cherche $P(\bar{X}_{36} > 501.5)$.
$Z = \frac{501.5 - 500}{1} = 1.5$.
$P(Z > 1.5) = 1 - \Phi(1.5) \approx 1 - 0.9332 = 0.0668$.
\end{correctionbox}

\begin{correctionbox}[Correction Exercice 11 : Application $\bar{X}_n$ (Tailles)]
$\mu=175, \sigma=8, n=64$. $\bar{X}_{64} \approx \mathcal{N}(\mu, \sigma^2/n)$.
$\mu_{\bar{X}} = 175$. $\sigma_{\bar{X}} = \frac{8}{\sqrt{64}} = \frac{8}{8} = 1$.
On cherche $P(\bar{X}_{64} < 173)$.
$Z = \frac{173 - 175}{1} = -2$.
$P(Z < -2) = \Phi(-2) = 1 - \Phi(2) \approx 1 - 0.9772 = 0.0228$.
\end{correctionbox}

\begin{correctionbox}[Correction Exercice 12 : Application $\bar{X}_n$ (Notes)]
$\mu=70, \sigma=12, n=36$. $\bar{X}_{36} \approx \mathcal{N}(\mu, \sigma^2/n)$.
$\mu_{\bar{X}} = 70$. $\sigma_{\bar{X}} = \frac{12}{\sqrt{36}} = \frac{12}{6} = 2$.
On cherche $P(\bar{X}_{36} < 67)$.
$Z = \frac{67 - 70}{2} = -1.5$.
$P(Z < -1.5) = \Phi(-1.5) = 1 - \Phi(1.5) \approx 1 - 0.9332 = 0.0668$.
\end{correctionbox}

\begin{correctionbox}[Correction Exercice 13 : CLT pour $S_n$ (Queue Droite)]
$\mu=10, \sigma=2, n=100$. $S_{100} \approx \mathcal{N}(n\mu, n\sigma^2)$.
$E[S_{100}] = 100 \times 10 = 1000$.
$\sigma_{S_n} = \sigma\sqrt{n} = 2 \times \sqrt{100} = 2 \times 10 = 20$.
On cherche $P(S_{100} > 1020)$.
$Z = \frac{1020 - 1000}{20} = \frac{20}{20} = 1$.
$P(Z > 1) = 1 - \Phi(1) \approx 1 - 0.8413 = 0.1587$.
\end{correctionbox}

\begin{correctionbox}[Correction Exercice 14 : CLT pour $S_n$ (Queue Gauche)]
$\mu=5, \sigma=4, n=64$. $S_{64} \approx \mathcal{N}(n\mu, n\sigma^2)$.
$E[S_{64}] = 64 \times 5 = 320$.
$\sigma_{S_n} = \sigma\sqrt{n} = 4 \times \sqrt{64} = 4 \times 8 = 32$.
On cherche $P(S_{64} \le 304)$.
$Z = \frac{304 - 320}{32} = \frac{-16}{32} = -0.5$.
$P(Z \le -0.5) = \Phi(-0.5) = 1 - \Phi(0.5)$. (Valeur non fournie).
\end{correctionbox}

\begin{correctionbox}[Correction Exercice 15 : CLT pour $S_n$ (Intervalle)]
$\mu=2, \sigma=3, n=36$. $S_{36} \approx \mathcal{N}(n\mu, n\sigma^2)$.
$E[S_{36}] = 36 \times 2 = 72$.
$\sigma_{S_n} = \sigma\sqrt{n} = 3 \times \sqrt{36} = 3 \times 6 = 18$.
On cherche $P(S_{36} > 90)$.
$Z = \frac{90 - 72}{18} = \frac{18}{18} = 1$.
$P(Z > 1) = 1 - \Phi(1) \approx 1 - 0.8413 = 0.1587$.
\end{correctionbox}

\begin{correctionbox}[Correction Exercice 16 : Application $S_n$ (Ascenseur)]
$\mu=70, \sigma=14, n=49$. $S_{49} \approx \mathcal{N}(n\mu, n\sigma^2)$.
$E[S_{49}] = 49 \times 70 = 3430$.
$\sigma_{S_n} = \sigma\sqrt{n} = 14 \times \sqrt{49} = 14 \times 7 = 98$.
On cherche $P(S_{49} > 3528)$.
$Z = \frac{3528 - 3430}{98} = \frac{98}{98} = 1$.
$P(Z > 1) = 1 - \Phi(1) \approx 1 - 0.8413 = 0.1587$.
\end{correctionbox}

\begin{correctionbox}[Correction Exercice 17 : Application $S_n$ (Rendement)]
$\mu=0.001, \sigma=0.01, n=100$. $S_{100} \approx \mathcal{N}(n\mu, n\sigma^2)$.
$E[S_{100}] = 100 \times 0.001 = 0.1$.
$\sigma_{S_n} = \sigma\sqrt{n} = 0.01 \times \sqrt{100} = 0.01 \times 10 = 0.1$.
On cherche $P(S_{100} > 0.2)$.
$Z = \frac{0.2 - 0.1}{0.1} = \frac{0.1}{0.1} = 1$.
$P(Z > 1) = 1 - \Phi(1) \approx 1 - 0.8413 = 0.1587$.
\end{correctionbox}

\begin{correctionbox}[Correction Exercice 18 : Application $S_n$ (Rendement)]
On utilise les mêmes paramètres : $S_{100} \approx \mathcal{N}(0.1, 0.1^2)$.
On cherche $P(S_{100} < 0)$.
$Z = \frac{0 - 0.1}{0.1} = -1$.
$P(Z < -1) = \Phi(-1) = 1 - \Phi(1) \approx 1 - 0.8413 = 0.1587$.
\end{correctionbox}

\begin{correctionbox}[Correction Exercice 19 : Inverse (Trouver $c$ pour $\bar{X}_n$)]
$\mu=50, \sigma=10, n=100$. $\bar{X}_{100} \approx \mathcal{N}(\mu, \sigma^2/n)$.
$\mu_{\bar{X}} = 50$. $\sigma_{\bar{X}} = \frac{10}{\sqrt{100}} = 1$.
On cherche $c$ tel que $P(\bar{X}_{100} \le c) \approx 0.8413$.
$P(Z \le \frac{c - 50}{1}) = 0.8413$.
D'après la table, le $z$ correspondant est $1.0$.
$\frac{c - 50}{1} = 1 \implies c = 51$.
\end{correctionbox}

\begin{correctionbox}[Correction Exercice 20 : Inverse (Trouver $c$ pour $S_n$)]
$\mu=10, \sigma=3, n=36$. $S_{36} \approx \mathcal{N}(n\mu, n\sigma^2)$.
$E[S_{36}] = 36 \times 10 = 360$.
$\sigma_{S_n} = \sigma\sqrt{n} = 3 \times \sqrt{36} = 18$.
On cherche $c$ tel que $P(S_{36} \le c) \approx 0.0013$.
$P(Z \le \frac{c - 360}{18}) = 0.0013$.
On sait $\Phi(3) \approx 0.9987$, donc $\Phi(-3) = 1 - 0.9987 = 0.0013$.
Le $z$ correspondant est $-3.0$.
$\frac{c - 360}{18} = -3 \implies c = 360 - 3(18) = 360 - 54 = 306$.
\end{correctionbox}

\begin{correctionbox}[Correction Exercice 21 : Inverse (Taille d'échantillon $n$)]
$\mu=0, \sigma=10$. On veut $P(|\bar{X}_n| \le 1) \ge 0.95$.
$P(-1 \le \bar{X}_n \le 1) \ge 0.95$.
Standardisation : $\sigma_{\bar{X}} = \frac{10}{\sqrt{n}}$.
$P\left( \frac{-1 - 0}{10/\sqrt{n}} \le Z \le \frac{1 - 0}{10/\sqrt{n}} \right) \ge 0.95$.
$P\left( \frac{-\sqrt{n}}{10} \le Z \le \frac{\sqrt{n}}{10} \right) \ge 0.95$.
On sait $P(-1.96 \le Z \le 1.96) = 0.95$.
On doit donc avoir $\frac{\sqrt{n}}{10} \ge 1.96$.
$\sqrt{n} \ge 19.6 \implies n \ge (19.6)^2 = 384.16$.
Il faut $n = 385$ au minimum.
\end{correctionbox}

\begin{correctionbox}[Correction Exercice 22 : Inverse (Taille d'échantillon $n$)]
$\mu=100, \sigma=20$. On veut $P(\bar{X}_n \ge 102) \le 0.0228$.
$P(Z \ge z) \le 0.0228$. On sait $P(Z \ge 2) = 1 - \Phi(2) \approx 0.0228$.
Donc on a besoin que notre Z-score soit $\ge 2$.
$Z = \frac{102 - 100}{20 / \sqrt{n}} = \frac{2}{20 / \sqrt{n}} = \frac{2\sqrt{n}}{20} = \frac{\sqrt{n}}{10}$.
On pose $Z \ge 2 \implies \frac{\sqrt{n}}{10} \ge 2$.
$\sqrt{n} \ge 20 \implies n \ge 400$.
Il faut $n = 400$ au minimum.
\end{correctionbox}

\begin{correctionbox}[Correction Exercice 23 : Paramètres (Proportion)]
$X_i \sim \text{Bern}(p)$ avec $p=0.25$.
1.  $\mu = E[X_i] = p = 0.25$.
    $\sigma^2 = \text{Var}(X_i) = p(1-p) = 0.25 \times 0.75 = 0.1875$.
2.  $\hat{p} = \bar{X}_n$. $n=400$.
    $E[\hat{p}] = \mu = 0.25$.
    $\text{Var}(\hat{p}) = \frac{\sigma^2}{n} = \frac{0.1875}{400} \approx 0.00046875$.
\end{correctionbox}

\begin{correctionbox}[Correction Exercice 24 : Calcul (Proportion)]
$p=0.5, n=100$. $\hat{p} \approx \mathcal{N}(p, \frac{p(1-p)}{n})$.
$E[\hat{p}] = 0.5$.
$\text{Var}(\hat{p}) = \frac{0.5 \times 0.5}{100} = \frac{0.25}{100} = 0.0025$.
$\sigma_{\hat{p}} = \sqrt{0.0025} = 0.05$.
On cherche $P(\hat{p} > 0.6)$.
$Z = \frac{0.6 - 0.5}{0.05} = \frac{0.1}{0.05} = 2$.
$P(Z > 2) = 1 - \Phi(2) \approx 1 - 0.9772 = 0.0228$.
\end{correctionbox}

\begin{correctionbox}[Correction Exercice 25 : Marge d'Erreur (Proportion)]
$n=1000, \hat{p}=0.54$.
On estime l'erreur standard $SE = \sqrt{\frac{\hat{p}(1-\hat{p})}{n}}$.
$SE = \sqrt{\frac{0.54 \times (1 - 0.54)}{1000}} = \sqrt{\frac{0.2484}{1000}} \approx 0.01576$.
La marge d'erreur à 95\% est $ME = 1.96 \times SE$.
$ME = 1.96 \times 0.01576 \approx 0.0309$.
(Soit $\pm 3.09\%$).
\end{correctionbox}

\subsection{Exercices Python}

Ces exercices appliquent le Théorème Central Limite (TCL) aux données financières. Nous considérerons l'ensemble des rendements journaliers de Google (GOOG) sur une longue période comme notre \textbf{population} (dont nous connaissons le "vrai" $\mu$ et $\sigma^2$). Nous simulerons ensuite un \textbf{échantillonnage} (tirer $n$ jours au hasard) pour voir comment la \textbf{distribution d'échantillonnage de la moyenne} ($\bar{X}_n$) se comporte.

\begin{codecell}
!pip install yfinance
import yfinance as yf
import pandas as pd
import numpy as np
import matplotlib.pyplot as plt
from scipy.stats import norm

# Definir le ticker et une longue periode pour notre "population"
ticker = "GOOG"
start_date = "2010-01-01"
end_date = "2024-12-31"

# Telecharger les prix de cloture ajustes
data = yf.download(ticker, start=start_date, end=end_date)["Adj Close"]

# Calculer les rendements journaliers (notre population X)
returns = data.pct_change().dropna()

# Calculer les "vrais" parametres de la population
population_mean = returns.mean()
population_var = returns.var()
population_std = returns.std()

print(f"--- Parametres de la Population (GOOG 2010-2024) ---")
print(f"Mu (moyenne) = {population_mean:.6f}")
print(f"Sigma^2 (variance) = {population_var:.6f}")
print(f"Sigma (ecart-type) = {population_std:.6f}")
print(f"N total (population) = {len(returns)}")
\end{codecell}

\begin{exercicebox}[Exercice 1 : Distribution de la Population vs. TCL]
La population des rendements journaliers $X_i$ n'est pas parfaitement normale. La distribution de $\bar{X}_n$, en revanche, devrait l'être.

\textbf{Votre tâche :}
\begin{enumerate}
    \item \textbf{(Plot)} Tracer l'histogramme de la \textbf{population} (la série \texttt{returns} complète).
    \item (Conclusion) La distribution de la population ressemble-t-elle à une loi normale parfaite ? (Regardez les queues).
\end{enumerate}
\end{exercicebox}

\begin{exercicebox}[Exercice 2 : Paramètres de la Distribution d'Échantillonnage (Théorie)]
Le TCL prédit les paramètres de la distribution de $\bar{X}_n$.
Supposons que nous prenions des échantillons de taille $n=30$.

\textbf{Votre tâche :}
\begin{enumerate}
    \item Calculer l'espérance \textbf{théorique} de la moyenne d'échantillon, $E[\bar{X}_{30}]$.
    \item Calculer la variance \textbf{théorique} de la moyenne d'échantillon, $\text{Var}(\bar{X}_{30}) = \sigma^2 / n$.
    \item Calculer l'erreur standard \textbf{théorique} (l'écart-type) de la moyenne d'échantillon, $\sigma_{\bar{X}_{30}} = \sigma / \sqrt{n}$.
    \item (Utilisez les $\mu$ et $\sigma^2$ de la population calculés dans la cellule de setup).
\end{enumerate}
\end{exercicebox}

\begin{exercicebox}[Exercice 3 : Simulation de la Distribution d'Échantillonnage]
Vérifions le TCL par simulation. Nous allons générer $k=1000$ échantillons de taille $n=30$ et calculer la moyenne de chacun.

\textbf{Votre tâche :}
\begin{enumerate}
    \item Créer une liste (ou un array) vide \texttt{sample\_means}.
    \item Boucler $k=1000$ fois :
        \begin{itemize}
            \item Tirer un échantillon de $n=30$ rendements de la \texttt{returns} (avec \texttt{np.random.choice(..., size=30, replace=True)}).
            \item Calculer la moyenne de cet échantillon.
            \item Ajouter cette moyenne à votre liste \texttt{sample\_means}.
        \end{itemize}
    \item Vous avez maintenant 1000 valeurs de $\bar{X}_{30}$.
\end{enumerate}
\end{exercicebox}

\begin{exercicebox}[Exercice 4 : Vérification des Paramètres (Empirique vs. Théorie)]
Utilisons les 1000 moyennes d'échantillon (\texttt{sample\_means}) de l'exercice 3.

\textbf{Votre tâche :}
\begin{enumerate}
    \item Calculer la moyenne \textbf{empirique} des 1000 moyennes d'échantillon.
    \item Comparer ce résultat à l'espérance \textbf{théorique} $E[\bar{X}_{30}]$ de l'exercice 2.
    \item Calculer la variance \textbf{empirique} des 1000 moyennes d'échantillon.
    \item Comparer ce résultat à la variance \textbf{théorique} $\text{Var}(\bar{X}_{30})$ de l'exercice 2.
\end{enumerate}
\end{exercicebox}

\begin{exercicebox}[Exercice 5 : Visualisation du TCL (Plot)]
C'est la visualisation la plus importante. Nous allons superposer la distribution empirique des moyennes (de l'Ex 3) et la distribution normale théorique (de l'Ex 2).

\textbf{Votre tâche :}
\begin{enumerate}
    \item \textbf{(Plot)} Tracer l'histogramme des 1000 \texttt{sample\_means} (de l'Ex 3). Assurez-vous d'utiliser \texttt{density=True}.
    \item \textbf{(Plot)} Sur le \textbf{même} graphique, tracer la PDF de la loi normale \textbf{théorique} prédite par le TCL.
    \item (Indice : $X \sim \mathcal{N}(\mu, \sigma^2/n)$. Utilisez \texttt{scipy.stats.norm.pdf()} avec \texttt{loc = population\_mean} et \texttt{scale = (population\_std / np.sqrt(30))}).
    \item (Conclusion) La prédiction du TCL correspond-elle à la simulation ?
\end{enumerate}
\end{exercicebox}

\begin{exercicebox}[Exercice 6 : L'Effet de la Taille $n$ (Plot)]
La convergence est plus rapide lorsque $n$ est grand.

\textbf{Votre tâche :}
\begin{enumerate}
    \item Répéter l'exercice 3, mais pour $n=5$ (créer 1000 \texttt{sample\_means\_5}).
    \item Répéter l'exercice 3, mais pour $n=100$ (créer 1000 \texttt{sample\_means\_100}).
    \item \textbf{(Plot)} Créer trois histogrammes côte à côte (\texttt{plt.subplot}) pour $n=5$, $n=30$ (de l'Ex 3), et $n=100$.
    \item (Conclusion) Que constatez-vous à propos de la variance (largeur de la cloche) de la distribution de $\bar{X}_n$ lorsque $n$ augmente ?
\end{enumerate}
\end{exercicebox}

\begin{exercicebox}[Exercice 7 : Application du TCL (Calcul de Probabilité pour $\bar{X}_n$)]
Utilisons le TCL pour répondre à une question pratique.
Quelle est la probabilité que le rendement \textbf{moyen} sur un mois de trading ($n=21$ jours) soit positif ?

\textbf{Votre tâche :}
\begin{enumerate}
    \item On cherche $P(\bar{X}_{21} > 0)$.
    \item Identifier $\mu$ et $\sigma$ (de la population).
    \item Calculer l'erreur standard $\sigma_{\bar{X}_{21}} = \sigma / \sqrt{21}$.
    \item Standardiser la valeur $0$ : $Z = (0 - \mu) / \sigma_{\bar{X}_{21}}$.
    \item Calculer la probabilité $P(Z > z) = 1 - \Phi(z)$ en utilisant \texttt{scipy.stats.norm.cdf()}.
\end{enumerate}
\end{exercicebox}

\begin{exercicebox}[Exercice 8 : Application du TCL (Calcul de Probabilité pour $S_n$)]
Quelle est la probabilité que le rendement \textbf{total} (la somme) sur une année ($n=252$ jours) soit supérieur à 10\% ?

\textbf{Votre tâche :}
\begin{enumerate}
    \item On cherche $P(S_{252} > 0.10)$.
    \item Le TCL dit $S_n \approx \mathcal{N}(n\mu, n\sigma^2)$.
    \item Calculer l'espérance de la somme : $E[S_{252}] = n\mu = 252 \times \mu$.
    \item Calculer l'écart-type de la somme : $\sigma_{S_{252}} = \sigma \sqrt{n} = \sigma \times \sqrt{252}$.
    \item Standardiser la valeur $0.10$ : $Z = (0.10 - E[S_{252}]) / \sigma_{S_{252}}$.
    \item Calculer la probabilité $P(Z > z) = 1 - \Phi(z)$.
\end{enumerate}
\end{exercicebox}

\begin{exercicebox}[Exercice 9 : TCL pour les Proportions (Binomiale)]
Le TCL s'applique aussi aux proportions (qui sont des moyennes de Bernoulli). Soit $p$ la probabilité qu'un jour soit un "jour de hausse" (rendement > 0).

\textbf{Votre tâche :}
\begin{enumerate}
    \item Estimer la "vraie" proportion $p$ (notre $\mu$) en calculant la proportion de jours de hausse dans toute la population \texttt{returns}.
    \item On sonde $n=100$ jours. Quelle est la probabilité que notre sondage ($\hat{p} = \bar{X}_{100}$) montre une majorité de jours de baisse ($\hat{p} < 0.5$) ?
    \item L'erreur standard pour une proportion est $\sigma_{\hat{p}} = \sqrt{p(1-p) / n}$.
    \item Standardiser 0.5 : $Z = (0.5 - p) / \sigma_{\hat{p}}$.
    \item Calculer la probabilité $P(Z < z) = \Phi(z)$.
\end{enumerate}
\end{exercicebox}

\begin{exercicebox}[Exercice 10 : Marge d'Erreur (Intervalle de Confiance)]
C'est l'application la plus courante du TCL dans les médias.
Quelle est la "marge d'erreur" à 95\% pour notre estimation de $p$ (proportion de jours de hausse) si l'on utilise un échantillon de $n=1000$ jours ?

\textbf{Votre tâche :}
\begin{enumerate}
    \item Utiliser le $p$ (proportion de la population) estimé à l'exercice 9.
    \item Calculer l'erreur standard pour $n=1000$ : $SE = \sqrt{p(1-p) / 1000}$.
    \item La marge d'erreur à 95\% est $ME = 1.96 \times SE$. (Car $P(-1.96 \le Z \le 1.96) \approx 0.95$).
    \item (Conclusion) Interpréter le résultat : "Notre estimation de la proportion de jours de hausse sera correcte à $\pm$ [ME] près, 95\% du temps."
\end{enumerate}
\end{exercicebox}