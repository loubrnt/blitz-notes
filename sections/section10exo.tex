\subsection{Exercices Python}

Ces exercices appliquent le Théorème Central Limite (TCL) aux données financières. Nous considérerons l'ensemble des rendements journaliers de Google (GOOG) sur une longue période comme notre \textbf{population} (dont nous connaissons le "vrai" $\mu$ et $\sigma^2$). Nous simulerons ensuite un \textbf{échantillonnage} (tirer $n$ jours au hasard) pour voir comment la \textbf{distribution d'échantillonnage de la moyenne} ($\bar{X}_n$) se comporte.

\begin{codecell}
!pip install yfinance
import yfinance as yf
import pandas as pd
import numpy as np
import matplotlib.pyplot as plt
from scipy.stats import norm

# Definir le ticker et une longue periode pour notre "population"
ticker = "GOOG"
start_date = "2010-01-01"
end_date = "2024-12-31"

# Telecharger les prix de cloture ajustes
data = yf.download(ticker, start=start_date, end=end_date)["Adj Close"]

# Calculer les rendements journaliers (notre population X)
returns = data.pct_change().dropna()

# Calculer les "vrais" parametres de la population
population_mean = returns.mean()
population_var = returns.var()
population_std = returns.std()

print(f"--- Parametres de la Population (GOOG 2010-2024) ---")
print(f"Mu (moyenne) = {population_mean:.6f}")
print(f"Sigma^2 (variance) = {population_var:.6f}")
print(f"Sigma (ecart-type) = {population_std:.6f}")
print(f"N total (population) = {len(returns)}")
\end{codecell}

\begin{exercicebox}[Exercice 1 : Distribution de la Population vs. TCL]
La population des rendements journaliers $X_i$ n'est pas parfaitement normale. La distribution de $\bar{X}_n$, en revanche, devrait l'être.

\textbf{Votre tâche :}
\begin{enumerate}
    \item \textbf{(Plot)} Tracer l'histogramme de la \textbf{population} (la série \texttt{returns} complète).
    \item (Conclusion) La distribution de la population ressemble-t-elle à une loi normale parfaite ? (Regardez les queues).
\end{enumerate}
\end{exercicebox}

\begin{exercicebox}[Exercice 2 : Paramètres de la Distribution d'Échantillonnage (Théorie)]
Le TCL prédit les paramètres de la distribution de $\bar{X}_n$.
Supposons que nous prenions des échantillons de taille $n=30$.

\textbf{Votre tâche :}
\begin{enumerate}
    \item Calculer l'espérance \textbf{théorique} de la moyenne d'échantillon, $E[\bar{X}_{30}]$.
    \item Calculer la variance \textbf{théorique} de la moyenne d'échantillon, $\text{Var}(\bar{X}_{30}) = \sigma^2 / n$.
    \item Calculer l'erreur standard \textbf{théorique} (l'écart-type) de la moyenne d'échantillon, $\sigma_{\bar{X}_{30}} = \sigma / \sqrt{n}$.
    \item (Utilisez les $\mu$ et $\sigma^2$ de la population calculés dans la cellule de setup).
\end{enumerate}
\end{exercicebox}

\begin{exercicebox}[Exercice 3 : Simulation de la Distribution d'Échantillonnage]
Vérifions le TCL par simulation. Nous allons générer $k=1000$ échantillons de taille $n=30$ et calculer la moyenne de chacun.

\textbf{Votre tâche :}
\begin{enumerate}
    \item Créer une liste (ou un array) vide \texttt{sample\_means}.
    \item Boucler $k=1000$ fois :
        \begin{itemize}
            \item Tirer un échantillon de $n=30$ rendements de la \texttt{returns} (avec \texttt{np.random.choice(..., size=30, replace=True)}).
            \item Calculer la moyenne de cet échantillon.
            \item Ajouter cette moyenne à votre liste \texttt{sample\_means}.
        \end{itemize}
    \item Vous avez maintenant 1000 valeurs de $\bar{X}_{30}$.
\end{enumerate}
\end{exercicebox}

\begin{exercicebox}[Exercice 4 : Vérification des Paramètres (Empirique vs. Théorie)]
Utilisons les 1000 moyennes d'échantillon (\texttt{sample\_means}) de l'exercice 3.

\textbf{Votre tâche :}
\begin{enumerate}
    \item Calculer la moyenne \textbf{empirique} des 1000 moyennes d'échantillon.
    \item Comparer ce résultat à l'espérance \textbf{théorique} $E[\bar{X}_{30}]$ de l'exercice 2.
    \item Calculer la variance \textbf{empirique} des 1000 moyennes d'échantillon.
    \item Comparer ce résultat à la variance \textbf{théorique} $\text{Var}(\bar{X}_{30})$ de l'exercice 2.
\end{enumerate}
\end{exercicebox}

\begin{exercicebox}[Exercice 5 : Visualisation du TCL (Plot)]
C'est la visualisation la plus importante. Nous allons superposer la distribution empirique des moyennes (de l'Ex 3) et la distribution normale théorique (de l'Ex 2).

\textbf{Votre tâche :}
\begin{enumerate}
    \item \textbf{(Plot)} Tracer l'histogramme des 1000 \texttt{sample\_means} (de l'Ex 3). Assurez-vous d'utiliser \texttt{density=True}.
    \item \textbf{(Plot)} Sur le \textbf{même} graphique, tracer la PDF de la loi normale \textbf{théorique} prédite par le TCL.
    \item (Indice : $X \sim \mathcal{N}(\mu, \sigma^2/n)$. Utilisez \texttt{scipy.stats.norm.pdf()} avec \texttt{loc = population\_mean} et \texttt{scale = (population\_std / np.sqrt(30))}).
    \item (Conclusion) La prédiction du TCL correspond-elle à la simulation ?
\end{enumerate}
\end{exercicebox}

\begin{exercicebox}[Exercice 6 : L'Effet de la Taille $n$ (Plot)]
La convergence est plus rapide lorsque $n$ est grand.

\textbf{Votre tâche :}
\begin{enumerate}
    \item Répéter l'exercice 3, mais pour $n=5$ (créer 1000 \texttt{sample\_means\_5}).
    \item Répéter l'exercice 3, mais pour $n=100$ (créer 1000 \texttt{sample\_means\_100}).
    \item \textbf{(Plot)} Créer trois histogrammes côte à côte (\texttt{plt.subplot}) pour $n=5$, $n=30$ (de l'Ex 3), et $n=100$.
    \item (Conclusion) Que constatez-vous à propos de la variance (largeur de la cloche) de la distribution de $\bar{X}_n$ lorsque $n$ augmente ?
\end{enumerate}
\end{exercicebox}

\begin{exercicebox}[Exercice 7 : Application du TCL (Calcul de Probabilité pour $\bar{X}_n$)]
Utilisons le TCL pour répondre à une question pratique.
Quelle est la probabilité que le rendement \textbf{moyen} sur un mois de trading ($n=21$ jours) soit positif ?

\textbf{Votre tâche :}
\begin{enumerate}
    \item On cherche $P(\bar{X}_{21} > 0)$.
    \item Identifier $\mu$ et $\sigma$ (de la population).
    \item Calculer l'erreur standard $\sigma_{\bar{X}_{21}} = \sigma / \sqrt{21}$.
    \item Standardiser la valeur $0$ : $Z = (0 - \mu) / \sigma_{\bar{X}_{21}}$.
    \item Calculer la probabilité $P(Z > z) = 1 - \Phi(z)$ en utilisant \texttt{scipy.stats.norm.cdf()}.
\end{enumerate}
\end{exercicebox}

\begin{exercicebox}[Exercice 8 : Application du TCL (Calcul de Probabilité pour $S_n$)]
Quelle est la probabilité que le rendement \textbf{total} (la somme) sur une année ($n=252$ jours) soit supérieur à 10\% ?

\textbf{Votre tâche :}
\begin{enumerate}
    \item On cherche $P(S_{252} > 0.10)$.
    \item Le TCL dit $S_n \approx \mathcal{N}(n\mu, n\sigma^2)$.
    \item Calculer l'espérance de la somme : $E[S_{252}] = n\mu = 252 \times \mu$.
    \item Calculer l'écart-type de la somme : $\sigma_{S_{252}} = \sigma \sqrt{n} = \sigma \times \sqrt{252}$.
    \item Standardiser la valeur $0.10$ : $Z = (0.10 - E[S_{252}]) / \sigma_{S_{252}}$.
    \item Calculer la probabilité $P(Z > z) = 1 - \Phi(z)$.
\end{enumerate}
\end{exercicebox}

\begin{exercicebox}[Exercice 9 : TCL pour les Proportions (Binomiale)]
Le TCL s'applique aussi aux proportions (qui sont des moyennes de Bernoulli). Soit $p$ la probabilité qu'un jour soit un "jour de hausse" (rendement > 0).

\textbf{Votre tâche :}
\begin{enumerate}
    \item Estimer la "vraie" proportion $p$ (notre $\mu$) en calculant la proportion de jours de hausse dans toute la population \texttt{returns}.
    \item On sonde $n=100$ jours. Quelle est la probabilité que notre sondage ($\hat{p} = \bar{X}_{100}$) montre une majorité de jours de baisse ($\hat{p} < 0.5$) ?
    \item L'erreur standard pour une proportion est $\sigma_{\hat{p}} = \sqrt{p(1-p) / n}$.
    \item Standardiser 0.5 : $Z = (0.5 - p) / \sigma_{\hat{p}}$.
    \item Calculer la probabilité $P(Z < z) = \Phi(z)$.
\end{enumerate}
\end{exercicebox}

\begin{exercicebox}[Exercice 10 : Marge d'Erreur (Intervalle de Confiance)]
C'est l'application la plus courante du TCL dans les médias.
Quelle est la "marge d'erreur" à 95\% pour notre estimation de $p$ (proportion de jours de hausse) si l'on utilise un échantillon de $n=1000$ jours ?

\textbf{Votre tâche :}
\begin{enumerate}
    \item Utiliser le $p$ (proportion de la population) estimé à l'exercice 9.
    \item Calculer l'erreur standard pour $n=1000$ : $SE = \sqrt{p(1-p) / 1000}$.
    \item La marge d'erreur à 95\% est $ME = 1.96 \times SE$. (Car $P(-1.96 \le Z \le 1.96) \approx 0.95$).
    \item (Conclusion) Interpréter le résultat : "Notre estimation de la proportion de jours de hausse sera correcte à $\pm$ [ME] près, 95\% du temps."
\end{enumerate}
\end{exercicebox}