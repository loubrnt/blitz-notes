\subsection{Exercices Python}

Les exercices suivants appliquent les concepts de dénombrement et de probabilité au célèbre jeu de données "Titanic". Ce dataset, chargé via la bibliothèque \texttt{seaborn}, contient des informations démographiques et de voyage sur les passagers du navire.

Le bloc de code ci-dessous initialise notre environnement en chargeant les données dans un DataFrame Pandas \texttt{df}. Pour garantir la consistance de nos calculs, nous définirons notre univers $S$ comme l'ensemble des passagers pour lesquels l'âge est connu (en supprimant les lignes avec un âge manquant).

\begin{codecell}
import pandas as pd
import seaborn as sns
import math

# Charger le dataset Titanic
df = sns.load_dataset("titanic")

# On retire les lignes ou l'age est inconnu pour simplifier les calculs
# C'est notre Univers S.
df = df.dropna(subset=["age"]) 

\end{codecell}

\begin{exercicebox}[Exercice 1 : Probabilité Naïve (Filtre multiple)]
Quelle est la probabilité qu'un passager, sélectionné au hasard dans l'univers \texttt{df}, soit un homme de plus de 40 ans ET voyageant en troisième classe ?

\textbf{Votre tâche :}
\begin{enumerate}
    \item Trouver $|S|$, la taille totale de l'univers \texttt{df}.
    \item Trouver $|A|$, le nombre de passagers remplissant les trois conditions (\texttt{sex == 'male'}, \texttt{age > 40}, \texttt{pclass == 3}).
    \item Calculer $P(A) = |A| / |S|$.
\end{enumerate}
\end{exercicebox}

\begin{exercicebox}[Exercice 2 : Dénombrement par Combinaisons ($\binom{n}{k}$)]
Pour une enquête de satisfaction, on veut créer un groupe de discussion (un "comité") composé de 5 personnes. Ces 5 personnes doivent être choisies exclusivement parmi les passagers ayant embarqué à Southampton (\texttt{embark\_town == 'Southampton'}).

Combien de comités uniques de 5 personnes est-il possible de former ?

\textbf{Votre tâche :}
\begin{enumerate}
    \item Trouver $n$, le nombre de passagers ayant embarqué à Southampton.
    \item Définir $k=5$.
    \item Calculer $\binom{n}{k}$ (par ex., avec \texttt{math.comb}).
\end{enumerate}
\end{exercicebox}

\begin{exercicebox}[Exercice 3 : Dénombrement par Permutations]
Lors d'un exercice de sécurité, on demande à 4 enfants (passagers de 12 ans ou moins) de s'aligner pour une photo de communication.

En supposant que l'on choisisse 4 enfants au hasard parmi tous les enfants du navire, et que l'ordre dans lequel ils sont alignés pour la photo est important, combien d'alignements différents sont possibles ?

\textbf{Votre tâche :}
\begin{enumerate}
    \item Trouver $n$, le nombre total d'enfants (âge $\le 12$) à bord.
    \item Définir $k=4$.
    \item Calculer $P(n, k)$ (par ex., avec \texttt{math.perm}).
\end{enumerate}
\end{exercicebox}

\begin{exercicebox}[Exercice 4 : Principe d'Inclusion-Exclusion]
Quelle est la probabilité qu'un passager sélectionné au hasard soit \textbf{soit un survivant} (ensemble $A$), \textbf{soit un passager de première classe} (ensemble $B$) (ou les deux) ?

\textbf{Votre tâche :}
\begin{enumerate}
    \item Trouver $|S|$.
    \item Trouver $|A|$ (nombre de survivants).
    \item Trouver $|B|$ (nombre de passagers en 1ère classe).
    \item Trouver $|A \cap B|$ (survivants de 1ère classe).
    \item Appliquer la formule : $P(A \cup B) = P(A) + P(B) - P(A \cap B)$.
\end{enumerate}
\end{exercicebox}

\begin{exercicebox}[Exercice 5 : Probabilité (Tirage sans remise)]
On sélectionne au hasard un échantillon de 10 passagers de l'univers \texttt{df}.

Quelle est la probabilité que cet échantillon contienne exactement \textbf{4 survivants} et \textbf{6 non-survivants} ?

\textbf{Votre tâche :}
\begin{enumerate}
    \item Trouver $N = |S|$, le nombre total de passagers.
    \item Trouver $m$, le nombre total de survivants dans $S$.
    \item Trouver $p$, le nombre total de non-survivants dans $S$.
    \item Calculer le dénominateur : $\binom{N}{10}$ (façons de choisir 10 passagers).
    \item Calculer le numérateur : $\binom{m}{4} \times \binom{p}{6}$ (façons de choisir 4 survivants ET 6 non-survivants).
    \item Calculer la probabilité (numérateur / dénominateur).
\end{enumerate}
\end{exercicebox}

\begin{exercicebox}[Exercice 6 : Probabilité Complémentaire]
On sélectionne au hasard un groupe de 5 passagers. Quelle est la probabilité que ce groupe contienne \textbf{au moins un} passager voyageant seul (\texttt{alone == True}) ?

\textbf{Votre tâche :}
\begin{enumerate}
    \item Calculer $P(E^c)$, la probabilité de l'événement complémentaire "aucun passager ne voyage seul".
    \item Trouver $N = |S|$, le nombre total de passagers.
    \item Trouver $n_{\text{non-seul}}$, le nombre de passagers qui ne voyagent *pas* seuls.
    \item Dénominateur $D = \binom{N}{5}$ (façons de choisir 5 passagers).
    \item Numérateur $N_c = \binom{n_{\text{non-seul}}}{5}$ (façons de choisir 5 passagers non-seuls).
    \item $P(E^c) = N_c / D$.
    \item Calculer le résultat final : $P(E) = 1 - P(E^c)$.
\end{enumerate}
\end{exercicebox}

\begin{exercicebox}[Exercice 7 : Probabilité et Analyse de Données]
Calculez la probabilité qu'un passager, choisi au hasard, ait payé un tarif (\texttt{fare}) supérieur au \textbf{tarif moyen} de l'ensemble du navire (\texttt{df}).

\textbf{Votre tâche :}
\begin{enumerate}
    \item Calculer le tarif moyen de tous les passagers dans \texttt{df}.
    \item Trouver $|S|$, la taille totale de l'univers \texttt{df}.
    \item Trouver $|A|$, le nombre de passagers dont le \texttt{fare} est strictement supérieur à ce tarif moyen.
    \item Calculer $P(A) = |A| / |S|$.
\end{enumerate}
\end{exercicebox}