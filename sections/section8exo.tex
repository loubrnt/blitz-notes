\subsection{Exercices Python}

La loi log-normale est fondamentale en finance. Elle repose sur l'idée que si les \textbf{log-rendements} d'une action $X_i = \ln(P_i / P_{i-1})$ sont (approximativement) normaux, alors le prix futur $P_t$, qui est un \textbf{produit} de ces rendements ($P_t = P_0 \times e^{X_1} \times \dots \times e^{X_t}$), suivra une loi log-normale.

Nous allons estimer les paramètres $\mu$ et $\sigma^2$ de la loi normale sous-jacente à partir des log-rendements journaliers de Microsoft (MSFT) et Google (GOOG), puis utiliser la théorie log-normale pour modéliser les prix.

\begin{codecell}
!pip install yfinance
import yfinance as yf
import pandas as pd
import numpy as np
from scipy.stats import norm # Moteur pour les calculs de CDF/PDF
import matplotlib.pyplot as plt

# Definir les tickers et la periode
tickers = ["MSFT", "GOOG"]
start_date = "2020-01-01"
end_date = "2024-12-31"

# Telecharger les prix de cloture ajustes
data = yf.download(tickers, start=start_date, end=end_date)["Adj Close"]

# Calculer les LOG-RENDEMENTS journaliers
log_returns = np.log(data / data.shift(1)).dropna()

# Renommer les colonnes
log_returns.columns = ["MSFT_LogReturn", "GOOG_LogReturn"]

# 'log_returns' est notre DataFrame.
# X_msft = log_returns["MSFT_LogReturn"]
# X_goog = log_returns["GOOG_LogReturn"]
\end{codecell}

\begin{exercicebox}[Exercice 1 : Estimer les Paramètres $\mu$ et $\sigma^2$]
Soit $P_t$ le prix de MSFT. Le modèle suppose que $X = \ln(P_t/P_{t-1}) \sim \mathcal{N}(\mu, \sigma^2)$. Les paramètres $\mu$ et $\sigma^2$ sont les paramètres "log-normaux".

\textbf{Votre tâche :}
\begin{enumerate}
    \item Estimer $\mu$ (l'espérance du log-rendement journalier) pour MSFT.
    \item Estimer $\sigma^2$ (la variance du log-rendement journalier) pour MSFT.
    \item Estimer $\sigma$ (l'écart-type du log-rendement journalier) pour MSFT.
\end{enumerate}
\end{exercicebox}

\begin{exercicebox}[Exercice 2 : Test de Normalité (Graphique)]
La théorie log-normale repose sur la normalité des log-rendements $X$. Vérifions-le visuellement.

\textbf{Votre tâche :}
\begin{enumerate}
    \item Utiliser $\mu$ et $\sigma$ (pour MSFT) de l'Exercice 1.
    \item \textbf{(Plot)} Tracer l'histogramme des log-rendements \textbf{empiriques} de MSFT (Indice : \texttt{plt.hist(..., density=True, bins=50)}).
    \item \textbf{(Plot)} Superposer la PDF \textbf{théorique} de la loi normale $\mathcal{N}(\mu, \sigma^2)$ sur cet histogramme.
    \item (Indice : Créez un \texttt{np.linspace}, calculez la PDF avec \texttt{norm.pdf(x, loc=mu, scale=sigma)}, puis \texttt{plt.plot()}).
    \item (Conclusion) La cloche théorique s'ajuste-t-elle bien aux données réelles ?
\end{enumerate}
\end{exercicebox}

\begin{exercicebox}[Exercice 3 : Asymétrie (Prix vs Log-Rendements)]
La théorie dit que les log-rendements $X$ sont symétriques (Normaux), mais que les prix $P_t$ sont asymétriques à droite (Log-Normaux).

\textbf{Votre tâche :}
\begin{enumerate}
    \item Calculer la moyenne et la médiane de la série des \textbf{log-rendements} de MSFT.
    \item Calculer la moyenne et la médiane de la série des \textbf{prix} de MSFT (la colonne \texttt{data['MSFT']}).
    \item Comparer les deux paires. Les log-rendements sont-ils symétriques (moyenne $\approx$ médiane) ? Les prix sont-ils asymétriques (moyenne $>$ médiane) ?
    \item \textbf{(Plot)} Créer deux histogrammes côte à côte (\texttt{plt.subplot}) pour visualiser la distribution des log-rendements et celle des prix.
\end{enumerate}
\end{exercicebox}

\begin{exercicebox}[Exercice 4 : Espérance vs Médiane (Théorique)]
Soit $Y = P_t/P_{t-1} = e^X$ la variable "ratio de prix journalier". $Y \sim \text{Log-}\mathcal{N}(\mu, \sigma^2)$.
Théorie : $\text{Med}(Y) = e^{\mu}$ et $E[Y] = e^{\mu + \sigma^2/2}$.

\textbf{Votre tâche :}
\begin{enumerate}
    \item Utiliser $\mu$ et $\sigma^2$ (pour MSFT) de l'Exercice 1.
    \item Calculer la médiane \textbf{théorique} $\text{Med}(Y)$.
    \item Calculer l'espérance \textbf{théorique} $E[Y]$.
    \item Vérifier que $E[Y] > \text{Med}(Y)$, confirmant l'asymétrie.
\end{enumerate}
\end{exercicebox}

\begin{exercicebox}[Exercice 5 : Espérance Théorique vs Empirique]
Vérifions le calcul de $E[Y]$ de l'exercice 4 de manière empirique.

\textbf{Votre tâche :}
\begin{enumerate}
    \item Créer la série $Y$ (ratio de prix journalier) : $Y = \exp(X_{\text{msft}})$.
    \item Calculer l'espérance \textbf{empirique} de $Y$ (la moyenne de cette série $Y$).
    \item Comparer cette valeur empirique à l'espérance \textbf{théorique} $e^{\mu + \sigma^2/2}$ calculée à l'exercice 4.
\end{enumerate}
\end{exercicebox}

\begin{exercicebox}[Exercice 6 : Variance Théorique vs Empirique]
Théorie : $\text{Var}(Y) = (e^{\sigma^2} - 1) \cdot e^{2\mu + \sigma^2}$.

\textbf{Votre tâche :}
\begin{enumerate}
    \item Utiliser $\mu$ et $\sigma^2$ (pour MSFT) de l'Exercice 1.
    \item Calculer la variance \textbf{théorique} $\text{Var}(Y)$ en utilisant la formule ci-dessus.
    \item Calculer la variance \textbf{empirique} de la série $Y$ (créée à l'Ex 5).
    \item Comparer les deux résultats.
\end{enumerate}
\end{exercicebox}

\begin{exercicebox}[Exercice 7 : Modélisation du Prix Futur (Paramètres)]
Modélisons le prix de GOOG dans $t=20$ jours ouvrés (environ 1 mois).
Le prix $P_{20}$ est log-normal si l'on suppose $P_{20} = P_0 \cdot e^{X_{20}}$, où $P_0$ est le prix actuel.
Le log-rendement total $X_{20} = \ln(P_{20}/P_0)$ suit $X_{20} \sim \mathcal{N}(t\mu, t\sigma^2)$.

\textbf{Votre tâche :}
\begin{enumerate}
    \item Estimer $\mu_G$ et $\sigma_G^2$ (journaliers) pour GOOG (similaire à l'Ex 1).
    \item Définir $t=20$.
    \item Calculer $\mu_{20} = t\mu_G$ (l'espérance du log-rendement sur 20 jours).
    \item Calculer $\sigma_{20}^2 = t\sigma_G^2$ (la variance du log-rendement sur 20 jours).
\end{enumerate}
\end{exercicebox}

\begin{exercicebox}[Exercice 8 : Calcul de Probabilité (Prix Futur)]
En utilisant les paramètres $\mu_{20}$ et $\sigma_{20} = \sqrt{\sigma_{20}^2}$ de l'exercice 7 pour GOOG :

\textbf{Votre tâche :}
\begin{enumerate}
    \item Calculer la probabilité que GOOG ait un rendement positif sur 20 jours.
    \item On cherche $P(P_{20} > P_0) \implies P(P_{20}/P_0 > 1) \implies P(\ln(P_{20}/P_0) > \ln(1))$.
    \item Calculer $P(X_{20} > 0)$.
    \item (Indice : Standardiser 0 avec $\mu_{20}$ et $\sigma_{20}$, puis utiliser $1 - \Phi(z)$).
    \item \textbf{(Plot)} Tracer la PDF de $X_{20} \sim \mathcal{N}(\mu_{20}, \sigma_{20}^2)$ et hachurer la zone $x > 0$.
\end{enumerate}
\end{exercicebox}

\begin{exercicebox}[Exercice 9 : Calcul de Probabilité (Perte > 5\%)]
En utilisant les paramètres $\mu_{20}$ et $\sigma_{20}$ de l'exercice 7 pour GOOG :

\textbf{Votre tâche :}
\begin{enumerate}
    \item Calculer la probabilité que GOOG perde plus de 5\% sur 20 jours.
    \item On cherche $P(P_{20} < 0.95 \times P_0) \implies P(P_{20}/P_0 < 0.95)$.
    \item Calculer $P(X_{20} < \ln(0.95))$.
    \item (Indice : Standardiser $\ln(0.95)$ avec $\mu_{20}$ et $\sigma_{20}$, puis utiliser $\Phi(z)$).
    \item \textbf{(Plot)} Tracer la PDF de $X_{20}$ et hachurer la zone $x < \ln(0.95)$.
\end{enumerate}
\end{exercicebox}

\begin{exercicebox}[Exercice 10 : Problème Inverse (Intervalle de Confiance)]
Trouvons l'intervalle de 95\% pour le prix de GOOG dans 20 jours.
Nous cherchons les bornes $y_1, y_2$ telles que $P(y_1 \le P_{20} \le y_2) = 0.95$.
On suppose un intervalle centré sur la loi normale sous-jacente (entre $z=-1.96$ et $z=+1.96$).

\textbf{Votre tâche :}
\begin{enumerate}
    \item Trouver $z_{inf} = -1.96$ et $z_{sup} = +1.96$.
    \item "Dé-standardiser" ces Z-scores pour trouver les log-rendements $x_1$ et $x_2$ :
        $x = \mu_t + z \sigma_t$ (en utilisant $\mu_{20}$ et $\sigma_{20}$ de l'Ex 7).
    \item Convertir ces log-rendements en ratios de prix $y = e^x$.
    \item (Conclusion) L'intervalle de 95\% pour le ratio de prix est $[y_1, y_2]$.
\end{enumerate}
\end{exercicebox}

\begin{exercicebox}[Exercice 11 : Calcul de la Médiane vs Espérance (Prix Futur)]
Pour le prix de GOOG dans 20 jours, $P_{20} = P_0 \cdot Y_{20}$, où $Y_{20} \sim \text{Log-}\mathcal{N}(\mu_{20}, \sigma_{20}^2)$.

\textbf{Votre tâche :}
\begin{enumerate}
    \item Calculer le ratio de prix \textbf{médian} attendu : $\text{Med}(Y_{20}) = e^{\mu_{20}}$.
    \item Calculer le ratio de prix \textbf{moyen} (espérance) attendu : $E[Y_{20}] = e^{\mu_{20} + \sigma_{20}^2 / 2}$.
    \item (Conclusion) Lequel est le plus élevé ? Pourquoi est-ce important pour un investisseur ?
\end{enumerate}
\end{exercicebox}