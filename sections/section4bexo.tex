\subsection{Exercices Python}

Les exercices suivants appliquent les concepts de la loi Gamma au jeu de données "Tips" (pourboires) de Seaborn, qui contient les montants des additions dans un restaurant.

\begin{codecell}
import pandas as pd
import seaborn as sns
import numpy as np
import math
import matplotlib.pyplot as plt

# Charger le dataset Tips
df = sns.load_dataset("tips")

# La colonne total_bill represente notre variable positive continue
# df est maintenant notre ensemble d observations
\end{codecell}

\begin{exercicebox}[Exercice 1 : Identification et statistiques descriptives]
Soit $Y$ la variable aléatoire représentant le montant total de l addition (\texttt{total\_bill}).

\textbf{Votre tâche :}
\begin{enumerate}
    \item Afficher les 5 premieres lignes du DataFrame pour comprendre la structure.
    \item Calculer la taille de l echantillon $n$ (nombre de factures).
    \item Calculer le montant minimum, maximum, la moyenne empirique et la variance empirique de \texttt{total\_bill}.
    \item Expliquer pourquoi la loi Gamma est un choix plausible pour modeliser cette variable.
\end{enumerate}
\end{exercicebox}

\begin{exercicebox}[Exercice 2 : Estimation des parametres k et lambda]
Pour ajuster une loi Gamma $\text{Gamma}(k, \lambda)$ aux données, nous estimons les parametres a partir des statistiques empiriques.

\textbf{Votre tâche :}
\begin{enumerate}
    \item En utilisant la relation $\mathbb{E}[Y] = \frac{k}{\lambda}$ et $\text{Var}(Y) = \frac{k}{\lambda^2}$, calculer les estimateurs :
        \[ \hat{k} = \frac{\bar{y}^2}{s^2} \quad \text{et} \quad \hat{\lambda} = \frac{\bar{y}}{s^2} \]
        où $\bar{y}$ est la moyenne empirique et $s^2$ la variance empirique.
    \item Arrondir $\hat{k}$ et $\hat{\lambda}$ à 3 decimales.
    \item Stocker ces valeurs dans des variables \texttt{k\_hat} et \texttt{lambda\_hat}.
\end{enumerate}
\end{exercicebox}

\begin{exercicebox}[Exercice 3 : Calcul de la densite (PDF)]
La fonction de densite de la loi Gamma est :
\[ f_Y(y) = \frac{\lambda^k y^{k-1} e^{-\lambda y}}{\Gamma(k)} \]

\textbf{Votre tâche :}
\begin{enumerate}
    \item Creer une fonction \texttt{gamma\_pdf(y, k, lambda)} qui calcule la densite en $y$.
    \item Utiliser cette fonction pour calculer $f_Y(20)$, $f_Y(30)$ et $f_Y(40)$ avec les parametres estimes precedemment.
    \item Arrondir les resultats a 6 decimales.
\end{enumerate}
\end{exercicebox}

\begin{exercicebox}[Exercice 4 : Fonction de repartition (CDF)]
La CDF d une loi Gamma ne s exprime pas simplement mais peut etre calculee numeriquement.

\textbf{Votre tâche :}
\begin{enumerate}
    \item Creer une fonction \texttt{gamma\_cdf(y, k, lambda)} qui calcule $P(Y \le y)$ en utilisant une integration numerique de la PDF.
    \item Calculer la probabilite qu une addition soit inferieure ou egale a 30 dollars : $F_Y(30) = P(Y \le 30)$.
    \item Calculer $P(Y > 50) = 1 - F_Y(50)$.
    \item Verifier que $F_Y(30) + P(Y > 30) \approx 1$.
\end{enumerate}
\end{exercicebox}

\begin{exercicebox}[Exercice 5 : Esperance theorique et empirique]
L esperance d une variable Gamma est $\mathbb{E}[Y] = \frac{k}{\lambda}$.

\textbf{Votre tâche :}
\begin{enumerate}
    \item Calculer l esperance theorique $\mathbb{E}[Y]$ avec les parametres $\hat{k}$ et $\hat{\lambda}$ estimes a l exercice 2.
    \item Comparer ce resultat avec la moyenne empirique \texttt{df['total\_bill'].mean()}.
    \item Calculer l erreur relative entre les deux valeurs.
\end{enumerate}
\end{exercicebox}

\begin{exercicebox}[Exercice 6 : Variance theorique et empirique]
La variance d une variable Gamma est $\text{Var}(Y) = \frac{k}{\lambda^2}$.

\textbf{Votre tâche :}
\begin{enumerate}
    \item Calculer la variance theorique $\text{Var}(Y)$ avec les parametres $\hat{k}$ et $\hat{\lambda}$.
    \item Comparer ce resultat avec la variance empirique \texttt{df['total\_bill'].var()}.
    \item Expliquer pourquoi les deux valeurs ne sont pas exactement identiques.
\end{enumerate}
\end{exercicebox}

\begin{exercicebox}[Exercice 7 : Probabilite d un intervalle]
Nous souhaitons calculer la probabilite qu une addition se situe dans un intervalle donne.

\textbf{Votre tâche :}
\begin{enumerate}
    \item Calculer $P(20 \le Y \le 40)$ en utilisant la CDF : $F_Y(40) - F_Y(20)$.
    \item Interpreter ce resultat en termes pourcentage.
    \item Calculer la probabilite qu une addition soit entre la moyenne empirique moins un ecart-type et la moyenne empirique plus un ecart-type.
\end{enumerate}
\end{exercicebox}

\begin{exercicebox}[Exercice 8 : Visualisation et ajustement]
La visualisation permet de valider graphiquement l adequation de la loi Gamma.

\textbf{Votre tâche :}
\begin{enumerate}
    \item Tracer l histogramme des \texttt{total\_bill} avec \texttt{density=True} et \texttt{bins=30}.
    \item Superposer la courbe de la PDF Gamma calculee sur un grille de valeurs entre 5 et 60.
    \item Ajouter un titre et des etiquettes d axes en francais.
    \item Commenter la qualite de l ajustement.
\end{enumerate}
\end{exercicebox}

\begin{exercicebox}[Exercice 9 : Somme de variables exponentielles]
Une interpretation de la loi Gamma est la somme de $k$ variables exponentielles de parametre $\lambda$.

\textbf{Votre tâche :}
\begin{enumerate}
    \item Simuler $k=5$ variables exponentielles independantes avec $\lambda = \hat{\lambda}$.
    \item Calculer leur somme $S = X_1 + X_2 + X_3 + X_4 + X_5$.
    \item Repeter cette simulation 10000 fois pour obtenir une distribution de $S$.
    \item Calculer la moyenne et la variance de ces sommes simulees et les comparer aux valeurs theoriques $\mathbb{E}[S] = k/\lambda$ et $\text{Var}(S) = k/\lambda^2$.
\end{enumerate}
\end{exercicebox}

\begin{exercicebox}[Exercice 10 : Cas particulier k=1 (loi exponentielle)]
Lorsque $k=1$, la loi Gamma se reducit a la loi exponentielle.

\textbf{Votre tâche :}
\begin{enumerate}
    \item Creer un sous-echantillon des petites additions (par exemple \texttt{total\_bill <= 15}) pour modeliser un scenario ou $k \approx 1$.
    \item Estimer $\lambda$ pour ce sous-echantillon comme $\hat{\lambda}_1 = 1/\bar{y}_\text{sub}$.
    \item Tracer l histogramme de ce sous-echantillon avec \texttt{density=True}.
    \item Superposer la densite exponentielle theorique $f(y) = \lambda e^{-\lambda y}$ et la densite Gamma(1, lambda).
    \item Confirmer que les deux courbes sont identiques.
\end{enumerate}
\end{exercicebox}