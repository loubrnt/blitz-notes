\subsection{Exercices Python}

Les exercices suivants appliquent les concepts de variables aléatoires discrètes, de leurs lois de probabilité (PMF, CDF) et de leurs caractéristiques (espérance, variance) au jeu de données "Taxis" de Seaborn.

\begin{codecell}
import pandas as pd
import seaborn as sns
import math

# Charger le dataset Taxis
df = sns.load_dataset("taxis")

# 'df' est maintenant notre Univers S.
# S_total = len(df)
\end{codecell}

\begin{exercicebox}[Exercice 1 : PMF (Fonction de Masse)]
Soit $X$ la variable aléatoire discrète représentant le nombre de passagers (\texttt{passengers}) dans un taxi.

\textbf{Votre tâche :}
\begin{enumerate}
    \item Calculer la Fonction de Masse (PMF) de $X$. Trouvez $P(X=1)$, $P(X=2)$, $P(X=3)$, etc., pour toutes les valeurs non nulles.
    \item Vérifier que $\sum_{k} P(X=k) = 1$.
\end{enumerate}
\end{exercicebox}

\begin{exercicebox}[Exercice 2 : CDF (Fonction de Répartition)]
En utilisant la variable $X$ (\texttt{passengers}) de l'exercice 1 :

\textbf{Votre tâche :}
\begin{enumerate}
    \item Calculer la valeur de la Fonction de Répartition (CDF) au point 2, c'est-à-dire $F_X(2) = P(X \le 2)$.
    \item Calculer $P(X > 3)$ en utilisant la CDF.
\end{enumerate}
\end{exercicebox}

\begin{exercicebox}[Exercice 3 : Espérance d'une VA Discrète]
En utilisant la variable $X$ (\texttt{passengers}) et sa PMF $P_X(k)$ calculée à l'exercice 1, calculez l'espérance de $X$.

L'espérance est $E[X] = \sum_{k} k \cdot P(X=k)$.

\textbf{Votre tâche :}
\begin{enumerate}
    \item Lister les valeurs $k$ possibles pour \texttt{passengers}.
    \item Pour chaque $k$, calculer le produit $k \times P(X=k)$.
    \item Sommer ces produits pour obtenir $E[X]$.
    \item (Vérification) Comparez votre résultat à la moyenne directe \texttt{df['passengers'].mean()}.
\end{enumerate}
\end{exercicebox}

\begin{exercicebox}[Exercice 4 : Variance d'une VA Discrète]
Calculez la variance de $X$ (\texttt{passengers}) en utilisant la formule $\text{Var}(X) = E[X^2] - (E[X])^2$.

\textbf{Votre tâche :}
\begin{enumerate}
    \item Utiliser l'espérance $E[X]$ calculée à l'exercice 3.
    \item Calculer l'espérance du carré, $E[X^2] = \sum_{k} k^2 \cdot P(X=k)$.
    \item Appliquer la formule de la variance.
    \item (Vérification) Comparez votre résultat à \texttt{df['passengers'].var()}.
\end{enumerate}
\end{exercicebox}

\begin{exercicebox}[Exercice 5 : Variable Aléatoire Indicatrice et Espérance]
Soit $I_A$ une variable aléatoire indicatrice pour l'événement $A$ = "le passager a donné un pourboire".

\textbf{Votre tâche :}
\begin{enumerate}
    \item Créer une nouvelle colonne \texttt{got\_tip} dans le DataFrame.
    \item Assigner $I_A = 1$ si \texttt{tip > 0}, et $I_A = 0$ sinon.
    \item Calculer l'espérance de cette variable, $E[I_A]$.
    \item Constatez que $E[I_A]$ est exactement la probabilité $P(A)$ que le passager donne un pourboire.
\end{enumerate}
\end{exercicebox}

\begin{exercicebox}[Exercice 6 : Loi de Bernoulli]
Soit $X$ une variable aléatoire de Bernoulli modélisant le type de paiement.
$X=1$ si le paiement est "credit\_card" (Succès) et $X=0$ si c'est "cash" (Échec).

\textbf{Votre tâche :}
\begin{enumerate}
    \item Calculer $p$, la probabilité de succès, $P(X=1)$.
    \item Calculer $1-p$, la probabilité d'échec, $P(X=0)$.
    \item Quelle est l'espérance $E[X]$ et la variance $\text{Var}(X)$ de cette variable ? (Utilisez les formules $p$ et $p(1-p)$).
\end{enumerate}
\end{exercicebox}

\begin{exercicebox}[Exercice 7 : Loi Binomiale (PMF et Espérance)]
Nous utilisons le paramètre $p$ (probabilité de payer par carte) de l'exercice 6.
On observe un échantillon de $n=10$ courses indépendantes. Soit $Y$ le nombre de courses payées par carte dans cet échantillon. $Y$ suit une loi binomiale $Y \sim \text{Bin}(n=10, p)$.

\textbf{Votre tâche :}
\begin{enumerate}
    \item En utilisant la PMF de la loi binomiale, calculer la probabilité d'avoir exactement $k=4$ paiements par carte, $P(Y=4)$.
    \item Calculer l'espérance $E[Y]$ et la variance $\text{Var}(Y)$ de cette variable binomiale.
\end{enumerate}
\end{exercicebox}

\begin{exercicebox}[Exercice 8 : Loi Géométrique]
Nous observons les trajets un par un (supposés indépendants) jusqu'à trouver notre premier "succès". 
Le "succès" est défini comme "trouver un trajet avec un pourboire (\texttt{tip}) de plus de 5 dollars".

Soit $X$ le nombre d'échecs (trajets avec $\texttt{tip} \le 5$) avant le premier succès. $X$ suit une loi géométrique $X \sim \text{Geom}(p)$.

\textbf{Votre tâche :}
\begin{enumerate}
    \item Calculer $p$, la probabilité de "succès" (trouver un $\texttt{tip} > 5$).
    \item En utilisant la PMF de la loi géométrique, calculer la probabilité d'avoir exactement $k=10$ échecs avant le premier succès.
\end{enumerate}
\end{exercicebox}

\begin{exercicebox}[Exercice 9 : Loi Hypergéométrique]
Considérons le tirage \textbf{sans remise}. Notre population est l'ensemble du \texttt{df}.
Soit $w$ le nombre total de paiements par "credit\_card" et $b$ le nombre total de paiements "cash".
On tire un échantillon de $m=20$ trajets. Soit $Z$ le nombre de paiements par carte dans cet échantillon. $Z$ suit une loi Hypergéométrique.

\textbf{Votre tâche :}
\begin{enumerate}
    \item Trouver $w$, $b$ et $m=20$.
    \item En utilisant la PMF de la loi hypergéométrique, calculer la probabilité d'avoir exactement $k=15$ paiements par carte dans l'échantillon, $P(Z=15)$.
\end{enumerate}
\end{exercicebox}

\begin{exercicebox}[Exercice 10 : Loi de Poisson]
La loi de Poisson modélise le nombre d'événements dans un intervalle de temps. Nous voulons modéliser le nombre de courses par heure dans un quartier.

\textbf{Votre tâche :}
\begin{enumerate}
    \item Filtrer le DataFrame pour ne garder que les courses dans le quartier "Manhattan" (\texttt{pickup\_borough == 'Manhattan'}).
    \item Convertir la colonne \texttt{pickup} en datetime.
    \item Agréger les données pour compter le nombre de courses par heure (vous pouvez "arrondir" l'heure de début de course).
    \item Calculer $\lambda$, le taux moyen de courses par heure à Manhattan (l'espérance de la loi de Poisson).
    \item En utilisant la PMF de la loi de Poisson avec ce $\lambda$, calculer la probabilité qu'il y ait exactement $k=50$ courses lors d'une heure donnée, $P(X=50)$.
\end{enumerate}
\end{exercicebox}