\subsection{Exercices Python}

Ces exercices appliquent les concepts de distributions multivariées (covariance, corrélation, variance d'une somme) à des données financières réelles. Nous allons analyser la relation entre les rendements boursiers de deux entreprises technologiques : Google (GOOG) et Microsoft (MSFT).

Nous travaillerons avec les \textbf{rendements journaliers} (variation en pourcentage), qui sont des variables aléatoires continues. L'espérance $E[X]$ sera estimée par la moyenne empirique (\texttt{.mean()}) et la variance $\text{Var}(X)$ par la variance empirique (\texttt{.var()}).

\begin{codecell}
!pip install yfinance
import yfinance as yf
import pandas as pd
import numpy as np

# Definir les tickers et la periode
tickers = ["GOOG", "MSFT"]
start_date = "2020-01-01"
end_date = "2024-12-31"

# Telecharger les prix de cloture ajustes
data = yf.download(tickers, start=start_date, end=end_date)["Adj Close"]

# Calculer les rendements journaliers en pourcentage
returns = data.pct_change().dropna()

# Renommer les colonnes pour plus de clarte
returns.columns = ["GOOG_Return", "MSFT_Return"]

# "returns" est notre DataFrame principal.
# X = returns["GOOG_Return"]
# Y = returns["MSFT_Return"]
\end{codecell}

\begin{exercicebox}[Exercice 1 : Espérances et Variances Marginales]
Soit $X$ la v.a. "Rendement journalier de GOOG" et $Y$ la v.a. "Rendement journalier de MSFT".

\textbf{Votre tâche :}
\begin{enumerate}
    \item Calculer l'espérance empirique $E[X]$ et $E[Y]$. (Que remarquez-vous sur leur ordre de grandeur ?)
    \item Calculer la variance empirique $\text{Var}(X)$ et $\text{Var}(Y)$.
    \item Calculer l'écart-type empirique $\sigma_X$ et $\sigma_Y$. Laquelle des deux actions est la plus "volatile" ?
\end{enumerate}
\end{exercicebox}

\begin{exercicebox}[Exercice 2 : Standardisation (Centrer-Réduire)]
Le concept de variable centrée réduite $Z = \frac{X - \mu_X}{\sigma_X}$ est très utilisé en finance (par ex: "Z-score").

\textbf{Votre tâche :}
\begin{enumerate}
    \item Récupérer $E[X]$ (la moyenne) et $\sigma_X$ (l'écart-type) des rendements de GOOG de l'exercice 1.
    \item Prendre le \textbf{dernier} rendement journalier de GOOG dans le jeu de données.
    \item Calculer le "Z-score" de ce dernier rendement.
    \item Interpréter ce score (par ex: "Le dernier jour, GOOG a performé à X écarts-types de sa moyenne...").
\end{enumerate}
\end{exercicebox}

\begin{exercicebox}[Exercice 3 : Covariance (Calcul via LOTUS)]
Calculez la covariance entre les rendements de GOOG ($X$) et de MSFT ($Y$) en utilisant la formule $\text{Cov}(X,Y) = E[XY] - E[X]E[Y]$.

\textbf{Votre tâche :}
\begin{enumerate}
    \item Récupérer $E[X]$ et $E[Y]$ de l'exercice 1.
    \item Calculer $E[XY]$ en utilisant le "Théorème de Transfert" (LOTUS) sur les données empiriques (Indice : calculez la moyenne de la série $X \times Y$).
    \item Appliquer la formule pour trouver $\text{Cov}(X,Y)$.
    \item Le signe est-il positif ou négatif ? Qu'est-ce que cela implique intuitivement ?
\end{enumerate}
\end{exercicebox}

\begin{exercicebox}[Exercice 4 : Corrélation (Calcul)]
La covariance de l'exercice 3 dépend des unités (rendement au carré). Nous allons la normaliser pour obtenir la corrélation $r \in [-1, 1]$.

\textbf{Votre tâche :}
\begin{enumerate}
    \item Récupérer $\text{Cov}(X,Y)$ (Exercice 3) et $\sigma_X, \sigma_Y$ (Exercice 1).
    \item Appliquer la formule : $\text{Corr}(X,Y) = \frac{\text{Cov}(X,Y)}{\sigma_X \sigma_Y}$.
    \item Interpréter ce coefficient. La relation linéaire entre GOOG et MSFT est-elle forte ou faible ?
\end{enumerate}
\end{exercicebox}

\begin{exercicebox}[Exercice 5 : Linéarité de l'Espérance (Portefeuille)]
Soit un portefeuille $P$ composé à 60\% de GOOG ($X$) et 40\% de MSFT ($Y$).
Le rendement du portefeuille est $P = 0.6X + 0.4Y$.
La théorie dit : $E[P] = E[0.6X + 0.4Y] = 0.6E[X] + 0.4E[Y]$.

\textbf{Votre tâche :}
\begin{enumerate}
    \item En utilisant $E[X]$ et $E[Y]$ de l'exercice 1, calculer l'espérance \textbf{théorique} $E[P]$.
    \item \textbf{Vérification empirique} : 
        \begin{itemize}
            \item Créer la série de données $P_{series} = 0.6 \times X + 0.4 \times Y$.
            \item Calculer l'espérance empirique de $P_{series}$ (\texttt{.mean()}).
        \end{itemize}
    \item Comparer votre résultat théorique (1) et empirique (2).
\end{enumerate}
\end{exercicebox}

\begin{exercicebox}[Exercice 6 : Variance d'un Portefeuille (Variance d'une Somme)]
Continuons avec le portefeuille $P = 0.6X + 0.4Y$.
La variance \textbf{théorique} est : $\text{Var}(P) = a^2\text{Var}(X) + b^2\text{Var}(Y) + 2ab\text{Cov}(X,Y)$.

\textbf{Votre tâche :}
\begin{enumerate}
    \item En utilisant $\text{Var}(X)$, $\text{Var}(Y)$ (Ex 1) et $\text{Cov}(X,Y)$ (Ex 3), calculer $\text{Var}(P)$ en appliquant la formule ci-dessus.
    \item \textbf{Vérification empirique} : 
        \begin{itemize}
            \item Utiliser la série $P_{series}$ de l'exercice 5.
            \item Calculer la variance empirique de $P_{series}$ (\texttt{.var()}).
        \end{itemize}
    \item Comparer votre résultat théorique (1) et empirique (2).
\end{enumerate}
\end{exercicebox}

\begin{exercicebox}[Exercice 7 : Le Bénéfice de la Diversification]
La diversification (le fait que $\text{Corr}(X,Y) \ne 1$) réduit le risque. Nous allons le prouver.
Le risque (l'écart-type) d'un portefeuille \textit{n'est pas} la moyenne pondérée des risques.

\textbf{Votre tâche :}
\begin{enumerate}
    \item Calculer le risque du portefeuille $\sigma_P$ (l'écart-type) en prenant la racine carrée de $\text{Var}(P)$ (calculée à l'exercice 6).
    \item Calculer la "moyenne pondérée des risques" : $\sigma_{moy} = 0.6 \sigma_X + 0.4 \sigma_Y$ (en utilisant $\sigma_X, \sigma_Y$ de l'Ex 1).
    \item Comparer $\sigma_P$ et $\sigma_{moy}$. Lequel est le plus petit ?
    \item (Conclusion) Pourquoi $\sigma_P < \sigma_{moy}$ ? (Indice : $\text{Corr}(X,Y)$).
\end{enumerate}
\end{exercicebox}

\begin{exercicebox}[Exercice 8 : Vérification des Bornes de Corrélation]
Le théorème stipule que si $Y = aX + b$, alors $\text{Corr}(X,Y) = \pm 1$. Vérifions cela.

\textbf{Votre tâche :}
\begin{enumerate}
    \item Soit $X$ la série des rendements de GOOG.
    \item Créer une nouvelle variable $Z = -3X + 0.005$ (une relation linéaire négative parfaite).
    \item Calculer la corrélation empirique entre $X$ et $Z$. (Vous pouvez utiliser \texttt{X.corr(Z)}).
    \item Le résultat est-il conforme au théorème ?
\end{enumerate}
\end{exercicebox}

\begin{exercicebox}[Exercice 9 : Loi Jointe (Discrétisation)]
Transformons nos variables continues $X$ et $Y$ en variables de Bernoulli discrètes.
Soit $X_{bern} = 1$ si le rendement de GOOG est positif ($> 0$), et $0$ sinon.
Soit $Y_{bern} = 1$ si le rendement de MSFT est positif ($> 0$), et $0$ sinon.
Nous voulons trouver la PMF jointe $P(X_{bern}=x, Y_{bern}=y)$.

\textbf{Votre tâche :}
\begin{enumerate}
    \item Créer les deux séries discrètes $X_{bern}$ et $Y_{bern}$.
    \item Utiliser \texttt{pandas.crosstab} pour créer un tableau de contingence (les effectifs).
    \item Normaliser ce tableau par l'effectif total pour obtenir la \textbf{loi jointe} (PMF jointe).
    \item Quelle est la probabilité que les deux actions aient un rendement positif le même jour, $P(X_{bern}=1, Y_{bern}=1)$ ?
\end{enumerate}
\end{exercicebox}

\begin{exercicebox}[Exercice 10 : Lois Marginales et Indépendance (Discret)]
En utilisant la loi jointe $P(X_{bern}, Y_{bern})$ de l'exercice 9.

\textbf{Votre tâche :}
\begin{enumerate}
    \item Calculer la loi marginale $P(X_{bern}=x)$ (somme des lignes).
    \item Calculer la loi marginale $P(Y_{bern}=y)$ (somme des colonnes).
    \item Les variables $X_{bern}$ et $Y_{bern}$ sont-elles indépendantes ? 
    \item Justifiez en comparant $P(X_{bern}=1, Y_{bern}=1)$ au produit $P(X_{bern}=1) \times P(Y_{bern}=1)$.
\end{enumerate}
\end{exercicebox}