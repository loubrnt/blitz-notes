\subsection{Exercices Python}

Les exercices suivants appliquent les concepts de variables aléatoires continues (PDF, CDF, espérance, variance) en utilisant la bibliothèque \texttt{NumPy} pour la simulation numérique afin de vérifier les résultats théoriques.

\begin{codecell}
import numpy as np
import math
\end{codecell}

\begin{exercicebox}[Exercice 1 : PDF CDF et Espérance (Simulation)]
Soit $X$ une v.a. continue avec la PDF $f(x) = 2x$ pour $x \in [0, 1]$, et $f(x)=0$ sinon.
Par calcul (que vous pouvez faire à la main), on trouve :
\begin{itemize}
    \item CDF : $F(x) = x^2$ (pour $x \in [0, 1]$)
    \item Espérance : $E[X] = 2/3$
\end{itemize}
Nous pouvons simuler cette variable en utilisant la méthode de la transformée inverse : si $U \sim \text{Unif}(0, 1)$, alors $X = F^{-1}(U) = \sqrt{U}$ suit la loi de $X$.

\textbf{Votre tâche (avec NumPy) :}
\begin{enumerate}
    \item Générer $N=100000$ échantillons $U$ d'une loi Uniforme(0, 1) avec \texttt{np.random.rand}.
    \item Transformer ces échantillons pour obtenir $N$ échantillons de $X$ (en prenant la racine carrée).
    \item Calculer l'espérance empirique $E[X]$ (la moyenne de vos échantillons $X$) et la comparer à la valeur théorique $2/3$.
\end{enumerate}
\end{exercicebox}

\begin{exercicebox}[Exercice 2 : Variance (Simulation)]
En utilisant les échantillons $X$ de l'exercice 1.
La valeur théorique (calculée à la main) de la variance est $\text{Var}(X) = 1/18$.

\textbf{Votre tâche (avec NumPy) :}
\begin{enumerate}
    \item Calculer la variance empirique $\text{Var}(X)$ de vos échantillons $X$ avec \texttt{np.var}.
    \item Comparer le résultat empirique à la valeur théorique $1/18$.
\end{enumerate}
\end{exercicebox}

\begin{exercicebox}[Exercice 3 : Loi Uniforme (Simulation vs Théorie)]
Soit $X \sim \text{Unif}(a=5, b=15)$. Les valeurs théoriques sont $E[X] = \frac{a+b}{2}$ et $\text{Var}(X) = \frac{(b-a)^2}{12}$.

\textbf{Votre tâche (avec NumPy) :}
\begin{enumerate}
    \item Calculer l'espérance et la variance théoriques.
    \item Générer $N=100000$ échantillons aléatoires de $X$ avec \texttt{np.random.uniform}.
    \item Calculer l'espérance empirique (\texttt{np.mean}) et la variance empirique (\texttt{np.var}) des échantillons.
    \item Comparer les résultats empiriques aux résultats théoriques.
\end{enumerate}
\end{exercicebox}

\begin{exercicebox}[Exercice 4 : Loi Uniforme (Vérification de la PDF)]
Pour $X \sim \text{Unif}(5, 15)$, la PDF est $f(x) = \frac{1}{10}$ sur $[5, 15]$.
La probabilité $P(7 \le X \le 10)$ est $\int_7^{10} \frac{1}{10} dx = \frac{10-7}{10} = 0.3$.

\textbf{Votre tâche (avec NumPy) :}
\begin{enumerate}
    \item Utiliser les échantillons de $X$ de l'exercice 3.
    \item Calculer la probabilité empirique $P(7 \le X \le 10)$ en comptant la proportion d'échantillons qui tombent dans cet intervalle.
    \item Comparer le résultat empirique à la valeur théorique $0.3$.
\end{enumerate}
\end{exercicebox}

\begin{exercicebox}[Exercice 5 : Loi Exponentielle (Simulation vs Théorie)]
Soit $X \sim \text{Exp}(\lambda=0.5)$. Les valeurs théoriques sont $E[X] = \frac{1}{\lambda}$ et $\text{Var}(X) = \frac{1}{\lambda^2}$.

Note : \texttt{np.random.exponential} prend un paramètre "scale" $\beta = 1/\lambda$.

\textbf{Votre tâche (avec NumPy) :}
\begin{enumerate}
    \item Définir $\lambda$ et calculer $E[X]$ et $\text{Var}(X)$ théoriques.
    \item Calculer le paramètre $\beta$ (scale) pour NumPy.
    \item Générer $N=100000$ échantillons aléatoires de $X$.
    \item Calculer et comparer les espérances et variances empiriques et théoriques.
\end{enumerate}
\end{exercicebox}

\begin{exercicebox}[Exercice 6 : Loi Exponentielle (Vérification de la CDF)]
Pour $X \sim \text{Exp}(\lambda=0.5)$, la CDF est $F(x) = 1 - e^{-\lambda x}$.
Calculons $P(X \le 3) = F(3) = 1 - e^{-0.5 \times 3}$.

\textbf{Votre tâche (avec NumPy) :}
\begin{enumerate}
    \item Calculer la valeur théorique $F(3)$.
    \item Utiliser les échantillons de $X$ de l'exercice 5.
    \item Calculer la probabilité empirique $P(X \le 3)$ en comptant la proportion d'échantillons $\le 3$.
    \item Comparer les deux valeurs.
\end{enumerate}
\end{exercicebox}

\begin{exercicebox}[Exercice 7 : Propriété de Non-Mémoire (Exponentielle)]
Nous allons vérifier numériquement la propriété de non-mémoire $P(X > s+t \mid X > s) = P(X > t)$ en utilisant les échantillons de $X$ de l'exercice 5 ($\lambda=0.5$).

\textbf{Votre tâche (avec NumPy) :}
\begin{enumerate}
    \item Choisir $s=1$ et $t=2$.
    \item Calculer $P(X > t)$ (théoriquement $e^{-\lambda t}$). Calculer la probabilité empirique (proportion d'échantillons $> t$).
    \item Calculer $P(X > s+t \mid X > s)$ empiriquement :
        \begin{itemize}
            \item Filtrer les échantillons pour ne garder que ceux où $X > s$.
            \item Parmi ce sous-ensemble, calculer la proportion de ceux où $X > s+t$.
        \end{itemize}
    \item Comparer les deux probabilités empiriques.
\end{enumerate}
\end{exercicebox}

\begin{exercicebox}[Exercice 8 : Théorème de Transfert (LOTUS)]
Soit $X \sim \text{Unif}(0, 2)$. La PDF est $f(x)=1/2$.
Soit $g(X) = X^2$. Nous voulons $E[g(X)] = E[X^2]$.
Théoriquement : $E[X^2] = \int_0^2 x^2 f(x) \, dx = \int_0^2 x^2 (1/2) \, dx = \frac{1}{2} [\frac{x^3}{3}]_0^2 = \frac{1}{2} (\frac{8}{3}) = 4/3$.

\textbf{Votre tâche (avec NumPy) :}
\begin{enumerate}
    \item Générer $N=100000$ échantillons $X \sim \text{Unif}(0, 2)$.
    \item Créer les échantillons $Y = g(X) = X^2$.
    \item Calculer l'espérance empirique $E[Y]$ (la moyenne de $Y$).
    \item Comparer le résultat empirique à la valeur théorique $4/3$.
\end{enumerate}
\end{exercicebox}

\begin{exercicebox}[Exercice 9 : Linéarité de l'Espérance (E[aX+b])]
Soit $X \sim \text{Unif}(5, 15)$ (de l'exercice 3). Nous savons que $E[X] = 10$.
Soit $Y = 5X - 3$.
Par linéarité, l'espérance théorique est $E[Y] = E[5X - 3] = 5E[X] - 3$.

\textbf{Votre tâche (avec NumPy) :}
\begin{enumerate}
    \item Calculer $E[Y]$ théoriquement en utilisant $E[X] = 10$.
    \item Utiliser les échantillons $X$ de l'exercice 3.
    \item Créer les échantillons $Y = 5 \times X - 3$.
    \item Calculer l'espérance empirique $E[Y]$ (la moyenne de $Y$).
    \item Comparer les deux résultats.
\end{enumerate}
\end{exercicebox}

\begin{exercicebox}[Exercice 10 : Propriétés de la Variance (Var(aX+b))]
Soit $X \sim \text{Unif}(5, 15)$ (de l'exercice 3). $\text{Var}(X) = \frac{(15-5)^2}{12} = 100/12 \approx 8.333$.
Soit $Y = 5X - 3$.
Théoriquement : $\text{Var}(Y) = \text{Var}(5X - 3) = \text{Var}(5X) = 5^2 \text{Var}(X) = 25 \times \text{Var}(X)$.

\textbf{Votre tâche (avec NumPy) :}
\begin{enumerate}
    \item Calculer $\text{Var}(Y)$ théoriquement en utilisant $\text{Var}(X) = 100/12$.
    \item Utiliser les échantillons $Y$ de l'exercice 9.
    \item Calculer la variance empirique $\text{Var}(Y)$ (avec \texttt{np.var}).
    \item Comparer les deux résultats.
\end{enumerate}
\end{exercicebox}