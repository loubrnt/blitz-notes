\subsection{Exercices}

\begin{exercicebox}[Exercice 1 :]
Soit la fonction $f(x) = cx^2$ pour $x \in [0, 2]$, et $f(x)=0$ sinon.
\begin{enumerate}
    \item Calculez la valeur de $c$ pour que $f(x)$ soit une PDF valide.
    \item En utilisant la valeur de $c$ trouvée, calculez $P(X > 1)$.
\end{enumerate}
\end{exercicebox}

\begin{exercicebox}[Exercice 2 :]
Soit une v.a. $X$ de PDF $f(x) = \frac{1}{2}x$ sur $[0, 2]$, et $f(x)=0$ sinon.
\begin{enumerate}
    \item Déterminez la fonction de répartition (CDF) $F(x)$.
    \item En utilisant la CDF, calculez $P(1 \le X \le 1.5)$.
\end{enumerate}
\end{exercicebox}

\begin{exercicebox}[Exercice 3 :]
La CDF d'une v.a. $X$ est donnée par :
$$F(x) = \begin{cases} 0 & \text{si } x < 0 \\ \sin(x) & \text{si } 0 \le x \le \pi/2 \\ 1 & \text{si } x > \pi/2 \end{cases}$$
Calculez la PDF $f(x)$ de $X$.
\end{exercicebox}

\begin{exercicebox}[Exercice 4 :]
Soit $X$ une variable aléatoire continue avec une PDF $f(x)$. Démontrez que pour tout $a \in \mathbb{R}$, $P(X = a) = 0$.
\end{exercicebox}

\begin{exercicebox}[Exercice 5 :]
Soit $X$ une v.a. de PDF $f(x) = 3x^2$ sur $[0, 1]$.
\begin{enumerate}
    \item Calculez l'espérance $E[X]$.
    \item Calculez $E[X^2]$.
    \item Déduisez-en la variance $\text{Var}(X)$.
\end{enumerate}
\end{exercicebox}

\begin{exercicebox}[Exercice 6 :]
Soit $X$ une v.a. de PDF $f(x) = 2x$ sur $[0, 1]$. En utilisant le théorème de LOTUS, calculez $E[1/X]$.
\end{exercicebox}

\begin{exercicebox}[Exercice 7 : Variance d'une transformation affine]
Soit $X$ une v.a. avec une variance $\text{Var}(X)$ finie. En utilisant les propriétés de l'espérance, démontrez que pour $a, b \in \mathbb{R}$ :
$$ \text{Var}(aX + b) = a^2 \text{Var}(X) $$
\end{exercicebox}

\begin{exercicebox}[Exercice 8 :]
Le temps d'attente $T$ (en minutes) d'un bus suit une loi $T \sim \text{Unif}(0, 12)$.
\begin{enumerate}
    \item Quelle est la probabilité d'attendre moins de 4 minutes ?
    \item Quelle est la probabilité d'attendre entre 5 et 10 minutes ?
\end{enumerate}
\end{exercicebox}

\begin{exercicebox}[Exercice 9 :]
Pour le temps d'attente $T \sim \text{Unif}(0, 12)$ de l'exercice précédent :
\begin{enumerate}
    \item Calculez le temps d'attente moyen $E[T]$.
    \item Calculez la variance $\text{Var}(T)$.
\end{enumerate}
\end{exercicebox}

\begin{exercicebox}[Exercice 10 :]
Soit $X \sim \text{Unif}(a, b)$. La PDF est $f(x) = \frac{1}{b-a}$ pour $x \in [a, b]$.
Démontrez par un calcul intégral direct que $E[X] = \frac{a+b}{2}$.
\end{exercicebox}

\begin{exercicebox}[Exercice 11 :]
Un générateur produit un nombre $X \sim \text{Unif}(a, b)$. On sait que $E[X] = 10$ et $\text{Var}(X) = 12$. Trouvez $a$ et $b$.
\end{exercicebox}

\begin{exercicebox}[Exercice 12 :]
La durée de vie $T$ (en années) d'un composant suit $T \sim \text{Exp}(\lambda)$. Sa durée de vie moyenne $E[T]$ est de 5 ans.
\begin{enumerate}
    \item Déterminez $\lambda$.
    \item Calculez la probabilité que le composant tombe en panne avant 2 ans, $P(T \le 2)$.
\end{enumerate}
\end{exercicebox}

\begin{exercicebox}[Exercice 13 :]
En utilisant le scénario de l'exercice 12 ($E[T]=5$ ans), calculez la probabilité que le composant dure 10 ans de plus, sachant qu'il a déjà duré 3 ans : $P(T > 13 \mid T > 3)$. Comparez ce résultat à $P(T > 10)$.
\end{exercicebox}

\begin{exercicebox}[Exercice 14 :]
Soit $X \sim \text{Exp}(\lambda)$. Sa PDF est $f(x) = \lambda e^{-\lambda x}$ pour $x \ge 0$.
En utilisant une intégration par parties, prouvez que $E[X] = 1/\lambda$.
\end{exercicebox}

\begin{exercicebox}[Exercice 15 :]
Soit $X \sim \text{Exp}(\lambda)$. Démontrez formellement la propriété d'absence de mémoire :
$$ P(X > s+t \mid X > s) = P(X > t) \quad \text{pour tous } s, t \ge 0 $$
\end{exercicebox}

\begin{exercicebox}[Exercice 16 :]
Soit $F(x)$ la CDF d'une v.a. continue $X$. Démontrez que pour $a < b$, on a :
$$ P(a < X \le b) = F(b) - F(a) $$
(Indice : Utilisez $P(A) = P(B) - P(B \setminus A)$ avec $A = \{X \le a\}$ et $B = \{X \le b\}$).
\end{exercicebox}

\begin{exercicebox}[Exercice 17 :]
Soit $X$ une v.a. continue de PDF $f(x)$. Démontrez que $E[aX + b] = aE[X] + b$ pour $a, b \in \mathbb{R}$, en utilisant la définition de l'espérance (LOTUS).
\end{exercicebox}

\begin{exercicebox}[Exercice 18 :]
Soit $X \sim \text{Unif}(0, 1)$. Soit $Y = 5X - 2$.
Trouvez la PDF $f_Y(y)$ de $Y$ en utilisant la méthode de la CDF ou la formule de changement de variable. Sur quel intervalle $Y$ est-elle définie ?
\end{exercicebox}

\begin{exercicebox}[Exercice 19 :]
Soit $X \sim \text{Unif}(-1, 1)$. Soit $Y = X^2$.
Trouvez la PDF $f_Y(y)$ de $Y$. (Attention : la fonction $g(x)=x^2$ n'est pas monotone sur $[-1, 1]$).
\end{exercicebox}

\begin{exercicebox}[Exercice 20 :]
Soit $X \sim \text{Exp}(1)$ (donc $f_X(x) = e^{-x}$ pour $x \ge 0$). Soit $Y = \sqrt{X}$.
Trouvez la PDF $f_Y(y)$ de $Y$.
\end{exercicebox}

\subsection{Correction}

\begin{correctionbox}[Exercice 1 :]
\begin{enumerate}
    \item Pour que $f(x)$ soit une PDF valide, on doit avoir $\int_{-\infty}^{\infty} f(x)\,dx = 1$. Alors :
    \[
    \int_0^2 c x^2 \, dx = c \left[ \frac{x^3}{3} \right]_0^2 = c \cdot \frac{8}{3} = 1 \implies c = \frac{3}{8}.
    \]
    
    \item Avec $c = \frac{3}{8}$, on calcule :
    \[
    P(X > 1) = \int_1^2 \frac{3}{8} x^2 \, dx = \frac{3}{8} \left[ \frac{x^3}{3} \right]_1^2 = \frac{3}{8} \left( \frac{8}{3} - \frac{1}{3} \right) = \frac{3}{8} \cdot \frac{7}{3} = \frac{7}{8}.
    \]
\end{enumerate}
\end{correctionbox}

\begin{correctionbox}[Exercice 2 :]
\begin{enumerate}
    \item La CDF est définie par $F(x) = \int_{-\infty}^x f(t)\,dt$.
    \item Si $x < 0$, $F(x) = 0$.
        \item Si $x \in [0,2]$, $F(x) = \int_0^x \frac{1}{2} t \, dt = \frac{1}{2} \cdot \frac{t^2}{2} \Big|_0^x = \frac{x^2}{4}$.
        \item Si $x > 2$, $F(x) = 1$.
    Donc :
    \[
    F(x) = 
    \begin{cases}
    0 & x < 0 \\
    \frac{x^2}{4} & 0 \le x \le 2 \\
    1 & x > 2
    \end{cases}
    \]

    \item $P(1 \le X \le 1.5) = F(1.5) - F(1) = \frac{(1.5)^2}{4} - \frac{1^2}{4} = \frac{2.25 - 1}{4} = \frac{1.25}{4} = 0.3125$.
\end{enumerate}
\end{correctionbox}

\begin{correctionbox}[Exercice 3 :]
La PDF est la dérivée de la CDF :
\[
f(x) = F'(x) = 
\begin{cases}
0 & x < 0 \\
\cos(x) & 0 \le x \le \pi/2 \\
0 & x > \pi/2
\end{cases}
\]
Donc $f(x) = \cos(x)$ pour $x \in [0, \pi/2]$, et $0$ sinon.

\end{correctionbox}

\begin{correctionbox}[Exercice 4 :]
Soit $X$ continue avec PDF $f(x)$. Alors :
\[
P(X = a) = \int_a^a f(x)\,dx = 0,
\]
car l'intégrale sur un intervalle de longueur nulle est nulle.

\end{correctionbox}

\begin{correctionbox}[Exercice 5 :]
\begin{enumerate}
    \item $E[X] = \int_0^1 x \cdot 3x^2 \, dx = 3 \int_0^1 x^3 \, dx = 3 \cdot \frac{1}{4} = \frac{3}{4}$.
    
    \item $E[X^2] = \int_0^1 x^2 \cdot 3x^2 \, dx = 3 \int_0^1 x^4 \, dx = 3 \cdot \frac{1}{5} = \frac{3}{5}$.
    
    \item $\text{Var}(X) = E[X^2] - (E[X])^2 = \frac{3}{5} - \left(\frac{3}{4}\right)^2 = \frac{3}{5} - \frac{9}{16} = \frac{48 - 45}{80} = \frac{3}{80}$.
\end{enumerate}
\end{correctionbox}

\begin{correctionbox}[Exercice 6 :]
Par LOTUS : $E[1/X] = \int_0^1 \frac{1}{x} \cdot 2x \, dx = \int_0^1 2 \, dx = 2$.
Mais attention : en $x=0$, $1/x$ n'est pas défini. Toutefois, comme il s'agit d'un point de mesure nulle, l'intégrale converge et vaut 2.

\end{correctionbox}

\begin{correctionbox}[Exercice 7 :]
On utilise $\text{Var}(Y) = E[Y^2] - (E[Y])^2$ avec $Y = aX + b$ :
\[
E[aX + b] = aE[X] + b, \quad E[(aX + b)^2] = E[a^2X^2 + 2abX + b^2] = a^2E[X^2] + 2abE[X] + b^2.
\]
Alors :
\begin{align*}
\text{Var}(aX + b) &= E[(aX + b)^2] - (E[aX + b])^2 \\
&= a^2E[X^2] + 2abE[X] + b^2 - (aE[X] + b)^2 \\
&= a^2E[X^2] + 2abE[X] + b^2 - \left(a^2(E[X])^2 + 2abE[X] + b^2\right) \\
&= a^2E[X^2] - a^2(E[X])^2 = a^2 \text{Var}(X).
\end{align*}
\end{correctionbox}

\begin{correctionbox}[Exercice 8 :]
$T \sim \text{Unif}(0, 12)$, donc $f_T(t) = \frac{1}{12}$ pour $t \in [0,12]$.
\begin{enumerate}
    \item $P(T < 4) = \int_0^4 \frac{1}{12} dt = \frac{4}{12} = \frac{1}{3}$.
    \item $P(5 \le T \le 10) = \int_5^{10} \frac{1}{12} dt = \frac{5}{12}$.
\end{enumerate}
\end{correctionbox}

\begin{correctionbox}[Exercice 9 :]
\begin{enumerate}
    \item $E[T] = \frac{0 + 12}{2} = 6$ minutes.
    \item $\text{Var}(T) = \frac{(12 - 0)^2}{12} = \frac{144}{12} = 12$.
\end{enumerate}
\end{correctionbox}

\begin{correctionbox}[Exercice 10 :]
\[
E[X] = \int_a^b x \cdot \frac{1}{b-a} dx = \frac{1}{b-a} \int_a^b x \, dx = \frac{1}{b-a} \cdot \frac{b^2 - a^2}{2} = \frac{b^2 - a^2}{2(b-a)} = \frac{(b-a)(b+a)}{2(b-a)} = \frac{a+b}{2}.
\]
\end{correctionbox}

\begin{correctionbox}[Exercice 11 :]
On sait : $E[X] = \frac{a+b}{2} = 10$, $\text{Var}(X) = \frac{(b-a)^2}{12} = 12$.
De la première équation : $a + b = 20$.
De la seconde : $(b - a)^2 = 144 \implies b - a = 12$ (puisque $b > a$).
En résolvant le système :
\[
a + b = 20, \quad b - a = 12 \implies 2b = 32 \implies b = 16, \quad a = 4.
\]
Donc $a = 4$, $b = 16$.

\end{correctionbox}

\begin{correctionbox}[Exercice 12 :]
\begin{enumerate}
    \item $E[T] = \frac{1}{\lambda} = 5 \implies \lambda = \frac{1}{5} = 0.2$.
    \item $P(T \le 2) = 1 - e^{-\lambda \cdot 2} = 1 - e^{-0.4} \approx 1 - 0.6703 = 0.3297$.
\end{enumerate}
\end{correctionbox}

\begin{correctionbox}[Exercice 13 :]
Par la propriété d'absence de mémoire :
\[
P(T > 13 \mid T > 3) = P(T > 10) = e^{-\lambda \cdot 10} = e^{-2} \approx 0.1353.
\]
Et $P(T > 10) = e^{-2} \approx 0.1353$. Les deux probabilités sont égales.

\end{correctionbox}

\begin{correctionbox}[Exercice 14 :]
\[
E[X] = \int_0^\infty x \lambda e^{-\lambda x} dx.
\]
Intégration par parties : $u = x$, $dv = \lambda e^{-\lambda x} dx$, donc $du = dx$, $v = -e^{-\lambda x}$.
\[
E[X] = \left[ -x e^{-\lambda x} \right]_0^\infty + \int_0^\infty e^{-\lambda x} dx = 0 + \left[ -\frac{1}{\lambda} e^{-\lambda x} \right]_0^\infty = \frac{1}{\lambda}.
\]
\end{correctionbox}

\begin{correctionbox}[Exercice 15 :]
\[
P(X > s+t \mid X > s) = \frac{P(X > s+t)}{P(X > s)} = \frac{e^{-\lambda(s+t)}}{e^{-\lambda s}} = e^{-\lambda t} = P(X > t).
\]
\end{correctionbox}

\begin{correctionbox}[Exercice 16 :]
Soit $A = \{X \le a\}$, $B = \{X \le b\}$ avec $a < b$. Alors $A \subset B$, et :
\[
P(a < X \le b) = P(B \setminus A) = P(B) - P(A) = F(b) - F(a).
\]
\end{correctionbox}

\begin{correctionbox}[Exercice 17 :]
\[
E[aX + b] = \int_{-\infty}^\infty (ax + b) f_X(x) dx = a \int_{-\infty}^\infty x f_X(x) dx + b \int_{-\infty}^\infty f_X(x) dx = aE[X] + b.
\]
\end{correctionbox}

\begin{correctionbox}[Exercice 18 :]
$X \sim \text{Unif}(0,1)$, $Y = 5X - 2$. Transformation linéaire croissante.
CDF : $F_Y(y) = P(Y \le y) = P(5X - 2 \le y) = P\left(X \le \frac{y+2}{5}\right) = F_X\left(\frac{y+2}{5}\right)$.
Comme $X \in [0,1]$, alors $Y \in [-2, 3]$.
PDF : $f_Y(y) = f_X\left(\frac{y+2}{5}\right) \cdot \left|\frac{d}{dy} \left(\frac{y+2}{5}\right)\right| = 1 \cdot \frac{1}{5} = \frac{1}{5}$ pour $y \in [-2, 3]$.
Donc $Y \sim \text{Unif}(-2, 3)$.

\end{correctionbox}

\begin{correctionbox}[Exercice 19 :]
$X \sim \text{Unif}(-1,1)$, $f_X(x) = \frac{1}{2}$ sur $[-1,1]$, $Y = X^2$.
Pour $y \in [0,1]$, on a deux branches : $x = \sqrt{y}$ et $x = -\sqrt{y}$.
Formule générale pour transformation non monotone :
\[
f_Y(y) = \sum_{i} f_X(x_i) \left| \frac{dx_i}{dy} \right|, \quad \frac{dx}{dy} = \frac{d}{dy} (\pm \sqrt{y}) = \pm \frac{1}{2\sqrt{y}}.
\]
Donc :
\[
f_Y(y) = f_X(\sqrt{y}) \cdot \frac{1}{2\sqrt{y}} + f_X(-\sqrt{y}) \cdot \frac{1}{2\sqrt{y}} = \frac{1}{2} \cdot \frac{1}{2\sqrt{y}} + \frac{1}{2} \cdot \frac{1}{2\sqrt{y}} = \frac{1}{4\sqrt{y}} + \frac{1}{4\sqrt{y}} = \frac{1}{2\sqrt{y}}.
\]
Ainsi, $f_Y(y) = \frac{1}{2\sqrt{y}}$ pour $y \in (0,1]$, et $0$ sinon.

\end{correctionbox}

\begin{correctionbox}[Exercice 20 :]
$X \sim \text{Exp}(1)$, $f_X(x) = e^{-x}$ pour $x \ge 0$, $Y = \sqrt{X} \Rightarrow X = Y^2$.
Transformation strictement croissante pour $y \ge 0$.
\[
f_Y(y) = f_X(y^2) \cdot \left| \frac{d}{dy}(y^2) \right| = e^{-y^2} \cdot 2y, \quad y \ge 0.
\]
Donc $f_Y(y) = 2y e^{-y^2}$ pour $y \ge 0$.

\end{correctionbox}

\subsection{Exercices Python}

Les exercices suivants appliquent les concepts de variables aléatoires continues (PDF, CDF, espérance, variance) en utilisant la bibliothèque \texttt{NumPy} pour la simulation numérique afin de vérifier les résultats théoriques.

\begin{codecell}
import numpy as np
import math
\end{codecell}

\begin{exercicebox}[Exercice 1 : PDF CDF et Espérance (Simulation)]
Soit $X$ une v.a. continue avec la PDF $f(x) = 2x$ pour $x \in [0, 1]$, et $f(x)=0$ sinon.
Par calcul (que vous pouvez faire à la main), on trouve :
\begin{itemize}
    \item CDF : $F(x) = x^2$ (pour $x \in [0, 1]$)
    \item Espérance : $E[X] = 2/3$
\end{itemize}
Nous pouvons simuler cette variable en utilisant la méthode de la transformée inverse : si $U \sim \text{Unif}(0, 1)$, alors $X = F^{-1}(U) = \sqrt{U}$ suit la loi de $X$.

\textbf{Votre tâche (avec NumPy) :}
\begin{enumerate}
    \item Générer $N=100000$ échantillons $U$ d'une loi Uniforme(0, 1) avec \texttt{np.random.rand}.
    \item Transformer ces échantillons pour obtenir $N$ échantillons de $X$ (en prenant la racine carrée).
    \item Calculer l'espérance empirique $E[X]$ (la moyenne de vos échantillons $X$) et la comparer à la valeur théorique $2/3$.
\end{enumerate}
\end{exercicebox}

\begin{exercicebox}[Exercice 2 : Variance (Simulation)]
En utilisant les échantillons $X$ de l'exercice 1.
La valeur théorique (calculée à la main) de la variance est $\text{Var}(X) = 1/18$.

\textbf{Votre tâche (avec NumPy) :}
\begin{enumerate}
    \item Calculer la variance empirique $\text{Var}(X)$ de vos échantillons $X$ avec \texttt{np.var}.
    \item Comparer le résultat empirique à la valeur théorique $1/18$.
\end{enumerate}
\end{exercicebox}

\begin{exercicebox}[Exercice 3 : Loi Uniforme (Simulation vs Théorie)]
Soit $X \sim \text{Unif}(a=5, b=15)$. Les valeurs théoriques sont $E[X] = \frac{a+b}{2}$ et $\text{Var}(X) = \frac{(b-a)^2}{12}$.

\textbf{Votre tâche (avec NumPy) :}
\begin{enumerate}
    \item Calculer l'espérance et la variance théoriques.
    \item Générer $N=100000$ échantillons aléatoires de $X$ avec \texttt{np.random.uniform}.
    \item Calculer l'espérance empirique (\texttt{np.mean}) et la variance empirique (\texttt{np.var}) des échantillons.
    \item Comparer les résultats empiriques aux résultats théoriques.
\end{enumerate}
\end{exercicebox}

\begin{exercicebox}[Exercice 4 : Loi Uniforme (Vérification de la PDF)]
Pour $X \sim \text{Unif}(5, 15)$, la PDF est $f(x) = \frac{1}{10}$ sur $[5, 15]$.
La probabilité $P(7 \le X \le 10)$ est $\int_7^{10} \frac{1}{10} dx = \frac{10-7}{10} = 0.3$.

\textbf{Votre tâche (avec NumPy) :}
\begin{enumerate}
    \item Utiliser les échantillons de $X$ de l'exercice 3.
    \item Calculer la probabilité empirique $P(7 \le X \le 10)$ en comptant la proportion d'échantillons qui tombent dans cet intervalle.
    \item Comparer le résultat empirique à la valeur théorique $0.3$.
\end{enumerate}
\end{exercicebox}

\begin{exercicebox}[Exercice 5 : Loi Exponentielle (Simulation vs Théorie)]
Soit $X \sim \text{Exp}(\lambda=0.5)$. Les valeurs théoriques sont $E[X] = \frac{1}{\lambda}$ et $\text{Var}(X) = \frac{1}{\lambda^2}$.

Note : \texttt{np.random.exponential} prend un paramètre "scale" $\beta = 1/\lambda$.

\textbf{Votre tâche (avec NumPy) :}
\begin{enumerate}
    \item Définir $\lambda$ et calculer $E[X]$ et $\text{Var}(X)$ théoriques.
    \item Calculer le paramètre $\beta$ (scale) pour NumPy.
    \item Générer $N=100000$ échantillons aléatoires de $X$.
    \item Calculer et comparer les espérances et variances empiriques et théoriques.
\end{enumerate}
\end{exercicebox}

\begin{exercicebox}[Exercice 6 : Loi Exponentielle (Vérification de la CDF)]
Pour $X \sim \text{Exp}(\lambda=0.5)$, la CDF est $F(x) = 1 - e^{-\lambda x}$.
Calculons $P(X \le 3) = F(3) = 1 - e^{-0.5 \times 3}$.

\textbf{Votre tâche (avec NumPy) :}
\begin{enumerate}
    \item Calculer la valeur théorique $F(3)$.
    \item Utiliser les échantillons de $X$ de l'exercice 5.
    \item Calculer la probabilité empirique $P(X \le 3)$ en comptant la proportion d'échantillons $\le 3$.
    \item Comparer les deux valeurs.
\end{enumerate}
\end{exercicebox}

\begin{exercicebox}[Exercice 7 : Propriété de Non-Mémoire (Exponentielle)]
Nous allons vérifier numériquement la propriété de non-mémoire $P(X > s+t \mid X > s) = P(X > t)$ en utilisant les échantillons de $X$ de l'exercice 5 ($\lambda=0.5$).

\textbf{Votre tâche (avec NumPy) :}
\begin{enumerate}
    \item Choisir $s=1$ et $t=2$.
    \item Calculer $P(X > t)$ (théoriquement $e^{-\lambda t}$). Calculer la probabilité empirique (proportion d'échantillons $> t$).
    \item Calculer $P(X > s+t \mid X > s)$ empiriquement :
        \begin{itemize}
            \item Filtrer les échantillons pour ne garder que ceux où $X > s$.
            \item Parmi ce sous-ensemble, calculer la proportion de ceux où $X > s+t$.
        \end{itemize}
    \item Comparer les deux probabilités empiriques.
\end{enumerate}
\end{exercicebox}

\begin{exercicebox}[Exercice 8 : Théorème de Transfert (LOTUS)]
Soit $X \sim \text{Unif}(0, 2)$. La PDF est $f(x)=1/2$.
Soit $g(X) = X^2$. Nous voulons $E[g(X)] = E[X^2]$.
Théoriquement : $E[X^2] = \int_0^2 x^2 f(x) \, dx = \int_0^2 x^2 (1/2) \, dx = \frac{1}{2} [\frac{x^3}{3}]_0^2 = \frac{1}{2} (\frac{8}{3}) = 4/3$.

\textbf{Votre tâche (avec NumPy) :}
\begin{enumerate}
    \item Générer $N=100000$ échantillons $X \sim \text{Unif}(0, 2)$.
    \item Créer les échantillons $Y = g(X) = X^2$.
    \item Calculer l'espérance empirique $E[Y]$ (la moyenne de $Y$).
    \item Comparer le résultat empirique à la valeur théorique $4/3$.
\end{enumerate}
\end{exercicebox}

\begin{exercicebox}[Exercice 9 : Linéarité de l'Espérance (E[aX+b])]
Soit $X \sim \text{Unif}(5, 15)$ (de l'exercice 3). Nous savons que $E[X] = 10$.
Soit $Y = 5X - 3$.
Par linéarité, l'espérance théorique est $E[Y] = E[5X - 3] = 5E[X] - 3$.

\textbf{Votre tâche (avec NumPy) :}
\begin{enumerate}
    \item Calculer $E[Y]$ théoriquement en utilisant $E[X] = 10$.
    \item Utiliser les échantillons $X$ de l'exercice 3.
    \item Créer les échantillons $Y = 5 \times X - 3$.
    \item Calculer l'espérance empirique $E[Y]$ (la moyenne de $Y$).
    \item Comparer les deux résultats.
\end{enumerate}
\end{exercicebox}

\begin{exercicebox}[Exercice 10 : Propriétés de la Variance (Var(aX+b))]
Soit $X \sim \text{Unif}(5, 15)$ (de l'exercice 3). $\text{Var}(X) = \frac{(15-5)^2}{12} = 100/12 \approx 8.333$.
Soit $Y = 5X - 3$.
Théoriquement : $\text{Var}(Y) = \text{Var}(5X - 3) = \text{Var}(5X) = 5^2 \text{Var}(X) = 25 \times \text{Var}(X)$.

\textbf{Votre tâche (avec NumPy) :}
\begin{enumerate}
    \item Calculer $\text{Var}(Y)$ théoriquement en utilisant $\text{Var}(X) = 100/12$.
    \item Utiliser les échantillons $Y$ de l'exercice 9.
    \item Calculer la variance empirique $\text{Var}(Y)$ (avec \texttt{np.var}).
    \item Comparer les deux résultats.
\end{enumerate}
\end{exercicebox}