\documentclass{article}

% --- GESTION DES MARGES DE PAGE ---
\usepackage[a4paper, top=2.5cm, bottom=2.5cm, left=3cm, right=3cm]{geometry}

% --- PRÉAMBULE STANDARD ---
\usepackage[utf8]{inputenc}
\usepackage[T1]{fontenc}
\usepackage{lmodern}
\usepackage[french]{babel}
\usepackage{parskip} % NOUVEAU : Supprime l'indentation et ajoute espace inter-paragraphe
\usepackage{xcolor}
\usepackage{tcolorbox}
\usepackage{listings}
\usepackage{amsmath}
\usepackage{graphicx} % Requis pour inclure des images
\usepackage{amssymb} % Pour les symboles mathématiques comme \subseteq
\usepackage{sectsty} % Pour le style des sections
\usepackage{etoolbox} % Pour les conditions
\usepackage[dvipsnames]{xcolor}
\usepackage{pgfplots} % Le package principal pour les graphiques
\usepackage{enumitem}
\usepackage{diagbox}

\setlist[itemize,1]{label=$\cdot$}

\usetikzlibrary{
    matrix, 
    patterns.meta,
    calc,
    positioning,
    decorations.pathreplacing,
    trees,
    backgrounds
}

% Redéfinir le symbole pour le premier niveau de la liste
\renewcommand{\labelitemi}{$\cdot$}
\renewcommand{\labelitemii}{$\circ$}
\renewcommand{\labelitemiii}{$\ast$}

% Styles de hachures (inchangés)
\tikzset{
  red_hatch/.style={
    pattern={Lines[angle=45, line width=0.8pt, distance=4pt]}, 
    pattern color=red
  },
  blue_hatch/.style={
    pattern={Lines[angle=-45, line width=0.8pt, distance=4pt]}, 
    pattern color=blue
  },
  purple_hatch/.style={
    pattern={Lines[angle=45, line width=0.8pt, distance=4pt]}, 
    pattern color=red,
    postaction={
      pattern={Lines[angle=-45, line width=0.8pt, distance=4pt]}, 
      pattern color=blue
    }
  }
}

% --- BIBLIOTHÈQUES TCOLORBOX ---
\tcbuselibrary{listings, skins, breakable}

% --- GESTION DES LIENS HYPERTEXTE ---
\usepackage[colorlinks=true, linkcolor=black, urlcolor=blue]{hyperref}

% --- SOULIGNER LES TITRES ET SOUS-TITRES ---
\sectionfont{\underline}
\subsectionfont{\underline}
\subsubsectionfont{\underline}

% --- PAGE DE GARDE AMÉLIORÉE ---
\makeatletter
\renewcommand{\maketitle}{%
\begin{titlepage}
\centering
\vspace*{\stretch{1.5}}
{\Huge \bfseries Mes Notes de Lecture\par}
\vspace{0.4cm}
\rule{0.8\linewidth}{0.4pt}
\vspace{1cm}
{\LARGE \bfseries Introduction à la Probabilité\par}
\vspace*{\stretch{2.5}}
{\Large \scshape Lou Brunet\par}
\vspace{0.5cm}
{\large \today\par}
\vspace*{\stretch{1}}
\end{titlepage}
}
\makeatother

% ==================================================================
% --- MODIFIÉ : DÉFINITION DES COULEURS STYLE VS CODE (LIGHT) ---
% ==================================================================
\definecolor{vscodeBlue}{HTML}{569CD6}
\definecolor{vscodeOrange}{HTML}{CE9178}
\definecolor{vscodeGreen}{HTML}{6A9955}
\definecolor{vscodePurple}{HTML}{C586C0}
\definecolor{vscodeGray}{HTML}{9B9B9B}
\definecolor{codeBackground}{HTML}{F8F8F8} % Fond gris très clair
\definecolor{codeText}{HTML}{242424}       % Texte principal (presque noir)
\definecolor{codeGray}{HTML}{A0A0A0}       % Numéros de ligne (gris moyen)

% ==================================================================
% --- MODIFIÉ : CONFIGURATION DU STYLE LISTINGS (LIGHT) ---
% ==================================================================
\lstdefinestyle{vscode}{
    language=Python,
    backgroundcolor=\color{codeBackground},     % Utilise le nouveau fond F8F8F8
    basicstyle=\ttfamily\small\color{codeText}, % Texte principal noir
    keywordstyle=\color{vscodeBlue},
    stringstyle=\color{vscodeOrange},
    commentstyle=\color{vscodeGreen},
    numberstyle=\tiny\color{codeGray},         % Numéros de ligne en gris
    otherkeywords={self, True, False, None},
    keywordstyle=[2]\color{vscodePurple},
    showstringspaces=false,
    breaklines=true,
    frame=none,
    tabsize=4
}

% --- STYLE DE L'OUTPUT ---
\lstdefinestyle{outputstyle}{
    basicstyle=\ttfamily\small\color{codeText}, % Texte principal noir
    breaklines=true,
    frame=none
}

% ==================================================================
% --- MODIFIÉ : DÉFINITION DES CELLULES DE CODE ET OUTPUT (LIGHT) ---
% ==================================================================
% Elles utilisent maintenant le même style "sidebar" que les autres boîtes.

\newtcblisting{codecell}{
  skin=enhanced, % Pour la bordure latérale
  arc=0mm,       % Coins carrés
  boxrule=0pt,   % Pas de cadre
  colback=codeBackground, % Fond clair (F8F8F8)
  borderline west={2pt}{0pt}{vscodeBlue}, % Barre latérale bleue
  fonttitle=\bfseries\color{vscodeBlue}, % Titre en bleu
  listing only,
  listing options={style=vscode, basicstyle=\ttfamily\footnotesize\color{codeText}}, % TEXTE NOIR
  left=3mm, right=3mm, top=2mm, bottom=2mm, % Padding (identique à sidebarstyle)
  boxsep=0mm, % (identique à sidebarstyle)
  breakable  % (identique à sidebarstyle)
}
\newtcblisting{outputcell}{
  skin=enhanced, % Pour la bordure latérale
  arc=0mm,       % Coins carrés
  boxrule=0pt,   % Pas de cadre
  colback=black!5, % Fond gris très clair (Output)
  borderline west={2pt}{0pt}{corrColor}, % Barre latérale grise
  fonttitle=\bfseries\color{corrColor}, % Titre en gris
  listing only,
  listing options={style=outputstyle, basicstyle=\ttfamily\footnotesize\color{codeText}}, % TEXTE NOIR
  left=3mm, right=3mm, top=2mm, bottom=2mm, % Padding (identique à sidebarstyle)
  boxsep=0mm, % (identique à sidebarstyle)
  breakable  % (identique à sidebarstyle)
}

% --- DÉFINITION DES COULEURS POUR DÉF/THÉO/PREUVE/INTUITION/EXEMPLE ---
\definecolor{defColor}{HTML}{1b1f3a}       % Bleu nuit pour les définitions
\definecolor{theoColor}{HTML}{53354a}      % Violet aubergine pour les théorèmes
\definecolor{proofColor}{HTML}{a64942}     % Rouge brique pour les preuves
\definecolor{intuitionColor}{HTML}{16A085} % Turquoise pour l'intuition
\definecolor{exampleColor}{HTML}{4a6982}   % Bleu ardoise pour les exemples
\definecolor{exoColor}{HTML}{1E8449}       % Vert
\definecolor{corrColor}{HTML}{7F8C8D}      % Gris
\definecolor{remarqueColor}{HTML}{D35400} 

% ==================================================================
% --- DÉFINITION DU STYLE DE BASE (MARGES RÉDUITES) ---
% ==================================================================

% --- STYLE "ORGANIQUE" AVEC BARRE LATÉRALE (Pour TOUTES les boîtes) ---
\tcbset{
    sidebarstyle/.style={
        skin=enhanced, % Nécessaire pour les bordures partielles
        arc=0mm,       % Coins carrés
        boxrule=0pt,   % Pas de cadre
        colback=white, % Fond blanc
        colframe=white,
        coltitle=white,
        fontupper=\color{black},
        left=3mm, right=3mm, top=2mm, bottom=2mm, % Padding interne réduit
        boxsep=0mm, % TRÈS IMPORTANT
        breakable
    }
}

% ==================================================================
% --- DÉFINITION DE TOUTES LES CELLULES (TEXTE) ---
% ==================================================================

% --- CELLULES "SIDEBAR" (Pour le contenu théorique) ---
\newtcolorbox{definitionbox}[1][]{
  sidebarstyle,
  borderline west={2pt}{0pt}{defColor}, % Barre latérale gauche
  fonttitle=\bfseries\color{defColor},  % Titre en couleur (sans fond)
  title=Définition\ifstrempty{#1}{}{ : #1}
}
\newtcolorbox{theorembox}[1][]{
  sidebarstyle,
  borderline west={2pt}{0pt}{theoColor},
  fonttitle=\bfseries\color{theoColor},
  title=Théorème\ifstrempty{#1}{}{ : #1}
}
\newtcolorbox{proofbox}[1][]{
  sidebarstyle,
  borderline west={2pt}{0pt}{proofColor},
  fonttitle=\bfseries\color{proofColor},
  title=Preuve\ifstrempty{#1}{}{ : #1}
}
\newtcolorbox{intuitionbox}[1][]{
  sidebarstyle,
  borderline west={2pt}{0pt}{intuitionColor},
  fonttitle=\bfseries\color{intuitionColor},
  title=Intuition\ifstrempty{#1}{}{ : #1}
}
\newtcolorbox{remarquebox}[1][]{
  sidebarstyle,
  borderline west={2pt}{0pt}{remarqueColor},
  fonttitle=\bfseries\color{remarqueColor},
  title=Remarque\ifstrempty{#1}{}{ : #1}
}


% --- CELLULES "SIDEBAR" (Pour les applications) ---
\newtcolorbox{examplebox}[1][]{
  sidebarstyle,
  borderline west={2pt}{0pt}{exampleColor},
  fonttitle=\bfseries\color{exampleColor},
  title=Exemple\ifstrempty{#1}{}{ : #1}
}
\newtcolorbox{exercicebox}[1][]{
    sidebarstyle,
    borderline west={2pt}{0pt}{corrColor},
    fonttitle=\bfseries\color{corrColor},
    title=#1 % <-- On utilise directement l'argument fourni
}
\newtcolorbox{correctionbox}[1][]{
    sidebarstyle,
    borderline west={2pt}{0pt}{exoColor},
    fonttitle=\bfseries\color{exoColor},
    title=#1
}


% =============================================
% --- CORPS DU DOCUMENT ---
% =============================================
\begin{document}

\maketitle

\newpage
\phantomsection
\addcontentsline{toc}{section}{Sommaire}
\tableofcontents 
\newpage

\newpage
\section{Probabilités et Dénombrement}

\subsection{Concepts fondamentaux}

Avant de pouvoir calculer des probabilités, il est essentiel d'établir un vocabulaire commun pour décrire les expériences aléatoires.

\begin{intuitionbox}[Nécessité d'un Cadre Formel]
Avant de calculer des probabilités, il est crucial de définir les règles du jeu :

\textbf{Qu'est-ce qui peut arriver ?}

On définit l'ensemble de tous les résultats possibles de l'expérience.

\textbf{À quoi s'intéresse-t-on ?} 

On identifie les sous-ensembles de résultats spécifiques qui nous intéressent.

Ces deux idées nous conduisent aux notions d'Univers et d'Événement, qui sont les piliers de toute théorie des probabilités.
\end{intuitionbox}

Cette intuition se traduit formellement par deux définitions clés :

\begin{definitionbox}[Concepts Fondamentaux]
\textbf{Univers (ou Espace Échantillon), $S$ :} 

L'ensemble de tous les résultats possibles d'une expérience aléatoire.

\textbf{Événement, $A$ :} 

Un sous-ensemble de l'univers ($A \subseteq S$). C'est un ensemble de résultats auxquels on s'intéresse.
\end{definitionbox}

Un exemple simple permet de solidifier ces concepts :

\begin{examplebox}[Univers et Événement]
Pour l'expérience du "lancer d'un dé à six faces" :

L'\textbf{univers} est $S = \{1, 2, 3, 4, 5, 6\}$.

"Obtenir un nombre impair" est un événement, représenté par le sous-ensemble $A = \{1, 3, 5\}$.
\end{examplebox}

\subsection{Définition Naïve de la Probabilité}

Pour de nombreuses expériences simples, comme lancer un dé non pipé, chaque résultat possible est "équiprobable". Cette hypothèse est la base de la première définition formelle de la probabilité.

\begin{definitionbox}[Probabilité Naïve]
Pour une expérience où chaque issue dans un espace échantillon fini $S$ est équiprobable, la probabilité d'un événement $A$ est le rapport du nombre d'issues favorables à $A$ sur le nombre total d'issues :
$$ P(A) = \frac{\text{Nombre d'issues favorables}}{\text{Nombre total d'issues}} = \frac{|A|}{|S|} $$
\end{definitionbox}

Appliquons cette formule à quelques cas classiques :

\begin{examplebox}[Applications de la définition naïve]
\begin{enumerate}
    \item \textbf{Lancer une pièce équilibrée :}
    L'espace échantillon est $S = \{\text{Pile, Face}\}$, donc $|S| = 2$.
    Si l'événement $A$ est "obtenir Pile", alors $A = \{\text{Pile}\}$ et $|A| = 1$.
    La probabilité est $P(A) = \frac{1}{2}$.

    \item \textbf{Lancer un dé à six faces non pipé :}
    L'espace échantillon est $S = \{1, 2, 3, 4, 5, 6\}$, donc $|S| = 6$.
    Si l'événement $B$ est "obtenir un nombre pair", alors $B = \{2, 4, 6\}$ et $|B| = 3$.
    La probabilité est $P(B) = \frac{3}{6} = \frac{1}{2}$.

    \item \textbf{Tirer une carte d'un jeu de 52 cartes :}
    L'espace échantillon $S$ contient 52 cartes, donc $|S| = 52$.
    Si l'événement $C$ est "tirer un Roi", il y a 4 Rois dans le jeu, donc $|C| = 4$.
    La probabilité est $P(C) = \frac{4}{52} = \frac{1}{13}$.
\end{enumerate}
\end{examplebox}

\subsection{Permutations (Arrangements)}

Le dénombrement, qui est l'art de compter les tailles $|A|$ et $|S|$, est fondamental pour appliquer la définition naïve. Le premier outil que nous verrons est la permutation, qui compte les arrangements \textbf{ordonnés}.

\begin{definitionbox}[Permutation de $k$ objets parmi $n$]
Le nombre de façons d'arranger $k$ objets choisis parmi $n$ objets distincts (où l'ordre compte et il n'y a pas de répétition) est noté $P(n, k)$ ou $A_n^k$ et est défini par :
$$ P(n, k) = \frac{n!}{(n-k)!} $$
où $n!$ est la factorielle de $n$, et par convention $0! = 1$.
\end{definitionbox}

Cette formule peut sembler abstraite, mais elle provient d'un raisonnement logique simple par "cases" :

\begin{intuitionbox}[Permutations de $k$ parmi $n$]
Pour placer $k$ objets dans un ordre spécifique en les choisissant parmi $n$ objets disponibles, on a $n$ choix pour la première position, $(n-1)$ choix pour la deuxième, ..., et $(n-k+1)$ choix pour la $k$-ième position. Cela donne $n \times (n-1) \times \cdots \times (n-k+1)$ arrangements. Ce produit contient $k$ termes. Il est égal à $\frac{n!}{(n-k)!}$, car cela revient à diviser la suite complète $n!$ par les facteurs non utilisés $(n-k) \times (n-k-1) \times \cdots \times 1$.
\end{intuitionbox}

Voyons une application classique de ce principe :

\begin{examplebox}[Permutations de $k$ parmi $n$]
\textbf{Podium d'une course :} Une course réunit 8 coureurs. Combien y a-t-il de podiums (1er, 2e, 3e) possibles ?

On cherche le nombre de façons d'ordonner 3 coureurs parmi 8 : $P(8, 3)$. 
$$ P(8, 3) = \frac{8!}{(8-3)!} = \frac{8!}{5!} = 8 \times 7 \times 6 = 336 $$
Il y a 336 podiums possibles.
\end{examplebox}

\subsection{Le Coefficient Binomial}

Que se passe-t-il si l'ordre ne compte pas ? Au lieu de compter des podiums, nous voulons compter des comités. C'est le rôle du coefficient binomial.

\begin{theorembox}[Formule du Coefficient Binomial]
Le nombre de façons de choisir $k$ objets parmi un ensemble de $n$ objets distincts (sans remise et sans ordre) est donné par le coefficient binomial :
$$ \binom{n}{k} = \frac{n!}{k!(n-k)!} $$
\end{theorembox}

% NOUVEAU :
La preuve de cette formule repose sur un argument combinatoire élégant : nous allons compter la même chose (les permutations) de deux façons différentes.
% FIN NOUVEAU

\newpage

\begin{proofbox}
Considérons le nombre de permutations de $k$ objets parmi $n$, noté $P(n,k)$.
\begin{enumerate}
    \item \textbf{Méthode 1 :} Par définition (vue ci-dessus), nous savons que $P(n,k) = \frac{n!}{(n-k)!}$.
    
    \item \textbf{Méthode 2 :} Nous pouvons construire une telle permutation en deux étapes successives :
    \begin{itemize}
        \item D'abord, \textbf{choisir un sous-ensemble} de $k$ objets parmi $n$ (l'ordre ne compte pas). C'est le nombre que nous cherchons, notons-le $\binom{n}{k}$.
        \item Ensuite, \textbf{ordonner} ces $k$ objets choisis. Il y a $k!$ façons de les arranger.
    \end{itemize}
    Le nombre total de permutations est donc le produit de ces étapes : $P(n,k) = \binom{n}{k} \times k!$.
\end{enumerate}
En égalisant les deux méthodes, on obtient :
\[ \binom{n}{k} \cdot k! = \frac{n!}{(n-k)!} \]
En divisant par $k!$, on trouve bien la formule :
\[ \binom{n}{k} = \frac{n!}{k!(n-k)!} \]
\end{proofbox}

% NOUVEAU :
L'intuition visuelle derrière cette preuve est de voir comment chaque "choix" (une colonne du tableau) génère $k!$ "ordres" (les lignes de cette colonne).
% FIN NOUVEAU

\begin{intuitionbox}
Pour rendre cela concret, voici le cas $\binom{5}{3}$.  
Il y a 10 sous-ensembles de 3 éléments parmi $\{a,b,c,d,e\}$. Chacun donne lieu à $3! = 6$ permutations.  
Le tableau ci-dessous montre \textbf{toutes les 60 permutations}, regroupées par sous-ensemble :

\vspace{3mm}

\begin{center}
\small
\renewcommand{\arraystretch}{0.9}
\setlength{\tabcolsep}{2pt}
\begin{tabular}{|c|c|c|c|c|c|c|c|c|c|}
\hline
\textbf{$\{a,b,c\}$} & \textbf{$\{a,b,d\}$} & \textbf{$\{a,b,e\}$} & \textbf{$\{a,c,d\}$} & \textbf{$\{a,c,e\}$} & \textbf{$\{a,d,e\}$} & \textbf{$\{b,c,d\}$} & \textbf{$\{b,c,e\}$} & \textbf{$\{b,d,e\}$} & \textbf{$\{c,d,e\}$} \\
\hline
$abc$ & $abd$ & $abe$ & $acd$ & $ace$ & $ade$ & $bcd$ & $bce$ & $bde$ & $cde$ \\
\hline
$acb$ & $adb$ & $aeb$ & $adc$ & $aec$ & $aed$ & $bdc$ & $bec$ & $bed$ & $ced$ \\
\hline
$bac$ & $bad$ & $bae$ & $cad$ & $cae$ & $dae$ & $cbd$ & $ceb$ & $dbe$ & $dce$ \\
\hline
$bca$ & $bda$ & $bea$ & $cda$ & $cea$ & $dea$ & $cdb$ & $ceb$ & $deb$ & $dec$ \\
\hline
$cab$ & $dab$ & $eab$ & $dac$ & $eac$ & $ead$ & $dbc$ & $ebc$ & $edb$ & $ecd$ \\
\hline
$cba$ & $dba$ & $eba$ & $dca$ & $eca$ & $eda$ & $dcb$ & $ebc$ & $edb$ & $edc$ \\
\hline
\end{tabular}
\end{center}

\vspace{3mm}

\smallskip

Chaque colonne correspond à \textbf{un seul et même choix non ordonné} (par exemple $\{a,b,c\}$), mais à 6 listes différentes selon l’ordre.  
Ainsi, pour obtenir le nombre de \textit{choix non ordonnés}, on divise le nombre total de listes ($60$) par le nombre d’ordres par groupe ($6$) :
\[
\binom{5}{3} = \frac{60}{6} = 10.
\]
\end{intuitionbox}

L'application la plus directe est le tirage d'un groupe où l'ordre n'importe pas :

\begin{examplebox}[Utilisation du Coefficient Binomial]
    \textbf{Comité d'étudiants :} De combien de manières peut-on former un comité de 3 étudiants à partir d'une classe de 10 ? L'ordre ne compte pas.
    $$ \binom{10}{3} = \frac{10!}{3!(10-3)!} = \frac{10 \times 9 \times 8}{3 \times 2 \times 1} = 120 \text{ comités possibles.} $$
\end{examplebox}

\newpage

\subsection{Identité de Vandermonde}

Les coefficients binomiaux obéissent à de nombreuses identités. L'identité de Vandermonde est l'une des plus utiles, car elle montre comment décomposer un problème de comptage complexe en sous-problèmes.

\begin{theorembox}[Identité de Vandermonde]
Cette identité offre une relation remarquable entre les coefficients binomiaux. Pour des entiers non négatifs $m, n$ et $k$, on a :
$$ \binom{m+n}{k} = \sum_{j=0}^{k} \binom{m}{j} \binom{n}{k-j} $$
\end{theorembox}

% NOUVEAU :
La preuve la plus intuitive est une "preuve par l'histoire" (proof by story), qui consiste à trouver un scénario de dénombrement que les deux côtés de l'équation résolvent.
% FIN NOUVEAU

\begin{proofbox}[Preuve combinatoire]
Imaginons un groupe composé de $m$ hommes et $n$ femmes. Nous souhaitons former un comité de $k$ personnes. Nous allons compter le nombre de comités possibles de deux façons.

\textbf{Côté gauche : $\binom{m+n}{k}$}
Le groupe total contient $m+n$ personnes. Le nombre de façons de choisir un comité de $k$ personnes parmi ce total est, par définition, $\binom{m+n}{k}$.

\textbf{Côté droit : $\sum_{j=0}^{k} \binom{m}{j} \binom{n}{k-j}$}
Nous pouvons compter le même nombre en conditionnant sur le nombre d'hommes (noté $j$) dans le comité. Un comité de $k$ personnes doit contenir $j$ hommes ET $k-j$ femmes, où $j$ peut aller de $0$ à $k$.
\begin{itemize}
    \item Pour $j=0$ : Choisir 0 homme ($\binom{m}{0}$) ET $k$ femmes ($\binom{n}{k}$).
    \item Pour $j=1$ : Choisir 1 homme ($\binom{m}{1}$) ET $k-1$ femmes ($\binom{n}{k-1}$).
    \item ...
    \item Pour $j=k$ : Choisir $k$ hommes ($\binom{m}{k}$) ET 0 femme ($\binom{n}{0}$).
\end{itemize}
Puisque ces cas (0 homme, 1 homme, etc.) sont mutuellement exclusifs, le nombre total de comités est la somme de toutes ces possibilités :
\[ \sum_{j=0}^{k} \binom{m}{j} \binom{n}{k-j} \]
Puisque les deux côtés comptent exactement la même chose (le nombre total de comités), ils doivent être égaux.
\end{proofbox}

Vérifions cette identité avec un exemple numérique concret, en reprenant l'analogie du comité :

\begin{examplebox}[Application de l'Identité de Vandermonde]
On veut former un comité de 3 personnes ($k=3$) à partir d'un groupe de 5 hommes ($m=5$) et 4 femmes ($n=4$).

\textbf{Méthode directe (côté gauche) :}
On choisit 3 personnes parmi les $5+4=9$ au total.
$$ \binom{9}{3} = \frac{9 \times 8 \times 7}{3 \times 2 \times 1} = 84 $$

\textbf{Méthode par cas (côté droit) :}
La somme est $\binom{5}{0}\binom{4}{3} + \binom{5}{1}\binom{4}{2} + \binom{5}{2}\binom{4}{1} + \binom{5}{3}\binom{4}{0} = 84$. Les deux méthodes donnent bien le même résultat.
\end{examplebox}

\newpage

\subsection{Bose-Einstein (Étoiles et Bâtons)}

Jusqu'à présent, nous avons supposé un "tirage sans remise". La statistique de Bose-Einstein, ou plus visuellement la méthode des "étoiles et bâtons", s'attaque au problème du \textbf{tirage avec remise} où l'ordre ne compte pas.

\begin{theorembox}[Combinaisons avec répétition]
Le nombre de façons de distribuer $k$ objets indiscernables dans $n$ boîtes discernables (ou de choisir $k$ objets parmi $n$ avec remise, où l'ordre ne compte pas) est donné par la formule :
$$ \binom{n+k-1}{k} = \binom{n+k-1}{n-1} $$
\end{theorembox}

% NOUVEAU :
La preuve de cette formule est l'un des résultats les plus élégants du dénombrement. L'astuce consiste à transformer le problème de distribution en un problème d'arrangement de symboles.
% FIN NOUVEAU

\begin{proofbox}[Par les "Étoiles et Bâtons"]
Nous cherchons à distribuer $k$ objets indiscernables ($\star$) dans $n$ boîtes discernables.
Nous pouvons représenter n'importe quelle distribution comme une séquence de symboles. Nous avons besoin de $k$ étoiles (les objets) et de $n-1$ bâtons ($|$) pour servir de séparateurs entre les $n$ boîtes.

Par exemple, pour distribuer $k=7$ étoiles dans $n=4$ boîtes, la séquence :
$$ \star\star\star \mid \star \mid \mid \star\star\star $$
correspond à : 3 étoiles dans la boîte 1, 1 étoile dans la boîte 2, 0 étoile dans la boîte 3 (l'espace entre deux bâtons), et 3 étoiles dans la boîte 4.

Chaque arrangement unique de ces symboles correspond à une distribution unique. Le problème revient donc à trouver le nombre de façons d'arranger ces $k$ étoiles et ces $n-1$ bâtons.

Nous avons un total de $n+k-1$ positions à remplir. Le nombre de façons de le faire est simplement le nombre de manières de choisir les $k$ positions pour les étoiles (les autres positions étant automatiquement remplies par des bâtons).
C'est exactement :
$$ \binom{n+k-1}{k} $$
(Ce qui est aussi égal à $\binom{n+k-1}{n-1}$, le nombre de façons de choisir les positions des $n-1$ bâtons).
\end{proofbox}

C'est la méthode parfaite pour tout problème de distribution d'objets identiques :

\begin{examplebox}[Distribution de biens identiques]
De combien de manières peut-on distribuer 10 croissants identiques à 4 enfants ?

Ici, $k=10$ (les croissants, objets indiscernables) et $n=4$ (les enfants, boîtes discernables).
Le nombre de distributions possibles est :
$$ \binom{4+10-1}{10} = \binom{13}{10} = \binom{13}{3} = \frac{13 \times 12 \times 11}{3 \times 2 \times 1} = 13 \times 2 \times 11 = 286 $$
Il y a 286 façons de distribuer les croissants.
\end{examplebox}

\subsection{Principe d'Inclusion-Exclusion}

Comment compter le nombre d'éléments dans l'union de plusieurs ensembles ? Si on additionne simplement leurs tailles, on compte les intersections plusieurs fois. Le principe d'inclusion-exclusion corrige systématiquement ce sur-comptage.

\begin{theorembox}[Principe d'Inclusion-Exclusion pour 3 ensembles]
Pour trois ensembles finis $A$, $B$ et $C$, le nombre d'éléments dans leur union est donné par :
$$ |A \cup B \cup C| = |A| + |B| + |C| - |A \cap B| - |A \cap C| - |B \cap C| + |A \cap B \cap C| $$
\end{theorembox}

% NOUVEAU :
La preuve pour 3 ensembles se fait en appliquant la formule pour 2 ensembles de manière répétée.
% FIN NOUVEAU

\begin{proofbox}
Nous utilisons la formule pour deux ensembles, $|X \cup Y| = |X| + |Y| - |X \cap Y|$, de manière imbriquée.
Posons $X = A \cup B$ et $Y = C$.
\begin{align*}
|A \cup B \cup C| &= |(A \cup B) \cup C| \\
&= |A \cup B| + |C| - |(A \cup B) \cap C|
\end{align*}
Nous devons maintenant développer les deux termes compliqués :
\begin{enumerate}
    \item $|A \cup B| = |A| + |B| - |A \cap B|$
    \item Par distributivité de l'intersection sur l'union, $(A \cup B) \cap C = (A \cap C) \cup (B \cap C)$.
\end{enumerate}
Appliquons la formule pour 2 ensembles à ce deuxième terme :
\[ |(A \cap C) \cup (B \cap C)| = |A \cap C| + |B \cap C| - |(A \cap C) \cap (B \cap C)| \]
Ce qui se simplifie en $|A \cap C| + |B \cap C| - |A \cap B \cap C|$.

Finalement, en substituant tout dans l'équation de départ :
\begin{align*}
|A \cup B \cup C| &= \underbrace{(|A| + |B| - |A \cap B|)}_{|A \cup B|} + |C| \\
                 &\quad - \underbrace{(|A \cap C| + |B \cap C| - |A \cap B \cap C|)}_{|(A \cup B) \cap C|}
\end{align*}
En réarrangeant les termes, on obtient la formule voulue :
\[ |A| + |B| + |C| - |A \cap B| - |A \cap C| - |B \cap C| + |A \cap B \cap C| \]
\end{proofbox}

La formule devient évidente lorsque l'on utilise un diagramme de Venn pour visualiser le sur-comptage et sa correction.

\begin{intuitionbox}[Visualisation avec 3 ensembles]
Le principe d'inclusion-exclusion permet de compter le nombre d'éléments dans une union d'ensembles sans double-comptage. Pour comprendre intuitivement pourquoi on ajoute et soustrait alternativement, considérons trois ensembles $A$, $B$ et $C$ :

\begin{center}
\begin{tikzpicture}[set/.style = {draw,
    circle,
    minimum size = 6cm,
    fill=Rhodamine,
    opacity = 0.4,
    text opacity = 1}]
 
\node (A) [set] {$A$};
\node (B) at (60:4cm) [set] {$B$};
\node (C) at (0:4cm) [set] {$C$};
 
\node at (barycentric cs:A=1,B=1) [left] {$X$};
\node at (barycentric cs:A=1,C=1) [below] {$Y$};
\node at (barycentric cs:B=1,C=1) [right] {$Z$};
\node at (barycentric cs:A=1,B=1,C=1) [] {$T$};
 
\end{tikzpicture}
\end{center}

\textbf{Le problème :} Si on additionne simplement $|A| + |B| + |C|$, on compte certaines zones plusieurs fois :
\begin{itemize}
    \item Les intersections deux à deux ($X$, $Y$, $Z$) sont comptées \textbf{deux fois}
    \item L'intersection triple ($T$) est comptée \textbf{trois fois}
\end{itemize}

\textbf{La solution :} On corrige en soustrayant les intersections deux à deux, mais alors l'intersection triple est comptée :
\begin{itemize}
    \item $+3$ fois dans la somme initiale
    \item $-3$ fois dans la soustraction des intersections deux à deux (car elle appartient à chacune)
    \item Donc $0$ fois au total ! Il faut la rajouter.
\end{itemize}

D'où la formule : $|A \cup B \cup C| = |A| + |B| + |C| - |A \cap B| - |A \cap C| - |B \cap C| + |A \cap B \cap C|$
\end{intuitionbox}

Ce que nous avons fait visuellement pour 3 ensembles peut être généralisé par récurrence à $n$ ensembles. La formule générale suit le même principe d'alternance des signes :

\begin{theorembox}[Principe d'Inclusion-Exclusion généralisé]
Pour $n$ ensembles finis $A_1, A_2, \dots, A_n$, on a :
\begin{align*}
|A_1 \cup A_2 \cup \cdots \cup A_n| = & \sum_{i=1}^n |A_i| \\
& - \sum_{1 \leq i < j \leq n} |A_i \cap A_j| \\
& + \sum_{1 \leq i < j < k \leq n} |A_i \cap A_j \cap A_k| \\
& - \cdots \\
& + (-1)^{n+1} |A_1 \cap A_2 \cap \cdots \cap A_n|
\end{align*}
Ce qui s'écrit plus compactement :
$$ \left| \bigcup_{i=1}^n A_i \right| = \sum_{k=1}^n (-1)^{k+1} \sum_{1 \leq i_1 < i_2 < \cdots < i_k \leq n} |A_{i_1} \cap A_{i_2} \cap \cdots \cap A_{i_k}| $$
\end{theorembox}

% NOUVEAU :
La preuve formelle que cette formule gigantesque fonctionne est fascinante. Il suffit de montrer que n'importe quel élément $x$ de l'union, peu importe à combien d'ensembles il appartient, est compté \textbf{exactement une fois} au final.
% FIN NOUVEAU

\begin{proofbox}[Preuve par comptage d'un élément]
Considérons un élément $x$ qui appartient à exactement $k$ ensembles parmi les $n$ ensembles $A_1, \ldots, A_n$ (où $k \ge 1$). Nous devons montrer que $x$ est compté exactement 1 fois par la formule.

Analysons combien de fois $x$ est compté dans chaque somme de la formule :
\begin{itemize}
    \item \textbf{Première somme ($\sum |A_i|$)} : $x$ est dans $k$ ensembles, donc il est ajouté $k$ fois. Le nombre de fois est $\binom{k}{1}$.
    
    \item \textbf{Deuxième somme ($-\sum |A_i \cap A_j|$)} : $x$ est compté (et soustrait) pour chaque \textit{paire} d'ensembles auxquels il appartient. Comme il appartient à $k$ ensembles, il y a $\binom{k}{2}$ telles paires.
    
    \item \textbf{Troisième somme ($+\sum |A_i \cap A_j \cap A_k|$)} : $x$ est ajouté pour chaque \textit{triplet} d'ensembles auxquels il appartient. Il y en a $\binom{k}{3}$.
    
    \item \textbf{Et ainsi de suite...}
\end{itemize}
Au total, l'élément $x$ est compté :
$$ \text{Total} = \binom{k}{1} - \binom{k}{2} + \binom{k}{3} - \cdots + (-1)^{k-1}\binom{k}{k} \text{ fois.} $$
Pour évaluer cette somme, rappelons l'identité fondamentale du binôme de Newton :
$$ (1 + x)^k = \sum_{j=0}^{k} \binom{k}{j} x^j = \binom{k}{0} + \binom{k}{1}x + \binom{k}{2}x^2 + \cdots $$
Si nous posons $x = -1$, nous obtenons :
$$ (1-1)^k = 0 = \binom{k}{0} - \binom{k}{1} + \binom{k}{2} - \binom{k}{3} + \cdots + (-1)^k\binom{k}{k} $$
Sachant que $\binom{k}{0} = 1$, on a :
$$ 0 = 1 - \left( \binom{k}{1} - \binom{k}{2} + \binom{k}{3} - \cdots + (-1)^{k-1}\binom{k}{k} \right) $$
En réarrangeant, on trouve :
$$ 1 = \binom{k}{1} - \binom{k}{2} + \binom{k}{3} - \cdots + (-1)^{k-1}\binom{k}{k} $$
Cela prouve que n'importe quel élément de l'union est compté exactement une fois.
\end{proofbox}

Ce principe est très utile en probabilité, car il permet de calculer $P(A \cup B \cup \dots)$ en se basant sur les probabilités des intersections, qui sont souvent plus faciles à trouver.

\begin{examplebox}[Application probabiliste]
On lance trois dés équilibrés. Quelle est la probabilité d'obtenir au moins un 6 ?

\textbf{Solution avec inclusion-exclusion :}

Soit $A$ = "le premier dé montre 6", $B$ = "le deuxième dé montre 6", $C$ = "le troisième dé montre 6".

On veut $P(A \cup B \cup C)$.

\begin{align*}
P(A \cup B \cup C) &= P(A) + P(B) + P(C) \\
&\quad - P(A \cap B) - P(A \cap C) - P(B \cap C) \\
&\quad + P(A \cap B \cap C) \\
&= \frac{1}{6} + \frac{1}{6} + \frac{1}{6} - \frac{1}{36} - \frac{1}{36} - \frac{1}{36} + \frac{1}{216} \\
&= \frac{3}{6} - \frac{3}{36} + \frac{1}{216} = \frac{1}{2} - \frac{1}{12} + \frac{1}{216} \\
&= \frac{108 - 18 + 1}{216} = \frac{91}{216} \approx 0.421
\end{align*}

\textbf{Vérification par la méthode complémentaire :}

La probabilité de n'obtenir aucun 6 est $\left(\frac{5}{6}\right)^3 = \frac{125}{216}$, donc la probabilité d'au moins un 6 est $1 - \frac{125}{216} = \frac{91}{216}$.
\end{examplebox}

\subsection{Exercices}

Cette série d'exercices vise à renforcer votre compréhension des concepts fondamentaux du dénombrement et de la probabilité naïve. La difficulté augmente progressivement.

% --- Concepts de Base et Probabilité Naïve ---

\begin{exercicebox}[Exercice 1 : Univers et Événements]
On lance deux dés à 6 faces, un rouge et un bleu.
\begin{enumerate}
    \item Décrivez l'univers $S$ de cette expérience. Quelle est sa taille $|S|$ ?
    \item Soit $A$ l'événement "la somme des dés est égale à 7". Listez les issues appartenant à $A$. Calculez $P(A)$.
    \item Soit $B$ l'événement "le dé rouge montre un 3". Listez les issues appartenant à $B$. Calculez $P(B)$.
    \item Décrivez l'événement $A \cap B$ et calculez sa probabilité.
\end{enumerate}
\end{exercicebox}

\begin{exercicebox}[Exercice 2 : Tirage de Cartes (Prob. Naïve)]
On tire une carte au hasard d'un jeu standard de 52 cartes.
\begin{enumerate}
    \item Quelle est la probabilité de tirer un Roi ?
    \item Quelle est la probabilité de tirer une carte rouge (Cœur ou Carreau) ?
    \item Quelle est la probabilité de tirer une figure (Valet, Dame, Roi) ?
    \item Quelle est la probabilité de tirer un As rouge ?
\end{enumerate}
\end{exercicebox}

\begin{exercicebox}[Exercice 3 : Urne Simple (Prob. Naïve)]
Une urne contient 5 boules rouges, 3 boules bleues et 2 boules vertes. On tire une boule au hasard.
\begin{enumerate}
    \item Quelle est la probabilité qu'elle soit bleue ?
    \item Quelle est la probabilité qu'elle ne soit pas verte ?
\end{enumerate}
\end{exercicebox}

% --- Permutations ---

\begin{exercicebox}[Exercice 4 : Anagrammes (Permutation Simple)]
Combien d'anagrammes distinctes peut-on former avec les lettres du mot "MATHS" ?
\end{exercicebox}

\begin{exercicebox}[Exercice 5 : Course (Arrangement)]
Dix athlètes participent à une course. Combien y a-t-il de classements possibles pour les 3 premières places (médaille d'or, d'argent, de bronze) ?
\end{exercicebox}

\begin{exercicebox}[Exercice 6 : Anagrammes (Permutation avec Répétition)]
Combien d'anagrammes distinctes peut-on former avec les lettres du mot "PROBABILITE" ?
\end{exercicebox}

% --- Combinaisons ---

\begin{exercicebox}[Exercice 7 : Choix d'un Comité (Combinaison)]
Une classe compte 15 étudiants. De combien de manières peut-on choisir un comité de 4 étudiants ?
\end{exercicebox}

\begin{exercicebox}[Exercice 8 : Mains de Poker (Combinaison)]
Dans un jeu de 52 cartes, combien de "mains" de 5 cartes différentes peut-on former ?
\end{exercicebox}

\begin{exercicebox}[Exercice 9 : Comité Mixte (Combinaison)]
À partir d'un groupe de 6 hommes et 4 femmes, combien de comités de 3 personnes peut-on former contenant exactement 2 hommes et 1 femme ?
\end{exercicebox}

\begin{exercicebox}[Exercice 10 : Probabilité avec Combinaisons]
On tire simultanément 3 cartes d'un jeu de 52 cartes. Quelle est la probabilité d'obtenir exactement 2 Rois ?
\end{exercicebox}

% --- Combinaisons avec Répétition (Étoiles et Bâtons) ---

\begin{exercicebox}[Exercice 11 : Distribution de Bonbons (Étoiles et Bâtons)]
De combien de manières peut-on distribuer 8 bonbons identiques à 3 enfants ? (Certains enfants peuvent ne rien recevoir).
\end{exercicebox}

\begin{exercicebox}[Exercice 12 : Solutions d'Équation (Étoiles et Bâtons)]
Combien y a-t-il de solutions entières non négatives ($x_i \ge 0$) à l'équation $x_1 + x_2 + x_3 + x_4 = 10$ ?
\end{exercicebox}

\begin{exercicebox}[Exercice 13 : Distribution avec Minimum (Étoiles et Bâtons avec Contrainte)]
De combien de manières peut-on distribuer 12 pommes identiques à 4 enfants, si chaque enfant doit recevoir au moins une pomme ?
\end{exercicebox}

% --- Principe d'Inclusion-Exclusion ---

\begin{exercicebox}[Exercice 14 : Divisibilité (Inclusion-Exclusion 2 Ensembles)]
Parmi les entiers de 1 à 100, combien sont divisibles par 2 OU par 3 ?
\end{exercicebox}

\begin{exercicebox}[Exercice 15 : Langues (Inclusion-Exclusion 2 Ensembles)]
Dans un groupe de 50 étudiants, 30 étudient l'anglais, 25 étudient l'espagnol et 10 étudient les deux langues. Combien d'étudiants étudient au moins une de ces deux langues ? Combien n'en étudient aucune ?
\end{exercicebox}

\begin{exercicebox}[Exercice 16 : Divisibilité (Inclusion-Exclusion 3 Ensembles)]
Parmi les entiers de 1 à 100, combien sont divisibles par 2, 3 OU 5 ?
\end{exercicebox}

% --- Problèmes Combinés et Plus Difficiles ---

\begin{exercicebox}[Exercice 17 : Chemins sur un Grillage (Combinaison)]
Sur un grillage, combien y a-t-il de chemins pour aller du point (0,0) au point (4,3) en se déplaçant uniquement vers la droite (D) ou vers le haut (H) ?
\end{exercicebox}

\begin{exercicebox}[Exercice 18 : Probabilité Hypergéométrique]
Une urne contient 7 boules blanches et 5 boules noires. On tire successivement et sans remise 4 boules. Quelle est la probabilité d'obtenir 2 blanches et 2 noires ?
\end{exercicebox}

\begin{exercicebox}[Exercice 19 : Arrangement Circulaire]
De combien de manières 6 personnes peuvent-elles s'asseoir autour d'une table ronde ? (Deux arrangements sont considérés identiques si chaque personne a les mêmes voisins).
\end{exercicebox}

\begin{exercicebox}[Exercice 20 : Problème des Dérangements (Inclusion-Exclusion)]
Quatre lettres sont adressées à quatre personnes différentes, avec les enveloppes correspondantes. On met chaque lettre au hasard dans une enveloppe. Quelle est la probabilité qu'\textit{aucune} lettre ne soit dans la bonne enveloppe ?
\end{exercicebox}



\subsection{Corrections des Exercices}

% --- Corrections : Concepts de Base et Probabilité Naïve ---

\begin{correctionbox}[Correction Exercice 1 : Univers et Événements]
1) L'univers $S$ est l'ensemble de toutes les paires $(r, b)$ où $r$ est le résultat du dé rouge et $b$ celui du dé bleu. $S = \{ (1,1), (1,2), \dots, (1,6), (2,1), \dots, (6,6) \}$. La taille de l'univers est $|S| = 6 \times 6 = 36$.

2) L'événement $A$ (somme égale à 7) est $A = \{ (1,6), (2,5), (3,4), (4,3), (5,2), (6,1) \}$. Il y a $|A|=6$ issues favorables. La probabilité est $P(A) = |A|/|S| = 6/36 = 1/6$.

3) L'événement $B$ (dé rouge montre 3) est $B = \{ (3,1), (3,2), (3,3), (3,4), (3,5), (3,6) \}$. Il y a $|B|=6$ issues favorables. La probabilité est $P(B) = |B|/|S| = 6/36 = 1/6$.

4) L'événement $A \cap B$ est l'ensemble des issues où la somme est 7 ET le dé rouge est 3. La seule issue possible est $(3,4)$. Donc $A \cap B = \{ (3,4) \}$. La probabilité est $P(A \cap B) = |A \cap B|/|S| = 1/36$.
\end{correctionbox}

\begin{correctionbox}[Correction Exercice 2 : Tirage de Cartes (Prob. Naïve)]
Le nombre total d'issues est $|S| = 52$.

a) Il y a 4 Rois. $P(\text{Roi}) = 4/52 = 1/13$.

b) Il y a 26 cartes rouges (13 Cœurs + 13 Carreaux). $P(\text{Rouge}) = 26/52 = 1/2$.

c) Il y a 12 figures (4 Valets + 4 Dames + 4 Rois). $P(\text{Figure}) = 12/52 = 3/13$.

d) Il y a 2 As rouges (As de Cœur, As de Carreau). $P(\text{As Rouge}) = 2/52 = 1/26$.
\end{correctionbox}

\begin{correctionbox}[Correction Exercice 3 : Urne Simple (Prob. Naïve)]
Le nombre total de boules est $5+3+2 = 10$.

a) Il y a 3 boules bleues. $P(\text{Bleue}) = 3/10$.

b) L'événement "ne pas être verte" est le complémentaire de "être verte". Il y a 2 boules vertes, donc $P(\text{Verte}) = 2/10$. La probabilité cherchée est $P(\text{Non Verte}) = 1 - P(\text{Verte}) = 1 - 2/10 = 8/10 = 4/5$. (Alternativement, il y a $5+3=8$ boules non vertes, donc $P=8/10$).
\end{correctionbox}

% --- Corrections : Permutations ---

\begin{correctionbox}[Correction Exercice 4 : Anagrammes (Permutation Simple)]
Le mot "MATHS" a 5 lettres distinctes. Le nombre d'anagrammes est le nombre de permutations de ces 5 lettres, soit $5! = 5 \times 4 \times 3 \times 2 \times 1 = 120$.
\end{correctionbox}

\begin{correctionbox}[Correction Exercice 5 : Course (Arrangement)]
On cherche le nombre de façons d'ordonner 3 athlètes parmi 10. C'est un arrangement (permutation de $k$ parmi $n$) :
$P(10, 3) = \frac{10!}{(10-3)!} = \frac{10!}{7!} = 10 \times 9 \times 8 = 720$.
Il y a 720 podiums possibles.
\end{correctionbox}

\begin{correctionbox}[Correction Exercice 6 : Anagrammes (Permutation avec Répétition)]
Le mot "PROBABILITE" a 11 lettres. Les répétitions sont : B (2 fois), I (2 fois). Les autres lettres (P, R, O, A, L, T, E) apparaissent une fois.
Le nombre d'anagrammes distinctes est :
$$ \frac{11!}{2! \times 2!} = \frac{39,916,800}{2 \times 2} = \frac{39,916,800}{4} = 9,979,200 $$
\end{correctionbox}

% --- Corrections : Combinaisons ---

\begin{correctionbox}[Correction Exercice 7 : Choix d'un Comité (Combinaison)]
L'ordre ne compte pas, c'est donc une combinaison de 4 étudiants parmi 15 :
$$ \binom{15}{4} = \frac{15!}{4!(15-4)!} = \frac{15!}{4!11!} = \frac{15 \times 14 \times 13 \times 12}{4 \times 3 \times 2 \times 1} = 15 \times 7 \times 13 \times 1 = 1365 $$
Il y a 1365 comités possibles.
\end{correctionbox}

\begin{correctionbox}[Correction Exercice 8 : Mains de Poker (Combinaison)]
On choisit 5 cartes parmi 52, sans ordre. C'est une combinaison :
$$ \binom{52}{5} = \frac{52!}{5!(52-5)!} = \frac{52!}{5!47!} = \frac{52 \times 51 \times 50 \times 49 \times 48}{5 \times 4 \times 3 \times 2 \times 1} = 2,598,960 $$
Il y a 2,598,960 mains de poker possibles.
\end{correctionbox}

\begin{correctionbox}[Correction Exercice 9 : Comité Mixte (Combinaison, Principe Multiplicatif)]
Il faut choisir 2 hommes parmi 6 ET 1 femme parmi 4. On multiplie les possibilités pour chaque choix :
Nombre de façons = (choix des hommes) $\times$ (choix des femmes)
$$ = \binom{6}{2} \times \binom{4}{1} = \frac{6 \times 5}{2 \times 1} \times \frac{4}{1} = 15 \times 4 = 60 $$
Il y a 60 comités possibles.
\end{correctionbox}

\begin{correctionbox}[Correction Exercice 10 : Probabilité avec Combinaisons]
L'univers $S$ est l'ensemble de toutes les mains de 3 cartes. $|S| = \binom{52}{3}$.
L'événement $A$ est "obtenir exactement 2 Rois". Pour cela, il faut choisir 2 Rois parmi les 4 Rois ET 1 carte qui n'est pas un Roi parmi les 48 autres cartes.
$|A| = \binom{4}{2} \times \binom{48}{1}$.
La probabilité est $P(A) = \frac{|A|}{|S|} = \frac{\binom{4}{2} \binom{48}{1}}{\binom{52}{3}}$.
$$ P(A) = \frac{\frac{4 \times 3}{2 \times 1} \times 48}{\frac{52 \times 51 \times 50}{3 \times 2 \times 1}} = \frac{6 \times 48}{22100} = \frac{288}{22100} \approx 0.013 $$
\end{correctionbox}

% --- Corrections : Combinaisons avec Répétition (Étoiles et Bâtons) ---

\begin{correctionbox}[Correction Exercice 11 : Distribution de Bonbons (Étoiles et Bâtons)]
C'est un problème de distribution de $k=8$ objets identiques (bonbons) dans $n=3$ boîtes distinctes (enfants). On utilise la formule $\binom{n+k-1}{k}$.
Nombre de manières = $\binom{3+8-1}{8} = \binom{10}{8} = \binom{10}{2} = \frac{10 \times 9}{2 \times 1} = 45$.
\end{correctionbox}

\begin{correctionbox}[Correction Exercice 12 : Solutions d'Équation (Étoiles et Bâtons)]
Cela revient à distribuer $k=10$ unités identiques dans $n=4$ variables distinctes.
Nombre de solutions = $\binom{n+k-1}{k} = \binom{4+10-1}{10} = \binom{13}{10} = \binom{13}{3} = \frac{13 \times 12 \times 11}{3 \times 2 \times 1} = 286$.
\end{correctionbox}

\begin{correctionbox}[Correction Exercice 13 : Distribution avec Minimum (Étoiles et Bâtons avec Contrainte)]
On doit distribuer $k=12$ pommes à $n=4$ enfants, avec $x_i \ge 1$.
On commence par donner une pomme à chaque enfant. Il reste $12 - 4 = 8$ pommes à distribuer sans contrainte (les $x'_i$ peuvent être nuls).
Le problème devient : distribuer $k'=8$ pommes à $n=4$ enfants.
Nombre de manières = $\binom{n+k'-1}{k'} = \binom{4+8-1}{8} = \binom{11}{8} = \binom{11}{3} = \frac{11 \times 10 \times 9}{3 \times 2 \times 1} = 165$.
\end{correctionbox}

% --- Corrections : Principe d'Inclusion-Exclusion ---

\begin{correctionbox}[Correction Exercice 14 : Divisibilité (Inclusion-Exclusion 2 Ensembles)]
Soit $A$ l'ensemble des entiers $\le 100$ divisibles par 2, et $B$ l'ensemble des entiers $\le 100$ divisibles par 3. On cherche $|A \cup B|$.
$|A| = \lfloor 100/2 \rfloor = 50$.
$|B| = \lfloor 100/3 \rfloor = 33$.
$|A \cap B|$ = ensemble des entiers divisibles par $2 \times 3 = 6$. $|A \cap B| = \lfloor 100/6 \rfloor = 16$.
Par inclusion-exclusion : $|A \cup B| = |A| + |B| - |A \cap B| = 50 + 33 - 16 = 67$.
\end{correctionbox}

\begin{correctionbox}[Correction Exercice 15 : Langues (Inclusion-Exclusion 2 Ensembles)]
Soit $E$ l'ensemble des étudiants étudiant l'anglais, $S$ l'ensemble de ceux étudiant l'espagnol.
$|E| = 30$, $|S| = 25$, $|E \cap S| = 10$.
Nombre d'étudiants étudiant au moins une langue : $|E \cup S| = |E| + |S| - |E \cap S| = 30 + 25 - 10 = 45$.
Nombre total d'étudiants = 50.
Nombre d'étudiants n'étudiant aucune de ces langues = Total - $|E \cup S| = 50 - 45 = 5$.
\end{correctionbox}

\begin{correctionbox}[Correction Exercice 16 : Divisibilité (Inclusion-Exclusion 3 Ensembles)]
Soit $A_2, A_3, A_5$ les ensembles des entiers $\le 100$ divisibles respectivement par 2, 3, 5. On cherche $|A_2 \cup A_3 \cup A_5|$.
$|A_2|=50$, $|A_3|=33$, $|A_5|=20$.
$|A_2 \cap A_3| = |A_6| = \lfloor 100/6 \rfloor = 16$.
$|A_2 \cap A_5| = |A_{10}| = \lfloor 100/10 \rfloor = 10$.
$|A_3 \cap A_5| = |A_{15}| = \lfloor 100/15 \rfloor = 6$.
$|A_2 \cap A_3 \cap A_5| = |A_{30}| = \lfloor 100/30 \rfloor = 3$.
Par inclusion-exclusion :
$|A_2 \cup A_3 \cup A_5| = (|A_2|+|A_3|+|A_5|) - (|A_2 \cap A_3|+|A_2 \cap A_5|+|A_3 \cap A_5|) + |A_2 \cap A_3 \cap A_5|$
$= (50+33+20) - (16+10+6) + 3 = 103 - 32 + 3 = 74$.
\end{correctionbox}

% --- Corrections : Problèmes Combinés et Plus Difficiles ---

\begin{correctionbox}[Correction Exercice 17 : Chemins sur un Grillage (Combinaison)]
Pour aller de (0,0) à (4,3), il faut faire un total de $4+3=7$ déplacements. Parmi ces 7 déplacements, il faut choisir les 4 moments où l'on va à droite (les 3 autres seront obligatoirement vers le haut), ou choisir les 3 moments où l'on va vers le haut.
Le nombre de chemins est $\binom{7}{4} = \binom{7}{3} = \frac{7 \times 6 \times 5}{3 \times 2 \times 1} = 35$.
\end{correctionbox}

\begin{correctionbox}[Correction Exercice 18 : Probabilité Hypergéométrique]
C'est un tirage sans remise. On peut utiliser la loi hypergéométrique ou le dénombrement.
Population totale = $7+5=12$ boules. On en tire $m=4$.
On veut $k=2$ blanches (parmi $w=7$) et $m-k=2$ noires (parmi $b=5$).
Probabilité = $\frac{\binom{w}{k} \binom{b}{m-k}}{\binom{w+b}{m}} = \frac{\binom{7}{2} \binom{5}{2}}{\binom{12}{4}}$.
$$ P = \frac{(\frac{7 \times 6}{2}) \times (\frac{5 \times 4}{2})}{(\frac{12 \times 11 \times 10 \times 9}{4 \times 3 \times 2 \times 1})} = \frac{21 \times 10}{495} = \frac{210}{495} = \frac{14}{33} \approx 0.424 $$
\end{correctionbox}

\begin{correctionbox}[Correction Exercice 19 : Arrangement Circulaire]
Pour $n$ objets distincts, le nombre d'arrangements circulaires est $(n-1)!$.
Ici, $n=6$. Le nombre de manières est $(6-1)! = 5! = 120$.
L'idée est de fixer une personne, puis d'arranger les 5 autres par rapport à elle.
\end{correctionbox}

\begin{correctionbox}[Correction Exercice 20 : Problème des Dérangements (Inclusion-Exclusion)]
On cherche le nombre de dérangements de 4 éléments, noté $D_4$ ou $!4$. La probabilité sera $D_4 / 4!$.
La formule générale des dérangements (obtenue par inclusion-exclusion) est $D_n = n! \sum_{i=0}^n \frac{(-1)^i}{i!}$.
Pour $n=4$:
$D_4 = 4! (1/0! - 1/1! + 1/2! - 1/3! + 1/4!)$
$D_4 = 24 (1 - 1 + 1/2 - 1/6 + 1/24)$
$D_4 = 24 (1/2 - 1/6 + 1/24) = 24 (12/24 - 4/24 + 1/24) = 24 (9/24) = 9$.
Il y a 9 dérangements possibles sur un total de $4! = 24$ permutations.
La probabilité est $P(\text{aucun match}) = D_4 / 4! = 9/24 = 3/8 = 0.375$.
\end{correctionbox}


\subsection{Exercices}

Cette série d'exercices vise à renforcer votre compréhension des concepts fondamentaux du dénombrement et de la probabilité naïve. La difficulté augmente progressivement.

%  Concepts de Base et Probabilité Naïve 

\begin{exercicebox}[Exercice 1 : Univers et Événements]
On lance deux dés à 6 faces, un rouge et un bleu.
\begin{enumerate}
    \item Décrivez l'univers $S$ de cette expérience. Quelle est sa taille $|S|$ ?
    \item Soit $A$ l'événement "la somme des dés est égale à 7". Listez les issues appartenant à $A$. Calculez $P(A)$.
    \item Soit $B$ l'événement "le dé rouge montre un 3". Listez les issues appartenant à $B$. Calculez $P(B)$.
    \item Décrivez l'événement $A \cap B$ et calculez sa probabilité.
\end{enumerate}
\end{exercicebox}

\begin{exercicebox}[Exercice 2 : Tirage de Cartes (Prob. Naïve)]
On tire une carte au hasard d'un jeu standard de 52 cartes.
\begin{enumerate}
    \item Quelle est la probabilité de tirer un Roi ?
    \item Quelle est la probabilité de tirer une carte rouge (Cœur ou Carreau) ?
    \item Quelle est la probabilité de tirer une figure (Valet, Dame, Roi) ?
    \item Quelle est la probabilité de tirer un As rouge ?
\end{enumerate}
\end{exercicebox}

\begin{exercicebox}[Exercice 3 : Urne Simple (Prob. Naïve)]
Une urne contient 5 boules rouges, 3 boules bleues et 2 boules vertes. On tire une boule au hasard.
\begin{enumerate}
    \item Quelle est la probabilité qu'elle soit bleue ?
    \item Quelle est la probabilité qu'elle ne soit pas verte ?
\end{enumerate}
\end{exercicebox}

%  Permutations 

\begin{exercicebox}[Exercice 4 : Anagrammes (Permutation Simple)]
Combien d'anagrammes distinctes peut-on former avec les lettres du mot "MATHS" ?
\end{exercicebox}

\begin{exercicebox}[Exercice 5 : Course (Arrangement)]
Dix athlètes participent à une course. Combien y a-t-il de classements possibles pour les 3 premières places (médaille d'or, d'argent, de bronze) ?
\end{exercicebox}

\begin{exercicebox}[Exercice 6 : Anagrammes (Permutation avec Répétition)]
Combien d'anagrammes distinctes peut-on former avec les lettres du mot "PROBABILITE" ?
\end{exercicebox}

%  Combinaisons 

\begin{exercicebox}[Exercice 7 : Choix d'un Comité (Combinaison)]
Une classe compte 15 étudiants. De combien de manières peut-on choisir un comité de 4 étudiants ?
\end{exercicebox}

\begin{exercicebox}[Exercice 8 : Mains de Poker (Combinaison)]
Dans un jeu de 52 cartes, combien de "mains" de 5 cartes différentes peut-on former ?
\end{exercicebox}

\begin{exercicebox}[Exercice 9 : Comité Mixte (Combinaison)]
À partir d'un groupe de 6 hommes et 4 femmes, combien de comités de 3 personnes peut-on former contenant exactement 2 hommes et 1 femme ?
\end{exercicebox}

\begin{exercicebox}[Exercice 10 : Probabilité avec Combinaisons]
On tire simultanément 3 cartes d'un jeu de 52 cartes. Quelle est la probabilité d'obtenir exactement 2 Rois ?
\end{exercicebox}

%  Combinaisons avec Répétition (Étoiles et Bâtons) 

\begin{exercicebox}[Exercice 11 : Distribution de Bonbons (Étoiles et Bâtons)]
De combien de manières peut-on distribuer 8 bonbons identiques à 3 enfants ? (Certains enfants peuvent ne rien recevoir).
\end{exercicebox}

\begin{exercicebox}[Exercice 12 : Solutions d'Équation (Étoiles et Bâtons)]
Combien y a-t-il de solutions entières non négatives ($x_i \ge 0$) à l'équation $x_1 + x_2 + x_3 + x_4 = 10$ ?
\end{exercicebox}

\begin{exercicebox}[Exercice 13 : Distribution avec Minimum (Étoiles et Bâtons avec Contrainte)]
De combien de manières peut-on distribuer 12 pommes identiques à 4 enfants, si chaque enfant doit recevoir au moins une pomme ?
\end{exercicebox}

%  Principe d'Inclusion-Exclusion 

\begin{exercicebox}[Exercice 14 : Divisibilité (Inclusion-Exclusion 2 Ensembles)]
Parmi les entiers de 1 à 100, combien sont divisibles par 2 OU par 3 ?
\end{exercicebox}

\begin{exercicebox}[Exercice 15 : Langues (Inclusion-Exclusion 2 Ensembles)]
Dans un groupe de 50 étudiants, 30 étudient l'anglais, 25 étudient l'espagnol et 10 étudient les deux langues. Combien d'étudiants étudient au moins une de ces deux langues ? Combien n'en étudient aucune ?
\end{exercicebox}

\begin{exercicebox}[Exercice 16 : Divisibilité (Inclusion-Exclusion 3 Ensembles)]
Parmi les entiers de 1 à 100, combien sont divisibles par 2, 3 OU 5 ?
\end{exercicebox}

%  Problèmes Combinés et Plus Difficiles 

\begin{exercicebox}[Exercice 17 : Chemins sur un Grillage (Combinaison)]
Sur un grillage, combien y a-t-il de chemins pour aller du point (0,0) au point (4,3) en se déplaçant uniquement vers la droite (D) ou vers le haut (H) ?
\end{exercicebox}

\begin{exercicebox}[Exercice 18 : Probabilité Hypergéométrique]
Une urne contient 7 boules blanches et 5 boules noires. On tire successivement et sans remise 4 boules. Quelle est la probabilité d'obtenir 2 blanches et 2 noires ?
\end{exercicebox}

\begin{exercicebox}[Exercice 19 : Arrangement Circulaire]
De combien de manières 6 personnes peuvent-elles s'asseoir autour d'une table ronde ? (Deux arrangements sont considérés identiques si chaque personne a les mêmes voisins).
\end{exercicebox}

\begin{exercicebox}[Exercice 20 : Problème des Dérangements (Inclusion-Exclusion)]
Quatre lettres sont adressées à quatre personnes différentes, avec les enveloppes correspondantes. On met chaque lettre au hasard dans une enveloppe. Quelle est la probabilité qu'\textit{aucune} lettre ne soit dans la bonne enveloppe ?
\end{exercicebox}



\subsection{Corrections des Exercices}

%  Corrections : Concepts de Base et Probabilité Naïve 

\begin{correctionbox}[Correction Exercice 1 : Univers et Événements]
1) L'univers $S$ est l'ensemble de toutes les paires $(r, b)$ où $r$ est le résultat du dé rouge et $b$ celui du dé bleu. $S = \{ (1,1), (1,2), \dots, (1,6), (2,1), \dots, (6,6) \}$. La taille de l'univers est $|S| = 6 \times 6 = 36$.

2) L'événement $A$ (somme égale à 7) est $A = \{ (1,6), (2,5), (3,4), (4,3), (5,2), (6,1) \}$. Il y a $|A|=6$ issues favorables. La probabilité est $P(A) = |A|/|S| = 6/36 = 1/6$.

3) L'événement $B$ (dé rouge montre 3) est $B = \{ (3,1), (3,2), (3,3), (3,4), (3,5), (3,6) \}$. Il y a $|B|=6$ issues favorables. La probabilité est $P(B) = |B|/|S| = 6/36 = 1/6$.

4) L'événement $A \cap B$ est l'ensemble des issues où la somme est 7 ET le dé rouge est 3. La seule issue possible est $(3,4)$. Donc $A \cap B = \{ (3,4) \}$. La probabilité est $P(A \cap B) = |A \cap B|/|S| = 1/36$.
\end{correctionbox}

\begin{correctionbox}[Correction Exercice 2 : Tirage de Cartes (Prob. Naïve)]
Le nombre total d'issues est $|S| = 52$.

a) Il y a 4 Rois. $P(\text{Roi}) = 4/52 = 1/13$.

b) Il y a 26 cartes rouges (13 Cœurs + 13 Carreaux). $P(\text{Rouge}) = 26/52 = 1/2$.

c) Il y a 12 figures (4 Valets + 4 Dames + 4 Rois). $P(\text{Figure}) = 12/52 = 3/13$.

d) Il y a 2 As rouges (As de Cœur, As de Carreau). $P(\text{As Rouge}) = 2/52 = 1/26$.
\end{correctionbox}

\begin{correctionbox}[Correction Exercice 3 : Urne Simple (Prob. Naïve)]
Le nombre total de boules est $5+3+2 = 10$.

a) Il y a 3 boules bleues. $P(\text{Bleue}) = 3/10$.

b) L'événement "ne pas être verte" est le complémentaire de "être verte". Il y a 2 boules vertes, donc $P(\text{Verte}) = 2/10$. La probabilité cherchée est $P(\text{Non Verte}) = 1 - P(\text{Verte}) = 1 - 2/10 = 8/10 = 4/5$. (Alternativement, il y a $5+3=8$ boules non vertes, donc $P=8/10$).
\end{correctionbox}

%  Corrections : Permutations 

\begin{correctionbox}[Correction Exercice 4 : Anagrammes (Permutation Simple)]
Le mot "MATHS" a 5 lettres distinctes. Le nombre d'anagrammes est le nombre de permutations de ces 5 lettres, soit $5! = 5 \times 4 \times 3 \times 2 \times 1 = 120$.
\end{correctionbox}

\begin{correctionbox}[Correction Exercice 5 : Course (Arrangement)]
On cherche le nombre de façons d'ordonner 3 athlètes parmi 10. C'est un arrangement (permutation de $k$ parmi $n$) :
$P(10, 3) = \frac{10!}{(10-3)!} = \frac{10!}{7!} = 10 \times 9 \times 8 = 720$.
Il y a 720 podiums possibles.
\end{correctionbox}

\begin{correctionbox}[Correction Exercice 6 : Anagrammes (Permutation avec Répétition)]
Le mot "PROBABILITE" a 11 lettres. Les répétitions sont : B (2 fois), I (2 fois). Les autres lettres (P, R, O, A, L, T, E) apparaissent une fois.
Le nombre d'anagrammes distinctes est :
$$ \frac{11!}{2! \times 2!} = \frac{39,916,800}{2 \times 2} = \frac{39,916,800}{4} = 9,979,200 $$
\end{correctionbox}

%  Corrections : Combinaisons 

\begin{correctionbox}[Correction Exercice 7 : Choix d'un Comité (Combinaison)]
L'ordre ne compte pas, c'est donc une combinaison de 4 étudiants parmi 15 :
$$ \binom{15}{4} = \frac{15!}{4!(15-4)!} = \frac{15!}{4!11!} = \frac{15 \times 14 \times 13 \times 12}{4 \times 3 \times 2 \times 1} = 15 \times 7 \times 13 \times 1 = 1365 $$
Il y a 1365 comités possibles.
\end{correctionbox}

\begin{correctionbox}[Correction Exercice 8 : Mains de Poker (Combinaison)]
On choisit 5 cartes parmi 52, sans ordre. C'est une combinaison :
$$ \binom{52}{5} = \frac{52!}{5!(52-5)!} = \frac{52!}{5!47!} = \frac{52 \times 51 \times 50 \times 49 \times 48}{5 \times 4 \times 3 \times 2 \times 1} = 2,598,960 $$
Il y a 2,598,960 mains de poker possibles.
\end{correctionbox}

\begin{correctionbox}[Correction Exercice 9 : Comité Mixte (Combinaison)]
Il faut choisir 2 hommes parmi 6 ET 1 femme parmi 4. On multiplie les possibilités pour chaque choix :
Nombre de façons = (choix des hommes) $\times$ (choix des femmes)
$$ = \binom{6}{2} \times \binom{4}{1} = \frac{6 \times 5}{2 \times 1} \times \frac{4}{1} = 15 \times 4 = 60 $$
Il y a 60 comités possibles.
\end{correctionbox}

\begin{correctionbox}[Correction Exercice 10 : Probabilité avec Combinaisons]
L'univers $S$ est l'ensemble de toutes les mains de 3 cartes. $|S| = \binom{52}{3}$.
L'événement $A$ est "obtenir exactement 2 Rois". Pour cela, il faut choisir 2 Rois parmi les 4 Rois ET 1 carte qui n'est pas un Roi parmi les 48 autres cartes.
$|A| = \binom{4}{2} \times \binom{48}{1}$.
La probabilité est $P(A) = \frac{|A|}{|S|} = \frac{\binom{4}{2} \binom{48}{1}}{\binom{52}{3}}$.
$$ P(A) = \frac{\frac{4 \times 3}{2 \times 1} \times 48}{\frac{52 \times 51 \times 50}{3 \times 2 \times 1}} = \frac{6 \times 48}{22100} = \frac{288}{22100} \approx 0.013 $$
\end{correctionbox}

%  Corrections : Combinaisons avec Répétition (Étoiles et Bâtons) 

\begin{correctionbox}[Correction Exercice 11 : Distribution de Bonbons (Étoiles et Bâtons)]
C'est un problème de distribution de $k=8$ objets identiques (bonbons) dans $n=3$ boîtes distinctes (enfants). On utilise la formule $\binom{n+k-1}{k}$.
Nombre de manières = $\binom{3+8-1}{8} = \binom{10}{8} = \binom{10}{2} = \frac{10 \times 9}{2 \times 1} = 45$.
\end{correctionbox}

\begin{correctionbox}[Correction Exercice 12 : Solutions d'Équation (Étoiles et Bâtons)]
Cela revient à distribuer $k=10$ unités identiques dans $n=4$ variables distinctes.
Nombre de solutions = $\binom{n+k-1}{k} = \binom{4+10-1}{10} = \binom{13}{10} = \binom{13}{3} = \frac{13 \times 12 \times 11}{3 \times 2 \times 1} = 286$.
\end{correctionbox}

\begin{correctionbox}[Correction Exercice 13 : Distribution avec Minimum (Étoiles et Bâtons avec Contrainte)]
On doit distribuer $k=12$ pommes à $n=4$ enfants, avec $x_i \ge 1$.
On commence par donner une pomme à chaque enfant. Il reste $12 - 4 = 8$ pommes à distribuer sans contrainte (les $x'_i$ peuvent être nuls).
Le problème devient : distribuer $k'=8$ pommes à $n=4$ enfants.
Nombre de manières = $\binom{n+k'-1}{k'} = \binom{4+8-1}{8} = \binom{11}{8} = \binom{11}{3} = \frac{11 \times 10 \times 9}{3 \times 2 \times 1} = 165$.
\end{correctionbox}

%  Corrections : Principe d'Inclusion-Exclusion 

\begin{correctionbox}[Correction Exercice 14 : Divisibilité (Inclusion-Exclusion 2 Ensembles)]
Soit $A$ l'ensemble des entiers $\le 100$ divisibles par 2, et $B$ l'ensemble des entiers $\le 100$ divisibles par 3. On cherche $|A \cup B|$.
$|A| = \lfloor 100/2 \rfloor = 50$.
$|B| = \lfloor 100/3 \rfloor = 33$.
$|A \cap B|$ = ensemble des entiers divisibles par $2 \times 3 = 6$. $|A \cap B| = \lfloor 100/6 \rfloor = 16$.
Par inclusion-exclusion : $|A \cup B| = |A| + |B| - |A \cap B| = 50 + 33 - 16 = 67$.
\end{correctionbox}

\begin{correctionbox}[Correction Exercice 15 : Langues (Inclusion-Exclusion 2 Ensembles)]
Soit $E$ l'ensemble des étudiants étudiant l'anglais, $S$ l'ensemble de ceux étudiant l'espagnol.
$|E| = 30$, $|S| = 25$, $|E \cap S| = 10$.
Nombre d'étudiants étudiant au moins une langue : $|E \cup S| = |E| + |S| - |E \cap S| = 30 + 25 - 10 = 45$.
Nombre total d'étudiants = 50.
Nombre d'étudiants n'étudiant aucune de ces langues = Total - $|E \cup S| = 50 - 45 = 5$.
\end{correctionbox}

\begin{correctionbox}[Correction Exercice 16 : Divisibilité (Inclusion-Exclusion 3 Ensembles)]
Soit $A_2, A_3, A_5$ les ensembles des entiers $\le 100$ divisibles respectivement par 2, 3, 5. On cherche $|A_2 \cup A_3 \cup A_5|$.
$|A_2|=50$, $|A_3|=33$, $|A_5|=20$.
$|A_2 \cap A_3| = |A_6| = \lfloor 100/6 \rfloor = 16$.
$|A_2 \cap A_5| = |A_{10}| = \lfloor 100/10 \rfloor = 10$.
$|A_3 \cap A_5| = |A_{15}| = \lfloor 100/15 \rfloor = 6$.
$|A_2 \cap A_3 \cap A_5| = |A_{30}| = \lfloor 100/30 \rfloor = 3$.
Par inclusion-exclusion :
$|A_2 \cup A_3 \cup A_5| = (|A_2|+|A_3|+|A_5|) - (|A_2 \cap A_3|+|A_2 \cap A_5|+|A_3 \cap A_5|) + |A_2 \cap A_3 \cap A_5|$
$= (50+33+20) - (16+10+6) + 3 = 103 - 32 + 3 = 74$.
\end{correctionbox}

%  Corrections : Problèmes Combinés et Plus Difficiles 

\begin{correctionbox}[Correction Exercice 17 : Chemins sur un Grillage (Combinaison)]
Pour aller de (0,0) à (4,3), il faut faire un total de $4+3=7$ déplacements. Parmi ces 7 déplacements, il faut choisir les 4 moments où l'on va à droite (les 3 autres seront obligatoirement vers le haut), ou choisir les 3 moments où l'on va vers le haut.
Le nombre de chemins est $\binom{7}{4} = \binom{7}{3} = \frac{7 \times 6 \times 5}{3 \times 2 \times 1} = 35$.
\end{correctionbox}

\begin{correctionbox}[Correction Exercice 18 : Probabilité Hypergéométrique]
C'est un tirage sans remise. On peut utiliser la loi hypergéométrique ou le dénombrement.
Population totale = $7+5=12$ boules. On en tire $m=4$.
On veut $k=2$ blanches (parmi $w=7$) et $m-k=2$ noires (parmi $b=5$).
Probabilité = $\frac{\binom{w}{k} \binom{b}{m-k}}{\binom{w+b}{m}} = \frac{\binom{7}{2} \binom{5}{2}}{\binom{12}{4}}$.
$$ P = \frac{(\frac{7 \times 6}{2}) \times (\frac{5 \times 4}{2})}{(\frac{12 \times 11 \times 10 \times 9}{4 \times 3 \times 2 \times 1})} = \frac{21 \times 10}{495} = \frac{210}{495} = \frac{14}{33} \approx 0.424 $$
\end{correctionbox}

\begin{correctionbox}[Correction Exercice 19 : Arrangement Circulaire]
Pour $n$ objets distincts, le nombre d'arrangements circulaires est $(n-1)!$.
Ici, $n=6$. Le nombre de manières est $(6-1)! = 5! = 120$.
L'idée est de fixer une personne, puis d'arranger les 5 autres par rapport à elle.
\end{correctionbox}

\begin{correctionbox}[Correction Exercice 20 : Problème des Dérangements (Inclusion-Exclusion)]
On cherche le nombre de dérangements de 4 éléments, noté $D_4$ ou $!4$. La probabilité sera $D_4 / 4!$.
La formule générale des dérangements (obtenue par inclusion-exclusion) est $D_n = n! \sum_{i=0}^n \frac{(-1)^i}{i!}$.
Pour $n=4$:
$D_4 = 4! (1/0! - 1/1! + 1/2! - 1/3! + 1/4!)$
$D_4 = 24 (1 - 1 + 1/2 - 1/6 + 1/24)$
$D_4 = 24 (1/2 - 1/6 + 1/24) = 24 (12/24 - 4/24 + 1/24) = 24 (9/24) = 9$.
Il y a 9 dérangements possibles sur un total de $4! = 24$ permutations.
La probabilité est $P(\text{aucun match}) = D_4 / 4! = 9/24 = 3/8 = 0.375$.
\end{correctionbox}

\subsection{Exercices Python}

Les exercices suivants appliquent les concepts de dénombrement et de probabilité au célèbre jeu de données "Titanic". Ce dataset, chargé via la bibliothèque \texttt{seaborn}, contient des informations démographiques et de voyage sur les passagers du navire.

Le bloc de code ci-dessous initialise notre environnement en chargeant les données dans un DataFrame Pandas \texttt{df}. Pour garantir la consistance de nos calculs, nous définirons notre univers $S$ comme l'ensemble des passagers pour lesquels l'âge est connu (en supprimant les lignes avec un âge manquant).

\begin{codecell}
import pandas as pd
import seaborn as sns
import math

# Charger le dataset Titanic
df = sns.load_dataset("titanic")

# On retire les lignes ou l'age est inconnu pour simplifier les calculs
# C'est notre Univers S.
df = df.dropna(subset=["age"]) 

\end{codecell}

\begin{exercicebox}[Exercice 1 : Probabilité Naïve (Filtre multiple)]
Quelle est la probabilité qu'un passager, sélectionné au hasard dans l'univers \texttt{df}, soit un homme de plus de 40 ans ET voyageant en troisième classe ?

\textbf{Votre tâche :}
\begin{enumerate}
    \item Trouver $|S|$, la taille totale de l'univers \texttt{df}.
    \item Trouver $|A|$, le nombre de passagers remplissant les trois conditions (\texttt{sex == 'male'}, \texttt{age > 40}, \texttt{pclass == 3}).
    \item Calculer $P(A) = |A| / |S|$.
\end{enumerate}
\end{exercicebox}

\begin{exercicebox}[Exercice 2 : Dénombrement par Combinaisons ($\binom{n}{k}$)]
Pour une enquête de satisfaction, on veut créer un groupe de discussion (un "comité") composé de 5 personnes. Ces 5 personnes doivent être choisies exclusivement parmi les passagers ayant embarqué à Southampton (\texttt{embark\_town == 'Southampton'}).

Combien de comités uniques de 5 personnes est-il possible de former ?

\textbf{Votre tâche :}
\begin{enumerate}
    \item Trouver $n$, le nombre de passagers ayant embarqué à Southampton.
    \item Définir $k=5$.
    \item Calculer $\binom{n}{k}$ (par ex., avec \texttt{math.comb}).
\end{enumerate}
\end{exercicebox}

\begin{exercicebox}[Exercice 3 : Dénombrement par Permutations]
Lors d'un exercice de sécurité, on demande à 4 enfants (passagers de 12 ans ou moins) de s'aligner pour une photo de communication.

En supposant que l'on choisisse 4 enfants au hasard parmi tous les enfants du navire, et que l'ordre dans lequel ils sont alignés pour la photo est important, combien d'alignements différents sont possibles ?

\textbf{Votre tâche :}
\begin{enumerate}
    \item Trouver $n$, le nombre total d'enfants (âge $\le 12$) à bord.
    \item Définir $k=4$.
    \item Calculer $P(n, k)$ (par ex., avec \texttt{math.perm}).
\end{enumerate}
\end{exercicebox}

\begin{exercicebox}[Exercice 4 : Principe d'Inclusion-Exclusion]
Quelle est la probabilité qu'un passager sélectionné au hasard soit \textbf{soit un survivant} (ensemble $A$), \textbf{soit un passager de première classe} (ensemble $B$) (ou les deux) ?

\textbf{Votre tâche :}
\begin{enumerate}
    \item Trouver $|S|$.
    \item Trouver $|A|$ (nombre de survivants).
    \item Trouver $|B|$ (nombre de passagers en 1ère classe).
    \item Trouver $|A \cap B|$ (survivants de 1ère classe).
    \item Appliquer la formule : $P(A \cup B) = P(A) + P(B) - P(A \cap B)$.
\end{enumerate}
\end{exercicebox}

\begin{exercicebox}[Exercice 5 : Probabilité (Tirage sans remise)]
On sélectionne au hasard un échantillon de 10 passagers de l'univers \texttt{df}.

Quelle est la probabilité que cet échantillon contienne exactement \textbf{4 survivants} et \textbf{6 non-survivants} ?

\textbf{Votre tâche :}
\begin{enumerate}
    \item Trouver $N = |S|$, le nombre total de passagers.
    \item Trouver $m$, le nombre total de survivants dans $S$.
    \item Trouver $p$, le nombre total de non-survivants dans $S$.
    \item Calculer le dénominateur : $\binom{N}{10}$ (façons de choisir 10 passagers).
    \item Calculer le numérateur : $\binom{m}{4} \times \binom{p}{6}$ (façons de choisir 4 survivants ET 6 non-survivants).
    \item Calculer la probabilité (numérateur / dénominateur).
\end{enumerate}
\end{exercicebox}

\begin{exercicebox}[Exercice 6 : Probabilité Complémentaire]
On sélectionne au hasard un groupe de 5 passagers. Quelle est la probabilité que ce groupe contienne \textbf{au moins un} passager voyageant seul (\texttt{alone == True}) ?

\textbf{Votre tâche :}
\begin{enumerate}
    \item Calculer $P(E^c)$, la probabilité de l'événement complémentaire "aucun passager ne voyage seul".
    \item Trouver $N = |S|$, le nombre total de passagers.
    \item Trouver $n_{\text{non-seul}}$, le nombre de passagers qui ne voyagent *pas* seuls.
    \item Dénominateur $D = \binom{N}{5}$ (façons de choisir 5 passagers).
    \item Numérateur $N_c = \binom{n_{\text{non-seul}}}{5}$ (façons de choisir 5 passagers non-seuls).
    \item $P(E^c) = N_c / D$.
    \item Calculer le résultat final : $P(E) = 1 - P(E^c)$.
\end{enumerate}
\end{exercicebox}

\begin{exercicebox}[Exercice 7 : Probabilité et Analyse de Données]
Calculez la probabilité qu'un passager, choisi au hasard, ait payé un tarif (\texttt{fare}) supérieur au \textbf{tarif moyen} de l'ensemble du navire (\texttt{df}).

\textbf{Votre tâche :}
\begin{enumerate}
    \item Calculer le tarif moyen de tous les passagers dans \texttt{df}.
    \item Trouver $|S|$, la taille totale de l'univers \texttt{df}.
    \item Trouver $|A|$, le nombre de passagers dont le \texttt{fare} est strictement supérieur à ce tarif moyen.
    \item Calculer $P(A) = |A| / |S|$.
\end{enumerate}
\end{exercicebox}
\section{Probabilités et Dénombrement}

\subsection{Concepts fondamentaux}

\begin{intuitionbox}[Nécessité d'un Cadre Formel]
Avant de calculer des probabilités, il est crucial de définir les règles du jeu :
\newline
\textbf{Qu'est-ce qui peut arriver ?}
\newline
On définit l'ensemble de tous les résultats possibles de l'expérience.
\newline
\textbf{À quoi s'intéresse-t-on ?} 
\newline
On identifie les sous-ensembles de résultats spécifiques qui nous intéressent.
\newline
Ces deux idées nous conduisent aux notions d'Univers et d'Événement, qui sont les piliers de toute théorie des probabilités.
\end{intuitionbox}

\begin{definitionbox}[Concepts Fondamentaux]
\textbf{Univers (ou Espace Échantillon), $S$ :} 
\newline
L'ensemble de tous les résultats possibles d'une expérience aléatoire.
\newline
\textbf{Événement, $A$ :} 
\newline
Un sous-ensemble de l'univers ($A \subseteq S$). C'est un ensemble de résultats auxquels on s'intéresse.
\end{definitionbox}

\begin{examplebox}[Univers et Événement]
Pour l'expérience du "lancer d'un dé à six faces" :
\newline
L'\textbf{univers} est $S = \{1, 2, 3, 4, 5, 6\}$.
"Obtenir un nombre impair" est un événement, représenté par le sous-ensemble $A = \{1, 3, 5\}$.
\end{examplebox}

\subsection{Définition Naïve de la Probabilité}

\begin{definitionbox}[Probabilité Naïve]
Pour une expérience où chaque issue dans un espace échantillon fini $S$ est équiprobable, la probabilité d'un événement $A$ est le rapport du nombre d'issues favorables à $A$ sur le nombre total d'issues :
$$ P(A) = \frac{\text{Nombre d'issues favorables}}{\text{Nombre total d'issues}} = \frac{|A|}{|S|} $$
\end{definitionbox}

\begin{examplebox}[Applications de la définition naïve]
\begin{enumerate}
    \item \textbf{Lancer une pièce équilibrée :}
    L'espace échantillon est $S = \{\text{Pile, Face}\}$, donc $|S| = 2$.
    Si l'événement $A$ est "obtenir Pile", alors $A = \{\text{Pile}\}$ et $|A| = 1$.
    La probabilité est $P(A) = \frac{1}{2}$.

    \item \textbf{Lancer un dé à six faces non pipé :}
    L'espace échantillon est $S = \{1, 2, 3, 4, 5, 6\}$, donc $|S| = 6$.
    Si l'événement $B$ est "obtenir un nombre pair", alors $B = \{2, 4, 6\}$ et $|B| = 3$.
    La probabilité est $P(B) = \frac{3}{6} = \frac{1}{2}$.

    \item \textbf{Tirer une carte d'un jeu de 52 cartes :}
    L'espace échantillon $S$ contient 52 cartes, donc $|S| = 52$.
    Si l'événement $C$ est "tirer un Roi", il y a 4 Rois dans le jeu, donc $|C| = 4$.
    La probabilité est $P(C) = \frac{4}{52} = \frac{1}{13}$.
\end{enumerate}
\end{examplebox}

\subsection{Permutations (Arrangements)}

\begin{definitionbox}[Permutation de $k$ objets parmi $n$]
Le nombre de façons d'arranger $k$ objets choisis parmi $n$ objets distincts (où l'ordre compte et il n'y a pas de répétition) est noté $P(n, k)$ ou $A_n^k$ et est défini par :
$$ P(n, k) = \frac{n!}{(n-k)!} $$
où $n!$ est la factorielle de $n$, et par convention $0! = 1$.
\end{definitionbox}

\begin{intuitionbox}[Permutations de $k$ parmi $n$]
Pour placer $k$ objets dans un ordre spécifique en les choisissant parmi $n$ objets disponibles, on a $n$ choix pour la première position, $(n-1)$ choix pour la deuxième, ..., et $(n-k+1)$ choix pour la $k$-ième position. Cela donne $n \times (n-1) \times \cdots \times (n-k+1)$ arrangements. Ce produit contient $k$ termes. Il est égal à $\frac{n!}{(n-k)!}$, car cela revient à diviser la suite complète $n!$ par les facteurs non utilisés $(n-k) \times (n-k-1) \times \cdots \times 1$.
\end{intuitionbox}

\begin{examplebox}[Permutations de $k$ parmi $n$]
\textbf{Podium d'une course :} Une course réunit 8 coureurs. Combien y a-t-il de podiums (1er, 2e, 3e) possibles ? \\
On cherche le nombre de façons d'ordonner 3 coureurs parmi 8 : $P(8, 3)$. 
$$ P(8, 3) = \frac{8!}{(8-3)!} = \frac{8!}{5!} = 8 \times 7 \times 6 = 336 $$
Il y a 336 podiums possibles.
\end{examplebox}

\subsection{Le Coefficient Binomial}

\begin{theorembox}[Formule du Coefficient Binomial]
Le nombre de façons de choisir $k$ objets parmi un ensemble de $n$ objets distincts (sans remise et sans ordre) est donné par le coefficient binomial :
$$ \binom{n}{k} = \frac{n!}{k!(n-k)!} $$
\end{theorembox}

\begin{intuitionbox}

L’idée est de relier $\binom{n}{k}$ à quelque chose de plus facile à compter : les \textbf{permutations} de $k$ objets parmi $n$, c’est-à-dire les listes ordonnées.  
On sait qu’il y en a :
\[
P(n,k) = \frac{n!}{(n-k)!}.
\]

D’un autre côté, on peut construire chaque permutation en deux étapes :
\begin{enumerate}
    \item Choisir un \textbf{sous-ensemble} de $k$ objets (sans ordre), il y a $\binom{n}{k}$ façons de le faire.
    \item Ordonner ces $k$ objets, il y a $k!$ façons de le faire.
\end{enumerate}
Donc, le nombre total de permutations est aussi $\binom{n}{k} \cdot k!$.

\medskip

\noindent En égalisant les deux expressions :
\[
\binom{n}{k} \cdot k! = \frac{n!}{(n-k)!}
\quad\Longrightarrow\quad
\binom{n}{k} = \frac{n!}{k!(n-k)!}.
\]

\medskip

\noindent Pour rendre cela concret, voici le cas $\binom{5}{3}$.  
Il y a 10 sous-ensembles de 3 éléments parmi $\{a,b,c,d,e\}$. Chacun donne lieu à $3! = 6$ permutations.  
Le tableau ci-dessous montre \textbf{toutes les 60 permutations}, regroupées par sous-ensemble :

\begin{center}
\small
\renewcommand{\arraystretch}{0.9}
\setlength{\tabcolsep}{2pt}
\begin{tabular}{|c|c|c|c|c|c|c|c|c|c|}
\hline
\textbf{$\{a,b,c\}$} & \textbf{$\{a,b,d\}$} & \textbf{$\{a,b,e\}$} & \textbf{$\{a,c,d\}$} & \textbf{$\{a,c,e\}$} & \textbf{$\{a,d,e\}$} & \textbf{$\{b,c,d\}$} & \textbf{$\{b,c,e\}$} & \textbf{$\{b,d,e\}$} & \textbf{$\{c,d,e\}$} \\
\hline
$abc$ & $abd$ & $abe$ & $acd$ & $ace$ & $ade$ & $bcd$ & $bce$ & $bde$ & $cde$ \\
\hline
$acb$ & $adb$ & $aeb$ & $adc$ & $aec$ & $aed$ & $bdc$ & $bec$ & $bed$ & $ced$ \\
\hline
$bac$ & $bad$ & $bae$ & $cad$ & $cae$ & $dae$ & $cbd$ & $ceb$ & $dbe$ & $dce$ \\
\hline
$bca$ & $bda$ & $bea$ & $cda$ & $cea$ & $dea$ & $cdb$ & $ceb$ & $deb$ & $dec$ \\
\hline
$cab$ & $dab$ & $eab$ & $dac$ & $eac$ & $ead$ & $dbc$ & $ebc$ & $edb$ & $ecd$ \\
\hline
$cba$ & $dba$ & $eba$ & $dca$ & $eca$ & $eda$ & $dcb$ & $ebc$ & $edb$ & $edc$ \\
\hline
\end{tabular}
\end{center}

\smallskip

Chaque colonne correspond à \textbf{un seul et même choix non ordonné} (par exemple $\{a,b,c\}$), mais à 6 listes différentes selon l’ordre.  
Ainsi, pour obtenir le nombre de \textit{choix non ordonnés}, on divise le nombre total de listes ($60$) par le nombre d’ordres par groupe ($6$) :
\[
\binom{5}{3} = \frac{60}{6} = 10.
\]

\medskip

\noindent C’est exactement ce que fait la formule :
\[
\binom{n}{k} = \frac{\text{nombre de permutations de } k \text{ parmi } n}{k!} = \frac{n!}{k!(n-k)!}.
\]

\end{intuitionbox}

\begin{examplebox}[Utilisation du Coefficient Binomial]
    \textbf{Comité d'étudiants :} De combien de manières peut-on former un comité de 3 étudiants à partir d'une classe de 10 ? L'ordre ne compte pas.
    $$ \binom{10}{3} = \frac{10!}{3!(10-3)!} = \frac{10 \times 9 \times 8}{3 \times 2 \times 1} = 120 \text{ comités possibles.} $$
\end{examplebox}

\subsection{Identité de Vandermonde}

\begin{theorembox}[Identité de Vandermonde]
Cette identité offre une relation remarquable entre les coefficients binomiaux. Pour des entiers non négatifs $m, n$ et $k$, on a :
$$ \binom{m+n}{k} = \sum_{j=0}^{k} \binom{m}{j} \binom{n}{k-j} $$
\end{theorembox}

\begin{intuitionbox}
C'est le "principe du diviser pour régner". Imaginez que vous devez choisir un comité de $k$ personnes à partir d'un groupe contenant $m$ hommes et $n$ femmes.
Le côté gauche, $\binom{m+n}{k}$, compte directement le nombre total de comités possibles.
Le côté droit arrive au même résultat en additionnant toutes les compositions possibles du comité : choisir 0 homme et $k$ femmes, PLUS 1 homme et $k-1$ femmes, PLUS 2 hommes et $k-2$ femmes, etc., jusqu'à choisir $k$ hommes et 0 femme. La somme de toutes ces possibilités doit être égale au total.
\end{intuitionbox}

\begin{examplebox}[Application de l'Identité de Vandermonde]
On veut former un comité de 3 personnes ($k=3$) à partir d'un groupe de 5 hommes ($m=5$) et 4 femmes ($n=4$).
\vspace{0.3cm}
\noindent\textbf{Méthode directe (côté gauche) :} \\
On choisit 3 personnes parmi les $5+4=9$ au total.
$$ \binom{9}{3} = \frac{9 \times 8 \times 7}{3 \times 2 \times 1} = 84 $$
\vspace{0.3cm}
\noindent\textbf{Méthode par cas (côté droit) :} \\
La somme est $\binom{5}{0}\binom{4}{3} + \binom{5}{1}\binom{4}{2} + \binom{5}{2}\binom{4}{1} + \binom{5}{3}\binom{4}{0} = 84$. Les deux méthodes donnent bien le même résultat.
\end{examplebox}

\subsection{Bose-Einstein (Étoiles et Bâtons)}

\begin{theorembox}[Combinaisons avec répétition]
Le nombre de façons de distribuer $k$ objets indiscernables dans $n$ boîtes discernables (ou de choisir $k$ objets parmi $n$ avec remise, où l'ordre ne compte pas) est donné par la formule :
$$ \binom{n+k-1}{k} = \binom{n+k-1}{n-1} $$
\end{theorembox}

\begin{intuitionbox}[Étoiles et Bâtons]
Imaginez que les $k$ objets sont des étoiles ($\star$) et que nous avons besoin de $n-1$ bâtons ($|$) pour les séparer en $n$ groupes. Par exemple, pour distribuer $k=7$ étoiles dans $n=4$ boîtes, une configuration possible serait :
$$ \star\star\star \mid \star \mid \mid \star\star\star $$
Cela correspond à 3 objets dans la première boîte, 1 dans la deuxième, 0 dans la troisième et 3 dans la quatrième.
Le problème revient à trouver le nombre de façons d'arranger ces $k$ étoiles et $n-1$ bâtons. Nous avons un total de $n+k-1$ positions, et nous devons choisir les $k$ positions pour les étoiles (ou les $n-1$ positions pour les bâtons). Le nombre de manières de le faire est précisément $\binom{n+k-1}{k}$.
\end{intuitionbox}

\begin{examplebox}[Distribution de biens identiques]
De combien de manières peut-on distribuer 10 croissants identiques à 4 enfants ?
\newline
Ici, $k=10$ (les croissants, objets indiscernables) et $n=4$ (les enfants, boîtes discernables).
Le nombre de distributions possibles est :
$$ \binom{4+10-1}{10} = \binom{13}{10} = \binom{13}{3} = \frac{13 \times 12 \times 11}{3 \times 2 \times 1} = 13 \times 2 \times 11 = 286 $$
Il y a 286 façons de distribuer les croissants.
\end{examplebox}

\subsection{Principe d'Inclusion-Exclusion}

\begin{theorembox}[Principe d'Inclusion-Exclusion pour 3 ensembles]
Pour trois ensembles finis $A$, $B$ et $C$, le nombre d'éléments dans leur union est donné par :
$$ |A \cup B \cup C| = |A| + |B| + |C| - |A \cap B| - |A \cap C| - |B \cap C| + |A \cap B \cap C| $$
\end{theorembox}

\begin{intuitionbox}[Visualisation avec 3 ensembles]
Le principe d'inclusion-exclusion permet de compter le nombre d'éléments dans une union d'ensembles sans double-comptage. Pour comprendre intuitivement pourquoi on ajoute et soustrait alternativement, considérons trois ensembles $A$, $B$ et $C$ :

\begin{center}
\begin{tikzpicture}[set/.style = {draw,
    circle,
    minimum size = 6cm,
    fill=Rhodamine,
    opacity = 0.4,
    text opacity = 1}]
 
\node (A) [set] {$A$};
\node (B) at (60:4cm) [set] {$B$};
\node (C) at (0:4cm) [set] {$C$};
 
\node at (barycentric cs:A=1,B=1) [left] {$X$};
\node at (barycentric cs:A=1,C=1) [below] {$Y$};
\node at (barycentric cs:B=1,C=1) [right] {$Z$};
\node at (barycentric cs:A=1,B=1,C=1) [] {$T$};
 
\end{tikzpicture}
\end{center}

\textbf{Le problème :} Si on additionne simplement $|A| + |B| + |C|$, on compte certaines zones plusieurs fois :
\begin{itemize}
    \item Les intersections deux à deux ($X$, $Y$, $Z$) sont comptées \textbf{deux fois}
    \item L'intersection triple ($T$) est comptée \textbf{trois fois}
\end{itemize}

\textbf{La solution :} On corrige en soustrayant les intersections deux à deux, mais alors l'intersection triple est comptée :
\begin{itemize}
    \item $+3$ fois dans la somme initiale
    \item $-3$ fois dans la soustraction des intersections deux à deux (car elle appartient à chacune)
    \item Donc $0$ fois au total ! Il faut la rajouter.
\end{itemize}

D'où la formule : $|A \cup B \cup C| = |A| + |B| + |C| - |A \cap B| - |A \cap C| - |B \cap C| + |A \cap B \cap C|$
\end{intuitionbox}

\begin{theorembox}[Principe d'Inclusion-Exclusion généralisé]
Pour $n$ ensembles finis $A_1, A_2, \dots, A_n$, on a :
\begin{align*}
|A_1 \cup A_2 \cup \cdots \cup A_n| = & \sum_{i=1}^n |A_i| \\
& - \sum_{1 \leq i < j \leq n} |A_i \cap A_j| \\
& + \sum_{1 \leq i < j < k \leq n} |A_i \cap A_j \cap A_k| \\
& - \cdots \\
& + (-1)^{n+1} |A_1 \cap A_2 \cap \cdots \cap A_n|
\end{align*}
Ce qui s'écrit plus compactement :
$$ \left| \bigcup_{i=1}^n A_i \right| = \sum_{k=1}^n (-1)^{k+1} \sum_{1 \leq i_1 < i_2 < \cdots < i_k \leq n} |A_{i_1} \cap A_{i_2} \cap \cdots \cap A_{i_k}| $$
\end{theorembox}

\begin{intuitionbox}[Généralisation]
La logique reste la même que pour trois ensembles, mais l'argument clé est de prouver que chaque élément est compté \textbf{exactement une fois}, peu importe le nombre d'ensembles auxquels il appartient.

Supposons qu'un élément $x$ est membre d'exactement $k$ ensembles parmi les $n$ ensembles $A_1, \ldots, A_n$. Analysons combien de fois $x$ est compté dans la formule :
\begin{itemize}
    \item \textbf{Première somme ($\sum |A_i|$)} : $x$ est dans $k$ ensembles, donc il est ajouté $k$ fois. Le nombre de fois est $\binom{k}{1}$.
    
    \item \textbf{Deuxième somme ($-\sum |A_i \cap A_j|$)} : On soustrait $x$ pour chaque paire d'ensembles auxquels il appartient. Il y a $\binom{k}{2}$ telles paires.
    
    \item \textbf{Troisième somme ($+\sum |A_i \cap A_j \cap A_k|$)} : On ajoute de nouveau $x$ pour chaque triplet d'ensembles auxquels il appartient. Il y en a $\binom{k}{3}$.
    
    \item \textbf{Et ainsi de suite...}
\end{itemize}

Au total, l'élément $x$ est compté :
$$ \binom{k}{1} - \binom{k}{2} + \binom{k}{3} - \cdots + (-1)^{k-1}\binom{k}{k} \text{ fois.} $$

Pour voir que cette somme vaut exactement 1, rappelons une identité fondamentale issue du binôme de Newton :
$$ (1-1)^k = \sum_{j=0}^{k} (-1)^j \binom{k}{j} = \binom{k}{0} - \binom{k}{1} + \binom{k}{2} - \cdots + (-1)^k \binom{k}{k} = 0 $$

En réarrangeant cette équation, sachant que $\binom{k}{0}=1$ :
$$ \binom{k}{0} = \binom{k}{1} - \binom{k}{2} + \binom{k}{3} - \cdots - (-1)^{k}\binom{k}{k} $$
$$ 1 = \binom{k}{1} - \binom{k}{2} + \binom{k}{3} - \cdots + (-1)^{k-1}\binom{k}{k} $$

Cela prouve que n'importe quel élément, qu'il soit dans un seul ensemble ($k=1$) ou dans plusieurs ($k>1$), contribue précisément pour 1 au décompte final. Le principe d'inclusion-exclusion est donc une méthode infaillible pour corriger les comptages multiples de manière systématique.
\end{intuitionbox}


\begin{examplebox}[Application probabiliste]
On lance trois dés équilibrés. Quelle est la probabilité d'obtenir au moins un 6 ?

\vspace{0.3cm}
\noindent\textbf{Solution avec inclusion-exclusion :}

Soit $A$ = "le premier dé montre 6", $B$ = "le deuxième dé montre 6", $C$ = "le troisième dé montre 6".

On veut $P(A \cup B \cup C)$.

\begin{align*}
P(A \cup B \cup C) &= P(A) + P(B) + P(C) \\
&\quad - P(A \cap B) - P(A \cap C) - P(B \cap C) \\
&\quad + P(A \cap B \cap C) \\
&= \frac{1}{6} + \frac{1}{6} + \frac{1}{6} - \frac{1}{36} - \frac{1}{36} - \frac{1}{36} + \frac{1}{216} \\
&= \frac{3}{6} - \frac{3}{36} + \frac{1}{216} = \frac{1}{2} - \frac{1}{12} + \frac{1}{216} \\
&= \frac{108 - 18 + 1}{216} = \frac{91}{216} \approx 0.421
\end{align*}

\vspace{0.3cm}
\noindent\textbf{Vérification par la méthode complémentaire :}
La probabilité de n'obtenir aucun 6 est $\left(\frac{5}{6}\right)^3 = \frac{125}{216}$, donc la probabilité d'au moins un 6 est $1 - \frac{125}{216} = \frac{91}{216}$.
\end{examplebox}



\subsection{Exercices}

%  Concepts de Base et Règle du Produit 

\begin{exercicebox}[Exercice 1 : Dés et Probabilité Conditionnelle Simple]
On lance deux dés équilibrés à 6 faces.
\begin{enumerate}
    \item Quelle est la probabilité que la somme des dés soit 8 ?
    \item Sachant que le premier dé a donné un 3, quelle est la probabilité que la somme soit 8 ?
    \item Sachant que la somme est 8, quelle est la probabilité que le premier dé ait donné un 3 ?
\end{enumerate}
\end{exercicebox}

\begin{exercicebox}[Exercice 2 : Tirage de Cartes (Sans Remise)]
On tire deux cartes successivement et sans remise d'un jeu standard de 52 cartes.
\begin{enumerate}
    \item Quelle est la probabilité que la deuxième carte soit un Roi, sachant que la première était un Roi ?
    \item Quelle est la probabilité de tirer deux Rois ?
\end{enumerate}
\end{exercicebox}

\begin{exercicebox}[Exercice 3 : Urne (Règle du Produit)]
Une urne contient 7 boules rouges et 3 boules bleues. On tire deux boules successivement et sans remise.
\begin{enumerate}
    \item Quelle est la probabilité que la première boule soit rouge ?
    \item Quelle est la probabilité que la deuxième boule soit bleue, sachant que la première était rouge ?
    \item Quelle est la probabilité de tirer une boule rouge puis une boule bleue ?
\end{enumerate}
\end{exercicebox}

\begin{exercicebox}[Exercice 4 : Famille (Condition Simple)]
Une famille a deux enfants. On suppose que la probabilité d'avoir un garçon (G) ou une fille (F) est la même (0.5) et que les naissances sont indépendantes.
\begin{enumerate}
    \item Quel est l'univers $S$ des possibilités ?
    \item Sachant que l'aîné est un garçon, quelle est la probabilité que la famille ait deux garçons ?
\end{enumerate}
\end{exercicebox}

\begin{exercicebox}[Exercice 5 : Famille (Condition "Au Moins")]
En utilisant le même scénario que l'exercice 4 (famille de deux enfants) :
Sachant qu'il y a \textit{au moins un} garçon dans la famille, quelle est la probabilité que la famille ait deux garçons ?
\end{exercicebox}

%  Indépendance 

\begin{exercicebox}[Exercice 6 : Indépendance (Dés)]
On lance deux dés équilibrés.
Soit $A$ l'événement "le premier dé donne 3" et $B$ l'événement "la somme des deux dés est 7".
Les événements $A$ et $B$ sont-ils indépendants ? Justifiez par le calcul.
\end{exercicebox}

\begin{exercicebox}[Exercice 7 : Indépendance (Cartes)]
On tire une carte d'un jeu de 52 cartes.
Soit $A$ l'événement "la carte est un Roi" et $B$ l'événement "la carte est un Cœur".
Les événements $A$ et $B$ sont-ils indépendants ?
\end{exercicebox}

\begin{exercicebox}[Exercice 8 : Indépendance vs Exclusion Mutuelle]
Soient $A$ et $B$ deux événements avec $P(A)=0.5$ et $P(B)=0.3$.
\begin{enumerate}
    \item Si $A$ et $B$ sont mutuellement exclusifs (disjoints), sont-ils indépendants ?
    \item Si $A$ et $B$ sont indépendants, quelle est $P(A \cup B)$ ?
\end{enumerate}
\end{exercicebox}

%  Formule des Probabilités Totales (LTP) 

\begin{exercicebox}[Exercice 9 : LTP (Deux Urnes)]
L'urne U1 contient 2 boules noires et 3 boules blanches. L'urne U2 contient 4 boules noires et 1 boule blanche.
On choisit une urne au hasard (chaque urne a 50\% de chance d'être choisie), puis on tire une boule de cette urne.
Quelle est la probabilité de tirer une boule blanche ?
\end{exercicebox}

\begin{exercicebox}[Exercice 10 : LTP (Usine)]
Une usine utilise deux machines, M1 et M2, pour produire des pièces. M1 produit 40\% des pièces et M2 produit 60\%. 5\% des pièces de M1 sont défectueuses, et 2\% des pièces de M2 sont défectueuses.
Si l'on choisit une pièce au hasard dans la production totale, quelle est la probabilité qu'elle soit défectueuse ?
\end{exercicebox}

\begin{exercicebox}[Exercice 11 : LTP (Pièce de Monnaie Inconnue)]
On a deux pièces. La pièce A est équilibrée ($P(\text{Pile})=0.5$). La pièce B est truquée ($P(\text{Pile})=0.8$).
On choisit une pièce au hasard et on la lance. Quelle est la probabilité d'obtenir Pile ?
\end{exercicebox}

%  Règle de Bayes 

\begin{exercicebox}[Exercice 12 : Bayes (Test Médical)]
Une maladie touche 1 personne sur 1000 ($P(M)=0.001$). Un test de dépistage donne un résultat positif chez 98\% des personnes malades ($P(T|M)=0.98$). Il donne aussi un résultat positif (un "faux positif") chez 3\% des personnes non malades ($P(T|\neg M)=0.03$).
Une personne reçoit un test positif. Quelle est la probabilité qu'elle soit réellement malade ?
\end{exercicebox}

\begin{exercicebox}[Exercice 13 : Bayes (Inversion d'Urnes)]
Reprenons le scénario de l'exercice 9 (U1 avec 2N/3B, U2 avec 4N/1B).
On a tiré une boule et on constate qu'elle est blanche. Quelle est la probabilité qu'elle provienne de l'urne U1 ?
\end{exercicebox}

\begin{exercicebox}[Exercice 14 : Bayes (Spam)]
Dans une boîte de réception, 60\% des emails sont des spams. 70\% des spams contiennent le mot "gratuit". Seuls 10\% des emails légitimes contiennent le mot "gratuit".
Vous recevez un email qui contient le mot "gratuit". Quelle est la probabilité que ce soit un spam ?
\end{exercicebox}

\begin{exercicebox}[Exercice 15 : Bayes (Usine Inversée)]
Reprenons le scénario de l'exercice 10 (M1: 40\% prod, 5\% défaut; M2: 60% prod, 2% défaut).
On trouve une pièce défectueuse. Quelle est la probabilité qu'elle ait été produite par la machine M1 ?
\end{exercicebox}

%  Règle de la Chaîne et Problèmes Combinés 

\begin{exercicebox}[Exercice 16 : Règle de la Chaîne (3 Cartes)]
On tire 3 cartes successivement et sans remise d'un jeu de 52 cartes.
Quelle est la probabilité de tirer 3 Piques ?
\end{exercicebox}

\begin{exercicebox}[Exercice 17 : Problème de Monty Hall (Calcul)]
En utilisant la formalisation du problème de Monty Hall (vous choisissez la Porte 1, la voiture est en $V \in \{1, 2, 3\}$, l'animateur ouvre $H \in \{2, 3\}$) :
Calculez $P(V=1 | H=3)$ (la probabilité que la voiture soit derrière votre porte, sachant que l'animateur a ouvert la 3). Supposez que $P(V=i)=1/3$ pour $i=1,2,3$.
\end{exercicebox}

\begin{exercicebox}[Exercice 18 : Bayes avec Mise à Jour (Pièce Truquée)]
Reprenons l'exercice 11 (Pièce A équilibrée, Pièce B truquée $P(\text{Pile})=0.8$).
On choisit une pièce au hasard. On la lance deux fois et on obtient Pile, puis Pile (PP).
Quelle est la probabilité que l'on ait choisi la pièce truquée (Pièce B) ?
\end{exercicebox}

\begin{exercicebox}[Exercice 19 : Indépendance Conditionnelle (Dés)]
On lance deux dés, $D_1$ et $D_2$. Soit $S = D_1 + D_2$ leur somme.
Soit $A$ l'événement "$D_1 = 1$", $B$ l'événement "$D_2 = 1$".
$A$ et $B$ sont indépendants. Sont-ils indépendants conditionnellement à l'événement $C = \{S = 2\}$ ?
\end{exercicebox}

\begin{exercicebox}[Exercice 20 : Jeu Séquentiel]
Alice et Bob jouent à un jeu. Ils lancent un dé à tour de rôle, en commençant par Alice. Le premier qui obtient un 6 gagne.
Quelle est la probabilité qu'Alice gagne ?
\end{exercicebox}

\subsection{Corrections des Exercices}

%  Corrections : Concepts de Base et Règle du Produit 

\begin{correctionbox}[Correction Exercice 1 : Dés et Probabilité Conditionnelle Simple]
L'univers $S$ a $|S| = 6 \times 6 = 36$ issues.

1.  Soit $A$ l'événement "la somme est 8". $A = \{(2,6), (3,5), (4,4), (5,3), (6,2)\}$.
    $|A|=5$, donc $P(A) = 5/36$.

2.  Soit $B$ l'événement "le premier dé donne 3". $B = \{(3,1), (3,2), (3,3), (3,4), (3,5), (3,6)\}$.
    On cherche $P(A|B)$. Sachant $B$, l'univers est réduit à ces 6 issues. Parmi celles-ci, seule l'issue $(3,5)$ donne une somme de 8.
    Donc, $P(A|B) = 1/6$.
    *Par formule :* $A \cap B = \{(3,5)\}$, $P(A \cap B) = 1/36$. $P(B) = 6/36 = 1/6$.
    $P(A|B) = \frac{P(A \cap B)}{P(B)} = \frac{1/36}{6/36} = 1/6$.

3.  On cherche $P(B|A)$. Sachant $A$, l'univers est réduit aux 5 issues de $A$. Parmi celles-ci, seule l'issue $(3,5)$ a 3 sur le premier dé.
    Donc, $P(B|A) = 1/5$.
    *Par formule :* $P(B|A) = \frac{P(A \cap B)}{P(A)} = \frac{1/36}{5/36} = 1/5$.
\end{correctionbox}

\begin{correctionbox}[Correction Exercice 2 : Tirage de Cartes (Sans Remise)]
Soit $K_1$ l'événement "Roi au 1er tirage" et $K_2$ "Roi au 2e tirage".

1.  On cherche $P(K_2|K_1)$. Si $K_1$ s'est produit, il reste 51 cartes dans le jeu, dont $4-1=3$ Rois.
    $P(K_2|K_1) = 3/51 = 1/17$.

2.  On cherche $P(K_1 \cap K_2)$. On utilise la règle du produit :
    $P(K_1 \cap K_2) = P(K_1) \times P(K_2|K_1)$
    $P(K_1) = 4/52 = 1/13$.
    $P(K_1 \cap K_2) = (4/52) \times (3/51) = (1/13) \times (1/17) = 1/221$.
\end{correctionbox}

\begin{correctionbox}[Correction Exercice 3 : Urne (Règle du Produit)]
Urne avec 7 Rouges (R) et 3 Bleues (B). Total = 10.
Soit $R_1$ "Rouge au 1er tirage" et $B_2$ "Bleue au 2e tirage".

1.  $P(R_1) = 7/10$.
2.  On cherche $P(B_2|R_1)$. Si $R_1$ s'est produit, il reste 9 boules (6R, 3B).
    $P(B_2|R_1) = 3/9 = 1/3$.
3.  On cherche $P(R_1 \cap B_2)$.
    $P(R_1 \cap B_2) = P(R_1) \times P(B_2|R_1) = (7/10) \times (1/3) = 7/30$.
\end{correctionbox}

\begin{correctionbox}[Correction Exercice 4 : Famille (Condition Simple)]
1.  L'univers est $S = \{GG, GF, FG, FF\}$, où le premier enfant est l'aîné. $|S|=4$, chaque issue a une probabilité de 1/4.
2.  Soit $A$ l'événement "l'aîné est un garçon" : $A = \{GG, GF\}$. $P(A) = 2/4 = 1/2$.
    Soit $B$ l'événement "la famille a deux garçons" : $B = \{GG\}$. $P(B) = 1/4$.
    On cherche $P(B|A)$. L'événement $A \cap B = \{GG\}$. $P(A \cap B) = 1/4$.
    $P(B|A) = \frac{P(A \cap B)}{P(A)} = \frac{1/4}{1/2} = 1/2$.
\end{correctionbox}

\begin{correctionbox}[Correction Exercice 5 : Famille (Condition "Au Moins")]
Soit $B$ l'événement "la famille a deux garçons" : $B = \{GG\}$.
Soit $C$ l'événement "il y a au moins un garçon" : $C = \{GG, GF, FG\}$. $P(C) = 3/4$.
On cherche $P(B|C)$.
L'événement $B \cap C = \{GG\}$. $P(B \cap C) = 1/4$.
$P(B|C) = \frac{P(B \cap C)}{P(C)} = \frac{1/4}{3/4} = 1/3$.
*Intuition :* L'univers de $C$ est $\{GG, GF, FG\}$. Parmi ces 3 issues équiprobables, une seule est $GG$.
\end{correctionbox}

%  Corrections : Indépendance 

\begin{correctionbox}[Correction Exercice 6 : Indépendance (Dés)]
$A$ = "premier dé = 3". $P(A) = 6/36 = 1/6$.
$B$ = "somme = 7". $B = \{(1,6), (2,5), (3,4), (4,3), (5,2), (6,1)\}$. $P(B) = 6/36 = 1/6$.
$A \cap B$ = "premier dé = 3 ET somme = 7" = $\%(3,4)\}$. $P(A \cap B) = 1/36$.
On teste si $P(A \cap B) = P(A)P(B)$.
$P(A)P(B) = (1/6) \times (1/6) = 1/36$.
Puisque $P(A \cap B) = P(A)P(B)$, les événements $A$ et $B$ sont indépendants.
\end{correctionbox}

\begin{correctionbox}[Correction Exercice 7 : Indépendance (Cartes)]
$A$ = "Roi". $P(A) = 4/52 = 1/13$.
$B$ = "Cœur". $P(B) = 13/52 = 1/4$.
$A \cap B$ = "Roi de Cœur". $P(A \cap B) = 1/52$.
On teste si $P(A \cap B) = P(A)P(B)$.
$P(A)P(B) = (1/13) \times (1/4) = 1/52$.
Puisque $P(A \cap B) = P(A)P(B)$, les événements $A$ et $B$ sont indépendants.
\end{correctionbox}

\begin{correctionbox}[Correction Exercice 8 : Indépendance vs Exclusion Mutuelle]
$P(A)=0.5, P(B)=0.3$.

1.  Si $A$ et $B$ sont mutuellement exclusifs, $A \cap B = \emptyset$, donc $P(A \cap B) = 0$.
    Pour qu'ils soient indépendants, il faudrait $P(A \cap B) = P(A)P(B) = 0.5 \times 0.3 = 0.15$.
    Puisque $0 \neq 0.15$, ils ne sont pas indépendants. (Deux événements non impossibles ne peuvent pas être à la fois mutuellement exclusifs et indépendants).

2.  Si $A$ et $B$ sont indépendants, $P(A \cap B) = P(A)P(B) = 0.15$.
    $P(A \cup B) = P(A) + P(B) - P(A \cap B) = 0.5 + 0.3 - 0.15 = 0.65$.
\end{correctionbox}

%  Corrections : Formule des Probabilités Totales (LTP) 

\begin{correctionbox}[Correction Exercice 9 : LTP (Deux Urnes)]
Soit $U_1$ et $U_2$ les événements "choisir l'urne 1" et "choisir l'urne 2". $P(U_1)=0.5, P(U_2)=0.5$.
Soit $W$ l'événement "tirer une boule blanche".
On a $P(W|U_1) = 3 / (2+3) = 3/5 = 0.6$.
On a $P(W|U_2) = 1 / (4+1) = 1/5 = 0.2$.
Par la formule des probabilités totales :
$P(W) = P(W|U_1)P(U_1) + P(W|U_2)P(U_2)$
$P(W) = (0.6 \times 0.5) + (0.2 \times 0.5) = 0.3 + 0.1 = 0.4$.
\end{correctionbox}

\begin{correctionbox}[Correction Exercice 10 : LTP (Usine)]
Soit $M_1$ et $M_2$ les machines. $P(M_1)=0.4, P(M_2)=0.6$.
Soit $D$ l'événement "la pièce est défectueuse".
On a $P(D|M_1) = 0.05$ et $P(D|M_2) = 0.02$.
Par la formule des probabilités totales :
$P(D) = P(D|M_1)P(M_1) + P(D|M_2)P(M_2)$
$P(D) = (0.05 \times 0.4) + (0.02 \times 0.6) = 0.020 + 0.012 = 0.032$.
La probabilité est de 3.2\%.
\end{correctionbox}

\begin{correctionbox}[Correction Exercice 11 : LTP (Pièce de Monnaie Inconnue)]
Soit $A$ "choisir pièce A" et $B$ "choisir pièce B". $P(A)=0.5, P(B)=0.5$.
Soit $H$ l'événement "obtenir Pile".
On a $P(H|A) = 0.5$ et $P(H|B) = 0.8$.
Par la formule des probabilités totales :
$P(H) = P(H|A)P(A) + P(H|B)P(B)$
$P(H) = (0.5 \times 0.5) + (0.8 \times 0.5) = 0.25 + 0.40 = 0.65$.
\end{correctionbox}

%  Corrections : Règle de Bayes 

\begin{correctionbox}[Correction Exercice 12 : Bayes (Test Médical)]
Soit $M$ "Malade" et $T$ "Test Positif".
$P(M) = 0.001$, donc $P(\neg M) = 0.999$.
$P(T|M) = 0.98$.
$P(T|\neg M) = 0.03$.
On cherche $P(M|T)$. Par la règle de Bayes : $P(M|T) = \frac{P(T|M)P(M)}{P(T)}$.

1.  Calculer $P(T)$ (dénominateur) avec la LTP :
    $P(T) = P(T|M)P(M) + P(T|\neg M)P(\neg M)$
    $P(T) = (0.98 \times 0.001) + (0.03 \times 0.999) = 0.00098 + 0.02997 = 0.03095$.

2.  Appliquer la règle de Bayes :
    $P(M|T) = \frac{0.00098}{0.03095} \approx 0.03166$.
    Il n'y a que 3.17\% de chance que la personne soit malade, même avec un test positif.
\end{correctionbox}

\begin{correctionbox}[Correction Exercice 13 : Bayes (Inversion d'Urnes)]
D'après l'exercice 9, on a :
$P(W) = 0.4$ (prob. totale de tirer une blanche).
$P(W|U_1) = 0.6$.
$P(U_1) = 0.5$.
On cherche $P(U_1|W)$. Par la règle de Bayes :
$P(U_1|W) = \frac{P(W|U_1)P(U_1)}{P(W)} = \frac{0.6 \times 0.5}{0.4} = \frac{0.3}{0.4} = 0.75$.
Sachant que la boule est blanche, il y a 75\% de chance qu'elle vienne de l'urne U1.
\end{correctionbox}

\begin{correctionbox}[Correction Exercice 14 : Bayes (Spam)]
Soit $S$ "Spam" et $G$ "Contient 'gratuit'".
$P(S) = 0.6$, donc $P(\neg S) = 0.4$.
$P(G|S) = 0.7$.
$P(G|\neg S) = 0.1$.
On cherche $P(S|G)$. Par la règle de Bayes : $P(S|G) = \frac{P(G|S)P(S)}{P(G)}$.

1.  Calculer $P(G)$ (dénominateur) avec la LTP :
    $P(G) = P(G|S)P(S) + P(G|\neg S)P(\neg S)$
    $P(G) = (0.7 \times 0.6) + (0.1 \times 0.4) = 0.42 + 0.04 = 0.46$.

2.  Appliquer la règle de Bayes :
    $P(S|G) = \frac{0.42}{0.46} \approx 0.913$.
    Il y a 91.3\% de chance que l'email soit un spam.
\end{correctionbox}

\begin{correctionbox}[Correction Exercice 15 : Bayes (Usine Inversée)]
D'après l'exercice 10, on a :
$P(D) = 0.032$ (prob. totale d'être défectueux).
$P(D|M_1) = 0.05$.
$P(M_1) = 0.4$.
On cherche $P(M_1|D)$. Par la règle de Bayes :
$P(M_1|D) = \frac{P(D|M_1)P(M_1)}{P(D)} = \frac{0.05 \times 0.4}{0.032} = \frac{0.02}{0.032} = 0.625$.
Sachant que la pièce est défectueuse, il y a 62.5\% de chance qu'elle vienne de M1.
\end{correctionbox}

%  Corrections : Règle de la Chaîne et Problèmes Combinés 

\begin{correctionbox}[Correction Exercice 16 : Règle de la Chaîne (3 Cartes)]
Soit $P_i$ l'événement "tirer un Pique au $i$-ème tirage". Il y a 13 Piques sur 52 cartes.
On cherche $P(P_1 \cap P_2 \cap P_3)$. On utilise la règle de la chaîne :
$P(P_1 \cap P_2 \cap P_3) = P(P_1) \times P(P_2|P_1) \times P(P_3|P_1 \cap P_2)$
$P(P_1) = 13/52$.
$P(P_2|P_1) = 12/51$ (il reste 12 Piques sur 51 cartes).
$P(P_3|P_1 \cap P_2) = 11/50$ (il reste 11 Piques sur 50 cartes).
$P = (13/52) \times (12/51) \times (11/50) = \frac{1}{4} \times \frac{4}{17} \times \frac{11}{50} = \frac{11}{17 \times 50} = 11/850 \approx 0.0129$.
\end{correctionbox}

\begin{correctionbox}[Correction Exercice 17 : Problème de Monty Hall (Calcul)]
On cherche $P(V=1 | H=3)$. On utilise la règle de Bayes :
$P(V=1 | H=3) = \frac{P(H=3 | V=1) P(V=1)}{P(H=3)}$.

*Numérateur :* $P(V=1) = 1/3$. $P(H=3 | V=1)$ est la probabilité que l'animateur ouvre la 3, sachant que vous avez choisi la 1 et que la voiture est en 1. Il peut ouvrir la 2 ou la 3 (deux chèvres). On suppose qu'il choisit au hasard : $P(H=3 | V=1) = 1/2$.
Numérateur = $(1/2) \times (1/3) = 1/6$.

*Dénominateur $P(H=3)$ par LTP (partition sur V) :*
$P(H=3) = P(H=3|V=1)P(V=1) + P(H=3|V=2)P(V=2) + P(H=3|V=3)P(V=3)$
- $P(H=3|V=1) = 1/2$ (calculé ci-dessus).
- $P(H=3|V=2) = 1$ (l'animateur doit ouvrir la 3, car vous avez choisi 1 et la voiture est en 2).
- $P(H=3|V=3) = 0$ (l'animateur ne peut pas ouvrir la porte 3 car elle contient la voiture).
$P(H=3) = (1/2 \times 1/3) + (1 \times 1/3) + (0 \times 1/3) = 1/6 + 1/3 + 0 = 1/2$.

*Résultat :* $P(V=1 | H=3) = \frac{1/6}{1/2} = 1/3$.
(La probabilité que la voiture soit derrière votre porte reste 1/3. La probabilité qu'elle soit derrière l'autre porte fermée (la 2) est $P(V=2|H=3) = 1 - P(V=1|H=3) = 2/3$. Il faut donc changer.)
\end{correctionbox}

\begin{correctionbox}[Correction Exercice 18 : Bayes avec Mise à Jour (Pièce Truquée)]
Soit $A$ "pièce A (équil.)" et $B$ "pièce B (truquée, p=0.8)". $P(A)=P(B)=0.5$.
Soit $E$ l'événement "obtenir Pile, Pile" (PP).
On cherche $P(B|E) = \frac{P(E|B)P(B)}{P(E)}$.

1.  Probabilités conditionnelles de l'évidence $E$ :
    $P(E|A) = P(\text{PP} | A) = 0.5 \times 0.5 = 0.25$ (indépendance des lancers).
    $P(E|B) = P(\text{PP} | B) = 0.8 \times 0.8 = 0.64$.

2.  Calculer $P(E)$ (dénominateur) avec la LTP :
    $P(E) = P(E|A)P(A) + P(E|B)P(B)$
    $P(E) = (0.25 \times 0.5) + (0.64 \times 0.5) = 0.125 + 0.320 = 0.445$.

3.  Appliquer la règle de Bayes :
    $P(B|E) = \frac{P(E|B)P(B)}{P(E)} = \frac{0.64 \times 0.5}{0.445} = \frac{0.32}{0.445} \approx 0.719$.
    Après avoir observé PP, la probabilité que ce soit la pièce truquée passe de 50\% à 71.9\%.
\end{correctionbox}

\begin{correctionbox}[Correction Exercice 19 : Indépendance Conditionnelle (Dés)]
$A = \{D_1=1\}$, $B = \{D_2=1\}$, $C = \{S=2\}$.
On teste si $P(A \cap B | C) = P(A|C) P(B|C)$.

L'événement $C = \{S=2\}$ ne peut se produire que d'une seule façon : $C = \{(1,1)\}$.
Donc, $C$ est l'événement $A \cap B$. $C \subseteq A$ et $C \subseteq B$.

Calculons les termes :
- $P(A|C) = P(A \cap C) / P(C)$. Puisque $C \subseteq A$, $A \cap C = C$.
  $P(A|C) = P(C) / P(C) = 1$.
- $P(B|C) = P(B \cap C) / P(C)$. Puisque $C \subseteq B$, $B \cap C = C$.
  $P(B|C) = P(C) / P(C) = 1$.
- $P(A \cap B | C) = P((A \cap B) \cap C) / P(C)$. Puisque $A \cap B = C$, $(A \cap B) \cap C = C$.
  $P(A \cap B | C) = P(C) / P(C) = 1$.

Test d'indépendance :
$P(A \cap B | C) = 1$.
$P(A|C) P(B|C) = 1 \times 1 = 1$.
Puisque $1=1$, les événements $A$ et $B$ sont bien indépendants conditionnellement à $C$.
*Intuition :* Sachant que la somme est 2, nous savons avec certitude que $D_1=1$ et $D_2=1$. Il n'y a plus d'aléa.
\end{correctionbox}

\begin{correctionbox}[Correction Exercice 20 : Jeu Séquentiel]
Soit $p=1/6$ la probabilité de gagner (obtenir un 6) et $q=5/6$ la probabilité de rater.
Alice gagne si elle réussit au tour 1, OU si (elle rate ET Bob rate) et elle réussit au tour 3, OU si (A rate, B rate, A rate, B rate) et elle réussit au tour 5, etc.

$P(\text{A gagne}) = P(\text{A au tour 1}) + P(\text{A au tour 3}) + P(\text{A au tour 5}) + \dots$
$P(\text{A gagne}) = p + (q \times q)p + (q \times q \times q \times q)p + \dots$
$P(\text{A gagne}) = p + q^2 p + q^4 p + \dots$
$P(\text{A gagne}) = p \times (1 + q^2 + q^4 + \dots)$
$P(\text{A gagne}) = p \sum_{k=0}^{\infty} (q^2)^k$

C'est une série géométrique de premier terme $p$ et de raison $r = q^2 = (5/6)^2 = 25/36$.
La somme est $\frac{\text{premier terme}}{1 - \text{raison}} = \frac{p}{1 - q^2}$.
$P(\text{A gagne}) = \frac{1/6}{1 - 25/36} = \frac{1/6}{11/36} = \frac{1}{6} \times \frac{36}{11} = 6/11$.
\end{correctionbox}

\subsection{Exercices Python}

Ces exercices appliquent les concepts de probabilité conditionnelle, de la règle de Bayes et de la formule des probabilités totales au jeu de données "Titanic".

\begin{codecell}
import pandas as pd
import seaborn as sns
import math

# Charger le dataset Titanic
df = sns.load_dataset("titanic")

# On retire les lignes ou l'age est inconnu pour simplifier les calculs
# C'est notre Univers S.
df = df.dropna(subset=["age"]) 

\end{codecell}

\begin{exercicebox}[Exercice 1 : Définition de $P(A|B)$]
Calculez la probabilité qu'un passager ait survécu ($A$), \textbf{sachant que} ce passager était en première classe ($B$).

\textbf{Votre tâche :}
\begin{enumerate}
    \item Soit $A$ = "le passager a survécu" (\texttt{survived == 1}).
    \item Soit $B$ = "le passager est en première classe" (\texttt{pclass == 1}).
    \item Trouver $|B|$ (le nombre de passagers en 1ère classe).
    \item Trouver $|A \cap B|$ (le nombre de survivants de 1ère classe).
    \item Calculer $P(A|B) = \frac{|A \cap B|}{|B|}$.
\end{enumerate}
\end{exercicebox}

\begin{exercicebox}[Exercice 2 : Règle du Produit (Tirage sans remise)]
On tire au hasard 2 passagers de l'univers \texttt{df} sans remise. Calculez la probabilité que le premier passager soit un survivant ($A_1$) ET que le second passager soit aussi un survivant ($A_2$).

Utilisez la Règle du Produit : $P(A_1 \cap A_2) = P(A_1)P(A_2|A_1)$.

\textbf{Votre tâche :}
\begin{enumerate}
    \item Trouver $|S|$ (total passagers) et $|A_1|$ (total survivants).
    \item Calculer $P(A_1) = |A_1| / |S|$.
    \item Calculer $P(A_2|A_1)$. (Indice : après avoir tiré un survivant, combien de passagers restent ? Combien de survivants restent ?).
    \item Calculer le produit $P(A_1) \times P(A_2|A_1)$.
\end{enumerate}
\end{exercicebox}

\begin{exercicebox}[Exercice 3 : Formule des Probabilités Totales]
Calculez la probabilité totale qu'un passager ait survécu ($B$) en utilisant la formule des probabilités totales. Utilisez la partition des trois classes de passagers ($A_1$=1ère, $A_2$=2e, $A_3$=3e classe).

La formule est : $P(B) = \sum_{i=1}^{3} P(B|A_i)P(A_i)$.

\textbf{Votre tâche :}
\begin{enumerate}
    \item Pour $i=1, 2, 3$, calculer $P(A_i)$, la probabilité d'appartenir à chaque classe (ex: $P(A_1) = |\text{pclass 1}| / |S|$).
    \item Pour $i=1, 2, 3$, calculer $P(B|A_i)$, la probabilité de survie sachant la classe (cf. Exercice 1).
    \item Appliquer la formule : $P(B) = P(B|A_1)P(A_1) + P(B|A_2)P(A_2) + P(B|A_3)P(A_3)$.
    \item (Vérification) Comparez votre résultat au calcul direct $P(B) = |\text{survivants}| / |S|$.
\end{enumerate}
\end{exercicebox}

\begin{exercicebox}[Exercice 4 : Règle de Bayes]
En utilisant les résultats de l'exercice précédent, appliquez la Règle de Bayes.

On observe qu'un passager a survécu ($B$). Quelle est la probabilité qu'il s'agisse d'un passager de première classe ($A_1$) ?

\textbf{Votre tâche :}
\begin{enumerate}
    \item On cherche $P(A_1|B) = \frac{P(B|A_1)P(A_1)}{P(B)}$.
    \item Récupérer $P(B|A_1)$ (la probabilité de survie en 1ère classe) de l'exercice 3.
    \item Récupérer $P(A_1)$ (la probabilité d'être en 1ère classe) de l'exercice 3.
    \item Récupérer $P(B)$ (la probabilité totale de survie) de l'exercice 3.
    \item Effectuer le calcul.
\end{enumerate}
\end{exercicebox}

\begin{exercicebox}[Exercice 5 : Indépendance de deux événements]
Les événements $A$ = "être une femme" (\texttt{sex == 'female'}) et $B$ = "avoir survécu" (\texttt{survived == 1}) sont-ils indépendants ?

\textbf{Votre tâche :}
\begin{enumerate}
    \item Prouver ou réfuter l'indépendance en vérifiant si $P(A \cap B) = P(A)P(B)$.
    \item Calculer $P(A) = |\text{femmes}| / |S|$.
    \item Calculer $P(B) = |\text{survivants}| / |S|$.
    \item Calculer $P(A \cap B) = |\text{femmes survivantes}| / |S|$.
    \item Comparer $P(A \cap B)$ au produit $P(A) \times P(B)$.
    \item (Alternative) Comparer $P(B|A)$ à $P(B)$. L'information "être une femme" change-t-elle la probabilité de survie ?
\end{enumerate}
\end{exercicebox}
\newpage

\section{Probabilité conditionnelle}

\begin{intuitionbox}[Question Fondamentale]
La probabilité conditionnelle est le concept qui répond à la question fondamentale : comment devons-nous mettre à jour nos croyances à la lumière des nouvelles informations que nous observons ?
\end{intuitionbox}

\subsection{Définition de la Probabilité Conditionnelle}

\begin{definitionbox}[Probabilité Conditionnelle]
Si $A$ et $B$ sont deux événements avec $P(B) > 0$, alors la probabilité conditionnelle de $A$ sachant $B$, notée $P(A|B)$, est définie comme :
$$P(A|B) = \frac{P(A \cap B)}{P(B)}$$
\end{definitionbox}

\begin{intuitionbox}
Imaginez que l'ensemble de tous les résultats possibles est un grand terrain. Savoir que l'événement $B$ s'est produit, c'est comme si on vous disait que le résultat se trouve dans une zone spécifique de ce terrain. La probabilité conditionnelle $P(A|B)$ ne s'intéresse plus au terrain entier, mais seulement à la proportion de la zone $B$ qui est également occupée par $A$. On "zoome" sur le monde où $B$ est vrai, et on recalcule les probabilités dans ce nouveau monde plus petit.
\end{intuitionbox}

\subsection{Règle du Produit (Intersection de deux événements)}

\begin{theorembox}[Probabilité de l'intersection de deux événements]
Pour tous événements $A$ et $B$ avec des probabilités positives, nous avons :
$$P(A \cap B) = P(A)P(B|A) = P(B)P(A|B)$$
Cela découle directement de la définition de la probabilité conditionnelle.
\end{theorembox}

\begin{intuitionbox}
Pour que deux événements se produisent, le premier doit se produire, PUIS le second doit se produire, sachant que le premier a eu lieu. Cette formule exprime mathématiquement cette idée séquentielle.
\end{intuitionbox}

\begin{examplebox}
Quelle est la probabilité de tirer deux As d'un jeu de 52 cartes sans remise ?
Soit $A$ l'événement "le premier tirage est un As", avec $P(A) = \frac{4}{52}$. Soit $B$ l'événement "le deuxième tirage est un As". Nous cherchons $P(A \cap B)$, que l'on calcule avec la formule $P(A \cap B) = P(A) \times P(B|A)$. La probabilité $P(B|A)$ correspond à tirer un As sachant que la première carte était un As. Il reste alors 51 cartes, dont 3 As. Donc, $P(B|A) = \frac{3}{51}$. Finalement, la probabilité de l'intersection est $P(A \cap B) = \frac{4}{52} \times \frac{3}{51} = \frac{12}{2652} \approx 0.0045$.
\end{examplebox}

\subsection{Règle de la Chaîne (Intersection de n événements)}

\begin{theorembox}[Probabilité de l'intersection de n événements]
Pour tous événements $A_1, \dots, A_n$ avec $P(A_1 \cap A_2 \cap \dots \cap A_{n-1}) > 0$, nous avons :
$$P(A_1 \cap \dots \cap A_n) = P(A_1)P(A_2|A_1)P(A_3|A_1 \cap A_2) \cdots P(A_n|A_1 \cap \dots \cap A_{n-1})$$
\end{theorembox}

\begin{intuitionbox}
Ceci est une généralisation de l'idée précédente, souvent appelée "règle de la chaîne" (chain rule). Pour qu'une séquence d'événements se produise, chaque événement doit se réaliser tour à tour, en tenant compte de tous les événements précédents qui se sont déjà produits.
\end{intuitionbox}

\begin{examplebox}
On tire 3 cartes sans remise. Quelle est la probabilité d'obtenir la séquence Roi, Dame, Valet ?
La probabilité de tirer un Roi en premier ($A_1$) est $P(A_1) = \frac{4}{52}$.
Ensuite, la probabilité de tirer une Dame ($A_2$) sachant qu'un Roi a été tiré est $P(A_2|A_1) = \frac{4}{51}$.
Enfin, la probabilité de tirer un Valet ($A_3$) sachant qu'un Roi et une Dame ont été tirés est $P(A_3|A_1 \cap A_2) = \frac{4}{50}$.
La probabilité totale de la séquence est donc le produit de ces probabilités : $P(A_1 \cap A_2 \cap A_3) = \frac{4}{52} \times \frac{4}{51} \times \frac{4}{50} \approx 0.00048$.
\end{examplebox}

\subsection{Règle de Bayes}

\begin{theorembox}[Règle de Bayes]
$$P(A|B) = \frac{P(B|A)P(A)}{P(B)}$$
\end{theorembox}

\begin{intuitionbox}
La règle de Bayes est la formule pour "inverser" une probabilité conditionnelle. Souvent, il est facile de connaître la probabilité d'un effet étant donné une cause ($P(\text{symptôme}|\text{maladie})$), mais ce qui nous intéresse vraiment, c'est la probabilité de la cause étant donné l'effet observé ($P(\text{maladie}|\text{symptôme})$). La règle de Bayes nous permet de faire ce retournement en utilisant notre connaissance initiale de la probabilité de la cause ($P(\text{maladie})$). C'est le fondement mathématique de la mise à jour de nos croyances.
\end{intuitionbox}

\begin{examplebox}[Dépistage médical]
Une maladie touche 1\% de la population ($P(M) = 0.01$). Un test de dépistage est fiable à 95\% : il est positif pour 95\% des malades ($P(T|M)=0.95$) et négatif pour 95\% des non-malades, ce qui implique un taux de faux positifs de $P(T|\neg M) = 0.05$.
Une personne est testée positive. Quelle est la probabilité qu'elle soit réellement malade, $P(M|T)$ ?
On cherche $P(M|T) = \frac{P(T|M)P(M)}{P(T)}$.
D'abord, on calcule $P(T)$ avec la formule des probabilités totales :
$P(T) = P(T|M)P(M) + P(T|\neg M)P(\neg M) = (0.95 \times 0.01) + (0.05 \times 0.99) = 0.0095 + 0.0495 = 0.059$.
Ensuite, on applique la règle de Bayes : $P(M|T) = \frac{0.95 \times 0.01}{0.059} \approx 0.161$.
Malgré un test positif, il n'y a que 16.1\% de chance que la personne soit malade.
\end{examplebox}

\subsection{Formule des Probabilités Totales}

\begin{theorembox}[Formule des probabilités totales]
Soit $A_1, \dots, A_n$ une partition de l'espace échantillon $S$ (c'est-à-dire que les $A_i$ sont des événements disjoints et leur union est $S$), avec $P(A_i) > 0$ pour tout $i$. Alors pour tout événement $B$ :
$$P(B) = \sum_{i=1}^{n} P(B|A_i)P(A_i)$$
\end{theorembox}

\begin{intuitionbox}
C'est une stratégie de "diviser pour régner". Pour calculer la probabilité totale d'un événement $B$, on peut décomposer le monde en plusieurs scénarios mutuellement exclusifs (la partition $A_i$). On calcule ensuite la probabilité de $B$ dans chacun de ces scénarios ($P(B|A_i)$), on pondère chaque résultat par la probabilité du scénario en question ($P(A_i)$), et on additionne le tout.

\begin{center}
\begin{tikzpicture}
% 1. Dessiner le grand rectangle et les lignes verticales de partition
\draw (0,0) rectangle (12,7);

% 3. Dessiner une grande ellipse pour la forme B
\filldraw[
    fill=gray!30, % Remplissage gris clair
    thick % Trait épais pour le contour
] (6, 3.5) ellipse (5.5cm and 2.5cm); % Centre (6,3.5), rayon x=5.5cm, rayon y=2.5cm

\foreach \x in {2,4,6,8,10} {
    \draw (\x,0) -- (\x,7);
}

% 2. Placer les étiquettes A_1, A_2, ... en bas
\foreach \i [evaluate=\i as \xpos using \i*2-1] in {1,...,6} {
    \node at (\xpos, -0.5) {$A_{\i}$};
}

% 4. Placer l'étiquette pour l'ensemble B
\node at (11, 6) {$B$}; % Ajusté pour être au-dessus de l'ellipse

% 5. Placer les étiquettes pour les intersections B ∩ A_i, toutes au même niveau
\node at (1.2, 3.5) {$B \cap A_1$};
\node at (3, 3.5) {$B \cap A_2$};
\node at (5, 3.5) {$B \cap A_3$};
\node at (7, 3.5) {$B \cap A_4$};
\node at (9, 3.5) {$B \cap A_5$};
\node at (10.8, 3.5) {$B \cap A_6$};
\end{tikzpicture}
\end{center}
\end{intuitionbox}

\begin{examplebox}
Une usine possède trois machines, M1, M2, et M3, qui produisent respectivement 50\%, 30\% et 20\% des articles. Leurs taux de production défectueuse sont de 4\%, 2\% et 5\%. Quelle est la probabilité qu'un article choisi au hasard soit défectueux ?
Soit $D$ l'événement "l'article est défectueux". Les machines forment une partition avec $P(M1)=0.5$, $P(M2)=0.3$, et $P(M3)=0.2$. Les probabilités conditionnelles de défaut sont $P(D|M1)=0.04$, $P(D|M2)=0.02$, et $P(D|M3)=0.05$.
En appliquant la formule, on obtient :
$P(D) = P(D|M1)P(M1) + P(D|M2)P(M2) + P(D|M3)P(M3) = (0.04 \times 0.5) + (0.02 \times 0.3) + (0.05 \times 0.2) = 0.02 + 0.006 + 0.01 = 0.036$.
La probabilité qu'un article soit défectueux est de 3.6\%.
\end{examplebox}

\begin{proofbox}[Démonstration de la formule des probabilités totales]
Puisque les $A_i$ forment une partition de $S$, on peut décomposer $B$ comme :
$$B = (B \cap A_1) \cup (B \cap A_2) \cup \cdots \cup (B \cap A_n)$$
Comme les $A_i$ sont disjoints, les événements $(B \cap A_i)$ le sont aussi. On peut donc sommer leurs probabilités :
$$P(B) = P(B \cap A_1) + P(B \cap A_2) + \cdots + P(B \cap A_n)$$
En appliquant le théorème de l'intersection des probabilités à chaque terme, on obtient :
$$P(B) = P(B|A_1)P(A_1) + P(B|A_2)P(A_2) + \cdots + P(B|A_n) = \sum_{i=1}^{n} P(B|A_i)P(A_i)$$
\end{proofbox}

\subsection{Règle de Bayes avec Conditionnement Additionnel}

\begin{theorembox}[Règle de Bayes avec conditionnement additionnel]
À condition que $P(A \cap E) > 0$ et $P(B \cap E) > 0$, nous avons :
$$P(A|B, E) = \frac{P(B|A, E)P(A|E)}{P(B|E)}$$
\end{theorembox}

\begin{intuitionbox}
Cette formule est simplement la règle de Bayes standard, mais appliquée à l'intérieur d'un univers que l'on a déjà "rétréci".

Imaginez que vous recevez une information \textbf{E} qui élimine une grande partie des possibilités. C'est votre nouveau point de départ, votre monde est plus petit. Toutes les probabilités que vous calculez désormais sont relatives à ce monde restreint.

Dans ce nouveau monde, vous recevez une autre information, l'évidence \textbf{B}. La règle de Bayes conditionnelle vous permet alors de mettre à jour votre croyance sur un événement \textbf{A}, en utilisant exactement la même logique que la règle de Bayes classique, mais en vous assurant que chaque calcul reste confiné à l'intérieur des frontières de l'univers défini par \textbf{E}.
\end{intuitionbox}

\subsection{Formule des Probabilités Totales avec Conditionnement Additionnel}

\begin{theorembox}[Formule des probabilités totales avec conditionnement additionnel]
Soit $A_1, \dots, A_n$ une partition de $S$. À condition que $P(A_i \cap E) > 0$ pour tout $i$, nous avons :
$$P(B|E) = \sum_{i=1}^{n} P(B|A_i, E)P(A_i|E)$$
\end{theorembox}

\begin{intuitionbox}
\begin{center}
\begin{tikzpicture}
  % Matrice principale, nommée "m"
  \matrix (m) [
    matrix of nodes,
    row sep = -\pgflinewidth,
    column sep = -\pgflinewidth,
    nodes={
      rectangle, draw=black, anchor=center,
      text height=4ex, text depth=0.5ex, minimum width=4em, fill=intuitionColor!10
    }
  ]
  {
    | |              & | |              & |[red_hatch]|    & | |              & | |              & | |            \\
    |[red_hatch]|    & |[purple_hatch]| & |[purple_hatch]| & | |              & |[red_hatch]|    & |[red_hatch]|  \\
    |[red_hatch]|    & |[blue_hatch]|   & |[red_hatch]|    & |[red_hatch]|    & |[red_hatch]|    & | |            \\
  };

  % --- DÉLIMITATION DES COLONNES AVEC ACCOLADES ---
  \draw [decorate, decoration={brace, amplitude=5pt, raise=4mm}]
    (m-1-1.north west) -- (m-1-2.north east) 
    node [midway, yshift=8mm, font=\bfseries] {A1};
    
  \draw [decorate, decoration={brace, amplitude=5pt, raise=4mm}]
    (m-1-3.north west) -- (m-1-4.north east) 
    node [midway, yshift=8mm, font=\bfseries] {A2};
    
  \draw [decorate, decoration={brace, amplitude=5pt, raise=4mm}]
    (m-1-5.north west) -- (m-1-6.north east) 
    node [midway, yshift=8mm, font=\bfseries] {A3};
\end{tikzpicture}
\end{center}
Imaginez que le graphique ci-dessus représente la carte d'un trésor. La carte est partitionnée en trois grandes régions : \textbf{A1}, \textbf{A2}, et \textbf{A3}. Sur cette carte, on a identifié deux types de terrains : une \textbf{zone marécageuse} (événement E, hachures rouges) qui s'étend sur \textbf{10 parcelles}, et une \textbf{zone près d'un vieux chêne} (événement B, hachures bleues) qui couvre \textbf{3 parcelles}.

On vous donne un premier indice : "Le trésor est dans la zone marécageuse (E)". Votre univers de recherche se réduit instantanément à ces 10 parcelles rouges. Puis, on vous donne un second indice : "Le trésor est aussi près d'un chêne (B)". Votre recherche se concentre alors sur les parcelles qui sont à la fois marécageuses et proches d'un chêne (les cases violettes, $B \cap E$).

La question est : "Sachant que le trésor est dans une parcelle violette, quelle est la probabilité qu'il se trouve dans la région A2 ?". On cherche donc $P(A_2 | B, E)$. La règle de Bayes nous permet de le calculer.

\textbf{Calcul des termes nécessaires :} D'abord, nous devons évaluer les probabilités à l'intérieur du "monde marécageux" (sachant E).

La \textbf{vraisemblance} est $P(B|A_2, E)$. En se limitant aux 4 parcelles marécageuses de la région A2, une seule est aussi près d'un chêne. Donc, $P(B|A_2, E) = 1/4$.

La \textbf{probabilité a priori} est $P(A_2|E)$. Sur les 10 parcelles marécageuses, 4 sont dans la région A2. Donc, $P(A_2|E) = 4/10$.

L'\textbf{évidence}, $P(B|E)$, est la probabilité de trouver un chêne dans l'ensemble de la zone marécageuse. On peut la calculer avec la formule des probabilités totales :
$$P(B|E) = P(B|A_1, E)P(A_1|E) + P(B|A_2, E)P(A_2|E) + P(B|A_3, E)P(A_3|E)$$
$$P(B|E) = (\frac{1}{3} \times \frac{3}{10}) + (\frac{1}{4} \times \frac{4}{10}) + (0 \times \frac{3}{10}) = \frac{1}{10} + \frac{1}{10} = \frac{2}{10}$$

\textbf{Application de la règle de Bayes :} Maintenant, nous assemblons le tout.
$$P(A_2|B, E) = \frac{P(B|A_2, E)P(A_2|E)}{P(B|E)} = \frac{(1/4) \times (4/10)}{2/10} = \frac{1/10}{2/10} = \frac{1}{2}$$
L'intuition confirme le calcul : sachant que le trésor est sur une parcelle violette, et qu'il n'y en a que deux (une en A1, une en A2), il y a bien une chance sur deux qu'il se trouve dans la région A2.
\end{intuitionbox}

\subsection{Indépendance de Deux Événements}

\begin{definitionbox}[Indépendance de deux événements]
Les événements $A$ et $B$ sont indépendants si :
$$P(A \cap B) = P(A)P(B)$$
Si $P(A) > 0$ et $P(B) > 0$, cela est équivalent à :
$$P(A|B) = P(A)$$
\end{definitionbox}

\begin{intuitionbox}
L'indépendance est l'absence d'information. Si deux événements sont indépendants, apprendre que l'un s'est produit ne change absolument rien à la probabilité de l'autre. Savoir qu'il pleut à Tokyo ($B$) ne modifie pas la probabilité que vous obteniez pile en lançant une pièce ($A$).
\end{intuitionbox}

\subsection{Indépendance Conditionnelle}

\begin{definitionbox}[Indépendance Conditionnelle]
Les événements $A$ et $B$ sont dits conditionnellement indépendants étant donné $E$ si :
$$P(A \cap B | E) = P(A|E)P(B|E)$$
\end{definitionbox}

\begin{intuitionbox}
L'indépendance peut apparaître ou disparaître quand on observe un autre événement. Par exemple, vos notes en maths ($A$) et en physique ($B$) ne sont probablement pas indépendantes. Mais si l'on sait que vous avez beaucoup travaillé ($E$), alors vos notes en maths et en physique pourraient devenir indépendantes. L'information "vous avez beaucoup travaillé" explique la corrélation ; une fois qu'on la connaît, connaître votre note en maths n'apporte plus d'information sur votre note en physique.
\end{intuitionbox}

\subsection{Le Problème de Monty Hall}

\begin{remarquebox}[Le problème de Monty Hall]
Imaginez que vous êtes à un jeu télévisé. Face à vous se trouvent trois portes fermées. Derrière l'une d'elles se trouve une voiture, et derrière les deux autres, des chèvres.
\begin{enumerate}
    \item Vous choisissez une porte (disons, la porte n°1).
    \item L'animateur, qui sait où se trouve la voiture, ouvre une autre porte (par exemple, la n°3) derrière laquelle se trouve une chèvre.
    \item Il vous demande alors : "Voulez-vous conserver votre choix initial (porte n°1) ou changer pour l'autre porte restante (la n°2) ?"
\end{enumerate}
\textbf{Question :} Avez-vous intérêt à changer de porte ? Votre probabilité de gagner la voiture est-elle plus grande si vous changez, si vous ne changez pas, ou est-elle la même dans les deux cas ?
\end{remarquebox}

\begin{correctionbox}[Solution du problème de Monty Hall]
La réponse est sans équivoque : il faut \textbf{toujours changer de porte}. Cette stratégie fait passer la probabilité de gagner de $1/3$ à $2/3$. L'intuition et la preuve ci-dessous détaillent ce résultat surprenant.
\end{correctionbox}

\begin{intuitionbox}[Le secret : l'information de l'animateur]
L'erreur commune est de supposer qu'il reste deux portes avec une chance égale de $1/2$. Cela ignore une information capitale : le choix de l'animateur n'est \textbf{pas aléatoire}. Il sait où se trouve la voiture et ouvrira toujours une porte perdante.

Le raisonnement correct se déroule en deux temps. D'abord, votre choix initial a $\mathbf{1/3}$ de chance d'être correct. Cela implique qu'il y a $\mathbf{2/3}$ de chance que la voiture soit derrière l'une des \textit{deux autres portes}. Ensuite, lorsque l'animateur ouvre l'une de ces deux portes, il ne fait que vous montrer où la voiture n'est \textit{pas} dans cet ensemble. La probabilité de $2/3$ se \textbf{concentre} alors entièrement sur la seule porte qu'il a laissée fermée. Changer de porte revient à miser sur cette probabilité de $2/3$.
\end{intuitionbox}

\begin{proofbox}[Preuve par l'arbre de décision]
L'analyse de la meilleure stratégie peut être visualisée à l'aide de l'arbre de décision ci-dessous. Il décompose le problème en deux scénarios initiaux : avoir choisi la bonne porte (probabilité $1/3$) ou une mauvaise porte (probabilité $2/3$).

\vspace{0.5cm}
\begin{center}
\begin{tikzpicture}[
  grow=right,
  level distance=4.5cm,
  level 1/.style={sibling distance=3cm},
  level 2/.style={sibling distance=2.5cm},
  edge from parent/.style={draw, -latex},
  % --- Définition des styles pour les cadres ---
  porte_style/.style={rectangle, rounded corners, draw=black, fill=gray!20, thick, inner sep=4pt, text width=2.5cm, align=center},
  gain_style/.style={rectangle, rounded corners, draw=green!60!black, fill=green!20, thick, inner sep=4pt},
  perte_style/.style={rectangle, rounded corners, draw=red!60!black, fill=red!20, thick, inner sep=4pt}
]

\node {S}
    % --- Branche du haut ---
    child {
        node[porte_style] {Bonne porte}
        child {
            node[gain_style] {Gain}
            edge from parent
            node[above, sloped] {$1/2$}
        }
        child {
            node[perte_style] {Perte}
            edge from parent
            node[below, sloped] {$1/2$}
        }
        edge from parent
        node[above, sloped] {1/3}
    }
    % --- Branche du bas ---
    child {
        node[porte_style] {Mauvaise porte}
        child {
            node[gain_style] {Gain}
            edge from parent
            node[above, sloped] {1}
        }
        edge from parent
        node[below, sloped] {2/3}
    };
\end{tikzpicture}
\end{center}
\vspace{0.5cm}

\noindent\textbf{Analyse de l'arbre :}

\vspace{0.3cm}
\noindent\textbf{Branche du bas (cas le plus probable) :}
\newline
Avec une probabilité de $\mathbf{2/3}$, votre choix initial se porte sur une "Mauvaise porte". L'animateur est alors obligé de révéler l'autre porte perdante. La seule porte restante est donc la bonne. L'arbre montre que cela mène à un "Gain" avec une probabilité de $\mathbf{1}$. Ce chemin correspond au résultat de la stratégie \textbf{"Changer"}.

\vspace{0.3cm}
\noindent\textbf{Branche du haut (cas le moins probable) :}
\newline
Avec une probabilité de $\mathbf{1/3}$, vous avez choisi la "Bonne porte" du premier coup. L'arbre se divise alors en deux issues équiprobables ($1/2$ chacune). L'issue "Gain" correspond à la stratégie \textbf{"Garder"} votre choix initial, tandis que l'issue "Perte" correspond à la stratégie \textbf{"Changer"} pour la porte perdante restante.

\vspace{0.3cm}
\noindent\textbf{Conclusion :}
\newline
Pour évaluer la meilleure stratégie, il suffit de sommer les probabilités de gain. La \textbf{probabilité de gain en changeant} est de $\mathbf{2/3}$, car vous gagnez uniquement si votre choix initial était mauvais (branche du bas). La \textbf{probabilité de gain en gardant} est de $\mathbf{1/3}$, car vous gagnez uniquement si votre choix initial était bon (branche "Gain" du haut). La stratégie optimale est donc bien de toujours changer de porte.
\end{proofbox}
\subsection{Exercices}

% --- Lois Binomiale et Bernoulli ---

\begin{exercicebox}[Exercice 1 : Loi Binomiale (Quiz)]
Un étudiant répond au hasard à un QCM de 10 questions. Chaque question a 4 choix de réponse, dont un seul est correct. Soit $X$ le nombre de bonnes réponses.
\begin{enumerate}
    \item Quelle loi suit $X$ ? Précisez ses paramètres.
    \item Quelle est la probabilité que l'étudiant ait exactement 5 bonnes réponses ?
    \item Quelle est la probabilité que l'étudiant ait au moins une bonne réponse ?
\end{enumerate}
\end{exercicebox}

\begin{exercicebox}[Exercice 2 : Loi Binomiale (Contrôle Qualité)]
Une usine produit des ampoules. 5\% des ampoules sont défectueuses. On prélève un lot de 20 ampoules. Soit $X$ le nombre d'ampoules défectueuses dans le lot.
\begin{enumerate}
    \item Quelle loi suit $X$ ? (On suppose le prélèvement "avec remise" ou d'une production très grande).
    \item Quelle est la probabilité qu'il n'y ait aucune ampoule défectueuse ?
    \item Quelle est la probabilité qu'il y ait exactement deux ampoules défectueuses ?
\end{enumerate}
\end{exercicebox}

\begin{exercicebox}[Exercice 3 : Espérance et Variance (Binomiale)]
Un archer touche la cible avec une probabilité $p=0.8$ à chaque tir. Il tire $n=40$ flèches. Soit $X$ le nombre de tirs réussis.
\begin{enumerate}
    \item Calculer l'espérance $E(X)$.
    \item Calculer la variance $\text{Var}(X)$ et l'écart-type $\text{SD}(X)$.
\end{enumerate}
\end{exercicebox}

\begin{exercicebox}[Exercice 4 : Loi de Bernoulli (Indicatrice)]
Soit $A$ un événement avec $P(A) = p$. Soit $I_A$ la variable indicatrice de $A$.
\begin{enumerate}
    \item Écrire la PMF de $I_A$.
    \item Calculer $E(I_A)$.
    \item Calculer $\text{Var}(I_A)$ en utilisant $\text{Var}(X) = E(X^2) - [E(X)]^2$. (Indice : $I_A^2 = I_A$).
\end{enumerate}
\end{exercicebox}

% --- Loi de Poisson ---

\begin{exercicebox}[Exercice 5 : Loi de Poisson (Emails)]
Un serveur de messagerie reçoit en moyenne 2 emails "spam" par minute. Soit $X$ le nombre de spams reçus en une minute.
\begin{enumerate}
    \item Quelle loi suit $X$ ?
    \item Quelle est la probabilité de recevoir exactement 3 spams en une minute ?
    \item Quelle est la probabilité de recevoir au plus 2 spams en une minute ?
\end{enumerate}
\end{exercicebox}

\begin{exercicebox}[Exercice 6 : Loi de Poisson (Échelle de temps)]
Une substance radioactive émet en moyenne $\lambda=4$ particules par seconde. Soit $Y$ le nombre de particules émises en 3 secondes.
\begin{enumerate}
    \item Quelle est la loi de $Y$ ? (Indice : ajuster le paramètre $\lambda$).
    \item Quelle est la probabilité que $Y=10$ ?
\end{enumerate}
\end{exercicebox}

\begin{exercicebox}[Exercice 7 : Approximation de Poisson (Binomiale)]
Un livre de 500 pages contient 1000 fautes de frappe distribuées au hasard. Soit $X$ le nombre de fautes de frappe sur une page donnée.
\begin{enumerate}
    \item Quelle est la loi exacte de $X$ ? (On suppose qu'une faute ne peut pas être à cheval sur deux pages).
    \item Par quelle loi peut-on approximer $X$ ? Précisez le paramètre.
    \item En utilisant l'approximation, calculez la probabilité qu'une page choisie au hasard contienne au moins une faute.
\end{enumerate}
\end{exercicebox}

% --- Lois Géométrique et Hypergéométrique ---

\begin{exercicebox}[Exercice 8 : Loi Géométrique (Échecs avant succès)]
On lance un dé équilibré jusqu'à obtenir un 6. Soit $X$ le nombre d'échecs (lancers qui ne sont pas 6) avant d'obtenir le premier 6.
\begin{enumerate}
    \item Quelle loi suit $X$ ? Précisez le paramètre $p$.
    \item Quelle est la probabilité d'échouer exactement 3 fois ? (c-à-d, le 6 arrive au 4ème lancer).
    \item Quelle est l'espérance du nombre d'échecs $E(X)$ ?
\end{enumerate}
\end{exercicebox}

\begin{exercicebox}[Exercice 9 : Loi Géométrique (Propriété)]
Soit $X \sim \text{Geom}(p)$ (comptant les échecs).
Quelle est la probabilité $P(X \ge k)$ ? C'est-à-dire, la probabilité d'avoir au moins $k$ échecs.
\end{exercicebox}

\begin{exercicebox}[Exercice 10 : Loi Hypergéométrique (Comité)]
Un club est composé de 12 hommes et 8 femmes. On choisit un comité de 5 personnes au hasard. Soit $X$ le nombre de femmes dans le comité.
\begin{enumerate}
    \item Quelle loi suit $X$ ? Précisez les paramètres.
    \item Quelle est la probabilité que le comité soit composé d'exactement 2 femmes ?
\end{enumerate}
\end{exercicebox}

\begin{exercicebox}[Exercice 11 : Loi Hypergéométrique (Pêche)]
Un lac contient 100 poissons, dont 10 ont été marqués. Un pêcheur attrape 8 poissons (sans remise). Soit $X$ le nombre de poissons marqués parmi les 8 attrapés.
\begin{enumerate}
    \item Quelle est la loi de $X$ ?
    \item Quelle est la probabilité qu'il n'attrape aucun poisson marqué ?
\end{enumerate}
\end{exercicebox}

\begin{exercicebox}[Exercice 12 : Binomiale vs Hypergéométrique]
Expliquez la différence fondamentale entre la loi Binomiale et la loi Hypergéométrique. Dans quel cas la loi Binomiale est-elle une bonne approximation de la loi Hypergéométrique ?
\end{exercicebox}

% --- PMF, CDF, Espérance Générale, LOTUS, Variance ---

\begin{exercicebox}[Exercice 13 : PMF (Trouver la constante)]
Soit $X$ une variable aléatoire discrète dont la PMF est donnée par $P(X=k) = c \cdot k^2$ pour $k \in \{1, 2, 3\}$. Pour toutes les autres valeurs, $P(X=x)=0$.
\begin{enumerate}
    \item Trouvez la valeur de la constante $c$.
    \item Calculez $P(X \ge 2)$.
\end{enumerate}
\end{exercicebox}

\begin{exercicebox}[Exercice 14 : Espérance (Jeu de Hasard)]
Un joueur paie 5€ pour jouer à un jeu. Il lance deux dés. Il gagne $S$ euros, où $S$ est la somme des deux dés. Soit $G$ son gain net (gain - mise).
\begin{enumerate}
    \item Rappeler $E(S)$ (espérance de la somme de two dés).
    \item Calculer $E(G)$. Le jeu est-il équitable ?
\end{enumerate}
\end{exercicebox}

\begin{exercicebox}[Exercice 15 : LOTUS (Jeu de Hasard 2)]
On lance un dé équilibré. Soit $X$ le résultat. Vous gagnez $g(X) = (X-3)^2$ euros.
\begin{enumerate}
    \item Calculez l'espérance de votre gain, $E[g(X)]$.
\end{enumerate}
\end{exercicebox}

\begin{exercicebox}[Exercice 16 : CDF (Fonction de Répartition)]
Soit $X$ une variable aléatoire avec la PMF suivante :
$P(X=0) = 0.2$, $P(X=1) = 0.5$, $P(X=2) = 0.3$.
\begin{enumerate}
    \item Écrivez la fonction de répartition $F_X(x) = P(X \le x)$.
    \item Calculez $P(0 < X \le 2)$.
    \item Calculez $P(X > 1)$.
\end{enumerate}
\end{exercicebox}

\begin{exercicebox}[Exercice 17 : Variance (Calcul)]
Pour la variable $X$ de l'exercice 16 :
\begin{enumerate}
    \item Calculez $E(X)$.
    \item Calculez $E(X^2)$ en utilisant LOTUS.
    \item Calculez $\text{Var}(X)$ en utilisant la formule $E(X^2) - [E(X)]^2$.
\end{enumerate}
\end{exercicebox}

\begin{exercicebox}[Exercice 18 : Propriétés de la Variance]
Soit $X$ une variable aléatoire avec $E(X)=10$ et $\text{Var}(X)=4$. Soit $Y = 5X - 2$.
\begin{enumerate}
    \item Calculez $E(Y)$.
    \item Calculez $\text{Var}(Y)$.
\end{enumerate}
\end{exercicebox}

% --- Linéarité et Indicateurs ---

\begin{exercicebox}[Exercice 19 : Linéarité de l'Espérance (Mélange)]
On lance un dé (résultat $D$) et une pièce (résultat $P$, 1 pour Pile, 0 for Face).
Soit $X = D + 3P$.
Calculez $E(X)$ en utilisant la linéarité de l'espérance.
\end{exercicebox}

\begin{exercicebox}[Exercice 20 : Indicateurs (Problème des Chapeaux)]
$n$ personnes jettent leur chapeau au centre d'une pièce. Les chapeaux sont mélangés, et chaque personne en reprend un au hasard. Soit $X$ le nombre de personnes qui reprennent leur propre chapeau.
\begin{enumerate}
    \item Définir $X$ comme une somme de $n$ variables indicatrices $I_i$. Que représente $I_i$ ?
    \item Quelle est la probabilité $P(I_i=1)$ ? (C-à-d, la probabilité que la personne $i$ reprenne son chapeau).
    \item Calculez $E(X)$ en utilisant la linéarité.
\end{enumerate}
\end{exercicebox}

\begin{exercicebox}[Exercice 21 : Indicateurs (Collectionneurs)]
On achète $n=5$ boîtes de céréales. Chaque boîte contient une figurine au hasard parmi $k=4$ types de figurines (types A, B, C, D). Soit $X$ le nombre de types de figurines *distincts* que nous avons obtenus.
\begin{enumerate}
    \item Soit $I_A$ l'indicatrice que nous avons obtenu au moins une figurine de type A.
    \item Calculez $P(I_A=0)$. (Probabilité de n'avoir *aucune* figurine A dans les 5 boîtes).
    \item Calculez $E(I_A)$.
    \item Exprimez $X$ en fonction d'indicatrices $I_A, I_B, I_C, I_D$ et calculez $E(X)$.
\end{enumerate}
\end{exercicebox}

\begin{exercicebox}[Exercice 22 : Variance (Propriétés)]
Prouvez que $\text{Var}(aX + b) = a^2 \text{Var}(X)$ pour des constantes $a$ et $b$.
\end{exercicebox}

% --- Exercices de Synthèse ---

\begin{exercicebox}[Exercice 23 : Identifier la Loi]
Pour chaque scénario, identifiez la loi discrète la plus appropriée (Bernoulli, Binomiale, Hypergéométrique, Géométrique, Poisson).
\begin{enumerate}
    \item On compte le nombre de "Face" lors de 20 lancers de pièce.
    \item On compte le nombre d'accidents à une intersection un jour donné (sachant un taux moyen de 1.5/jour).
    \item On compte le nombre de Rois dans une main de 5 cartes tirées d'un jeu de 52 cartes.
    \item On compte le nombre de lancers de dé nécessaires avant d'obtenir le premier 3 (en comptant les échecs).
    \item On vérifie si un seul composant électronique est défectueux ou non.
\end{enumerate}
\end{exercicebox}

\begin{exercicebox}[Exercice 24 : Espérance (Loi Hypergéométrique)]
Soit $X \sim \text{HG}(w, b, m)$ (tirage de $m$ boules parmi $w$ blanches et $b$ noires).
En utilisant des variables indicatrices, montrez que $E(X) = m \left( \frac{w}{w+b} \right)$.
(Indice : Soit $I_j$ l'indicatrice que la $j$-ème boule tirée est blanche, pour $j=1 \dots m$. $X = \sum I_j$. Calculez $E(I_j)$.)
\end{exercicebox}

\begin{exercicebox}[Exercice 25 : Variance (Poisson)]
On admet que si $X \sim \text{Bin}(n, p)$, $\text{Var}(X) = np(1-p)$.
En utilisant l'approximation Poisson $X_n \sim \text{Bin}(n, \lambda/n)$, que devient la variance lorsque $n \to \infty$ ?
\end{exercicebox}



\subsection{Corrections des Exercices}

\begin{correctionbox}[Correction Exercice 1 : Loi Binomiale (Quiz)]
1. $X$ suit une loi Binomiale, car c'est la somme de $n=10$ succès indépendants (bonnes réponses), chacun avec une probabilité $p = 1/4 = 0.25$. $X \sim \text{Bin}(10, 0.25)$.
2. $P(X=5) = \binom{10}{5} (0.25)^5 (1-0.25)^{10-5} = 252 \times (0.25)^5 (0.75)^5 \approx 0.0584$.
3. $P(X \ge 1) = 1 - P(X=0)$.
$P(X=0) = \binom{10}{0} (0.25)^0 (0.75)^{10} = (0.75)^{10} \approx 0.0563$.
$P(X \ge 1) = 1 - 0.0563 = 0.9437$.
\end{correctionbox}

\begin{correctionbox}[Correction Exercice 2 : Loi Binomiale (Contrôle Qualité)]
1. $X \sim \text{Bin}(n=20, p=0.05)$.
2. $P(X=0) = \binom{20}{0} (0.05)^0 (0.95)^{20} = (0.95)^{20} \approx 0.3585$.
3. $P(X=2) = \binom{20}{2} (0.05)^2 (0.95)^{18} = 190 \times (0.0025) \times (0.95)^{18} \approx 0.1887$.
\end{correctionbox}

\begin{correctionbox}[Correction Exercice 3 : Espérance et Variance (Binomiale)]
1. $X \sim \text{Bin}(40, 0.8)$. L'espérance est $E(X) = np = 40 \times 0.8 = 32$. (On s'attend à 32 tirs réussis).
2. $\text{Var}(X) = np(1-p) = 40 \times 0.8 \times 0.2 = 6.4$.
$\text{SD}(X) = \sqrt{\text{Var}(X)} = \sqrt{6.4} \approx 2.53$.
\end{correctionbox}

\begin{correctionbox}[Correction Exercice 4 : Loi de Bernoulli (Indicatrice)]
1. $I_A$ suit la loi de Bernoulli $\text{Bern}(p)$.
$P(I_A=1) = p$ et $P(I_A=0) = 1-p$.
2. $E(I_A) = 1 \cdot P(I_A=1) + 0 \cdot P(I_A=0) = 1 \cdot p + 0 = p$.
3. On utilise LOTUS pour $E(I_A^2)$:
$E(I_A^2) = (1^2) \cdot P(I_A=1) + (0^2) \cdot P(I_A=0) = 1 \cdot p + 0 = p$.
(Note : $I_A^2 = I_A$ car $1^2=1$ et $0^2=0$, donc $E(I_A^2) = E(I_A) = p$).
$\text{Var}(I_A) = E(I_A^2) - [E(I_A)]^2 = p - p^2 = p(1-p)$.
\end{correctionbox}

\begin{correctionbox}[Correction Exercice 5 : Loi de Poisson (Emails)]
1. $X$ suit une loi de Poisson de paramètre $\lambda=2$. $X \sim \text{Poisson}(2)$.
2. $P(X=3) = \frac{e^{-2} 2^3}{3!} = \frac{e^{-2} \times 8}{6} \approx 0.1804$.
3. $P(X \le 2) = P(X=0) + P(X=1) + P(X=2)$
$P(X=0) = \frac{e^{-2} 2^0}{0!} = e^{-2}$
$P(X=1) = \frac{e^{-2} 2^1}{1!} = 2e^{-2}$
$P(X=2) = \frac{e^{-2} 2^2}{2!} = 2e^{-2}$
$P(X \le 2) = e^{-2}(1 + 2 + 2) = 5e^{-2} \approx 0.6767$.
\end{correctionbox}

\begin{correctionbox}[Correction Exercice 6 : Loi de Poisson (Échelle de temps)]
1. Le taux est $\lambda=4$ par seconde. Sur 3 secondes, le taux moyen est $\lambda' = 4 \times 3 = 12$.
$Y \sim \text{Poisson}(12)$.
2. $P(Y=10) = \frac{e^{-12} 12^{10}}{10!} \approx 0.1048$.
\end{correctionbox}

\begin{correctionbox}[Correction Exercice 7 : Approximation de Poisson (Binomiale)]
1. C'est un tirage sans remise (une faute ne peut pas être comptée deux fois). C'est $\text{Hypergéométrique}$. (Ou $\text{Binomiale}$ si on considère que chaque mot a une prob $p$ d'être une faute, mais $n$ (nb de mots) est inconnu).
On peut aussi voir cela comme $n=1000$ essais (fautes) de Bernoulli où le succès est "tomber sur la page X" (prob $p=1/500$). $X \sim \text{Bin}(1000, 1/500)$.
2. La loi Binomiale $\text{Bin}(n=1000, p=1/500)$ a $n$ grand et $p$ petit.
On approxime par Poisson avec $\lambda = np = 1000 \times (1/500) = 2$.
$X \approx \text{Poisson}(2)$.
3. $P(X \ge 1) = 1 - P(X=0) = 1 - \frac{e^{-2} 2^0}{0!} = 1 - e^{-2} \approx 1 - 0.1353 = 0.8647$.
\end{correctionbox}

\begin{correctionbox}[Correction Exercice 8 : Loi Géométrique (Échecs avant succès)]
1. $X$ compte le nombre d'échecs avant le premier succès. Le succès est "obtenir 6", $p=1/6$.
$X \sim \text{Geom}(p=1/6)$.
2. On cherche $P(X=3)$. $P(X=k) = (1-p)^k p$.
$P(X=3) = (5/6)^3 \times (1/6) = \frac{125}{216} \times \frac{1}{6} = \frac{125}{1296} \approx 0.0965$.
3. $E(X) = \frac{q}{p} = \frac{1-p}{p} = \frac{5/6}{1/6} = 5$. (On s'attend à 5 échecs en moyenne).
\end{correctionbox}

\begin{correctionbox}[Correction Exercice 9 : Loi Géométrique (Propriété)]
$P(X \ge k)$ est la probabilité d'avoir au moins $k$ échecs.
Cela signifie que les $k$ premiers essais ont *tous* été des échecs.
La probabilité d'un échec est $q=1-p$. La probabilité de $k$ échecs consécutifs est $q^k$.
(Après ces $k$ échecs, peu importe ce qui se passe, la condition $X \ge k$ est remplie).
Donc, $P(X \ge k) = (1-p)^k$.
\end{correctionbox}

\begin{correctionbox}[Correction Exercice 10 : Loi Hypergéométrique (Comité)]
1. Tirage sans remise. Population totale $w+b = 20$. On s'intéresse aux femmes (disons "blanches", $w=8$). Les hommes sont "noires" ($b=12$). On tire $m=5$ personnes.
$X \sim \text{HG}(w=8, b=12, m=5)$.
2. $P(X=2) = \frac{\binom{w}{k} \binom{b}{m-k}}{\binom{w+b}{m}} = \frac{\binom{8}{2} \binom{12}{3}}{\binom{20}{5}} = \frac{28 \times 220}{15504} = \frac{6160}{15504} \approx 0.3973$.
\end{correctionbox}

\begin{correctionbox}[Correction Exercice 11 : Loi Hypergéométrique (Pêche)]
1. $w=10$ (marqués), $b=90$ (non marqués). $w+b=100$. $m=8$ (tirage).
$X \sim \text{HG}(w=10, b=90, m=8)$.
2. $P(X=0) = \frac{\binom{10}{0} \binom{90}{8}}{\binom{100}{8}} = \frac{1 \times \frac{90!}{8!82!}}{\frac{100!}{8!92!}} = \frac{90! 92!}{82! 100!} = \frac{90 \cdot \dots \cdot 83}{100 \cdot \dots \cdot 93} \approx 0.4166$.
\end{correctionbox}

\begin{correctionbox}[Correction Exercice 12 : Binomiale vs Hypergéométrique]
La différence est l'indépendance des tirages.
\begin{itemize}
    \item \textbf{Binomiale :} Modélise $n$ tirages \textit{avec remise} (ou indépendants). La probabilité de succès $p$ est constante.
    \item \textbf{Hypergéométrique :} Modélise $m$ tirages \textit{sans remise} d'une population finie. La probabilité de succès change à chaque tirage.
\end{itemize}
La Binomiale approxime bien l'Hypergéométrique lorsque la taille de la population ($w+b$) est très grande par rapport à la taille de l'échantillon ($m$). Dans ce cas, $p \approx w/(w+b)$ est presque constant.
\end{correctionbox}

\begin{correctionbox}[Correction Exercice 13 : PMF (Trouver la constante)]
1. La somme des probabilités doit valoir 1 : $\sum P(X=k) = 1$.
$P(X=1) + P(X=2) + P(X=3) = 1$
$c \cdot 1^2 + c \cdot 2^2 + c \cdot 3^2 = 1$
$c(1 + 4 + 9) = 1 \implies 14c = 1 \implies c = 1/14$.
2. $P(X \ge 2) = P(X=2) + P(X=3) = c \cdot 2^2 + c \cdot 3^2 = 4c + 9c = 13c = 13/14$.
\end{correctionbox}

\begin{correctionbox}[Correction Exercice 14 : Espérance (Jeu de Hasard)]
1. $E(S) = E(D_1 + D_2) = E(D_1) + E(D_2) = 3.5 + 3.5 = 7$.
2. $G = S - 5$ (Gain = Somme - Mise).
Par linéarité : $E(G) = E(S - 5) = E(S) - E(5) = E(S) - 5$.
$E(G) = 7 - 5 = 2$.
L'espérance de gain net est de 2€. Le jeu est très en faveur du joueur (et non équitable).
\end{correctionbox}

\begin{correctionbox}[Correction Exercice 15 : LOTUS (Jeu de Hasard 2)]
On cherche $E[g(X)] = E[(X-3)^2]$. On utilise LOTUS : $E[g(X)] = \sum g(x) P(X=x)$.
$E[(X-3)^2] = \sum_{k=1}^6 (k-3)^2 P(X=k) = \sum_{k=1}^6 (k-3)^2 \cdot (1/6)$
$= \frac{1}{6} \left[ (1-3)^2 + (2-3)^2 + (3-3)^2 + (4-3)^2 + (5-3)^2 + (6-3)^2 \right]$
$= \frac{1}{6} \left[ (-2)^2 + (-1)^2 + 0^2 + 1^2 + 2^2 + 3^2 \right]$
$= \frac{1}{6} [ 4 + 1 + 0 + 1 + 4 + 9 ] = \frac{19}{6} \approx 3.167€$.
\end{correctionbox}

\begin{correctionbox}[Correction Exercice 16 : CDF (Fonction de Répartition)]
1. $F_X(x)$ accumule les probabilités :
$F_X(x) = \begin{cases} 
      0 & \text{si } x < 0 \\
      0.2 & \text{si } 0 \le x < 1 \\
      0.2 + 0.5 = 0.7 & \text{si } 1 \le x < 2 \\
      0.7 + 0.3 = 1.0 & \text{si } x \ge 2 
   \end{cases}$
2. $P(0 < X \le 2) = P(X=1) + P(X=2) = 0.5 + 0.3 = 0.8$.
(Alternativement : $F_X(2) - F_X(0) = 1.0 - 0.2 = 0.8$).
3. $P(X > 1) = P(X=2) = 0.3$.
(Alternativement : $1 - P(X \le 1) = 1 - F_X(1) = 1 - 0.7 = 0.3$).
\end{correctionbox}

\begin{correctionbox}[Correction Exercice 17 : Variance (Calcul)]
1. $E(X) = \sum x P(X=x) = (0)(0.2) + (1)(0.5) + (2)(0.3) = 0 + 0.5 + 0.6 = 1.1$.
2. $E(X^2) = \sum x^2 P(X=x) = (0^2)(0.2) + (1^2)(0.5) + (2^2)(0.3) = 0 + 0.5 + (4)(0.3) = 0.5 + 1.2 = 1.7$.
3. $\text{Var}(X) = E(X^2) - [E(X)]^2 = 1.7 - (1.1)^2 = 1.7 - 1.21 = 0.49$.
\end{correctionbox}

\begin{correctionbox}[Correction Exercice 18 : Propriétés de la Variance]
1. $E(Y) = E(5X - 2) = E(5X) - E(2) = 5E(X) - 2 = 5(10) - 2 = 48$.
2. $\text{Var}(Y) = \text{Var}(5X - 2) = \text{Var}(5X) = 5^2 \text{Var}(X) = 25 \times 4 = 100$.
(La constante $b=-2$ n'affecte pas la dispersion).
\end{correctionbox}

\begin{correctionbox}[Correction Exercice 19 : Linéarité de l'Espérance (Mélange)]
On cherche $E(X) = E(D + 3P)$.
Par linéarité : $E(X) = E(D) + E(3P) = E(D) + 3E(P)$.
$E(D) = 3.5$ (espérance d'un dé).
$P$ est une Bernoulli $\text{Bern}(0.5)$. $E(P) = p = 0.5$.
$E(X) = 3.5 + 3(0.5) = 3.5 + 1.5 = 5.0$.
\end{correctionbox}

\begin{correctionbox}[Correction Exercice 20 : Indicateurs (Problème des Chapeaux)]
1. $I_i$ est la variable indicatrice de l'événement "la personne $i$ reprend son propre chapeau".
$X = \sum_{i=1}^n I_i$.
2. Il y a $n!$ permutations (arrangements) possibles des chapeaux. Il y a $(n-1)!$ permutations où la personne $i$ a son propre chapeau.
$P(I_i=1) = \frac{(n-1)!}{n!} = \frac{1}{n}$.
3. Par linéarité : $E(X) = E(\sum I_i) = \sum E(I_i)$.
$E(I_i) = P(I_i=1) = 1/n$.
$E(X) = \sum_{i=1}^n (1/n) = n \times (1/n) = 1$.
En moyenne, quel que soit $n$, une seule personne reprend son chapeau !
\end{correctionbox}

\begin{correctionbox}[Correction Exercice 21 : Indicateurs (Collectionneurs)]
1. $I_A = 1$ si on a au moins une fig. A, $I_A = 0$ sinon.
2. $P(I_A=0) = P(\text{n'avoir aucune fig. A})$. À chaque boîte, $P(\text{pas A}) = 3/4$.
Pour 5 boîtes indépendantes : $P(I_A=0) = (3/4)^5$.
3. $E(I_A) = P(I_A=1) = 1 - P(I_A=0) = 1 - (3/4)^5$.
4. $X$ est le nombre de types distincts, $X = I_A + I_B + I_C + I_D$.
Par linéarité : $E(X) = E(I_A) + E(I_B) + E(I_C) + E(I_D)$.
Par symétrie, $E(I_A) = E(I_B) = E(I_C) = E(I_D) = 1 - (3/4)^5$.
$E(X) = 4 \times \left( 1 - (3/4)^5 \right) \approx 4 \times (1 - 0.2373) = 4 \times 0.7627 = 3.0508$.
\end{correctionbox}

\begin{correctionbox}[Correction Exercice 22 : Variance (Propriétés)]
Soit $\mu = E(X)$. Alors $E(aX+b) = aE(X)+b = a\mu+b$.
Par définition de la variance :
$\text{Var}(aX+b) = E\left[ ( (aX+b) - E[aX+b] )^2 \right]$
$= E\left[ ( (aX+b) - (a\mu+b) )^2 \right]$
$= E\left[ ( aX - a\mu )^2 \right]$
$= E\left[ ( a(X - \mu) )^2 \right]$
$= E\left[ a^2 (X - \mu)^2 \right]$
$= a^2 E\left[ (X - \mu)^2 \right]$ (car $a^2$ est une constante)
$= a^2 \text{Var}(X)$.
\end{correctionbox}

\begin{correctionbox}[Correction Exercice 23 : Identifier la Loi]
1. \textbf{Binomiale} (n=20 essais de Bernoulli indépendants).
2. \textbf{Poisson} (comptage d'événements rares dans un intervalle fixe).
3. \textbf{Hypergéométrique} (tirage sans remise d'une population finie).
4. \textbf{Géométrique} (comptage d'échecs avant le premier succès).
5. \textbf{Bernoulli} (un seul essai, deux issues).
\end{correctionbox}

\begin{correctionbox}[Correction Exercice 24 : Espérance (Loi Hypergéométrique)]
Soit $I_j = 1$ si la $j$-ème boule tirée (pour $j=1 \dots m$) est blanche, $I_j=0$ sinon.
Le nombre total de blanches est $X = \sum_{j=1}^m I_j$.
Par linéarité, $E(X) = \sum_{j=1}^m E(I_j)$.
$E(I_j) = P(I_j=1)$. Quelle est la probabilité que la $j$-ème boule tirée soit blanche ?
Par symétrie (ou en considérant un tirage aléatoire), n'importe quelle boule a la même probabilité d'être à la $j$-ème position. Il y a $w$ blanches sur $w+b$ boules au total.
$P(I_j=1) = \frac{w}{w+b}$ (c'est vrai pour $j=1$, $j=2$, ... $j=m$).
$E(X) = \sum_{j=1}^m \frac{w}{w+b} = m \left( \frac{w}{w+b} \right)$.
\end{correctionbox}

\begin{correctionbox}[Correction Exercice 25 : Variance (Poisson)]
$X_n \sim \text{Bin}(n, \lambda/n)$.
$\text{Var}(X_n) = np(1-p) = n(\lambda/n)(1 - \lambda/n) = \lambda(1 - \lambda/n)$.
Lorsque $n \to \infty$ :
$\lim_{n \to \infty} \text{Var}(X_n) = \lim_{n \to \infty} \lambda(1 - \lambda/n)$
Puisque $\lambda/n \to 0$, la limite est $\lambda(1 - 0) = \lambda$.
On en déduit que pour $X \sim \text{Poisson}(\lambda)$, la variance est $\text{Var}(X) = \lambda$.
(L'espérance et la variance sont égales pour la loi de Poisson).
\end{correctionbox}

\subsection{Exercices Python}

Les exercices suivants appliquent les concepts de variables aléatoires discrètes, de leurs lois de probabilité (PMF, CDF) et de leurs caractéristiques (espérance, variance) au jeu de données "Taxis" de Seaborn.

\begin{codecell}
import pandas as pd
import seaborn as sns
import math

# Charger le dataset Taxis
df = sns.load_dataset("taxis")

# 'df' est maintenant notre Univers S.
# S_total = len(df)
\end{codecell}

\begin{exercicebox}[Exercice 1 : PMF (Fonction de Masse)]
Soit $X$ la variable aléatoire discrète représentant le nombre de passagers (\texttt{passengers}) dans un taxi.

\textbf{Votre tâche :}
\begin{enumerate}
    \item Calculer la Fonction de Masse (PMF) de $X$. Trouvez $P(X=1)$, $P(X=2)$, $P(X=3)$, etc., pour toutes les valeurs non nulles.
    \item Vérifier que $\sum_{k} P(X=k) = 1$.
\end{enumerate}
\end{exercicebox}

\begin{exercicebox}[Exercice 2 : CDF (Fonction de Répartition)]
En utilisant la variable $X$ (\texttt{passengers}) de l'exercice 1 :

\textbf{Votre tâche :}
\begin{enumerate}
    \item Calculer la valeur de la Fonction de Répartition (CDF) au point 2, c'est-à-dire $F_X(2) = P(X \le 2)$.
    \item Calculer $P(X > 3)$ en utilisant la CDF.
\end{enumerate}
\end{exercicebox}

\begin{exercicebox}[Exercice 3 : Espérance d'une VA Discrète]
En utilisant la variable $X$ (\texttt{passengers}) et sa PMF $P_X(k)$ calculée à l'exercice 1, calculez l'espérance de $X$.

L'espérance est $E[X] = \sum_{k} k \cdot P(X=k)$.

\textbf{Votre tâche :}
\begin{enumerate}
    \item Lister les valeurs $k$ possibles pour \texttt{passengers}.
    \item Pour chaque $k$, calculer le produit $k \times P(X=k)$.
    \item Sommer ces produits pour obtenir $E[X]$.
    \item (Vérification) Comparez votre résultat à la moyenne directe \texttt{df['passengers'].mean()}.
\end{enumerate}
\end{exercicebox}

\begin{exercicebox}[Exercice 4 : Variance d'une VA Discrète]
Calculez la variance de $X$ (\texttt{passengers}) en utilisant la formule $\text{Var}(X) = E[X^2] - (E[X])^2$.

\textbf{Votre tâche :}
\begin{enumerate}
    \item Utiliser l'espérance $E[X]$ calculée à l'exercice 3.
    \item Calculer l'espérance du carré, $E[X^2] = \sum_{k} k^2 \cdot P(X=k)$.
    \item Appliquer la formule de la variance.
    \item (Vérification) Comparez votre résultat à \texttt{df['passengers'].var()}.
\end{enumerate}
\end{exercicebox}

\begin{exercicebox}[Exercice 5 : Variable Aléatoire Indicatrice et Espérance]
Soit $I_A$ une variable aléatoire indicatrice pour l'événement $A$ = "le passager a donné un pourboire".

\textbf{Votre tâche :}
\begin{enumerate}
    \item Créer une nouvelle colonne \texttt{got\_tip} dans le DataFrame.
    \item Assigner $I_A = 1$ si \texttt{tip > 0}, et $I_A = 0$ sinon.
    \item Calculer l'espérance de cette variable, $E[I_A]$.
    \item Constatez que $E[I_A]$ est exactement la probabilité $P(A)$ que le passager donne un pourboire.
\end{enumerate}
\end{exercicebox}

\begin{exercicebox}[Exercice 6 : Loi de Bernoulli]
Soit $X$ une variable aléatoire de Bernoulli modélisant le type de paiement.
$X=1$ si le paiement est "credit\_card" (Succès) et $X=0$ si c'est "cash" (Échec).

\textbf{Votre tâche :}
\begin{enumerate}
    \item Calculer $p$, la probabilité de succès, $P(X=1)$.
    \item Calculer $1-p$, la probabilité d'échec, $P(X=0)$.
    \item Quelle est l'espérance $E[X]$ et la variance $\text{Var}(X)$ de cette variable ? (Utilisez les formules $p$ et $p(1-p)$).
\end{enumerate}
\end{exercicebox}

\begin{exercicebox}[Exercice 7 : Loi Binomiale (PMF et Espérance)]
Nous utilisons le paramètre $p$ (probabilité de payer par carte) de l'exercice 6.
On observe un échantillon de $n=10$ courses indépendantes. Soit $Y$ le nombre de courses payées par carte dans cet échantillon. $Y$ suit une loi binomiale $Y \sim \text{Bin}(n=10, p)$.

\textbf{Votre tâche :}
\begin{enumerate}
    \item En utilisant la PMF de la loi binomiale, calculer la probabilité d'avoir exactement $k=4$ paiements par carte, $P(Y=4)$.
    \item Calculer l'espérance $E[Y]$ et la variance $\text{Var}(Y)$ de cette variable binomiale.
\end{enumerate}
\end{exercicebox}

\begin{exercicebox}[Exercice 8 : Loi Géométrique]
Nous observons les trajets un par un (supposés indépendants) jusqu'à trouver notre premier "succès". 
Le "succès" est défini comme "trouver un trajet avec un pourboire (\texttt{tip}) de plus de 5 dollars".

Soit $X$ le nombre d'échecs (trajets avec $\texttt{tip} \le 5$) avant le premier succès. $X$ suit une loi géométrique $X \sim \text{Geom}(p)$.

\textbf{Votre tâche :}
\begin{enumerate}
    \item Calculer $p$, la probabilité de "succès" (trouver un $\texttt{tip} > 5$).
    \item En utilisant la PMF de la loi géométrique, calculer la probabilité d'avoir exactement $k=10$ échecs avant le premier succès.
\end{enumerate}
\end{exercicebox}

\begin{exercicebox}[Exercice 9 : Loi Hypergéométrique]
Considérons le tirage \textbf{sans remise}. Notre population est l'ensemble du \texttt{df}.
Soit $w$ le nombre total de paiements par "credit\_card" et $b$ le nombre total de paiements "cash".
On tire un échantillon de $m=20$ trajets. Soit $Z$ le nombre de paiements par carte dans cet échantillon. $Z$ suit une loi Hypergéométrique.

\textbf{Votre tâche :}
\begin{enumerate}
    \item Trouver $w$, $b$ et $m=20$.
    \item En utilisant la PMF de la loi hypergéométrique, calculer la probabilité d'avoir exactement $k=15$ paiements par carte dans l'échantillon, $P(Z=15)$.
\end{enumerate}
\end{exercicebox}

\begin{exercicebox}[Exercice 10 : Loi de Poisson]
La loi de Poisson modélise le nombre d'événements dans un intervalle de temps. Nous voulons modéliser le nombre de courses par heure dans un quartier.

\textbf{Votre tâche :}
\begin{enumerate}
    \item Filtrer le DataFrame pour ne garder que les courses dans le quartier "Manhattan" (\texttt{pickup\_borough == 'Manhattan'}).
    \item Convertir la colonne \texttt{pickup} en datetime.
    \item Agréger les données pour compter le nombre de courses par heure (vous pouvez "arrondir" l'heure de début de course).
    \item Calculer $\lambda$, le taux moyen de courses par heure à Manhattan (l'espérance de la loi de Poisson).
    \item En utilisant la PMF de la loi de Poisson avec ce $\lambda$, calculer la probabilité qu'il y ait exactement $k=50$ courses lors d'une heure donnée, $P(X=50)$.
\end{enumerate}
\end{exercicebox}
\newpage
\section{Espérance et Variance }

\subsection{Espérance d'une variable aléatoire discrète}

Maintenant que nous avons défini les variables aléatoires discrètes et leur distribution (PMF), l'étape suivante est de résumer ces distributions. La mesure la plus importante est leur "centre", ou leur valeur moyenne.

\begin{definitionbox}[Espérance]
L'espérance (ou valeur attendue) d'une variable aléatoire discrète $X$, qui prend les valeurs distinctes $x_1, x_2, \dots$, est définie par :
$$ E(X) = \sum_j x_j P(X=x_j) $$
\end{definitionbox}

Cette formule est une moyenne pondérée de toutes les valeurs possibles.

\begin{intuitionbox}
L'espérance représente la valeur moyenne que l'on obtiendrait si l'on répétait l'expérience un très grand nombre de fois. C'est le \textbf{centre de gravité} de la distribution de probabilité. Si les probabilités étaient des masses placées sur une tige aux positions $x_j$, l'espérance serait le point d'équilibre.
\end{intuitionbox}

L'exemple le plus simple est le lancer d'un dé.

\begin{examplebox}[Lancer d'un dé]
Soit $X$ le résultat d'un lancer de dé équilibré. Chaque face a une probabilité de $1/6$. L'espérance est :
$$ E(X) = 1\left(\frac{1}{6}\right) + 2\left(\frac{1}{6}\right) + 3\left(\frac{1}{6}\right) + 4\left(\frac{1}{6}\right) + 5\left(\frac{1}{6}\right) + 6\left(\frac{1}{6}\right) = \frac{21}{6} = 3.5 $$
Même si 3.5 n'est pas un résultat possible, c'est la valeur moyenne sur un grand nombre de lancers.
\end{examplebox}

\subsection{Linéarité de l'espérance}

Le calcul de l'espérance deviendrait très fastidieux si nous devions toujours utiliser la définition. Heureusement, l'espérance possède une propriété fondamentale qui simplifie énormément les calculs.

\begin{theorembox}[Linéarité de l'espérance]
Pour toutes variables aléatoires $X$ et $Y$, et pour toute constante $c$, on a :
\begin{align*}
E(X+Y) &= E(X) + E(Y) \\
E(cX) &= cE(X)
\end{align*}
Cette propriété est extrêmement puissante car elle ne requiert pas que $X$ et $Y$ soient indépendantes.
\end{theorembox}

La preuve de $E(cX) = cE(X)$ est directe à partir de la définition. La preuve pour la somme $E(X+Y)$ est plus complexe mais essentielle.

\begin{proofbox}
La première propriété est directe :
$$ E(cX) = \sum_x (cx) P(X=x) = c \sum_x x P(X=x) = cE(X) $$
Pour la seconde, nous devons utiliser la définition de l'espérance pour une fonction de deux variables (une extension de LOTUS). Soit $S = X+Y$. L'espérance $E(S)$ se calcule en sommant sur toutes les paires possibles $(x, y)$:
\begin{align*}
E(X+Y) &= \sum_x \sum_y (x+y) P(X=x, Y=y) \\
&= \sum_x \sum_y x P(X=x, Y=y) + \sum_x \sum_y y P(X=x, Y=y) \\
&= \sum_x x \left( \sum_y P(X=x, Y=y) \right) + \sum_y y \left( \sum_x P(X=x, Y=y) \right)
\end{align*}
Par la loi des probabilités totales (ou "marginalisation"), la somme interne $\sum_y P(X=x, Y=y)$ est simplement $P(X=x)$. De même, $\sum_x P(X=x, Y=y) = P(Y=y)$.
$$ E(X+Y) = \sum_x x P(X=x) + \sum_y y P(Y=y) = E(X) + E(Y) $$
Notez que l'indépendance n'a jamais été requise pour cette preuve.
\end{proofbox}

Cette propriété est incroyablement utile.

\begin{intuitionbox}
Cette propriété formalise une idée très simple : "la moyenne d'une somme est la somme des moyennes". Si vous jouez à deux jeux de hasard, votre gain moyen total est simplement la somme de ce que vous gagnez en moyenne à chaque jeu, que les jeux soient liés ou non.
\end{intuitionbox}

Cette propriété rend le calcul de l'espérance d'une somme trivial, comme le montre l'exemple des deux dés.

\begin{examplebox}[Somme de deux dés]
Soit $X_1$ le résultat du premier dé et $X_2$ celui du second. On sait que $E(X_1) = 3.5$ et $E(X_2) = 3.5$.
Soit $S = X_1 + X_2$ la somme des deux dés. Grâce à la linéarité, on peut calculer l'espérance de la somme sans avoir à lister les 36 résultats possibles :
$$ E(S) = E(X_1 + X_2) = E(X_1) + E(X_2) = 3.5 + 3.5 = 7 $$
\end{examplebox}

\subsection{Espérance de la loi binomiale}

Nous pouvons maintenant utiliser cette puissante propriété de linéarité pour trouver l'espérance de nos distributions de référence, en évitant des sommes complexes.

\begin{theorembox}[Espérance de la loi binomiale]
Si $X \sim \text{Bin}(n, p)$, alors son espérance est $E(X) = np$.
\end{theorembox}

Ce résultat est profondément intuitif.

\begin{intuitionbox}
Ce résultat est très naturel. Si vous lancez une pièce 100 fois ($n=100$) avec une probabilité de 50\% d'obtenir Pile ($p=0.5$), vous vous attendez en moyenne à obtenir $100 \times 0.5 = 50$ Piles. La formule $np$ généralise cette idée.
\end{intuitionbox}

La preuve formelle est un exemple parfait de l'élégance de la linéarité, utilisant les variables indicatrices.

\begin{proofbox}
Le calcul direct de l'espérance avec la PMF binomiale est possible, mais long. En utilisant la linéarité de l'espérance, on obtient une preuve beaucoup plus courte et élégante.

On peut voir une variable binomiale $X$ comme la somme de $n$ variables de Bernoulli indépendantes, $X = I_1 + I_2 + \dots + I_n$, où chaque $I_j$ représente le succès (1) ou l'échec (0) du $j$-ième essai.

Chaque $I_j$ a pour espérance $E(I_j) = 1 \cdot p + 0 \cdot (1-p) = p$.

Par linéarité de l'espérance, on a :
$$ E(X) = E(I_1) + E(I_2) + \dots + E(I_n) = \underbrace{p + p + \dots + p}_{n \text{ fois}} = np $$
\end{proofbox}

\subsection{Espérance de la loi géométrique}

Calculons maintenant l'espérance pour la loi qui modélise le temps d'attente.

\begin{theorembox}[Espérance de la loi géométrique]
L'espérance d'une variable aléatoire $X \sim \text{Geom}(p)$ (comptant le nombre d'échecs) est :
$$ E(X) = \frac{1-p}{p} = \frac{q}{p} $$
\end{theorembox}

L'intuition est aussi très forte ici :

\begin{intuitionbox}
Si un événement a 1 chance sur 10 de se produire ($p=0.1$), il est logique de penser qu'il faudra en moyenne 9 échecs ($q/p = 0.9/0.1=9$) avant qu'il ne se produise. L'espérance du nombre total d'essais (échecs + 1 succès) serait alors $1/p$.
\end{intuitionbox}

Contrairement à la loi binomiale, la preuve la plus directe ne repose pas sur la linéarité mais sur une manipulation de séries.

\begin{proofbox}[Démonstration de l'espérance géométrique via les séries entières]
Soit $X \sim \text{Geom}(p)$, où $X$ compte le nombre d'échecs avant le premier succès. La PMF est $P(X=k) = q^k p$ pour $k=0, 1, 2, \dots$, avec $q=1-p$.

Par définition, l'espérance est :
$$ E(X) = \sum_{k=0}^{\infty} k \cdot P(X=k) = \sum_{k=0}^{\infty} k q^k p $$
Le terme pour $k=0$ est nul, on peut donc commencer la somme à $k=1$ :
$$ E(X) = p \sum_{k=1}^{\infty} k q^k $$
L'astuce consiste à reconnaître que la somme ressemble à la dérivée d'une série géométrique. Rappelons la formule de la série géométrique pour $|q|<1$ :
$$ \sum_{k=0}^{\infty} q^k = \frac{1}{1-q} $$
En dérivant les deux côtés par rapport à $q$, on obtient :
$$ \frac{d}{dq} \left( \sum_{k=0}^{\infty} q^k \right) = \frac{d}{dq} \left( \frac{1}{1-q} \right) $$
$$ \sum_{k=1}^{\infty} k q^{k-1} = \frac{1}{(1-q)^2} $$
Pour faire apparaître ce terme dans notre formule d'espérance, on factorise $q$ dans la somme :
$$ E(X) = p \cdot q \sum_{k=1}^{\infty} k q^{k-1} $$
On peut maintenant remplacer la somme par son expression analytique :
$$ E(X) = p \cdot q \cdot \frac{1}{(1-q)^2} $$
Puisque $p = 1-q$, on a :
$$ E(X) = p \cdot q \cdot \frac{1}{p^2} = \frac{q}{p} $$
Ce qui démontre que l'espérance du nombre d'échecs avant le premier succès est $\frac{q}{p}$.
\end{proofbox}

\subsection{Loi du statisticien inconscient (LOTUS)}

Souvent, nous ne sommes pas intéressés par l'espérance de $X$ elle-même, mais par l'espérance d'une fonction de $X$, par exemple $E(X^2)$ ou $E(e^X)$.

\begin{theorembox}[Théorème de Transfert (LOTUS)]
Si $X$ est une variable aléatoire discrète et $g(x)$ est une fonction de $\mathbb{R}$ dans $\mathbb{R}$, alors l'espérance de la variable aléatoire $g(X)$ est donnée par :
$$ E[g(X)] = \sum_x g(x) P(X=x) $$
La somme porte sur toutes les valeurs possibles de $X$. Ce théorème est utile car il évite d'avoir à trouver la PMF de $g(X)$.
\end{theorembox}

La preuve dans le cas discret consiste simplement à regrouper les termes.

\begin{proofbox}
Soit $Y = g(X)$. Par définition, l'espérance de $Y$ est $E(Y) = \sum_y y P(Y=y)$.
L'ensemble des valeurs $y$ que $Y$ peut prendre est $\{g(x) \mid x \in \text{support de } X\}$.
Pour une valeur $y$ donnée, l'événement $\{Y=y\}$ est l'union de tous les événements $\{X=x\}$ tels que $g(x)=y$.
$$ P(Y=y) = P(g(X)=y) = \sum_{x: g(x)=y} P(X=x) $$
En substituant cela dans la définition de $E(Y)$ :
$$ E(Y) = \sum_y y \left( \sum_{x: g(x)=y} P(X=x) \right) $$
On peut réécrire $y$ comme $g(x)$ à l'intérieur de la seconde somme :
$$ E(g(X)) = \sum_y \sum_{x: g(x)=y} g(x) P(X=x) $$
Cette double somme parcourt toutes les valeurs de $y$, et pour chaque $y$, elle parcourt tous les $x$ correspondants. Cela revient à simplement sommer sur tous les $x$ possibles dès le départ :
$$ E[g(X)] = \sum_x g(x) P(X=x) $$
\end{proofbox}

Ce théorème justifie son nom : c'est ce que l'on ferait "inconsciemment".

\begin{intuitionbox}
Pour trouver la valeur moyenne d'une fonction d'une variable aléatoire (par exemple, le carré du résultat d'un dé), vous n'avez pas besoin de déterminer d'abord la distribution de ce carré. Vous pouvez simplement prendre chaque valeur possible du résultat original, lui appliquer la fonction, et pondérer ce nouveau résultat par la probabilité du résultat original.
\end{intuitionbox}

Utilisons ce théorème pour calculer $E(X^2)$ pour notre dé.

\begin{examplebox}[Calcul de $E(X^2)$ pour un dé]
Soit $X$ le résultat d'un lancer de dé. Calculons l'espérance de $Y=X^2$. La fonction est $g(x)=x^2$.
\begin{align*}
E(X^2) &= \sum_{k=1}^6 k^2 P(X=k) \\
&= 1^2\left(\frac{1}{6}\right) + 2^2\left(\frac{1}{6}\right) + 3^2\left(\frac{1}{6}\right) + 4^2\left(\frac{1}{6}\right) + 5^2\left(\frac{1}{6}\right) + 6^2\left(\frac{1}{6}\right) \\
&= \frac{1+4+9+16+25+36}{6} = \frac{91}{6} \approx 15.17
\end{align*}
\end{examplebox}

\subsection{Variance}

L'espérance nous donne le centre d'une distribution, mais elle ne dit rien sur sa "largeur" ou sa "dispersion". C'est le rôle de la variance.

\begin{definitionbox}[Variance et écart-type]
La \textbf{variance} d'une variable aléatoire $X$ mesure la dispersion de sa distribution autour de son espérance. Elle est définie par :
$$ \text{Var}(X) = E\left[ (X - E(X))^2 \right] $$
La racine carrée de la variance est appelée l' \textbf{écart-type} :
$$ \text{SD}(X) = \sqrt{\text{Var}(X)} $$
\end{definitionbox}

L'idée est de mesurer l'écart quadratique moyen à l'espérance.

\begin{intuitionbox}
La variance est la "distance carrée moyenne à la moyenne". On prend l'écart de chaque valeur par rapport à la moyenne, on le met au carré (pour que les écarts positifs et négatifs ne s'annulent pas), puis on en calcule la moyenne. L'écart-type est souvent plus interprétable car il ramène cette mesure de dispersion dans les mêmes unités que la variable aléatoire elle-même.
\end{intuitionbox}

La définition $E[(X-E(X))^2]$ est excellente pour l'interprétation, mais pénible pour le calcul. Une formule alternative est presque toujours utilisée.

\begin{theorembox}[Formule de calcul de la variance]
Pour toute variable aléatoire $X$, une formule plus pratique pour le calcul de la variance est :
$$ \text{Var}(X) = E(X^2) - [E(X)]^2 $$
\end{theorembox}

La preuve est une simple expansion algébrique utilisant la linéarité de l'espérance.

\begin{proofbox}
Soit $\mu = E(X)$. On part de la définition de la variance :
\begin{align*}
\text{Var}(X) &= E[ (X - \mu)^2 ] \\
&= E[ X^2 - 2X\mu + \mu^2 ] \quad \text{(On développe le carré)} \\
&= E(X^2) - E(2\mu X) + E(\mu^2) \quad \text{(Par linéarité de l'espérance)} \\
&= E(X^2) - 2\mu E(X) + \mu^2 \quad \text{(Car $2\mu$ et $\mu^2$ sont des constantes)} \\
&= E(X^2) - 2\mu(\mu) + \mu^2 \quad \text{(Car $E(X) = \mu$)} \\
&= E(X^2) - 2\mu^2 + \mu^2 \\
&= E(X^2) - \mu^2 = E(X^2) - [E(X)]^2
\end{align*}
\end{proofbox}

Nous pouvons maintenant calculer la variance de notre lancer de dé.

\begin{examplebox}[Variance d'un lancer de dé]
Nous avons déjà calculé pour un dé que $E(X) = 3.5$ et $E(X^2) = 91/6$. On peut maintenant trouver la variance facilement :
$$ \text{Var}(X) = E(X^2) - [E(X)]^2 = \frac{91}{6} - (3.5)^2 = \frac{91}{6} - 12.25 = 15.166... - 12.25 \approx 2.917 $$
L'écart-type est $\text{SD}(X) = \sqrt{2.917} \approx 1.708$.
\end{examplebox}

\subsection{Exercices}

% --- Espérance de base et LOTUS ---

\begin{exercicebox}[Exercice 1 : Calcul d'Espérance (PMF Simple)]
Une variable aléatoire $X$ a la distribution de probabilité suivante :
$P(X=-1) = 0.3$, $P(X=0) = 0.5$, $P(X=2) = 0.2$.
Calculez l'espérance $E(X)$.
\end{exercicebox}

\begin{exercicebox}[Exercice 2 : LOTUS (Calcul de $E(X^2)$)]
En utilisant la même variable aléatoire $X$ que dans l'exercice 1, calculez $E(X^2)$.
\end{exercicebox}

\begin{exercicebox}[Exercice 3 : Variance (Calcul de base)]
En utilisant les résultats des exercices 1 et 2, calculez la variance $\text{Var}(X)$.
\end{exercicebox}

\begin{exercicebox}[Exercice 4 : Espérance (Jeu Simple)]
Un jeu consiste à payer 2 pour lancer un dé à 6 faces. Si le dé tombe sur 6, vous gagnez 10. Sinon, vous ne gagnez rien. Soit $G$ votre gain net (gain - mise).
\begin{enumerate}
    \item Quelle est la PMF de $G$ ?
    \item Calculez $E(G)$. Le jeu est-il favorable au joueur ?
\end{enumerate}
\end{exercicebox}

\begin{exercicebox}[Exercice 5 : Variance (Jeu Simple)]
En utilisant la variable aléatoire $G$ de l'exercice 4 :
\begin{enumerate}
    \item Calculez $E(G^2)$.
    \item Calculez $\text{Var}(G)$.
\end{enumerate}
\end{exercicebox}

\begin{exercicebox}[Exercice 6 : Espérance de Bernoulli]
Soit $X$ une variable aléatoire $X \sim \text{Bern}(p)$ (variable indicatrice). En utilisant la définition de l'espérance, montrez que $E(X) = p$.
\end{exercicebox}

\begin{exercicebox}[Exercice 7 : Variance de Bernoulli]
En utilisant le résultat de l'exercice 6 et le théorème de LOTUS, montrez que $\text{Var}(X) = p(1-p)$ pour $X \sim \text{Bern}(p)$. (Indice : $X^2 = X$ pour une variable de Bernoulli).
\end{exercicebox}

% --- Linéarité de l'Espérance ---

\begin{exercicebox}[Exercice 8 : Linéarité (Simple)]
Soient $X$ et $Y$ deux variables aléatoires. On sait que $E(X) = 10$ et $E(Y) = -5$.
Calculez $E(3X - 2Y + 4)$.
\end{exercicebox}

\begin{exercicebox}[Exercice 9 : Linéarité (Trois Dés)]
On lance trois dés équilibrés à 6 faces. Soit $S$ la somme des trois résultats.
En utilisant la linéarité de l'espérance, calculez $E(S)$.
\end{exercicebox}

\begin{exercicebox}[Exercice 10 : Linéarité (Somme de Bernoulli)]
Soit $X \sim \text{Bin}(n, p)$. On rappelle que $X$ peut s'écrire comme la somme de $n$ variables de Bernoulli indépendantes $X = I_1 + \dots + I_n$, où $E(I_j) = p$.
Utilisez la linéarité de l'espérance pour prouver que $E(X) = np$.
\end{exercicebox}

% --- Espérances des Lois Classiques ---

\begin{exercicebox}[Exercice 11 : Espérance Binomiale (Application)]
Un QCM (questionnaire à choix multiples) comporte 40 questions. Chaque question a 4 options de réponse, dont une seule est correcte. Un étudiant répond à tout au hasard.
Quel est le nombre attendu (l'espérance) de bonnes réponses ?
\end{exercicebox}

\begin{exercicebox}[Exercice 12 : Espérance Géométrique (Application)]
On lance une paire de dés équilibrés. Un "succès" est d'obtenir un double-six.
\begin{enumerate}
    \item Quelle est la probabilité $p$ d'un succès ?
    \item Soit $X$ le nombre d'échecs avant le premier double-six. Quelle est l'espérance $E(X)$ ?
\end{enumerate}
\end{exercicebox}

\begin{exercicebox}[Exercice 13 : Espérance Géométrique (Attente Totale)]
En reprenant la situation de l'exercice 12 ($p=1/36$), soit $Y$ le \textit{nombre total de lancers} nécessaires pour obtenir le premier double-six ($Y = X + 1$).
Calculez $E(Y)$.
\end{exercicebox}

\begin{exercicebox}[Exercice 14 : Espérance (Loi Hypergéométrique)]
On tire 5 cartes d'un jeu de 52 cartes sans remise. Soit $X$ le nombre d'As tirés. On peut écrire $X = I_1 + I_2 + I_3 + I_4 + I_5$, où $I_j=1$ si la $j$-ème carte tirée est un As, et 0 sinon.
\begin{enumerate}
    \item Quelle est la probabilité $P(I_1 = 1)$ (que la 1ère carte soit un As) ?
    \item Quelle est la probabilité $P(I_2 = 1)$ (que la 2ème carte soit un As) ? (Indice : Pensez par symétrie ou utilisez la LTP).
    \item Calculez $E(X)$ en utilisant la linéarité.
\end{enumerate}
\end{exercicebox}

% --- Variance et E[X^2] ---

\begin{exercicebox}[Exercice 15 : Espérance et Variance (Dé à 4 faces)]
Soit $X$ le résultat d'un lancer de dé équilibré à 4 faces ($X \in \{1, 2, 3, 4\}$).
\begin{enumerate}
    \item Calculez $E(X)$.
    \item Calculez $E(X^2)$.
    \item Calculez $\text{Var}(X)$.
\end{enumerate}
\end{exercicebox}

\begin{exercicebox}[Exercice 16 : Formule de la Variance (Inverse)]
Une variable aléatoire $Y$ a une espérance $E(Y) = 5$ et une variance $\text{Var}(Y) = 4$.
Quelle est la valeur de $E(Y^2)$ ?
\end{exercicebox}

\begin{exercicebox}[Exercice 17 : Formule de la Variance (Inverse 2)]
Une variable aléatoire $W$ a $E(W^2) = 50$ et $\text{Var}(W) = 1$.
Quelles sont les deux valeurs possibles pour $E(W)$ ?
\end{exercicebox}

\begin{exercicebox}[Exercice 18 : Variance Nulle]
Une variable aléatoire $X$ a une variance $\text{Var}(X) = 0$. Que pouvez-vous conclure sur la distribution de $X$ ?
(Indice : $\text{Var}(X) = E[(X-\mu)^2]$).
\end{exercicebox}

\begin{exercicebox}[Exercice 19 : LOTUS et Linéarité]
Soit $X$ une variable aléatoire avec $E(X)=3$ et $E(X^2)=10$.
Calculez $E[(X+1)^2]$.
(Indice : Développez $(X+1)^2$ avant de prendre l'espérance).
\end{exercicebox}

\begin{exercicebox}[Exercice 20 : Synthèse (Jeu de Roulette)]
À la roulette, vous misez 1 sur "Rouge". Il y a 18 cases rouges, 18 noires, et 1 verte (le 0). Total = 37 cases.
Si "Rouge" sort, vous récupérez votre mise de 1 et gagnez 1 de plus (gain net $G=+1$).
Si "Noir" or "Vert" sort, vous perdez votre mise (gain net $G=-1$).
\begin{enumerate}
    \item Calculez $E(G)$.
    \item Calculez $E(G^2)$.
    \item Calculez $\text{Var}(G)$.
\end{enumerate}
\end{exercicebox}

\subsection{Corrections des Exercices}

% --- Corrections : Concepts de Base (PMF, CDF) ---

\begin{correctionbox}[Correction Exercice 1 : Identification de Variables Aléatoires]
1.  \textbf{Discrète}. $X$ ne peut prendre que des valeurs entières $\{0, 1, \dots, 10\}$.
2.  \textbf{Continue}. Le temps peut prendre n'importe quelle valeur dans un intervalle (par ex. $T \in [2.5, 5]$ heures).
3.  \textbf{Discrète}. $X$ ne peut prendre que des valeurs entières $\{0, 1, 2, \dots\}$.
4.  \textbf{Continue}. La température peut prendre n'importe quelle valeur dans un intervalle (par ex. $T \in [15.0, 25.0]^\circ\text{C}$).
5.  \textbf{Discrète}. $X$ ne peut prendre que des valeurs entières $\{0, 1, 2, \dots\}$ (si on compte les échecs) ou $\{1, 2, 3, \dots\}$ (si on compte les lancers).
\end{correctionbox}

\begin{correctionbox}[Correction Exercice 2 : Construction d'une PMF]
On lance un dé à 4 faces (1, 2, 3, 4). $X$ est le résultat.
1.  Valeurs possibles : $S_X = \{1, 2, 3, 4\}$.
2.  PMF : Le dé est équilibré, donc chaque face a la même probabilité $1/4$.
    $P(X=1) = 1/4$
    $P(X=2) = 1/4$
    $P(X=3) = 1/4$
    $P(X=4) = 1/4$
    Et $P(X=k) = 0$ pour tout autre $k$.
3.  Vérification : $\sum P(X=k) = 1/4 + 1/4 + 1/4 + 1/4 = 4/4 = 1$.
\end{correctionbox}

\begin{correctionbox}[Correction Exercice 3 : PMF d'une Somme]
$Y = D_1 + D_2$, où $D_1, D_2 \in \{1, 2, 3, 4\}$. Il y a $4 \times 4 = 16$ issues équiprobables (prob. 1/16 chacune).
1.  Valeurs possibles : Min = $1+1=2$. Max = $4+4=8$. $S_Y = \{2, 3, 4, 5, 6, 7, 8\}$.
2.  PMF (en comptant les issues favorables sur 16) :
    - $P(Y=2) = P(1,1) \implies 1/16$
    - $P(Y=3) = P(1,2) + P(2,1) \implies 2/16$
    - $P(Y=4) = P(1,3) + P(2,2) + P(3,1) \implies 3/16$
    - $P(Y=5) = P(1,4) + P(2,3) + P(3,2) + P(4,1) \implies 4/16$
    - $P(Y=6) = P(2,4) + P(3,3) + P(4,2) \implies 3/16$
    - $P(Y=7) = P(3,4) + P(4,3) \implies 2/16$
    - $P(Y=8) = P(4,4) \implies 1/16$
    (Vérification : $1+2+3+4+3+2+1 = 16$. La somme est $16/16 = 1$).
\end{correctionbox}

\begin{correctionbox}[Correction Exercice 4 : Construction d'une CDF]
On utilise la PMF de l'exercice 3. $F_Y(y) = P(Y \le y)$.
1.  CDF aux points de masse :
    - $F_Y(2) = P(Y \le 2) = P(Y=2) = 1/16$
    - $F_Y(3) = P(Y \le 3) = P(Y=2)+P(Y=3) = 1/16 + 2/16 = 3/16$
    - $F_Y(4) = P(Y \le 4) = 3/16 + P(Y=4) = 3/16 + 3/16 = 6/16$
    - $F_Y(5) = P(Y \le 5) = 6/16 + P(Y=5) = 6/16 + 4/16 = 10/16$
    - $F_Y(6) = P(Y \le 6) = 10/16 + P(Y=6) = 10/16 + 3/16 = 13/16$
    - $F_Y(7) = P(Y \le 7) = 13/16 + P(Y=7) = 13/16 + 2/16 = 15/16$
    - $F_Y(8) = P(Y \le 8) = 15/16 + P(Y=8) = 15/16 + 1/16 = 16/16 = 1$
2.  $F_Y(1.5) = P(Y \le 1.5) = 0$ (car la valeur minimale est 2).
3.  $F_Y(5.2) = P(Y \le 5.2) = P(Y \le 5) = F_Y(5) = 10/16$.
4.  $F_Y(10) = P(Y \le 10) = 1$ (car la valeur maximale est 8).
\end{correctionbox}

% --- Corrections : Loi de Bernoulli et Loi Binomiale ---

\begin{correctionbox}[Correction Exercice 5 : Loi de Bernoulli]
1.  $X$ suit une \textbf{loi de Bernoulli}. Le paramètre est $p=0.05$. On note $X \sim \text{Bern}(0.05)$.
2.  La PMF est :
    $P(X=1) = p = 0.05$ (succès = défectueux)
    $P(X=0) = 1-p = 0.95$ (échec = non défectueux)
\end{correctionbox}

\begin{correctionbox}[Correction Exercice 6 : Loi Binomiale (Calcul Direct)]
1.  $X$ est le nombre de succès (Pile) en $n=5$ essais indépendants avec probabilité $p=0.7$.
    $X$ suit une \textbf{loi Binomiale}. $X \sim \text{Bin}(n=5, p=0.7)$.
2.  $P(X=k) = \binom{n}{k} p^k (1-p)^{n-k}$.
    $P(X=3) = \binom{5}{3} (0.7)^3 (1-0.7)^{5-3} = 10 \times (0.343) \times (0.3)^2 = 10 \times 0.343 \times 0.09 = 0.3087$.
3.  $P(X=5) = \binom{5}{5} (0.7)^5 (0.3)^0 = 1 \times (0.7)^5 \times 1 = 0.16807$.
\end{correctionbox}

\begin{correctionbox}[Correction Exercice 7 : Loi Binomiale (Calcul Cumulé)]
On a $X \sim \text{Bin}(5, 0.7)$.
1.  $P(X=0) = \binom{5}{0} (0.7)^0 (0.3)^5 = 1 \times 1 \times (0.3)^5 = 0.00243$.
2.  L'événement "au moins 1 Pile" ($X \ge 1$) est le complémentaire de "0 Pile" ($X=0$).
    $P(X \ge 1) = 1 - P(X=0) = 1 - 0.00243 = 0.99757$.
\end{correctionbox}

\begin{correctionbox}[Correction Exercice 8 : Problème Binomial (Contrôle Qualité)]
Le tirage est \textit{avec remise}, donc les essais sont indépendants. C'est une loi binomiale.
$n = 20$ (nombre d'essais).
$p = 0.10$ (probabilité de succès = défectueux).
On cherche $P(X=2)$.
$P(X=2) = \binom{20}{2} (0.1)^2 (1-0.1)^{20-2}$
$P(X=2) = \frac{20 \times 19}{2} (0.1)^2 (0.9)^{18} = 190 \times 0.01 \times (0.9)^{18}$
$P(X=2) = 1.9 \times (0.9)^{18} \approx 1.9 \times 0.15009 \approx 0.2852$.
\end{correctionbox}

% --- Corrections : Loi Hypergéométrique ---

\begin{correctionbox}[Correction Exercice 9 : Loi Hypergéométrique (Urne)]
Le tirage est \textit{sans remise} d'une population finie.
1.  $X$ suit une \textbf{loi Hypergéométrique}.
    Paramètres : $w=7$ (blanches, succès), $b=5$ (noires, échecs), $m=4$ (nombre de tirages).
    $X \sim \text{HG}(w=7, b=5, m=4)$.
2.  On cherche $P(X=2)$.
    $P(X=k) = \frac{\binom{w}{k} \binom{b}{m-k}}{\binom{w+b}{m}}$
    $P(X=2) = \frac{\binom{7}{2} \binom{5}{4-2}}{\binom{12}{4}} = \frac{\binom{7}{2} \binom{5}{2}}{\binom{12}{4}}$
    $P(X=2) = \frac{(\frac{7 \times 6}{2}) \times (\frac{5 \times 4}{2})}{(\frac{12 \times 11 \times 10 \times 9}{4 \times 3 \times 2 \times 1})} = \frac{21 \times 10}{495} = \frac{210}{495} = \frac{14}{33} \approx 0.4242$.
\end{correctionbox}

\begin{correctionbox}[Correction Exercice 10 : Problème Hypergéométrique (Comité)]
Tirage sans remise. C'est une loi Hypergéométrique.
$w=10$ (hommes), $b=8$ (femmes), $m=6$ (taille du comité). Total $N=18$.
On cherche $P(X=3)$ (exactement 3 hommes, ce qui implique $m-k = 6-3=3$ femmes).
$P(X=3) = \frac{\binom{10}{3} \binom{8}{3}}{\binom{18}{6}}$
$P(X=3) = \frac{(\frac{10 \times 9 \times 8}{3 \times 2 \times 1}) \times (\frac{8 \times 7 \times 6}{3 \times 2 \times 1})}{(\frac{18 \times 17 \times 16 \times 15 \times 14 \times 13}{6 \times 5 \times 4 \times 3 \times 2 \times 1})} = \frac{120 \times 56}{18564} = \frac{6720}{18564} \approx 0.362$.
\end{correctionbox}

\begin{correctionbox}[Correction Exercice 11 : Binomiale vs Hypergéométrique]
Population totale $N=10000$. 10\% défectueux, donc $w=1000$ (défectueux), $b=9000$ (non défectueux).
Tirage de $m=20$ \textit{sans remise}.
1.  Loi exacte : \textbf{Loi Hypergéométrique}.
    $X \sim \text{HG}(w=1000, b=9000, m=20)$.
2.  Probabilité exacte $P(X=2)$ :
    $P(X=2) = \frac{\binom{1000}{2} \binom{9000}{18}}{\binom{10000}{20}}$
    $P(X=2) = \frac{(\frac{1000 \times 999}{2}) \times (\frac{9000 \times \dots \times 8983}{18!})}{(\frac{10000 \times \dots \times 9981}{20!})} \approx 0.2854$.
    (Le calcul est très complexe, mais on peut montrer qu'il est très proche de la binomiale).
3.  Le résultat de l'exercice 8 (Binomiale) était $\approx 0.2852$.
    L'approximation binomiale est excellente. La raison est que la taille de l'échantillon ($m=20$) est très petite par rapport à la taille de la population ($N=10000$). Le fait de ne pas remettre les 20 articles change à peine les probabilités pour les tirages suivants.
\end{correctionbox}

% --- Corrections : Loi Géométrique ---

\begin{correctionbox}[Correction Exercice 12 : Loi Géométrique (Calcul Direct)]
1.  $X$ est le nombre d'échecs avant le premier succès. $X$ suit une \textbf{loi Géométrique}.
    Le succès est "obtenir 6", donc $p = 1/6$. $X \sim \text{Geom}(p=1/6)$.
2.  "Premier 6 au 3ème lancer" signifie 2 échecs (lancers 1 et 2) puis 1 succès (lancer 3).
    C'est $P(X=2)$. $q = 1-p = 5/6$.
    $P(X=2) = q^2 p^1 = (5/6)^2 (1/6) = 25/216 \approx 0.1157$.
3.  "Premier 6 au 1er lancer" signifie 0 échec. C'est $P(X=0)$.
    $P(X=0) = q^0 p^1 = 1 \times (1/6) = 1/6$.
\end{correctionbox}

\begin{correctionbox}[Correction Exercice 13 : Loi Géométrique (Calcul Cumulé)]
$p=0.2$ (succès), $q=0.8$ (échec). $X$ compte les échecs. $X \sim \text{Geom}(0.2)$.
1.  "Exactement 4 tirs au total" signifie 3 échecs suivis d'un succès. On cherche $P(X=3)$.
    $P(X=3) = q^3 p^1 = (0.8)^3 (0.2) = 0.512 \times 0.2 = 0.1024$.
2.  "Plus de 2 tirs au total" signifie qu'il faut au moins 3 tirs. C'est l'événement "les 2 premiers tirs sont des échecs".
    La probabilité est $P(\text{Echec 1} \cap \text{Echec 2}) = q \times q = q^2$.
    $P(X \ge 2) = (0.8)^2 = 0.64$.
\end{correctionbox}

\begin{correctionbox}[Correction Exercice 14 : Variante de la Loi Géométrique]
$Y$ est le nombre total d'essais ($k=1, 2, 3, \dots$). $p$ est la prob. de succès.
1.  Pour que $Y=k$, il faut $k-1$ échecs, suivis d'un succès.
    $P(Y=k) = (1-p)^{k-1} p = q^{k-1} p$, pour $k=1, 2, \dots$
2.  Avec $p=1/6$, on cherche $P(Y=3)$.
    $P(Y=3) = (5/6)^{3-1} (1/6) = (5/6)^2 (1/6) = 25/216$.
    C'est le même résultat que $P(X=2)$ de l'exercice 12. Les deux définitions décrivent la même situation (3 lancers au total).
\end{correctionbox}

% --- Corrections : Loi de Poisson ---

\begin{correctionbox}[Correction Exercice 15 : Loi de Poisson (Calcul Direct)]
$X \sim \text{Poisson}(\lambda=5)$. PMF : $P(X=k) = \frac{e^{-\lambda} \lambda^k}{k!}$.
1.  $P(X=0) = \frac{e^{-5} 5^0}{0!} = \frac{e^{-5} \times 1}{1} = e^{-5} \approx 0.0067$.
2.  $P(X=5) = \frac{e^{-5} 5^5}{5!} = \frac{e^{-5} \times 3125}{120} = e^{-5} \times \frac{625}{24} \approx 26.04 \times e^{-5} \approx 0.1755$.
\end{correctionbox}

\begin{correctionbox}[Correction Exercice 16 : Loi de Poisson (Calcul Cumulé)]
$X \sim \text{Poisson}(\lambda=2)$. On cherche $P(X \le 2)$.
$P(X \le 2) = P(X=0) + P(X=1) + P(X=2)$
$P(X=0) = \frac{e^{-2} 2^0}{0!} = e^{-2}$
$P(X=1) = \frac{e^{-2} 2^1}{1!} = 2e^{-2}$
$P(X=2) = \frac{e^{-2} 2^2}{2!} = \frac{4e^{-2}}{2} = 2e^{-2}$
$P(X \le 2) = e^{-2} + 2e^{-2} + 2e^{-2} = 5e^{-2} \approx 5 \times 0.1353 = 0.6767$.
\end{correctionbox}

\begin{correctionbox}[Correction Exercice 17 : Loi de Poisson (Changement de $\lambda$)]
1.  Pour une page, $X \sim \text{Poisson}(\lambda=0.5)$.
    $P(X=0) = \frac{e^{-0.5} (0.5)^0}{0!} = e^{-0.5} \approx 0.6065$.
2.  Si le taux est 0.5 faute/page, le taux pour 10 pages est $\lambda_Y = 0.5 \times 10 = 5$.
    $Y \sim \text{Poisson}(\lambda_Y=5)$.
3.  On cherche $P(Y=0)$.
    $P(Y=0) = \frac{e^{-5} 5^0}{0!} = e^{-5} \approx 0.0067$.
\end{correctionbox}

\begin{correctionbox}[Correction Exercice 18 : Approximation Binomiale par Poisson]
1.  C'est un tirage de $n=10000$ clients, où chaque client est un essai de Bernoulli avec $p=0.0003$. La loi exacte est $X \sim \text{Bin}(10000, 0.0003)$.
2.  Le paramètre $\lambda$ pour l'approximation Poisson est $\lambda = np = 10000 \times 0.0003 = 3$.
3.  On utilise $Y \sim \text{Poisson}(\lambda=3)$ pour approximer $X$.
    $P(X=2) \approx P(Y=2) = \frac{e^{-3} 3^2}{2!} = \frac{9e^{-3}}{2} = 4.5 e^{-3} \approx 4.5 \times 0.04979 \approx 0.224$.
\end{correctionbox}

% --- Corrections : Synthèse et Variables Indicatrices ---

\begin{correctionbox}[Correction Exercice 19 : Choisir la Bonne Loi]
1.  Tirage sans remise d'une population finie : \textbf{Loi Hypergéométrique}.
2.  Comptage d'événements sur un intervalle de temps fixe : \textbf{Loi de Poisson}.
3.  Comptage d'essais jusqu'au premier succès : \textbf{Loi Géométrique}.
4.  Comptage de succès sur un nombre fixe d'essais indépendants : \textbf{Loi Binomiale}.
5.  Comptage d'événements rares sur un intervalle (temps/espace) : \textbf{Loi de Poisson}.
\end{correctionbox}

\begin{correctionbox}[Correction Exercice 20 : Variable Indicatrice]
$A$ = "obtenir 6". $P(A) = 1/6$.
$I_A = 1$ si $A$ se produit, $I_A = 0$ sinon.
1.  C'est une expérience avec deux issues (succès/échec). $I_A$ suit une \textbf{Loi de Bernoulli}.
    Le paramètre est $p = P(A) = 1/6$. $I_A \sim \text{Bern}(1/6)$.
2.  La PMF de $I_A$ est :
    $P(I_A = 1) = p = 1/6$
    $P(I_A = 0) = 1-p = 5/6$
\end{correctionbox}

\subsection{Exercices Pratiques (Python)}

Ces exercices vous aideront à calculer et à vérifier empiriquement les concepts d'espérance et de variance en utilisant des simulations.

Pour ces exercices, vous aurez besoin de la bibliothèque \texttt{numpy}.

\begin{codecell}
pip install numpy
\end{codecell}

\begin{exercicebox}[Exercice 1 : $E(X)$ $E(X^2)$ et Variance (Dé)]
Nous allons simuler $N$ lancers d'un dé à 6 faces pour vérifier empiriquement la définition de l'espérance, le théorème LOTUS, et la formule de calcul de la variance.

\textbf{Votre tâche :}
\begin{enumerate}
    \item Simulez 100 000 lancers d'un dé équilibré (valeurs de 1 à 6) et stockez les résultats dans un tableau NumPy.
    \item Calculez l'espérance empirique $E(X)$ en prenant la moyenne du tableau.
    \item En utilisant LOTUS, calculez l'espérance empirique $E(X^2)$ (en créant un nouveau tableau des carrés, puis en prenant sa moyenne).
    \item Calculez la variance empirique en utilisant la formule : $\text{Var}(X) = E(X^2) - [E(X)]^2$.
    \item Comparez votre résultat à la variance calculée directement avec \texttt{numpy.var()}.
\end{enumerate}

\begin{codecell}
import numpy as np

N_simulations = 100000

# 1. Simuler N lancers d'un de a 6 faces
# lancers = ...

# 2. Calculer E(X) (moyenne empirique)
# E_X = ...
# print(f"E(X) empirique: {E_X:.4f} (Theorique: 3.5)")

# 3. Calculer E(X^2) (LOTUS)
# lancers_carres = ...
# E_X2 = ...
# print(f"E(X^2) empirique: {E_X2:.4f} (Theorique: 91/6 = 15.1667)")

# 4. Calculer Var(X) avec la formule
# var_calc = ...
# print(f"Variance (calculee): {var_calc:.4f}")

# 5. Calculer Var(X) avec la fonction numpy
# var_np = ...
# print(f"Variance (numpy.var): {var_np:.4f}")
# print(f"Difference: {np.abs(var_calc - var_np):.6f}")
\end{codecell}
\end{exercicebox}

\begin{exercicebox}[Exercice 2 : Linearite de l'Esperance]
Vérifions empiriquement que $E(X+Y) = E(X) + E(Y)$. Nous allons simuler deux variables aléatoires différentes : $X$ (un dé à 4 faces) et $Y$ (un dé à 6 faces).

\textbf{Votre tâche :}
\begin{enumerate}
    \item Simulez $N=100000$ lancers d'un dé à 4 faces ($X$).
    \item Simulez $N=100000$ lancers d'un dé à 6 faces ($Y$).
    \item Créez la variable aléatoire $Z = X + Y$.
    \item Calculez les moyennes empiriques $E(X)$, $E(Y)$, et $E(Z)$.
    \item Vérifiez que $E(Z)$ est très proche de $E(X) + E(Y)$.
\end{enumerate}

\begin{codecell}
import numpy as np

N_simulations = 100000

# 1. Simuler X (de a 4 faces) et Y (de a 6 faces)
# X = ...
# Y = ...

# 2. Creer Z = X + Y
# Z = ...

# 3. Calculer les moyennes empiriques
# E_X = ...
# E_Y = ...
# E_Z = ...

# 4. Verifier la linearite
# print(f"E(X) = {E_X:.4f}")
# print(f"E(Y) = {E_Y:.4f}")
# print(f"E(X) + E(Y) = {E_X + E_Y:.4f}")
# print(f"E(Z) = E(X+Y) = {E_Z:.4f}")
\end{codecell}
\end{exercicebox}

\begin{exercicebox}[Exercice 3 : Esperance Binomiale (Simulation)]
La théorie nous dit que pour $X \sim \text{Bin}(n, p)$, $E(X) = np$. Nous allons vérifier cela par simulation.

\textbf{Votre tâche :}
\begin{enumerate}
    \item Définissez les paramètres $n=20$ et $p=0.4$.
    \item Simulez 100 000 réalisations d'une variable aléatoire $X \sim \text{Bin}(n, p)$ en utilisant \texttt{numpy.random.binomial()}.
    \item Calculez la moyenne empirique de vos simulations.
    \item Comparez la moyenne empirique à l'espérance théorique $np$.
\end{enumerate}

\begin{codecell}
import numpy as np

n, p = 20, 0.4
N_simulations = 100000

# 1. Simuler N fois une loi Bin(n, p)
# resultats_bin = ...

# 2. Calculer la moyenne empirique
# moyenne_empirique = ...

# 3. Calculer la moyenne theorique
# moyenne_theorique = ...

# 4. Afficher
# print(f"Moyenne empirique: {moyenne_empirique:.4f}")
# print(f"Esperance theorique (np): {moyenne_theorique:.4f}")
\end{codecell}
\end{exercicebox}

\begin{exercicebox}[Exercice 4 : Esperance Geometrique (Simulation)]
Pour $X \sim \text{Geom}(p)$ (comptant les échecs), $E(X) = q/p$. Vérifions cela.

\textbf{Votre tâche :}
\begin{enumerate}
    \item Définissez $p=0.2$ (et $q=1-p$).
    \item Simulez 100 000 réalisations d'une variable $Y \sim \text{Geom}(p)$ en utilisant \texttt{numpy.random.geometric()}.
    \item \textbf{Attention :} \texttt{numpy.random.geometric} compte le nombre d'essais ($k=1, 2, \dots$). Pour obtenir $X$ (le nombre d'échecs, $k=0, 1, \dots$), vous devez soustraire 1 de chaque résultat.
    \item Calculez la moyenne empirique de $X$ (le nombre d'échecs).
    \item Comparez cette moyenne à l'espérance théorique $q/p$.
\end{enumerate}

\begin{codecell}
import numpy as np

p = 0.2
q = 1 - p
N_simulations = 100000

# 1. Simuler N fois une loi Geom(p) (nb d'essais)
# resultats_geom_essais = ...

# 3. Convertir en nombre d'echecs
# resultats_geom_echecs = ...

# 4. Calculer la moyenne empirique des echecs
# moyenne_empirique = ...

# 5. Calculer la moyenne theorique des echecs
# moyenne_theorique = ...

# 6. Afficher
# print(f"Moyenne empirique (echecs): {moyenne_empirique:.4f}")
# print(f"Esperance theorique (q/p): {moyenne_theorique:.4f}")
\end{codecell}
\end{exercicebox}

\begin{exercicebox}[Exercice 5 : Esperance et Variance de Bernoulli]
La variable aléatoire de Bernoulli $X \sim \text{Bern}(p)$ est la brique de base. Théoriquement, $E(X) = p$ et $\text{Var}(X) = p(1-p)$.

\textbf{Votre tâche :}
\begin{enumerate}
    \item Définissez $p=0.8$.
    \item Simulez 100 000 essais de Bernoulli (résultats 0 ou 1) avec probabilité $p$. (Indice : \texttt{numpy.random.choice} ou \texttt{numpy.random.binomial} avec $n=1$).
    \item Calculez l'espérance empirique (la moyenne) et la variance empirique (\texttt{numpy.var}).
    \item Comparez-les aux valeurs théoriques $p$ et $p(1-p)$.
\end{enumerate}

\begin{codecell}
import numpy as np

p = 0.8
N_simulations = 100000

# 1. Simuler N essais de Bernoulli
# essais = ...

# 2. Calculer l'esperance et la variance empiriques
# E_empirique = ...
# Var_empirique = ...

# 3. Calculer les valeurs theoriques
# E_theorique = ...
# Var_theorique = ...

# 4. Afficher
# print(f"Esperance: Empirique={E_empirique:.4f}, Theorique={E_theorique:.4f}")
# print(f"Variance:  Empirique={Var_empirique:.4f}, Theorique={Var_theorique:.4f}")
\end{codecell}
\end{exercicebox}
\subsection{Exercices}

% --- Section 1 : PDF & CDF (Calculs Intégraux de Base) ---

\begin{exercicebox}[Exercice 1 : Validation d'une PDF]
Soit la fonction $f(x) = cx^2$ pour $x \in [0, 1]$, et $f(x)=0$ sinon.
Calculez la valeur de la constante $c$ pour que $f(x)$ soit une fonction de densité de probabilité (PDF) valide.
\end{exercicebox}

\begin{exercicebox}[Exercice 2 : Calcul de Probabilité (PDF)]
En utilisant la PDF $f(x) = 3x^2$ pour $x \in [0, 1]$ (de l'exercice 1), calculez $P(X > 1/2)$.
\end{exercicebox}

\begin{exercicebox}[Exercice 3 : Calcul de la CDF]
Pour la PDF $f(x) = 3x^2$ sur $[0, 1]$, déterminez l'expression de la fonction de répartition (CDF) $F(x)$ pour $x \in [0, 1]$.
\end{exercicebox}

\begin{exercicebox}[Exercice 4 : Calcul de la PDF à partir de la CDF]
La CDF d'une variable aléatoire $X$ est donnée par :
$$F(x) = \begin{cases} 0 & \text{si } x < 0 \\ x^4 & \text{si } 0 \le x \le 1 \\ 1 & \text{si } x > 1 \end{cases}$$
Calculez la PDF $f(x)$ de $X$ pour $x \in [0, 1]$.
\end{exercicebox}

\begin{exercicebox}[Exercice 5 : Calcul de Probabilité (CDF)]
En utilisant la CDF $F(x) = x^4$ pour $x \in [0, 1]$ (de l'exercice 4), calculez $P(0.5 \le X \le 1)$.
\end{exercicebox}

\begin{exercicebox}[Exercice 6 : PDF (Trouver la constante)]
Soit la fonction $f(x) = c/x^3$ pour $x \in [1, 2]$, et $f(x)=0$ sinon.
Calculez la valeur de la constante $c$.
\end{exercicebox}

\begin{exercicebox}[Exercice 7 : Calcul de Probabilité (PDF non-polynomiale)]
En utilisant la PDF $f(x) = (8/3) \cdot (1/x^3)$ sur $[1, 2]$ (de l'exercice 6), calculez $P(X \le 1.5)$.
\end{exercicebox}

% --- Section 2 : Espérance, Variance, LOTUS (Calculs Intégraux) ---

\begin{exercicebox}[Exercice 8 : Calcul d'Espérance]
Soit $X$ une v.a. de PDF $f(x) = 3x^2$ sur $[0, 1]$. Calculez $E[X]$.
\end{exercicebox}

\begin{exercicebox}[Exercice 9 : Calcul de Moment (LOTUS)]
Pour la même PDF $f(x) = 3x^2$ sur $[0, 1]$, calculez $E[X^2]$.
\end{exercicebox}

\begin{exercicebox}[Exercice 10 : Calcul de Variance]
En utilisant les résultats des exercices 8 et 9, calculez $\text{Var}(X)$.
\end{exercicebox}

\begin{exercicebox}[Exercice 11 : Application de LOTUS]
Soit $X$ une v.a. de PDF $f(x) = 2x$ sur $[0, 1]$. Calculez $E[\sqrt{X}]$.
\end{exercicebox}

\begin{exercicebox}[Exercice 12 : Calcul d'Espérance]
En utilisant la PDF $f(x) = 2/x^3$ sur $[1, \infty)$, calculez $E[X]$.
\end{exercicebox}

\begin{exercicebox}[Exercice 13 : Calcul avec LOTUS]
En utilisant la PDF $f(x) = 2/x^3$ sur $[1, \infty)$, calculez $E[1/X]$.
\end{exercicebox}

% --- Section 3 : Loi Uniforme (Applications) ---

\begin{exercicebox}[Exercice 14 : Scénario (Loi Uniforme - Bus)]
Un bus arrive à un arrêt toutes les 15 minutes exactement. Vous arrivez à l'arrêt à un moment aléatoire. Votre temps d'attente $T$ suit une loi $T \sim \text{Unif}(0, 15)$. Calculez la probabilité que vous attendiez 3 minutes ou moins.
\end{exercicebox}

\begin{exercicebox}[Exercice 15 : Probabilité (Loi Uniforme - Bus)]
En utilisant le scénario de l'exercice 14 ($T \sim \text{Unif}(0, 15)$), calculez la probabilité que vous attendiez entre 5 et 10 minutes.
\end{exercicebox}

\begin{exercicebox}[Exercice 16 : Espérance et Variance (Loi Uniforme - Bus)]
Pour le temps d'attente $T \sim \text{Unif}(0, 15)$, calculez le temps d'attente moyen $E[T]$ et la variance $\text{Var}(T)$.
\end{exercicebox}

\begin{exercicebox}[Exercice 17 : Problème Inverse (Loi Uniforme)]
Un générateur de nombres aléatoires produit un nombre $X$ suivant $\text{Unif}(a, b)$. On sait que $E[X] = 5$ et $\text{Var}(X) = 3$.
En utilisant les formules $E[X] = (a+b)/2$ et $\text{Var}(X) = (b-a)^2/12$, trouvez $a$ et $b$.
\end{exercicebox}

\begin{exercicebox}[Exercice 18 : LOTUS (Loi Uniforme)]
Soit $X \sim \text{Unif}(0, 2)$. La PDF est $f(x)=1/2$ sur $[0, 2]$.
Calculez $E[X^3]$.
\end{exercicebox}

% --- Section 4 : Loi Exponentielle (Applications) ---

\begin{exercicebox}[Exercice 19 : Scénario (Loi Exponentielle - Appels)]
Les appels à un service client arrivent selon un processus de Poisson. Le temps $T$ (en minutes) entre deux appels suit une loi exponentielle avec $\lambda = 0.1$ appels/minute.
Calculez le temps moyen $E[T]$ entre les appels.
\end{exercicebox}

\begin{exercicebox}[Exercice 20 : Probabilité (Loi Exponentielle - Appels)]
En utilisant $T \sim \text{Exp}(0.1)$ de l'exercice 19, calculez la probabilité qu'il n'y ait pas d'appel pendant au moins 5 minutes, $P(T > 5)$.
\end{exercicebox}

\begin{exercicebox}[Exercice 21 : Probabilité (Loi Exponentielle - Appels)]
En utilisant $T \sim \text{Exp}(0.1)$, calculez la probabilité que le prochain appel arrive entre la 2ème et la 3ème minute, $P(2 \le T \le 3)$.
\end{exercicebox}

\begin{exercicebox}[Exercice 22 : Scénario (Loi Exponentielle - Durée de vie)]
La durée de vie (en années) d'un composant suit $T \sim \text{Exp}(\lambda)$. On sait que sa durée de vie moyenne $E[T]$ est de 8 ans.
Calculez la probabilité que le composant tombe en panne avant 2 ans, $P(T \le 2)$.
\end{exercicebox}

\begin{exercicebox}[Exercice 23 : Non-Mémoire (Loi Exponentielle)]
En utilisant le scénario de l'exercice 22 ($T \sim \text{Exp}(1/8)$), calculez la probabilité que le composant dure 10 ans de plus, sachant qu'il a déjà duré 5 ans : $P(T > 15 \mid T > 5)$.
\end{exercicebox}

\begin{exercicebox}[Exercice 24 : Problème Inverse (Loi Exponentielle)]
Soit $X \sim \text{Exp}(\lambda)$.
Trouvez $\lambda$ tel que $P(X \le 100) = 0.9$. (C'est-à-dire que 90% des observations sont inférieures à 100).
\end{exercicebox}

\begin{exercicebox}[Exercice 25 : Preuve de l'Espérance (Exponentielle)]
Soit $X \sim \text{Exp}(\lambda)$. Sa PDF est $f(x) = \lambda e^{-\lambda x}$ pour $x \ge 0$.
En utilisant l'intégration par parties sur $E[X] = \int_0^\infty x \lambda e^{-\lambda x} \, dx$, prouvez que $E[X] = 1/\lambda$.
\end{exercicebox}


\subsection{Corrections des Exercices}

\begin{correctionbox}[Correction Exercice 1 : Validation d'une PDF]
$$\int_0^1 cx^2 \, \mathrm{d}x = c \left[ \frac{x^3}{3} \right]_0^1 = c \left( \frac{1^3}{3} - 0 \right) = \frac{c}{3}$$
$$\frac{c}{3} = 1 \implies c = 3$$
\end{correctionbox}

\begin{correctionbox}[Correction Exercice 2 : Calcul de Probabilité (PDF)]
$f(x) = 3x^2$ sur $[0, 1]$.
$$P(X > 1/2) = \int_{1/2}^1 3x^2 \, \mathrm{d}x = \left[ x^3 \right]_{1/2}^1$$
$$= (1)^3 - (1/2)^3 = 1 - 1/8 = 7/8 \text{ (ou } 0.875 \text{)}$$
\end{correctionbox}

\begin{correctionbox}[Correction Exercice 3 : Calcul de la CDF]
Pour $x \in [0, 1]$ :
$$F(x) = \int_0^x f(t) \, \mathrm{d}t = \int_0^x 3t^2 \, \mathrm{d}t = \left[ t^3 \right]_0^x = x^3$$
\end{correctionbox}

\begin{correctionbox}[Correction Exercice 4 : Calcul de la PDF à partir de la CDF]
$f(x) = F'(x) = \frac{d}{dx} (x^4) = 4x^3$.
Donc, $f(x) = 4x^3$ pour $x \in [0, 1]$.
\end{correctionbox}

\begin{correctionbox}[Correction Exercice 5 : Calcul de Probabilité (CDF)]
$P(0.5 \le X \le 1) = F(1) - F(0.5) = (1)^4 - (0.5)^4 = 1 - 0.0625 = 0.9375$.
\end{correctionbox}

\begin{correctionbox}[Correction Exercice 6 : PDF (Trouver la constante)]
$$\int_1^2 \frac{c}{x^3} \, \mathrm{d}x = c \int_1^2 x^{-3} \, \mathrm{d}x = c \left[ \frac{x^{-2}}{-2} \right]_1^2 = c \left[ -\frac{1}{2x^2} \right]_1^2$$
$$= c \left( (-\frac{1}{2 \cdot 2^2}) - (-\frac{1}{2 \cdot 1^2}) \right) = c \left( -\frac{1}{8} + \frac{1}{2} \right) = c \left( \frac{3}{8} \right)$$
$$c(3/8) = 1 \implies c = 8/3$$
\end{correctionbox}

\begin{correctionbox}[Correction Exercice 7 : Calcul de Probabilité (PDF non-polynomiale)]
$f(x) = (8/3)x^{-3}$ sur $[1, 2]$.
$$P(X \le 1.5) = \int_1^{1.5} \frac{8}{3}x^{-3} \, \mathrm{d}x = \frac{8}{3} \left[ \frac{x^{-2}}{-2} \right]_1^{1.5} = \frac{8}{3} \left[ -\frac{1}{2x^2} \right]_1^{1.5}$$
$$= \frac{8}{3} \left( (-\frac{1}{2 \cdot (1.5)^2}) - (-\frac{1}{2 \cdot 1^2}) \right) = \frac{8}{3} \left( -\frac{1}{4.5} + \frac{1}{2} \right)$$
$$= \frac{8}{3} \left( -\frac{2}{9} + \frac{1}{2} \right) = \frac{8}{3} \left( \frac{-4+9}{18} \right) = \frac{8}{3} \left( \frac{5}{18} \right) = \frac{40}{54} = \frac{20}{27}$$
\end{correctionbox}

\begin{correctionbox}[Correction Exercice 8 : Calcul d'Espérance]
$f(x) = 3x^2$ sur $[0, 1]$.
$$E[X] = \int_0^1 x \cdot (3x^2) \, \mathrm{d}x = \int_0^1 3x^3 \, \mathrm{d}x = \left[ \frac{3x^4}{4} \right]_0^1 = 3/4$$
\end{correctionbox}

\begin{correctionbox}[Correction Exercice 9 : Calcul de Moment (LOTUS)]
$f(x) = 3x^2$ sur $[0, 1]$.
$$E[X^2] = \int_0^1 x^2 \cdot (3x^2) \, \mathrm{d}x = \int_0^1 3x^4 \, \mathrm{d}x = \left[ \frac{3x^5}{5} \right]_0^1 = 3/5$$
\end{correctionbox}

\begin{correctionbox}[Correction Exercice 10 : Calcul de Variance]
$\text{Var}(X) = E[X^2] - (E[X])^2 = (3/5) - (3/4)^2$
$$= \frac{3}{5} - \frac{9}{16} = \frac{48 - 45}{80} = \frac{3}{80}$$
\end{correctionbox}

\begin{correctionbox}[Correction Exercice 11 : Application de LOTUS]
$f(x) = 2x$ sur $[0, 1]$. $g(x) = \sqrt{X} = x^{1/2}$.
$$E[\sqrt{X}] = \int_0^1 x^{1/2} \cdot (2x) \, \mathrm{d}x = \int_0^1 2x^{3/2} \, \mathrm{d}x$$
$$= \left[ 2 \frac{x^{5/2}}{5/2} \right]_0^1 = \left[ \frac{4}{5} x^{5/2} \right]_0^1 = 4/5$$
\end{correctionbox}

\begin{correctionbox}[Correction Exercice 12 : Calcul d'Espérance]
$f(x) = 2/x^3$ sur $[1, \infty)$.
$$E[X] = \int_1^\infty x \cdot \frac{2}{x^3} \, \mathrm{d}x = \int_1^\infty \frac{2}{x^2} \, \mathrm{d}x = \left[ -\frac{2}{x} \right]_1^\infty$$
$$= \left( \lim_{b \to \infty} -\frac{2}{b} \right) - \left( -\frac{2}{1} \right) = 0 - (-2) = 2$$
\end{correctionbox}

\begin{correctionbox}[Correction Exercice 13 : Calcul avec LOTUS]
$f(x) = 2/x^3$ sur $[1, \infty)$. $g(x) = 1/x$.
$$E[1/X] = \int_1^\infty (1/x) \cdot \frac{2}{x^3} \, \mathrm{d}x = \int_1^\infty \frac{2}{x^4} \, \mathrm{d}x = \left[ \frac{2x^{-3}}{-3} \right]_1^\infty$$
$$= \left[ -\frac{2}{3x^3} \right]_1^\infty = (0) - (-\frac{2}{3}) = 2/3$$
\end{correctionbox}

\begin{correctionbox}[Correction Exercice 14 : Scénario (Loi Uniforme - Bus)]
$T \sim \text{Unif}(0, 15)$. $f(t) = 1/15$ sur $[0, 15]$.
$$P(T \le 3) = \int_0^3 \frac{1}{15} \, \mathrm{d}t = \frac{1}{15} [t]_0^3 = \frac{3}{15} = 0.2$$
\end{correctionbox}

\begin{correctionbox}[Correction Exercice 15 : Probabilité (Loi Uniforme - Bus)]
$T \sim \text{Unif}(0, 15)$.
$$P(5 \le T \le 10) = \int_5^{10} \frac{1}{15} \, \mathrm{d}t = \frac{1}{15} [t]_5^{10} = \frac{10-5}{15} = \frac{5}{15} = 1/3$$
\end{correctionbox}

\begin{correctionbox}[Correction Exercice 16 : Espérance et Variance (Loi Uniforme - Bus)]
$T \sim \text{Unif}(a=0, b=15)$.
$E[T] = \frac{a+b}{2} = \frac{0+15}{2} = 7.5$ minutes.
$\text{Var}(T) = \frac{(b-a)^2}{12} = \frac{(15-0)^2}{12} = \frac{225}{12} = \frac{75}{4} = 18.75 \text{ min}^2$.
\end{correctionbox}

\begin{correctionbox}[Correction Exercice 17 : Problème Inverse (Loi Uniforme)]
$E[X] = \frac{a+b}{2} = 5 \implies a+b = 10$.
$\text{Var}(X) = \frac{(b-a)^2}{12} = 3 \implies (b-a)^2 = 36 \implies b-a = 6$.
Système : (1) $b+a=10$, (2) $b-a=6$.
(1)+(2) : $2b = 16 \implies b = 8$.
(1) : $a = 10 - b = 10 - 8 = 2$.
$X \sim \text{Unif}(2, 8)$.
\end{correctionbox}

\begin{correctionbox}[Correction Exercice 18 : LOTUS (Loi Uniforme)]
$f(x) = 1/2$ sur $[0, 2]$. $g(x) = x^3$.
$$E[X^3] = \int_0^2 x^3 \cdot \frac{1}{2} \, \mathrm{d}x = \frac{1}{2} \left[ \frac{x^4}{4} \right]_0^2$$
$$= \frac{1}{2} \left( \frac{2^4}{4} - 0 \right) = \frac{1}{2} \left( \frac{16}{4} \right) = 2$$
\end{correctionbox}

\begin{correctionbox}[Correction Exercice 19 : Scénario (Loi Exponentielle - Appels)]
$T \sim \text{Exp}(0.1)$.
$E[T] = 1/\lambda = 1/0.1 = 10$ minutes.
\end{correctionbox}

\begin{correctionbox}[Correction Exercice 20 : Probabilité (Loi Exponentielle - Appels)]
$\lambda = 0.1$. On cherche $P(T > 5)$.
$$P(T > 5) = e^{-\lambda t} = e^{-0.1 \times 5} = e^{-0.5} \approx 0.6065$$
\end{correctionbox}

\begin{correctionbox}[Correction Exercice 21 : Probabilité (Loi Exponentielle - Appels)]
$\lambda = 0.1$. $P(2 \le T \le 3) = F(3) - F(2)$.
$F(t) = 1 - e^{-\lambda t}$.
$P = (1 - e^{-0.1 \times 3}) - (1 - e^{-0.1 \times 2}) = (1 - e^{-0.3}) - (1 - e^{-0.2})$
$$= e^{-0.2} - e^{-0.3} \approx 0.8187 - 0.7408 = 0.0779$$
\end{correctionbox}

\begin{correctionbox}[Correction Exercice 22 : Scénario (Loi Exponentielle - Durée de vie)]
$E[T] = 8 \implies \lambda = 1/8 = 0.125$.
On cherche $P(T \le 2)$.
$$P(T \le 2) = F(2) = 1 - e^{-\lambda t} = 1 - e^{-0.125 \times 2} = 1 - e^{-0.25}$$
$$\approx 1 - 0.7788 = 0.2212$$
\end{correctionbox}

\begin{correctionbox}[Correction Exercice 23 : Non-Mémoire (Loi Exponentielle)]
$\lambda = 1/8$. $P(T > 15 \mid T > 5) = P(T > 10+5 \mid T > 5)$.
Par non-mémoire, c'est $P(T > 10)$.
$$P(T > 10) = e^{-\lambda t} = e^{-(1/8) \times 10} = e^{-10/8} = e^{-1.25} \approx 0.2865$$
\end{correctionbox}

\begin{correctionbox}[Correction Exercice 24 : Problème Inverse (Loi Exponentielle)]
On cherche $\lambda$ tel que $F(100) = 0.9$.
$$F(100) = 1 - e^{-\lambda \times 100} = 0.9$$
$$0.1 = e^{-100\lambda}$$
$$\ln(0.1) = -100\lambda$$
$$-\ln(10) = -100\lambda$$
$$\lambda = \frac{\ln(10)}{100} \approx \frac{2.3026}{100} \approx 0.023$$
\end{correctionbox}

\begin{correctionbox}[Correction Exercice 25 : Preuve de l'Espérance (Exponentielle)]
$E[X] = \int_0^\infty x \lambda e^{-\lambda x} \, dx$.
IPP : $u=x \implies du=dx$ ; $dv=\lambda e^{-\lambda x}dx \implies v=-e^{-\lambda x}$.
$$E[X] = [u v]_0^\infty - \int_0^\infty v \, du = \left[ -x e^{-\lambda x} \right]_0^\infty - \int_0^\infty (-e^{-\lambda x}) \, dx$$
Le premier terme est $(0 - 0)$ (par croissance comparée).
$$E[X] = \int_0^\infty e^{-\lambda x} \, dx = \left[ -\frac{1}{\lambda} e^{-\lambda x} \right]_0^\infty$$
$$= \left( \lim_{b \to \infty} -\frac{1}{\lambda} e^{-\lambda b} \right) - \left( -\frac{1}{\lambda} e^0 \right) = 0 - (-1/\lambda) = 1/\lambda$$
\end{correctionbox}

\subsection{Exercices Python}

Les exercices suivants appliquent les concepts de variables aléatoires continues (PDF, CDF, espérance, variance) en utilisant la bibliothèque \texttt{NumPy} pour la simulation numérique afin de vérifier les résultats théoriques.

\begin{codecell}
import numpy as np
import math
\end{codecell}

\begin{exercicebox}[Exercice 1 : PDF CDF et Espérance (Simulation)]
Soit $X$ une v.a. continue avec la PDF $f(x) = 2x$ pour $x \in [0, 1]$, et $f(x)=0$ sinon.
Par calcul (que vous pouvez faire à la main), on trouve :
\begin{itemize}
    \item CDF : $F(x) = x^2$ (pour $x \in [0, 1]$)
    \item Espérance : $E[X] = 2/3$
\end{itemize}
Nous pouvons simuler cette variable en utilisant la méthode de la transformée inverse : si $U \sim \text{Unif}(0, 1)$, alors $X = F^{-1}(U) = \sqrt{U}$ suit la loi de $X$.

\textbf{Votre tâche (avec NumPy) :}
\begin{enumerate}
    \item Générer $N=100000$ échantillons $U$ d'une loi Uniforme(0, 1) avec \texttt{np.random.rand}.
    \item Transformer ces échantillons pour obtenir $N$ échantillons de $X$ (en prenant la racine carrée).
    \item Calculer l'espérance empirique $E[X]$ (la moyenne de vos échantillons $X$) et la comparer à la valeur théorique $2/3$.
\end{enumerate}
\end{exercicebox}

\begin{exercicebox}[Exercice 2 : Variance (Simulation)]
En utilisant les échantillons $X$ de l'exercice 1.
La valeur théorique (calculée à la main) de la variance est $\text{Var}(X) = 1/18$.

\textbf{Votre tâche (avec NumPy) :}
\begin{enumerate}
    \item Calculer la variance empirique $\text{Var}(X)$ de vos échantillons $X$ avec \texttt{np.var}.
    \item Comparer le résultat empirique à la valeur théorique $1/18$.
\end{enumerate}
\end{exercicebox}

\begin{exercicebox}[Exercice 3 : Loi Uniforme (Simulation vs Théorie)]
Soit $X \sim \text{Unif}(a=5, b=15)$. Les valeurs théoriques sont $E[X] = \frac{a+b}{2}$ et $\text{Var}(X) = \frac{(b-a)^2}{12}$.

\textbf{Votre tâche (avec NumPy) :}
\begin{enumerate}
    \item Calculer l'espérance et la variance théoriques.
    \item Générer $N=100000$ échantillons aléatoires de $X$ avec \texttt{np.random.uniform}.
    \item Calculer l'espérance empirique (\texttt{np.mean}) et la variance empirique (\texttt{np.var}) des échantillons.
    \item Comparer les résultats empiriques aux résultats théoriques.
\end{enumerate}
\end{exercicebox}

\begin{exercicebox}[Exercice 4 : Loi Uniforme (Vérification de la PDF)]
Pour $X \sim \text{Unif}(5, 15)$, la PDF est $f(x) = \frac{1}{10}$ sur $[5, 15]$.
La probabilité $P(7 \le X \le 10)$ est $\int_7^{10} \frac{1}{10} dx = \frac{10-7}{10} = 0.3$.

\textbf{Votre tâche (avec NumPy) :}
\begin{enumerate}
    \item Utiliser les échantillons de $X$ de l'exercice 3.
    \item Calculer la probabilité empirique $P(7 \le X \le 10)$ en comptant la proportion d'échantillons qui tombent dans cet intervalle.
    \item Comparer le résultat empirique à la valeur théorique $0.3$.
\end{enumerate}
\end{exercicebox}

\begin{exercicebox}[Exercice 5 : Loi Exponentielle (Simulation vs Théorie)]
Soit $X \sim \text{Exp}(\lambda=0.5)$. Les valeurs théoriques sont $E[X] = \frac{1}{\lambda}$ et $\text{Var}(X) = \frac{1}{\lambda^2}$.

Note : \texttt{np.random.exponential} prend un paramètre "scale" $\beta = 1/\lambda$.

\textbf{Votre tâche (avec NumPy) :}
\begin{enumerate}
    \item Définir $\lambda$ et calculer $E[X]$ et $\text{Var}(X)$ théoriques.
    \item Calculer le paramètre $\beta$ (scale) pour NumPy.
    \item Générer $N=100000$ échantillons aléatoires de $X$.
    \item Calculer et comparer les espérances et variances empiriques et théoriques.
\end{enumerate}
\end{exercicebox}

\begin{exercicebox}[Exercice 6 : Loi Exponentielle (Vérification de la CDF)]
Pour $X \sim \text{Exp}(\lambda=0.5)$, la CDF est $F(x) = 1 - e^{-\lambda x}$.
Calculons $P(X \le 3) = F(3) = 1 - e^{-0.5 \times 3}$.

\textbf{Votre tâche (avec NumPy) :}
\begin{enumerate}
    \item Calculer la valeur théorique $F(3)$.
    \item Utiliser les échantillons de $X$ de l'exercice 5.
    \item Calculer la probabilité empirique $P(X \le 3)$ en comptant la proportion d'échantillons $\le 3$.
    \item Comparer les deux valeurs.
\end{enumerate}
\end{exercicebox}

\begin{exercicebox}[Exercice 7 : Propriété de Non-Mémoire (Exponentielle)]
Nous allons vérifier numériquement la propriété de non-mémoire $P(X > s+t \mid X > s) = P(X > t)$ en utilisant les échantillons de $X$ de l'exercice 5 ($\lambda=0.5$).

\textbf{Votre tâche (avec NumPy) :}
\begin{enumerate}
    \item Choisir $s=1$ et $t=2$.
    \item Calculer $P(X > t)$ (théoriquement $e^{-\lambda t}$). Calculer la probabilité empirique (proportion d'échantillons $> t$).
    \item Calculer $P(X > s+t \mid X > s)$ empiriquement :
        \begin{itemize}
            \item Filtrer les échantillons pour ne garder que ceux où $X > s$.
            \item Parmi ce sous-ensemble, calculer la proportion de ceux où $X > s+t$.
        \end{itemize}
    \item Comparer les deux probabilités empiriques.
\end{enumerate}
\end{exercicebox}

\begin{exercicebox}[Exercice 8 : Théorème de Transfert (LOTUS)]
Soit $X \sim \text{Unif}(0, 2)$. La PDF est $f(x)=1/2$.
Soit $g(X) = X^2$. Nous voulons $E[g(X)] = E[X^2]$.
Théoriquement : $E[X^2] = \int_0^2 x^2 f(x) \, dx = \int_0^2 x^2 (1/2) \, dx = \frac{1}{2} [\frac{x^3}{3}]_0^2 = \frac{1}{2} (\frac{8}{3}) = 4/3$.

\textbf{Votre tâche (avec NumPy) :}
\begin{enumerate}
    \item Générer $N=100000$ échantillons $X \sim \text{Unif}(0, 2)$.
    \item Créer les échantillons $Y = g(X) = X^2$.
    \item Calculer l'espérance empirique $E[Y]$ (la moyenne de $Y$).
    \item Comparer le résultat empirique à la valeur théorique $4/3$.
\end{enumerate}
\end{exercicebox}

\begin{exercicebox}[Exercice 9 : Linéarité de l'Espérance (E[aX+b])]
Soit $X \sim \text{Unif}(5, 15)$ (de l'exercice 3). Nous savons que $E[X] = 10$.
Soit $Y = 5X - 3$.
Par linéarité, l'espérance théorique est $E[Y] = E[5X - 3] = 5E[X] - 3$.

\textbf{Votre tâche (avec NumPy) :}
\begin{enumerate}
    \item Calculer $E[Y]$ théoriquement en utilisant $E[X] = 10$.
    \item Utiliser les échantillons $X$ de l'exercice 3.
    \item Créer les échantillons $Y = 5 \times X - 3$.
    \item Calculer l'espérance empirique $E[Y]$ (la moyenne de $Y$).
    \item Comparer les deux résultats.
\end{enumerate}
\end{exercicebox}

\begin{exercicebox}[Exercice 10 : Propriétés de la Variance (Var(aX+b))]
Soit $X \sim \text{Unif}(5, 15)$ (de l'exercice 3). $\text{Var}(X) = \frac{(15-5)^2}{12} = 100/12 \approx 8.333$.
Soit $Y = 5X - 3$.
Théoriquement : $\text{Var}(Y) = \text{Var}(5X - 3) = \text{Var}(5X) = 5^2 \text{Var}(X) = 25 \times \text{Var}(X)$.

\textbf{Votre tâche (avec NumPy) :}
\begin{enumerate}
    \item Calculer $\text{Var}(Y)$ théoriquement en utilisant $\text{Var}(X) = 100/12$.
    \item Utiliser les échantillons $Y$ de l'exercice 9.
    \item Calculer la variance empirique $\text{Var}(Y)$ (avec \texttt{np.var}).
    \item Comparer les deux résultats.
\end{enumerate}
\end{exercicebox}
\newpage
\section{Espérance et autres notions associées aux variables aléatoires discrètes }

\subsection{Espérance d'une variable aléatoire discrète}

\begin{definitionbox}[Espérance]
L'espérance (ou valeur attendue) d'une variable aléatoire discrète $X$, qui prend les valeurs distinctes $x_1, x_2, \dots$, est définie par :
$$ E(X) = \sum_j x_j P(X=x_j) $$
\end{definitionbox}

\begin{intuitionbox}
L'espérance représente la valeur moyenne que l'on obtiendrait si l'on répétait l'expérience un très grand nombre de fois. C'est le \textbf{centre de gravité} de la distribution de probabilité. Si les probabilités étaient des masses placées sur une tige aux positions $x_j$, l'espérance serait le point d'équilibre.
\end{intuitionbox}

\begin{examplebox}[Lancer d'un dé]
Soit $X$ le résultat d'un lancer de dé équilibré. Chaque face a une probabilité de $1/6$. L'espérance est :
$$ E(X) = 1\left(\frac{1}{6}\right) + 2\left(\frac{1}{6}\right) + 3\left(\frac{1}{6}\right) + 4\left(\frac{1}{6}\right) + 5\left(\frac{1}{6}\right) + 6\left(\frac{1}{6}\right) = \frac{21}{6} = 3.5 $$
Même si 3.5 n'est pas un résultat possible, c'est la valeur moyenne sur un grand nombre de lancers.
\end{examplebox}

\subsection{Linéarité de l'espérance}

\begin{theorembox}[Linéarité de l'espérance]
Pour toutes variables aléatoires $X$ et $Y$, et pour toute constante $c$, on a :
\begin{align*}
E(X+Y) &= E(X) + E(Y) \\
E(cX) &= cE(X)
\end{align*}
Cette propriété est extrêmement puissante car elle ne requiert pas que $X$ et $Y$ soient indépendantes.
\end{theorembox}

\begin{intuitionbox}
Cette propriété formalise une idée très simple : "la moyenne d'une somme est la somme des moyennes". Si vous jouez à deux jeux de hasard, votre gain moyen total est simplement la somme de ce que vous gagnez en moyenne à chaque jeu, que les jeux soient liés ou non.
\end{intuitionbox}

\begin{examplebox}[Somme de deux dés]
Soit $X_1$ le résultat du premier dé et $X_2$ celui du second. On sait que $E(X_1) = 3.5$ et $E(X_2) = 3.5$.
Soit $S = X_1 + X_2$ la somme des deux dés. Grâce à la linéarité, on peut calculer l'espérance de la somme sans avoir à lister les 36 résultats possibles :
$$ E(S) = E(X_1 + X_2) = E(X_1) + E(X_2) = 3.5 + 3.5 = 7 $$
\end{examplebox}

\subsection{Espérance de la loi binomiale}

\begin{theorembox}[Espérance de la loi binomiale]
Si $X \sim \text{Bin}(n, p)$, alors son espérance est $E(X) = np$.
\end{theorembox}

\begin{intuitionbox}
Ce résultat est très naturel. Si vous lancez une pièce 100 fois ($n=100$) avec une probabilité de 50\% d'obtenir Pile ($p=0.5$), vous vous attendez en moyenne à obtenir $100 \times 0.5 = 50$ Piles. La formule $np$ généralise cette idée.
\end{intuitionbox}

\begin{proofbox}
Le calcul direct de l'espérance avec la PMF binomiale est possible, mais long. En utilisant la linéarité de l'espérance, on obtient une preuve beaucoup plus courte et élégante.
\newline
On peut voir une variable binomiale $X$ comme la somme de $n$ variables de Bernoulli indépendantes, $X = I_1 + I_2 + \dots + I_n$, où chaque $I_j$ représente le succès (1) ou l'échec (0) du $j$-ième essai.
\newline
Chaque $I_j$ a pour espérance $E(I_j) = 1 \cdot p + 0 \cdot (1-p) = p$.
\newline
Par linéarité de l'espérance, on a :
$$ E(X) = E(I_1) + E(I_2) + \dots + E(I_n) = \underbrace{p + p + \dots + p}_{n \text{ fois}} = np $$
\end{proofbox}

\subsection{Espérance de la loi géométrique}
\begin{theorembox}[Espérance de la loi géométrique]
L'espérance d'une variable aléatoire $X \sim \text{Geom}(p)$ (comptant le nombre d'échecs) est :
$$ E(X) = \frac{1-p}{p} = \frac{q}{p} $$
\end{theorembox}

\begin{intuitionbox}
Si un événement a 1 chance sur 10 de se produire ($p=0.1$), il est logique de penser qu'il faudra en moyenne 9 échecs ($q/p = 0.9/0.1=9$) avant qu'il ne se produise. L'espérance du nombre total d'essais (échecs + 1 succès) serait alors $1/p$.
\end{intuitionbox}

\begin{proofbox}[Démonstration de l'espérance géométrique via les séries entières]
Soit $X \sim \text{Geom}(p)$, où $X$ compte le nombre d'échecs avant le premier succès. La PMF est $P(X=k) = q^k p$ pour $k=0, 1, 2, \dots$, avec $q=1-p$.
\newline
Par définition, l'espérance est :
$$ E(X) = \sum_{k=0}^{\infty} k \cdot P(X=k) = \sum_{k=0}^{\infty} k q^k p $$
Le terme pour $k=0$ est nul, on peut donc commencer la somme à $k=1$ :
$$ E(X) = p \sum_{k=1}^{\infty} k q^k $$
L'astuce consiste à reconnaître que la somme ressemble à la dérivée d'une série géométrique. Rappelons la formule de la série géométrique pour $|q|<1$ :
$$ \sum_{k=0}^{\infty} q^k = \frac{1}{1-q} $$
En dérivant les deux côtés par rapport à $q$, on obtient :
$$ \frac{d}{dq} \left( \sum_{k=0}^{\infty} q^k \right) = \frac{d}{dq} \left( \frac{1}{1-q} \right) $$
$$ \sum_{k=1}^{\infty} k q^{k-1} = \frac{1}{(1-q)^2} $$
Pour faire apparaître ce terme dans notre formule d'espérance, on factorise $q$ dans la somme :
$$ E(X) = p \cdot q \sum_{k=1}^{\infty} k q^{k-1} $$
On peut maintenant remplacer la somme par son expression analytique :
$$ E(X) = p \cdot q \cdot \frac{1}{(1-q)^2} $$
Puisque $p = 1-q$, on a :
$$ E(X) = p \cdot q \cdot \frac{1}{p^2} = \frac{q}{p} $$
Ce qui démontre que l'espérance du nombre d'échecs avant le premier succès est $\frac{q}{p}$.
\end{proofbox}

\subsection{Loi du statisticien inconscient (LOTUS)}

\begin{theorembox}[Théorème de Transfert (LOTUS)]
Si $X$ est une variable aléatoire discrète et $g(x)$ est une fonction de $\mathbb{R}$ dans $\mathbb{R}$, alors l'espérance de la variable aléatoire $g(X)$ est donnée par :
$$ E[g(X)] = \sum_x g(x) P(X=x) $$
La somme porte sur toutes les valeurs possibles de $X$. Ce théorème est utile car il évite d'avoir à trouver la PMF de $g(X)$.
\end{theorembox}

\begin{intuitionbox}
Pour trouver la valeur moyenne d'une fonction d'une variable aléatoire (par exemple, le carré du résultat d'un dé), vous n'avez pas besoin de déterminer d'abord la distribution de ce carré. Vous pouvez simplement prendre chaque valeur possible du résultat original, lui appliquer la fonction, et pondérer ce nouveau résultat par la probabilité du résultat original.
\end{intuitionbox}

\begin{examplebox}[Calcul de $E(X^2)$ pour un dé]
Soit $X$ le résultat d'un lancer de dé. Calculons l'espérance de $Y=X^2$. La fonction est $g(x)=x^2$.
\begin{align*}
E(X^2) &= \sum_{k=1}^6 k^2 P(X=k) \\
&= 1^2\left(\frac{1}{6}\right) + 2^2\left(\frac{1}{6}\right) + 3^2\left(\frac{1}{6}\right) + 4^2\left(\frac{1}{6}\right) + 5^2\left(\frac{1}{6}\right) + 6^2\left(\frac{1}{6}\right) \\
&= \frac{1+4+9+16+25+36}{6} = \frac{91}{6} \approx 15.17
\end{align*}
\end{examplebox}

\subsection{Variance}

\begin{definitionbox}[Variance et écart-type]
La \textbf{variance} d'une variable aléatoire $X$ mesure la dispersion de sa distribution autour de son espérance. Elle est définie par :
$$ \text{Var}(X) = E\left[ (X - E(X))^2 \right] $$
La racine carrée de la variance est appelée l' \textbf{écart-type} :
$$ \text{SD}(X) = \sqrt{\text{Var}(X)} $$
\end{definitionbox}

\begin{intuitionbox}
La variance est la "distance carrée moyenne à la moyenne". On prend l'écart de chaque valeur par rapport à la moyenne, on le met au carré (pour que les écarts positifs et négatifs ne s'annulent pas), puis on en calcule la moyenne. L'écart-type est souvent plus interprétable car il ramène cette mesure de dispersion dans les mêmes unités que la variable aléatoire elle-même.
\end{intuitionbox}

\begin{theorembox}[Formule de calcul de la variance]
Pour toute variable aléatoire $X$, une formule plus pratique pour le calcul de la variance est :
$$ \text{Var}(X) = E(X^2) - [E(X)]^2 $$
\end{theorembox}

\begin{examplebox}[Variance d'un lancer de dé]
Nous avons déjà calculé pour un dé que $E(X) = 3.5$ et $E(X^2) = 91/6$. On peut maintenant trouver la variance facilement :
$$ \text{Var}(X) = E(X^2) - [E(X)]^2 = \frac{91}{6} - (3.5)^2 = \frac{91}{6} - 12.25 = 15.166... - 12.25 \approx 2.917 $$
L'écart-type est $\text{SD}(X) = \sqrt{2.917} \approx 1.708$.
\end{examplebox}

\subsection{Exercices}

\begin{exercicebox}[Espérance d'un jeu]
Un jeu consiste à lancer un dé à six faces. Si vous obtenez un 6, vous gagnez 10€. Si vous obtenez un 4 ou un 5, vous gagnez 1€. Sinon, vous ne gagnez rien. Quelle est l'espérance de gain pour une partie ?
\end{exercicebox}

\begin{correctionbox}
Soit $X$ la variable aléatoire représentant le gain. Les valeurs possibles de $X$ sont 10, 1, et 0.
Les probabilités associées sont :
$P(X=10) = P(\text{obtenir un 6}) = 1/6$.
$P(X=1) = P(\text{obtenir un 4 ou 5}) = 2/6 = 1/3$.
$P(X=0) = P(\text{obtenir 1, 2 ou 3}) = 3/6 = 1/2$.

L'espérance de gain est :
$$ E(X) = 10 \cdot P(X=10) + 1 \cdot P(X=1) + 0 \cdot P(X=0) $$
$$ E(X) = 10 \cdot \frac{1}{6} + 1 \cdot \frac{2}{6} + 0 \cdot \frac{3}{6} = \frac{10+2}{6} = \frac{12}{6} = 2 $$
L'espérance de gain est de 2€ par partie.
\end{correctionbox}

\begin{exercicebox}[Linéarité de l'espérance]
On lance deux dés non truqués. Soit $X$ la somme des résultats. Calculez $E(X)$ en utilisant la linéarité de l'espérance.
\end{exercicebox}

\begin{correctionbox}
Soit $D_1$ le résultat du premier dé et $D_2$ le résultat du second dé. On a $X = D_1 + D_2$.
L'espérance du résultat d'un seul dé est $E(D_1) = E(D_2) = 3.5$.
Par linéarité de l'espérance :
$$ E(X) = E(D_1 + D_2) = E(D_1) + E(D_2) = 3.5 + 3.5 = 7 $$
L'espérance de la somme est 7.
\end{correctionbox}

\begin{exercicebox}[Espérance binomiale]
Un étudiant répond au hasard à un QCM de 20 questions, chaque question ayant 4 options de réponse (une seule correcte). Quelle est l'espérance du nombre de bonnes réponses ?
\end{exercicebox}

\begin{correctionbox}
Soit $X$ le nombre de bonnes réponses. Chaque question est une épreuve de Bernoulli avec une probabilité de succès $p=1/4$. Le nombre total d'épreuves est $n=20$.
$X$ suit donc une loi binomiale $X \sim \text{Bin}(n=20, p=0.25)$.
L'espérance d'une loi binomiale est $E(X) = np$.
$$ E(X) = 20 \times 0.25 = 5 $$
L'étudiant peut s'attendre à avoir 5 bonnes réponses en moyenne.
\end{correctionbox}

\begin{exercicebox}[Loi Géométrique]
On lance une pièce de monnaie jusqu'à obtenir "Pile" pour la première fois. La probabilité d'obtenir "Pile" est $p=0.5$.
\begin{enumerate}
    \item Quelle est la probabilité que le premier "Pile" apparaisse au 4ème lancer (c'est-à-dire après 3 "Face") ?
    \item Quel est le nombre moyen d'échecs ("Face") attendu avant le premier succès ?
\end{enumerate}
\end{exercicebox}

\begin{correctionbox}
Soit $X$ le nombre d'échecs avant le premier succès. $X \sim \text{Geom}(p=0.5)$.
1. On cherche $P(X=3)$. La PMF est $P(X=k) = (1-p)^k p$.
$$ P(X=3) = (0.5)^3 \times 0.5 = 0.125 \times 0.5 = 0.0625 $$
La probabilité est de 6.25\%.

2. On cherche l'espérance $E(X)$.
$$ E(X) = \frac{1-p}{p} = \frac{0.5}{0.5} = 1 $$
On s'attend en moyenne à 1 échec ("Face") avant le premier "Pile".
\end{correctionbox}

\begin{exercicebox}[Variance d'un dé]
Calculez la variance du résultat $X$ d'un lancer de dé équilibré.
\end{exercicebox}

\begin{correctionbox}
On utilise la formule $\text{Var}(X) = E(X^2) - [E(X)]^2$.
On sait déjà que $E(X)=3.5$.
Calculons $E(X^2)$ avec LOTUS :
$$ E(X^2) = \sum_{k=1}^6 k^2 P(X=k) = \frac{1}{6}(1^2+2^2+3^2+4^2+5^2+6^2) $$
$$ E(X^2) = \frac{1}{6}(1+4+9+16+25+36) = \frac{91}{6} \approx 15.167 $$
Maintenant, la variance :
$$ \text{Var}(X) = \frac{91}{6} - (3.5)^2 = \frac{91}{6} - 12.25 = \frac{91 - 73.5}{6} = \frac{17.5}{6} \approx 2.917 $$
\end{correctionbox}

\begin{exercicebox}[Loi de Poisson]
Un livre de 500 pages contient 250 fautes de frappe distribuées au hasard.
\begin{enumerate}
    \item Quel est le nombre moyen de fautes par page ?
    \item Quelle est la probabilité qu'une page choisie au hasard contienne exactement 2 fautes ?
\end{enumerate}
\end{exercicebox}

\begin{correctionbox}
1. Le taux moyen d'erreurs est $\lambda = \frac{250 \text{ fautes}}{500 \text{ pages}} = 0.5$ fautes par page.
Le nombre de fautes par page, $X$, suit une loi de Poisson $X \sim \text{Poisson}(\lambda=0.5)$.

2. On cherche $P(X=2)$. La PMF de Poisson est $P(X=k) = \frac{e^{-\lambda}\lambda^k}{k!}$.
$$ P(X=2) = \frac{e^{-0.5}(0.5)^2}{2!} = \frac{0.6065 \times 0.25}{2} \approx 0.0758 $$
La probabilité est d'environ 7.58\%.
\end{correctionbox}

\begin{exercicebox}[Variance et constante]
Soit $X$ une variable aléatoire avec $E(X)=10$ et $\text{Var}(X)=2$. Calculez l'espérance et la variance de $Y = 3X + 5$.
\end{exercicebox}

\begin{correctionbox}
On utilise les propriétés de l'espérance et de la variance.
Pour l'espérance :
$E(Y) = E(3X+5) = E(3X) + E(5) = 3E(X) + 5$.
$$ E(Y) = 3(10) + 5 = 35 $$
Pour la variance :
$\text{Var}(Y) = \text{Var}(3X+5) = \text{Var}(3X) = 3^2 \text{Var}(X)$.
$$ \text{Var}(Y) = 9 \times 2 = 18 $$
\end{correctionbox}

\begin{exercicebox}[Variable indicatrice]
On lance une pièce deux fois. Soit $A$ l'événement "obtenir au moins un Pile". Soit $I_A$ la variable indicatrice de cet événement. Donnez la PMF, l'espérance et la variance de $I_A$.
\end{exercicebox}

\begin{correctionbox}
L'univers est $\{PP, PF, FP, FF\}$. L'événement $A$ est $\{PP, PF, FP\}$.
La probabilité de A est $P(A) = 3/4$.
La variable $I_A$ vaut 1 si $A$ se produit, 0 sinon. C'est une loi de Bernoulli.
$I_A \sim \text{Bern}(p=3/4)$.

PMF : $P(I_A=1) = 3/4$ et $P(I_A=0) = 1/4$.
Espérance : $E(I_A) = p = 3/4$.
Variance : $\text{Var}(I_A) = p(1-p) = \frac{3}{4} \left(1-\frac{3}{4}\right) = \frac{3}{4} \cdot \frac{1}{4} = \frac{3}{16}$.
\end{correctionbox}

\begin{exercicebox}[Poisson comme approximation]
Une compagnie aérienne observe que 0.2\% des passagers qui réservent un vol ne se présentent pas. Si un avion a 200 sièges et que la compagnie vend 200 billets, quelle est la probabilité qu'exactement 3 passagers ne se présentent pas ? Utilisez l'approximation de Poisson.
\end{exercicebox}

\begin{correctionbox}
C'est une situation binomiale avec $n=200$ (grand) et $p=0.002$ (petit). On peut l'approximer par une loi de Poisson.
Le paramètre $\lambda$ est $\lambda = np = 200 \times 0.002 = 0.4$.
Soit $X$ le nombre de passagers absents, $X \sim \text{Poisson}(0.4)$. On cherche $P(X=3)$.
$$ P(X=3) = \frac{e^{-0.4}(0.4)^3}{3!} = \frac{0.6703 \times 0.064}{6} \approx 0.00715 $$
La probabilité est d'environ 0.715\%.
\end{correctionbox}

\begin{exercicebox}[LOTUS]
Une variable aléatoire discrète $X$ a la PMF suivante : $P(X=-1)=0.2$, $P(X=0)=0.5$, $P(X=1)=0.3$. Calculez $E[ (X+1)^2 ]$.
\end{exercicebox}

\begin{correctionbox}
On utilise le théorème de transfert (LOTUS) avec la fonction $g(x) = (x+1)^2$.
$$ E[g(X)] = \sum_x g(x) P(X=x) $$
$$ E[(X+1)^2] = (-1+1)^2 P(X=-1) + (0+1)^2 P(X=0) + (1+1)^2 P(X=1) $$
$$ E[(X+1)^2] = (0)^2 \cdot (0.2) + (1)^2 \cdot (0.5) + (2)^2 \cdot (0.3) $$
$$ E[(X+1)^2] = 0 + 1 \cdot 0.5 + 4 \cdot 0.3 = 0.5 + 1.2 = 1.7 $$
\end{correctionbox}

\begin{exercicebox}[Espérance d'une fonction]
Soit $X$ une variable aléatoire représentant le résultat d'un lancer d'un dé équilibré à 4 faces (valeurs 1, 2, 3, 4). Calculez l'espérance de $Y = 1/X$.
\end{exercicebox}

\begin{correctionbox}
Chaque face a une probabilité de $1/4$. On utilise le théorème de transfert (LOTUS) avec $g(x) = 1/x$.
$$ E(Y) = E(1/X) = \sum_{k=1}^4 \frac{1}{k} P(X=k) $$
$$ E(1/X) = \frac{1}{1}\left(\frac{1}{4}\right) + \frac{1}{2}\left(\frac{1}{4}\right) + \frac{1}{3}\left(\frac{1}{4}\right) + \frac{1}{4}\left(\frac{1}{4}\right) $$
$$ E(1/X) = \frac{1}{4} \left(1 + \frac{1}{2} + \frac{1}{3} + \frac{1}{4}\right) = \frac{1}{4} \left(\frac{12+6+4+3}{12}\right) = \frac{1}{4} \cdot \frac{25}{12} = \frac{25}{48} \approx 0.521 $$
\end{correctionbox}

\begin{exercicebox}[Loi Géométrique : "Sans mémoire"]
Un archer touche sa cible avec une probabilité $p=1/3$. Sachant qu'il a déjà manqué ses 5 premiers tirs, quelle est la probabilité qu'il touche la cible au 8ème tir (c'est-à-dire après 2 échecs supplémentaires) ?
\end{exercicebox}

\begin{correctionbox}
La loi géométrique est "sans mémoire". Le fait qu'il ait déjà manqué 5 tirs ne change pas les probabilités pour les tirs futurs. Le problème revient à se demander la probabilité qu'il lui faille 3 tirs supplémentaires pour réussir, ce qui équivaut à 2 échecs suivis d'un succès.
Soit $X$ le nombre d'échecs avant le premier succès. On cherche $P(X=2)$.
$$ P(X=2) = (1-p)^2 p = \left(\frac{2}{3}\right)^2 \left(\frac{1}{3}\right) = \frac{4}{9} \cdot \frac{1}{3} = \frac{4}{27} $$
\end{correctionbox}

\begin{exercicebox}[Loi de Poisson : Événements rares]
Dans une grande forêt, on observe en moyenne 0.8 ours par kilomètre carré. On étudie une zone de 5 km².
\begin{enumerate}
    \item Quel est le nombre moyen d'ours attendu dans cette zone ?
    \item Quelle est la probabilité de n'observer aucun ours dans cette zone ?
\end{enumerate}
\end{exercicebox}

\begin{correctionbox}
1. Le taux d'ours est de 0.8 par km². Pour une zone de 5 km², le paramètre $\lambda$ de la loi de Poisson est :
$$ \lambda = 0.8 \text{ ours/km²} \times 5 \text{ km²} = 4 $$
On s'attend donc à observer 4 ours en moyenne dans cette zone.

2. Soit $X$ le nombre d'ours observés, $X \sim \text{Poisson}(\lambda=4)$. On cherche $P(X=0)$.
$$ P(X=0) = \frac{e^{-4} 4^0}{0!} = e^{-4} \approx 0.0183 $$
La probabilité de n'observer aucun ours est d'environ 1.83\%.
\end{correctionbox}

\begin{exercicebox}[Variance d'une loi de Bernoulli]
Soit $X \sim \text{Bern}(p)$. Montrez en utilisant la formule $\text{Var}(X) = E(X^2) - [E(X)]^2$ que $\text{Var}(X) = p(1-p)$.
\end{exercicebox}

\begin{correctionbox}
Pour une variable de Bernoulli, $X$ ne peut prendre que les valeurs 0 et 1.
L'espérance est $E(X) = 1 \cdot p + 0 \cdot (1-p) = p$.
Pour calculer $E(X^2)$, on utilise LOTUS. Comme $X$ ne prend que les valeurs 0 et 1, $X^2$ est identique à $X$.
En effet, $0^2=0$ et $1^2=1$. Donc $X^2=X$.
Par conséquent, $E(X^2) = E(X) = p$.
On applique la formule de la variance :
$$ \text{Var}(X) = E(X^2) - [E(X)]^2 = p - p^2 = p(1-p) $$
\end{correctionbox}

\begin{exercicebox}[Loi Hypergéométrique]
Dans un groupe de 10 amis (6 hommes et 4 femmes), on tire au sort 3 personnes pour organiser une fête. Quelle est la probabilité que le groupe tiré au sort soit composé exclusivement de femmes ?
\end{exercicebox}

\begin{correctionbox}
Il s'agit d'un tirage sans remise. Soit $X$ le nombre de femmes tirées.
$X \sim \text{HG}(w=4 \text{ femmes}, b=6 \text{ hommes}, m=3 \text{ tirages})$.
On cherche $P(X=3)$.
$$ P(X=3) = \frac{\binom{\text{femmes}}{3} \binom{\text{hommes}}{0}}{\binom{\text{total}}{3}} = \frac{\binom{4}{3} \binom{6}{0}}{\binom{10}{3}} $$
$$ \binom{4}{3} = 4 \quad ; \quad \binom{6}{0} = 1 \quad ; \quad \binom{10}{3} = \frac{10 \cdot 9 \cdot 8}{3 \cdot 2 \cdot 1} = 120 $$
$$ P(X=3) = \frac{4 \times 1}{120} = \frac{4}{120} = \frac{1}{30} \approx 0.0333 $$
La probabilité est d'environ 3.33\%.
\end{correctionbox}

\begin{exercicebox}[Loi Binomiale : "Au moins"]
Un test de dépistage rapide a une probabilité de 0.1 de donner un faux positif. Si 10 personnes saines passent ce test, quelle est la probabilité qu'au moins deux d'entre elles reçoivent un faux positif ?
\end{exercicebox}

\begin{correctionbox}
Soit $X$ le nombre de faux positifs. $X \sim \text{Bin}(n=10, p=0.1)$.
On cherche $P(X \ge 2)$. Il est plus simple de calculer le complémentaire : $1 - P(X < 2)$.
$P(X < 2) = P(X=0) + P(X=1)$.
$$ P(X=0) = \binom{10}{0}(0.1)^0(0.9)^{10} = (0.9)^{10} \approx 0.3487 $$
$$ P(X=1) = \binom{10}{1}(0.1)^1(0.9)^9 = 10 \cdot 0.1 \cdot (0.9)^9 \approx 0.3874 $$
$P(X < 2) \approx 0.3487 + 0.3874 = 0.7361$.
$$ P(X \ge 2) = 1 - 0.7361 = 0.2639 $$
La probabilité d'avoir au moins deux faux positifs est d'environ 26.39\%.
\end{correctionbox}

\begin{exercicebox}[Écart-type]
Une machine distribue des boissons. La quantité versée $X$ (en cL) a pour espérance $E(X)=20$ et on a $E(X^2)=404$. Quel est l'écart-type de la quantité versée ?
\end{exercicebox}

\begin{correctionbox}
L'écart-type est la racine carrée de la variance. Calculons d'abord la variance.
$$ \text{Var}(X) = E(X^2) - [E(X)]^2 = 404 - (20)^2 = 404 - 400 = 4 $$
La variance est de 4 cL².
L'écart-type est :
$$ \text{SD}(X) = \sqrt{\text{Var}(X)} = \sqrt{4} = 2 $$
L'écart-type est de 2 cL.
\end{correctionbox}

\begin{exercicebox}[Espérance et prise de décision]
Vous avez le choix entre deux loteries.
Loterie A : Vous gagnez 100€ avec une probabilité de 0.1, sinon rien.
Loterie B : Vous gagnez 20€ avec une probabilité de 0.4, sinon rien.
Quelle loterie est la plus avantageuse en termes d'espérance de gain ?
\end{exercicebox}

\begin{correctionbox}
Calculons l'espérance de gain pour chaque loterie.
Soit $G_A$ le gain de la loterie A.
$$ E(G_A) = 100 \cdot P(G_A=100) + 0 \cdot P(G_A=0) = 100 \cdot 0.1 = 10€ $$
Soit $G_B$ le gain de la loterie B.
$$ E(G_B) = 20 \cdot P(G_B=20) + 0 \cdot P(G_B=0) = 20 \cdot 0.4 = 8€ $$
L'espérance de gain de la loterie A (10€) est supérieure à celle de la loterie B (8€). La loterie A est donc plus avantageuse en moyenne.
\end{correctionbox}

\begin{exercicebox}[Poisson : Somme de variables]
Deux sources radioactives émettent des particules indépendamment. La source 1 émet des particules selon une loi de Poisson de paramètre $\lambda_1=2$ par minute. La source 2 suit une loi de Poisson de paramètre $\lambda_2=3$ par minute. Quelle est la probabilité qu'un total de 4 particules soit émis en une minute ?
\end{exercicebox}

\begin{correctionbox}
Une propriété importante de la loi de Poisson est que la somme de deux variables de Poisson indépendantes est aussi une variable de Poisson dont le paramètre est la somme des paramètres.
Soit $X_1 \sim \text{Poisson}(2)$ et $X_2 \sim \text{Poisson}(3)$.
Le nombre total de particules $Y = X_1 + X_2$ suit une loi de Poisson de paramètre $\lambda = \lambda_1 + \lambda_2 = 2+3=5$.
Donc, $Y \sim \text{Poisson}(5)$.
On cherche $P(Y=4)$.
$$ P(Y=4) = \frac{e^{-5} 5^4}{4!} = \frac{e^{-5} \cdot 625}{24} \approx 0.1755 $$
La probabilité est d'environ 17.55\%.
\end{correctionbox}

\begin{exercicebox}[Contexte et choix du modèle]
Une petite ville compte 5000 habitants. En moyenne, 1 personne sur 1000 est allergique à une substance X.
\begin{enumerate}
    \item Quel modèle (Binomial ou Poisson) utiliseriez-vous pour estimer la probabilité qu'il y ait exactement 5 personnes allergiques dans cette ville ? Justifiez.
    \item Calculez cette probabilité.
\end{enumerate}
\end{exercicebox}

\begin{correctionbox}
1. Le modèle exact est une loi binomiale avec $n=5000$ et $p=1/1000=0.001$. Cependant, comme $n$ est très grand et $p$ est très petit, la loi de Poisson est une excellente approximation et bien plus simple à calculer. On utilisera donc une loi de Poisson.

2. Le paramètre de la loi de Poisson est $\lambda = np = 5000 \times 0.001 = 5$.
Soit $X$ le nombre de personnes allergiques, $X \sim \text{Poisson}(5)$.
On cherche $P(X=5)$.
$$ P(X=5) = \frac{e^{-5} 5^5}{5!} = \frac{e^{-5} \cdot 3125}{120} \approx 0.1755 $$
La probabilité est d'environ 17.55\%.
\end{correctionbox}
\subsection{Exercices}

% --- Lois Jointes et Marginales (Discret) ---

\begin{exercicebox}[Exercice 1 : Loi Jointe et Marginales]
Soit le tableau suivant représentant la loi de probabilité jointe $P(X=x, Y=y)$ d'un couple de variables aléatoires $(X, Y)$.

\begin{center}
\begin{tabular}{|c|ccc|}
\hline
\diagbox{$X$}{$Y$} & 0 & 1 & 2 \\ \hline
0 & 0.1 & 0.2 & 0.1 \\
1 & 0.3 & 0.1 & 0.2 \\ \hline
\end{tabular}
\end{center}

\begin{enumerate}
    \item Vérifiez qu'il s'agit bien d'une loi de probabilité.
    \item Calculez la loi marginale de $X$, $P(X=x)$.
    \item Calculez la loi marginale de $Y$, $P(Y=y)$.
\end{enumerate}
\end{exercicebox}

\begin{exercicebox}[Exercice 2 : Calcul de Probabilité Jointe]
En utilisant la loi jointe de l'exercice 1 :
\begin{enumerate}
    \item Calculez $P(X=0, Y \le 1)$.
    \item Calculez $P(X=Y)$.
    \item Calculez $P(X > Y)$.
\end{enumerate}
\end{exercicebox}

\begin{exercicebox}[Exercice 3 : Indépendance (Loi Jointe)]
En utilisant la loi jointe de l'exercice 1 :
\begin{enumerate}
    \item Calculez $P(X=0) \times P(Y=0)$.
    \item Comparez ce résultat à $P(X=0, Y=0)$.
    \item Les variables $X$ et $Y$ sont-elles indépendantes ? Justifiez.
\end{enumerate}
\end{exercicebox}

% --- Espérance, Covariance et Corrélation ---

\begin{exercicebox}[Exercice 4 : Espérances Marginales]
En utilisant les lois marginales calculées à l'exercice 1 :
\begin{enumerate}
    \item Calculez l'espérance $E[X]$.
    \item Calculez l'espérance $E[Y]$.
\end{enumerate}
\end{exercicebox}

\begin{exercicebox}[Exercice 5 : Espérance d'une Fonction (LOTUS)]
En utilisant la loi jointe de l'exercice 1, calculez $E[XY]$.
(Indice : $E[XY] = \sum_x \sum_y (xy) P(X=x, Y=y)$).
\end{exercicebox}

\begin{exercicebox}[Exercice 6 : Covariance (Calcul)]
En utilisant les résultats des exercices 4 et 5, calculez la covariance $\text{Cov}(X,Y)$.
\end{exercicebox}

\begin{exercicebox}[Exercice 7 : Variances Marginales]
En utilisant les lois marginales de l'exercice 1 et les espérances de l'exercice 4 :
\begin{enumerate}
    \item Calculez $E[X^2]$ et $\text{Var}(X)$.
    \item Calculez $E[Y^2]$ et $\text{Var}(Y)$.
\end{enumerate}
\end{exercicebox}

\begin{exercicebox}[Exercice 8 : Corrélation (Calcul)]
En utilisant les résultats des exercices 6 et 7, calculez le coefficient de corrélation $\text{Corr}(X,Y)$.
\end{exercicebox}

% --- Propriétés de la Variance et de la Covariance ---

\begin{exercicebox}[Exercice 9 : Variance d'une Somme (Non Indépendant)]
Soient $X$ et $Y$ deux variables aléatoires telles que $\text{Var}(X) = 10$, $\text{Var}(Y) = 5$ et $\text{Cov}(X,Y) = 2$.
Calculez $\text{Var}(X+Y)$.
\end{exercicebox}

\begin{exercicebox}[Exercice 10 : Variance d'une Différence (Indépendant)]
Soient $X$ et $Y$ deux variables aléatoires \textbf{indépendantes} telles que $\text{Var}(X) = 16$ et $\text{Var}(Y) = 9$.
\begin{enumerate}
    \item Que vaut $\text{Cov}(X,Y)$ ?
    \item Calculez $\text{Var}(X-Y)$. (Rappel : $\text{Var}(X-Y) = \text{Var}(X) + \text{Var}(Y) - 2\text{Cov}(X,Y)$).
\end{enumerate}
\end{exercicebox}

\begin{exercicebox}[Exercice 11 : Bilinéarité de la Covariance]
Soient $X, Y, Z$ trois variables aléatoires. Exprimez $\text{Cov}(X+Y, Z)$ en fonction des covariances des variables individuelles.
\end{exercicebox}

\begin{exercicebox}[Exercice 12 : Variance d'une Combinaison Linéaire]
Soient $X$ et $Y$ deux variables aléatoires indépendantes avec $\text{Var}(X) = 4$ et $\text{Var}(Y) = 2$.
Calculez $\text{Var}(3X - 5Y + 1)$.
\end{exercicebox}

\begin{exercicebox}[Exercice 13 : Variance d'une Somme (Dés)]
On lance deux dés équilibrés $D_1$ et $D_2$. Soit $S = D_1 + D_2$.
On rappelle que pour un dé, $\text{Var}(D_i) = 35/12$.
\begin{enumerate}
    \item Les variables $D_1$ et $D_2$ sont-elles indépendantes ?
    \item Calculez $\text{Var}(S)$.
\end{enumerate}
\end{exercicebox}

\begin{exercicebox}[Exercice 14 : Covariance et Variance]
Soit $X$ une variable aléatoire. En utilisant la bilinéarité de la covariance, montrez que $\text{Cov}(X, X) = \text{Var}(X)$.
\end{exercicebox}

\begin{exercicebox}[Exercice 15 : Covariance avec une Constante]
Soit $X$ une variable aléatoire et $c$ une constante.
Montrez que $\text{Cov}(X, c) = 0$. (Indice : $E[c]=c$ et $E[Xc] = cE[X]$).
\end{exercicebox}

% --- Standardisation et Somme de Poissons ---

\begin{exercicebox}[Exercice 16 : Standardisation (Centrer-Réduire)]
Soit $X$ une variable aléatoire avec $E[X] = 10$ et $\text{Var}(X) = 4$.
Soit $Z = \frac{X - E[X]}{\sqrt{\text{Var}(X)}} = \frac{X - 10}{2}$ la variable standardisée.
\begin{enumerate}
    \item Calculez $E[Z]$.
    \item Calculez $\text{Var}(Z)$.
\end{enumerate}
\end{exercicebox}

\begin{exercicebox}[Exercice 17 : Corrélation et Standardisation]
Soit $\text{Corr}(X,Y) = 0.5$. Soient $Z_X$ et $Z_Y$ les versions standardisées de $X$ et $Y$.
Que vaut $\text{Cov}(Z_X, Z_Y)$ ? (Indice : regardez l'intuition de la corrélation).
\end{exercicebox}

\begin{exercicebox}[Exercice 18 : Somme de Lois de Poisson]
Un magasin reçoit des clients au comptoir A selon $X \sim \text{Poisson}(\lambda_1=5 \text{ clients/heure})$ et au comptoir B selon $Y \sim \text{Poisson}(\lambda_2=3 \text{ clients/heure})$. On suppose que $X$ et $Y$ sont indépendantes.
Soit $S = X+Y$ le nombre total de clients arrivant au magasin en une heure.
\begin{enumerate}
    \item Quelle est la loi de $S$ ? Donnez son nom et son paramètre.
    \item Quelle est la probabilité qu'exactement 6 clients au total arrivent en une heure, $P(S=6)$ ?
\end{enumerate}
\end{exercicebox}

\begin{exercicebox}[Exercice 19 : Corrélation Nulle mais Dépendance]
Soit $X$ une variable aléatoire $X \in \{-1, 0, 1\}$, avec $P(X=-1)=1/3$, $P(X=0)=1/3$, $P(X=1)=1/3$.
Soit $Y = X^2$.
\begin{enumerate}
    \item Calculez $E[X]$.
    \item Calculez $E[XY]$. (Indice : $E[XY] = E[X^3]$).
    \item Calculez $\text{Cov}(X,Y)$.
    \item Les variables $X$ et $Y$ sont-elles indépendantes ?
\end{enumerate}
\end{exercicebox}

\begin{exercicebox}[Exercice 20 : Bornes de la Corrélation]
Soit $X$ une variable aléatoire et $Y = -3X + 5$.
Sans faire de calcul, que vaut $\text{Corr}(X,Y)$ ? Justifiez.
\end{exercicebox}

\subsection{Corrections des Exercices}

% --- Corrections : Lois Jointes et Marginales (Discret) ---

\begin{correctionbox}[Correction Exercice 1 : Loi Jointe et Marginales]
1.  On somme toutes les probabilités du tableau :
    $0.1 + 0.2 + 0.1 + 0.3 + 0.1 + 0.2 = 1.0$.
    Puisque la somme est 1 et toutes les probabilités sont non négatives, c'est une loi valide.

2.  Loi marginale de $X$ (somme des lignes) :
    $P(X=0) = P(X=0, Y=0) + P(X=0, Y=1) + P(X=0, Y=2) = 0.1 + 0.2 + 0.1 = 0.4$.
    $P(X=1) = P(X=1, Y=0) + P(X=1, Y=1) + P(X=1, Y=2) = 0.3 + 0.1 + 0.2 = 0.6$.

3.  Loi marginale de $Y$ (somme des colonnes) :
    $P(Y=0) = P(X=0, Y=0) + P(X=1, Y=0) = 0.1 + 0.3 = 0.4$.
    $P(Y=1) = P(X=0, Y=1) + P(X=1, Y=1) = 0.2 + 0.1 = 0.3$.
    $P(Y=2) = P(X=0, Y=2) + P(X=1, Y=2) = 0.1 + 0.2 = 0.3$.
\end{correctionbox}

\begin{correctionbox}[Correction Exercice 2 : Calcul de Probabilité Jointe]
1.  $P(X=0, Y \le 1) = P(X=0, Y=0) + P(X=0, Y=1) = 0.1 + 0.2 = 0.3$.
2.  $P(X=Y) = P(X=0, Y=0) + P(X=1, Y=1) = 0.1 + 0.1 = 0.2$.
3.  $P(X > Y) = P(X=1, Y=0) = 0.3$. (C'est la seule case où $x > y$).
\end{correctionbox}

\begin{correctionbox}[Correction Exercice 3 : Indépendance (Loi Jointe)]
On utilise les lois marginales de l'exercice 1 : $P(X=0)=0.4$ et $P(Y=0)=0.4$.
1.  $P(X=0) \times P(Y=0) = 0.4 \times 0.4 = 0.16$.
2.  Dans le tableau joint, $P(X=0, Y=0) = 0.1$.
3.  Puisque $P(X=0, Y=0) \neq P(X=0) \times P(Y=0)$ (car $0.1 \neq 0.16$), les variables $X$ et $Y$ \textbf{ne sont pas indépendantes}. (Un seul contre-exemple suffit).
\end{correctionbox}

% --- Corrections : Espérance, Covariance et Corrélation ---

\begin{correctionbox}[Correction Exercice 4 : Espérances Marginales]
1.  $E[X] = \sum_x x P(X=x) = (0)(P(X=0)) + (1)(P(X=1))$
    $E[X] = (0)(0.4) + (1)(0.6) = 0.6$.
2.  $E[Y] = \sum_y y P(Y=y) = (0)(P(Y=0)) + (1)(P(Y=1)) + (2)(P(Y=2))$
    $E[Y] = (0)(0.4) + (1)(0.3) + (2)(0.3) = 0 + 0.3 + 0.6 = 0.9$.
\end{correctionbox}

\begin{correctionbox}[Correction Exercice 5 : Espérance d'une Fonction (LOTUS)]
On somme $(xy)P(X=x, Y=y)$ sur les 6 cases. Les termes où $x=0$ ou $y=0$ sont nuls.
$E[XY] = (0 \cdot 0)(0.1) + (0 \cdot 1)(0.2) + (0 \cdot 2)(0.1) + (1 \cdot 0)(0.3) + (1 \cdot 1)(0.1) + (1 \cdot 2)(0.2)$
$E[XY] = 0 + 0 + 0 + 0 + (1)(0.1) + (2)(0.2) = 0.1 + 0.4 = 0.5$.
\end{correctionbox}

\begin{correctionbox}[Correction Exercice 6 : Covariance (Calcul)]
On utilise la formule $\text{Cov}(X,Y) = E[XY] - E[X]E[Y]$.
$$ \text{Cov}(X,Y) = 0.5 - (0.6)(0.9) = 0.5 - 0.54 = -0.04 $$
\end{correctionbox}

\begin{correctionbox}[Correction Exercice 7 : Variances Marginales]
1.  Pour $X$:
    $E[X^2] = (0^2)(0.4) + (1^2)(0.6) = 0.6$.
    $\text{Var}(X) = E[X^2] - (E[X])^2 = 0.6 - (0.6)^2 = 0.6 - 0.36 = 0.24$.
2.  Pour $Y$:
    $E[Y^2] = (0^2)(0.4) + (1^2)(0.3) + (2^2)(0.3) = 0 + 0.3 + (4)(0.3) = 0.3 + 1.2 = 1.5$.
    $\text{Var}(Y) = E[Y^2] - (E[Y])^2 = 1.5 - (0.9)^2 = 1.5 - 0.81 = 0.69$.
\end{correctionbox}

\begin{correctionbox}[Correction Exercice 8 : Corrélation (Calcul)]
On utilise la formule $\text{Corr}(X,Y) = \frac{\text{Cov}(X,Y)}{\sqrt{\text{Var}(X)\text{Var}(Y)}}$.
$$ \text{Corr}(X,Y) = \frac{-0.04}{\sqrt{0.24 \times 0.69}} = \frac{-0.04}{\sqrt{0.1656}} \approx \frac{-0.04}{0.4069} \approx -0.098 $$
La corrélation est très faible et négative.
\end{correctionbox}

% --- Corrections : Propriétés de la Variance et de la Covariance ---

\begin{correctionbox}[Correction Exercice 9 : Variance d'une Somme (Non Indépendant)]
On utilise la formule générale :
$$ \text{Var}(X+Y) = \text{Var}(X) + \text{Var}(Y) + 2\text{Cov}(X,Y) $$
$$ \text{Var}(X+Y) = 10 + 5 + 2(2) = 15 + 4 = 19 $$
\end{correctionbox}

\begin{correctionbox}[Correction Exercice 10 : Variance d'une Différence (Indépendant)]
1.  Puisque $X$ et $Y$ sont indépendantes, leur covariance est nulle : $\text{Cov}(X,Y) = 0$.
2.  On utilise la formule générale :
    $$ \text{Var}(X-Y) = \text{Var}(X + (-1)Y) = \text{Var}(X) + \text{Var}(-1 \cdot Y) + 2\text{Cov}(X, -Y) $$
    $$ = \text{Var}(X) + (-1)^2 \text{Var}(Y) - 2\text{Cov}(X, Y) $$
    $$ \text{Var}(X-Y) = \text{Var}(X) + \text{Var}(Y) - 2(0) $$
    $$ \text{Var}(X-Y) = 16 + 9 = 25 $$
\end{correctionbox}

\begin{correctionbox}[Correction Exercice 11 : Bilinéarité de la Covariance]
La covariance est linéaire sur son premier argument :
$$ \text{Cov}(X+Y, Z) = \text{Cov}(X, Z) + \text{Cov}(Y, Z) $$
\end{correctionbox}

\begin{correctionbox}[Correction Exercice 12 : Variance d'une Combinaison Linéaire]
On utilise $\text{Var}(aX + bY + c) = a^2 \text{Var}(X) + b^2 \text{Var}(Y) + 2ab\text{Cov}(X,Y)$.
Ici $a=3$, $b=-5$, $c=1$. $X$ et $Y$ sont indépendantes, donc $\text{Cov}(X,Y)=0$.
$$ \text{Var}(3X - 5Y + 1) = (3)^2 \text{Var}(X) + (-5)^2 \text{Var}(Y) + 0 $$
$$ = 9 \times (4) + 25 \times (2) = 36 + 50 = 86 $$
(Note : la constante additive $c=1$ ne change pas la variance).
\end{correctionbox}

\begin{correctionbox}[Correction Exercice 13 : Variance d'une Somme (Dés)]
1.  Oui, les lancers de deux dés standards sont des événements physiquement indépendants.
2.  Puisqu'ils sont indépendants, $\text{Cov}(D_1, D_2) = 0$.
    $$ \text{Var}(S) = \text{Var}(D_1 + D_2) = \text{Var}(D_1) + \text{Var}(D_2) $$
    $$ \text{Var}(S) = \frac{35}{12} + \frac{35}{12} = \frac{70}{12} = \frac{35}{6} $$
\end{correctionbox}

\begin{correctionbox}[Correction Exercice 14 : Covariance et Variance]
Par définition, $\text{Cov}(A, B) = E[(A-\mu_A)(B-\mu_B)]$.
Posons $A=X$ et $B=X$. Alors $\mu_A = \mu_X$ et $\mu_B = \mu_X$.
$$ \text{Cov}(X, X) = E[(X-\mu_X)(X-\mu_X)] = E[(X-\mu_X)^2] $$
C'est la définition de $\text{Var}(X)$.
\end{correctionbox}

\begin{correctionbox}[Correction Exercice 15 : Covariance avec une Constante]
On utilise la formule de calcul $\text{Cov}(X,c) = E[Xc] - E[X]E[c]$.
Par linéarité, $E[Xc] = cE[X]$.
L'espérance d'une constante est la constante elle-même : $E[c] = c$.
$$ \text{Cov}(X, c) = cE[X] - E[X]c = 0 $$
\end{correctionbox}

% --- Corrections : Standardisation et Somme de Poissons ---

\begin{correctionbox}[Correction Exercice 16 : Standardisation (Centrer-Réduire)]
$Z = \frac{X - 10}{2} = \frac{1}{2}X - 5$.
1.  Calcul de $E[Z]$ par linéarité :
    $$ E[Z] = E\left[ \frac{1}{2}X - 5 \right] = \frac{1}{2}E[X] - 5 = \frac{1}{2}(10) - 5 = 5 - 5 = 0 $$
2.  Calcul de $\text{Var}(Z)$ par les propriétés de la variance :
    $$ \text{Var}(Z) = \text{Var}\left( \frac{1}{2}X - 5 \right) = \left(\frac{1}{2}\right)^2 \text{Var}(X) = \frac{1}{4} \text{Var}(X) $$
    $$ \text{Var}(Z) = \frac{1}{4}(4) = 1 $$
    Par définition, une variable standardisée a une moyenne de 0 et une variance de 1.
\end{correctionbox}

\begin{correctionbox}[Correction Exercice 17 : Corrélation et Standardisation]
La corrélation $\text{Corr}(X,Y)$ EST, par définition, la covariance des versions standardisées $Z_X$ et $Z_Y$.
$$ \text{Cov}(Z_X, Z_Y) = \text{Corr}(X,Y) = 0.5 $$
\end{correctionbox}

\begin{correctionbox}[Correction Exercice 18 : Somme de Lois de Poisson]
1.  Puisque $X$ et $Y$ sont des v.a. de Poisson \textbf{indépendantes}, leur somme $S=X+Y$ suit aussi une \textbf{loi de Poisson}.
    Le nouveau paramètre est la somme des paramètres : $\lambda_S = \lambda_1 + \lambda_2 = 5 + 3 = 8$.
    Donc, $S \sim \text{Poisson}(\lambda=8)$.
2.  On cherche $P(S=6)$ pour $S \sim \text{Poisson}(8)$.
    $$ P(S=6) = \frac{e^{-8} 8^6}{6!} = \frac{e^{-8} \times 262144}{720} = 364.08 \times e^{-8} \approx 0.122 $$
\end{correctionbox}

\begin{correctionbox}[Correction Exercice 19 : Corrélation Nulle mais Dépendance]
1.  $E[X] = (-1)(1/3) + (0)(1/3) + (1)(1/3) = -1/3 + 0 + 1/3 = 0$.
2.  $E[XY] = E[X(X^2)] = E[X^3]$.
    $E[X^3] = (-1)^3(1/3) + (0)^3(1/3) + (1)^3(1/3) = (-1)(1/3) + 0 + (1)(1/3) = 0$.
3.  $\text{Cov}(X,Y) = E[XY] - E[X]E[Y] = 0 - (0)E[Y] = 0$.
    Les variables sont \textbf{non corrélées}.
4.  Les variables $X$ et $Y$ sont-elles indépendantes ? Non.
    Test : $P(X=1, Y=0) \stackrel{?}{=} P(X=1)P(Y=0)$.
    - $P(X=1, Y=0) = P(X=1, X^2=0) = 0$.
    - $P(X=1) = 1/3$.
    - $P(Y=0) = P(X^2=0) = P(X=0) = 1/3$.
    - $P(X=1)P(Y=0) = (1/3)(1/3) = 1/9$.
    Puisque $0 \neq 1/9$, elles \textbf{ne sont pas indépendantes}.
    C'est un exemple classique de dépendance non linéaire avec covariance nulle.
\end{correctionbox}

\begin{correctionbox}[Correction Exercice 20 : Bornes de la Corrélation]
$Y$ est une fonction linéaire parfaite de $X$ : $Y = aX + b$ avec $a=-3$ et $b=5$.
La corrélation $\text{Corr}(X,Y)$ mesure la force de la relation \textit{linéaire}. Puisqu'elle est parfaite, la corrélation doit être $\pm 1$.
Le coefficient $a = -3$ est négatif, donc la relation est décroissante.
Par conséquent, $\text{Corr}(X,Y) = -1$.
\end{correctionbox}

\subsection{Exercices Python}

Ces exercices appliquent les concepts de distributions multivariées (covariance, corrélation, variance d'une somme) à des données financières réelles. Nous allons analyser la relation entre les rendements boursiers de deux entreprises technologiques : Google (GOOG) et Microsoft (MSFT).

Nous travaillerons avec les \textbf{rendements journaliers} (variation en pourcentage), qui sont des variables aléatoires continues. L'espérance $E[X]$ sera estimée par la moyenne empirique (\texttt{.mean()}) et la variance $\text{Var}(X)$ par la variance empirique (\texttt{.var()}).

\begin{codecell}
!pip install yfinance
import yfinance as yf
import pandas as pd
import numpy as np

# Definir les tickers et la periode
tickers = ["GOOG", "MSFT"]
start_date = "2020-01-01"
end_date = "2024-12-31"

# Telecharger les prix de cloture ajustes
data = yf.download(tickers, start=start_date, end=end_date)["Adj Close"]

# Calculer les rendements journaliers en pourcentage
returns = data.pct_change().dropna()

# Renommer les colonnes pour plus de clarte
returns.columns = ["GOOG_Return", "MSFT_Return"]

# "returns" est notre DataFrame principal.
# X = returns["GOOG_Return"]
# Y = returns["MSFT_Return"]
\end{codecell}

\begin{exercicebox}[Exercice 1 : Espérances et Variances Marginales]
Soit $X$ la v.a. "Rendement journalier de GOOG" et $Y$ la v.a. "Rendement journalier de MSFT".

\textbf{Votre tâche :}
\begin{enumerate}
    \item Calculer l'espérance empirique $E[X]$ et $E[Y]$. (Que remarquez-vous sur leur ordre de grandeur ?)
    \item Calculer la variance empirique $\text{Var}(X)$ et $\text{Var}(Y)$.
    \item Calculer l'écart-type empirique $\sigma_X$ et $\sigma_Y$. Laquelle des deux actions est la plus "volatile" ?
\end{enumerate}
\end{exercicebox}

\begin{exercicebox}[Exercice 2 : Standardisation (Centrer-Réduire)]
Le concept de variable centrée réduite $Z = \frac{X - \mu_X}{\sigma_X}$ est très utilisé en finance (par ex: "Z-score").

\textbf{Votre tâche :}
\begin{enumerate}
    \item Récupérer $E[X]$ (la moyenne) et $\sigma_X$ (l'écart-type) des rendements de GOOG de l'exercice 1.
    \item Prendre le \textbf{dernier} rendement journalier de GOOG dans le jeu de données.
    \item Calculer le "Z-score" de ce dernier rendement.
    \item Interpréter ce score (par ex: "Le dernier jour, GOOG a performé à X écarts-types de sa moyenne...").
\end{enumerate}
\end{exercicebox}

\begin{exercicebox}[Exercice 3 : Covariance (Calcul via LOTUS)]
Calculez la covariance entre les rendements de GOOG ($X$) et de MSFT ($Y$) en utilisant la formule $\text{Cov}(X,Y) = E[XY] - E[X]E[Y]$.

\textbf{Votre tâche :}
\begin{enumerate}
    \item Récupérer $E[X]$ et $E[Y]$ de l'exercice 1.
    \item Calculer $E[XY]$ en utilisant le "Théorème de Transfert" (LOTUS) sur les données empiriques (Indice : calculez la moyenne de la série $X \times Y$).
    \item Appliquer la formule pour trouver $\text{Cov}(X,Y)$.
    \item Le signe est-il positif ou négatif ? Qu'est-ce que cela implique intuitivement ?
\end{enumerate}
\end{exercicebox}

\begin{exercicebox}[Exercice 4 : Corrélation (Calcul)]
La covariance de l'exercice 3 dépend des unités (rendement au carré). Nous allons la normaliser pour obtenir la corrélation $r \in [-1, 1]$.

\textbf{Votre tâche :}
\begin{enumerate}
    \item Récupérer $\text{Cov}(X,Y)$ (Exercice 3) et $\sigma_X, \sigma_Y$ (Exercice 1).
    \item Appliquer la formule : $\text{Corr}(X,Y) = \frac{\text{Cov}(X,Y)}{\sigma_X \sigma_Y}$.
    \item Interpréter ce coefficient. La relation linéaire entre GOOG et MSFT est-elle forte ou faible ?
\end{enumerate}
\end{exercicebox}

\begin{exercicebox}[Exercice 5 : Linéarité de l'Espérance (Portefeuille)]
Soit un portefeuille $P$ composé à 60\% de GOOG ($X$) et 40\% de MSFT ($Y$).
Le rendement du portefeuille est $P = 0.6X + 0.4Y$.
La théorie dit : $E[P] = E[0.6X + 0.4Y] = 0.6E[X] + 0.4E[Y]$.

\textbf{Votre tâche :}
\begin{enumerate}
    \item En utilisant $E[X]$ et $E[Y]$ de l'exercice 1, calculer l'espérance \textbf{théorique} $E[P]$.
    \item \textbf{Vérification empirique} : 
        \begin{itemize}
            \item Créer la série de données $P_{series} = 0.6 \times X + 0.4 \times Y$.
            \item Calculer l'espérance empirique de $P_{series}$ (\texttt{.mean()}).
        \end{itemize}
    \item Comparer votre résultat théorique (1) et empirique (2).
\end{enumerate}
\end{exercicebox}

\begin{exercicebox}[Exercice 6 : Variance d'un Portefeuille (Variance d'une Somme)]
Continuons avec le portefeuille $P = 0.6X + 0.4Y$.
La variance \textbf{théorique} est : $\text{Var}(P) = a^2\text{Var}(X) + b^2\text{Var}(Y) + 2ab\text{Cov}(X,Y)$.

\textbf{Votre tâche :}
\begin{enumerate}
    \item En utilisant $\text{Var}(X)$, $\text{Var}(Y)$ (Ex 1) et $\text{Cov}(X,Y)$ (Ex 3), calculer $\text{Var}(P)$ en appliquant la formule ci-dessus.
    \item \textbf{Vérification empirique} : 
        \begin{itemize}
            \item Utiliser la série $P_{series}$ de l'exercice 5.
            \item Calculer la variance empirique de $P_{series}$ (\texttt{.var()}).
        \end{itemize}
    \item Comparer votre résultat théorique (1) et empirique (2).
\end{enumerate}
\end{exercicebox}

\begin{exercicebox}[Exercice 7 : Le Bénéfice de la Diversification]
La diversification (le fait que $\text{Corr}(X,Y) \ne 1$) réduit le risque. Nous allons le prouver.
Le risque (l'écart-type) d'un portefeuille \textit{n'est pas} la moyenne pondérée des risques.

\textbf{Votre tâche :}
\begin{enumerate}
    \item Calculer le risque du portefeuille $\sigma_P$ (l'écart-type) en prenant la racine carrée de $\text{Var}(P)$ (calculée à l'exercice 6).
    \item Calculer la "moyenne pondérée des risques" : $\sigma_{moy} = 0.6 \sigma_X + 0.4 \sigma_Y$ (en utilisant $\sigma_X, \sigma_Y$ de l'Ex 1).
    \item Comparer $\sigma_P$ et $\sigma_{moy}$. Lequel est le plus petit ?
    \item (Conclusion) Pourquoi $\sigma_P < \sigma_{moy}$ ? (Indice : $\text{Corr}(X,Y)$).
\end{enumerate}
\end{exercicebox}

\begin{exercicebox}[Exercice 8 : Vérification des Bornes de Corrélation]
Le théorème stipule que si $Y = aX + b$, alors $\text{Corr}(X,Y) = \pm 1$. Vérifions cela.

\textbf{Votre tâche :}
\begin{enumerate}
    \item Soit $X$ la série des rendements de GOOG.
    \item Créer une nouvelle variable $Z = -3X + 0.005$ (une relation linéaire négative parfaite).
    \item Calculer la corrélation empirique entre $X$ et $Z$. (Vous pouvez utiliser \texttt{X.corr(Z)}).
    \item Le résultat est-il conforme au théorème ?
\end{enumerate}
\end{exercicebox}

\begin{exercicebox}[Exercice 9 : Loi Jointe (Discrétisation)]
Transformons nos variables continues $X$ et $Y$ en variables de Bernoulli discrètes.
Soit $X_{bern} = 1$ si le rendement de GOOG est positif ($> 0$), et $0$ sinon.
Soit $Y_{bern} = 1$ si le rendement de MSFT est positif ($> 0$), et $0$ sinon.
Nous voulons trouver la PMF jointe $P(X_{bern}=x, Y_{bern}=y)$.

\textbf{Votre tâche :}
\begin{enumerate}
    \item Créer les deux séries discrètes $X_{bern}$ et $Y_{bern}$.
    \item Utiliser \texttt{pandas.crosstab} pour créer un tableau de contingence (les effectifs).
    \item Normaliser ce tableau par l'effectif total pour obtenir la \textbf{loi jointe} (PMF jointe).
    \item Quelle est la probabilité que les deux actions aient un rendement positif le même jour, $P(X_{bern}=1, Y_{bern}=1)$ ?
\end{enumerate}
\end{exercicebox}

\begin{exercicebox}[Exercice 10 : Lois Marginales et Indépendance (Discret)]
En utilisant la loi jointe $P(X_{bern}, Y_{bern})$ de l'exercice 9.

\textbf{Votre tâche :}
\begin{enumerate}
    \item Calculer la loi marginale $P(X_{bern}=x)$ (somme des lignes).
    \item Calculer la loi marginale $P(Y_{bern}=y)$ (somme des colonnes).
    \item Les variables $X_{bern}$ et $Y_{bern}$ sont-elles indépendantes ? 
    \item Justifiez en comparant $P(X_{bern}=1, Y_{bern}=1)$ au produit $P(X_{bern}=1) \times P(Y_{bern}=1)$.
\end{enumerate}
\end{exercicebox}
\newpage
\section{La Loi Normale (ou Gaussienne)}

\subsection{Introduction et Fonction de Densité (PDF)}

Après les lois discrètes et les lois continues de base (Uniforme, Exponentielle), nous abordons la distribution la plus célèbre et la plus utilisée en probabilités et statistiques.

\begin{definitionbox}[Loi Normale]
Une variable aléatoire continue $X$ suit une \textbf{loi normale} (ou loi de Gauss) de paramètres $\mu$ (l'espérance) et $\sigma^2$ (la variance), notée $X \sim \mathcal{N}(\mu, \sigma^2)$, si sa fonction de densité de probabilité (PDF) est donnée par :
$$ f(x; \mu, \sigma) = \frac{1}{\sigma \sqrt{2\pi}} e^{ -\frac{1}{2} \left( \frac{x-\mu}{\sigma} \right)^2 } $$
pour tout $x \in (-\infty, \infty)$, où $\sigma > 0$.
\end{definitionbox}

Cette formule, bien qu'imposante, décrit une forme très familière : la courbe en cloche.

\begin{intuitionbox}[La Courbe en Cloche]
La loi normale est sans doute la distribution la plus importante en probabilités et statistiques. Pourquoi ? Parce qu'elle modélise remarquablement bien de nombreux phénomènes naturels et processus aléatoires où les valeurs tendent à se regrouper autour d'une moyenne, avec des écarts symétriques devenant de plus en plus rares à mesure qu'on s'éloigne de cette moyenne. Pensez à la taille des individus dans une population, aux erreurs de mesure répétées, ou même aux notes d'un grand groupe d'étudiants à un examen bien conçu. 

Sa densité a une forme caractéristique de \textbf{cloche symétrique} :
\begin{itemize}
    \item \textbf{Le Centre ($\mu$)} : Le paramètre $\mu$ représente l'\textbf{espérance} (la moyenne) de la distribution. C'est le centre de symétrie de la courbe, là où la cloche atteint son \textbf{sommet}. C'est la valeur la plus probable (le mode) et aussi la valeur qui coupe la distribution en deux moitiés égales (la médiane). Changer $\mu$ \textit{translate} la cloche horizontalement sans changer sa forme.
    \item \textbf{La Dispersion ($\sigma$)} : Le paramètre $\sigma$ est l'\textbf{écart-type} ($\sigma^2$ est la variance). Il mesure la \textbf{dispersion} des valeurs autour de la moyenne $\mu$. Géométriquement, $\sigma$ contrôle la \textbf{largeur} de la cloche.
        \begin{itemize}
            \item Un \textit{petit} $\sigma$ signifie que les données sont très concentrées autour de la moyenne, donnant une cloche \textbf{étroite et pointue}.
            \item Un \textit{grand} $\sigma$ signifie que les données sont plus étalées, donnant une cloche \textbf{large et aplatie}.
        \end{itemize}
    Les points d'inflexion de la courbe (là où la courbure change de sens) se situent exactement à $\mu \pm \sigma$.
\end{itemize}

\tcblower
\centering
\begin{tikzpicture}
    \begin{axis}[
        title={La Courbe en Cloche (PDF de la Loi Normale)},
        xlabel={$x$},
        ylabel={$f(x)$},
        axis lines=middle,
        no markers,
        samples=100,
        domain=-4:4,
        height=8cm,
        width=\linewidth-1cm,
        tick label style={font=\tiny},
        legend style={at={(0.5,-0.15)}, anchor=north, font=\small},
        legend columns=2
    ]
    % N(0, 1)
    \addplot [blue, ultra thick] {1/(sqrt(2*pi))*exp(-x^2/2)};
    \addlegendentry{$\mu=0, \sigma=1$};
    % N(0, 0.25) => sigma=0.5
    \addplot [red, ultra thick] {1/(0.5*sqrt(2*pi))*exp(-x^2/(2*0.5^2))};
    \addlegendentry{$\mu=0, \sigma=0.5$ (étroite)};
    % N(1, 2.25) => sigma=1.5
    \addplot [green!70!black, ultra thick] {1/(1.5*sqrt(2*pi))*exp(-(x-1)^2/(2*1.5^2))};
    \addlegendentry{$\mu=1, \sigma=1.5$ (large, décalée)};

    \draw [dashed] (axis cs:0,0) -- (axis cs:0, {1/(sqrt(2*pi))}) node[above, font=\tiny] {pic à $\mu=0$}; % Ligne pour mu=0
    \draw [dashed] (axis cs:1,0) -- (axis cs:1, {1/(1.5*sqrt(2*pi))}) node[above right, font=\tiny] {pic à $\mu=1$}; % Ligne pour mu=1
    \end{axis}
\end{tikzpicture}
\par\small\textit{Influence de $\mu$ (position) et $\sigma$ (largeur) sur la forme de la cloche.}
\end{intuitionbox}

Mais d'où vient cette formule spécifique ? Il existe une dérivation fascinante à partir d'hypothèses fondamentales sur les erreurs aléatoires (argument d'Herschel-Maxwell).

\begin{proofbox}[Dérivation de la Densité Normale à partir des Principes Fondamentaux]

\textbf{Contexte Visuel :} Imaginons un nuage de points dispersés autour d'une cible à l'origine $(0,0)$, comme des impacts de fléchettes. Le graphique ci-dessous illustre cette dispersion. On s'intéresse à la probabilité de tomber dans une petite zone, comme $dA$, autour d'un point $(x, y)$.

\begin{center}
\begin{tikzpicture}
\begin{axis}[
    axis lines=middle, % Axes qui passent par l'origine (0,0)
    xlabel=$x$,       % Étiquette axe X
    ylabel=$y$,       % Étiquette axe Y
    axis line style={magenta}, % Couleur des axes
    xlabel style={anchor=west, magenta}, % Style de l'étiquette X
    ylabel style={anchor=south, magenta}, % Style de l'étiquette Y
    xmin=-3.5, xmax=3.5, % Limites du graphique
    ymin=-3.5, ymax=3.5,
    tick label style={font=\tiny} % Police plus petite pour les graduations
]

% Ajout des points du nuage. Ce sont des coordonnées approximatives.
\addplot [only marks, mark=*, cyan, mark size=1.5pt]
coordinates {
    (0.2, 0.1) (-0.5, 0.2) (0.1, -0.3) (0.5, -0.8) (-0.3, -1.2) (0,0.1)
    (1.5, 1.5) (0.8, 0.8) (2.0, -0.5) (2.8, -1.4) (2.5, -0.2)
    (-1.8, -1.3) (-2.5, 0.5) (-1.5, 0.3) (-2.2, -0.8) (-0.8, 0.5)
    (0.5, 1.2) (0.7, 2.8) (0.2, 3.2) (-0.5, 1.5) (-1.0, -2.0)
    (1.8, -1.0) (1.0, -1.5) (0.3, -2.5) (-1.8, -2.8) (1.2, 0.5)
    (-1.2, -1.5) (-0.8, 1.1)
};

% --- Annotations ---

% Points et boîte pour 'dA'
\addplot [only marks, mark=*, red, mark size=1.5pt] coordinates {(1.3, 2.0)};
\draw [red, thick] (axis cs:1.1, 1.8) rectangle (axis cs:1.5, 2.2);
\node [red, above] at (axis cs:1.3, 2.2) {$dA$};

% Points et boîte pour 'dB'
\addplot [only marks, mark=*, cyan, mark size=1.5pt] 
coordinates {(-1.2, 2.0) (-1.3, 2.3) (-1.0, 2.2) (-1.1, 1.9)};
\draw [blue, thick] (axis cs:-1.8, 1.2) rectangle (axis cs:-0.8, 2.8);
\node [blue, above] at (axis cs:-1.3, 2.8) {$dB$};

\end{axis}
\end{tikzpicture}
\end{center}

\textbf{Objectif :} Expliquer comment arriver à la formule mathématique de la courbe en cloche (densité de probabilité normale) en partant de principes fondamentaux sur les erreurs aléatoires.

\textbf{1. Le Point de Départ : Densité et Aire $dA$}
Dans une distribution continue, la probabilité de tomber \textit{exactement} sur un point $(x, y)$ est nulle. On ne peut donc pas parler de "probabilité d'un point". On parle de la probabilité de tomber \textit{dans une petite zone}, comme un rectangle $dA = dx \cdot dy$ autour du point $(x, y)$.
Cette probabilité, notée $P(\text{dans } dA)$, est \textit{proportionnelle} à l'aire de la zone $dA$. La \textit{constante de proportionnalité} est la \textbf{fonction de densité de probabilité} $p(x, y)$ évaluée en ce point. En d'autres termes, la densité $p(x, y)$ \textit{représente} localement la concentration de probabilité. Ainsi, la probabilité de tomber dans la zone $dA$ est approximativement :
$$ P(\text{dans } dA) \approx p(x, y) \cdot dA $$

\textbf{2. Les Hypothèses Fondamentales}
On pose deux hypothèses sur la nature de ces erreurs (représentées par la densité $p(x, y)$) :
\begin{enumerate}
    \item \textbf{Indépendance des axes :} L'erreur horizontale ($x$) est indépendante de l'erreur verticale ($y$). Cela implique que la densité jointe $p(x, y)$ peut s'écrire comme le produit de la densité marginale sur $x$, notée $f(x)$, et de la densité marginale sur $y$, notée $f(y)$. Donc, $p(x, y) = f(x) \cdot f(y)$.
    \item \textbf{Symétrie de rotation (Isotropie) :} La densité ne dépend que de la distance $r = \sqrt{x^2 + y^2}$ au centre, pas de l'angle. Il existe donc une fonction $\phi(r)$ telle que la densité en $(x,y)$ est $p(x, y) = \phi(\sqrt{x^2 + y^2})$.
\end{enumerate}

\textbf{3. L'Équation Fonctionnelle}
En égalant les deux expressions pour la même densité $p(x, y)$ (à une constante près), on obtient :
$$ f(x) \cdot f(y) = \phi(\sqrt{x^2 + y^2}) $$
Pour $y=0$, on a $f(x) \cdot f(0) = \phi(x)$. Posons $f(0) = \lambda$. Alors $\phi(x) = \lambda f(x)$.
L'équation devient :
$$ f(x) \cdot f(y) = \lambda f(\sqrt{x^2 + y^2}) $$

\textbf{4. Résolution de l'Équation Fonctionnelle}
Posons $g(x) = f(x)/\lambda$, avec $g(0)=1$. L'équation se simplifie en :
$$ g(x) g(y) = g(\sqrt{x^2 + y^2}) $$
Posons $g(x) = h(x^2)$. L'équation devient $h(x^2)h(y^2) = h(x^2+y^2)$. Avec $a=x^2$ et $b=y^2$, on a :
$$ h(a) h(b) = h(a+b) $$
La solution continue de cette équation de Cauchy est $h(a) = e^{Aa}$ pour une constante $A$.
Retour aux fonctions : $g(x) = h(x^2) = e^{Ax^2}$. $f(x) = \lambda g(x) = \lambda e^{Ax^2}$.
Comme la densité doit diminuer loin du centre, $A$ doit être négative. Posons $A = -k$ avec $k>0$.
$$ f(x) = \lambda e^{-k x^2} $$

\textbf{5. Normalisation et Identification des Paramètres}
\begin{enumerate}
    \item \textbf{Condition $\int f(x) dx = 1$} : L'intégrale Gaussienne $\int_{-\infty}^{\infty} e^{-k x^2} \, \mathrm{d}x = \sqrt{\frac{\pi}{k}}$.
    Donc, $\int_{-\infty}^{\infty} f(x) dx = \lambda \sqrt{\frac{\pi}{k}} = 1 \implies \lambda = \sqrt{\frac{k}{\pi}}$.
    \item \textbf{Lien avec la Variance ($\sigma^2$)} : Pour une distribution centrée, $\sigma^2 = E[X^2] = \int x^2 f(x) dx$.
    $$ \sigma^2 = \int_{-\infty}^{\infty} x^2 \left( \sqrt{\frac{k}{\pi}} e^{-k x^2} \right) \, \mathrm{d}x = \sqrt{\frac{k}{\pi}} \left( \frac{1}{2k} \sqrt{\frac{\pi}{k}} \right) = \frac{1}{2k} $$
    Donc, $k = \frac{1}{2\sigma^2}$.
    \item \textbf{Substitution Finale :} Remplaçons $k$ dans $\lambda$ et $f(x)$.
    $$ \lambda = \sqrt{\frac{1/(2\sigma^2)}{\pi}} = \frac{1}{\sigma\sqrt{2\pi}} $$
    $$ f(x) = \frac{1}{\sigma\sqrt{2\pi}} e^{-\frac{1}{2\sigma^2} x^2} = \frac{1}{\sigma\sqrt{2\pi}} e^{ -\frac{x^2}{2\sigma^2} } $$
    \item \textbf{Généralisation (Moyenne $\mu$)} : Pour centrer la distribution sur $\mu$, on remplace $x$ par $(x-\mu)$ dans l'exposant :
    $$ f(x; \mu, \sigma) = \frac{1}{\sigma \sqrt{2\pi}} e^{ -\frac{(x-\mu)^2}{2\sigma^2} } $$
\end{enumerate}
C'est la fonction de densité de la loi normale $\mathcal{N}(\mu, \sigma^2)$.
\end{proofbox}

\subsection{La Loi Normale Centrée Réduite $\mathcal{N}(0, 1)$}

Avant d'explorer les propriétés de la loi normale générale, concentrons-nous sur son cas le plus simple et le plus fondamental.

\begin{definitionbox}[Loi Normale Standard (ou Centrée Réduite)]
Un cas particulier extraordinairement utile est la loi normale avec une moyenne $\mu=0$ et une variance $\sigma^2=1$ (donc $\sigma=1$). On l'appelle la \textbf{loi normale standard} ou \textbf{centrée réduite}, et on la note souvent $Z$. Sa PDF est traditionnellement notée $\phi(z)$ :
$$ \phi(z) = \frac{1}{\sqrt{2\pi}} e^{-z^2/2} $$
Sa fonction de répartition (CDF), qui donne $P(Z \le z)$, est notée $\Phi(z)$ :
$$ \Phi(z) = P(Z \le z) = \int_{-\infty}^z \frac{1}{\sqrt{2\pi}} e^{-t^2/2} \, \mathrm{d}t $$
\end{definitionbox}

Pourquoi cette version standard est-elle si importante ? Elle sert de référence universelle.

\begin{intuitionbox}[La Référence Universelle et le Changement d'Unités]
Pourquoi cette loi $\mathcal{N}(0, 1)$ est-elle si centrale ? Imaginez que vous ayez des mesures en degrés Celsius ($\mathcal{N}(\mu_C, \sigma_C^2)$) et d'autres en degrés Fahrenheit ($\mathcal{N}(\mu_F, \sigma_F^2)$). Comment les comparer ? La loi normale standard fournit un \textbf{système d'unités universel}.

Toute variable normale $X \sim \mathcal{N}(\mu, \sigma^2)$ peut être transformée ("standardisée") en une variable $Z \sim \mathcal{N}(0, 1)$ par un simple changement d'échelle et de position : $Z = (X-\mu)/\sigma$. 

Cela signifie qu'au lieu de devoir calculer des aires (probabilités) pour une infinité de courbes en cloche différentes (une pour chaque paire $\mu, \sigma$), on peut tout ramener à \textbf{une seule courbe de référence}, $\mathcal{N}(0, 1)$. Les aires sous cette courbe standard ($\Phi(z)$) ont été calculées une fois pour toutes et sont disponibles dans des tables ou des logiciels. On n'a plus qu'à convertir notre problème dans cette "langue" standard, trouver la probabilité, et interpréter le résultat.
\end{intuitionbox}

La notation est très standardisée pour cette loi.

\begin{remarquebox}[Notation $\phi$ et $\Phi$]
Les symboles $\phi$ (phi minuscule) pour la PDF et $\Phi$ (phi majuscule) pour la CDF de la loi normale standard sont quasi universels. Il est important de ne pas les confondre. $\phi(z)$ est la \textit{hauteur} de la courbe en $z$, tandis que $\Phi(z)$ est l'\textit{aire} sous la courbe à gauche de $z$.
\end{remarquebox}

Un détail technique important concerne le calcul de $\Phi(z)$.

\begin{remarquebox}[Absence de Primitive Simple]
L'intégrale $\int e^{-t^2/2} \, \mathrm{d}t$, nécessaire pour calculer $\Phi(z)$, n'a \textbf{pas d'expression analytique} en termes de fonctions élémentaires (polynômes, exponentielles, log, sin, cos...). C'est une fonction spéciale, connue sous le nom de \textbf{fonction d'erreur} (liée à $\Phi$ par une transformation simple). C'est la raison pour laquelle on dépend de tables ou de calculs numériques pour obtenir les valeurs de $\Phi(z)$. Heureusement, ces outils sont omniprésents aujourd'hui.
\end{remarquebox}

\subsection{Standardisation : Le Score Z}

Formalisons cette transformation clé qui relie toute loi normale à la loi standard.

\begin{theorembox}[Standardisation d'une Variable Normale]
Si $X \sim \mathcal{N}(\mu, \sigma^2)$, alors la variable $Z$ définie par :
$$ Z = \frac{X - \mu}{\sigma} $$
suit la loi normale standard, $Z \sim \mathcal{N}(0, 1)$.
\end{theorembox}

La preuve formelle utilise un changement de variable dans la fonction de répartition.

\begin{proofbox}
Soit $F_X(x)$ la CDF de $X$ et $F_Z(z)$ la CDF de $Z$. Nous voulons montrer que $F_Z(z) = \Phi(z)$.
\begin{align*}
F_Z(z) &= P(Z \le z) \\
&= P\left( \frac{X-\mu}{\sigma} \le z \right) \\
&= P(X - \mu \le z\sigma) \\
&= P(X \le \mu + z\sigma) \\
&= F_X(\mu + z\sigma)
\end{align*}
Par définition de la CDF de $X$ :
$$ F_X(x) = \int_{-\infty}^x \frac{1}{\sigma \sqrt{2\pi}} e^{ -\frac{1}{2} \left( \frac{t-\mu}{\sigma} \right)^2 } \, dt $$
Donc,
$$ F_Z(z) = \int_{-\infty}^{\mu + z\sigma} \frac{1}{\sigma \sqrt{2\pi}} e^{ -\frac{1}{2} \left( \frac{t-\mu}{\sigma} \right)^2 } \, dt $$
Effectuons le changement de variable $u = (t-\mu)/\sigma$. Alors $t = \mu + u\sigma$ et $dt = \sigma du$.
Les bornes d'intégration changent :
\begin{itemize}
    \item Quand $t \to -\infty$, $u \to -\infty$.
    \item Quand $t = \mu + z\sigma$, $u = ((\mu + z\sigma)-\mu)/\sigma = z$.
\end{itemize}
L'intégrale devient :
$$ F_Z(z) = \int_{-\infty}^{z} \frac{1}{\sigma \sqrt{2\pi}} e^{ -\frac{1}{2} u^2 } (\sigma du) $$
$$ F_Z(z) = \int_{-\infty}^{z} \frac{1}{\sqrt{2\pi}} e^{ -u^2/2 } \, du $$
C'est exactement la définition de $\Phi(z)$, la CDF de la loi normale standard. Ainsi, $Z \sim \mathcal{N}(0, 1)$.
\end{proofbox}

Cette transformation a une interprétation très concrète.

\begin{intuitionbox}[Mesurer en "Unités d'Écart-Type"]
Transformer $X$ en $Z$ s'appelle \textbf{standardiser} la variable. Le résultat, $z = \frac{x-\mu}{\sigma}$, est appelé le \textbf{Score Z} (ou cote Z). Ce score Z est une mesure \textit{sans unité} qui indique \textbf{à combien d'écarts-types} une valeur observée $x$ se situe par rapport à la moyenne $\mu$ de sa distribution.
\begin{itemize}
    \item $z = 0$ : $x$ est exactement à la moyenne ($\mathbf{x = \mu}$).
    \item $z = +1$ : $x$ est un écart-type \textit{au-dessus} de la moyenne ($\mathbf{x = \mu + \sigma}$).
    \item $z = -2$ : $x$ est deux écarts-types \textit{en dessous} de la moyenne ($\mathbf{x = \mu - 2\sigma}$).
\end{itemize}
Cette transformation est extrêmement utile pour :
\begin{enumerate}
    \item \textbf{Comparer des valeurs} issues de distributions normales différentes. Un score Z de +1.5 a toujours la même signification relative, que l'on parle de QI, de taille, ou de température.
    \item \textbf{Calculer des probabilités} en utilisant la table unique de la loi $\mathcal{N}(0, 1)$.
\end{enumerate}
\end{intuitionbox}

Un exemple classique est la comparaison de notes.

\begin{examplebox}[Comparaison de Performances]
Un étudiant A obtient 80 points à un examen où la moyenne est $\mu_A=70$ et l'écart-type $\sigma_A=5$. Un étudiant B obtient 85 points à un autre examen où $\mu_B=75$ et $\sigma_B=10$. Qui a le mieux réussi relativement à son groupe ?

Calculons les Z-scores :
$$ Z_A = \frac{80 - 70}{5} = \frac{10}{5} = +2.0 $$
$$ Z_B = \frac{85 - 75}{10} = \frac{10}{10} = +1.0 $$
L'étudiant A a un score Z plus élevé (+2.0 contre +1.0), ce qui signifie qu'il se situe plus d'écarts-types au-dessus de la moyenne de son groupe que l'étudiant B. L'étudiant A a donc relativement mieux réussi.
\end{examplebox}

\subsection{Propriétés Importantes de la Loi Normale}

La loi normale possède des propriétés de stabilité remarquables sous certaines transformations.

\begin{theorembox}[Stabilité par Transformation Linéaire]
Si $X \sim \mathcal{N}(\mu, \sigma^2)$ et $Y = aX + b$ (avec $a \neq 0$), alors $Y$ suit aussi une loi normale :
$$ Y \sim \mathcal{N}(a\mu + b, \, (a\sigma)^2) $$
L'espérance est transformée linéairement ($E[aX+b] = aE[X]+b$), et la variance est multipliée par $a^2$ ($\text{Var}(aX+b) = a^2\text{Var}(X)$).
\end{theorembox}

\begin{proofbox}
Nous utilisons le fait que si $X \sim \mathcal{N}(\mu, \sigma^2)$, alors $Z = (X-\mu)/\sigma \sim \mathcal{N}(0,1)$.
Exprimons $X$ en fonction de $Z$ : $X = \mu + \sigma Z$.
Substituons cela dans l'expression de $Y$:
$$ Y = a(\mu + \sigma Z) + b = (a\mu + b) + (a\sigma)Z $$
Posons $\mu_Y = a\mu + b$ et $\sigma_Y = |a|\sigma$. Alors $Y = \mu_Y + \sigma_Y Z$ (si $a>0$) ou $Y = \mu_Y - \sigma_Y Z$ (si $a<0$).
Dans les deux cas, $Y$ est une transformation linéaire d'une variable normale standard $Z$.
La CDF de $Y$ peut être exprimée en termes de la CDF $\Phi$ de $Z$.
Si $a>0$ :
$$ P(Y \le y) = P(\mu_Y + a\sigma Z \le y) = P(a\sigma Z \le y - \mu_Y) = P\left( Z \le \frac{y - \mu_Y}{a\sigma} \right) = \Phi\left(\frac{y - \mu_Y}{a\sigma}\right) $$
C'est la CDF d'une loi $\mathcal{N}(\mu_Y, (a\sigma)^2)$.
Le cas $a<0$ est similaire et mène au même résultat pour la distribution (la variance dépend de $a^2$).
Ainsi, $Y \sim \mathcal{N}(a\mu + b, (a\sigma)^2)$.
\end{proofbox}

Cette propriété est très utile pour les changements d'unités.

\begin{examplebox}[Changement d'Unités]
Si la température en Celsius $T_C$ suit $\mathcal{N}(20, 5^2)$, quelle est la loi de la température en Fahrenheit $T_F = \frac{9}{5}T_C + 32$ ?

$a = 9/5$, $b=32$.

Nouvelle moyenne : $E[T_F] = \frac{9}{5}(20) + 32 = 36 + 32 = 68$.

Nouvel écart-type : $\sigma_{T_F} = |a|\sigma_{T_C} = \frac{9}{5}(5) = 9$. Nouvelle variance : $\sigma_{T_F}^2 = 9^2 = 81$.

Donc, $T_F \sim \mathcal{N}(68, 9^2)$.
\end{examplebox}

Une autre propriété cruciale concerne la somme de variables normales indépendantes.

\begin{theorembox}[Stabilité par Addition (Indépendance)]
Si $X \sim \mathcal{N}(\mu_X, \sigma_X^2)$ et $Y \sim \mathcal{N}(\mu_Y, \sigma_Y^2)$ sont des variables aléatoires \textbf{indépendantes}, alors leur somme $S = X + Y$ suit aussi une loi normale :
$$ S \sim \mathcal{N}(\mu_X + \mu_Y, \, \sigma_X^2 + \sigma_Y^2) $$
Les moyennes s'ajoutent, et (grâce à l'indépendance) les variances s'ajoutent.
\end{theorembox}

La preuve formelle de ce théorème est plus avancée et utilise généralement les fonctions caractéristiques ou les fonctions génératrices des moments.

\begin{proofbox}[Idée de la preuve (via Fonctions Caractéristiques)]
La fonction caractéristique $\varphi_X(t)$ d'une variable aléatoire $X$ est définie comme $\varphi_X(t) = E[e^{itX}]$.
Pour une loi normale $X \sim \mathcal{N}(\mu, \sigma^2)$, sa fonction caractéristique est $\varphi_X(t) = e^{i\mu t - \frac{1}{2}\sigma^2 t^2}$.
Si $X$ et $Y$ sont indépendantes, la fonction caractéristique de leur somme $S=X+Y$ est le produit de leurs fonctions caractéristiques : $\varphi_S(t) = \varphi_X(t) \varphi_Y(t)$.
\begin{align*}
\varphi_S(t) &= \left( e^{i\mu_X t - \frac{1}{2}\sigma_X^2 t^2} \right) \left( e^{i\mu_Y t - \frac{1}{2}\sigma_Y^2 t^2} \right) \\
&= e^{i(\mu_X + \mu_Y)t - \frac{1}{2}(\sigma_X^2 + \sigma_Y^2)t^2}
\end{align*}
On reconnaît ici la fonction caractéristique d'une loi normale avec pour moyenne $\mu_X + \mu_Y$ et pour variance $\sigma_X^2 + \sigma_Y^2$. Comme la fonction caractéristique détermine de manière unique la distribution, on conclut que $S \sim \mathcal{N}(\mu_X + \mu_Y, \sigma_X^2 + \sigma_Y^2)$.
\end{proofbox}

Il est essentiel de se souvenir de la condition d'indépendance pour l'addition des variances.

\begin{remarquebox}[Attention à l'Indépendance]
La propriété d'addition des variances ($\sigma_S^2 = \sigma_X^2 + \sigma_Y^2$) est cruciale et ne tient \textbf{que si $X$ et $Y$ sont indépendantes}. Si elles ne le sont pas, la variance de la somme inclut un terme de covariance : $\text{Var}(X+Y) = \text{Var}(X) + \text{Var}(Y) + 2\text{Cov}(X, Y)$. Cependant, la somme de variables normales (même dépendantes) reste normale (si elles sont conjointement normales).
\end{remarquebox}

Appliquons ce théorème à un exemple concret.

\begin{examplebox}[Poids Total]
Le poids d'une pomme suit $\mathcal{N}(150g, 10^2)$. Le poids d'une orange suit $\mathcal{N}(200g, 15^2)$. On suppose les poids indépendants. Quel est la loi du poids total d'une pomme et d'une orange ?

Soit $P$ le poids de la pomme, $O$ celui de l'orange. $T = P+O$.

$E[T] = E[P] + E[O] = 150 + 200 = 350g$.

$\text{Var}(T) = \text{Var}(P) + \text{Var}(O) = 10^2 + 15^2 = 100 + 225 = 325$.

Donc, $T \sim \mathcal{N}(350, 325)$. L'écart-type du poids total est $\sqrt{325} \approx 18.03g$.
\end{examplebox}

\subsection{La Règle Empirique (68-95-99.7)}

Une conséquence directe des aires sous la courbe normale standard est une règle approximative très utile.

\begin{theorembox}[Règle Empirique]
Pour toute variable $X \sim \mathcal{N}(\mu, \sigma^2)$ :
\begin{itemize}
    \item $P(\mu - \sigma \le X \le \mu + \sigma) \approx 0.6827$ (Environ \textbf{68\%} des valeurs dans $\mu \pm \sigma$).
    \item $P(\mu - 2\sigma \le X \le \mu + 2\sigma) \approx 0.9545$ (Environ \textbf{95\%} des valeurs dans $\mu \pm 2\sigma$).
    \item $P(\mu - 3\sigma \le X \le \mu + 3\sigma) \approx 0.9973$ (Environ \textbf{99.7\%} des valeurs dans $\mu \pm 3\sigma$).
\end{itemize}
\end{theorembox}

\begin{proofbox}[Dérivation à partir de $\Phi(z)$]
Ces valeurs sont obtenues en calculant les aires sous la PDF de la loi normale standard $\mathcal{N}(0, 1)$ entre les Z-scores correspondants.
\begin{itemize}
    \item $P(-1 \le Z \le 1) = \Phi(1) - \Phi(-1) = \Phi(1) - (1 - \Phi(1)) = 2\Phi(1) - 1$.
    Avec $\Phi(1) \approx 0.8413$, on obtient $2(0.8413) - 1 \approx 0.6826$.
    \item $P(-2 \le Z \le 2) = \Phi(2) - \Phi(-2) = 2\Phi(2) - 1$.
    Avec $\Phi(2) \approx 0.9772$, on obtient $2(0.9772) - 1 \approx 0.9544$.
    \item $P(-3 \le Z \le 3) = \Phi(3) - \Phi(-3) = 2\Phi(3) - 1$.
    Avec $\Phi(3) \approx 0.99865$, on obtient $2(0.99865) - 1 \approx 0.9973$.
\end{itemize}
Ces valeurs sont souvent arrondies à 68%, 95%, et 99.7% pour faciliter la mémorisation.
\end{proofbox}

Cette règle fournit des repères très pratiques.

\begin{intuitionbox}[Repères Essentiels sur la Cloche]
Cette règle, dérivée directement des aires sous la courbe $\mathcal{N}(0, 1)$ entre $z=\pm 1$, $z=\pm 2$ et $z=\pm 3$, fournit des repères extrêmement utiles pour interpréter l'écart-type $\sigma$. Elle nous dit où se trouve la grande majorité des données.


Une observation qui tombe en dehors de l'intervalle $\mu \pm 3\sigma$ est très inhabituelle (elle n'a que 0.3% de chances de se produire). C'est souvent considéré comme une \textit{valeur aberrante} (outlier) potentielle.
\end{intuitionbox}

\subsection{Calcul de Probabilités Normales}

En pratique, pour calculer une probabilité $P(a \le X \le b)$ pour une loi $\mathcal{N}(\mu, \sigma^2)$, on utilise systématiquement la standardisation.

\begin{examplebox}[Utilisation du Z-score]
Supposons que le QI d'une population suit $\mathcal{N}(100, 15^2)$. Quelle est la probabilité $P(X > 130)$ ?

1.  \textbf{Standardiser :} $z = \frac{130 - 100}{15} = 2$. On cherche $P(Z > 2)$.
2.  \textbf{Utiliser la CDF Standard :} $P(Z > 2) = 1 - P(Z \le 2) = 1 - \Phi(2)$.
3.  \textbf{Chercher dans la table / Calculer :} $\Phi(2) \approx 0.9772$.
4.  \textbf{Résultat :} $P(X > 130) = 1 - 0.9772 = 0.0228$. Environ 2.3% de la population a un QI supérieur à 130.
\end{examplebox}

Pour les intervalles, on utilise la propriété $P(a \le Z \le b) = \Phi(b) - \Phi(a)$.

\begin{examplebox}[Probabilité entre deux valeurs]
Quelle est la probabilité $P(85 \le X \le 115)$ ? ($\mu=100, \sigma=15$)

1.  \textbf{Standardiser :} $z_1 = \frac{85 - 100}{15} = -1$, $z_2 = \frac{115 - 100}{15} = 1$. On cherche $P(-1 \le Z \le 1)$.
2.  \textbf{Utiliser la CDF Standard :} $P(-1 \le Z \le 1) = \Phi(1) - \Phi(-1)$.
3.  \textbf{Utiliser la symétrie :} $\Phi(-z) = 1 - \Phi(z)$. Donc $\Phi(-1) = 1 - \Phi(1)$.
    $P(-1 \le Z \le 1) = \Phi(1) - (1 - \Phi(1)) = 2\Phi(1) - 1$.
4.  \textbf{Chercher dans la table / Calculer :} $\Phi(1) \approx 0.8413$.
5.  \textbf{Résultat :} $P(85 \le X \le 115) \approx 2(0.8413) - 1 = 1.6826 - 1 = 0.6826$. (On retrouve la règle des 68% !)
\end{examplebox}

On peut aussi inverser le processus : trouver la valeur $x$ correspondant à une probabilité donnée.

\begin{examplebox}[Trouver une valeur pour une probabilité donnée (Problème Inverse)]
Quel est le QI minimum requis pour être dans le top 10\% de la population ? ($\mu=100, \sigma=15$).

1.  \textbf{Trouver le Z-score correspondant :} On cherche $x$ tel que $P(X > x) = 0.10$. Cela équivaut à $P(Z > z) = 0.10$, où $z = (x-100)/15$.
    Si $P(Z > z) = 0.10$, alors $P(Z \le z) = \Phi(z) = 1 - 0.10 = 0.90$.
2.  \textbf{Chercher dans la table inverse / Calculer :} On cherche la valeur $z$ pour laquelle l'aire à gauche est 0.90 (le 90ème percentile). On trouve $z \approx 1.28$.
3.  \textbf{Convertir en X :} On utilise la relation $z = (x-\mu)/\sigma$ pour trouver $x$:
    $1.28 = \frac{x - 100}{15}$
    $x = 100 + 1.28 \times 15 = 100 + 19.2 = 119.2$.
    Il faut un QI d'environ 119.2 pour être dans le top 10\%.
\end{examplebox}
\subsection{Exercices}

\textit{Pour tous les exercices de calcul, vous pouvez utiliser les valeurs suivantes pour la fonction de répartition de la loi normale standard $\Phi(z) = P(Z \le z)$ :}
\begin{itemize}
    \item $\Phi(0) = 0.5$
    \item $\Phi(0.675) \approx 0.75$
    \item $\Phi(1) \approx 0.8413$
    \item $\Phi(1.28) \approx 0.90$
    \item $\Phi(1.5) \approx 0.9332$
    \item $\Phi(1.96) \approx 0.975$
    \item $\Phi(2) \approx 0.9772$
    \item $\Phi(2.5) \approx 0.9938$
    \item $\Phi(3) \approx 0.9987$
\end{itemize}
\textit{Et rappelez-vous la propriété de symétrie : $\Phi(-z) = 1 - \Phi(z)$.}

% --- Section 1 : Concepts, Z-Scores & Règle Empirique ---

\begin{exercicebox}[Exercice 1 : Concepts (PDF)]
Soit deux lois normales $A \sim \mathcal{N}(10, 4)$ et $B \sim \mathcal{N}(10, 9)$.
\begin{enumerate}
    \item Laquelle des deux distributions a le pic le plus élevé ?
    \item Laquelle des deux distributions est la plus "aplatie" ?
\end{enumerate}
\end{exercicebox}

\begin{exercicebox}[Exercice 2 : Z-Score (Calcul)]
La taille des étudiants suit $X \sim \mathcal{N}(175, 6^2)$, où les unités sont en cm. Un étudiant mesure 184 cm. Calculez son Z-score.
\end{exercicebox}

\begin{exercicebox}[Exercice 3 : Z-Score (Interprétation)]
Un score de $Z = -2.5$ est obtenu pour une variable $X$. Qu'est-ce que cela signifie en termes de moyenne ($\mu$) et d'écart-type ($\sigma$) ?
\end{exercicebox}

\begin{exercicebox}[Exercice 4 : Z-Score (Calcul Inverse)]
Pour une distribution $\mathcal{N}(50, 100)$, quelle valeur $x$ correspond à un Z-score de $z = 1.5$ ?
\end{exercicebox}

\begin{exercicebox}[Exercice 5 : Z-Score (Comparaison)]
Alice obtient 115 à un test de QI $\mathcal{N}(100, 15^2)$. Bob obtient 24 à un test d'aptitude $\mathcal{N}(20, 2^2)$. Qui a le mieux réussi par rapport à son groupe ?
\end{exercicebox}

\begin{exercicebox}[Exercice 6 : Règle Empirique (68-95-99.7)]
Le poids de paquets de café suit $\mathcal{N}(500g, 10^2)$.
En utilisant la règle empirique, quel pourcentage approximatif de paquets pèse entre 490g et 510g ?
\end{exercicebox}

\begin{exercicebox}[Exercice 7 : Règle Empirique (Queue)]
En utilisant la règle empirique pour $\mathcal{N}(\mu, \sigma^2)$, quelle est la probabilité approximative $P(X > \mu + 2\sigma)$ ?
\end{exercicebox}

% --- Section 2 : Calculs de Probabilités (Standardisation) ---

\begin{exercicebox}[Exercice 8 : Loi Standard (Lecture Directe)]
Soit $Z \sim \mathcal{N}(0, 1)$. Calculez $P(Z \le 1.5)$.
\end{exercicebox}

\begin{exercicebox}[Exercice 9 : Loi Standard (Queue Droite)]
Soit $Z \sim \mathcal{N}(0, 1)$. Calculez $P(Z > 1)$.
\end{exercicebox}

\begin{exercicebox}[Exercice 10 : Loi Standard (Queue Gauche)]
Soit $Z \sim \mathcal{N}(0, 1)$. Calculez $P(Z \le -2)$.
\end{exercicebox}

\begin{exercicebox}[Exercice 11 : Loi Standard (Intervalle)]
Soit $Z \sim \mathcal{N}(0, 1)$. Calculez $P(-1 \le Z \le 2)$.
\end{exercicebox}

\begin{exercicebox}[Exercice 12 : Calcul (Standardisation)]
Soit $X \sim \mathcal{N}(50, 25)$. (Attention : $\sigma^2=25$).
Calculez $P(X \le 60)$.
\end{exercicebox}

\begin{exercicebox}[Exercice 13 : Calcul (Standardisation)]
Soit $X \sim \mathcal{N}(100, 225)$. (Attention : $\sigma^2=225$).
Calculez $P(X > 77.5)$.
\end{exercicebox}

\begin{exercicebox}[Exercice 14 : Calcul (Intervalle)]
La durée de vie d'une batterie suit $X \sim \mathcal{N}(40 \text{ heures}, 16)$.
Calculez $P(38 \le X \le 42)$.
\end{exercicebox}

\begin{exercicebox}[Exercice 15 : Propriété de la CDF]
En utilisant les définitions, montrez que $P(Z > z) = P(Z \le -z)$.
\end{exercicebox}

% --- Section 3 : Problèmes Inverses ---

\begin{exercicebox}[Exercice 16 : Inverse (Z-Score)]
Soit $Z \sim \mathcal{N}(0, 1)$. Trouvez la valeur $z$ telle que $P(Z \le z) = 0.975$.
\end{exercicebox}

\begin{exercicebox}[Exercice 17 : Inverse (Z-Score Queue Droite)]
Soit $Z \sim \mathcal{N}(0, 1)$. Trouvez la valeur $z$ telle que $P(Z > z) = 0.1587$.
(Indice : $1 - 0.1587 = 0.8413$).
\end{exercicebox}

\begin{exercicebox}[Exercice 18 : Inverse (Z-Score Intervalle Central)]
Soit $Z \sim \mathcal{N}(0, 1)$. Trouvez la valeur $z$ (positive) telle que $P(-z \le Z \le z) = 0.95$.
(Indice : si l'aire centrale est 0.95, combien vaut l'aire à gauche de $z$ ?)
\end{exercicebox}

\begin{exercicebox}[Exercice 19 : Problème Inverse (Application)]
Le score à un examen suit $\mathcal{N}(100, 15^2)$. Pour obtenir la note "A", un étudiant doit être dans le top 2.5\% (c'est-à-dire $P(X > x) = 0.025$).
Quel score $x$ minimum faut-il obtenir ?
\end{exercicebox}

\begin{exercicebox}[Exercice 20 : Problème Inverse (Application)]
Une machine remplit des sacs de farine $\mathcal{N}(1000g, 20^2)$. On veut garantir que 99.87\% des sacs pèsent *plus* qu'un certain poids $x$.
Quelle est la valeur de $x$ ?
\end{exercicebox}

% --- Section 4 : Propriétés (Stabilité, Sommes) ---

\begin{exercicebox}[Exercice 21 : Stabilité par Transformation Linéaire]
Si $X \sim \mathcal{N}(10, 4)$ (donc $\sigma=2$), quelle est la loi de la variable $Y = 3X + 5$ ?
\end{exercicebox}

\begin{exercicebox}[Exercice 22 : Stabilité (Changement d'Unités)]
La taille $T_{cm}$ d'une population suit $\mathcal{N}(170, 100)$. Quelle est la loi de la taille $T_{m}$ en mètres ? (Rappel : $T_{m} = T_{cm} / 100$).
\end{exercicebox}

\begin{exercicebox}[Exercice 23 : Stabilité par Addition (Indépendante)]
Soit $X \sim \mathcal{N}(50, 10)$ et $Y \sim \mathcal{N}(30, 6)$ deux variables indépendantes.
Quelle est la loi de la somme $S = X + Y$ ?
\end{exercicebox}

\begin{exercicebox}[Exercice 24 : Stabilité par Différence (Indépendante)]
En utilisant les variables $X$ et $Y$ de l'exercice 23, quelle est la loi de la différence $D = X - Y$ ?
(Indice : $D = X + (-1)Y$).
\end{exercicebox}

\begin{exercicebox}[Exercice 25 : Somme de $n$ variables i.i.d.]
On prélève 10 pommes d'un lot où le poids d'une pomme suit $\mathcal{N}(150g, 10^2)$. Soit $P_{total}$ le poids total des 10 pommes (supposées indépendantes).
Quelle est la loi de $P_{total}$ ?
\end{exercicebox}

\subsection{Corrections des Exercices}

\begin{correctionbox}[Correction Exercice 1 : Concepts (PDF)]
1.  La variance de A ($\sigma_A^2=4$) est plus petite que celle de B ($\sigma_B^2=9$). Une variance plus petite signifie une distribution plus concentrée, donc un pic plus élevé (l'aire totale devant rester 1). C'est la distribution A.
2.  La distribution B ($\sigma_B=3$) a un écart-type plus grand que A ($\sigma_A=2$), elle est donc plus dispersée, c'est-à-dire plus "large et aplatie".
\end{correctionbox}

\begin{correctionbox}[Correction Exercice 2 : Z-Score (Calcul)]
On a $x=184$, $\mu=175$, $\sigma=6$.
$$Z = \frac{x - \mu}{\sigma} = \frac{184 - 175}{6} = \frac{9}{6} = 1.5$$
L'étudiant se situe à 1.5 écarts-types au-dessus de la moyenne.
\end{correctionbox}

\begin{correctionbox}[Correction Exercice 3 : Z-Score (Interprétation)]
Cela signifie que la valeur $X$ observée est $2.5$ écarts-types \textbf{en dessous} de la moyenne $\mu$. (c-à-d, $x = \mu - 2.5\sigma$).
\end{correctionbox}

\begin{correctionbox}[Correction Exercice 4 : Z-Score (Calcul Inverse)]
On a $\mu=50$ et $\sigma^2=100 \implies \sigma=10$. On cherche $x$.
$z = \frac{x - \mu}{\sigma} \implies x = \mu + z\sigma$
$x = 50 + (1.5)(10) = 50 + 15 = 65$.
\end{correctionbox}

\begin{correctionbox}[Correction Exercice 5 : Z-Score (Comparaison)]
On compare les Z-scores :
$Z_{Alice} = \frac{115 - 100}{15} = \frac{15}{15} = 1.0$.
$Z_{Bob} = \frac{24 - 20}{2} = \frac{4}{2} = 2.0$.
Bob a un Z-score plus élevé (2.0 contre 1.0), il a donc mieux réussi par rapport à son groupe.
\end{correctionbox}

\begin{correctionbox}[Correction Exercice 6 : Règle Empirique (68-95-99.7)]
On a $\mu=500$ et $\sigma=10$. L'intervalle [490, 510] correspond à $[\mu - 1\sigma, \mu + 1\sigma]$.
Selon la règle empirique, environ \textbf{68\%} des paquets se trouvent dans cet intervalle.
\end{correctionbox}

\begin{correctionbox}[Correction Exercice 7 : Règle Empirique (Queue)]
La règle dit que $P(\mu - 2\sigma \le X \le \mu + 2\sigma) \approx 0.95$.
L'aire totale est 1. L'aire en dehors de cet intervalle est $1 - 0.95 = 0.05$.
En raison de la symétrie, cette aire de 0.05 est répartie également entre les deux queues (gauche et droite).
L'aire de la queue droite $P(X > \mu + 2\sigma)$ est donc $0.05 / 2 = 0.025$ (soit 2.5\%).
\end{correctionbox}

\begin{correctionbox}[Correction Exercice 8 : Loi Standard (Lecture Directe)]
$P(Z \le 1.5) = \Phi(1.5)$. En utilisant la table fournie, $P(Z \le 1.5) \approx 0.9332$.
\end{correctionbox}

\begin{correctionbox}[Correction Exercice 9 : Loi Standard (Queue Droite)]
$P(Z > 1) = 1 - P(Z \le 1) = 1 - \Phi(1)$.
$P(Z > 1) \approx 1 - 0.8413 = 0.1587$.
\end{correctionbox}

\begin{correctionbox}[Correction Exercice 10 : Loi Standard (Queue Gauche)]
$P(Z \le -2) = \Phi(-2)$. Par symétrie, $\Phi(-2) = 1 - \Phi(2)$.
$P(Z \le -2) \approx 1 - 0.9772 = 0.0228$.
\end{correctionbox}

\begin{correctionbox}[Correction Exercice 11 : Loi Standard (Intervalle)]
$P(-1 \le Z \le 2) = P(Z \le 2) - P(Z \le -1) = \Phi(2) - \Phi(-1)$.
$\Phi(-1) = 1 - \Phi(1) \approx 1 - 0.8413 = 0.1587$.
$P(-1 \le Z \le 2) \approx 0.9772 - 0.1587 = 0.8185$.
\end{correctionbox}

\begin{correctionbox}[Correction Exercice 12 : Calcul (Standardisation)]
$X \sim \mathcal{N}(50, 25) \implies \mu=50, \sigma=5$.
On cherche $P(X \le 60)$.
$Z = \frac{60 - 50}{5} = \frac{10}{5} = 2$.
$P(X \le 60) = P(Z \le 2) = \Phi(2) \approx 0.9772$.
\end{correctionbox}

\begin{correctionbox}[Correction Exercice 13 : Calcul (Standardisation)]
$X \sim \mathcal{N}(100, 225) \implies \mu=100, \sigma=15$.
On cherche $P(X > 77.5)$.
$Z = \frac{77.5 - 100}{15} = \frac{-22.5}{15} = -1.5$.
$P(X > 77.5) = P(Z > -1.5) = 1 - P(Z \le -1.5) = 1 - \Phi(-1.5)$.
$\Phi(-1.5) = 1 - \Phi(1.5) \approx 1 - 0.9332 = 0.0668$.
$P(Z > -1.5) = 1 - 0.0668 = 0.9332$. (Logique : $P(Z > -1.5) = P(Z \le 1.5)$ par symétrie).
\end{correctionbox}

\begin{correctionbox}[Correction Exercice 14 : Calcul (Intervalle)]
$X \sim \mathcal{N}(40, 16) \implies \mu=40, \sigma=4$.
On cherche $P(38 \le X \le 42)$.
$z_1 = \frac{38 - 40}{4} = -0.5$. $z_2 = \frac{42 - 40}{4} = 0.5$.
$P(38 \le X \le 42) = P(-0.5 \le Z \le 0.5) = \Phi(0.5) - \Phi(-0.5)$.
$\Phi(-0.5) = 1 - \Phi(0.5)$.
$P = \Phi(0.5) - (1 - \Phi(0.5)) = 2\Phi(0.5) - 1$.
(La valeur $\Phi(0.5) \approx 0.6915$ n'est pas fournie, mais $P(-1 \le Z \le 1) \approx 0.68$, donc on s'attend à une valeur plus petite).
\end{correctionbox}

\begin{correctionbox}[Correction Exercice 15 : Propriété de la CDF]
$P(Z > z) = 1 - P(Z \le z) = 1 - \Phi(z)$.
$P(Z \le -z) = \Phi(-z)$.
Par symétrie de la PDF $\phi(z) = \phi(-z)$, l'aire à droite de $z$ est égale à l'aire à gauche de $-z$.
Donc $P(Z > z) = P(Z \le -z)$, ce qui implique $1 - \Phi(z) = \Phi(-z)$.
\end{correctionbox}

\begin{correctionbox}[Correction Exercice 16 : Inverse (Z-Score)]
On cherche $z$ tel que $P(Z \le z) = 0.975$.
C'est $\Phi(z) = 0.975$. D'après la table, $z = 1.96$.
\end{correctionbox}

\begin{correctionbox}[Correction Exercice 17 : Inverse (Z-Score Queue Droite)]
On cherche $z$ tel que $P(Z > z) = 0.1587$.
Cela signifie $P(Z \le z) = 1 - 0.1587 = 0.8413$.
$\Phi(z) = 0.8413$. D'après la table, $z = 1$.
\end{correctionbox}

\begin{correctionbox}[Correction Exercice 18 : Inverse (Z-Score Intervalle Central)]
Si $P(-z \le Z \le z) = 0.95$, l'aire restante dans les deux queues est $1 - 0.95 = 0.05$.
Par symétrie, l'aire dans la queue gauche $P(Z < -z)$ est $0.05 / 2 = 0.025$.
L'aire totale à gauche de $z$ est $P(Z \le z) = 0.95 + 0.025 = 0.975$.
On cherche $z$ tel que $\Phi(z) = 0.975$.
D'après la table, $z = 1.96$. (On retrouve $\mu \pm 1.96\sigma$ comme l'intervalle à 95\% exact).
\end{correctionbox}

\begin{correctionbox}[Correction Exercice 19 : Problème Inverse (Application)]
$X \sim \mathcal{N}(100, 15^2)$. On cherche $x$ tel que $P(X > x) = 0.025$.
Standardisation : $P(Z > z) = 0.025$, où $z = (x-100)/15$.
$P(Z \le z) = 1 - 0.025 = 0.975$.
D'après la table, $z$ tel que $\Phi(z)=0.975$ est $z=1.96$.
On résout : $1.96 = \frac{x - 100}{15} \implies x = 100 + 1.96(15) = 100 + 29.4 = 129.4$.
Il faut un score minimum de 129.4.
\end{correctionbox}

\begin{correctionbox}[Correction Exercice 20 : Problème Inverse (Application)]
$X \sim \mathcal{N}(1000, 20^2)$. On cherche $x$ tel que $P(X > x) = 0.9987$.
Standardisation : $P(Z > z) = 0.9987$, où $z = (x-1000)/20$.
$P(Z \le z) = 1 - 0.9987 = 0.0013$.
C'est une valeur très faible. Utilisons la symétrie : $\Phi(-z) = 1 - \Phi(z)$.
$\Phi(z) = 0.0013$. On cherche $z$ dans la table.
On voit que $\Phi(3) = 0.9987$, donc $\Phi(-3) = 1 - 0.9987 = 0.0013$.
Le Z-score est $z = -3$.
On résout : $-3 = \frac{x - 1000}{20} \implies x = 1000 - 3(20) = 1000 - 60 = 940$.
Le poids garanti est 940g.
\end{correctionbox}

\begin{correctionbox}[Correction Exercice 21 : Stabilité par Transformation Linéaire]
$X \sim \mathcal{N}(10, 4) \implies \mu_X=10, \sigma_X^2=4$. $Y = aX + b$ avec $a=3, b=5$.
$Y$ suit une loi normale.
$E[Y] = a\mu_X + b = 3(10) + 5 = 35$.
$\text{Var}(Y) = a^2 \text{Var}(X) = 3^2 \times 4 = 9 \times 4 = 36$.
Donc, $Y \sim \mathcal{N}(35, 36)$.
\end{correctionbox}

\begin{correctionbox}[Correction Exercice 22 : Stabilité (Changement d'Unités)]
$T_{cm} \sim \mathcal{N}(170, 100) \implies \mu=170, \sigma^2=100$.
$T_m = a T_{cm} + b$ avec $a=1/100 = 0.01$ et $b=0$.
$E[T_m] = a\mu + b = 0.01(170) + 0 = 1.7$.
$\text{Var}(T_m) = a^2 \text{Var}(T_{cm}) = (0.01)^2 \times 100 = 0.0001 \times 100 = 0.01$.
Donc, $T_m \sim \mathcal{N}(1.7, 0.01)$. (L'écart-type est $\sigma_m = \sqrt{0.01} = 0.1$ m, ce qui est logique : 10cm = 0.1m).
\end{correctionbox}

\begin{correctionbox}[Correction Exercice 23 : Stabilité par Addition (Indépendante)]
$X \sim \mathcal{N}(50, 10)$, $Y \sim \mathcal{N}(30, 6)$. $X, Y$ indépendantes.
$S = X + Y$ suit une loi normale.
$E[S] = E[X] + E[Y] = 50 + 30 = 80$.
$\text{Var}(S) = \text{Var}(X) + \text{Var}(Y) = 10 + 6 = 16$.
Donc, $S \sim \mathcal{N}(80, 16)$.
\end{correctionbox}

\begin{correctionbox}[Correction Exercice 24 : Stabilité par Différence (Indépendante)]
$D = X - Y = X + (-1)Y$. C'est une somme de variables normales indépendantes (si $X, Y$ le sont, $X, -Y$ le sont aussi). $D$ suit une loi normale.
$E[D] = E[X] + E[-Y] = E[X] - E[Y] = 50 - 30 = 20$.
$\text{Var}(D) = \text{Var}(X) + \text{Var}(-1 \cdot Y) = \text{Var}(X) + (-1)^2 \text{Var}(Y)$
$\text{Var}(D) = \text{Var}(X) + \text{Var}(Y) = 10 + 6 = 16$.
Donc, $D \sim \mathcal{N}(20, 16)$. (Note : les variances s'ajoutent toujours !)
\end{correctionbox}

\begin{correctionbox}[Correction Exercice 25 : Somme de $n$ variables i.i.d.]
Soit $P_i \sim \mathcal{N}(150, 100)$ le poids de la $i$-ème pomme.
$P_{total} = P_1 + \dots + P_{10}$. C'est une somme de 10 v.a. normales indépendantes.
$P_{total}$ suit une loi normale.
$E[P_{total}] = E[P_1] + \dots + E[P_{10}] = 10 \times E[P_1] = 10 \times 150 = 1500g$.
$\text{Var}(P_{total}) = \text{Var}(P_1) + \dots + \text{Var}(P_{10}) = 10 \times \text{Var}(P_1) = 10 \times 100 = 1000$.
Donc, $P_{total} \sim \mathcal{N}(1500, 1000)$.
\end{correctionbox}

\subsection{Exercices Python}

Ces exercices appliquent les concepts de la loi normale au jeu de données "Yahoo Finance". Nous allons modéliser les \textbf{rendements journaliers} (variation en pourcentage) des actions, qui sont souvent (par approximation) considérés comme suivant une loi normale.

Nous allons travailler avec les rendements de Google ($X$) et de Microsoft ($Y$).

\begin{codecell}
!pip install yfinance
import yfinance as yf
import pandas as pd
import numpy as np
from scipy.stats import norm # Moteur pour les calculs de CDF/PDF

# Definir les tickers et la periode
tickers = ["GOOG", "MSFT"]
start_date = "2020-01-01"
end_date = "2024-12-31"

# Telecharger les prix de cloture ajustes
data = yf.download(tickers, start=start_date, end=end_date)["Adj Close"]

# Calculer les rendements journaliers en pourcentage
returns = data.pct_change().dropna()

# Renommer les colonnes pour plus de clarte
returns.columns = ["GOOG_Return", "MSFT_Return"]

# 'returns' est notre DataFrame principal.
# X = returns["GOOG_Return"]
# Y = returns["MSFT_Return"]
\end{codecell}

\begin{exercicebox}[Exercice 1 : Estimation des Paramètres $\mu$ et $\sigma^2$]
Soit $X$ la v.a. "Rendement journalier de GOOG" et $Y$ la v.a. "Rendement journalier de MSFT".
Nous supposons $X \sim \mathcal{N}(\mu_X, \sigma_X^2)$ et $Y \sim \mathcal{N}(\mu_Y, \sigma_Y^2)$.

\textbf{Votre tâche :}
\begin{enumerate}
    \item Estimer $\mu_X$ et $\mu_Y$ (les espérances) en calculant la moyenne empirique (\texttt{.mean()}) des deux séries.
    \item Estimer $\sigma_X^2$ et $\sigma_Y^2$ (les variances) en calculant la variance empirique (\texttt{.var()}).
    \item Estimer $\sigma_X$ et $\sigma_Y$ (les écarts-types) en prenant la racine carrée des variances (ou \texttt{.std()}).
\end{enumerate}
\end{exercicebox}

\begin{exercicebox}[Exercice 2 : Standardisation (Score Z)]
Le Z-score nous dit à combien d'écarts-types une observation se situe de la moyenne.

\textbf{Votre tâche :}
\begin{enumerate}
    \item Utiliser les $\mu_X$ et $\sigma_X$ (pour GOOG) estimés à l'exercice 1.
    \item Trouver le rendement du dernier jour disponible dans vos données pour GOOG.
    \item Calculer le Z-score de ce rendement : $Z = (x - \mu_X) / \sigma_X$.
    \item Interpréter ce score (ex: "Le rendement de ce jour était à [Z] écarts-types de la moyenne...").
\end{enumerate}
\end{exercicebox}

\begin{exercicebox}[Exercice 3 : Calcul de Probabilité (Utilisation de $\Phi$)]
En utilisant les paramètres $\mu_X$ et $\sigma_X$ pour GOOG (Ex 1) et en supposant la distribution normale :

\textbf{Votre tâche :}
\begin{enumerate}
    \item Calculer la probabilité qu'un jour donné, le rendement de GOOG soit "calme", c'est-à-dire $P(-0.01 \le X \le 0.01)$.
    \item (Indice : Standardisez $a=-0.01$ et $b=0.01$ en $z_a, z_b$, puis calculez $\Phi(z_b) - \Phi(z_a)$. Vous aurez besoin de \texttt{scipy.stats.norm.cdf()}).
\end{enumerate}
\end{exercicebox}

\begin{exercicebox}[Exercice 4 : Problème Inverse (Value at Risk)]
Le "Value at Risk" (VaR) est une valeur $x$ telle qu'il y a une probabilité $p$ de perdre plus que $x$.

\textbf{Votre tâche :}
\begin{enumerate}
    \item Toujours avec les paramètres de GOOG, trouvez la valeur $x$ (le rendement) qui correspond au "top 5\%" des pires jours.
    \item Autrement dit, trouvez $x$ tel que $P(X \le x) = 0.05$.
    \item (Indice : Trouvez le Z-score $z$ tel que $\Phi(z) = 0.05$ (vous aurez besoin de \texttt{scipy.stats.norm.ppf()}), puis "dé-standardisez" : $x = \mu_X + z \sigma_X$).
\end{enumerate}
\end{exercicebox}

\begin{exercicebox}[Exercice 5 : PDF (Calcul de la Densité)]
La PDF $f(x)$ n'est pas une probabilité, mais une "densité". La valeur $f(\mu)$ est le point le plus haut de la cloche.

\textbf{Votre tâche :}
\begin{enumerate}
    \item Utiliser les $\mu_X$ et $\sigma_X$ de GOOG (Ex 1).
    \item Calculer la valeur de la densité de probabilité au point $x=\mu_X$ (le pic de la cloche).
    \item Calculer la valeur de la densité au point $x=0.0$.
    \item (Indice : Vous aurez besoin de \texttt{scipy.stats.norm.pdf(x, loc=mu, scale=sigma)}).
\end{enumerate}
\end{exercicebox}

\begin{exercicebox}[Exercice 6 : Vérification de la Règle Empirique (68-95-99.7)]
La théorie dit que $\approx 68\%$ des données devraient être dans $\mu \pm \sigma$. Nous allons vérifier si les rendements boursiers respectent cette règle.

\textbf{Votre tâche :}
\begin{enumerate}
    \item Utiliser les $\mu_X$ et $\sigma_X$ de GOOG (Ex 1).
    \item Calculer la proportion \textbf{réelle} des rendements qui tombent dans l'intervalle $[\mu_X - \sigma_X, \mu_X + \sigma_X]$.
    \item (Bonus) Faire de même pour $[\mu_X - 2\sigma_X, \mu_X + 2\sigma_X]$ (théorie: 95\%).
    \item (Conclusion) Les rendements de GOOG semblent-ils suivre parfaitement la règle ?
\end{enumerate}
\end{exercicebox}

\begin{exercicebox}[Exercice 7 : Calcul des Quartiles]
L'écart interquartile (IQR) est une autre mesure de dispersion, $IQR = Q_3 - Q_1$.
$Q_1$ est la valeur $x$ telle que $P(X \le x) = 0.25$.
$Q_3$ est la valeur $x$ telle que $P(X \le x) = 0.75$.

\textbf{Votre tâche :}
\begin{enumerate}
    \item En utilisant le modèle normal pour GOOG ($\mu_X, \sigma_X$), trouver les Z-scores $z_1$ et $z_3$ pour $p=0.25$ et $p=0.75$ (\texttt{norm.ppf}).
    \item "Dé-standardiser" ces Z-scores pour trouver $Q_1$ et $Q_3$.
    \item Calculer l'IQR \textbf{théorique} ($Q_3 - Q_1$).
    \item Comparer cet IQR théorique à l'IQR \textbf{empirique} de la série $X$ (Indice : \texttt{X.quantile(0.75) - X.quantile(0.25)}).
\end{enumerate}
\end{exercicebox}

\begin{exercicebox}[Exercice 8 : Stabilité par Transformation Linéaire (Y=aX+b)]
Soit un produit financier (ETF) qui amplifie par 2 les mouvements de GOOG, $Y = 2X$.
La théorie prédit $E[Y] = 2E[X]$ et $\text{Var}(Y) = 2^2 \text{Var}(X) = 4\text{Var}(X)$.

\textbf{Votre tâche :}
\begin{enumerate}
    \item Calculer $E[Y]$ et $\text{Var}(Y)$ \textbf{théoriquement} en utilisant $E[X]$ et $\text{Var}(X)$ de l'Ex 1.
    \item \textbf{Vérification empirique} :
        \begin{itemize}
            \item Créer la série de données $Y_{series} = 2 \times X$.
            \item Calculer la moyenne (\texttt{.mean()}) et la variance (\texttt{.var()}) de $Y_{series}$.
        \end{itemize}
    \item Comparer les résultats théoriques et empiriques.
\end{enumerate}
\end{exercicebox}

\begin{exercicebox}[Exercice 9 : Stabilité par Addition (Portefeuille S=X+Y)]
Théorie : $E[X+Y] = E[X] + E[Y]$ et $\text{Var}(X+Y) = \text{Var}(X) + \text{Var}(Y) + 2\text{Cov}(X,Y)$.
*Note : $X$ et $Y$ (GOOG, MSFT) ne sont PAS indépendantes.*

\textbf{Votre tâche :}
\begin{enumerate}
    \item Créer la série $S = X + Y$.
    \item Calculer $E[S]$ (la moyenne de $S$).
    \item Vérifier que $E[S] = E[X] + E[Y]$ (en utilisant les valeurs de l'Ex 1).
    \item Calculer $\text{Var}(S)$ (la variance de $S$).
    \item Est-ce que $\text{Var}(S) = \text{Var}(X) + \text{Var}(Y)$ ? Pourquoi y a-t-il une différence ?
\end{enumerate}
\end{exercicebox}

\begin{exercicebox}[Exercice 10 : Covariance et Variance du Portefeuille (Corrigé)]
Reprenons l'exercice 9. La différence est due à la covariance.

\textbf{Votre tâche :}
\begin{enumerate}
    \item Calculer $\text{Cov}(X,Y)$ (vous pouvez utiliser \texttt{returns.cov()}).
    \item Calculer la variance \textbf{théorique} de $S = X+Y$ avec la formule complète : $\text{Var}(S_{theo}) = \text{Var}(X) + \text{Var}(Y) + 2\text{Cov}(X,Y)$.
    \item Comparer $\text{Var}(S_{theo})$ à la variance empirique $\text{Var}(S)$ calculée à l'exercice 9.
\end{enumerate}
\end{exercicebox}

\begin{exercicebox}[Exercice 11 : Distribution d'un Portefeuille Pondéré]
Un portefeuille $P$ est composé de 60\% de GOOG ($X$) et 40\% de MSFT ($Y$). Donc $P = 0.6X + 0.4Y$.
Si $X$ et $Y$ sont (conjointement) normales, $P$ est aussi normale.

\textbf{Votre tâche :}
\begin{enumerate}
    \item Calculer l'espérance $E[P]$ (en utilisant la linéarité : $0.6E[X] + 0.4E[Y]$).
    \item Calculer la variance $\text{Var}(P)$ (en utilisant la formule complète : $\text{Var}(0.6X + 0.4Y) = 0.6^2\text{Var}(X) + 0.4^2\text{Var}(Y) + 2(0.6)(0.4)\text{Cov}(X,Y)$).
    \item Énoncer la loi de probabilité approximative du portefeuille : $P \sim \mathcal{N}(\mu_P, \sigma_P^2)$.
\end{enumerate}
\end{exercicebox}

\begin{exercicebox}[Exercice 12 : Distribution de la Différence (Pairs Trading)]
Un "spread" $D$ est la différence de rendement $D = X - Y$.
Théorie : $E[D] = E[X] - E[Y]$ et $\text{Var}(D) = \text{Var}(X) + \text{Var}(Y) - 2\text{Cov}(X,Y)$.

\textbf{Votre tâche :}
\begin{enumerate}
    \item Calculer $E[D]$ et $\text{Var}(D)$ \textbf{théoriquement} en utilisant les valeurs des exercices précédents.
    \item \textbf{Vérification empirique} :
        \begin{itemize}
            \item Créer la série de données $D_{series} = X - Y$.
            \item Calculer la moyenne (\texttt{.mean()}) et la variance (\texttt{.var()}) de $D_{series}$.
        \end{itemize}
    \item Comparer les résultats théoriques et empiriques.
\end{enumerate}
\end{exercicebox}
\newpage
\section{La Loi Normale (ou Gaussienne)}

\subsection{Introduction et Fonction de Densité (PDF)}

Après les lois discrètes et les lois continues de base (Uniforme, Exponentielle), nous abordons la distribution la plus célèbre et la plus utilisée en probabilités et statistiques.

\begin{definitionbox}[Loi Normale]
Une variable aléatoire continue $X$ suit une \textbf{loi normale} (ou loi de Gauss) de paramètres $\mu$ (l'espérance) et $\sigma^2$ (la variance), notée $X \sim \mathcal{N}(\mu, \sigma^2)$, si sa fonction de densité de probabilité (PDF) est donnée par :
$$ f(x; \mu, \sigma) = \frac{1}{\sigma \sqrt{2\pi}} e^{ -\frac{1}{2} \left( \frac{x-\mu}{\sigma} \right)^2 } $$
pour tout $x \in (-\infty, \infty)$, où $\sigma > 0$.
\end{definitionbox}

Cette formule, bien qu'imposante, décrit une forme très familière : la courbe en cloche.

\begin{intuitionbox}[La Courbe en Cloche]
La loi normale est sans doute la distribution la plus importante en probabilités et statistiques. Pourquoi ? Parce qu'elle modélise remarquablement bien de nombreux phénomènes naturels et processus aléatoires où les valeurs tendent à se regrouper autour d'une moyenne, avec des écarts symétriques devenant de plus en plus rares à mesure qu'on s'éloigne de cette moyenne. Pensez à la taille des individus dans une population, aux erreurs de mesure répétées, ou même aux notes d'un grand groupe d'étudiants à un examen bien conçu. 

Sa densité a une forme caractéristique de \textbf{cloche symétrique} :
\begin{itemize}
    \item \textbf{Le Centre ($\mu$)} : Le paramètre $\mu$ représente l'\textbf{espérance} (la moyenne) de la distribution. C'est le centre de symétrie de la courbe, là où la cloche atteint son \textbf{sommet}. C'est la valeur la plus probable (le mode) et aussi la valeur qui coupe la distribution en deux moitiés égales (la médiane). Changer $\mu$ \textit{translate} la cloche horizontalement sans changer sa forme.
    \item \textbf{La Dispersion ($\sigma$)} : Le paramètre $\sigma$ est l'\textbf{écart-type} ($\sigma^2$ est la variance). Il mesure la \textbf{dispersion} des valeurs autour de la moyenne $\mu$. Géométriquement, $\sigma$ contrôle la \textbf{largeur} de la cloche.
        \begin{itemize}
            \item Un \textit{petit} $\sigma$ signifie que les données sont très concentrées autour de la moyenne, donnant une cloche \textbf{étroite et pointue}.
            \item Un \textit{grand} $\sigma$ signifie que les données sont plus étalées, donnant une cloche \textbf{large et aplatie}.
        \end{itemize}
    Les points d'inflexion de la courbe (là où la courbure change de sens) se situent exactement à $\mu \pm \sigma$.
\end{itemize}

\tcblower
\centering
\begin{tikzpicture}
    \begin{axis}[
        title={La Courbe en Cloche (PDF de la Loi Normale)},
        xlabel={$x$},
        ylabel={$f(x)$},
        axis lines=middle,
        no markers,
        samples=100,
        domain=-4:4,
        height=8cm,
        width=\linewidth-1cm,
        tick label style={font=\tiny},
        legend style={at={(0.5,-0.15)}, anchor=north, font=\small},
        legend columns=2
    ]
    % N(0, 1)
    \addplot [blue, ultra thick] {1/(sqrt(2*pi))*exp(-x^2/2)};
    \addlegendentry{$\mu=0, \sigma=1$};
    % N(0, 0.25) => sigma=0.5
    \addplot [red, ultra thick] {1/(0.5*sqrt(2*pi))*exp(-x^2/(2*0.5^2))};
    \addlegendentry{$\mu=0, \sigma=0.5$ (étroite)};
    % N(1, 2.25) => sigma=1.5
    \addplot [green!70!black, ultra thick] {1/(1.5*sqrt(2*pi))*exp(-(x-1)^2/(2*1.5^2))};
    \addlegendentry{$\mu=1, \sigma=1.5$ (large, décalée)};

    \draw [dashed] (axis cs:0,0) -- (axis cs:0, {1/(sqrt(2*pi))}) node[above, font=\tiny] {pic à $\mu=0$}; % Ligne pour mu=0
    \draw [dashed] (axis cs:1,0) -- (axis cs:1, {1/(1.5*sqrt(2*pi))}) node[above right, font=\tiny] {pic à $\mu=1$}; % Ligne pour mu=1
    \end{axis}
\end{tikzpicture}
\par\small\textit{Influence de $\mu$ (position) et $\sigma$ (largeur) sur la forme de la cloche.}
\end{intuitionbox}

Mais d'où vient cette formule spécifique ? Il existe une dérivation fascinante à partir d'hypothèses fondamentales sur les erreurs aléatoires (argument d'Herschel-Maxwell).

\begin{proofbox}[Dérivation de la Densité Normale à partir des Principes Fondamentaux]

\textbf{Contexte Visuel :} Imaginons un nuage de points dispersés autour d'une cible à l'origine $(0,0)$, comme des impacts de fléchettes. Le graphique ci-dessous illustre cette dispersion. On s'intéresse à la probabilité de tomber dans une petite zone, comme $dA$, autour d'un point $(x, y)$.

\begin{center}
\begin{tikzpicture}
\begin{axis}[
    axis lines=middle, % Axes qui passent par l'origine (0,0)
    xlabel=$x$,       % Étiquette axe X
    ylabel=$y$,       % Étiquette axe Y
    axis line style={magenta}, % Couleur des axes
    xlabel style={anchor=west, magenta}, % Style de l'étiquette X
    ylabel style={anchor=south, magenta}, % Style de l'étiquette Y
    xmin=-3.5, xmax=3.5, % Limites du graphique
    ymin=-3.5, ymax=3.5,
    tick label style={font=\tiny} % Police plus petite pour les graduations
]

% Ajout des points du nuage. Ce sont des coordonnées approximatives.
\addplot [only marks, mark=*, cyan, mark size=1.5pt]
coordinates {
    (0.2, 0.1) (-0.5, 0.2) (0.1, -0.3) (0.5, -0.8) (-0.3, -1.2) (0,0.1)
    (1.5, 1.5) (0.8, 0.8) (2.0, -0.5) (2.8, -1.4) (2.5, -0.2)
    (-1.8, -1.3) (-2.5, 0.5) (-1.5, 0.3) (-2.2, -0.8) (-0.8, 0.5)
    (0.5, 1.2) (0.7, 2.8) (0.2, 3.2) (-0.5, 1.5) (-1.0, -2.0)
    (1.8, -1.0) (1.0, -1.5) (0.3, -2.5) (-1.8, -2.8) (1.2, 0.5)
    (-1.2, -1.5) (-0.8, 1.1)
};

% --- Annotations ---

% Points et boîte pour 'dA'
\addplot [only marks, mark=*, red, mark size=1.5pt] coordinates {(1.3, 2.0)};
\draw [red, thick] (axis cs:1.1, 1.8) rectangle (axis cs:1.5, 2.2);
\node [red, above] at (axis cs:1.3, 2.2) {$dA$};

% Points et boîte pour 'dB'
\addplot [only marks, mark=*, cyan, mark size=1.5pt] 
coordinates {(-1.2, 2.0) (-1.3, 2.3) (-1.0, 2.2) (-1.1, 1.9)};
\draw [blue, thick] (axis cs:-1.8, 1.2) rectangle (axis cs:-0.8, 2.8);
\node [blue, above] at (axis cs:-1.3, 2.8) {$dB$};

\end{axis}
\end{tikzpicture}
\end{center}

\textbf{Objectif :} Expliquer comment arriver à la formule mathématique de la courbe en cloche (densité de probabilité normale) en partant de principes fondamentaux sur les erreurs aléatoires.

\textbf{1. Le Point de Départ : Densité et Aire $dA$}
Dans une distribution continue, la probabilité de tomber \textit{exactement} sur un point $(x, y)$ est nulle. On ne peut donc pas parler de "probabilité d'un point". On parle de la probabilité de tomber \textit{dans une petite zone}, comme un rectangle $dA = dx \cdot dy$ autour du point $(x, y)$.
Cette probabilité, notée $P(\text{dans } dA)$, est \textit{proportionnelle} à l'aire de la zone $dA$. La \textit{constante de proportionnalité} est la \textbf{fonction de densité de probabilité} $p(x, y)$ évaluée en ce point. En d'autres termes, la densité $p(x, y)$ \textit{représente} localement la concentration de probabilité. Ainsi, la probabilité de tomber dans la zone $dA$ est approximativement :
$$ P(\text{dans } dA) \approx p(x, y) \cdot dA $$

\textbf{2. Les Hypothèses Fondamentales}
On pose deux hypothèses sur la nature de ces erreurs (représentées par la densité $p(x, y)$) :
\begin{enumerate}
    \item \textbf{Indépendance des axes :} L'erreur horizontale ($x$) est indépendante de l'erreur verticale ($y$). Cela implique que la densité jointe $p(x, y)$ peut s'écrire comme le produit de la densité marginale sur $x$, notée $f(x)$, et de la densité marginale sur $y$, notée $f(y)$. Donc, $p(x, y) = f(x) \cdot f(y)$.
    \item \textbf{Symétrie de rotation (Isotropie) :} La densité ne dépend que de la distance $r = \sqrt{x^2 + y^2}$ au centre, pas de l'angle. Il existe donc une fonction $\phi(r)$ telle que la densité en $(x,y)$ est $p(x, y) = \phi(\sqrt{x^2 + y^2})$.
\end{enumerate}

\textbf{3. L'Équation Fonctionnelle}
En égalant les deux expressions pour la même densité $p(x, y)$ (à une constante près), on obtient :
$$ f(x) \cdot f(y) = \phi(\sqrt{x^2 + y^2}) $$
Pour $y=0$, on a $f(x) \cdot f(0) = \phi(x)$. Posons $f(0) = \lambda$. Alors $\phi(x) = \lambda f(x)$.
L'équation devient :
$$ f(x) \cdot f(y) = \lambda f(\sqrt{x^2 + y^2}) $$

\textbf{4. Résolution de l'Équation Fonctionnelle}
Posons $g(x) = f(x)/\lambda$, avec $g(0)=1$. L'équation se simplifie en :
$$ g(x) g(y) = g(\sqrt{x^2 + y^2}) $$
Posons $g(x) = h(x^2)$. L'équation devient $h(x^2)h(y^2) = h(x^2+y^2)$. Avec $a=x^2$ et $b=y^2$, on a :
$$ h(a) h(b) = h(a+b) $$
La solution continue de cette équation de Cauchy est $h(a) = e^{Aa}$ pour une constante $A$.
Retour aux fonctions : $g(x) = h(x^2) = e^{Ax^2}$. $f(x) = \lambda g(x) = \lambda e^{Ax^2}$.
Comme la densité doit diminuer loin du centre, $A$ doit être négative. Posons $A = -k$ avec $k>0$.
$$ f(x) = \lambda e^{-k x^2} $$

\textbf{5. Normalisation et Identification des Paramètres}
\begin{enumerate}
    \item \textbf{Condition $\int f(x) dx = 1$} : L'intégrale Gaussienne $\int_{-\infty}^{\infty} e^{-k x^2} \, \mathrm{d}x = \sqrt{\frac{\pi}{k}}$.
    Donc, $\int_{-\infty}^{\infty} f(x) dx = \lambda \sqrt{\frac{\pi}{k}} = 1 \implies \lambda = \sqrt{\frac{k}{\pi}}$.
    \item \textbf{Lien avec la Variance ($\sigma^2$)} : Pour une distribution centrée, $\sigma^2 = E[X^2] = \int x^2 f(x) dx$.
    $$ \sigma^2 = \int_{-\infty}^{\infty} x^2 \left( \sqrt{\frac{k}{\pi}} e^{-k x^2} \right) \, \mathrm{d}x = \sqrt{\frac{k}{\pi}} \left( \frac{1}{2k} \sqrt{\frac{\pi}{k}} \right) = \frac{1}{2k} $$
    Donc, $k = \frac{1}{2\sigma^2}$.
    \item \textbf{Substitution Finale :} Remplaçons $k$ dans $\lambda$ et $f(x)$.
    $$ \lambda = \sqrt{\frac{1/(2\sigma^2)}{\pi}} = \frac{1}{\sigma\sqrt{2\pi}} $$
    $$ f(x) = \frac{1}{\sigma\sqrt{2\pi}} e^{-\frac{1}{2\sigma^2} x^2} = \frac{1}{\sigma\sqrt{2\pi}} e^{ -\frac{x^2}{2\sigma^2} } $$
    \item \textbf{Généralisation (Moyenne $\mu$)} : Pour centrer la distribution sur $\mu$, on remplace $x$ par $(x-\mu)$ dans l'exposant :
    $$ f(x; \mu, \sigma) = \frac{1}{\sigma \sqrt{2\pi}} e^{ -\frac{(x-\mu)^2}{2\sigma^2} } $$
\end{enumerate}
C'est la fonction de densité de la loi normale $\mathcal{N}(\mu, \sigma^2)$.
\end{proofbox}

\subsection{La Loi Normale Centrée Réduite $\mathcal{N}(0, 1)$}

Avant d'explorer les propriétés de la loi normale générale, concentrons-nous sur son cas le plus simple et le plus fondamental.

\begin{definitionbox}[Loi Normale Standard (ou Centrée Réduite)]
Un cas particulier extraordinairement utile est la loi normale avec une moyenne $\mu=0$ et une variance $\sigma^2=1$ (donc $\sigma=1$). On l'appelle la \textbf{loi normale standard} ou \textbf{centrée réduite}, et on la note souvent $Z$. Sa PDF est traditionnellement notée $\phi(z)$ :
$$ \phi(z) = \frac{1}{\sqrt{2\pi}} e^{-z^2/2} $$
Sa fonction de répartition (CDF), qui donne $P(Z \le z)$, est notée $\Phi(z)$ :
$$ \Phi(z) = P(Z \le z) = \int_{-\infty}^z \frac{1}{\sqrt{2\pi}} e^{-t^2/2} \, \mathrm{d}t $$
\end{definitionbox}

Pourquoi cette version standard est-elle si importante ? Elle sert de référence universelle.

\begin{intuitionbox}[La Référence Universelle et le Changement d'Unités]
Pourquoi cette loi $\mathcal{N}(0, 1)$ est-elle si centrale ? Imaginez que vous ayez des mesures en degrés Celsius ($\mathcal{N}(\mu_C, \sigma_C^2)$) et d'autres en degrés Fahrenheit ($\mathcal{N}(\mu_F, \sigma_F^2)$). Comment les comparer ? La loi normale standard fournit un \textbf{système d'unités universel}.

Toute variable normale $X \sim \mathcal{N}(\mu, \sigma^2)$ peut être transformée ("standardisée") en une variable $Z \sim \mathcal{N}(0, 1)$ par un simple changement d'échelle et de position : $Z = (X-\mu)/\sigma$. 

Cela signifie qu'au lieu de devoir calculer des aires (probabilités) pour une infinité de courbes en cloche différentes (une pour chaque paire $\mu, \sigma$), on peut tout ramener à \textbf{une seule courbe de référence}, $\mathcal{N}(0, 1)$. Les aires sous cette courbe standard ($\Phi(z)$) ont été calculées une fois pour toutes et sont disponibles dans des tables ou des logiciels. On n'a plus qu'à convertir notre problème dans cette "langue" standard, trouver la probabilité, et interpréter le résultat.
\end{intuitionbox}

La notation est très standardisée pour cette loi.

\begin{remarquebox}[Notation $\phi$ et $\Phi$]
Les symboles $\phi$ (phi minuscule) pour la PDF et $\Phi$ (phi majuscule) pour la CDF de la loi normale standard sont quasi universels. Il est important de ne pas les confondre. $\phi(z)$ est la \textit{hauteur} de la courbe en $z$, tandis que $\Phi(z)$ est l'\textit{aire} sous la courbe à gauche de $z$.
\end{remarquebox}

Un détail technique important concerne le calcul de $\Phi(z)$.

\begin{remarquebox}[Absence de Primitive Simple]
L'intégrale $\int e^{-t^2/2} \, \mathrm{d}t$, nécessaire pour calculer $\Phi(z)$, n'a \textbf{pas d'expression analytique} en termes de fonctions élémentaires (polynômes, exponentielles, log, sin, cos...). C'est une fonction spéciale, connue sous le nom de \textbf{fonction d'erreur} (liée à $\Phi$ par une transformation simple). C'est la raison pour laquelle on dépend de tables ou de calculs numériques pour obtenir les valeurs de $\Phi(z)$. Heureusement, ces outils sont omniprésents aujourd'hui.
\end{remarquebox}

\subsection{Standardisation : Le Score Z}

Formalisons cette transformation clé qui relie toute loi normale à la loi standard.

\begin{theorembox}[Standardisation d'une Variable Normale]
Si $X \sim \mathcal{N}(\mu, \sigma^2)$, alors la variable $Z$ définie par :
$$ Z = \frac{X - \mu}{\sigma} $$
suit la loi normale standard, $Z \sim \mathcal{N}(0, 1)$.
\end{theorembox}

La preuve formelle utilise un changement de variable dans la fonction de répartition.

\begin{proofbox}
Soit $F_X(x)$ la CDF de $X$ et $F_Z(z)$ la CDF de $Z$. Nous voulons montrer que $F_Z(z) = \Phi(z)$.
\begin{align*}
F_Z(z) &= P(Z \le z) \\
&= P\left( \frac{X-\mu}{\sigma} \le z \right) \\
&= P(X - \mu \le z\sigma) \\
&= P(X \le \mu + z\sigma) \\
&= F_X(\mu + z\sigma)
\end{align*}
Par définition de la CDF de $X$ :
$$ F_X(x) = \int_{-\infty}^x \frac{1}{\sigma \sqrt{2\pi}} e^{ -\frac{1}{2} \left( \frac{t-\mu}{\sigma} \right)^2 } \, dt $$
Donc,
$$ F_Z(z) = \int_{-\infty}^{\mu + z\sigma} \frac{1}{\sigma \sqrt{2\pi}} e^{ -\frac{1}{2} \left( \frac{t-\mu}{\sigma} \right)^2 } \, dt $$
Effectuons le changement de variable $u = (t-\mu)/\sigma$. Alors $t = \mu + u\sigma$ et $dt = \sigma du$.
Les bornes d'intégration changent :
\begin{itemize}
    \item Quand $t \to -\infty$, $u \to -\infty$.
    \item Quand $t = \mu + z\sigma$, $u = ((\mu + z\sigma)-\mu)/\sigma = z$.
\end{itemize}
L'intégrale devient :
$$ F_Z(z) = \int_{-\infty}^{z} \frac{1}{\sigma \sqrt{2\pi}} e^{ -\frac{1}{2} u^2 } (\sigma du) $$
$$ F_Z(z) = \int_{-\infty}^{z} \frac{1}{\sqrt{2\pi}} e^{ -u^2/2 } \, du $$
C'est exactement la définition de $\Phi(z)$, la CDF de la loi normale standard. Ainsi, $Z \sim \mathcal{N}(0, 1)$.
\end{proofbox}

Cette transformation a une interprétation très concrète.

\begin{intuitionbox}[Mesurer en "Unités d'Écart-Type"]
Transformer $X$ en $Z$ s'appelle \textbf{standardiser} la variable. Le résultat, $z = \frac{x-\mu}{\sigma}$, est appelé le \textbf{Score Z} (ou cote Z). Ce score Z est une mesure \textit{sans unité} qui indique \textbf{à combien d'écarts-types} une valeur observée $x$ se situe par rapport à la moyenne $\mu$ de sa distribution.
\begin{itemize}
    \item $z = 0$ : $x$ est exactement à la moyenne ($\mathbf{x = \mu}$).
    \item $z = +1$ : $x$ est un écart-type \textit{au-dessus} de la moyenne ($\mathbf{x = \mu + \sigma}$).
    \item $z = -2$ : $x$ est deux écarts-types \textit{en dessous} de la moyenne ($\mathbf{x = \mu - 2\sigma}$).
\end{itemize}
Cette transformation est extrêmement utile pour :
\begin{enumerate}
    \item \textbf{Comparer des valeurs} issues de distributions normales différentes. Un score Z de +1.5 a toujours la même signification relative, que l'on parle de QI, de taille, ou de température.
    \item \textbf{Calculer des probabilités} en utilisant la table unique de la loi $\mathcal{N}(0, 1)$.
\end{enumerate}
\end{intuitionbox}

Un exemple classique est la comparaison de notes.

\begin{examplebox}[Comparaison de Performances]
Un étudiant A obtient 80 points à un examen où la moyenne est $\mu_A=70$ et l'écart-type $\sigma_A=5$. Un étudiant B obtient 85 points à un autre examen où $\mu_B=75$ et $\sigma_B=10$. Qui a le mieux réussi relativement à son groupe ?

Calculons les Z-scores :
$$ Z_A = \frac{80 - 70}{5} = \frac{10}{5} = +2.0 $$
$$ Z_B = \frac{85 - 75}{10} = \frac{10}{10} = +1.0 $$
L'étudiant A a un score Z plus élevé (+2.0 contre +1.0), ce qui signifie qu'il se situe plus d'écarts-types au-dessus de la moyenne de son groupe que l'étudiant B. L'étudiant A a donc relativement mieux réussi.
\end{examplebox}

\subsection{Propriétés Importantes de la Loi Normale}

La loi normale possède des propriétés de stabilité remarquables sous certaines transformations.

\begin{theorembox}[Stabilité par Transformation Linéaire]
Si $X \sim \mathcal{N}(\mu, \sigma^2)$ et $Y = aX + b$ (avec $a \neq 0$), alors $Y$ suit aussi une loi normale :
$$ Y \sim \mathcal{N}(a\mu + b, \, (a\sigma)^2) $$
L'espérance est transformée linéairement ($E[aX+b] = aE[X]+b$), et la variance est multipliée par $a^2$ ($\text{Var}(aX+b) = a^2\text{Var}(X)$).
\end{theorembox}

\begin{proofbox}
Nous utilisons le fait que si $X \sim \mathcal{N}(\mu, \sigma^2)$, alors $Z = (X-\mu)/\sigma \sim \mathcal{N}(0,1)$.
Exprimons $X$ en fonction de $Z$ : $X = \mu + \sigma Z$.
Substituons cela dans l'expression de $Y$:
$$ Y = a(\mu + \sigma Z) + b = (a\mu + b) + (a\sigma)Z $$
Posons $\mu_Y = a\mu + b$ et $\sigma_Y = |a|\sigma$. Alors $Y = \mu_Y + \sigma_Y Z$ (si $a>0$) ou $Y = \mu_Y - \sigma_Y Z$ (si $a<0$).
Dans les deux cas, $Y$ est une transformation linéaire d'une variable normale standard $Z$.
La CDF de $Y$ peut être exprimée en termes de la CDF $\Phi$ de $Z$.
Si $a>0$ :
$$ P(Y \le y) = P(\mu_Y + a\sigma Z \le y) = P(a\sigma Z \le y - \mu_Y) = P\left( Z \le \frac{y - \mu_Y}{a\sigma} \right) = \Phi\left(\frac{y - \mu_Y}{a\sigma}\right) $$
C'est la CDF d'une loi $\mathcal{N}(\mu_Y, (a\sigma)^2)$.
Le cas $a<0$ est similaire et mène au même résultat pour la distribution (la variance dépend de $a^2$).
Ainsi, $Y \sim \mathcal{N}(a\mu + b, (a\sigma)^2)$.
\end{proofbox}

Cette propriété est très utile pour les changements d'unités.

\begin{examplebox}[Changement d'Unités]
Si la température en Celsius $T_C$ suit $\mathcal{N}(20, 5^2)$, quelle est la loi de la température en Fahrenheit $T_F = \frac{9}{5}T_C + 32$ ?

$a = 9/5$, $b=32$.

Nouvelle moyenne : $E[T_F] = \frac{9}{5}(20) + 32 = 36 + 32 = 68$.

Nouvel écart-type : $\sigma_{T_F} = |a|\sigma_{T_C} = \frac{9}{5}(5) = 9$. Nouvelle variance : $\sigma_{T_F}^2 = 9^2 = 81$.

Donc, $T_F \sim \mathcal{N}(68, 9^2)$.
\end{examplebox}

Une autre propriété cruciale concerne la somme de variables normales indépendantes.

\begin{theorembox}[Stabilité par Addition (Indépendance)]
Si $X \sim \mathcal{N}(\mu_X, \sigma_X^2)$ et $Y \sim \mathcal{N}(\mu_Y, \sigma_Y^2)$ sont des variables aléatoires \textbf{indépendantes}, alors leur somme $S = X + Y$ suit aussi une loi normale :
$$ S \sim \mathcal{N}(\mu_X + \mu_Y, \, \sigma_X^2 + \sigma_Y^2) $$
Les moyennes s'ajoutent, et (grâce à l'indépendance) les variances s'ajoutent.
\end{theorembox}

La preuve formelle de ce théorème est plus avancée et utilise généralement les fonctions caractéristiques ou les fonctions génératrices des moments.

\begin{proofbox}[Idée de la preuve (via Fonctions Caractéristiques)]
La fonction caractéristique $\varphi_X(t)$ d'une variable aléatoire $X$ est définie comme $\varphi_X(t) = E[e^{itX}]$.
Pour une loi normale $X \sim \mathcal{N}(\mu, \sigma^2)$, sa fonction caractéristique est $\varphi_X(t) = e^{i\mu t - \frac{1}{2}\sigma^2 t^2}$.
Si $X$ et $Y$ sont indépendantes, la fonction caractéristique de leur somme $S=X+Y$ est le produit de leurs fonctions caractéristiques : $\varphi_S(t) = \varphi_X(t) \varphi_Y(t)$.
\begin{align*}
\varphi_S(t) &= \left( e^{i\mu_X t - \frac{1}{2}\sigma_X^2 t^2} \right) \left( e^{i\mu_Y t - \frac{1}{2}\sigma_Y^2 t^2} \right) \\
&= e^{i(\mu_X + \mu_Y)t - \frac{1}{2}(\sigma_X^2 + \sigma_Y^2)t^2}
\end{align*}
On reconnaît ici la fonction caractéristique d'une loi normale avec pour moyenne $\mu_X + \mu_Y$ et pour variance $\sigma_X^2 + \sigma_Y^2$. Comme la fonction caractéristique détermine de manière unique la distribution, on conclut que $S \sim \mathcal{N}(\mu_X + \mu_Y, \sigma_X^2 + \sigma_Y^2)$.
\end{proofbox}

Il est essentiel de se souvenir de la condition d'indépendance pour l'addition des variances.

\begin{remarquebox}[Attention à l'Indépendance]
La propriété d'addition des variances ($\sigma_S^2 = \sigma_X^2 + \sigma_Y^2$) est cruciale et ne tient \textbf{que si $X$ et $Y$ sont indépendantes}. Si elles ne le sont pas, la variance de la somme inclut un terme de covariance : $\text{Var}(X+Y) = \text{Var}(X) + \text{Var}(Y) + 2\text{Cov}(X, Y)$. Cependant, la somme de variables normales (même dépendantes) reste normale (si elles sont conjointement normales).
\end{remarquebox}

Appliquons ce théorème à un exemple concret.

\begin{examplebox}[Poids Total]
Le poids d'une pomme suit $\mathcal{N}(150g, 10^2)$. Le poids d'une orange suit $\mathcal{N}(200g, 15^2)$. On suppose les poids indépendants. Quel est la loi du poids total d'une pomme et d'une orange ?

Soit $P$ le poids de la pomme, $O$ celui de l'orange. $T = P+O$.

$E[T] = E[P] + E[O] = 150 + 200 = 350g$.

$\text{Var}(T) = \text{Var}(P) + \text{Var}(O) = 10^2 + 15^2 = 100 + 225 = 325$.

Donc, $T \sim \mathcal{N}(350, 325)$. L'écart-type du poids total est $\sqrt{325} \approx 18.03g$.
\end{examplebox}

\subsection{La Règle Empirique (68-95-99.7)}

Une conséquence directe des aires sous la courbe normale standard est une règle approximative très utile.

\begin{theorembox}[Règle Empirique]
Pour toute variable $X \sim \mathcal{N}(\mu, \sigma^2)$ :
\begin{itemize}
    \item $P(\mu - \sigma \le X \le \mu + \sigma) \approx 0.6827$ (Environ \textbf{68\%} des valeurs dans $\mu \pm \sigma$).
    \item $P(\mu - 2\sigma \le X \le \mu + 2\sigma) \approx 0.9545$ (Environ \textbf{95\%} des valeurs dans $\mu \pm 2\sigma$).
    \item $P(\mu - 3\sigma \le X \le \mu + 3\sigma) \approx 0.9973$ (Environ \textbf{99.7\%} des valeurs dans $\mu \pm 3\sigma$).
\end{itemize}
\end{theorembox}

\begin{proofbox}[Dérivation à partir de $\Phi(z)$]
Ces valeurs sont obtenues en calculant les aires sous la PDF de la loi normale standard $\mathcal{N}(0, 1)$ entre les Z-scores correspondants.
\begin{itemize}
    \item $P(-1 \le Z \le 1) = \Phi(1) - \Phi(-1) = \Phi(1) - (1 - \Phi(1)) = 2\Phi(1) - 1$.
    Avec $\Phi(1) \approx 0.8413$, on obtient $2(0.8413) - 1 \approx 0.6826$.
    \item $P(-2 \le Z \le 2) = \Phi(2) - \Phi(-2) = 2\Phi(2) - 1$.
    Avec $\Phi(2) \approx 0.9772$, on obtient $2(0.9772) - 1 \approx 0.9544$.
    \item $P(-3 \le Z \le 3) = \Phi(3) - \Phi(-3) = 2\Phi(3) - 1$.
    Avec $\Phi(3) \approx 0.99865$, on obtient $2(0.99865) - 1 \approx 0.9973$.
\end{itemize}
Ces valeurs sont souvent arrondies à 68%, 95%, et 99.7% pour faciliter la mémorisation.
\end{proofbox}

Cette règle fournit des repères très pratiques.

\begin{intuitionbox}[Repères Essentiels sur la Cloche]
Cette règle, dérivée directement des aires sous la courbe $\mathcal{N}(0, 1)$ entre $z=\pm 1$, $z=\pm 2$ et $z=\pm 3$, fournit des repères extrêmement utiles pour interpréter l'écart-type $\sigma$. Elle nous dit où se trouve la grande majorité des données.


Une observation qui tombe en dehors de l'intervalle $\mu \pm 3\sigma$ est très inhabituelle (elle n'a que 0.3% de chances de se produire). C'est souvent considéré comme une \textit{valeur aberrante} (outlier) potentielle.
\end{intuitionbox}

\subsection{Calcul de Probabilités Normales}

En pratique, pour calculer une probabilité $P(a \le X \le b)$ pour une loi $\mathcal{N}(\mu, \sigma^2)$, on utilise systématiquement la standardisation.

\begin{examplebox}[Utilisation du Z-score]
Supposons que le QI d'une population suit $\mathcal{N}(100, 15^2)$. Quelle est la probabilité $P(X > 130)$ ?

1.  \textbf{Standardiser :} $z = \frac{130 - 100}{15} = 2$. On cherche $P(Z > 2)$.
2.  \textbf{Utiliser la CDF Standard :} $P(Z > 2) = 1 - P(Z \le 2) = 1 - \Phi(2)$.
3.  \textbf{Chercher dans la table / Calculer :} $\Phi(2) \approx 0.9772$.
4.  \textbf{Résultat :} $P(X > 130) = 1 - 0.9772 = 0.0228$. Environ 2.3% de la population a un QI supérieur à 130.
\end{examplebox}

Pour les intervalles, on utilise la propriété $P(a \le Z \le b) = \Phi(b) - \Phi(a)$.

\begin{examplebox}[Probabilité entre deux valeurs]
Quelle est la probabilité $P(85 \le X \le 115)$ ? ($\mu=100, \sigma=15$)

1.  \textbf{Standardiser :} $z_1 = \frac{85 - 100}{15} = -1$, $z_2 = \frac{115 - 100}{15} = 1$. On cherche $P(-1 \le Z \le 1)$.
2.  \textbf{Utiliser la CDF Standard :} $P(-1 \le Z \le 1) = \Phi(1) - \Phi(-1)$.
3.  \textbf{Utiliser la symétrie :} $\Phi(-z) = 1 - \Phi(z)$. Donc $\Phi(-1) = 1 - \Phi(1)$.
    $P(-1 \le Z \le 1) = \Phi(1) - (1 - \Phi(1)) = 2\Phi(1) - 1$.
4.  \textbf{Chercher dans la table / Calculer :} $\Phi(1) \approx 0.8413$.
5.  \textbf{Résultat :} $P(85 \le X \le 115) \approx 2(0.8413) - 1 = 1.6826 - 1 = 0.6826$. (On retrouve la règle des 68% !)
\end{examplebox}

On peut aussi inverser le processus : trouver la valeur $x$ correspondant à une probabilité donnée.

\begin{examplebox}[Trouver une valeur pour une probabilité donnée (Problème Inverse)]
Quel est le QI minimum requis pour être dans le top 10\% de la population ? ($\mu=100, \sigma=15$).

1.  \textbf{Trouver le Z-score correspondant :} On cherche $x$ tel que $P(X > x) = 0.10$. Cela équivaut à $P(Z > z) = 0.10$, où $z = (x-100)/15$.
    Si $P(Z > z) = 0.10$, alors $P(Z \le z) = \Phi(z) = 1 - 0.10 = 0.90$.
2.  \textbf{Chercher dans la table inverse / Calculer :} On cherche la valeur $z$ pour laquelle l'aire à gauche est 0.90 (le 90ème percentile). On trouve $z \approx 1.28$.
3.  \textbf{Convertir en X :} On utilise la relation $z = (x-\mu)/\sigma$ pour trouver $x$:
    $1.28 = \frac{x - 100}{15}$
    $x = 100 + 1.28 \times 15 = 100 + 19.2 = 119.2$.
    Il faut un QI d'environ 119.2 pour être dans le top 10\%.
\end{examplebox}

\subsection{Exercices}

% --- PDF, CDF et Loi Normale Standard ---

\begin{exercicebox}[Exercice 1 : Concepts de Base $\Phi(z)$]
Soit $Z \sim \mathcal{N}(0, 1)$ la loi normale standard. Sa CDF est $\Phi(z)$.
Exprimez les probabilités suivantes en termes de $\Phi(z)$ :
\begin{enumerate}
    \item $P(Z \le 1.5)$
    \item $P(Z > 1)$
    \item $P(Z \le -1.5)$ (Indice : utilisez la symétrie $\Phi(-z) = 1 - \Phi(z)$)
    \item $P(-1.5 \le Z \le 1.5)$
\end{enumerate}
\end{exercicebox}

\begin{exercicebox}[Exercice 2 : Utilisation d'une Table $\Phi(z)$]
En utilisant une table ou une calculatrice pour la loi $\mathcal{N}(0, 1)$, on sait que $\Phi(1) \approx 0.8413$, $\Phi(1.96) \approx 0.975$ et $\Phi(2) \approx 0.9772$.
Calculez :
\begin{enumerate}
    \item $P(Z > 1)$
    \item $P(Z \le -2)$
    \item $P(-1.96 \le Z \le 1.96)$
\end{enumerate}
\end{exercicebox}

\begin{exercicebox}[Exercice 3 : Propriétés de la PDF $\phi(z)$]
Soit $\phi(z)$ la PDF de la loi $\mathcal{N}(0, 1)$.
\begin{enumerate}
    \item Quelle est la valeur de $\phi(0)$ ? (Le pic de la courbe).
    \item Que vaut $\phi(z)$ par rapport à $\phi(-z)$ ?
    \item Que vaut $\int_{-\infty}^{\infty} \phi(z) \, dz$ ?
\end{enumerate}
\end{exercicebox}

% --- Standardisation (Z-score) et Calcul de Probabilités ---

\begin{exercicebox}[Exercice 4 : Calcul de Z-scores]
Une variable aléatoire $X$ suit une loi normale $\mathcal{N}(\mu=50, \sigma^2=100)$. Notez que $\sigma=10$.
Calculez le Z-score pour les valeurs suivantes de $X$ :
\begin{enumerate}
    \item $x = 60$
    \item $x = 50$
    \item $x = 35$
\end{enumerate}
\end{exercicebox}

\begin{exercicebox}[Exercice 5 : Calcul de Probabilité (Général)]
La taille des hommes adultes dans un pays suit une loi normale $\mathcal{N}(175 \text{ cm}, 7^2 \text{ cm}^2)$.
Soit $X$ la taille d'un homme choisi au hasard. Calculez :
\begin{enumerate}
    \item $P(X \le 182 \text{ cm})$ (Indice : Standardisez $x=182$ et utilisez $\Phi(1) \approx 0.8413$)
    \item $P(X > 168 \text{ cm})$
\end{enumerate}
\end{exercicebox}

\begin{exercicebox}[Exercice 6 : Calcul de Probabilité (Intervalle)]
Les scores à un test de QI suivent une loi normale $\mathcal{N}(100, 15^2)$.
Quelle est la probabilité qu'une personne choisie au hasard ait un QI compris entre 85 et 115 ?
(Indice : Standardisez les deux bornes).
\end{exercicebox}

\begin{exercicebox}[Exercice 7 : Calcul de Probabilité (Queue Extrême)]
En utilisant la même loi $\mathcal{N}(100, 15^2)$ pour le QI :
Quelle est la probabilité qu'une personne ait un QI supérieur à 130 ?
(Indice : Utilisez $\Phi(2) \approx 0.9772$).
\end{exercicebox}

% --- Problèmes Inverses (Trouver x) ---

\begin{exercicebox}[Exercice 8 : Problème Inverse (Z-score)]
Soit $Z \sim \mathcal{N}(0, 1)$. Trouvez la valeur $z$ telle que :
(Utilisez $\Phi(1.28) \approx 0.90$ et $\Phi(1.645) \approx 0.95$)
\begin{enumerate}
    \item $P(Z \le z) = 0.90$
    \item $P(Z > z) = 0.05$ (Indice : si $P(Z>z)=0.05$, que vaut $P(Z \le z)$ ?)
    \item $P(Z \le z) = 0.10$ (Indice : utilisez la symétrie)
\end{enumerate}
\end{exercicebox}

\begin{exercicebox}[Exercice 9 : Problème Inverse (Général)]
Les scores au test $\mathcal{N}(100, 15^2)$ sont utilisés pour sélectionner des candidats. Seul le top 5\% des scores est accepté.
Quel est le score minimum requis pour être accepté ?
(Indice : Utilisez $z \approx 1.645$ pour le top 5\%).
\end{exercicebox}

\begin{exercicebox}[Exercice 10 : Problème Inverse (Intervalle Central)]
Soit $Z \sim \mathcal{N}(0, 1)$. Trouvez la valeur $z$ telle que $P(-z \le Z \le z) = 0.95$.
(Indice : si 95\% est au centre, combien reste-t-il dans chaque queue ? Utilisez $\Phi(1.96) \approx 0.975$).
\end{exercicebox}

\begin{exercicebox}[Exercice 11 : Problème Inverse (Général)]
La durée de vie d'une batterie suit $\mathcal{N}(500 \text{ heures}, 50^2 \text{ heures}^2)$.
Le fabricant veut offrir une garantie. Il ne veut remplacer que 2.5\% des batteries.
Quelle durée de garantie (en heures) doit-il proposer ?
(Indice : $P(Z \le -1.96) \approx 0.025$).
\end{exercicebox}

% --- Règle Empirique (68-95-99.7) ---

\begin{exercicebox}[Exercice 12 : Règle Empirique (Application)]
Le poids de paquets de café suit $\mathcal{N}(250g, 5^2g^2)$.
En utilisant la règle empirique (68-95-99.7), donnez un intervalle qui contient :
\begin{enumerate}
    \item Environ 68\% des poids.
    \item Environ 95\% des poids.
    \item Environ 99.7\% des poids.
\end{enumerate}
\end{exercicebox}

\begin{exercicebox}[Exercice 13 : Règle Empirique (Probabilité)]
En utilisant la situation de l'exercice 12 ($\mathcal{N}(250, 5^2)$) et la règle empirique :
\begin{enumerate}
    \item Estimez $P(245 \le X \le 255)$.
    \item Estimez $P(X \le 240)$. (Indice : L'intervalle $\mu \pm 2\sigma$ est [240, 260] et contient 95\%. Utilisez la symétrie).
\end{enumerate}
\end{exercicebox}

% --- Propriétés (Transformations Linéaires et Sommes) ---

\begin{exercicebox}[Exercice 14 : Transformation Linéaire (Celsius -> Fahrenheit)]
La température $T_C$ à midi en été dans une ville suit $\mathcal{N}(25, 3^2)$ (en degrés Celsius).
On convertit la température en Fahrenheit : $T_F = 1.8 \times T_C + 32$.
Quelle est la loi de $T_F$ ? (Donnez sa moyenne et sa variance).
\end{exercicebox}

\begin{exercicebox}[Exercice 15 : Transformation Linéaire (Z-score)]
Soit $X \sim \mathcal{N}(\mu, \sigma^2)$. Soit $Y = aX+b$.
Trouvez $a$ et $b$ (en fonction de $\mu$ et $\sigma$) tels que $Y \sim \mathcal{N}(0, 1)$.
\end{exercicebox}

\begin{exercicebox}[Exercice 16 : Somme de Normales Indépendantes]
Soit $X \sim \mathcal{N}(10, 3^2)$ et $Y \sim \mathcal{N}(20, 4^2)$. $X$ et $Y$ sont indépendantes.
Soit $S = X + Y$.
\begin{enumerate}
    \item Quelle est la loi de $S$ ? (Donnez sa moyenne et sa variance).
    \item Quel est l'écart-type de $S$ ?
\end{enumerate}
\end{exercicebox}

\begin{exercicebox}[Exercice 17 : Différence de Normales Indépendantes]
En utilisant $X$ et $Y$ de l'exercice 16, soit $D = Y - X$.
\begin{enumerate}
    \item Quelle est la loi de $D$ ? (Donnez sa moyenne et sa variance).
    \item Quel est l'écart-type de $D$ ? (Comparez-le à celui de $S$).
\end{enumerate}
\end{exercicebox}

\begin{exercicebox}[Exercice 18 : Application (Somme)]
Le poids d'une boîte vide $B$ suit $\mathcal{N}(100g, 5^2)$. Le poids du contenu $C$ suit $\mathcal{N}(800g, 10^2)$. $B$ et $C$ sont indépendants.
Soit $T = B+C$ le poids total.
\begin{enumerate}
    \item Quelle est la loi de $T$ ?
    \item Calculez $P(T > 925g)$. (Utilisez $\Phi(2) \approx 0.9772$).
\end{enumerate}
\end{exercicebox}

\begin{exercicebox}[Exercice 19 : Moyenne d'un Échantillon (Avancé)]
Soient $X_1, X_2, X_3, X_4$ quatre observations indépendantes de la loi $\mathcal{N}(10, 4^2)$.
Soit $\bar{X} = \frac{X_1 + X_2 + X_3 + X_4}{4}$ la moyenne de l'échantillon.
\begin{enumerate}
    \item Soit $S = X_1+X_2+X_3+X_4$. Quelle est la loi de $S$ ?
    \item En utilisant la transformation linéaire $\bar{X} = \frac{1}{4}S$, quelle est la loi de $\bar{X}$ ?
\end{enumerate}
\end{exercicebox}

\begin{exercicebox}[Exercice 20 : Comparaison (Différence)]
Alice et Bob passent un examen. Les notes d'Alice $A$ suivent $\mathcal{N}(80, 5^2)$. Les notes de Bob $B$ suivent $\mathcal{N}(78, 3^2)$. On suppose leurs notes indépendantes.
Quelle est la probabilité que Bob ait une meilleure note qu'Alice ?
(Indice : Calculez $P(B > A)$, ce qui est équivalent à $P(B - A > 0)$).
\end{exercicebox}

\subsection{Corrections des Exercices}

% --- Corrections : PDF, CDF et Loi Normale Standard ---

\begin{correctionbox}[Correction Exercice 1 : Concepts de Base $\Phi(z)$]
1.  $P(Z \le 1.5) = \Phi(1.5)$.
2.  $P(Z > 1) = 1 - P(Z \le 1) = 1 - \Phi(1)$.
3.  $P(Z \le -1.5) = 1 - P(Z \le 1.5) = 1 - \Phi(1.5)$.
4.  $P(-1.5 \le Z \le 1.5) = P(Z \le 1.5) - P(Z \le -1.5) = \Phi(1.5) - (1 - \Phi(1.5)) = 2\Phi(1.5) - 1$.
\end{correctionbox}

\begin{correctionbox}[Correction Exercice 2 : Utilisation d'une Table $\Phi(z)$]
Données : $\Phi(1) \approx 0.8413$, $\Phi(1.96) \approx 0.975$, $\Phi(2) \approx 0.9772$.
1.  $P(Z > 1) = 1 - \Phi(1) \approx 1 - 0.8413 = 0.1587$.
2.  $P(Z \le -2) = 1 - \Phi(2) \approx 1 - 0.9772 = 0.0228$.
3.  $P(-1.96 \le Z \le 1.96) = \Phi(1.96) - \Phi(-1.96) = \Phi(1.96) - (1 - \Phi(1.96))$
    $= 2\Phi(1.96) - 1 \approx 2(0.975) - 1 = 1.95 - 1 = 0.95$.
    (C'est l'intervalle de confiance à 95\%).
\end{correctionbox}

\begin{correctionbox}[Correction Exercice 3 : Propriétés de la PDF $\phi(z)$]
$\phi(z) = \frac{1}{\sqrt{2\pi}} e^{-z^2/2}$.
1.  $\phi(0) = \frac{1}{\sqrt{2\pi}} e^{0} = \frac{1}{\sqrt{2\pi}} \approx 0.3989$.
2.  Puisque $z^2 = (-z)^2$, on a $\phi(z) = \phi(-z)$. La fonction est paire (symétrique par rapport à l'axe y).
3.  Par définition d'une PDF, l'aire totale sous la courbe doit être 1. $\int_{-\infty}^{\infty} \phi(z) \, dz = 1$.
\end{correctionbox}

% --- Corrections : Standardisation (Z-score) et Calcul de Probabilités ---

\begin{correctionbox}[Correction Exercice 4 : Calcul de Z-scores]
$X \sim \mathcal{N}(\mu=50, \sigma^2=100) \implies \sigma=10$.
$Z = \frac{X - \mu}{\sigma}$.
1.  $x = 60 \implies z = (60 - 50) / 10 = 10 / 10 = 1$.
2.  $x = 50 \implies z = (50 - 50) / 10 = 0 / 10 = 0$.
3.  $x = 35 \implies z = (35 - 50) / 10 = -15 / 10 = -1.5$.
\end{correctionbox}

\begin{correctionbox}[Correction Exercice 5 : Calcul de Probabilité (Général)]
$X \sim \mathcal{N}(175, 7^2)$. $\mu=175, \sigma=7$.
1.  $P(X \le 182) = P\left(Z \le \frac{182 - 175}{7}\right) = P(Z \le \frac{7}{7}) = P(Z \le 1) = \Phi(1) \approx 0.8413$.
2.  $P(X > 168) = P\left(Z > \frac{168 - 175}{7}\right) = P(Z > \frac{-7}{7}) = P(Z > -1)$.
    Par symétrie, $P(Z > -1) = P(Z < 1) = \Phi(1) \approx 0.8413$.
\end{correctionbox}

\begin{correctionbox}[Correction Exercice 6 : Calcul de Probabilité (Intervalle)]
$X \sim \mathcal{N}(100, 15^2)$. $\mu=100, \sigma=15$.
On cherche $P(85 \le X \le 115)$.
$z_1 = (85 - 100) / 15 = -15 / 15 = -1$.
$z_2 = (115 - 100) / 15 = 15 / 15 = 1$.
$P(-1 \le Z \le 1) = \Phi(1) - \Phi(-1) = \Phi(1) - (1 - \Phi(1)) = 2\Phi(1) - 1$.
En utilisant $\Phi(1) \approx 0.8413$, $P \approx 2(0.8413) - 1 = 1.6826 - 1 = 0.6826$.
(On retrouve la règle des 68\%).
\end{correctionbox}

\begin{correctionbox}[Correction Exercice 7 : Calcul de Probabilité (Queue Extrême)]
$X \sim \mathcal{N}(100, 15^2)$.
On cherche $P(X > 130)$.
$z = (130 - 100) / 15 = 30 / 15 = 2$.
$P(X > 130) = P(Z > 2) = 1 - P(Z \le 2) = 1 - \Phi(2)$.
En utilisant $\Phi(2) \approx 0.9772$, $P \approx 1 - 0.9772 = 0.0228$.
\end{correctionbox}

% --- Corrections : Problèmes Inverses (Trouver x) ---

\begin{correctionbox}[Correction Exercice 8 : Problème Inverse (Z-score)]
1.  $P(Z \le z) = 0.90 \implies z = \Phi^{-1}(0.90) \approx 1.28$.
2.  $P(Z > z) = 0.05 \implies P(Z \le z) = 1 - 0.05 = 0.95$.
    $z = \Phi^{-1}(0.95) \approx 1.645$.
3.  $P(Z \le z) = 0.10$. C'est dans la queue gauche. Par symétrie, $z = - \Phi^{-1}(1 - 0.10) = - \Phi^{-1}(0.90)$.
    $z \approx -1.28$.
\end{correctionbox}

\begin{correctionbox}[Correction Exercice 9 : Problème Inverse (Général)]
$X \sim \mathcal{N}(100, 15^2)$. On cherche $x$ tel que $P(X > x) = 0.05$.
1.  Trouver le Z-score : $P(Z > z) = 0.05 \implies P(Z \le z) = 0.95 \implies z \approx 1.645$.
2.  Convertir en $x$ : $z = (x-\mu)/\sigma \implies x = \mu + z\sigma$.
    $x = 100 + (1.645)(15) = 100 + 24.675 = 124.675$.
    Le score minimum est d'environ 125.
\end{correctionbox}

\begin{correctionbox}[Correction Exercice 10 : Problème Inverse (Intervalle Central)]
$P(-z \le Z \le z) = 0.95$.
Si 95\% est au centre, il reste $1 - 0.95 = 0.05$ (ou 5\%) dans les deux queues.
Par symétrie, chaque queue a $0.05 / 2 = 0.025$.
La probabilité à gauche de $z$ est $P(Z \le z) = 0.95 + 0.025 = 0.975$.
On cherche $z = \Phi^{-1}(0.975)$.
En utilisant l'indice, $z \approx 1.96$.
\end{correctionbox}

\begin{correctionbox}[Correction Exercice 11 : Problème Inverse (Général)]
$X \sim \mathcal{N}(500, 50^2)$. On cherche $x$ tel que $P(X \le x) = 0.025$.
1.  Trouver le Z-score : $P(Z \le z) = 0.025$. C'est la queue gauche.
    En utilisant l'indice $P(Z \le -1.96) \approx 0.025$, on a $z \approx -1.96$.
2.  Convertir en $x$ : $x = \mu + z\sigma$.
    $x = 500 + (-1.96)(50) = 500 - 98 = 402$.
    Le fabricant doit proposer une garantie de 402 heures.
\end{correctionbox}

% --- Corrections : Règle Empirique (68-95-99.7) ---

\begin{correctionbox}[Correction Exercice 12 : Règle Empirique (Application)]
$X \sim \mathcal{N}(\mu=250, \sigma=5)$.
1.  68\% $\implies \mu \pm 1\sigma = 250 \pm 5 \implies [245, 255]$.
2.  95\% $\implies \mu \pm 2\sigma = 250 \pm 2(5) = 250 \pm 10 \implies [240, 260]$.
3.  99.7\% $\implies \mu \pm 3\sigma = 250 \pm 3(5) = 250 \pm 15 \implies [235, 265]$.
\end{correctionbox}

\begin{correctionbox}[Correction Exercice 13 : Règle Empirique (Probabilité)]
1.  $P(245 \le X \le 255)$ est l'intervalle $\mu \pm 1\sigma$.
    La probabilité est d'environ 68\% ou 0.68.
2.  L'intervalle $\mu \pm 2\sigma$ est $[240, 260]$ et contient 95\% des données.
    Il reste $100\% - 95\% = 5\%$ dans les deux queues (i.e., $P(X < 240) + P(X > 260) = 0.05$).
    Par symétrie, la queue gauche $P(X < 240)$ est $0.05 / 2 = 0.025$.
    La probabilité est d'environ 2.5\% ou 0.025.
\end{correctionbox}

% --- Corrections : Propriétés (Transformations Linéaires et Sommes) ---

\begin{correctionbox}[Correction Exercice 14 : Transformation Linéaire]
$T_C \sim \mathcal{N}(25, 3^2)$. $T_F = a T_C + b$ avec $a=1.8$ et $b=32$.
Loi de $T_F$ : $T_F \sim \mathcal{N}(a\mu + b, (a\sigma)^2)$.
Moyenne : $E[T_F] = 1.8(25) + 32 = 45 + 32 = 77$.
Variance : $\text{Var}(T_F) = (1.8)^2 \text{Var}(T_C) = (1.8)^2 (3^2) = (1.8 \times 3)^2 = (5.4)^2 = 29.16$.
Donc, $T_F \sim \mathcal{N}(77, 29.16)$.
\end{correctionbox}

\begin{correctionbox}[Correction Exercice 15 : Transformation Linéaire (Z-score)]
On veut $Y = aX+b \sim \mathcal{N}(0, 1)$.
$E[Y] = aE[X] + b = a\mu + b$. On veut $a\mu + b = 0$.
$\text{Var}(Y) = a^2 \text{Var}(X) = a^2 \sigma^2$. On veut $a^2 \sigma^2 = 1$.
De $\text{Var}(Y)=1 \implies a^2 = 1/\sigma^2 \implies a = 1/\sigma$ (en supposant $a>0$).
De $E[Y]=0 \implies (1/\sigma)\mu + b = 0 \implies b = -\mu/\sigma$.
Les constantes sont $a = 1/\sigma$ et $b = -\mu/\sigma$. (C'est la définition de la standardisation).
\end{correctionbox}

\begin{correctionbox}[Correction Exercice 16 : Somme de Normales Indépendantes]
$X \sim \mathcal{N}(10, 9)$ et $Y \sim \mathcal{N}(20, 16)$. $S = X+Y$.
1.  La somme de normales indépendantes est une normale.
    $E[S] = E[X] + E[Y] = 10 + 20 = 30$.
    $\text{Var}(S) = \text{Var}(X) + \text{Var}(Y) = 9 + 16 = 25$.
    Donc, $S \sim \mathcal{N}(30, 25)$.
2.  $\text{Var}(S) = 25 \implies \sigma_S = \sqrt{25} = 5$.
\end{correctionbox}

\begin{correctionbox}[Correction Exercice 17 : Différence de Normales Indépendantes]
$D = Y - X$.
1.  La différence est aussi une normale.
    $E[D] = E[Y] - E[X] = 20 - 10 = 10$.
    $\text{Var}(D) = \text{Var}(Y + (-1)X) = \text{Var}(Y) + (-1)^2 \text{Var}(X) = \text{Var}(Y) + \text{Var}(X)$.
    $\text{Var}(D) = 16 + 9 = 25$.
    Donc, $D \sim \mathcal{N}(10, 25)$.
2.  $\sigma_D = \sqrt{25} = 5$. (Identique à $\sigma_S$. La variance s'additionne toujours).
\end{correctionbox}

\begin{correctionbox}[Correction Exercice 18 : Application (Somme)]
$B \sim \mathcal{N}(100, 25)$, $C \sim \mathcal{N}(800, 100)$. $T = B+C$.
1.  $E[T] = E[B] + E[C] = 100 + 800 = 900$.
    $\text{Var}(T) = \text{Var}(B) + \text{Var}(C) = 25 + 100 = 125$.
    $T \sim \mathcal{N}(900, 125)$.
2.  $P(T > 925)$. $\sigma_T = \sqrt{125} = \sqrt{25 \times 5} = 5\sqrt{5} \approx 11.18$.
    $z = (925 - 900) / \sqrt{125} = 25 / (5\sqrt{5}) = 5/\sqrt{5} = \sqrt{5} \approx 2.236$.
    $P(T > 925) = P(Z > 2.236) = 1 - \Phi(2.236) \approx 1 - 0.9873 = 0.0127$.
    (Note : L'indice $\Phi(2) \approx 0.9772$ semble être une approximation pour un $z$ de 2, qui n'est pas le bon $z$ ici).
\end{correctionbox}

\begin{correctionbox}[Correction Exercice 19 : Moyenne d'un Échantillon (Avancé)]
$X_i \sim \mathcal{N}(10, 16)$ (indép.). $\bar{X} = \frac{1}{4} S$ où $S = X_1+X_2+X_3+X_4$.
1.  $S$ est une somme de normales indépendantes.
    $E[S] = E[X_1] + \dots + E[X_4] = 4 \times 10 = 40$.
    $\text{Var}(S) = \text{Var}(X_1) + \dots + \text{Var}(X_4) = 4 \times 16 = 64$.
    $S \sim \mathcal{N}(40, 64)$.
2.  $\bar{X}$ est une transformation linéaire de $S$.
    $E[\bar{X}] = E[\frac{1}{4}S] = \frac{1}{4}E[S] = \frac{1}{4}(40) = 10$.
    $\text{Var}(\bar{X}) = \text{Var}(\frac{1}{4}S) = (\frac{1}{4})^2 \text{Var}(S) = \frac{1}{16}(64) = 4$.
    $\bar{X} \sim \mathcal{N}(10, 4)$.
\end{correctionbox}

\begin{correctionbox}[Correction Exercice 20 : Comparaison (Différence)]
$A \sim \mathcal{N}(80, 25)$, $B \sim \mathcal{N}(78, 9)$. Indép.
On cherche $P(B > A)$, ce qui est $P(B - A > 0)$.
Soit $D = B - A$. $D$ suit une loi normale.
$E[D] = E[B] - E[A] = 78 - 80 = -2$.
$\text{Var}(D) = \text{Var}(B) + \text{Var}(A) = 9 + 25 = 34$.
Donc $D \sim \mathcal{N}(-2, 34)$. $\sigma_D = \sqrt{34} \approx 5.83$.
On cherche $P(D > 0)$.
$z = (0 - (-2)) / \sqrt{34} = 2 / \sqrt{34} \approx 0.342$.
$P(D > 0) = P(Z > 0.342) = 1 - \Phi(0.342) \approx 1 - 0.6338 = 0.3662$.
Il y a environ 36.6\% de chance que Bob ait une meilleure note.
\end{correctionbox}
\subsection{Exercices}

\textit{Pour tous les exercices de calcul, vous pouvez utiliser les valeurs suivantes pour la fonction de répartition de la loi normale standard $\Phi(z) = P(Z \le z)$ :}
\begin{itemize}
    \item $\Phi(0) = 0.5$
    \item $\Phi(0.5) \approx 0.6915$
    \item $\Phi(1) \approx 0.8413$
    \item $\Phi(1.5) \approx 0.9332$
    \item $\Phi(1.96) \approx 0.975$
    \item $\Phi(2) \approx 0.9772$
    \item $\Phi(2.5) \approx 0.9938$
    \item $\Phi(3) \approx 0.9987$
\end{itemize}
\textit{Et rappelez-vous la propriété de symétrie : $\Phi(-z) = 1 - \Phi(z)$.}

% --- Section 1 : Définitions et Concepts ---

\begin{exercicebox}[Exercice 1 : Définition (Paramètres)]
Soit $Y \sim \text{Log-}\mathcal{N}(3, 16)$.
\begin{enumerate}
    \item Soit $X = \ln(Y)$. Quelle est la loi de $X$ ?
    \item Que valent $E[X]$ et $\text{Var}(X)$ ?
\end{enumerate}
\end{exercicebox}

\begin{exercicebox}[Exercice 2 : Définition (Inverse)]
Soit $X \sim \mathcal{N}(0, 1)$. Si $Y = e^X$, quelle est la notation de la loi de $Y$ ?
\end{exercicebox}

\begin{exercicebox}[Exercice 3 : Intuition (Somme vs Produit)]
La taille d'une population de bactéries au temps $t$ est $P(t) = P(0) \times F_1 \times F_2 \times \dots \times F_t$, où les $F_i$ sont des facteurs de croissance aléatoires.
Pourquoi est-il plus probable que $P(t)$ suive une loi log-normale plutôt qu'une loi normale ?
\end{exercicebox}

\begin{exercicebox}[Exercice 4 : Intuition (Transformation)]
Si $X \sim \mathcal{N}(0, \sigma^2)$, on sait que la distribution de $X$ est symétrique autour de 0.
Pourquoi la distribution de $Y = e^X$ n'est-elle pas symétrique ?
\end{exercicebox}

\begin{exercicebox}[Exercice 5 : PDF (Propriétés)]
Soit $Y \sim \text{Log-}\mathcal{N}(\mu, \sigma^2)$. Quelle est la probabilité $P(Y \le 0)$ ?
\end{exercicebox}

% --- Section 2 : Moments et Mesures Centrales ---

\begin{exercicebox}[Exercice 6 : Espérance]
Soit $Y \sim \text{Log-}\mathcal{N}(\mu=4, \sigma^2=2)$.
Calculez l'espérance $E[Y]$.
\end{exercicebox}

\begin{exercicebox}[Exercice 7 : Médiane]
Soit $Y \sim \text{Log-}\mathcal{N}(\mu=4, \sigma^2=2)$.
Calculez la médiane $\text{Med}(Y)$.
\end{exercicebox}

\begin{exercicebox}[Exercice 8 : Mode]
Soit $Y \sim \text{Log-}\mathcal{N}(\mu=4, \sigma^2=2)$.
Calculez le mode $\text{Mode}(Y)$.
\end{exercicebox}

\begin{exercicebox}[Exercice 9 : Relation d'Ordre]
En utilisant les résultats des exercices 6, 7 et 8, vérifiez la relation d'ordre (inégalité) entre le mode, la médiane et l'espérance.
\end{exercicebox}

\begin{exercicebox}[Exercice 10 : Variance]
Soit $Y \sim \text{Log-}\mathcal{N}(\mu=1, \sigma^2=1)$.
Calculez la variance $\text{Var}(Y)$.
\end{exercicebox}

\begin{exercicebox}[Exercice 11 : Moments (Problème Inverse)]
Soit $Y$ une variable log-normale. On sait que sa médiane est $e^5$ et que son espérance est $e^{5.5}$.
\begin{enumerate}
    \item Trouvez $\mu$.
    \item Trouvez $\sigma^2$.
\end{enumerate}
\end{exercicebox}

\begin{exercicebox}[Exercice 12 : Impact de $\sigma$ sur l'Espérance]
Soient $A \sim \text{Log-}\mathcal{N}(0, 1)$ et $B \sim \text{Log-}\mathcal{N}(0, 2)$.
Les deux ont la même médiane ($e^0=1$). Laquelle a l'espérance la plus élevée ? Pourquoi ?
\end{exercicebox}

% --- Section 3 : Calculs de Probabilités ---

\begin{exercicebox}[Exercice 13 : Calcul de CDF (Simple)]
Soit $Y \sim \text{Log-}\mathcal{N}(\mu=5, \sigma^2=1)$. (donc $\sigma=1$).
Calculez $P(Y \le e^5)$.
\end{exercicebox}

\begin{exercicebox}[Exercice 14 : Calcul de CDF]
Soit $Y \sim \text{Log-}\mathcal{N}(\mu=3, \sigma=2)$.
Calculez $P(Y \le e^4)$.
\end{exercicebox}

\begin{exercicebox}[Exercice 15 : Calcul de Probabilité (Queue Droite)]
Soit $Y \sim \text{Log-}\mathcal{N}(\mu=0, \sigma=1)$.
Calculez $P(Y > 1)$.
\end{exercicebox}

\begin{exercicebox}[Exercice 16 : Calcul de Probabilité (Queue Gauche)]
Soit $Y \sim \text{Log-}\mathcal{N}(\mu=1, \sigma=1)$.
Calculez $P(Y \le e^{-1})$.
\end{exercicebox}

\begin{exercicebox}[Exercice 17 : Calcul de Probabilité (Intervalle)]
Soit $Y \sim \text{Log-}\mathcal{N}(\mu=10, \sigma=3)$.
Calculez $P(e \le Y \le e^{16})$.
\end{exercicebox}

\begin{exercicebox}[Exercice 18 : Application (Revenu)]
Le revenu $Y$ (en milliers) suit $\text{Log-}\mathcal{N}(\mu=3, \sigma=1)$.
Quelle est la probabilité qu'un individu ait un revenu $Y$ inférieur à $e^2$ (milliers) ?
\end{exercicebox}

% --- Section 4 : Problèmes Inverses ---

\begin{exercicebox}[Exercice 19 : Problème Inverse (Médiane)]
Soit $Y \sim \text{Log-}\mathcal{N}(\mu=5, \sigma=2)$. Trouvez la valeur $y$ telle que $P(Y \le y) = 0.5$.
\end{exercicebox}

\begin{exercicebox}[Exercice 20 : Problème Inverse (Percentile)]
Soit $Y \sim \text{Log-}\mathcal{N}(\mu=0, \sigma=1)$. Trouvez le 84.13ème percentile de $Y$.
\end{exercicebox}

\begin{exercicebox}[Exercice 21 : Problème Inverse (Percentile)]
Soit $Y \sim \text{Log-}\mathcal{N}(\mu=2, \sigma=3)$. Trouvez la valeur $y$ telle que $P(Y \le y) \approx 0.9772$.
\end{exercicebox}

\begin{exercicebox}[Exercice 22 : Problème Inverse (Quantile bas)]
Soit $Y \sim \text{Log-}\mathcal{N}(\mu=10, \sigma=2)$. Trouvez la valeur $y$ telle que $P(Y > y) \approx 0.9938$.
\end{exercicebox}

% --- Section 5 : Applications (Modèle Financier) ---

\begin{exercicebox}[Exercice 23 : Modèle Financier (Paramètres)]
Le log-rendement journalier $x_i$ d'un actif a $\mu=E[x_i]=0.01$ et $\sigma^2=\text{Var}(x_i)=0.04$.
Soit $X = \sum_{i=1}^{16} x_i$ le log-rendement total sur 16 jours. On suppose l'indépendance.
Quelle est la loi de $X$ ? (Donnez son nom, $E[X]$ et $\text{Var}(X)$).
\end{exercicebox}

\begin{exercicebox}[Exercice 24 : Modèle Financier (Loi du Prix)]
En utilisant l'exercice 23, soit $S(0)$ le prix initial et $S(16)$ le prix après 16 jours.
Quelle est la loi du \textit{ratio} de prix $Y = S(16)/S(0)$ ?
\end{exercicebox}

\begin{exercicebox}[Exercice 25 : Modèle Financier (Calcul de Prob.)]
Un actif a $S(0)=100$. Le log-rendement sur 1 an $X = \ln(S(1)/S(0))$ suit $\mathcal{N}(\mu=0.05, \sigma^2=0.09)$.
Quelle est la probabilité que le prix de l'actif après 1 an soit inférieur à 100 ? (c-à-d, $P(S(1) < 100)$).
\end{exercicebox}

\subsection{Corrections des Exercices}

\begin{correctionbox}[Correction Exercice 1 : Définition (Paramètres)]
1.  Par définition, si $Y \sim \text{Log-}\mathcal{N}(\mu, \sigma^2)$, alors $X = \ln(Y)$ suit une loi normale $\mathcal{N}(\mu, \sigma^2)$.
    Ici, $X \sim \mathcal{N}(3, 16)$.
2.  $E[X] = \mu = 3$.
    $\text{Var}(X) = \sigma^2 = 16$.
\end{correctionbox}

\begin{correctionbox}[Correction Exercice 2 : Définition (Inverse)]
Si $X \sim \mathcal{N}(0, 1)$, alors $Y = e^X$ suit une loi log-normale avec $\mu=0$ et $\sigma^2=1$.
La notation est $Y \sim \text{Log-}\mathcal{N}(0, 1)$.
\end{correctionbox}

\begin{correctionbox}[Correction Exercice 3 : Intuition (Somme vs Produit)]
La loi normale résulte de l'addition de nombreux petits effets (TCL).
La loi log-normale résulte de la multiplication de nombreux petits facteurs.
La croissance d'une population de bactéries est un processus multiplicatif ($P(t) = P(t-1) \times F_t$).
En prenant le logarithme, $\ln(P(t)) = \ln(P(0)) + \sum \ln(F_i)$, on obtient une somme qui tend vers une loi normale. Donc $P(t)$ lui-même tend vers une loi log-normale.
De plus, $P(t)$ doit être positif, ce que garantit la loi log-normale (support $(0, \infty)$).
\end{correctionbox}

\begin{correctionbox}[Correction Exercice 4 : Intuition (Transformation)]
La transformation $Y = e^X$ n'est pas linéaire. Elle est convexe.
Elle "étire" la partie droite de la distribution normale ($x>0 \implies e^x$ croît exponentiellement) et "compresse" la partie gauche ($x<0 \implies e^x$ s'écrase vers 0).
Une distribution symétrique (Normale) transformée par une fonction asymétrique (Exponentielle) donne une distribution asymétrique (Log-Normale).
\end{correctionbox}

\begin{correctionbox}[Correction Exercice 5 : PDF (Propriétés)]
Le support de la loi log-normale $Y$ est $(0, \infty)$. La variable ne peut jamais être négative ou nulle.
Par conséquent, $P(Y \le 0) = 0$.
\end{correctionbox}

\begin{correctionbox}[Correction Exercice 6 : Espérance]
$Y \sim \text{Log-}\mathcal{N}(\mu=4, \sigma^2=2)$.
$E[Y] = e^{\mu + \sigma^2/2} = e^{4 + 2/2} = e^{4 + 1} = e^5$.
\end{correctionbox}

\begin{correctionbox}[Correction Exercice 7 : Médiane]
$Y \sim \text{Log-}\mathcal{N}(\mu=4, \sigma^2=2)$.
$\text{Med}(Y) = e^{\mu} = e^4$.
\end{correctionbox}

\begin{correctionbox}[Correction Exercice 8 : Mode]
$Y \sim \text{Log-}\mathcal{N}(\mu=4, \sigma^2=2)$.
$\text{Mode}(Y) = e^{\mu - \sigma^2} = e^{4 - 2} = e^2$.
\end{correctionbox}

\begin{correctionbox}[Correction Exercice 9 : Relation d'Ordre]
Les valeurs sont :
$\text{Mode} = e^2 \approx 7.39$
$\text{Médiane} = e^4 \approx 54.60$
$\text{Espérance} = e^5 \approx 148.41$
On vérifie bien $\text{Mode} < \text{Médiane} < \text{Espérance}$, ce qui est la signature d'une asymétrie à droite.
\end{correctionbox}

\begin{correctionbox}[Correction Exercice 10 : Variance]
$Y \sim \text{Log-}\mathcal{N}(\mu=1, \sigma^2=1)$.
$\text{Var}(Y) = (e^{\sigma^2} - 1) \cdot e^{2\mu + \sigma^2}$
$\text{Var}(Y) = (e^{1} - 1) \cdot e^{2(1) + 1} = (e - 1)e^3$.
\end{correctionbox}

\begin{correctionbox}[Correction Exercice 11 : Moments (Problème Inverse)]
1.  $\text{Med}(Y) = e^{\mu}$. On nous donne $\text{Med}(Y) = e^5$. Donc $\mu = 5$.
2.  $E[Y] = e^{\mu + \sigma^2/2}$. On nous donne $E[Y] = e^{5.5}$.
    Donc, $e^{5.5} = e^{\mu + \sigma^2/2} = e^{5 + \sigma^2/2}$.
    En égalant les exposants : $5.5 = 5 + \sigma^2/2$.
    $0.5 = \sigma^2/2 \implies \sigma^2 = 1$.
\end{correctionbox}

\begin{correctionbox}[Correction Exercice 12 : Impact de $\sigma$ sur l'Espérance]
$E[A] = e^{\mu_A + \sigma_A^2/2} = e^{0 + 1/2} = e^{0.5}$.
$E[B] = e^{\mu_B + \sigma_B^2/2} = e^{0 + 2/2} = e^{1}$.
$E[B] > E[A]$. L'espérance $E[Y] = e^{\mu + \sigma^2/2}$ croît avec $\sigma^2$. Cela est dû à l'inégalité de Jensen ($\mathbb{E}[e^X] \geq e^{\mathbb{E}[X]}$) : une plus grande variance (incertitude) sur $X=\ln(Y)$ augmente l'espérance de $Y=e^X$ à cause de la convexité de l'exponentielle.
\end{correctionbox}

\begin{correctionbox}[Correction Exercice 13 : Calcul de CDF (Simple)]
On cherche $P(Y \le e^5)$ pour $Y \sim \text{Log-}\mathcal{N}(5, 1)$.
$$P(Y \le e^5) = P(\ln(Y) \le \ln(e^5)) = P(X \le 5)$$
Puisque $X = \ln(Y) \sim \mathcal{N}(\mu=5, \sigma^2=1)$, on cherche la probabilité que $X$ soit inférieure à sa propre moyenne. Par symétrie de la loi normale, c'est 0.5.
Formellement : $Z = \frac{5 - 5}{1} = 0$. $P(Z \le 0) = \Phi(0) = 0.5$.
\end{correctionbox}

\begin{correctionbox}[Correction Exercice 14 : Calcul de CDF]
$Y \sim \text{Log-}\mathcal{N}(\mu=3, \sigma=2)$.
$P(Y \le e^4) = P(\ln(Y) \le \ln(e^4)) = P(X \le 4)$, où $X \sim \mathcal{N}(3, 4)$.
Standardisation : $Z = \frac{X - \mu}{\sigma} = \frac{4 - 3}{2} = \frac{1}{2} = 0.5$.
$P(X \le 4) = P(Z \le 0.5) = \Phi(0.5) \approx 0.6915$.
\end{correctionbox}

\begin{correctionbox}[Correction Exercice 15 : Calcul de Probabilité (Queue Droite)]
$Y \sim \text{Log-}\mathcal{N}(0, 1)$.
$P(Y > 1) = 1 - P(Y \le 1) = 1 - P(\ln(Y) \le \ln(1)) = 1 - P(X \le 0)$.
$X \sim \mathcal{N}(0, 1)$. $P(X \le 0) = \Phi(0) = 0.5$.
$P(Y > 1) = 1 - 0.5 = 0.5$. (Logique : $e^0=1$ est la médiane, donc 50% est au-dessus).
\end{correctionbox}

\begin{correctionbox}[Correction Exercice 16 : Calcul de Probabilité (Queue Gauche)]
$Y \sim \text{Log-}\mathcal{N}(1, 1)$.
$P(Y \le e^{-1}) = P(\ln(Y) \le \ln(e^{-1})) = P(X \le -1)$, où $X \sim \mathcal{N}(1, 1)$.
Standardisation : $Z = \frac{X - \mu}{\sigma} = \frac{-1 - 1}{1} = -2$.
$P(X \le -1) = P(Z \le -2) = \Phi(-2) = 1 - \Phi(2) \approx 1 - 0.9772 = 0.0228$.
\end{correctionbox}

\begin{correctionbox}[Correction Exercice 17 : Calcul de Probabilité (Intervalle)]
$Y \sim \text{Log-}\mathcal{N}(10, 9)$. ($X \sim \mathcal{N}(10, 9)$, $\sigma=3$).
$P(e \le Y \le e^{16}) = P(\ln(e) \le \ln(Y) \le \ln(e^{16})) = P(1 \le X \le 16)$.
Standardisation :
$z_1 = \frac{1 - 10}{3} = -3$.
$z_2 = \frac{16 - 10}{3} = 2$.
$P(1 \le X \le 16) = P(-3 \le Z \le 2) = \Phi(2) - \Phi(-3)$.
$\Phi(-3) = 1 - \Phi(3) \approx 1 - 0.9987 = 0.0013$.
$P = \Phi(2) - \Phi(-3) \approx 0.9772 - 0.0013 = 0.9759$.
\end{correctionbox}

\begin{correctionbox}[Correction Exercice 18 : Application (Revenu)]
$Y \sim \text{Log-}\mathcal{N}(3, 1)$. ($X \sim \mathcal{N}(3, 1)$, $\sigma=1$).
$P(Y < e^2) = P(\ln(Y) < \ln(e^2)) = P(X < 2)$.
Standardisation : $Z = \frac{2 - 3}{1} = -1$.
$P(X < 2) = P(Z < -1) = \Phi(-1) = 1 - \Phi(1) \approx 1 - 0.8413 = 0.1587$.
\end{correctionbox}

\begin{correctionbox}[Correction Exercice 19 : Problème Inverse (Médiane)]
On cherche $y$ tel que $P(Y \le y) = 0.5$.
C'est la définition de la médiane. $\text{Med}(Y) = e^{\mu} = e^5$.
Donc $y = e^5$.
\end{correctionbox}

\begin{correctionbox}[Correction Exercice 20 : Problème Inverse (Percentile)]
$Y \sim \text{Log-}\mathcal{N}(0, 1)$. On cherche $y$ tel que $P(Y \le y) = 0.8413$.
$P(\ln(Y) \le \ln(y)) = 0.8413$.
$P(X \le \ln(y)) = 0.8413$, où $X \sim \mathcal{N}(0, 1)$.
$P(Z \le \ln(y)) = 0.8413$.
On cherche $z = \ln(y)$ tel que $\Phi(z) = 0.8413$. D'après la table, $z=1$.
$\ln(y) = 1 \implies y = e^1 = e$.
\end{correctionbox}

\begin{correctionbox}[Correction Exercice 21 : Problème Inverse (Percentile)]
$Y \sim \text{Log-}\mathcal{N}(2, 9)$. ($X \sim \mathcal{N}(2, 9)$, $\sigma=3$).
On cherche $y$ tel que $P(Y \le y) \approx 0.9772$.
$P(\ln(Y) \le \ln(y)) = P(X \le \ln(y)) = 0.9772$.
Standardisation : $P\left(Z \le \frac{\ln(y) - 2}{3}\right) = 0.9772$.
On cherche $z$ tel que $\Phi(z) = 0.9772$. D'après la table, $z=2$.
$\frac{\ln(y) - 2}{3} = 2 \implies \ln(y) - 2 = 6 \implies \ln(y) = 8 \implies y = e^8$.
\end{correctionbox}

\begin{correctionbox}[Correction Exercice 22 : Problème Inverse (Quantile bas)]
$Y \sim \text{Log-}\mathcal{N}(10, 4)$. ($X \sim \mathcal{N}(10, 4)$, $\sigma=2$).
On cherche $y$ tel que $P(Y > y) \approx 0.9938$.
$P(Y \le y) = 1 - 0.9938 = 0.0062$.
$P(X \le \ln(y)) = 0.0062$.
Standardisation : $P\left(Z \le \frac{\ln(y) - 10}{2}\right) = 0.0062$.
On cherche $z$ tel que $\Phi(z) = 0.0062$.
La table ne donne pas cette valeur, mais $\Phi(2.5) \approx 0.9938$.
Par symétrie, $\Phi(-2.5) = 1 - \Phi(2.5) \approx 1 - 0.9938 = 0.0062$.
Donc, $z = -2.5$.
$\frac{\ln(y) - 10}{2} = -2.5 \implies \ln(y) - 10 = -5 \implies \ln(y) = 5 \implies y = e^5$.
\end{correctionbox}

\begin{correctionbox}[Correction Exercice 23 : Modèle Financier (Paramètres)]
$X$ est une somme de 16 v.a. i.i.d. $x_i \sim \mathcal{N}(0.01, 0.04)$.
$X$ suit une loi normale.
$E[X] = \sum E[x_i] = t \cdot \mu = 16 \times 0.01 = 0.16$.
$\text{Var}(X) = \sum \text{Var}(x_i) = t \cdot \sigma^2 = 16 \times 0.04 = 0.64$.
Donc, $X \sim \mathcal{N}(0.16, 0.64)$.
\end{correctionbox}

\begin{correctionbox}[Correction Exercice 24 : Modèle Financier (Loi du Prix)]
$Y = S(16)/S(0)$. $X = \ln(Y) = \ln(S(16)/S(0))$.
D'après l'exercice 23, $X \sim \mathcal{N}(0.16, 0.64)$.
Par définition, $Y = e^X$ suit une loi log-normale.
$Y \sim \text{Log-}\mathcal{N}(\mu_X=0.16, \sigma_X^2=0.64)$.
\end{correctionbox}

\begin{correctionbox}[Correction Exercice 25 : Modèle Financier (Calcul de Prob.)]
$X = \ln(S(1)/S(0)) \sim \mathcal{N}(0.05, 0.09)$. (donc $\mu=0.05, \sigma=0.3$).
On cherche $P(S(1) < 100)$.
$P(S(1) < 100) = P(S(1)/S(0) < 100/100) = P(S(1)/S(0) < 1)$.
On applique le log :
$P(\ln(S(1)/S(0)) < \ln(1)) = P(X < 0)$.
On standardise $X$ :
$Z = \frac{X - \mu}{\sigma} = \frac{0 - 0.05}{0.3} = -0.05 / 0.3 \approx -0.167$.
$P(X < 0) = P(Z < -0.167)$.
C'est $\Phi(-0.167) = 1 - \Phi(0.167)$. Puisque $\Phi(0) = 0.5$, $\Phi(0.167)$ est légèrement supérieur à 0.5, donc $\Phi(-0.167)$ est légèrement inférieur à 0.5. La probabilité est légèrement inférieure à 50\%.
\end{correctionbox}

\subsection{Exercices Python}

La loi log-normale est fondamentale en finance. Elle repose sur l'idée que si les \textbf{log-rendements} d'une action $X_i = \ln(P_i / P_{i-1})$ sont (approximativement) normaux, alors le prix futur $P_t$, qui est un \textbf{produit} de ces rendements ($P_t = P_0 \times e^{X_1} \times \dots \times e^{X_t}$), suivra une loi log-normale.

Nous allons estimer les paramètres $\mu$ et $\sigma^2$ de la loi normale sous-jacente à partir des log-rendements journaliers de Microsoft (MSFT) et Google (GOOG), puis utiliser la théorie log-normale pour modéliser les prix.

\begin{codecell}
!pip install yfinance
import yfinance as yf
import pandas as pd
import numpy as np
from scipy.stats import norm # Moteur pour les calculs de CDF/PDF

# Definir les tickers et la periode
tickers = ["MSFT", "GOOG"]
start_date = "2020-01-01"
end_date = "2024-12-31"

# Telecharger les prix de cloture ajustes
data = yf.download(tickers, start=start_date, end=end_date)["Adj Close"]

# Calculer les LOG-RENDEMENTS journaliers
log_returns = np.log(data / data.shift(1)).dropna()

# Renommer les colonnes
log_returns.columns = ["MSFT_LogReturn", "GOOG_LogReturn"]

# 'log_returns' est notre DataFrame.
# X_msft = log_returns["MSFT_LogReturn"]
# X_goog = log_returns["GOOG_LogReturn"]
\end{codecell}

\begin{exercicebox}[Exercice 1 : Estimer les Paramètres $\mu$ et $\sigma^2$]
Soit $P_t$ le prix de MSFT. Le modèle suppose que $X = \ln(P_t/P_{t-1}) \sim \mathcal{N}(\mu, \sigma^2)$. Les paramètres $\mu$ et $\sigma^2$ sont les paramètres "log-normaux".

\textbf{Votre tâche :}
\begin{enumerate}
    \item Estimer $\mu$ (l'espérance du log-rendement journalier) pour MSFT.
    \item Estimer $\sigma^2$ (la variance du log-rendement journalier) pour MSFT.
    \item Estimer $\sigma$ (l'écart-type du log-rendement journalier) pour MSFT.
\end{enumerate}
\end{exercicebox}

\begin{exercicebox}[Exercice 2 : Test de Normalité (Règle 68-95-99.7)]
La théorie log-normale repose sur la normalité des log-rendements $X$. Vérifions-le.

\textbf{Votre tâche :}
\begin{enumerate}
    \item Utiliser $\mu$ et $\sigma$ (pour MSFT) de l'Exercice 1.
    \item Calculer la proportion \textbf{empirique} des log-rendements de MSFT qui tombent dans l'intervalle $[\mu - \sigma, \mu + \sigma]$.
    \item Comparer ce pourcentage à la valeur théorique (68.27\%). Le modèle semble-t-il bien s'ajuster ?
\end{enumerate}
\end{exercicebox}

\begin{exercicebox}[Exercice 3 : Asymétrie (Prix vs Log-Rendements)]
La théorie dit que les log-rendements $X$ sont symétriques (Normaux), mais que les prix $P_t$ sont asymétriques à droite (Log-Normaux).

\textbf{Votre tâche :}
\begin{enumerate}
    \item Calculer la moyenne et la médiane de la série des \textbf{log-rendements} de MSFT.
    \item Calculer la moyenne et la médiane de la série des \textbf{prix} de MSFT (la colonne \texttt{data['MSFT']}).
    \item Comparer les deux paires. Les log-rendements sont-ils symétriques (moyenne $\approx$ médiane) ? Les prix sont-ils asymétriques (moyenne $>$ médiane) ?
\end{enumerate}
\end{exercicebox}

\begin{exercicebox}[Exercice 4 : Espérance vs Médiane (Théorique)]
Soit $Y = P_t/P_{t-1} = e^X$ la variable "ratio de prix journalier". $Y \sim \text{Log-}\mathcal{N}(\mu, \sigma^2)$.
Théorie : $\text{Med}(Y) = e^{\mu}$ et $E[Y] = e^{\mu + \sigma^2/2}$.

\textbf{Votre tâche :}
\begin{enumerate}
    \item Utiliser $\mu$ et $\sigma^2$ (pour MSFT) de l'Exercice 1.
    \item Calculer la médiane \textbf{théorique} $\text{Med}(Y)$.
    \item Calculer l'espérance \textbf{théorique} $E[Y]$.
    \item Vérifier que $E[Y] > \text{Med}(Y)$, confirmant l'asymétrie.
\end{enumerate}
\end{exercicebox}

\begin{exercicebox}[Exercice 5 : Espérance Théorique vs Empirique]
Vérifions le calcul de $E[Y]$ de l'exercice 4 de manière empirique.

\textbf{Votre tâche :}
\begin{enumerate}
    \item Créer la série $Y$ (ratio de prix journalier) : $Y = \exp(X_{\text{msft}})$.
    \item Calculer l'espérance \textbf{empirique} de $Y$ (la moyenne de cette série $Y$).
    \item Comparer cette valeur empirique à l'espérance \textbf{théorique} $e^{\mu + \sigma^2/2}$ calculée à l'exercice 4.
\end{enumerate}
\end{exercicebox}

\begin{exercicebox}[Exercice 6 : Variance Théorique vs Empirique]
Théorie : $\text{Var}(Y) = (e^{\sigma^2} - 1) \cdot e^{2\mu + \sigma^2}$.

\textbf{Votre tâche :}
\begin{enumerate}
    \item Utiliser $\mu$ et $\sigma^2$ (pour MSFT) de l'Exercice 1.
    \item Calculer la variance \textbf{théorique} $\text{Var}(Y)$ en utilisant la formule ci-dessus.
    \item Calculer la variance \textbf{empirique} de la série $Y$ (créée à l'Ex 5).
    \item Comparer les deux résultats.
\end{enumerate}
\end{exercicebox}

\begin{exercicebox}[Exercice 7 : Modélisation du Prix Futur (Paramètres)]
Modélisons le prix de GOOG dans $t=20$ jours ouvrés (environ 1 mois).
Le prix $P_{20}$ est log-normal si l'on suppose $P_{20} = P_0 \cdot e^{X_{20}}$, où $P_0$ est le prix actuel.
Le log-rendement total $X_{20} = \ln(P_{20}/P_0)$ suit $X_{20} \sim \mathcal{N}(t\mu, t\sigma^2)$.

\textbf{Votre tâche :}
\begin{enumerate}
    \item Estimer $\mu_G$ et $\sigma_G^2$ (journaliers) pour GOOG (similaire à l'Ex 1).
    \item Définir $t=20$.
    \item Calculer $\mu_{20} = t\mu_G$ (l'espérance du log-rendement sur 20 jours).
    \item Calculer $\sigma_{20}^2 = t\sigma_G^2$ (la variance du log-rendement sur 20 jours).
\end{enumerate}
\end{exercicebox}

\begin{exercicebox}[Exercice 8 : Calcul de Probabilité (Prix Futur)]
En utilisant les paramètres $\mu_{20}$ et $\sigma_{20} = \sqrt{\sigma_{20}^2}$ de l'exercice 7 pour GOOG :

\textbf{Votre tâche :}
\begin{enumerate}
    \item Calculer la probabilité que GOOG ait un rendement positif sur 20 jours.
    \item On cherche $P(P_{20} > P_0) \implies P(P_{20}/P_0 > 1) \implies P(\ln(P_{20}/P_0) > \ln(1))$.
    \item Calculer $P(X_{20} > 0)$.
    \item (Indice : Standardiser 0 avec $\mu_{20}$ et $\sigma_{20}$, puis utiliser $1 - \Phi(z)$).
\end{enumerate}
\end{exercicebox}

\begin{exercicebox}[Exercice 9 : Calcul de Probabilité (Perte > 5\%)]
En utilisant les paramètres $\mu_{20}$ et $\sigma_{20}$ de l'exercice 7 pour GOOG :

\textbf{Votre tâche :}
\begin{enumerate}
    \item Calculer la probabilité que GOOG perde plus de 5\% sur 20 jours.
    \item On cherche $P(P_{20} < 0.95 \times P_0) \implies P(P_{20}/P_0 < 0.95)$.
    \item Calculer $P(X_{20} < \ln(0.95))$.
    \item (Indice : Standardiser $\ln(0.95)$ avec $\mu_{20}$ et $\sigma_{20}$, puis utiliser $\Phi(z)$).
\end{enumerate}
\end{exercicebox}

\begin{exercicebox}[Exercice 10 : Problème Inverse (Intervalle de Confiance)]
Trouvons l'intervalle de 95\% pour le prix de GOOG dans 20 jours.
Nous cherchons les bornes $y_1, y_2$ telles que $P(y_1 \le P_{20} \le y_2) = 0.95$.
On suppose un intervalle centré sur la loi normale sous-jacente (entre $z=-1.96$ et $z=+1.96$).

\textbf{Votre tâche :}
\begin{enumerate}
    \item Trouver $z_{inf} = -1.96$ et $z_{sup} = +1.96$.
    \item "Dé-standardiser" ces Z-scores pour trouver les log-rendements $x_1$ et $x_2$ :
        $x = \mu_t + z \sigma_t$ (en utilisant $\mu_{20}$ et $\sigma_{20}$ de l'Ex 7).
    \item Convertir ces log-rendements en ratios de prix $y = e^x$.
    \item (Conclusion) L'intervalle de 95\% pour le ratio de prix est $[y_1, y_2]$.
\end{enumerate}
\end{exercicebox}

\begin{exercicebox}[Exercice 11 : Calcul de la Médiane vs Espérance (Prix Futur)]
Pour le prix de GOOG dans 20 jours, $P_{20} = P_0 \cdot Y_{20}$, où $Y_{20} \sim \text{Log-}\mathcal{N}(\mu_{20}, \sigma_{20}^2)$.

\textbf{Votre tâche :}
\begin{enumerate}
    \item Calculer le ratio de prix \textbf{médian} attendu : $\text{Med}(Y_{20}) = e^{\mu_{20}}$.
    \item Calculer le ratio de prix \textbf{moyen} (espérance) attendu : $E[Y_{20}] = e^{\mu_{20} + \sigma_{20}^2 / 2}$.
    \item (Conclusion) Lequel est le plus élevé ? Pourquoi est-ce important pour un investisseur ?
\end{enumerate}
\end{exercicebox}

\begin{exercicebox}[Exercice 12 : Mode (Prix Futur)]
Théorie : Le mode (la valeur la plus probable) du ratio de prix $Y_{20}$ est $\text{Mode}(Y_{20}) = e^{\mu_{20} - \sigma_{20}^2}$.

\textbf{Votre tâche :}
\begin{enumerate}
    \item Calculer le ratio de prix \textbf{modal} (le plus probable) pour GOOG dans 20 jours.
    \item Comparer le Mode (Ex 12), la Médiane (Ex 11) et l'Espérance (Ex 11).
    \item Vérifier que $\text{Mode} < \text{Médiane} < \text{Espérance}$, confirmant l'asymétrie à droite.
\end{enumerate}
\end{exercicebox}
\newpage

\section{Moments d'une distribution}

\subsection{Définitions fondamentales des moments}

Après avoir défini l'espérance ($\mu$) et la variance ($\sigma^2$), qui sont les moments d'ordre 1 et 2, nous pouvons généraliser cette idée pour capturer des informations plus subtiles sur la forme d'une distribution.

\begin{definitionbox}[Types de Moments]
Soit $X$ une variable aléatoire ayant une espérance $\mu$ et une variance $\sigma^2$. Pour tout entier positif $m$, on définit les moments suivants :
\begin{itemize}
    \item \textbf{$m$-ième moment (non centré)} : $E[X^m]$.
    \item \textbf{$m$-ième moment centré} : $E[(X - \mu)^m]$.
    \item \textbf{$m$-ième moment standardisé} : $E\left[\left(\frac{X - \mu}{\sigma}\right)^m\right]$.
\end{itemize}
Les moments centrés et standardisés permettent d'étudier les propriétés de la distribution indépendamment de sa position ($\mu$) et de son échelle ($\sigma$).
\end{definitionbox}

\subsection{Asymétrie (Skewness)}

Le premier moment nous donne la tendance centrale. Le deuxième moment (la variance) nous donne la dispersion. Le troisième moment, lui, va nous renseigner sur la \textit{symétrie} de la distribution.

\begin{definitionbox}[Asymétrie (Skewness)]
L'\textbf{asymétrie} (ou \textit{skewness}) d'une variable aléatoire $X$ de moyenne $\mu$ et d'écart-type $\sigma$ est définie comme le \textbf{troisième moment standardisé} :
$$ \text{Skew}(X) = E\left[ \left( \frac{X - \mu}{\sigma} \right)^3 \right]. $$
\end{definitionbox}

\begin{intuitionbox}[Comprendre la Formule du Skewness]
Pour une variable aléatoire $X$ de moyenne $\mu$ et d'écart-type $\sigma$, le \textbf{skewness} est défini comme :
\[
\text{Skew}(X) = \frac{E[(X - \mu)^3]}{\sigma^3}
\]

\medskip

\textbf{Logique du numérateur : le moment centré d'ordre 3}
\begin{itemize}
    \item Le terme $(X - \mu)^3$ est le \textbf{cube de l'écart à la moyenne}
    \item Contrairement à $(X - \mu)^2$ (toujours positif), le cube \textbf{conserve le signe} de l'écart
    \item Il pondère différemment les observations à gauche et à droite de la moyenne
\end{itemize}

\medskip

% --- MODIFIÉ : Tableau supprimé et fusionné dans la liste ---
\textbf{Interprétation intuitive}
\begin{itemize}
    \item \textbf{Skewness = 0 (Symétrique)} : La distribution est symétrique. Les écarts positifs et négatifs s'annulent. Typiquement : Moyenne = Médiane = Mode.
    \item \textbf{Skewness > 0 (Queue à droite)} : La distribution présente une queue longue à droite. Les grandes valeurs positives sont amplifiées par le cube. Les valeurs extrêmes tirent la moyenne vers la droite.
    \item \textbf{Skewness < 0 (Queue à gauche)} : La distribution présente une queue longue à gauche. Les écarts négatifs dominent. Les valeurs extrêmes tirent la moyenne vers la gauche.
\end{itemize}
% --- FIN MODIFICATION ---

\medskip

\textbf{Pourquoi $\sigma^3$ au dénominateur ?}
\begin{itemize}
    \item Le moment d'ordre 3 est homogène à des unités au cube
    \item On divise par $\sigma^3$ pour obtenir un coefficient \textbf{sans dimension}
    \item Permet la comparaison entre distributions de différentes échelles
\end{itemize}
\end{intuitionbox}

\begin{remarquebox}[Pourquoi Standardiser ?]
En standardisant d'abord ($\frac{X-\mu}{\sigma}$), la définition de $\text{Skew}(X)$ ne dépend ni de la position ($\mu$) ni de l'échelle ($\sigma$) de la distribution, ce qui est raisonnable puisque ces informations sont déjà fournies par la moyenne et l'écart-type. De plus, cette standardisation garantit que l'asymétrie est invariante par changement d'unité de mesure (par exemple, passer des pouces aux mètres n'affecte pas la valeur de l'asymétrie).
\end{remarquebox}

\subsection{Propriétés de symétrie}

Le skewness est une mesure numérique de l'asymétrie. Mais nous pouvons aussi définir la symétrie de manière formelle.

\begin{definitionbox}[Symétrie d'une Variable Aléatoire]
On dit qu'une variable aléatoire $X$ a une distribution \textbf{symétrique} autour de $\mu$ si la variable $X - \mu$ a la même distribution que $\mu - X$. On dit aussi que $X$ est symétrique ou que sa distribution est symétrique. Ces trois formulations ont le même sens.
\end{definitionbox}

\begin{theorembox}[Symétrie en Termes de Fonction de Densité]
Soit $X$ une variable aléatoire continue de fonction de densité de probabilité (PDF) $f$. Alors, $X$ est symétrique autour de $\mu$ si et seulement si :
$$ f(x) = f(2\mu - x) \quad \text{pour tout } x. $$
\end{theorembox}

\begin{proofbox}[Preuve du Théorème de Symétrie]
Soit $F$ la fonction de répartition (CDF) de $X$. Si la symétrie tient, alors :
$$ F(x) = P(X \le x) = P(X - \mu \le x - \mu) = P(\mu - X \le x - \mu) = P(X \ge 2\mu - x) = 1 - F(2\mu - x). $$

En prenant la dérivée des deux côtés par rapport à $x$, on obtient :
$$ f(x) = \frac{d}{dx}F(x) = \frac{d}{dx}[1 - F(2\mu - x)] = f(2\mu - x). $$

Cela démontre que la condition $f(x) = f(2\mu - x)$ est nécessaire et suffisante pour la symétrie.
\end{proofbox}

\subsection{Aplatissement (Kurtosis)}

Après l'asymétrie (ordre 3), le moment d'ordre 4 nous informe sur "l'épaisseur" des queues de la distribution, c'est-à-dire la probabilité d'obtenir des valeurs très éloignées de la moyenne.

\begin{definitionbox}[Kurtosis (Aplatissement)]
Pour une variable aléatoire $X$ de moyenne $\mu$ et d'écart-type $\sigma$, le \textbf{kurtosis} est défini comme le \textbf{quatrième moment standardisé} :
$$ \text{Kurtosis}(X) = E\left[ \left( \frac{X - \mu}{\sigma} \right)^4 \right]. $$

Dans la pratique, on utilise plus souvent le \textbf{kurtosis excessif} (ou excès de kurtosis), défini comme :
$$ \text{Excess Kurtosis}(X) = E\left[ \left( \frac{X - \mu}{\sigma} \right)^4 \right] - 3. $$
La soustraction de 3 fait en sorte que le kurtosis d'une loi normale soit égal à 0.
\end{definitionbox}

\begin{intuitionbox}[Comprendre la Kurtosis]
Pour une variable aléatoire $X$, le \textbf{kurtosis} est défini comme :
\[
\text{Kurt}(X) = \frac{E[(X - \mu)^4]}{\sigma^4}
\]
et l'\textbf{excess kurtosis} (kurtosis excédentaire) comme : $\text{Excess Kurtosis} = \text{Kurt}(X) - 3$.

\medskip

\textbf{Pourquoi le moment d'ordre 4 ?}
\begin{itemize}
    \item Comme la variance, on utilise une puissance paire (pas d'effet de signe)
    \item La puissance 4 \textbf{amplifie énormément les écarts extrêmes}
    \item Mesure le \textbf{poids des queues} et la \textbf{concentration autour de la moyenne}
\end{itemize}

\medskip

% --- MODIFIÉ : Tableau supprimé et fusionné dans la liste ---
\textbf{Interprétation intuitive (basée sur l'Excess Kurtosis)}
\begin{itemize}
    \item \textbf{Leptokurtique (Excess Kurtosis > 0)} : Kurtosis total > 3. Distribution pointue avec des queues épaisses. Les événements extrêmes sont plus probables que pour une loi normale.
    \item \textbf{Mésocurtique (Excess Kurtosis = 0)} : Kurtosis total = 3. C'est la référence (loi normale).
    \item \textbf{Platykurtique (Excess Kurtosis < 0)} : Kurtosis total < 3. Distribution aplatie avec des queues légères et un centre large. Les événements extrêmes sont moins probables.
\end{itemize}
% --- FIN MODIFICATION ---

\medskip

\textbf{Application en finance}
\begin{itemize}
    \item Les rendements financiers ont souvent un excès de kurtosis positif
    \item Indique une probabilité plus élevée d'événements extrêmes que la loi normale
    \item Justifie le "vol smile" dans les options
\end{itemize}

\medskip

\textbf{Pourquoi $\sigma^4$ au dénominateur ?}
\begin{itemize}
    \item Le moment d'ordre 4 est homogène à des unités$^4$
    \item On divise par $\sigma^4$ pour un coefficient \textbf{sans dimension}
\end{itemize}
\end{intuitionbox}

\subsection{Exemples de distributions}

Pour bien fixer les idées, comparons le skewness et le kurtosis de plusieurs distributions classiques. Notez que dans les graphiques suivants, le "Kurtosis" affiché est l'\textit{excess kurtosis} (centré à 0).

\begin{examplebox}[La Distribution Normale (Mésokurtique)]

\begin{center}
\begin{tikzpicture}
  \begin{axis}[
    width=0.8\textwidth,
    height=0.5\textwidth,
    xlabel={$x$},
    title={Densité Normale ($\mu=0$, $\sigma=1$)},
    grid=both,
    grid style={line width=.1pt, draw=gray!30},
    major grid style={line width=.2pt,draw=gray!50},
    domain=-4:4,
    samples=100,
    enlargelimits=false,
    axis lines=middle,
    xmin=-4, xmax=4,
    ymin=0, ymax=0.55
  ]
 
  % Courbe de densité
  \addplot [thick, color=blue, fill=blue!20, fill opacity=0.5] 
    {1/(sqrt(2*pi))*exp(-x^2/2)} \closedcycle;
 
  % Ligne de la moyenne
  \addplot [red, dashed, thick] coordinates {(0,0) (0,0.4)};
  \node[red, fill=white, font=\scriptsize, rounded corners, inner sep=2pt, opacity=0.8] at (axis cs:0.5,0.35) {Moyenne};
 
  % Boîte de texte avec moments seulement
  \node [draw=black, fill=white, rounded corners, font=\scriptsize, align=left, anchor=north east, opacity=0.8] 
    at (axis description cs:0.95,0.95) { % MODIFIÉ
    \textbf{Moments :}\\
    • Moyenne = 0.00\\
    • Variance = 1.00\\
    • Skewness = 0.00\\
    • Kurtosis = 0.00
    };
    
  \end{axis}
\end{tikzpicture}
\end{center}

La distribution normale est l'archétype de la courbe en cloche. Imaginez une cible : la majorité des flèches touchent le centre, et plus on s'éloigne du centre, moins il y a de chances d'être touché. C'est une distribution parfaitement symétrique, ce qui se traduit par un \textbf{skewness nul (0.00)}. Son pic est ni trop pointu, ni trop plat : c'est notre point de référence, on dit qu'elle est \textbf{mésokurtique}, d'où son kurtosis de \textbf{0.00}. C'est la base de nombreuses analyses statistiques car elle modélise naturellement beaucoup de phénomènes.

\end{examplebox}

\begin{examplebox}[La Distribution Exponentielle (Asymétrique à Droite)]

\begin{center}
\begin{tikzpicture}
  \begin{axis}[
    width=0.8\textwidth,
    height=0.5\textwidth,
    xlabel={$x$},
    title={Densité Exponentielle ($\lambda=1$)},
    grid=both,
    grid style={line width=.1pt, draw=gray!30},
    major grid style={line width=.2pt,draw=gray!50},
    domain=0:6,
    samples=100,
    enlargelimits=false,
    axis lines=middle,
    xmin=0, xmax=6,
    ymin=0, ymax=1.1
  ]
 
  % Courbe de densité exponentielle
  \addplot [thick, color=blue, fill=blue!20, fill opacity=0.5] 
    {exp(-x)} \closedcycle;
 
  % Ligne de la moyenne
  \addplot [red, dashed, thick] coordinates {(1,0) (1,0.37)};
  \node[red, fill=white, font=\scriptsize, rounded corners, inner sep=2pt, opacity=0.8] at (axis cs:1.5,0.3) {Moyenne};
 
  % Boîte de texte avec moments seulement
  \node [draw=black, fill=white, rounded corners, font=\scriptsize, align=left, anchor=north east, opacity=0.8] 
    at (axis description cs:0.95,0.95) { % MODIFIÉ
    \textbf{Moments :}\\
    • Moyenne = 1.00\\
    • Variance = 1.00\\
    • Skewness = 2.00\\
    • Kurtosis = 6.00
    };
    
  \end{axis}
\end{tikzpicture}
\end{center}

Imaginez le temps d'attente avant un événement rare, comme un appel téléphonique. La plupart du temps, l'appel arrive vite, mais il peut parfois y avoir de longues attentes. C'est exactement ce que modélise la distribution exponentielle : un pic à gauche et une longue queue à droite. Cela se traduit par un \textbf{skewness positif élevé (2.00)}, indiquant une asymétrie marquée. Elle est aussi \textbf{leptokurtique} (\textbf{kurtosis = 6.00}) : son pic est pointu, et la longue queue droite signifie qu'il y a une probabilité non négligeable de valeurs extrêmes.

\end{examplebox}

\begin{examplebox}[La Distribution Uniforme (Platykurtique)]

\begin{center}
\begin{tikzpicture}
  \begin{axis}[
    width=0.8\textwidth,
    height=0.5\textwidth,
    xlabel={$x$},
    title={Densité Uniforme ($a=0$, $b=2$)},
    grid=both,
    grid style={line width=.1pt, draw=gray!30},
    major grid style={line width=.2pt,draw=gray!50},
    domain=-0.5:2.5,
    samples=100,
    enlargelimits=false,
    axis lines=middle,
    ymin=0,
    ymax=.75,
    xmin=-1, xmax=4
  ]
 
  % Courbe de densité uniforme
  \addplot [thick, color=blue, fill=blue!20, fill opacity=0.5, const plot] 
    coordinates {(-0.5,0) (0,0) (0,0.5) (2,0.5) (2,0) (2.5,0)};
 
  % Ligne de la moyenne
  \addplot [red, dashed, thick] coordinates {(1,0) (1,0.5)};
  \node[red, fill=white, font=\scriptsize, rounded corners, inner sep=2pt, opacity=0.8] at (axis cs:1.3,0.4) {Moyenne};
 
  % Boîte de texte avec moments seulement
  \node [draw=black, fill=white, rounded corners, font=\scriptsize, align=left, anchor=north east, opacity=0.8] 
    at (axis description cs:0.95,0.95) { % MODIFIÉ
    \textbf{Moments :}\\
    • Moyenne = 1.00\\
    • Variance = 0.33\\
    • Skewness = 0.00\\
    • Kurtosis = -1.20
    };
    
  \end{axis}
\end{tikzpicture}
\end{center}

La distribution uniforme, c'est le "tirage au sort parfait" : chaque valeur sur un intervalle a la même chance d'être tirée. Visuellement, c'est un rectangle, donc aucune valeur n'est privilégiée. Elle est symétrique (\textbf{skewness = 0.00}), mais contrairement à la normale, elle est "plate", sans pic central. Cela se traduit par un \textbf{kurtosis négatif (-1.20)}, ce qui signifie qu'elle est \textbf{platykurtique}. Elle est donc très différente des distributions avec un pic central comme la normale.

\end{examplebox}

\begin{examplebox}[La Distribution Log-Normale (Fortement Leptokurtique)]

\begin{center}
\begin{tikzpicture}
  \begin{axis}[
    width=0.8\textwidth,
    height=0.5\textwidth,
    xlabel={$x$},
    title={Densité Log-Normale ($\sigma=0.7$)},
    grid=both,
    grid style={line width=.1pt, draw=gray!30},
    major grid style={line width=.2pt,draw=gray!50},
    domain=0:6,
    samples=100,
    enlargelimits=false,
    axis lines=middle,
    xmin=0, xmax=6,
    ymin=0, ymax=1
  ]
 
  % Courbe de densité log-normale
  \addplot [thick, color=blue, fill=blue!20, fill opacity=0.5] 
    {1/(x*0.7*sqrt(2*pi))*exp(-(ln(x))^2/(2*0.7^2))};
 
  % Ligne de la moyenne
  \addplot [red, dashed, thick] coordinates {(1.28,0) (1.28,0.4)};
  \node[red, fill=white, font=\scriptsize, rounded corners, inner sep=2pt, opacity=0.8] at (axis cs:1.8,0.5) {Moyenne};
 
  % Boîte de texte avec moments seulement
  \node [draw=black, fill=white, rounded corners, font=\scriptsize, align=left, anchor=north east, opacity=0.8] 
    at (axis description cs:0.95,0.95) { % MODIFIÉ
    \textbf{Moments :}\\
    • Moyenne = 1.28\\
    • Variance = 1.03\\
    • Skewness = 2.89\\
    • Kurtosis = 20.78
    };
    
  \end{axis}
\end{tikzpicture}
\end{center}

La log-normale est une distribution très asymétrique. Imaginez la richesse d'une population : la majorité est modeste, mais il existe une petite proportion de très riches, ce qui "étire" la droite de la courbe. Cela donne un \textbf{skewness très élevé (2.89)}. Elle est extrêmement \textbf{leptokurtique} (\textbf{kurtosis = 20.78}) : un pic très aigu et une queue droite très lourde. Cela signifie qu'il y a un risque élevé de valeurs extrêmement grandes, ce qui la rend très utile pour modéliser des phénomènes avec de rares événements extrêmes.

\end{examplebox}


\subsection{Exercices}

% --- Définitions Fondamentales des Moments ---

\begin{exercicebox}[Exercice 1 : Associer les Moments]
Associez chaque description au moment correspondant :
\begin{enumerate}
    \item $E[X]$
    \item $\sqrt{E[(X-\mu)^2]}$
    \item $E[(X-\mu)^2]$
    \item $E[\left(\frac{X-\mu}{\sigma}\right)^3]$
    \item $E[\left(\frac{X-\mu}{\sigma}\right)^4] - 3$
\end{enumerate}

\textbf{Termes :} Variance, Asymétrie (Skewness), Espérance (Moyenne), Écart-type, Excès de Kurtosis.
\end{exercicebox}

\begin{exercicebox}[Exercice 2 : Calcul des Moments Centrés]
Soit $X$ une v.a. avec $\mu = 2$. La PMF de $X$ est :
$P(X=0)=0.2$, $P(X=2)=0.6$, $P(X=4)=0.2$.
\begin{enumerate}
    \item Vérifiez que $E[X] = 2$.
    \item Calculez le 2ème moment centré, $E[(X-\mu)^2]$ (la Variance).
    \item Calculez le 3ème moment centré, $E[(X-\mu)^3]$.
\end{enumerate}
\end{exercicebox}

\begin{exercicebox}[Exercice 3 : Calcul des Moments Standardisés]
En utilisant la v.a. $X$ de l'exercice 2 (avec $\mu=2$ et $\text{Var}(X) = 1.6$) :
\begin{enumerate}
    \item Calculez le Skewness, $\text{Skew}(X)$.
    \item Que pouvez-vous conclure sur la symétrie de cette distribution ?
\end{enumerate}
(Rappel : $\sigma = \sqrt{1.6} \approx 1.265$).
\end{exercicebox}

% --- Asymétrie (Skewness) ---

\begin{exercicebox}[Exercice 4 : Interprétation Visuelle du Skewness]
Pour chacune des distributions décrites, indiquez si le skewness est \textbf{positif (> 0)}, \textbf{négatif (< 0)} ou \textbf{nul (= 0)}.
\begin{enumerate}
    \item La distribution $\text{Exp}(\lambda)$ (queue longue à droite).
    \item Une distribution où Moyenne = 10, Médiane = 12, Mode = 13 (queue longue à gauche).
    \item La distribution $\mathcal{N}(\mu, \sigma^2)$.
    \item La distribution des revenus dans un pays (beaucoup de salaires bas, quelques salaires très élevés).
\end{enumerate}
\end{exercicebox}

\begin{exercicebox}[Exercice 5 : Skewness d'une Distribution Simple]
Soit $X$ une v.a. : $P(X=0)=0.1$, $P(X=1)=0.8$, $P(X=10)=0.1$.
\begin{enumerate}
    \item Calculez $\mu = E[X]$.
    \item Calculez $\sigma^2 = \text{Var}(X)$.
    \item Calculez $E[(X-\mu)^3]$.
    \item Le skewness est-il positif, négatif ou nul ? (Le calcul complet n'est pas nécessaire si vous justifiez).
\end{enumerate}
\end{exercicebox}

\begin{exercicebox}[Exercice 6 : Skewness d'une Variable de Bernoulli]
Soit $X \sim \text{Bern}(p)$. On rappelle que $\mu = p$ et $\sigma^2 = p(1-p)$.
\begin{enumerate}
    \item Calculez $E[X^3]$. (Indice : $X^3 = X$).
    \item Calculez le 3ème moment centré $E[(X-p)^3]$ en développant l'expression.
    \item Calculez le skewness $\text{Skew}(X) = \frac{E[(X-p)^3]}{\sigma^3}$.
    \item Pour quelle valeur de $p$ cette distribution est-elle symétrique ?
\end{enumerate}
\end{exercicebox}

% --- Symétrie ---

\begin{exercicebox}[Exercice 7 : Définition de la Symétrie]
Soit $X \sim \text{Unif}(-5, 5)$.
\begin{enumerate}
    \item Autour de quel point $\mu$ cette distribution est-elle symétrique ?
    \item Montrez que $f(x) = f(2\mu - x)$ en utilisant la PDF de la loi uniforme.
\end{enumerate}
\end{exercicebox}

\begin{exercicebox}[Exercice 8 : Symétrie et Moments Centrés]
Soit $X$ une v.a. symétrique autour de sa moyenne $\mu$.
Montrez que tous ses moments centrés d'ordre impair sont nuls, c'est-à-dire $E[(X-\mu)^k] = 0$ pour $k=1, 3, 5, \dots$
(Indice : Soit $Y = X-\mu$. $Y$ est symétrique autour de 0. Que vaut $E[Y^k]$ ?).
\end{exercicebox}

\begin{exercicebox}[Exercice 9 : Symétrie de la Loi Normale]
Soit $X \sim \mathcal{N}(\mu, \sigma^2)$.
\begin{enumerate}
    \item La distribution est-elle symétrique ?
    \item Que vaut $\text{Skew}(X)$ ? (Sans calcul, en utilisant le résultat de l'exercice 8).
\end{enumerate}
\end{exercicebox}

\begin{exercicebox}[Exercice 10 : Symétrie et Skewness]
Si $\text{Skew}(X) = 0$, peut-on affirmer que la distribution de $X$ est symétrique ?
(Indice : Pensez à une distribution qui n'est pas symétrique mais où les asymétries se compensent, ex: $P(X=-4)=0.1, P(X=-1)=0.4, P(X=2)=0.5$).
\end{exercicebox}

% --- Aplatissement (Kurtosis) ---

\begin{exercicebox}[Exercice 11 : Interprétation du Kurtosis]
Associez chaque type de distribution à sa description du kurtosis (excédentaire) :
\begin{enumerate}
    \item \textbf{Leptokurtique}
    \item \textbf{Mésokurtique}
    \item \textbf{Platykurtique}
\end{enumerate}

\textbf{Descriptions :}
A. Excès de Kurtosis = 0 (Référence de la loi normale).
B. Excès de Kurtosis < 0 (Distribution "plate" avec des queues fines).
C. Excès de Kurtosis > 0 (Distribution "pointue" avec des queues épaisses).
\end{exercicebox}

\begin{exercicebox}[Exercice 12 : Kurtosis et queues]
Une distribution A a un excès de kurtosis de 5. Une distribution B a un excès de kurtosis de 1.
Laquelle des deux distributions est la plus susceptible de produire des valeurs "extrêmes" (très loin de la moyenne) ?
\end{exercicebox}

\begin{exercicebox}[Exercice 13 : Kurtosis de la Loi Normale]
Soit $Z \sim \mathcal{N}(0, 1)$.
\begin{enumerate}
    \item Que vaut le 4ème moment standardisé $E[Z^4]$ ? (Indice : Pour une $\mathcal{N}(0,1)$, $E[Z^4]=3$).
    \item Que vaut l'excès de kurtosis de $Z$ ?
\end{enumerate}
\end{exercicebox}

\begin{exercicebox}[Exercice 14 : Kurtosis de la Loi Uniforme]
D'après les exemples du cours, la loi $\text{Unif}(a, b)$ a un excès de kurtosis de -1.2.
Est-elle leptokurtique, mésokurtique ou platykurtique ?
\end{exercicebox}

% --- Synthèse et Propriétés ---

\begin{exercicebox}[Exercice 15 : Invariance des Moments Standardisés]
Soit $X$ une v.a. avec $\mu=10$, $\sigma=2$, $\text{Skew}(X) = 0.5$ et $\text{Excess Kurtosis}(X) = 1$.
Soit $Y = 3X + 5$.
\begin{enumerate}
    \item Calculez $E[Y]$ et $\text{Var}(Y)$.
    \item Que vaut $\text{Skew}(Y)$ ?
    \item Que vaut $\text{Excess Kurtosis}(Y)$ ?
\end{enumerate}
(Indice : Les moments standardisés sont invariants par transformation linéaire $aX+b$ avec $a>0$).
\end{exercicebox}

\begin{exercicebox}[Exercice 16 : Moments d'une Distribution Inconnue]
Une v.a. $Z$ a été standardisée ($E[Z]=0, \text{Var}(Z)=1$).
On sait que $E[Z^3] = 0.8$ et $E[Z^4] = 4.5$.
\begin{enumerate}
    \item Calculez le skewness de $Z$.
    \item Calculez l'excès de kurtosis de $Z$.
\end{enumerate}
\end{exercicebox}

\begin{exercicebox}[Exercice 17 : Identifier la Distribution]
Une distribution $X$ est analysée. On trouve :
$\text{Skew}(X) = 0$ et $\text{Excess Kurtosis}(X) = 0$.
Quelle distribution célèbre partage ces deux propriétés ?
\end{exercicebox}

\begin{exercicebox}[Exercice 18 : Identifier la Distribution (2)]
Une distribution $Y$ est analysée. On trouve :
$\text{Skew}(Y) = 2.0$ et $\text{Excess Kurtosis}(Y) = 6.0$.
Quelle distribution vue en cours (et dans les exemples) correspond à ces valeurs ?
\end{exercicebox}

\begin{exercicebox}[Exercice 19 : Moments non centrés vs centrés]
Soit $X$ une v.a. avec $E[X] = \mu$.
Exprimez le 2ème moment centré $E[(X-\mu)^2]$ en fonction des moments non centrés $E[X^2]$ et $E[X]$. (C'est la formule de calcul de la variance).
\end{exercicebox}

\begin{exercicebox}[Exercice 20 : Moments centrés vs non centrés]
Soit $X$ une v.a. avec $E[X] = \mu$.
Exprimez le 3ème moment centré $E[(X-\mu)^3]$ en fonction des moments non centrés ($E[X^3]$, $E[X^2]$, $E[X]$).
(Indice : Développez $(X-\mu)^3 = X^3 - 3X^2\mu + 3X\mu^2 - \mu^3$ et prenez l'espérance).
\end{exercicebox}

\subsection{Corrections des Exercices}

% --- Corrections : Définitions Fondamentales des Moments ---

\begin{correctionbox}[Correction Exercice 1 : Associer les Moments]
1.  $E[X]$ $\rightarrow$ \textbf{Espérance (Moyenne)}
2.  $\sqrt{E[(X-\mu)^2]}$ $\rightarrow$ \textbf{Écart-type}
3.  $E[(X-\mu)^2]$ $\rightarrow$ \textbf{Variance}
4.  $E[\left(\frac{X-\mu}{\sigma}\right)^3]$ $\rightarrow$ \textbf{Asymétrie (Skewness)}
5.  $E[\left(\frac{X-\mu}{\sigma}\right)^4] - 3$ $\rightarrow$ \textbf{Excès de Kurtosis}
\end{correctionbox}

\begin{correctionbox}[Correction Exercice 2 : Calcul des Moments Centrés]
$P(X=0)=0.2$, $P(X=2)=0.6$, $P(X=4)=0.2$.
1.  $E[X] = (0)(0.2) + (2)(0.6) + (4)(0.2) = 0 + 1.2 + 0.8 = 2.0$. $\mu=2$.
2.  Variance = $E[(X-2)^2]$
    $= (0-2)^2(0.2) + (2-2)^2(0.6) + (4-2)^2(0.2)$
    $= (-2)^2(0.2) + (0)^2(0.6) + (2)^2(0.2)$
    $= 4(0.2) + 0 + 4(0.2) = 0.8 + 0.8 = 1.6$.
3.  3ème moment centré = $E[(X-2)^3]$
    $= (0-2)^3(0.2) + (2-2)^3(0.6) + (4-2)^3(0.2)$
    $= (-8)(0.2) + (0)(0.6) + (8)(0.2)$
    $= -1.6 + 0 + 1.6 = 0$.
\end{correctionbox}

\begin{correctionbox}[Correction Exercice 3 : Calcul des Moments Standardisés]
De l'exercice 2, $\mu=2$, $\text{Var}(X) = 1.6$ et $E[(X-\mu)^3] = 0$.
1.  $\text{Skew}(X) = \frac{E[(X-\mu)^3]}{\sigma^3} = \frac{0}{(1.6)^{3/2}} = 0$.
2.  Le skewness est nul. Cela indique que la distribution est symétrique, ce que l'on peut vérifier (les probabilités $P(\mu-2k)$ et $P(\mu+2k)$ sont égales).
\end{correctionbox}

% --- Corrections : Asymétrie (Skewness) ---

\begin{correctionbox}[Correction Exercice 4 : Interprétation Visuelle du Skewness]
1.  \textbf{Positif (> 0)}. La loi exponentielle a une queue longue à droite.
2.  \textbf{Négatif (< 0)}. L'ordre $\text{Moyenne} < \text{Médiane} < \text{Mode}$ est caractéristique d'une queue à gauche.
3.  \textbf{Nul (= 0)}. La loi normale est parfaitement symétrique.
4.  \textbf{Positif (> 0)}. La grande majorité des gens a un revenu bas/moyen, et une petite minorité a un revenu très élevé, créant une queue longue à droite.
\end{correctionbox}

\begin{correctionbox}[Correction Exercice 5 : Skewness d'une Distribution Simple]
$P(X=0)=0.1$, $P(X=1)=0.8$, $P(X=10)=0.1$.
1.  $\mu = E[X] = (0)(0.1) + (1)(0.8) + (10)(0.1) = 0 + 0.8 + 1.0 = 1.8$.
2.  $E[X^2] = (0^2)(0.1) + (1^2)(0.8) + (10^2)(0.1) = 0 + 0.8 + 10 = 10.8$.
    $\sigma^2 = E[X^2] - \mu^2 = 10.8 - (1.8)^2 = 10.8 - 3.24 = 7.56$.
3.  $E[(X-\mu)^3] = (0-1.8)^3(0.1) + (1-1.8)^3(0.8) + (10-1.8)^3(0.1)$
    $= (-5.832)(0.1) + (-0.512)(0.8) + (551.368)(0.1)$
    $= -0.5832 - 0.4096 + 55.1368 = 54.144$.
4.  Puisque $E[(X-\mu)^3] = 54.144 > 0$, le skewness est \textbf{positif}. Cela est dû à la valeur extrême $X=10$ qui tire la distribution vers la droite.
\end{correctionbox}

\begin{correctionbox}[Correction Exercice 6 : Skewness d'une Variable de Bernoulli]
$X \sim \text{Bern}(p)$, $\mu=p$, $\sigma^2=p(1-p)$.
1.  $X$ ne vaut que 0 ou 1, donc $X^3 = X$. $E[X^3] = E[X] = p$.
2.  $E[(X-p)^3] = E[X^3 - 3X^2p + 3Xp^2 - p^3]$
    $= E[X^3] - 3pE[X^2] + 3p^2E[X] - p^3$
    (Car $X^2=X$ et $X^3=X$)
    $= E[X] - 3pE[X] + 3p^2E[X] - p^3$
    $= p - 3p(p) + 3p^2(p) - p^3 = p - 3p^2 + 3p^3 - p^3 = p - 3p^2 + 2p^3$
    $= p(1 - 3p + 2p^2) = p(1-p)(1-2p)$.
3.  $\text{Skew}(X) = \frac{E[(X-p)^3]}{\sigma^3} = \frac{p(1-p)(1-2p)}{[p(1-p)]^{3/2}} = \frac{1-2p}{\sqrt{p(1-p)}}$.
4.  La distribution est symétrique si $\text{Skew}(X) = 0$. Cela se produit si $1-2p = 0$, donc $p=0.5$.
\end{correctionbox}

% --- Corrections : Symétrie ---

\begin{correctionbox}[Correction Exercice 7 : Définition de la Symétrie]
$X \sim \text{Unif}(-5, 5)$.
1.  La moyenne est $\mu = (-5+5)/2 = 0$. La distribution est symétrique autour de $\mu=0$.
2.  On doit montrer $f(x) = f(2\mu - x) = f(-x)$.
    La PDF est $f(x) = 1/10$ si $x \in [-5, 5]$ et $0$ sinon.
    - Si $x \in [-5, 5]$, alors $-x \in [-5, 5]$. $f(x) = 1/10$ et $f(-x) = 1/10$. Ils sont égaux.
    - Si $x > 5$, $f(x) = 0$. Alors $-x < -5$, $f(-x) = 0$. Ils sont égaux.
    La condition est vérifiée pour tout $x$.
\end{correctionbox}

\begin{correctionbox}[Correction Exercice 8 : Symétrie et Moments Centrés]
Soit $Y = X-\mu$. Si $X$ est symétrique autour de $\mu$, $Y$ est symétrique autour de 0. Sa PDF $f_Y(y)$ est paire : $f_Y(y) = f_Y(-y)$.
On veut calculer $E[Y^k] = \int_{-\infty}^{\infty} y^k f_Y(y) dy$ pour $k$ impair.
La fonction $y^k$ est impaire (car $k$ est impair).
La fonction $f_Y(y)$ est paire.
Le produit d'une fonction impaire et d'une fonction paire est une fonction impaire.
L'intégrale d'une fonction impaire sur un intervalle symétrique $(-\infty, \infty)$ est 0.
Donc, $E[(X-\mu)^k] = E[Y^k] = 0$ pour $k=1, 3, 5, \dots$.
\end{correctionbox}

\begin{correctionbox}[Correction Exercice 9 : Symétrie de la Loi Normale]
1.  Oui, la PDF de la loi normale est parfaitement symétrique autour de sa moyenne $\mu$.
2.  Le Skewness est basé sur le 3ème moment centré ($k=3$, qui est impair). D'après l'exercice 8, tous les moments centrés impairs d'une distribution symétrique sont nuls.
    Donc $\text{Skew}(X) = 0$.
\end{correctionbox}

\begin{correctionbox}[Correction Exercice 10 : Symétrie et Skewness]
Non. $\text{Skew}(X)=0$ est une condition nécessaire mais non suffisante pour la symétrie. Une distribution peut avoir un skewness nul tout en n'étant pas symétrique, si les asymétries de gauche et de droite "s'annulent" dans le calcul du 3ème moment.
\end{correctionbox}

% --- Corrections : Aplatissement (Kurtosis) ---

\begin{correctionbox}[Correction Exercice 11 : Interprétation du Kurtosis]
1.  \textbf{Leptokurtique} $\rightarrow$ \textbf{C.} Excès de Kurtosis > 0 (Pointue, queues épaisses).
2.  \textbf{Mésokurtique} $\rightarrow$ \textbf{A.} Excès de Kurtosis = 0 (Référence normale).
3.  \textbf{Platykurtique} $\rightarrow$ \textbf{B.} Excès de Kurtosis < 0 (Plate, queues fines).
\end{correctionbox}

\begin{correctionbox}[Correction Exercice 12 : Kurtosis et queues]
Un excès de kurtosis plus élevé signifie des queues plus épaisses.
La \textbf{Distribution A} (kurtosis=5) est beaucoup plus susceptible de produire des valeurs extrêmes que la Distribution B (kurtosis=1).
\end{correctionbox}

\begin{correctionbox}[Correction Exercice 13 : Kurtosis de la Loi Normale]
$Z \sim \mathcal{N}(0, 1)$.
1.  $E[Z^4]$ est le 4ème moment standardisé. Par définition, c'est le kurtosis (non excessif). Pour la loi normale, $\text{Kurtosis}(Z) = 3$. Donc $E[Z^4] = 3$.
2.  Excès de Kurtosis = $\text{Kurtosis}(Z) - 3 = 3 - 3 = 0$.
\end{correctionbox}

\begin{correctionbox}[Correction Exercice 14 : Kurtosis de la Loi Uniforme]
L'excès de kurtosis est -1.2, ce qui est négatif.
La distribution uniforme est \textbf{platykurtique} (plus "plate" que la normale).
\end{correctionbox}

% --- Corrections : Synthèse et Propriétés ---

\begin{correctionbox}[Correction Exercice 15 : Invariance des Moments Standardisés]
$Y = 3X + 5$.
1.  $E[Y] = 3E[X] + 5 = 3(10) + 5 = 35$.
    $\text{Var}(Y) = 3^2 \text{Var}(X) = 9 (2^2) = 36$.
2.  Le skewness est invariant par transformation linéaire affine (si $a>0$). $\text{Skew}(Y) = \text{Skew}(X) = 0.5$.
3.  L'excès de kurtosis est invariant par transformation linéaire affine. $\text{Excess Kurtosis}(Y) = \text{Excess Kurtosis}(X) = 1$.
\end{correctionbox}

\begin{correctionbox}[Correction Exercice 16 : Moments d'une Distribution Inconnue]
$Z$ est déjà standardisée ($\mu=0, \sigma=1$).
1.  $\text{Skew}(Z) = E\left[ \left( \frac{Z - 0}{1} \right)^3 \right] = E[Z^3] = 0.8$.
2.  $\text{Excess Kurtosis}(Z) = E\left[ \left( \frac{Z - 0}{1} \right)^4 \right] - 3 = E[Z^4] - 3 = 4.5 - 3 = 1.5$.
\end{correctionbox}

\begin{correctionbox}[Correction Exercice 17 : Identifier la Distribution]
La \textbf{Loi Normale} $\mathcal{N}(\mu, \sigma^2)$ est la distribution de référence qui est symétrique ($\text{Skew}=0$) et mésokurtique ($\text{Excess Kurtosis}=0$).
\end{correctionbox}

\begin{correctionbox}[Correction Exercice 18 : Identifier la Distribution (2)]
La \textbf{Loi Exponentielle} $\text{Exp}(\lambda)$ (pour n'importe quel $\lambda$) a un skewness de 2.0 et un excès de kurtosis de 6.0.
\end{correctionbox}

\begin{correctionbox}[Correction Exercice 19 : Moments non centrés vs centrés]
C'est la dérivation de la formule de calcul de la variance.
$E[(X-\mu)^2] = E[X^2 - 2X\mu + \mu^2]$
$= E[X^2] - E[2X\mu] + E[\mu^2]$ (par linéarité)
$= E[X^2] - 2\mu E[X] + \mu^2$ (car $\mu$ est une constante)
$= E[X^2] - 2\mu(\mu) + \mu^2 = E[X^2] - 2\mu^2 + \mu^2$
$= E[X^2] - \mu^2 = E[X^2] - (E[X])^2$.
\end{correctionbox}

\begin{correctionbox}[Correction Exercice 20 : Moments centrés vs non centrés]
On développe et on utilise la linéarité de l'espérance, en se rappelant que $\mu=E[X]$ est une constante.
$E[(X-\mu)^3] = E[X^3 - 3X^2\mu + 3X\mu^2 - \mu^3]$
$= E[X^3] - E[3X^2\mu] + E[3X\mu^2] - E[\mu^3]$
$= E[X^3] - 3\mu E[X^2] + 3\mu^2 E[X] - \mu^3$
En remplaçant $E[X]$ par $\mu$ :
$= E[X^3] - 3\mu E[X^2] + 3\mu^2(\mu) - \mu^3$
$= E[X^3] - 3\mu E[X^2] + 3\mu^3 - \mu^3$
$= E[X^3] - 3E[X]E[X^2] + 2(E[X])^3$.
\end{correctionbox}

\subsection{Exercices Pratiques (Python)}

Dans ce chapitre, nous avons vu que le 3ème moment (skewness) et le 4ème moment (kurtosis) décrivent la forme d'une distribution. En finance, ces mesures sont cruciales pour évaluer le risque.

Un rendement avec un \textbf{skewness négatif} signifie que les pertes extrêmes sont plus probables que les gains extrêmes. Un \textbf{excès de kurtosis positif} (leptokurtique) signifie que les "queues" de la distribution sont épaisses, indiquant qu'il y a une probabilité plus élevée d'événements extrêmes (gains ou pertes majeurs) que ne le prédit une loi normale.

Nous allons calculer ces moments pour des actifs financiers réels.

\begin{codecell}
# Cellule d'installation et d'importation
pip install numpy pandas yfinance scipy
\end{codecell}

\begin{codecell}
import numpy as np
import pandas as pd
import yfinance as yf
from scipy import stats
\end{codecell}

\begin{exercicebox}[Exercice 1 : Calcul Manuel des Moments]
Calculons les 4 moments des rendements quotidiens de l'action Tesla (TSLA), connue pour sa volatilité.

\textbf{Votre tâche :}
\begin{enumerate}
    \item Télécharger 5 ans de données pour TSLA.
    \item Calculer les rendements logarithmiques quotidiens $X$.
    \item Calculer la moyenne $\mu = E[X]$ et l'écart-type $\sigma$.
    \item Standardiser les rendements : $Z = (X - \mu) / \sigma$.
    \item Calculer le Skewness (3ème moment standardisé) : $E[Z^3]$.
    \item Calculer l'Excès de Kurtosis (4ème moment standardisé - 3) : $E[Z^4] - 3$.
\end{enumerate}

\begin{codecell}
# 1. Telecharger les donnees
ticker = 'TSLA'
data = yf.download(ticker, period='5y')

# 2. Calculer les rendements log
log_returns = np.log(data['Close'] / data['Close'].shift(1)).dropna()

# 3. Calculer mu et sigma
# mu = ...
# sigma = ...

# 4. Standardiser les rendements
# z_scores = ...

# 5. Calculer le Skewness
# Indice : (z_scores**3).mean()
# skewness_manuel = ...

# 6. Calculer l'Exces de Kurtosis
# Indice : (z_scores**4).mean() - 3
# kurtosis_manuel = ...

# print(f"--- Calculs Manuels pour {ticker} ---")
# print(f"Skewness: {skewness_manuel:.4f}")
# print(f"Exces de Kurtosis: {kurtosis_manuel:.4f}")
\end{codecell}
\end{exercicebox}

\begin{exercicebox}[Exercice 2 : Vérification avec SciPy]
Vérifions nos calculs manuels de l'exercice 1 en utilisant les fonctions optimisées de la bibliothèque scipy.stats.

\textbf{Votre tâche :}
\begin{enumerate}
    \item Utiliser stats.skew(log\_returns) pour calculer le skewness.
    \item Utiliser stats.kurtosis(log\_returns) pour calculer l'excès de kurtosis. (Note : cette fonction calcule l'excès de kurtosis par défaut, en soustrayant 3).
    \item Afficher et comparer les résultats avec ceux de l'exercice 1.
\end{enumerate}

\begin{codecell}
from scipy import stats

# log_returns de l'exercice 1

# 1. Calculer le skewness avec scipy
# skew_scipy = ...

# 2. Calculer l'exces de kurtosis avec scipy
# kurtosis_scipy = ...

# print(f"--- Verification avec SciPy pour {ticker} ---")
# print(f"Skewness: {skew_scipy:.4f}")
# print(f"Exces de Kurtosis: {kurtosis_scipy:.4f}")
\end{codecell}
\end{exercicebox}

\begin{exercicebox}[Exercice 3 : Interprétation des Moments]
Cet exercice est purement théorique et ne nécessite pas de code. Répondez en vous basant sur les résultats (probablement) obtenus aux exercices 1 et 2 pour TSLA.

\begin{enumerate}
    \item Le skewness calculé est-il positif, négatif ou proche de zéro ? Que cela implique-t-il sur la probabilité des gains quotidiens extrêmes par rapport aux pertes quotidiennes extrêmes ?
    \item L'excès de kurtosis est-il positif, négatif ou proche de zéro ? La distribution est-elle leptokurtique, mésokurtique ou platykurtique ?
    \item Que signifie cette valeur de kurtosis pour un investisseur en termes de risque, comparé à une distribution normale ?
\end{enumerate}
\end{exercicebox}

\begin{exercicebox}[Exercice 4 : Comparaison de Distributions]
Comparons les moments de TSLA (actif volatil) à ceux d'un indice large comme le S\&P 500 (GSPC) (actif plus stable).

\textbf{Votre tâche :}
\begin{enumerate}
    \item Télécharger 5 ans de données pour GSPC.
    \item Calculer ses rendements logarithmiques (log\_returns\_sp500).
    \item Calculer le skewness et l'excès de kurtosis pour GSPC en utilisant scipy.stats
    \item Comparer les kurtosis de TSLA et GSPC. Lequel a les "queues les plus épaisses" (fatter tails) ?
\end{enumerate}

\begin{codecell}
# 1. Telecharger les donnees pour S&P 500
ticker_sp500 = 'GSPC'
# data_sp500 = ...

# 2. Calculer les rendements log
# log_returns_sp500 = ...
log_returns_sp500 = log_returns_sp500.dropna()

# 3. Calculer skew et kurtosis pour S&P 500
# skew_sp500 = ...
# kurtosis_sp500 = ...

# print(f"--- Comparaison des Moments (5 ans) ---")
# print(f"TSLA Skew: {skew_scipy:.4f} | Kurtosis: {kurtosis_scipy:.4f}")
# print(f"SP500 Skew: {skew_sp500:.4f} | Kurtosis: {kurtosis_sp500:.4f}")
\end{codecell}
\end{exercicebox}
\subsection{Exercices}

% --- Section 1 : Définitions, Skewness et Kurtosis ---

\begin{exercicebox}[Exercice 1 : Concepts (Moments)]
\begin{enumerate}
    \item À quoi correspond le 1er moment non centré, $E[X^1]$ ?
    \item À quoi correspond le 2ème moment centré, $E[(X-\mu)^2]$ ?
\end{enumerate}
\end{exercicebox}

\begin{exercicebox}[Exercice 2 : Interprétation (Skewness)]
Une distribution des salaires dans une entreprise a un skewness de +2.5. Qu'est-ce que cela signifie concrètement sur la répartition des salaires ?
\end{exercicebox}

\begin{exercicebox}[Exercice 3 : Interprétation (Skewness)]
Si une distribution est parfaitement symétrique, que vaut son skewness ?
\end{exercicebox}

\begin{exercicebox}[Exercice 4 : Interprétation (Kurtosis)]
Une distribution des rendements d'un actif financier a un "excess kurtosis" de +5.0.
\begin{enumerate}
    \item Comment appelle-t-on ce type de distribution ?
    \item Qu'est-ce que cela implique sur la probabilité des "krachs" (événements extrêmes) par rapport à une loi normale ?
\end{enumerate}
\end{exercicebox}

\begin{exercicebox}[Exercice 5 : Interprétation (Kurtosis)]
Une distribution a un "excess kurtosis" de -1.0.
\begin{enumerate}
    \item Comment appelle-t-on ce type de distribution ?
    \item Comment décririez-vous visuellement son "pic" et ses "queues" par rapport à une loi normale ?
\end{enumerate}
\end{exercicebox}

\begin{exercicebox}[Exercice 6 : Comparaison (Skewness)]
La distribution exponentielle est-elle symétrique, asymétrique à gauche ou asymétrique à droite ? Quel est le signe de son skewness ?
\end{exercicebox}

\begin{exercicebox}[Exercice 7 : Comparaison (Kurtosis)]
La distribution uniforme est-elle mésokurtique, leptokurtique ou platykurtique ? Son "excess kurtosis" est-il positif, négatif ou nul ?
\end{exercicebox}

% --- Section 2 : Symétrie ---

\begin{exercicebox}[Exercice 8 : Propriété de Symétrie]
Soit $X$ une variable aléatoire symétrique autour de sa moyenne $\mu$.
Montrez que son 3ème moment centré, $E[(X-\mu)^3]$, est nul (et donc que son skewness est nul).
\end{exercicebox}

\begin{exercicebox}[Exercice 9 : Vérification de Symétrie (PDF)]
La PDF d'une loi $X$ est $f(x) = \frac{1}{2}e^{-|x-3|}$.
Cette distribution est-elle symétrique ? Si oui, autour de quel point $\mu$ ?
\end{exercicebox}

\begin{exercicebox}[Exercice 10 : Symétrie et Moments]
Si une distribution a un skewness de 0, est-elle forcément symétrique ? (Indice : Pensez à une distribution qui aurait des queues asymétriques mais qui s'annuleraient pour le moment d'ordre 3).
\end{exercicebox}

% --- Section 3 : Moments d'Échantillon vs Population ---

\begin{exercicebox}[Exercice 11 : Définitions (Échantillon vs Pop.)]
Quelle est la différence conceptuelle entre $\sigma^2$ et $s^2$ ?
\end{exercicebox}

\begin{exercicebox}[Exercice 12 : Correction de Bessel ($n-1$)]
Pourquoi utilise-t-on $n-1$ au dénominateur pour $s^2$ ? Que se passerait-il si nous avions un échantillon de $n=1$ et que nous divisions par $n$ ?
\end{exercicebox}

\begin{exercicebox}[Exercice 13 : Calcul (Moments d'Échantillon)]
On observe l'échantillon de données suivant : $\{ 2, 3, 10 \}$.
\begin{enumerate}
    \item Calculez la moyenne d'échantillon $\bar{X}$.
    \item Calculez la variance d'échantillon non biaisée $s^2$.
\end{enumerate}
\end{exercicebox}

% --- Section 4 : Fonctions Génératrices (MGF) - Calculs ---

\begin{exercicebox}[Exercice 14 : MGF (Propriété de base)]
Soit $M_X(t)$ la MGF d'une variable $X$. Que vaut $M_X(0)$ ?
\end{exercicebox}

\begin{exercicebox}[Exercice 15 : MGF (Série de Taylor)]
Le développement en série de Taylor d'une MGF est $M_X(t) = 1 + 5t + 14t^2 + O(t^3)$.
(Rappel : $M_X(t) = 1 + E[X]t + E[X^2]\frac{t^2}{2!} + \dots$)
\begin{enumerate}
    \item Trouvez $E[X]$.
    \item Trouvez $E[X^2]$.
    \item Calculez $\text{Var}(X)$.
\end{enumerate}
\end{exercicebox}

\begin{exercicebox}[Exercice 16 : MGF (Calcul de moments)]
La MGF de la loi Exponentielle de paramètre $\lambda$ est $M_X(t) = \frac{\lambda}{\lambda - t}$.
\begin{enumerate}
    \item Calculez $M_X'(t)$ et trouvez $E[X] = M_X'(0)$.
    \item Calculez $M_X''(t)$ et trouvez $E[X^2] = M_X''(0)$.
\end{enumerate}
\end{exercicebox}

\begin{exercicebox}[Exercice 17 : MGF (Variance de l'Exponentielle)]
En utilisant les résultats de l'exercice 16, déduisez $\text{Var}(X)$ pour une loi Exponentielle($\lambda$).
\end{exercicebox}

\begin{exercicebox}[Exercice 18 : MGF (Loi Normale)]
La MGF de $X \sim \mathcal{N}(\mu, \sigma^2)$ est $M_X(t) = \exp(\mu t + \frac{1}{2}\sigma^2 t^2)$.
Calculez $M_X'(t)$ et vérifiez que $M_X'(0) = \mu$.
\end{exercicebox}

\begin{exercicebox}[Exercice 19 : MGF (Loi Normale)]
En utilisant la MGF de l'exercice 18, calculez $M_X''(t)$ et montrez que $E[X^2] = \mu^2 + \sigma^2$.
\end{exercicebox}

\begin{exercicebox}[Exercice 20 : MGF (Loi Normale)]
En utilisant les résultats des exercices 18 et 19, retrouvez la formule de la variance $\text{Var}(X)$.
\end{exercicebox}

% --- Section 5 : MGF - Sommes de Variables Indépendantes ---

\begin{exercicebox}[Exercice 21 : Propriété des Sommes]
Soient $X$ et $Y$ deux variables aléatoires \textbf{indépendantes}. Soit $S = X+Y$.
Comment la MGF de $S$, $M_S(t)$, est-elle liée à $M_X(t)$ et $M_Y(t)$ ?
\end{exercicebox}

\begin{exercicebox}[Exercice 22 : Application (Somme de Poissons)]
$X \sim \text{Poisson}(\lambda_1)$ a pour MGF $M_X(t) = e^{\lambda_1(e^t - 1)}$.
$Y \sim \text{Poisson}(\lambda_2)$ a pour MGF $M_Y(t) = e^{\lambda_2(e^t - 1)}$.
$X$ et $Y$ sont indépendantes.
\begin{enumerate}
    \item Trouvez la MGF de $S = X+Y$.
    \item En reconnaissant la forme de la MGF de $S$, quelle est la loi de $S$ ?
\end{enumerate}
\end{exercicebox}

\begin{exercicebox}[Exercice 23 : Application (Somme de Binomiales)]
La MGF de $X \sim \text{Bin}(n, p)$ est $M_X(t) = (p e^t + (1-p))^n$.
Soit $Y \sim \text{Bin}(m, p)$ (même $p$), indépendante de $X$.
Quelle est la loi de $S = X+Y$ ? Justifiez avec les MGF.
\end{exercicebox}

\begin{exercicebox}[Exercice 24 : Application (Transformation Linéaire)]
Soit $M_X(t)$ la MGF de $X$. Soit $Y = aX + b$.
Exprimez $M_Y(t)$ en fonction de $M_X(t)$.
(Indice : $M_Y(t) = E[e^{t(aX+b)}]$).
\end{exercicebox}

\begin{exercicebox}[Exercice 25 : Application (Moyenne d'Échantillon)]
Soient $X_1, \dots, X_n$ des v.a. i.i.d. (indépendantes et identiquement distribuées) avec la MGF $M_X(t)$.
Soit $\bar{X} = \frac{1}{n} \sum X_i$ la moyenne d'échantillon.
Exprimez la MGF de $\bar{X}$ en fonction de $M_X(t)$.
(Indice : utilisez les résultats des exercices 21 et 24).
\end{exercicebox}


\subsection{Corrections des Exercices}

\begin{correctionbox}[Correction Exercice 1 : Concepts (Moments)]
1.  Le 1er moment non centré $E[X^1]$ est l'espérance $\mu$.
2.  Le 2ème moment centré $E[(X-\mu)^2]$ est la variance $\sigma^2$.
\end{correctionbox}

\begin{correctionbox}[Correction Exercice 2 : Interprétation (Skewness)]
Un skewness de +2.5 est fortement positif. Cela signifie que la distribution des salaires est \textbf{asymétrique à droite}.
Concrètement : la grande majorité des employés a un salaire regroupé (autour de la médiane), mais il existe une "longue queue" de quelques individus avec des salaires très élevés. Ces valeurs extrêmes "tirent" la moyenne vers la droite (Moyenne > Médiane).
\end{correctionbox}

\begin{correctionbox}[Correction Exercice 3 : Interprétation (Skewness)]
Si une distribution est parfaitement symétrique (comme la loi normale ou la loi uniforme), les écarts positifs et négatifs à la moyenne s'annulent parfaitement. Le skewness est \textbf{nul (0)}.
\end{correctionbox}

\begin{correctionbox}[Correction Exercice 4 : Interprétation (Kurtosis)]
1.  Un excess kurtosis > 0 est dit \textbf{leptokurtique}.
2.  Cela signifie que la distribution a des "queues plus épaisses" que la loi normale. La probabilité d'événements extrêmes (très grands gains ou très grandes pertes, comme un "krach") est \textbf{plus élevée} que ce qu'un modèle normal ne prédirait.
\end{correctionbox}

\begin{correctionbox}[Correction Exercice 5 : Interprétation (Kurtosis)]
1.  Un excess kurtosis < 0 est dit \textbf{platykurtique}.
2.  Visuellement, par rapport à une loi normale, la distribution est plus "aplatie" ou "carrée". Elle a un pic central moins prononcé et des queues plus "légères" (les événements extrêmes sont moins probables). La distribution uniforme est un exemple classique.
\end{correctionbox}

\begin{correctionbox}[Correction Exercice 6 : Comparaison (Skewness)]
La distribution exponentielle (ex: temps d'attente) a un pic à 0 et une longue queue vers les grandes valeurs. Elle est \textbf{asymétrique à droite}, et son skewness est \textbf{positif} (Skew = 2).
\end{correctionbox}

\begin{correctionbox}[Correction Exercice 7 : Comparaison (Kurtosis)]
La distribution uniforme est "plate" et n'a pas de pic central ni de queues épaisses. Elle est \textbf{platykurtique}. Son "excess kurtosis" est \textbf{négatif} (Kurtosis Excessif = -1.2).
\end{correctionbox}

\begin{correctionbox}[Correction Exercice 8 : Propriété de Symétrie]
Soit $Y = X - \mu$. La symétrie implique que $Y$ et $-Y$ (qui est $\mu - X$) ont la même distribution.
$E[(X-\mu)^3] = E[Y^3]$.
Puisque $Y$ et $-Y$ ont la même distribution, ils ont les mêmes moments : $E[Y^k] = E[(-Y)^k]$ pour tout $k$.
Pour $k=3$ :
$E[Y^3] = E[(-Y)^3] = E[(-1)^3 Y^3] = E[-Y^3] = -E[Y^3]$.
La seule valeur qui est égale à son opposé est 0.
$E[Y^3] = -E[Y^3] \implies 2 E[Y^3] = 0 \implies E[Y^3] = 0$.
Donc, $E[(X-\mu)^3] = 0$.
\end{correctionbox}

\begin{correctionbox}[Correction Exercice 9 : Vérification de Symétrie (PDF)]
On teste la condition $f(x) = f(2\mu - x)$. Le candidat pour $\mu$ est 3.
$f(2\mu - x) = f(2(3) - x) = f(6 - x)$.
$f(6 - x) = \frac{1}{2}e^{-|(6-x)-3|} = \frac{1}{2}e^{-|3-x|}$.
Puisque $|3-x| = |-(x-3)| = |x-3|$, on a $f(6-x) = \frac{1}{2}e^{-|x-3|} = f(x)$.
Oui, la distribution est symétrique autour de $\mu=3$.
\end{correctionbox}

\begin{correctionbox}[Correction Exercice 10 : Symétrie et Moments]
Non. Le Skewness nul est une condition nécessaire pour la symétrie, but pas suffisante.
On peut construire des distributions (assez exotiques) qui sont non symétriques, mais où les asymétries se compensent d'une manière qui annule le 3ème moment, résultant en un skewness de 0. Cependant, dans la plupart des cas pratiques, Skewness = 0 est un très bon indicateur de symétrie.
\end{correctionbox}

\begin{correctionbox}[Correction Exercice 11 : Définitions (Échantillon vs Pop.)]
$\sigma^2$ (variance de population) est un \textbf{paramètre} théorique. C'est la "vraie" variance de l'ensemble de la population (souvent inconnue).
$s^2$ (variance d'échantillon) est une \textbf{statistique}. C'est une \textit{estimation} de $\sigma^2$ calculée à partir d'un sous-ensemble de données (l'échantillon).
\end{correctionbox}

\begin{correctionbox}[Correction Exercice 12 : Correction de Bessel ($n-1$)]
Si $n=1$ (un seul échantillon $X_1$), la moyenne d'échantillon est $\bar{X} = X_1$.
La somme des carrés des écarts est $\sum (X_i - \bar{X})^2 = (X_1 - X_1)^2 = 0$.
Si on divisait par $n=1$, on obtiendrait $s^2 = 0/1 = 0$. Cela estimerait faussement que la population n'a pas de variance, ce qui est absurde.
Diviser par $n-1$ (donc $1-1=0$) donne $0/0$, une forme indéfinie, ce qui nous dit correctement qu'on ne peut pas estimer une dispersion à partir d'un seul point.
\end{correctionbox}

\begin{correctionbox}[Correction Exercice 13 : Calcul (Moments d'Échantillon)]
Échantillon : $\{ 2, 3, 10 \}$. $n=3$.
1.  $\bar{X} = \frac{1}{3} (2 + 3 + 10) = \frac{15}{3} = 5$.
2.  $s^2 = \frac{1}{n-1} \sum (X_i - \bar{X})^2 = \frac{1}{2} \left[ (2-5)^2 + (3-5)^2 + (10-5)^2 \right]$
    $s^2 = \frac{1}{2} \left[ (-3)^2 + (-2)^2 + (5)^2 \right]$
    $s^2 = \frac{1}{2} [ 9 + 4 + 25 ] = \frac{38}{2} = 19$.
\end{correctionbox}

\begin{correctionbox}[Correction Exercice 14 : MGF (Propriété de base)]
$M_X(t) = E[e^{tX}]$.
En $t=0$, $M_X(0) = E[e^{0 \cdot X}] = E[e^0] = E[1] = 1$.
Toute MGF valide doit valoir 1 en $t=0$.
\end{correctionbox}

\begin{correctionbox}[Correction Exercice 15 : MGF (Série de Taylor)]
On identifie les coefficients de la série $M_X(t) = 1 + \frac{E[X]}{1!}t + \frac{E[X^2]}{2!}t^2 + \dots$
1.  Le coefficient de $t$ est $E[X]$. On lit $5t$. Donc $E[X] = 5$.
2.  Le coefficient de $t^2$ est $E[X^2]/2!$. On lit $14t^2$.
    $E[X^2] / 2 = 14 \implies E[X^2] = 28$.
3.  $\text{Var}(X) = E[X^2] - (E[X])^2 = 28 - (5)^2 = 28 - 25 = 3$.
\end{correctionbox}

\begin{correctionbox}[Correction Exercice 16 : MGF (Calcul de moments)]
$M_X(t) = \lambda (\lambda - t)^{-1}$.
1.  $M_X'(t) = \lambda \cdot (-1) (\lambda - t)^{-2} \cdot (-1) = \lambda (\lambda - t)^{-2}$.
    $E[X] = M_X'(0) = \lambda (\lambda - 0)^{-2} = \lambda (\lambda^{-2}) = 1/\lambda$.
2.  $M_X''(t) = \lambda \cdot (-2) (\lambda - t)^{-3} \cdot (-1) = 2\lambda (\lambda - t)^{-3}$.
    $E[X^2] = M_X''(0) = 2\lambda (\lambda - 0)^{-3} = 2\lambda (\lambda^{-3}) = 2/\lambda^2$.
\end{correctionbox}

\begin{correctionbox}[Correction Exercice 17 : MGF (Variance de l'Exponentielle)]
$\text{Var}(X) = E[X^2] - (E[X])^2$
$\text{Var}(X) = (2/\lambda^2) - (1/\lambda)^2 = 2/\lambda^2 - 1/\lambda^2 = 1/\lambda^2$.
\end{correctionbox}

\begin{correctionbox}[Correction Exercice 18 : MGF (Loi Normale)]
$M_X(t) = \exp(\mu t + \frac{1}{2}\sigma^2 t^2)$.
On utilise la règle de la chaîne : $(\exp(u))' = \exp(u) \cdot u'$.
$M_X'(t) = \exp(\mu t + \frac{1}{2}\sigma^2 t^2) \cdot \frac{d}{dt}(\mu t + \frac{1}{2}\sigma^2 t^2)$
$M_X'(t) = \exp(\mu t + \frac{1}{2}\sigma^2 t^2) \cdot (\mu + \sigma^2 t)$.
$E[X] = M_X'(0) = \exp(0) \cdot (\mu + 0) = 1 \cdot \mu = \mu$.
\end{correctionbox}

\begin{correctionbox}[Correction Exercice 19 : MGF (Loi Normale)]
On dérive $M_X'(t)$ (règle du produit $u \cdot v$) :
$u = \exp(\mu t + \frac{1}{2}\sigma^2 t^2) \implies u' = u \cdot (\mu + \sigma^2 t)$
$v = (\mu + \sigma^2 t) \implies v' = \sigma^2$
$M_X''(t) = u'v + uv'$
$M_X''(t) = [\exp(\dots)(\mu + \sigma^2 t)] \cdot (\mu + \sigma^2 t) + [\exp(\dots)] \cdot (\sigma^2)$
On évalue en $t=0$ :
$E[X^2] = M_X''(0) = [\exp(0)(\mu)] \cdot (\mu) + [\exp(0)] \cdot (\sigma^2)$
$E[X^2] = (1 \cdot \mu) \cdot \mu + 1 \cdot \sigma^2 = \mu^2 + \sigma^2$.
\end{correctionbox}

\begin{correctionbox}[Correction Exercice 20 : MGF (Loi Normale)]
$\text{Var}(X) = E[X^2] - (E[X])^2$
$\text{Var}(X) = (\mu^2 + \sigma^2) - (\mu)^2 = \sigma^2$.
La MGF confirme bien que le paramètre $\sigma^2$ est la variance.
\end{correctionbox}

\begin{correctionbox}[Correction Exercice 21 : Propriété des Sommes]
Si $X$ et $Y$ sont indépendantes, la MGF de la somme est le \textbf{produit} des MGF :
$M_{X+Y}(t) = M_X(t) \cdot M_Y(t)$.
\end{correctionbox}

\begin{correctionbox}[Correction Exercice 22 : Application (Somme de Poissons)]
1.  $M_S(t) = M_X(t) \cdot M_Y(t) = e^{\lambda_1(e^t - 1)} \cdot e^{\lambda_2(e^t - 1)}$
    $M_S(t) = e^{\lambda_1(e^t - 1) + \lambda_2(e^t - 1)} = e^{(\lambda_1 + \lambda_2)(e^t - 1)}$.
2.  On reconnaît la MGF d'une loi de Poisson. Le paramètre est ce qui multiplie $(e^t - 1)$.
    Donc $S$ suit une loi de Poisson de paramètre $(\lambda_1 + \lambda_2)$.
    $S \sim \text{Poisson}(\lambda_1 + \lambda_2)$.
\end{correctionbox}

\begin{correctionbox}[Correction Exercice 23 : Application (Somme de Binomiales)]
$M_S(t) = M_X(t) \cdot M_Y(t) = (p e^t + (1-p))^n \cdot (p e^t + (1-p))^m$
$M_S(t) = (p e^t + (1-p))^{n+m}$.
On reconnaît la MGF d'une loi Binomiale avec $n+m$ essais et une probabilité de succès $p$.
$S \sim \text{Bin}(n+m, p)$.
\end{correctionbox}

\begin{correctionbox}[Correction Exercice 24 : Application (Transformation Linéaire)]
$M_Y(t) = E[e^{tY}] = E[e^{t(aX + b)}] = E[e^{taX + tb}]$
$M_Y(t) = E[e^{taX} \cdot e^{tb}]$.
Puisque $e^{tb}$ est une constante :
$M_Y(t) = e^{tb} E[e^{(ta)X}]$.
Par définition, $E[e^{(ta)X}]$ est la MGF de $X$ évaluée au point $(ta)$.
$M_Y(t) = e^{tb} M_X(ta)$.
\end{correctionbox}

\begin{correctionbox}[Correction Exercice 25 : Application (Moyenne d'Échantillon)]
Soit $S = \sum X_i$. $\bar{X} = S / n = (1/n)S$.
1.  D'abord, la MGF de la somme $S$ (Exercice 21) :
    $M_S(t) = M_{X_1}(t) \cdot \dots \cdot M_{X_n}(t) = [M_X(t)]^n$ (car i.i.d.)
2.  Ensuite, on utilise la transformation linéaire $\bar{X} = aS + b$ avec $a=1/n$ et $b=0$ (Exercice 24) :
    $M_{\bar{X}}(t) = e^{b t} M_S(a t) = e^0 M_S(t/n)$
    $M_{\bar{X}}(t) = M_S(t/n)$.
3.  On combine les deux :
    $M_{\bar{X}}(t) = [M_X(t/n)]^n$.
\end{correctionbox}

\subsection{Exercices Python}

La loi log-normale est fondamentale en finance. Elle repose sur l'idée que si les \textbf{log-rendements} d'une action $X_i = \ln(P_i / P_{i-1})$ sont (approximativement) normaux, alors le prix futur $P_t$, qui est un \textbf{produit} de ces rendements ($P_t = P_0 \times e^{X_1} \times \dots \times e^{X_t}$), suivra une loi log-normale.

Nous allons estimer les paramètres $\mu$ et $\sigma^2$ de la loi normale sous-jacente à partir des log-rendements journaliers de Microsoft (MSFT) et Google (GOOG), puis utiliser la théorie log-normale pour modéliser les prix.

\begin{codecell}
!pip install yfinance
import yfinance as yf
import pandas as pd
import numpy as np
from scipy.stats import norm # Moteur pour les calculs de CDF/PDF
import matplotlib.pyplot as plt

# Definir les tickers et la periode
tickers = ["MSFT", "GOOG"]
start_date = "2020-01-01"
end_date = "2024-12-31"

# Telecharger les prix de cloture ajustes
data = yf.download(tickers, start=start_date, end=end_date)["Adj Close"]

# Calculer les LOG-RENDEMENTS journaliers
log_returns = np.log(data / data.shift(1)).dropna()

# Renommer les colonnes
log_returns.columns = ["MSFT_LogReturn", "GOOG_LogReturn"]

# 'log_returns' est notre DataFrame.
# X_msft = log_returns["MSFT_LogReturn"]
# X_goog = log_returns["GOOG_LogReturn"]
\end{codecell}

\begin{exercicebox}[Exercice 1 : Estimer les Paramètres $\mu$ et $\sigma^2$]
Soit $P_t$ le prix de MSFT. Le modèle suppose que $X = \ln(P_t/P_{t-1}) \sim \mathcal{N}(\mu, \sigma^2)$. Les paramètres $\mu$ et $\sigma^2$ sont les paramètres "log-normaux".

\textbf{Votre tâche :}
\begin{enumerate}
    \item Estimer $\mu$ (l'espérance du log-rendement journalier) pour MSFT.
    \item Estimer $\sigma^2$ (la variance du log-rendement journalier) pour MSFT.
    \item Estimer $\sigma$ (l'écart-type du log-rendement journalier) pour MSFT.
\end{enumerate}
\end{exercicebox}

\begin{exercicebox}[Exercice 2 : Test de Normalité (Graphique)]
La théorie log-normale repose sur la normalité des log-rendements $X$. Vérifions-le visuellement.

\textbf{Votre tâche :}
\begin{enumerate}
    \item Utiliser $\mu$ et $\sigma$ (pour MSFT) de l'Exercice 1.
    \item \textbf{(Plot)} Tracer l'histogramme des log-rendements \textbf{empiriques} de MSFT (Indice : \texttt{plt.hist(..., density=True, bins=50)}).
    \item \textbf{(Plot)} Superposer la PDF \textbf{théorique} de la loi normale $\mathcal{N}(\mu, \sigma^2)$ sur cet histogramme.
    \item (Indice : Créez un \texttt{np.linspace}, calculez la PDF avec \texttt{norm.pdf(x, loc=mu, scale=sigma)}, puis \texttt{plt.plot()}).
    \item (Conclusion) La cloche théorique s'ajuste-t-elle bien aux données réelles ?
\end{enumerate}
\end{exercicebox}

\begin{exercicebox}[Exercice 3 : Asymétrie (Prix vs Log-Rendements)]
La théorie dit que les log-rendements $X$ sont symétriques (Normaux), mais que les prix $P_t$ sont asymétriques à droite (Log-Normaux).

\textbf{Votre tâche :}
\begin{enumerate}
    \item Calculer la moyenne et la médiane de la série des \textbf{log-rendements} de MSFT.
    \item Calculer la moyenne et la médiane de la série des \textbf{prix} de MSFT (la colonne \texttt{data['MSFT']}).
    \item Comparer les deux paires. Les log-rendements sont-ils symétriques (moyenne $\approx$ médiane) ? Les prix sont-ils asymétriques (moyenne $>$ médiane) ?
    \item \textbf{(Plot)} Créer deux histogrammes côte à côte (\texttt{plt.subplot}) pour visualiser la distribution des log-rendements et celle des prix.
\end{enumerate}
\end{exercicebox}

\begin{exercicebox}[Exercice 4 : Espérance vs Médiane (Théorique)]
Soit $Y = P_t/P_{t-1} = e^X$ la variable "ratio de prix journalier". $Y \sim \text{Log-}\mathcal{N}(\mu, \sigma^2)$.
Théorie : $\text{Med}(Y) = e^{\mu}$ et $E[Y] = e^{\mu + \sigma^2/2}$.

\textbf{Votre tâche :}
\begin{enumerate}
    \item Utiliser $\mu$ et $\sigma^2$ (pour MSFT) de l'Exercice 1.
    \item Calculer la médiane \textbf{théorique} $\text{Med}(Y)$.
    \item Calculer l'espérance \textbf{théorique} $E[Y]$.
    \item Vérifier que $E[Y] > \text{Med}(Y)$, confirmant l'asymétrie.
\end{enumerate}
\end{exercicebox}

\begin{exercicebox}[Exercice 5 : Espérance Théorique vs Empirique]
Vérifions le calcul de $E[Y]$ de l'exercice 4 de manière empirique.

\textbf{Votre tâche :}
\begin{enumerate}
    \item Créer la série $Y$ (ratio de prix journalier) : $Y = \exp(X_{\text{msft}})$.
    \item Calculer l'espérance \textbf{empirique} de $Y$ (la moyenne de cette série $Y$).
    \item Comparer cette valeur empirique à l'espérance \textbf{théorique} $e^{\mu + \sigma^2/2}$ calculée à l'exercice 4.
\end{enumerate}
\end{exercicebox}

\begin{exercicebox}[Exercice 6 : Variance Théorique vs Empirique]
Théorie : $\text{Var}(Y) = (e^{\sigma^2} - 1) \cdot e^{2\mu + \sigma^2}$.

\textbf{Votre tâche :}
\begin{enumerate}
    \item Utiliser $\mu$ et $\sigma^2$ (pour MSFT) de l'Exercice 1.
    \item Calculer la variance \textbf{théorique} $\text{Var}(Y)$ en utilisant la formule ci-dessus.
    \item Calculer la variance \textbf{empirique} de la série $Y$ (créée à l'Ex 5).
    \item Comparer les deux résultats.
\end{enumerate}
\end{exercicebox}

\begin{exercicebox}[Exercice 7 : Modélisation du Prix Futur (Paramètres)]
Modélisons le prix de GOOG dans $t=20$ jours ouvrés (environ 1 mois).
Le prix $P_{20}$ est log-normal si l'on suppose $P_{20} = P_0 \cdot e^{X_{20}}$, où $P_0$ est le prix actuel.
Le log-rendement total $X_{20} = \ln(P_{20}/P_0)$ suit $X_{20} \sim \mathcal{N}(t\mu, t\sigma^2)$.

\textbf{Votre tâche :}
\begin{enumerate}
    \item Estimer $\mu_G$ et $\sigma_G^2$ (journaliers) pour GOOG (similaire à l'Ex 1).
    \item Définir $t=20$.
    \item Calculer $\mu_{20} = t\mu_G$ (l'espérance du log-rendement sur 20 jours).
    \item Calculer $\sigma_{20}^2 = t\sigma_G^2$ (la variance du log-rendement sur 20 jours).
\end{enumerate}
\end{exercicebox}

\begin{exercicebox}[Exercice 8 : Calcul de Probabilité (Prix Futur)]
En utilisant les paramètres $\mu_{20}$ et $\sigma_{20} = \sqrt{\sigma_{20}^2}$ de l'exercice 7 pour GOOG :

\textbf{Votre tâche :}
\begin{enumerate}
    \item Calculer la probabilité que GOOG ait un rendement positif sur 20 jours.
    \item On cherche $P(P_{20} > P_0) \implies P(P_{20}/P_0 > 1) \implies P(\ln(P_{20}/P_0) > \ln(1))$.
    \item Calculer $P(X_{20} > 0)$.
    \item (Indice : Standardiser 0 avec $\mu_{20}$ et $\sigma_{20}$, puis utiliser $1 - \Phi(z)$).
    \item \textbf{(Plot)} Tracer la PDF de $X_{20} \sim \mathcal{N}(\mu_{20}, \sigma_{20}^2)$ et hachurer la zone $x > 0$.
\end{enumerate}
\end{exercicebox}

\begin{exercicebox}[Exercice 9 : Calcul de Probabilité (Perte > 5\%)]
En utilisant les paramètres $\mu_{20}$ et $\sigma_{20}$ de l'exercice 7 pour GOOG :

\textbf{Votre tâche :}
\begin{enumerate}
    \item Calculer la probabilité que GOOG perde plus de 5\% sur 20 jours.
    \item On cherche $P(P_{20} < 0.95 \times P_0) \implies P(P_{20}/P_0 < 0.95)$.
    \item Calculer $P(X_{20} < \ln(0.95))$.
    \item (Indice : Standardiser $\ln(0.95)$ avec $\mu_{20}$ et $\sigma_{20}$, puis utiliser $\Phi(z)$).
    \item \textbf{(Plot)} Tracer la PDF de $X_{20}$ et hachurer la zone $x < \ln(0.95)$.
\end{enumerate}
\end{exercicebox}

\begin{exercicebox}[Exercice 10 : Problème Inverse (Intervalle de Confiance)]
Trouvons l'intervalle de 95\% pour le prix de GOOG dans 20 jours.
Nous cherchons les bornes $y_1, y_2$ telles que $P(y_1 \le P_{20} \le y_2) = 0.95$.
On suppose un intervalle centré sur la loi normale sous-jacente (entre $z=-1.96$ et $z=+1.96$).

\textbf{Votre tâche :}
\begin{enumerate}
    \item Trouver $z_{inf} = -1.96$ et $z_{sup} = +1.96$.
    \item "Dé-standardiser" ces Z-scores pour trouver les log-rendements $x_1$ et $x_2$ :
        $x = \mu_t + z \sigma_t$ (en utilisant $\mu_{20}$ et $\sigma_{20}$ de l'Ex 7).
    \item Convertir ces log-rendements en ratios de prix $y = e^x$.
    \item (Conclusion) L'intervalle de 95\% pour le ratio de prix est $[y_1, y_2]$.
\end{enumerate}
\end{exercicebox}

\begin{exercicebox}[Exercice 11 : Calcul de la Médiane vs Espérance (Prix Futur)]
Pour le prix de GOOG dans 20 jours, $P_{20} = P_0 \cdot Y_{20}$, où $Y_{20} \sim \text{Log-}\mathcal{N}(\mu_{20}, \sigma_{20}^2)$.

\textbf{Votre tâche :}
\begin{enumerate}
    \item Calculer le ratio de prix \textbf{médian} attendu : $\text{Med}(Y_{20}) = e^{\mu_{20}}$.
    \item Calculer le ratio de prix \textbf{moyen} (espérance) attendu : $E[Y_{20}] = e^{\mu_{20} + \sigma_{20}^2 / 2}$.
    \item (Conclusion) Lequel est le plus élevé ? Pourquoi est-ce important pour un investisseur ?
\end{enumerate}
\end{exercicebox}
\newpage

\section{Moments d'une distribution}

\subsection{Définitions fondamentales des moments}

Après avoir défini l'espérance ($\mu$) et la variance ($\sigma^2$), qui sont les moments d'ordre 1 et 2, nous pouvons généraliser cette idée pour capturer des informations plus subtiles sur la forme d'une distribution.

\begin{definitionbox}[Types de Moments]
Soit $X$ une variable aléatoire ayant une espérance $\mu$ et une variance $\sigma^2$. Pour tout entier positif $m$, on définit les moments suivants :
\begin{itemize}
    \item \textbf{$m$-ième moment (non centré)} : $E[X^m]$.
    \item \textbf{$m$-ième moment centré} : $E[(X - \mu)^m]$.
    \item \textbf{$m$-ième moment standardisé} : $E\left[\left(\frac{X - \mu}{\sigma}\right)^m\right]$.
\end{itemize}
Les moments centrés et standardisés permettent d'étudier les propriétés de la distribution indépendamment de sa position ($\mu$) et de son échelle ($\sigma$).
\end{definitionbox}

\subsection{Asymétrie (Skewness)}

Le premier moment nous donne la tendance centrale. Le deuxième moment (la variance) nous donne la dispersion. Le troisième moment, lui, va nous renseigner sur la \textit{symétrie} de la distribution.

\begin{definitionbox}[Asymétrie (Skewness)]
L'\textbf{asymétrie} (ou \textit{skewness}) d'une variable aléatoire $X$ de moyenne $\mu$ et d'écart-type $\sigma$ est définie comme le \textbf{troisième moment standardisé} :
$$ \text{Skew}(X) = E\left[ \left( \frac{X - \mu}{\sigma} \right)^3 \right]. $$
\end{definitionbox}

\begin{intuitionbox}[Comprendre la Formule du Skewness]
Pour une variable aléatoire $X$ de moyenne $\mu$ et d'écart-type $\sigma$, le \textbf{skewness} est défini comme :
\[
\text{Skew}(X) = \frac{E[(X - \mu)^3]}{\sigma^3}
\]

\medskip

\textbf{Logique du numérateur : le moment centré d'ordre 3}
\begin{itemize}
    \item Le terme $(X - \mu)^3$ est le \textbf{cube de l'écart à la moyenne}
    \item Contrairement à $(X - \mu)^2$ (toujours positif), le cube \textbf{conserve le signe} de l'écart
    \item Il pondère différemment les observations à gauche et à droite de la moyenne
\end{itemize}

\medskip

% --- MODIFIÉ : Tableau supprimé et fusionné dans la liste ---
\textbf{Interprétation intuitive}
\begin{itemize}
    \item \textbf{Skewness = 0 (Symétrique)} : La distribution est symétrique. Les écarts positifs et négatifs s'annulent. Typiquement : Moyenne = Médiane = Mode.
    \item \textbf{Skewness > 0 (Queue à droite)} : La distribution présente une queue longue à droite. Les grandes valeurs positives sont amplifiées par le cube. Les valeurs extrêmes tirent la moyenne vers la droite.
    \item \textbf{Skewness < 0 (Queue à gauche)} : La distribution présente une queue longue à gauche. Les écarts négatifs dominent. Les valeurs extrêmes tirent la moyenne vers la gauche.
\end{itemize}
% --- FIN MODIFICATION ---

\medskip

\textbf{Pourquoi $\sigma^3$ au dénominateur ?}
\begin{itemize}
    \item Le moment d'ordre 3 est homogène à des unités au cube
    \item On divise par $\sigma^3$ pour obtenir un coefficient \textbf{sans dimension}
    \item Permet la comparaison entre distributions de différentes échelles
\end{itemize}
\end{intuitionbox}

\begin{remarquebox}[Pourquoi Standardiser ?]
En standardisant d'abord ($\frac{X-\mu}{\sigma}$), la définition de $\text{Skew}(X)$ ne dépend ni de la position ($\mu$) ni de l'échelle ($\sigma$) de la distribution, ce qui est raisonnable puisque ces informations sont déjà fournies par la moyenne et l'écart-type. De plus, cette standardisation garantit que l'asymétrie est invariante par changement d'unité de mesure (par exemple, passer des pouces aux mètres n'affecte pas la valeur de l'asymétrie).
\end{remarquebox}

\subsection{Propriétés de symétrie}

Le skewness est une mesure numérique de l'asymétrie. Mais nous pouvons aussi définir la symétrie de manière formelle.

\begin{definitionbox}[Symétrie d'une Variable Aléatoire]
On dit qu'une variable aléatoire $X$ a une distribution \textbf{symétrique} autour de $\mu$ si la variable $X - \mu$ a la même distribution que $\mu - X$. On dit aussi que $X$ est symétrique ou que sa distribution est symétrique. Ces trois formulations ont le même sens.
\end{definitionbox}

\begin{theorembox}[Symétrie en Termes de Fonction de Densité]
Soit $X$ une variable aléatoire continue de fonction de densité de probabilité (PDF) $f$. Alors, $X$ est symétrique autour de $\mu$ si et seulement si :
$$ f(x) = f(2\mu - x) \quad \text{pour tout } x. $$
\end{theorembox}

\begin{proofbox}[Preuve du Théorème de Symétrie]
Soit $F$ la fonction de répartition (CDF) de $X$. Si la symétrie tient, alors :
$$ F(x) = P(X \le x) = P(X - \mu \le x - \mu) = P(\mu - X \le x - \mu) = P(X \ge 2\mu - x) = 1 - F(2\mu - x). $$

En prenant la dérivée des deux côtés par rapport à $x$, on obtient :
$$ f(x) = \frac{d}{dx}F(x) = \frac{d}{dx}[1 - F(2\mu - x)] = f(2\mu - x). $$

Cela démontre que la condition $f(x) = f(2\mu - x)$ est nécessaire et suffisante pour la symétrie.
\end{proofbox}

\subsection{Aplatissement (Kurtosis)}

Après l'asymétrie (ordre 3), le moment d'ordre 4 nous informe sur "l'épaisseur" des queues de la distribution, c'est-à-dire la probabilité d'obtenir des valeurs très éloignées de la moyenne.

\begin{definitionbox}[Kurtosis (Aplatissement)]
Pour une variable aléatoire $X$ de moyenne $\mu$ et d'écart-type $\sigma$, le \textbf{kurtosis} est défini comme le \textbf{quatrième moment standardisé} :
$$ \text{Kurtosis}(X) = E\left[ \left( \frac{X - \mu}{\sigma} \right)^4 \right]. $$

Dans la pratique, on utilise plus souvent le \textbf{kurtosis excessif} (ou excès de kurtosis), défini comme :
$$ \text{Excess Kurtosis}(X) = E\left[ \left( \frac{X - \mu}{\sigma} \right)^4 \right] - 3. $$
La soustraction de 3 fait en sorte que le kurtosis d'une loi normale soit égal à 0.
\end{definitionbox}

\begin{intuitionbox}[Comprendre la Kurtosis]
Pour une variable aléatoire $X$, le \textbf{kurtosis} est défini comme :
\[
\text{Kurt}(X) = \frac{E[(X - \mu)^4]}{\sigma^4}
\]
et l'\textbf{excess kurtosis} (kurtosis excédentaire) comme : $\text{Excess Kurtosis} = \text{Kurt}(X) - 3$.

\medskip

\textbf{Pourquoi le moment d'ordre 4 ?}
\begin{itemize}
    \item Comme la variance, on utilise une puissance paire (pas d'effet de signe)
    \item La puissance 4 \textbf{amplifie énormément les écarts extrêmes}
    \item Mesure le \textbf{poids des queues} et la \textbf{concentration autour de la moyenne}
\end{itemize}

\medskip

% --- MODIFIÉ : Tableau supprimé et fusionné dans la liste ---
\textbf{Interprétation intuitive (basée sur l'Excess Kurtosis)}
\begin{itemize}
    \item \textbf{Leptokurtique (Excess Kurtosis > 0)} : Kurtosis total > 3. Distribution pointue avec des queues épaisses. Les événements extrêmes sont plus probables que pour une loi normale.
    \item \textbf{Mésocurtique (Excess Kurtosis = 0)} : Kurtosis total = 3. C'est la référence (loi normale).
    \item \textbf{Platykurtique (Excess Kurtosis < 0)} : Kurtosis total < 3. Distribution aplatie avec des queues légères et un centre large. Les événements extrêmes sont moins probables.
\end{itemize}
% --- FIN MODIFICATION ---

\medskip

\textbf{Application en finance}
\begin{itemize}
    \item Les rendements financiers ont souvent un excès de kurtosis positif
    \item Indique une probabilité plus élevée d'événements extrêmes que la loi normale
    \item Justifie le "vol smile" dans les options
\end{itemize}

\medskip

\textbf{Pourquoi $\sigma^4$ au dénominateur ?}
\begin{itemize}
    \item Le moment d'ordre 4 est homogène à des unités$^4$
    \item On divise par $\sigma^4$ pour un coefficient \textbf{sans dimension}
\end{itemize}
\end{intuitionbox}

\subsection{Exemples de distributions}

Pour bien fixer les idées, comparons le skewness et le kurtosis de plusieurs distributions classiques. Notez que dans les graphiques suivants, le "Kurtosis" affiché est l'\textit{excess kurtosis} (centré à 0).

\begin{examplebox}[La Distribution Normale (Mésokurtique)]

\begin{center}
\begin{tikzpicture}
  \begin{axis}[
    width=0.8\textwidth,
    height=0.5\textwidth,
    xlabel={$x$},
    title={Densité Normale ($\mu=0$, $\sigma=1$)},
    grid=both,
    grid style={line width=.1pt, draw=gray!30},
    major grid style={line width=.2pt,draw=gray!50},
    domain=-4:4,
    samples=100,
    enlargelimits=false,
    axis lines=middle,
    xmin=-4, xmax=4,
    ymin=0, ymax=0.55
  ]
 
  % Courbe de densité
  \addplot [thick, color=blue, fill=blue!20, fill opacity=0.5] 
    {1/(sqrt(2*pi))*exp(-x^2/2)} \closedcycle;
 
  % Ligne de la moyenne
  \addplot [red, dashed, thick] coordinates {(0,0) (0,0.4)};
  \node[red, fill=white, font=\scriptsize, rounded corners, inner sep=2pt, opacity=0.8] at (axis cs:0.5,0.35) {Moyenne};
 
  % Boîte de texte avec moments seulement
  \node [draw=black, fill=white, rounded corners, font=\scriptsize, align=left, anchor=north east, opacity=0.8] 
    at (axis description cs:0.95,0.95) { % MODIFIÉ
    \textbf{Moments :}\\
    • Moyenne = 0.00\\
    • Variance = 1.00\\
    • Skewness = 0.00\\
    • Kurtosis = 0.00
    };
    
  \end{axis}
\end{tikzpicture}
\end{center}

La distribution normale est l'archétype de la courbe en cloche. Imaginez une cible : la majorité des flèches touchent le centre, et plus on s'éloigne du centre, moins il y a de chances d'être touché. C'est une distribution parfaitement symétrique, ce qui se traduit par un \textbf{skewness nul (0.00)}. Son pic est ni trop pointu, ni trop plat : c'est notre point de référence, on dit qu'elle est \textbf{mésokurtique}, d'où son kurtosis de \textbf{0.00}. C'est la base de nombreuses analyses statistiques car elle modélise naturellement beaucoup de phénomènes.

\end{examplebox}

\begin{examplebox}[La Distribution Exponentielle (Asymétrique à Droite)]

\begin{center}
\begin{tikzpicture}
  \begin{axis}[
    width=0.8\textwidth,
    height=0.5\textwidth,
    xlabel={$x$},
    title={Densité Exponentielle ($\lambda=1$)},
    grid=both,
    grid style={line width=.1pt, draw=gray!30},
    major grid style={line width=.2pt,draw=gray!50},
    domain=0:6,
    samples=100,
    enlargelimits=false,
    axis lines=middle,
    xmin=0, xmax=6,
    ymin=0, ymax=1.1
  ]
 
  % Courbe de densité exponentielle
  \addplot [thick, color=blue, fill=blue!20, fill opacity=0.5] 
    {exp(-x)} \closedcycle;
 
  % Ligne de la moyenne
  \addplot [red, dashed, thick] coordinates {(1,0) (1,0.37)};
  \node[red, fill=white, font=\scriptsize, rounded corners, inner sep=2pt, opacity=0.8] at (axis cs:1.5,0.3) {Moyenne};
 
  % Boîte de texte avec moments seulement
  \node [draw=black, fill=white, rounded corners, font=\scriptsize, align=left, anchor=north east, opacity=0.8] 
    at (axis description cs:0.95,0.95) { % MODIFIÉ
    \textbf{Moments :}\\
    • Moyenne = 1.00\\
    • Variance = 1.00\\
    • Skewness = 2.00\\
    • Kurtosis = 6.00
    };
    
  \end{axis}
\end{tikzpicture}
\end{center}

Imaginez le temps d'attente avant un événement rare, comme un appel téléphonique. La plupart du temps, l'appel arrive vite, mais il peut parfois y avoir de longues attentes. C'est exactement ce que modélise la distribution exponentielle : un pic à gauche et une longue queue à droite. Cela se traduit par un \textbf{skewness positif élevé (2.00)}, indiquant une asymétrie marquée. Elle est aussi \textbf{leptokurtique} (\textbf{kurtosis = 6.00}) : son pic est pointu, et la longue queue droite signifie qu'il y a une probabilité non négligeable de valeurs extrêmes.

\end{examplebox}

\begin{examplebox}[La Distribution Uniforme (Platykurtique)]

\begin{center}
\begin{tikzpicture}
  \begin{axis}[
    width=0.8\textwidth,
    height=0.5\textwidth,
    xlabel={$x$},
    title={Densité Uniforme ($a=0$, $b=2$)},
    grid=both,
    grid style={line width=.1pt, draw=gray!30},
    major grid style={line width=.2pt,draw=gray!50},
    domain=-0.5:2.5,
    samples=100,
    enlargelimits=false,
    axis lines=middle,
    ymin=0,
    ymax=.75,
    xmin=-1, xmax=4
  ]
 
  % Courbe de densité uniforme
  \addplot [thick, color=blue, fill=blue!20, fill opacity=0.5, const plot] 
    coordinates {(-0.5,0) (0,0) (0,0.5) (2,0.5) (2,0) (2.5,0)};
 
  % Ligne de la moyenne
  \addplot [red, dashed, thick] coordinates {(1,0) (1,0.5)};
  \node[red, fill=white, font=\scriptsize, rounded corners, inner sep=2pt, opacity=0.8] at (axis cs:1.3,0.4) {Moyenne};
 
  % Boîte de texte avec moments seulement
  \node [draw=black, fill=white, rounded corners, font=\scriptsize, align=left, anchor=north east, opacity=0.8] 
    at (axis description cs:0.95,0.95) { % MODIFIÉ
    \textbf{Moments :}\\
    • Moyenne = 1.00\\
    • Variance = 0.33\\
    • Skewness = 0.00\\
    • Kurtosis = -1.20
    };
    
  \end{axis}
\end{tikzpicture}
\end{center}

La distribution uniforme, c'est le "tirage au sort parfait" : chaque valeur sur un intervalle a la même chance d'être tirée. Visuellement, c'est un rectangle, donc aucune valeur n'est privilégiée. Elle est symétrique (\textbf{skewness = 0.00}), mais contrairement à la normale, elle est "plate", sans pic central. Cela se traduit par un \textbf{kurtosis négatif (-1.20)}, ce qui signifie qu'elle est \textbf{platykurtique}. Elle est donc très différente des distributions avec un pic central comme la normale.

\end{examplebox}

\begin{examplebox}[La Distribution Log-Normale (Fortement Leptokurtique)]

\begin{center}
\begin{tikzpicture}
  \begin{axis}[
    width=0.8\textwidth,
    height=0.5\textwidth,
    xlabel={$x$},
    title={Densité Log-Normale ($\sigma=0.7$)},
    grid=both,
    grid style={line width=.1pt, draw=gray!30},
    major grid style={line width=.2pt,draw=gray!50},
    domain=0:6,
    samples=100,
    enlargelimits=false,
    axis lines=middle,
    xmin=0, xmax=6,
    ymin=0, ymax=1
  ]
 
  % Courbe de densité log-normale
  \addplot [thick, color=blue, fill=blue!20, fill opacity=0.5] 
    {1/(x*0.7*sqrt(2*pi))*exp(-(ln(x))^2/(2*0.7^2))};
 
  % Ligne de la moyenne
  \addplot [red, dashed, thick] coordinates {(1.28,0) (1.28,0.4)};
  \node[red, fill=white, font=\scriptsize, rounded corners, inner sep=2pt, opacity=0.8] at (axis cs:1.8,0.5) {Moyenne};
 
  % Boîte de texte avec moments seulement
  \node [draw=black, fill=white, rounded corners, font=\scriptsize, align=left, anchor=north east, opacity=0.8] 
    at (axis description cs:0.95,0.95) { % MODIFIÉ
    \textbf{Moments :}\\
    • Moyenne = 1.28\\
    • Variance = 1.03\\
    • Skewness = 2.89\\
    • Kurtosis = 20.78
    };
    
  \end{axis}
\end{tikzpicture}
\end{center}

La log-normale est une distribution très asymétrique. Imaginez la richesse d'une population : la majorité est modeste, mais il existe une petite proportion de très riches, ce qui "étire" la droite de la courbe. Cela donne un \textbf{skewness très élevé (2.89)}. Elle est extrêmement \textbf{leptokurtique} (\textbf{kurtosis = 20.78}) : un pic très aigu et une queue droite très lourde. Cela signifie qu'il y a un risque élevé de valeurs extrêmement grandes, ce qui la rend très utile pour modéliser des phénomènes avec de rares événements extrêmes.

\end{examplebox}

Nous avons défini les moments d'une \textit{distribution} (moments de population), tels que $\mu = E[X]$ ou $\sigma^2 = E[(X-\mu)^2]$. Ce sont des valeurs théoriques, la "vérité" sous-jacente.

En pratique, nous ne connaissons presque jamais cette "vérité". Nous ne disposons que de données. Notre but est d'utiliser ces données pour \textit{estimer} les moments de la population.

\subsection{Moments d'échantillon (Sample Moments)}

\begin{definitionbox}[Moments d'Échantillon]
Soit $X_1, X_2, \dots, X_n$ un échantillon de $n$ observations.
\begin{itemize}
    \item La \textbf{moyenne d'échantillon} (notre "meilleure estimation" de $\mu$) est :
    $$ \bar{X} = \frac{1}{n} \sum_{i=1}^n X_i $$
    \item La \textbf{variance d'échantillon (non biaisée)} (notre "meilleure estimation" de $\sigma^2$) est :
    $$ s^2 = \frac{1}{n-1} \sum_{i=1}^n (X_i - \bar{X})^2 $$
\end{itemize}
De même, on peut calculer un \textit{skewness d'échantillon} et un \textit{kurtosis d'échantillon} en utilisant $\bar{X}$ et $s$, qui seront nos estimations du vrai skewness et du vrai kurtosis de la population.
\end{definitionbox}

\begin{examplebox}[Application : Contrôle Qualité ]
Imaginez une usine qui produit des sacs de sucre de 1kg.
\begin{itemize}
    \item \textbf{Population :} L'infinité de tous les sacs de sucre que la machine produira.
    \item \textbf{Moment de population (inconnu) :} Le poids moyen \textit{réel} $\mu$ que la machine verse, et la variance \textit{réelle} $\sigma^2$ (sa constance).
    \item \textbf{Problème :} Nous ne pouvons pas peser tous les sacs !
    \item \textbf{Solution :} Nous prélevons un \textbf{échantillon} de $n=10$ sacs.
    
    Nous les pesons : $\{ 1002g, 998g, 1001g, 995g, 1003g, 1000g, 997g, 1005g, 999g, 1000g \}$.
    
    \item \textbf{Calcul des moments d'échantillon :}
    \begin{itemize}
        \item $\bar{X} = (1002 + 998 + \dots + 1000) / 10 = 1000g$.
        \item $s^2 = \frac{1}{10-1} \left( (1002-1000)^2 + (998-1000)^2 + \dots \right) = 7.33 g^2$.
    \end{itemize}
    \item \textbf{Conclusion :} Notre meilleure estimation est que la machine est bien réglée sur $\mu = 1000g$. L'écart-type de notre échantillon est $s = \sqrt{7.33} \approx 2.7g$. Nous pouvons utiliser cela pour affirmer, par exemple, que 95\% des sacs se situent probablement entre $1000 \pm 2s$ (si la distribution est normale).
\end{itemize}
\end{examplebox}

\begin{remarquebox}[L'Intuition du "$n-1$"]
Pourquoi diviser par $n-1$ pour la variance ? C'est la \textbf{correction de Bessel}.

Imaginez un échantillon de 1 seule personne ($n=1$). Sa taille est 170cm.
\begin{itemize}
    \item Quelle est la moyenne de l'échantillon ? $\bar{X} = 170$ cm.
    \item Quelle est la variance de l'échantillon ? $\sum (X_i - \bar{X})^2 = (170 - 170)^2 = 0$.
    \item Si on divisait par $n=1$, on estimerait que la variance de la population est 0. C'est absurde ! Cela voudrait dire que tout le monde mesure 170cm.
\end{itemize}
En divisant par $n-1$ (donc $1-1=0$), la formule devient $0/0$ (indéfinie), ce qui nous dit à juste titre : "Je ne peux pas estimer la dispersion avec une seule personne."

\textbf{Intuition plus générale :} Nous "perdons un degré de liberté". Pour calculer la variance, nous avons besoin de connaître la moyenne. Mais nous ne connaissons pas la vraie moyenne $\mu$. Nous devons donc utiliser $\bar{X}$, une \textit{estimation}. Le fait d'utiliser une estimation calculée \textit{à partir de ce même échantillon} introduit un léger biais (nos données sont, par définition, centrées sur $\bar{X}$). Diviser par $n-1$ au lieu de $n$ "gonfle" légèrement le résultat pour compenser ce biais.
\end{remarquebox}

\subsection{Fonctions génératrices des moments (MGF)}

\begin{definitionbox}[Fonction Génératrice des Moments (MGF)]
La \textbf{fonction génératrice des moments} (MGF) d'une variable aléatoire $X$, notée $M_X(t)$, est définie comme :
$$ M_X(t) = E[e^{tX}] $$
\end{definitionbox}

\begin{intuitionbox}[L'ADN, le Code-Barres, ou le Fichier .zip]
Ce concept est abstrait, alors utilisons des analogies :

\textbf{Analogie 1 : L'ADN ou l'Empreinte Digitale}
\begin{itemize}
    \item La MGF est l'**empreinte digitale unique** d'une distribution.
    \item Elle "compresse" \textit{toutes} les informations sur votre distribution (moyenne, variance, skewness, kurtosis, etc.) en une seule, unique fonction.
    \item Si deux distributions ont la même MGF, elles sont identiques. C'est la \textbf{propriété d'unicité}.
\end{itemize}

\textbf{Analogie 2 : Le Code-Barres}
\begin{itemize}
    \item Pensez à une distribution (ex: Loi Normale) comme à un produit au supermarché.
    \item La MGF, $M_X(t)$, est son **code-barres unique**.
    \item Le processus de "génération de moments" (que nous verrons ci-dessous) est le \textbf{scanner}.
    \item En scannant le code-barres ($M_X(t)$), vous pouvez obtenir n'importe quelle information :
        \item Scan 1 ($M_X'(0)$) $\to$ vous donne le prix ($E[X]$).
        \item Scan 2 ($M_X''(0)$) $\to$ vous donne le poids ($E[X^2]$).
        \item Scan 3 ($M_X'''(0)$) $\to$ vous donne le pays d'origine ($E[X^3]$).
\end{itemize}

\textbf{Pourquoi $e^{tX}$ ?}
La "magie" vient du développement en série de Taylor de $e^x$:
$$ e^{tX} = 1 + (tX) + \frac{(tX)^2}{2!} + \frac{(tX)^3}{3!} + \dots $$
Quand on prend l'espérance, $E[\cdot]$, les puissances de $X$ (c'est-à-dire $X, X^2, X^3\dots$) apparaissent. Ce sont les moments ! La MGF "stocke" tous ces moments en les organisant comme coefficients d'un polynôme infini en $t$.
\end{intuitionbox}

\subsection{Génération des moments via les MGF}

\begin{theorembox}[Moments par Dérivation]
Si la MGF $M_X(t)$ existe, alors le $m$-ième moment non centré $E[X^m]$ est la $m$-ième dérivée de $M_X(t)$, évaluée en $t=0$ :
$$ E[X^m] = \frac{d^m}{dt^m} M_X(t) \bigg|_{t=0} = M_X^{(m)}(0) $$
\end{theorembox}

\begin{examplebox}[Application : La Loi de Poisson]
Une loi de Poisson modélise le nombre d'événements (ex: appels à un centre d'appels) par heure. Soit $X \sim \text{Poisson}(\lambda)$, où $\lambda$ est le nombre moyen d'appels.

La MGF (l'ADN) d'une loi de Poisson est (on l'admet) :
$$ M_X(t) = e^{\lambda(e^t - 1)} $$

Utilisons notre "scanner" (les dérivées) pour trouver les moments.

\textbf{1. Trouver la Moyenne $E[X]$ :}
On dérive une fois (règle de la chaîne) :
$$ M_X'(t) = \frac{d}{dt} \left( e^{\lambda(e^t - 1)} \right) = \underbrace{e^{\lambda(e^t - 1)}}_{\text{répète}} \cdot \underbrace{(\lambda e^t)}_{\text{dérivée interne}} $$
Maintenant, on évalue en $t=0$ :
$$ E[X] = M_X'(0) = e^{\lambda(e^0 - 1)} \cdot (\lambda e^0) = e^{\lambda(1 - 1)} \cdot (\lambda \cdot 1) = e^0 \cdot \lambda = 1 \cdot \lambda = \lambda $$
\textbf{Résultat :} La moyenne est $\lambda$, ce qui est la définition même du paramètre de la loi de Poisson. Parfait.

\textbf{2. Trouver $E[X^2]$ (pour la variance) :}
On dérive une seconde fois (règle du produit sur $M_X'(t) = (\lambda e^t) \cdot (e^{\lambda(e^t - 1)})$) :
$$ M_X''(t) = \underbrace{(\lambda e^t)}_{\text{dérivée de u}} \cdot \underbrace{(e^{\lambda(e^t - 1)})}_{\text{v}} + \underbrace{(\lambda e^t)}_{\text{u}} \cdot \underbrace{(e^{\lambda(e^t - 1)} \cdot \lambda e^t)}_{\text{dérivée de v}} $$
Maintenant, on évalue en $t=0$ (tous les $e^0$ deviennent 1) :
$$ E[X^2] = M_X''(0) = (\lambda \cdot 1) \cdot (e^{\lambda(1-1)}) + (\lambda \cdot 1) \cdot (e^{\lambda(1-1)} \cdot \lambda \cdot 1) $$
$$ E[X^2] = (\lambda) \cdot (e^0) + (\lambda) \cdot (e^0 \cdot \lambda) = \lambda \cdot 1 + \lambda \cdot (1 \cdot \lambda) = \lambda + \lambda^2 $$

\textbf{3. Trouver la Variance $\text{Var}(X)$ :}
$\text{Var}(X) = E[X^2] - (E[X])^2 = (\lambda + \lambda^2) - (\lambda)^2 = \lambda$
\textbf{Résultat :} Nous avons prouvé par les MGF que pour une loi de Poisson, $\text{Moyenne} = \text{Variance} = \lambda$. C'est une propriété fondamentale de cette loi.
\end{examplebox}

\subsection{Sommes de variables aléatoires indépendantes via les MGF}

C'est la super-puissance des MGF.

\begin{theorembox}[MGF d'une Somme]
Soient $X$ et $Y$ deux variables aléatoires \textbf{indépendantes}. Soit $S = X + Y$. Alors la MGF de $S$ est le produit des MGF individuelles :
$$ M_S(t) = M_{X+Y}(t) = M_X(t) \cdot M_Y(t) $$
\end{theorembox}

\begin{intuitionbox}[La Magie de l'Exponentielle]
Pourquoi est-ce vrai ? $M_{X+Y}(t) = E[e^{t(X+Y)}] = E[e^{tX} \cdot e^{tY}]$.
Parce que $X$ et $Y$ sont indépendantes, $E[f(X)g(Y)] = E[f(X)]E[g(Y)]$.
Donc, $E[e^{tX} \cdot e^{tY}] = E[e^{tX}] \cdot E[e^{tY}] = M_X(t) \cdot M_Y(t)$.

Les MGF transforment une opération analytiquement horrible (la "convolution" de densités) en une simple multiplication algébrique.
\end{intuitionbox}

\begin{examplebox}[Application : Portefeuille d'Actifs ou Tailles Humaines]
C'est l'un des théorèmes les plus importants des statistiques.
\textbf{Problème :} Soit $X$ la taille d'un homme, $X \sim N(\mu_X, \sigma_X^2)$. Soit $Y$ la taille d'une femme, $Y \sim N(\mu_Y, \sigma_Y^2)$. Si on les choisit au hasard, quelle est la loi de la somme de leurs tailles $S = X+Y$ ?

\begin{enumerate}
    \item \textbf{ADN de $X$} : La MGF d'une loi Normale $N(\mu, \sigma^2)$ est $M(t) = \exp(\mu t + \frac{1}{2}\sigma^2 t^2)$.
    \item \textbf{ADN de $X$ et $Y$} :
    $M_X(t) = \exp(\mu_X t + \frac{1}{2}\sigma_X^2 t^2)$
    $M_Y(t) = \exp(\mu_Y t + \frac{1}{2}\sigma_Y^2 t^2)$
    
    \item \textbf{ADN de $S = X+Y$} (on multiplie) :
    $M_S(t) = M_X(t) \cdot M_Y(t) = \exp(\mu_X t + \frac{1}{2}\sigma_X^2 t^2) \cdot \exp(\mu_Y t + \frac{1}{2}\sigma_Y^2 t^2)$
    
    \item \textbf{Simplification} (en additionnant les exposants) :
    $M_S(t) = \exp\left( (\mu_X t + \mu_Y t) + (\frac{1}{2}\sigma_X^2 t^2 + \frac{1}{2}\sigma_Y^2 t^2) \right)$
    $M_S(t) = \exp\left( (\mu_X + \mu_Y)t + \frac{1}{2}(\sigma_X^2 + \sigma_Y^2)t^2 \right)$
    
    \item \textbf{Conclusion (par Unicité)} :
    Regardez cet ADN ! C'est l'ADN d'une loi Normale !
    Le nouveau $\mu$ est $(\mu_X + \mu_Y)$.
    La nouvelle $\sigma^2$ est $(\sigma_X^2 + \sigma_Y^2)$.
\end{enumerate}

\textbf{Résultat :} Nous avons prouvé que \textbf{la somme de deux Normales indépendantes est une nouvelle Normale}.
Si $X \sim N(175cm, 7^2)$ et $Y \sim N(165cm, 6^2)$, alors $S \sim N(340cm, 7^2 + 6^2 = 85)$.
Notez que les écarts-types \textit{ne s'additionnent pas} ($\sqrt{85} \approx 9.2 \ne 7+6$). Ce sont les variances qui s'additionnent.
\end{examplebox}
\subsection{Exercices}

\textit{Pour tous les exercices de calcul, vous pouvez utiliser les valeurs (arrondies) suivantes pour la fonction de répartition de la loi normale standard $\Phi(z) = P(Z \le z)$ :}
\begin{itemize}
    \item $\Phi(0) = 0.5$
    \item $\Phi(0.08) \approx 0.5319$
    \item $\Phi(0.1) \approx 0.5398$
    \item $\Phi(1.0) \approx 0.8413$
    \item $\Phi(1.5) \approx 0.9332$
    \item $\Phi(1.58) \approx 0.9429$
    \item $\Phi(1.645) \approx 0.95$
    \item $\Phi(1.75) \approx 0.9599$
    \item $\Phi(1.96) \approx 0.975$
    \item $\Phi(2.0) \approx 0.9772$
    \item $\Phi(2.33) \approx 0.99$
    \item $\Phi(2.5) \approx 0.9938$
    \item $\Phi(3.0) \approx 0.9987$
    \item $\Phi(10.0) \approx 1.0$
\end{itemize}
\textit{Et rappelez-vous la propriété de symétrie : $\Phi(-z) = 1 - \Phi(z)$.}

% --- Section 1 : Inégalité de Chebyshev ---

\begin{exercicebox}[Exercice 1 : Chebyshev (Calcul de base)]
Une variable aléatoire $Y$ a une moyenne $\mu=50$ et une variance $\sigma^2=16$.
Utilisez l'inégalité de Chebyshev pour trouver une borne supérieure à $P(|Y - 50| \ge 10)$.
\end{exercicebox}

\begin{exercicebox}[Exercice 2 : Chebyshev (En termes d'écarts-types)]
Quelle est la borne \textit{universelle} (valable pour toute distribution) pour la probabilité qu'une variable aléatoire s'écarte de sa moyenne de plus de 5 écarts-types ?
\end{exercicebox}

\begin{exercicebox}[Exercice 3 : Chebyshev (Application à $\bar{X}_n$)]
Soit $X_i$ une suite de v.a. i.i.d. avec $\mu=100$ et $\sigma^2=400$. Soit $n=100$.
\begin{enumerate}
    \item Calculez $E[\bar{X}_{100}]$ et $\text{Var}(\bar{X}_{100})$.
    \item En utilisant Chebyshev, bornez $P(|\bar{X}_{100} - 100| \ge 4)$.
\end{enumerate}
\end{exercicebox}

% --- Section 2 : CLT (Forme $\bar{X}_n$) ---

\begin{exercicebox}[Exercice 4 : CLT (Distribution de $\bar{X}_n$)]
On prélève un échantillon de $n=64$ observations d'une population de moyenne $\mu=20$ et de variance $\sigma^2=16$.
Quelle est la distribution approximative de la moyenne d'échantillon $\bar{X}_{64}$ selon le TCL ?
\end{exercicebox}

\begin{exercicebox}[Exercice 5 : CLT (Calcul de probabilité pour $\bar{X}_n$)]
En utilisant les informations de l'exercice 4, calculez $P(\bar{X}_{64} \le 21)$.
\end{exercicebox}

\begin{exercicebox}[Exercice 6 : CLT (Calcul de probabilité pour $\bar{X}_n$)]
On prélève $n=100$ observations d'une population de moyenne $\mu=50$ et d'écart-type $\sigma=5$.
Calculez $P(49 \le \bar{X}_{100} \le 51)$.
\end{exercicebox}

\begin{exercicebox}[Exercice 7 : CLT (Inverse pour $\bar{X}_n$)]
Soit $\bar{X}_n$ la moyenne de $n=36$ v.a. i.i.d. de moyenne $\mu=10$ et de variance $\sigma^2=81$.
Trouvez la valeur $c$ telle que $P(\bar{X}_{36} \le c) \approx 0.9772$.
\end{exercicebox}

% --- Section 3 : CLT (Forme $S_n$) ---

\begin{exercicebox}[Exercice 8 : CLT (Distribution de $S_n$)]
On prélève un échantillon de $n=100$ observations d'une population de moyenne $\mu=10$ et de variance $\sigma^2=4$.
Quelle est la distribution approximative de la somme $S_{100} = \sum X_i$ selon le TCL ?
\end{exercicebox}

\begin{exercicebox}[Exercice 9 : CLT (Calcul de probabilité pour $S_n$)]
En utilisant les informations de l'exercice 8, calculez $P(S_{100} > 1020)$.
\end{exercicebox}

\begin{exercicebox}[Exercice 10 : CLT (Application $S_n$)]
Un ascenseur doit transporter $n=49$ personnes. Le poids de chaque personne est une v.a. i.i.d. de moyenne $\mu=70$kg et d'écart-type $\sigma=14$kg.
Calculez la probabilité que le poids total $S_{49}$ dépasse 3500 kg.
\end{exercicebox}

\begin{exercicebox}[Exercice 11 : CLT (Inverse pour $S_n$)]
Soit $S_n$ la somme de $n=64$ v.a. i.i.d. de moyenne $\mu=5$ et de variance $\sigma^2=1$.
Trouvez la valeur $c$ telle que $P(S_{64} > c) \approx 0.05$.
\end{exercicebox}

% --- Section 4 : Approximation Normale de la Loi Binomiale ---

\begin{exercicebox}[Exercice 12 : Règle d'Approximation]
Peut-on approximer une loi $X \sim \text{Bin}(n=40, p=0.1)$ par une loi normale en utilisant la règle $np \ge 10$ et $n(1-p) \ge 10$ ?
\end{exercicebox}

\begin{exercicebox}[Exercice 13 : Règle d'Approximation]
Peut-on approximer une loi $X \sim \text{Bin}(n=500, p=0.04)$ par une loi normale en utilisant la règle $np \ge 10$ et $n(1-p) \ge 10$ ?
\end{exercicebox}

\begin{exercicebox}[Exercice 14 : Paramètres d'Approximation]
Soit $X \sim \text{Bin}(100, 0.3)$. On souhaite l'approximer par $Y \sim \mathcal{N}(\mu_Y, \sigma_Y^2)$.
Calculez $\mu_Y$ et $\sigma_Y^2$.
\end{exercicebox}

\begin{exercicebox}[Exercice 15 : Correction de Continuité (Règles)]
Soit $X$ une variable binomiale (discrète) et $Y$ son approximation normale (continue).
Traduisez les probabilités discrètes suivantes en probabilités continues :
\begin{enumerate}
    \item $P(X = 50)$
    \item $P(X \ge 50)$
    \item $P(X \le 49)$
    \item $P(X < 49)$
\end{enumerate}
\end{exercicebox}

\begin{exercicebox}[Exercice 16 : Correction de Continuité (Règles)]
Traduisez les probabilités discrètes suivantes en probabilités continues :
\begin{enumerate}
    \item $P(X > 30)$
    \item $P(30 < X < 40)$
    \item $P(30 \le X \le 40)$
\end{enumerate}
\end{exercicebox}

% --- Section 5 : Calculs Complets (Approximation Binomiale) ---

\begin{exercicebox}[Exercice 17 : Binomiale (Calcul P(X=k))]
On lance une pièce équilibrée 100 fois. Soit $X$ le nombre de "Pile".
Utilisez l'approximation normale (avec correction de continuité) pour estimer $P(X=50)$.
(C'est l'exemple du texte).
\end{exercicebox}

\begin{exercicebox}[Exercice 18 : Binomiale (Calcul $P(X \ge k)$)]
On lance une pièce équilibrée 100 fois ($X \sim \text{Bin}(100, 0.5)$).
Estimez $P(X \ge 60)$.
\end{exercicebox}

\begin{exercicebox}[Exercice 19 : Binomiale (Calcul $P(X \le k)$)]
Un traitement a un taux de succès de $p=0.2$. On l'administre à $n=400$ patients. Soit $X$ le nombre de succès.
Estimez $P(X \le 70)$.
\end{exercicebox}

\begin{exercicebox}[Exercice 20 : Binomiale (Calcul $P(X > k)$)]
En utilisant la situation de l'exercice 19 ($X \sim \text{Bin}(400, 0.2)$), estimez $P(X > 92)$.
\end{exercicebox}

\begin{exercicebox}[Exercice 21 : Binomiale (Calcul $P(k_1 \le X \le k_2)$)]
Un sondage est mené auprès de $n=100$ personnes. On suppose que $p=0.6$ est la probabilité qu'une personne soutienne une mesure. Soit $X$ le nombre de supporters.
Estimez $P(50 \le X \le 65)$.
\end{exercicebox}

% --- Section 6 : Synthèse ---

\begin{exercicebox}[Exercice 22 : TCL vs Chebyshev]
Soit $\bar{X}_{100}$ la moyenne de $n=100$ v.a. i.i.d. de moyenne $\mu=10$ et $\sigma^2=25$.
Calculez $P(|\bar{X}_{100} - 10| \ge 1)$ en utilisant :
\begin{enumerate}
    \item L'inégalité de Chebyshev.
    \item Le Théorème Central Limite.
\end{enumerate}
\end{exercicebox}

\begin{exercicebox}[Exercice 23 : Taille d'Échantillon (CLT)]
Soient $X_i$ des v.a. i.i.d. avec $\mu=0$ et $\sigma^2=1$.
Combien d'échantillons $n$ faut-il pour garantir que $P(|\bar{X}_n| \le 0.1) \ge 0.95$ ?
(Indice : $P(-1.96 \le Z \le 1.96) = 0.95$).
\end{exercicebox}

\begin{exercicebox}[Exercice 24 : LLN vs CLT]
Considérons $\bar{X}_n$ pour des $X_i$ i.i.d. avec $\mu=10, \sigma^2=100$.
\begin{enumerate}
    \item Calculez $P(9 \le \bar{X}_n \le 11)$ pour $n=100$.
    \item Calculez $P(9 \le \bar{X}_n \le 11)$ pour $n=10000$.
    \item Comment ce résultat illustre-t-il la LLN ?
\end{enumerate}
\end{exercicebox}

\begin{exercicebox}[Exercice 25 : Binomiale (Sans Correction)]
Soit $X \sim \text{Bin}(400, 0.1)$. Estimez $P(X \le 30.5)$ \textit{sans} utiliser la correction de continuité (c'est-à-dire en approximant $P(X \le 30.5) \approx P(Y \le 30.5)$).
(Ceci permet de comparer avec l'exercice 19).
\end{exercicebox}


\subsection{Corrections des Exercices}

\begin{correctionbox}[Correction Exercice 1 : Chebyshev (Calcul de base)]
$P(|Y - \mu| \ge k) \le \frac{\sigma^2}{k^2}$.
$P(|Y - 50| \ge 10) \le \frac{16}{10^2} = \frac{16}{100} = 0.16$.
La probabilité est d'au maximum 16\%.
\end{correctionbox}

\begin{correctionbox}[Correction Exercice 2 : Chebyshev (En termes d'écarts-types)]
On pose $k = 5\sigma$.
$P(|Y - \mu| \ge 5\sigma) \le \frac{\sigma^2}{(5\sigma)^2} = \frac{\sigma^2}{25\sigma^2} = \frac{1}{25} = 0.04$.
La probabilité est d'au maximum 4\%.
\end{correctionbox}

\begin{correctionbox}[Correction Exercice 3 : Chebyshev (Application à $\bar{X}_n$)]
1.  $E[\bar{X}_{100}] = \mu = 100$.
    $\text{Var}(\bar{X}_{100}) = \frac{\sigma^2}{n} = \frac{400}{100} = 4$.
2.  On applique Chebyshev à $\bar{X}_{100}$ (avec $\mu=100, \sigma^2=4$) et $k=4$.
    $P(|\bar{X}_{100} - 100| \ge 4) \le \frac{\text{Var}(\bar{X}_{100})}{k^2} = \frac{4}{4^2} = \frac{4}{16} = 0.25$.
\end{correctionbox}

\begin{correctionbox}[Correction Exercice 4 : CLT (Distribution de $\bar{X}_n$)]
$E[\bar{X}_{64}] = \mu = 20$.
$\text{Var}(\bar{X}_{64}) = \frac{\sigma^2}{n} = \frac{16}{64} = 0.25$.
Selon le TCL, $\bar{X}_{64} \approx \mathcal{N}(20, 0.25)$.
\end{correctionbox}

\begin{correctionbox}[Correction Exercice 5 : CLT (Calcul de probabilité pour $\bar{X}_n$)]
On standardise $\bar{X}_{64} \approx \mathcal{N}(20, 0.25)$, donc $\sigma_{\bar{X}} = \sqrt{0.25} = 0.5$.
On cherche $P(\bar{X}_{64} \le 21)$.
$Z = \frac{\bar{X}_n - \mu}{\sigma_{\bar{X}}} = \frac{21 - 20}{0.5} = \frac{1}{0.5} = 2$.
$P(\bar{X}_{64} \le 21) = P(Z \le 2) = \Phi(2) \approx 0.9772$.
\end{correctionbox}

\begin{correctionbox}[Correction Exercice 6 : CLT (Calcul de probabilité pour $\bar{X}_n$)]
$\mu=50, \sigma=5, n=100$. $\bar{X}_{100} \approx \mathcal{N}(\mu, \sigma^2/n)$.
$\mu_{\bar{X}} = 50$. $\sigma_{\bar{X}} = \sigma / \sqrt{n} = 5 / \sqrt{100} = 0.5$.
On cherche $P(49 \le \bar{X}_{100} \le 51)$.
$Z_1 = \frac{49 - 50}{0.5} = -2$. $Z_2 = \frac{51 - 50}{0.5} = 2$.
$P(-2 \le Z \le 2) = \Phi(2) - \Phi(-2) = \Phi(2) - (1 - \Phi(2)) = 2\Phi(2) - 1$.
$P \approx 2(0.9772) - 1 = 1.9544 - 1 = 0.9544$. (La règle des 95\%).
\end{correctionbox}

\begin{correctionbox}[Correction Exercice 7 : CLT (Inverse pour $\bar{X}_n$)]
$\mu=10, \sigma^2=81, n=36$. $\bar{X}_{36} \approx \mathcal{N}(\mu, \sigma^2/n)$.
$\mu_{\bar{X}} = 10$. $\sigma_{\bar{X}} = \sigma / \sqrt{n} = 9 / \sqrt{36} = 9 / 6 = 1.5$.
On cherche $c$ tel que $P(\bar{X}_{36} \le c) \approx 0.9772$.
$P(Z \le \frac{c - 10}{1.5}) = 0.9772$.
D'après la table, le $z$ correspondant est $2.0$.
$\frac{c - 10}{1.5} = 2 \implies c - 10 = 3 \implies c = 13$.
\end{correctionbox}

\begin{correctionbox}[Correction Exercice 8 : CLT (Distribution de $S_n$)]
$E[S_{100}] = n\mu = 100 \times 10 = 1000$.
$\text{Var}(S_{100}) = n\sigma^2 = 100 \times 4 = 400$.
Selon le TCL, $S_{100} \approx \mathcal{N}(1000, 400)$.
\end{correctionbox}

\begin{correctionbox}[Correction Exercice 9 : CLT (Calcul de probabilité pour $S_n$)]
On standardise $S_{100} \approx \mathcal{N}(1000, 400)$, donc $\sigma_{S_n} = \sqrt{400} = 20$.
On cherche $P(S_{100} > 1020)$.
$Z = \frac{S_n - n\mu}{\sigma_{S_n}} = \frac{1020 - 1000}{20} = \frac{20}{20} = 1$.
$P(S_{100} > 1020) = P(Z > 1) = 1 - \Phi(1) \approx 1 - 0.8413 = 0.1587$.
\end{correctionbox}

\begin{correctionbox}[Correction Exercice 10 : CLT (Application $S_n$)]
$n=49, \mu=70, \sigma=14$. $S_{49} \approx \mathcal{N}(n\mu, n\sigma^2)$.
$E[S_{49}] = n\mu = 49 \times 70 = 3430$.
$\sigma_{S_n} = \sigma \sqrt{n} = 14 \times \sqrt{49} = 14 \times 7 = 98$.
On cherche $P(S_{49} > 3500)$.
$Z = \frac{3500 - 3430}{98} = \frac{70}{98} \approx 0.714$.
$P(Z > 0.714)$. (Valeur non fournie, mais le calcul est le Z-score).
\end{correctionbox}

\begin{correctionbox}[Correction Exercice 11 : CLT (Inverse pour $S_n$)]
$n=64, \mu=5, \sigma^2=1$. $S_{64} \approx \mathcal{N}(n\mu, n\sigma^2)$.
$E[S_{64}] = 64 \times 5 = 320$.
$\sigma_{S_n} = \sigma \sqrt{n} = 1 \times \sqrt{64} = 8$.
On cherche $c$ tel que $P(S_{64} > c) \approx 0.05$.
$P(Z > \frac{c - 320}{8}) = 0.05$.
$P(Z \le z) = 0.95$. D'après la table, $z \approx 1.645$.
$\frac{c - 320}{8} = 1.645 \implies c = 320 + 8(1.645) = 320 + 13.16 = 333.16$.
\end{correctionbox}

\begin{correctionbox}[Correction Exercice 12 : Règle d'Approximation]
$n=40, p=0.1$.
$np = 40 \times 0.1 = 4$.
$n(1-p) = 40 \times 0.9 = 36$.
Puisque $np=4$ est $< 10$, la règle n'est pas satisfaite. L'approximation normale n'est pas recommandée.
\end{correctionbox}

\begin{correctionbox}[Correction Exercice 13 : Règle d'Approximation]
$n=500, p=0.04$.
$np = 500 \times 0.04 = 20$.
$n(1-p) = 500 \times 0.96 = 480$.
Les deux conditions ($20 \ge 10$ et $480 \ge 10$) sont satisfaites. L'approximation est valide.
\end{correctionbox}

\begin{correctionbox}[Correction Exercice 14 : Paramètres d'Approximation]
$X \sim \text{Bin}(100, 0.3)$.
$\mu_Y = np = 100 \times 0.3 = 30$.
$\sigma_Y^2 = np(1-p) = 100 \times 0.3 \times 0.7 = 21$.
L'approximation est $Y \sim \mathcal{N}(30, 21)$.
\end{correctionbox}

\begin{correctionbox}[Correction Exercice 15 : Correction de Continuité (Règles)]
1. $P(X = 50) \approx P(49.5 \le Y \le 50.5)$
2. $P(X \ge 50) \approx P(Y \ge 49.5)$
3. $P(X \le 49) \approx P(Y \le 49.5)$
4. $P(X < 49) = P(X \le 48) \approx P(Y \le 48.5)$
\end{correctionbox}

\begin{correctionbox}[Correction Exercice 16 : Correction de Continuité (Règles)]
1. $P(X > 30) = P(X \ge 31) \approx P(Y \ge 30.5)$
2. $P(30 < X < 40) = P(31 \le X \le 39) \approx P(30.5 \le Y \le 39.5)$
3. $P(30 \le X \le 40) \approx P(29.5 \le Y \le 40.5)$
\end{correctionbox}

\begin{correctionbox}[Correction Exercice 17 : Binomiale (Calcul P(X=k))]
$X \sim \text{Bin}(100, 0.5)$. $\mu = 50$, $\sigma^2 = 25$, $\sigma = 5$.
On cherche $P(X=50) \approx P(49.5 \le Y \le 50.5)$.
$Z_1 = \frac{49.5 - 50}{5} = -0.1$. $Z_2 = \frac{50.5 - 50}{5} = 0.1$.
$P(-0.1 \le Z \le 0.1) = \Phi(0.1) - \Phi(-0.1) = \Phi(0.1) - (1 - \Phi(0.1)) = 2\Phi(0.1) - 1$.
$P \approx 2(0.5398) - 1 = 1.0796 - 1 = 0.0796$.
\end{correctionbox}

\begin{correctionbox}[Correction Exercice 18 : Binomiale (Calcul $P(X \ge k)$)]
$X \sim \text{Bin}(100, 0.5)$. $\mu = 50$, $\sigma = 5$.
On cherche $P(X \ge 60) \approx P(Y \ge 59.5)$.
$Z = \frac{59.5 - 50}{5} = \frac{9.5}{5} = 1.9$.
$P(Z \ge 1.9) = 1 - \Phi(1.9)$. (Valeur $\Phi(1.9)$ non fournie, mais $1-\Phi(1.96) \approx 0.025$).
\end{correctionbox}

\begin{correctionbox}[Correction Exercice 19 : Binomiale (Calcul $P(X \le k)$)]
$X \sim \text{Bin}(400, 0.2)$. $np = 80$, $n(1-p)=320$. (Règle OK).
$\mu = 80$. $\sigma^2 = 80 \times 0.8 = 64$. $\sigma = 8$.
On cherche $P(X \le 70) \approx P(Y \le 70.5)$.
$Z = \frac{70.5 - 80}{8} = \frac{-9.5}{8} \approx -1.19$.
$P(Z \le -1.19) = 1 - \Phi(1.19)$. (Valeur non fournie).
\end{correctionbox}

\begin{correctionbox}[Correction Exercice 20 : Binomiale (Calcul $P(X > k)$)]
$X \sim \text{Bin}(400, 0.2)$. $\mu = 80$, $\sigma = 8$.
On cherche $P(X > 92) = P(X \ge 93) \approx P(Y \ge 92.5)$.
$Z = \frac{92.5 - 80}{8} = \frac{12.5}{8} = 1.5625$.
$P(Z \ge 1.5625) \approx P(Z \ge 1.58) = 1 - \Phi(1.58) \approx 1 - 0.9429 = 0.0571$.
\end{correctionbox}

\begin{correctionbox}[Correction Exercice 21 : Binomiale (Calcul $P(k_1 \le X \le k_2)$)]
$X \sim \text{Bin}(100, 0.6)$. $np = 60$, $n(1-p)=40$. (Règle OK).
$\mu = 60$. $\sigma^2 = 60 \times 0.4 = 24$. $\sigma = \sqrt{24} \approx 4.9$.
On cherche $P(50 \le X \le 65) \approx P(49.5 \le Y \le 65.5)$.
$Z_1 = \frac{49.5 - 60}{4.9} = \frac{-10.5}{4.9} \approx -2.14$.
$Z_2 = \frac{65.5 - 60}{4.9} = \frac{5.5}{4.9} \approx 1.12$.
$P(-2.14 \le Z \le 1.12) = \Phi(1.12) - \Phi(-2.14) = \Phi(1.12) - (1 - \Phi(2.14))$. (Valeurs non fournies).
\end{correctionbox}

\begin{correctionbox}[Correction Exercice 22 : TCL vs Chebyshev]
$n=100, \mu=10, \sigma^2=25$. $\bar{X}_{100} \approx \mathcal{N}(10, 25/100=0.25)$. $\sigma_{\bar{X}} = 0.5$.
On cherche $P(|\bar{X}_{100} - 10| \ge 1)$.
1.  **Chebyshev** : $k=1, \text{Var}(\bar{X}_{100}) = 0.25$.
    $P \le \frac{\text{Var}(\bar{X}_{100})}{k^2} = \frac{0.25}{1^2} = 0.25$. (Borne $\le 25\%$).
2.  **CLT** : $P(|\bar{X}_{100} - 10| \ge 1) = P(Z \ge \frac{1}{0.5}) + P(Z \le \frac{-1}{0.5}) = P(Z \ge 2) + P(Z \le -2)$.
    $P = (1 - \Phi(2)) + \Phi(-2) = (1 - \Phi(2)) + (1 - \Phi(2)) = 2(1 - \Phi(2))$.
    $P \approx 2(1 - 0.9772) = 2(0.0228) = 0.0456$. (Probabilité $\approx 4.56\%$).
\end{correctionbox}

\begin{correctionbox}[Correction Exercice 23 : Taille d'Échantillon (CLT)]
$\bar{X}_n \approx \mathcal{N}(\mu, \sigma^2/n) = \mathcal{N}(0, 1/n)$. $\sigma_{\bar{X}} = 1/\sqrt{n}$.
On veut $P(|\bar{X}_n| \le 0.1) \ge 0.95$, soit $P(-0.1 \le \bar{X}_n \le 0.1) \ge 0.95$.
On standardise : $P\left( \frac{-0.1 - 0}{1/\sqrt{n}} \le Z \le \frac{0.1 - 0}{1/\sqrt{n}} \right) \ge 0.95$.
$P(-0.1\sqrt{n} \le Z \le 0.1\sqrt{n}) \ge 0.95$.
On sait que $P(-1.96 \le Z \le 1.96) = 0.95$.
On doit donc avoir $0.1\sqrt{n} \ge 1.96$.
$\sqrt{n} \ge 19.6 \implies n \ge (19.6)^2 = 384.16$.
Il faut $n=385$ échantillons au minimum.
\end{correctionbox}

\begin{correctionbox}[Correction Exercice 24 : LLN vs CLT]
$\mu=10, \sigma^2=100$.
1.  **$n=100$** : $\bar{X}_{100} \approx \mathcal{N}(10, 100/100=1)$. $\sigma_{\bar{X}} = 1$.
    $P(9 \le \bar{X}_{100} \le 11) = P(\frac{9-10}{1} \le Z \le \frac{11-10}{1}) = P(-1 \le Z \le 1)$.
    $P = 2\Phi(1) - 1 \approx 2(0.8413) - 1 = 0.6826$.
2.  **$n=10000$** : $\bar{X}_{10000} \approx \mathcal{N}(10, 100/10000=0.01)$. $\sigma_{\bar{X}} = 0.1$.
    $P(9 \le \bar{X}_{10000} \le 11) = P(\frac{9-10}{0.1} \le Z \le \frac{11-10}{0.1}) = P(-10 \le Z \le 10)$.
    $P \approx \Phi(10) - \Phi(-10) \approx 1 - (1-1) = 1$.
3.  **Illustration** : Le CLT montre \textit{comment} la LLN fonctionne. En augmentant $n$, la variance de $\bar{X}_n$ s'effondre (de 1 à 0.01), forçant la distribution de $\bar{X}_n$ à se concentrer massivement autour de $\mu=10$. La probabilité que $\bar{X}_n$ soit proche de $\mu$ tend vers 1.
\end{correctionbox}

\begin{correctionbox}[Correction Exercice 25 : Binomiale (Sans Correction)]
$X \sim \text{Bin}(400, 0.1)$. On approxime $P(X \le 30.5)$.
$\mu = np = 40$. $\sigma^2 = np(1-p) = 36$. $\sigma = 6$.
$Y \sim \mathcal{N}(40, 36)$.
On cherche $P(Y \le 30.5)$.
$Z = \frac{30.5 - 40}{6} = \frac{-9.5}{6} \approx -1.583$.
$P(Z \le -1.58) = \Phi(-1.58) = 1 - \Phi(1.58) \approx 1 - 0.9429 = 0.0571$.
(Note : C'est le même calcul que l'exercice 19, $P(X \le 30)$, car $P(X \le 30) \approx P(Y \le 30.5)$).
\end{correctionbox}

\subsection{Exercices Python}

Ces exercices utilisent les \textbf{Lois des Grands Nombres (LLN)} pour estimer des paramètres de population à partir de simulations. La LLN garantit que la moyenne d'échantillon ($\bar{X}_n$) converge vers la vraie espérance ($\mu$) lorsque $n$ devient grand.

Nous utiliserons les données de \texttt{yfinance} pour établir des "vraies" valeurs ($\mu, \sigma^2$), puis nous simulerons des échantillons plus petits pour voir comment la moyenne de l'échantillon ($\bar{X}_n$) s'approche de $\mu$.

\begin{codecell}
!pip install yfinance
import yfinance as yf
import pandas as pd
import numpy as np
import matplotlib.pyplot as plt
from scipy.stats import norm

# Definir les tickers et la periode
tickers = ["GOOG"]
start_date = "2010-01-01"
end_date = "2024-12-31"

# Telecharger les prix de cloture ajustes
data = yf.download(tickers, start=start_date, end=end_date)["Adj Close"]

# Calculer les rendements journaliers en pourcentage
returns = data.pct_change().dropna()

# 'returns' est notre DataFrame principal.
# Considerons cette grande serie de donnees comme notre "Population"
# pour les besoins de ces exercices.
population_mean = returns.mean()
population_var = returns.var()
population_std = returns.std()

print(f"--- Population (GOOG 2010-2024) ---")
print(f"Vraie Moyenne (mu) = {population_mean:.6f}")
print(f"Vraie Variance (sigma^2) = {population_var:.6f}")
\end{codecell}

\begin{exercicebox}[Exercice 1 : Vérification de la Loi Faible (WLLN)]
La WLLN dit que $P(|\bar{X}_n - \mu| > \epsilon) \to 0$ lorsque $n \to \infty$. Nous allons vérifier que la variance de $\bar{X}_n$ diminue avec $n$, ce qui est la clé de la preuve de Chebyshev.

La théorie dit : $\text{Var}(\bar{X}_n) = \frac{\sigma^2}{n}$.

\textbf{Votre tâche :}
\begin{enumerate}
    \item Utiliser $\sigma^2$ (la variance de la population) calculée ci-dessus.
    \item Calculer la variance \textbf{théorique} de la moyenne d'échantillon $\text{Var}(\bar{X}_n)$ pour $n=10$, $n=100$, et $n=1000$.
    \item (Conclusion) Comment la variance de notre estimateur $\bar{X}_n$ évolue-t-elle lorsque $n$ augmente ? Qu'est-ce que cela implique sur la précision de notre estimation ?
\end{enumerate}
\end{exercicebox}

\begin{exercicebox}[Exercice 2 : Inégalité de Chebyshev]
Chebyshev donne une borne universelle : $P(|\bar{X}_n - \mu| \ge k) \le \frac{\text{Var}(\bar{X}_n)}{k^2}$.
Utilisons $n=100$ et $\epsilon = 0.01$ (soit un écart de 1\% du rendement journalier).

\textbf{Votre tâche :}
\begin{enumerate}
    \item Utiliser $\text{Var}(\bar{X}_{100})$ calculée à l'exercice 1.
    \item Fixer $k = \epsilon = 0.01$.
    \item Calculer la borne supérieure de probabilité (le côté droit de l'inégalité).
    \item (Conclusion) Interpréter cette borne : "Pour un échantillon de 100 jours, la probabilité que notre moyenne d'échantillon soit erronée de plus de 1\% est, au maximum, de...".
\end{enumerate}
\end{exercicebox}

\begin{exercicebox}[Exercice 3 : Simulation de la Loi Forte (SLLN) - Trajectoire]
La SLLN dit que $P(\lim_{n \to \infty} \bar{X}_n = \mu) = 1$. Nous allons simuler une "trajectoire" de $\bar{X}_n$ pour visualiser cette convergence.

\textbf{Votre tâche :}
\begin{enumerate}
    \item Prendre les 1000 premiers rendements de la série \texttt{returns}.
    \item Calculer la moyenne d'échantillon cumulative $\bar{X}_n$ pour $n=1, 2, 3, \dots, 1000$.
    \item (Indice : utiliser \texttt{.expanding().mean()} de pandas).
    \item \textbf{(Plot)} Tracer $\bar{X}_n$ en fonction de $n$ (de 1 à 1000).
    \item \textbf{(Plot)} Tracer une ligne horizontale constante à la "vraie" moyenne $\mu$ (la \texttt{population\_mean}).
    \item (Conclusion) La trajectoire de $\bar{X}_n$ converge-t-elle vers $\mu$ ?
\end{enumerate}
\end{exercicebox}

\begin{exercicebox}[Exercice 4 : Méthode de Monte-Carlo (Estimation de Probabilité)]
Nous voulons estimer $p = P(X > 0.02)$, la probabilité d'un "gros jour positif" (rendement > 2\%).
La vraie valeur $p$ est la proportion empirique sur toute la "population".
L'estimation $\bar{Z}_n$ est la proportion sur un échantillon de $n$ jours.

\textbf{Votre tâche :}
\begin{enumerate}
    \item Calculer la "vraie" probabilité $p$ (notre $\mu$) en comptant la proportion de \texttt{returns > 0.02} sur tout le dataset.
    \item Simuler 500 expériences. Dans \textbf{chaque} expérience :
        \begin{itemize}
            \item Tirer un échantillon de $n=50$ jours (avec remise) de \texttt{returns}.
            \item Estimer $\bar{Z}_{50}$ (la proportion de jours $> 0.02$ dans cet échantillon).
        \end{itemize}
    \item \textbf{(Plot)} Tracer l'histogramme de vos 500 estimations $\bar{Z}_{50}$.
    \item \textbf{(Plot)} Ajouter une ligne verticale à la "vraie" moyenne $p$.
    \item (Conclusion) Les estimations sont-elles centrées autour de la vraie valeur, comme prédit par la LLN ?
\end{enumerate}
\end{exercicebox}

\begin{exercicebox}[Exercice 5 : Estimation de $\pi$ (Monte-Carlo Pur)]
Appliquons l'exemple du cours pour estimer $\pi$ en utilisant la LLN.
Nous cherchons $\mu = \pi/4$. Nous estimons $\mu$ par $\bar{Z}_n = \frac{\text{Points dans le cercle}}{n}$.

\textbf{Votre tâche (avec NumPy) :}
\begin{enumerate}
    \item Définir $n = 1,000,000$.
    \item Générer $n$ coordonnées $X \sim U(0, 1)$ et $n$ coordonnées $Y \sim U(0, 1)$.
    \item Calculer $Z_i = 1$ si $X_i^2 + Y_i^2 \le 1$, et $0$ sinon. (Indice : \texttt{np.where} ou une comparaison booléenne).
    \item Calculer $\bar{Z}_n$ (la moyenne de $Z$).
    \item Calculer votre estimation de $\pi \approx 4 \cdot \bar{Z}_n$.
    \item (Conclusion) Votre estimation est-elle proche de \texttt{math.pi} ?
\end{enumerate}
\end{exercicebox}
\newpage

\section{Le Théorème Central Limite (TCL)}

\subsection{Introduction : L'omniprésence de la loi normale}

Dans la section précédente, la Loi des Grands Nombres (LLN) nous a donné une garantie fondamentale : la moyenne d'échantillon $\bar{X}_n$ converge vers la vraie moyenne $\mu$ lorsque $n$ devient grand.
$$ \bar{X}_n \xrightarrow{p.s.} \mu $$
La LLN nous dit \textbf{où} la moyenne d'échantillon converge (vers la constante $\mu$), mais elle ne nous dit rien sur la \textit{forme} de la distribution de $\bar{X}_n$ autour de $\mu$ pour un $n$ grand, mais fini.

Le \textbf{Théorème Central Limite (TCL)} comble cette lacune. Il décrit la \textit{manière} dont $\bar{X}_n$ converge, en nous donnant la forme de sa distribution. C'est sans doute le théorème le plus important des statistiques.

\begin{intuitionbox}[L'Idée Fondamentale]
Intuitivement, ce résultat affirme qu'une \textbf{somme} d'un grand nombre de variables aléatoires indépendantes et identiquement distribuées (i.i.d.) tend, le plus souvent, à suivre une \textbf{loi normale} (aussi appelée loi de Laplace-Gauss ou "courbe en cloche").

Ce théorème et ses généralisations offrent une explication à l'omniprésence de la loi normale dans la nature. De nombreux phénomènes (la taille d'un individu, l'erreur de mesure d'un instrument, le bruit de fond d'un signal) sont le résultat de l'addition d'un très grand nombre de petites perturbations aléatoires. Le TCL nous dit que le résultat de cette somme sera, inévitablement, distribué selon une loi normale.
\end{intuitionbox}

\subsection{L'illustration : la somme des "Pile ou Face"}

Prenons l'exemple le plus simple pour illustrer ce phénomène : le jeu de "pile ou face".



\begin{examplebox}[Distribution de la Somme de $n$ Lancers]
Soit $X_i$ le résultat du $i$-ème lancer, avec $X_i = 1$ pour "Face" (probabilité 0,5) et $X_i = 0$ pour "Pile" (probabilité 0,5). La distribution d'origine (pour $n=1$) n'est pas du tout une courbe en cloche : c'est une distribution discrète avec deux bâtons de même hauteur.

Considérons la \textbf{somme} $S_n = X_1 + X_2 + \dots + X_n$, qui représente le nombre total de "Face" obtenus en $n$ lancers.

\begin{itemize}
    \item \textbf{Pour $n=1$ :} La distribution de $S_1$ est :
    \begin{itemize}
        \item Valeurs de la somme : \{0, 1\}
        \item Fréquences : \{0.5, 0.5\}
    \end{itemize}
    
    \item \textbf{Pour $n=2$ :} Les sommes possibles sont \{0, 1, 2\}. La distribution de $S_2$ est :
    \begin{itemize}
        \item Valeurs de la somme : \{0, 1, 2\}
        \item Fréquences : \{0.25, 0.5, 0.25\} (elle forme un triangle).
    \end{itemize}
    
    \item \textbf{Pour $n=3$ :} Les sommes possibles sont \{0, 1, 2, 3\}. La distribution de $S_3$ est :
    \begin{itemize}
        \item Valeurs de la somme : \{0, 1, 2, 3\}
        \item Fréquences : \{0.125, 0.375, 0.375, 0.125\}
    \end{itemize}
\end{itemize}

\begin{center}
\begin{tikzpicture}
  \begin{axis}[
    width=0.8\textwidth,
    height=0.5\textwidth,
    xlabel={Valeurs de la somme},
    title={Fonction de fréquence pour des tirages à pile ou face},
    grid=both,
    grid style={line width=.1pt, draw=gray!30},
    major grid style={line width=.2pt,draw=gray!50},
    domain=0:4,
    samples=100,
    enlargelimits=false,
    axis lines=middle,
    xmin=0, xmax=4,
    ymin=0, ymax=0.6
  ]
  % n=1
  \addplot [thick, color=blue, fill=blue!20, fill opacity=0.5] coordinates {(0,0.5) (1,0.5)} \closedcycle;
  % n=2
  \addplot [thick, color=green, fill=green!20, fill opacity=0.5] coordinates {(0,0.25) (1,0.5) (2,0.25)} \closedcycle;
  % n=3
  \addplot [thick, color=red, fill=red!20, fill opacity=0.5] coordinates {(0,0.125) (1,0.375) (2,0.375) (3,0.125)} \closedcycle;
  % n=4 (suggéré)
  \addplot [thick, color=black, fill=black!20, fill opacity=0.5] coordinates {(0,0.0625) (1,0.25) (2,0.375) (3,0.25) (4,0.0625)} \closedcycle;
  
  \legend{$n=1$,$n=2$,$n=3$,$n=4$}
  \end{axis}
\end{tikzpicture}
\end{center}

Graphiquement, on constate que plus le nombre de tirages $n$ augmente (par exemple, jusqu'à $n=12$), plus la courbe de fréquence (qui reste discrète) se rapproche d'une courbe en cloche symétrique, caractéristique de la loi normale.
\end{examplebox}

\subsection{Distribution de la population vs. Distribution d'échantillonnage}

Le point le plus remarquable du TCL est qu'il fonctionne \textit{quelle que soit} la distribution de départ.

\begin{intuitionbox}[Population vs. Échantillonnage]
Imaginez deux univers de distributions :

\begin{itemize}
    \item \textbf{1. La Distribution de la Population ($X_i$) :} C'est la loi de nos variables $X_i$ individuelles. Elle peut avoir \textbf{n'importe quelle forme} (par exemple, une distribution bimodale, asymétrique, ou uniforme). Cette distribution a une "vraie" moyenne $\mu$ et un "vrai" écart-type $\sigma$.
    
    \item \textbf{2. La Distribution d'Échantillonnage ($\bar{X}_n$) :} C'est la distribution de la \textit{moyenne} $\bar{X}_n = (X_1 + \dots + X_n)/n$, calculée sur des échantillons de taille $n$. C'est la distribution de "toutes les moyennes d'échantillon possibles".
\end{itemize}

Le TCL énonce la relation magique entre les deux :

\textbf{Quelle que soit la forme de la distribution de la population, plus la taille de l'échantillon $n$ croît, plus la distribution d'échantillonnage de la moyenne $\bar{X}_n$ est proche d'une loi normale (gaussienne).}

De plus, les paramètres de cette loi normale sont :
\begin{itemize}
    \item \textbf{Moyenne :} La distribution de $\bar{X}_n$ est centrée sur la même moyenne $\mu$ que la population.
    \item \textbf{Écart-type :} La distribution de $\bar{X}_n$ est beaucoup plus resserrée. Son écart-type (appelé "erreur standard") est $\sigma_{\bar{X}} = \frac{\sigma}{\sqrt{n}}$.
\end{itemize}
Cette dispersion $\sigma/\sqrt{n}$ qui tend vers 0 est la manifestation de la Loi des Grands Nombres. Le TCL précise que la \textit{forme} de cette convergence est gaussienne.
\end{intuitionbox}

\subsection{Énoncé formel du Théorème Central Limite}

Pour énoncer le théorème formellement, nous devons d'abord définir les propriétés de la somme $S_n$ et de la moyenne $\bar{X}_n$.

Soit $X_1, \dots, X_n$ des variables aléatoires i.i.d. avec $E[X_i] = \mu$ et $\text{Var}(X_i) = \sigma^2$.

\begin{itemize}
    \item \textbf{La Somme $S_n = \sum X_i$} :
    \begin{itemize}
        \item Espérance : $E[S_n] = E[\sum X_i] = \sum E[X_i] = n\mu$
        \item Variance : $\text{Var}(S_n) = \text{Var}(\sum X_i) = \sum \text{Var}(X_i) = n\sigma^2$
        \item Écart-type : $\sigma_{S_n} = \sqrt{n\sigma^2} = \sigma\sqrt{n}$
    \end{itemize}
    
    \item \textbf{La Moyenne $\bar{X}_n = S_n / n$} :
    \begin{itemize}
        \item Espérance : $E[\bar{X}_n] = E[S_n / n] = \frac{1}{n} E[S_n] = \frac{1}{n} (n\mu) = \mu$
        \item Variance : $\text{Var}(\bar{X}_n) = \text{Var}(S_n / n) = \frac{1}{n^2} \text{Var}(S_n) = \frac{1}{n^2} (n\sigma^2) = \frac{\sigma^2}{n}$
        \item Écart-type : $\sigma_{\bar{X}_n} = \sqrt{\sigma^2 / n} = \frac{\sigma}{\sqrt{n}}$
    \end{itemize}
\end{itemize}

Nous voyons que la distribution de $S_n$ s'étale (variance $\to \infty$) tandis que celle de $\bar{X}_n$ se contracte (variance $\to 0$). Pour étudier la \textit{forme} de la convergence, nous créons une variable "stable" en la centrant (soustrayant la moyenne) et en la réduisant (divisant par l'écart-type). C'est la variable $Z_n$.

\begin{theorembox}[Théorème Central Limite (Lindeberg-Lévy)]
Soit $X_1, X_2, \dots, X_n$ une suite de variables aléatoires \textbf{i.i.d.} (indépendantes et identiquement distribuées) suivant la même loi $D$.
Supposons que l'\textbf{espérance $\mu$} et l'\textbf{écart-type $\sigma$} de cette loi $D$ existent, sont finis, et $\sigma \neq 0$.

Considérons la variable aléatoire standardisée $Z_n$ :
$$ Z_n = \frac{S_n - E[S_n]}{\sigma_{S_n}} = \frac{S_n - n\mu}{\sigma\sqrt{n}} $$
Cette variable est équivalente à la moyenne standardisée :
$$ Z_n = \frac{\bar{X}_n - E[\bar{X}_n]}{\sigma_{\bar{X}_n}} = \frac{\bar{X}_n - \mu}{\sigma / \sqrt{n}} $$
(Pour tout $n$, $Z_n$ est une variable centrée-réduite : $E[Z_n] = 0$ et $\text{Var}(Z_n) = 1$).

Alors, la suite de variables aléatoires $Z_1, Z_2, \dots, Z_n, \dots$ \textbf{converge en loi} vers une variable aléatoire $Z$ qui suit la \textbf{loi normale centrée réduite $N(0, 1)$}, lorsque $n$ tend vers l'infini.

Cela signifie que si $\Phi$ est la fonction de répartition de la loi $N(0, 1)$, alors pour tout réel $z$ :
$$ \lim_{n \to \infty} P(Z_n \le z) = \lim_{n \to \infty} P\left( \frac{\bar{X}_n - \mu}{\sigma/\sqrt{n}} \le z \right) = \Phi(z) $$
\end{theorembox}

\subsection{Applications Pratiques du TCL}

Le TCL n'est pas seulement une curiosité mathématique ; c'est le fondement de l'inférence statistique. Voici comment l'appliquer concrètement pour résoudre des problèmes.

\begin{examplebox}[La taille des individus]
\textbf{Contexte :} La taille des individus dans une population suit une courbe en cloche. Pourquoi ? Car elle est la \textbf{somme} de milliers de petites influences (gènes, nutrition, etc.). Le TCL s'applique.

\textbf{Données :} Supposons que dans une population, la taille $X$ des individus ait une espérance $\mu = 175$ cm et un écart-type $\sigma = 8$ cm. (Note : la loi de $X$ n'est pas forcément normale, même si en pratique elle l'est).

\textbf{Problème :} On prélève un échantillon aléatoire de $n=64$ individus. Quelle est la probabilité que la \textbf{moyenne de cet échantillon} ($\bar{X}_{64}$) soit supérieure à 177 cm ?

\textbf{Solution :}
\begin{enumerate}
    \item \textbf{Identifier les paramètres :}
    \begin{itemize}
        \item Moyenne de la population : $\mu = 175$ cm
        \item Écart-type de la population : $\sigma = 8$ cm
        \item Taille de l'échantillon : $n = 64$
    \end{itemize}
    
    \item \textbf{Appliquer le TCL :}
    Puisque $n=64$ est grand (généralement $n \ge 30$ est suffisant), le TCL s'applique. La distribution d'échantillonnage de la moyenne $\bar{X}_n$ suit approximativement une loi normale.
    $$ \bar{X}_n \approx N\left(\mu, \frac{\sigma^2}{n}\right) $$
    
    \item \textbf{Calculer les paramètres de la loi normale de $\bar{X}_n$ :}
    \begin{itemize}
        \item Espérance de $\bar{X}_n$ : $E[\bar{X}_n] = \mu = 175$ cm.
        \item Écart-type de $\bar{X}_n$ (appelé "Erreur Standard") :
        $$ \sigma_{\bar{X}_n} = \frac{\sigma}{\sqrt{n}} = \frac{8}{\sqrt{64}} = \frac{8}{8} = 1 \text{ cm} $$
    \end{itemize}
    Donc, $\bar{X}_{64} \approx N(175, 1^2)$.
    
    \item \textbf{Standardiser (Calculer le Z-score) :}
    Nous cherchons $P(\bar{X}_{64} > 177)$. Nous transformons cette valeur en un score $Z$ pour utiliser la loi normale centrée réduite $N(0, 1)$.
    $$ Z = \frac{\bar{X}_n - \mu}{\sigma_{\bar{X}_n}} = \frac{177 - 175}{1} = 2 $$
    
    \item \textbf{Trouver la probabilité :}
    Chercher $P(\bar{X}_{64} > 177)$ revient à chercher $P(Z > 2)$.
    En utilisant la table de la loi normale (ou une calculatrice) :
    $$ P(Z > 2) = 1 - P(Z \le 2) = 1 - \Phi(2) $$
    Sachant que $\Phi(2) \approx 0.9772$,
    $$ P(Z > 2) = 1 - 0.9772 = 0.0228 $$
\end{enumerate}
\textbf{Conclusion :} Il y a environ 2.28\% de chances qu'un échantillon de 64 personnes ait une taille moyenne supérieure à 177 cm.
\end{examplebox}

\begin{examplebox}[Remplissage de bouteilles]
\textbf{Contexte :} Une machine remplit des bouteilles de soda. Le volume versé $X_i$ fluctue légèrement. La loi de $X_i$ est inconnue.

\textbf{Données :} La machine est réglée pour verser en moyenne $\mu = 500$ ml. L'écart-type du processus est connu et vaut $\sigma = 6$ ml. Pour un contrôle, on prélève un échantillon de $n=36$ bouteilles.

\textbf{Problème :} On considère que la machine est déréglée si la moyenne de l'échantillon $\bar{X}_{36}$ est inférieure à 498 ml. Quelle est la probabilité d'une "fausse alarme" (c'est-à-dire, la machine fonctionne bien à $\mu=500$, mais l'échantillon a une moyenne $\bar{X}_{36} < 498$) ?

\textbf{Solution :}
\begin{enumerate}
    \item \textbf{Identifier les paramètres :}
    $\mu = 500$ ml, $\sigma = 6$ ml, $n = 36$.
    
    \item \textbf{Appliquer le TCL :}
    $n=36 \ge 30$, donc le TCL s'applique.
    $$ \bar{X}_{36} \approx N\left(\mu, \frac{\sigma^2}{n}\right) $$
    
    \item \textbf{Calculer les paramètres de $\bar{X}_{36}$ :}
    \begin{itemize}
        \item Espérance : $E[\bar{X}_{36}] = \mu = 500$ ml.
        \item Erreur Standard : $\sigma_{\bar{X}} = \frac{\sigma}{\sqrt{n}} = \frac{6}{\sqrt{36}} = \frac{6}{6} = 1$ ml.
    \end{itemize}
    Donc, $\bar{X}_{36} \approx N(500, 1^2)$.
    
    \item \textbf{Standardiser (Calculer le Z-score) :}
    Nous cherchons la probabilité $P(\bar{X}_{36} < 498)$.
    $$ Z = \frac{\bar{X}_n - \mu}{\sigma_{\bar{X}_n}} = \frac{498 - 500}{1} = -2 $$
    
    \item \textbf{Trouver la probabilité :}
    Chercher $P(\bar{X}_{36} < 498)$ revient à chercher $P(Z < -2)$.
    $$ P(Z < -2) = \Phi(-2) $$
    Par symétrie de la loi normale, $\Phi(-z) = 1 - \Phi(z)$.
    $$ P(Z < -2) = 1 - \Phi(2) = 1 - 0.9772 = 0.0228 $$
\end{enumerate}
\textbf{Conclusion :} Il y a 2.28\% de chances d'avoir une fausse alarme, c'est-à-dire de croire à tort que la machine est déréglée alors qu'elle fonctionne normalement.
\end{examplebox}

\begin{examplebox}[Rendement d'un portefeuille (sur la Somme)]
\textbf{Contexte :} Le rendement quotidien $X_i$ d'un actif est très volatile. On s'intéresse au rendement annuel \textbf{total}, qui est la \textbf{somme} des rendements quotidiens.

\textbf{Données :} Supposons que le rendement quotidien $X_i$ ait une espérance $\mu = 0.04\%$ et un écart-type $\sigma = 1\%$. (La loi de $X_i$ est inconnue, mais $\mu$ et $\sigma$ existent). Il y a $n=252$ jours de trading dans l'année.

\textbf{Problème :} Quelle est la probabilité que le rendement annuel total $S_{252} = X_1 + \dots + X_{252}$ soit négatif (inférieur à 0) ?

\textbf{Solution :}
\begin{enumerate}
    \item \textbf{Identifier les paramètres (pour une seule v.a. $X_i$) :}
    $\mu = 0.0004$, $\sigma = 0.01$, $n = 252$.
    
    \item \textbf{Appliquer le TCL (pour la somme $S_n$) :}
    $n=252$ est grand. Le TCL s'applique à la somme $S_n$.
    $$ S_n \approx N\left(n\mu, n\sigma^2\right) $$
    
    \item \textbf{Calculer les paramètres de la loi normale de $S_{252}$ :}
    \begin{itemize}
        \item Espérance de $S_{252}$ : $E[S_n] = n\mu = 252 \times 0.0004 = 0.1008$ (soit 10.08\%).
        \item Variance de $S_{252}$ : $\text{Var}(S_n) = n\sigma^2 = 252 \times (0.01)^2 = 252 \times 0.0001 = 0.0252$.
        \item Écart-type de $S_{252}$ : $\sigma_{S_n} = \sqrt{n\sigma^2} = \sqrt{0.0252} \approx 0.1587$ (soit 15.87\%).
    \end{itemize}
    Donc, $S_{252} \approx N(0.1008, 0.1587^2)$.
    
    \item \textbf{Standardiser (Calculer le Z-score) :}
    Nous cherchons $P(S_{252} < 0)$.
    $$ Z = \frac{S_n - E[S_n]}{\sigma_{S_n}} = \frac{0 - 0.1008}{0.1587} \approx -0.635 $$
    
    \item \textbf{Trouver la probabilité :}
    Chercher $P(S_{252} < 0)$ revient à chercher $P(Z < -0.635)$.
    $$ P(Z < -0.635) = \Phi(-0.635) = 1 - \Phi(0.635) $$
    En interpolant dans la table, $\Phi(0.635) \approx 0.7373$.
    $$ P(Z < -0.635) \approx 1 - 0.7373 = 0.2627 $$
\end{enumerate}
\textbf{Conclusion :} Malgré une espérance de rendement quotidien positive, il y a environ 26.3\% de chances que le rendement annuel total soit négatif.
\end{examplebox}

\begin{examplebox}[Estimation d'une proportion (Marge d'erreur)]
\textbf{Contexte :} On veut estimer la proportion $p$ de votants qui approuvent un candidat. On modélise chaque personne $i$ par une variable de Bernoulli $X_i$ (1 si "oui", 0 si "non").
L'espérance de la population est $\mu = E[X_i] = p$.
La variance de la population est $\sigma^2 = \text{Var}(X_i) = p(1-p)$.
Le résultat du sondage est la moyenne d'échantillon $\bar{X}_n = \hat{p}$ (la proportion observée).

\textbf{Données :} On sonde $n=1000$ personnes. Le résultat est que 540 personnes disent "oui". Donc $\hat{p} = 540/1000 = 0.54$.

\textbf{Problème :} Calculer l'intervalle de confiance à 95\% pour la vraie proportion $p$ (la fameuse "marge d'erreur").

\textbf{Solution :}
\begin{enumerate}
    \item \textbf{Appliquer le TCL :}
    $n=1000$ est grand. Le TCL nous dit que la proportion d'échantillon $\hat{p} = \bar{X}_n$ suit une loi normale :
    $$ \hat{p} \approx N\left(p, \frac{p(1-p)}{n}\right) $$
    
    \item \textbf{Formule de l'Intervalle de Confiance :}
    Un intervalle de confiance à 95\% est centré sur notre estimation $\hat{p}$ et s'étend de $\pm 1.96$ erreurs standard (car $P(-1.96 \le Z \le 1.96) = 0.95$).
    $$ I.C._{95\%} = \left[ \hat{p} - 1.96 \cdot \sigma_{\hat{p}} \ ; \ \hat{p} + 1.96 \cdot \sigma_{\hat{p}} \right] $$
    où $\sigma_{\hat{p}} = \sqrt{p(1-p)/n}$.
    
    \item \textbf{Estimer l'Erreur Standard :}
    Problème : nous ne connaissons pas $p$ (c'est ce que nous cherchons !). Nous ne pouvons donc pas calculer $\sigma_{\hat{p}}$.
    \textbf{Solution :} Nous l'estimons en utilisant notre meilleur estimateur pour $p$, qui est $\hat{p} = 0.54$.
    $$ \text{Erreur Standard Estimée (SE)} = \sqrt{\frac{\hat{p}(1-\hat{p})}{n}} $$
    $$ SE = \sqrt{\frac{0.54 \times (1 - 0.54)}{1000}} = \sqrt{\frac{0.54 \times 0.46}{1000}} = \sqrt{\frac{0.2484}{1000}} \approx \sqrt{0.0002484} \approx 0.01576 $$
    
    \item \textbf{Calculer la Marge d'Erreur :}
    La marge d'erreur (ME) est la demi-largeur de l'intervalle.
    $$ ME = 1.96 \times SE = 1.96 \times 0.01576 \approx 0.0309 $$
    
    \item \textbf{Construire l'Intervalle :}
    $$ I.C._{95\%} = [ 0.54 - 0.0309 \ ; \ 0.54 + 0.0309 ] = [ 0.5091 \ ; \ 0.5709 ] $$
\end{enumerate}
\textbf{Conclusion :} Avec 54\% d'intentions de vote sur un échantillon de 1000 personnes, nous sommes confiants à 95\% que la vraie proportion $p$ dans la population se situe entre 50.9\% et 57.1\%. La marge d'erreur du sondage est de $\pm 3.1\%$.
\end{examplebox}
\subsection{Exercices}

\textit{Pour tous les exercices de calcul, vous pouvez utiliser les valeurs (arrondies) suivantes pour la fonction de répartition de la loi normale standard $\Phi(z) = P(Z \le z)$ :}
\begin{itemize}
    \item $\Phi(0) = 0.5$
    \item $\Phi(1.0) \approx 0.8413$
    \item $\Phi(1.5) \approx 0.9332$
    \item $\Phi(1.96) \approx 0.975$
    \item $\Phi(2.0) \approx 0.9772$
    \item $\Phi(2.5) \approx 0.9938$
    \item $\Phi(3.0) \approx 0.9987$
\end{itemize}
\textit{Et rappelez-vous la propriété de symétrie : $\Phi(-z) = 1 - \Phi(z)$.}

% --- Section 1 : Paramètres de S_n et X_n barre ---

\begin{exercicebox}[Exercice 1 : Paramètres de $\bar{X}_n$]
Soit $X_i$ une suite de v.a. i.i.d. avec $\mu=50$ et $\sigma^2=100$. Soit $n=25$.
Calculez $E[\bar{X}_{25}]$ et $\text{Var}(\bar{X}_{25})$.
\end{exercicebox}

\begin{exercicebox}[Exercice 2 : Paramètres de $S_n$]
Soit $X_i$ une suite de v.a. i.i.d. avec $\mu=10$ et $\sigma=3$. Soit $n=16$.
Calculez $E[S_{16}]$ et l'écart-type $\sigma_{S_{16}}$.
\end{exercicebox}

\begin{exercicebox}[Exercice 3 : Paramètres de $\bar{X}_n$ (bis)]
Soit $X_i$ une suite de v.a. i.i.d. avec $\mu=70$ et $\sigma=10$. Soit $n=400$.
Calculez $E[\bar{X}_{400}]$ et l'écart-type $\sigma_{\bar{X}_{400}}$.
\end{exercicebox}

\begin{exercicebox}[Exercice 4 : Paramètres de $S_n$ (bis)]
Soit $X_i$ une suite de v.a. i.i.d. avec $\mu=0.5$ et $\sigma^2=0.01$. Soit $n=64$.
Calculez $E[S_{64}]$ et $\text{Var}(S_{64})$.
\end{exercicebox}

\begin{exercicebox}[Exercice 5 : Retrouver $\sigma^2$ (via $\bar{X}_n$)]
La moyenne d'échantillon $\bar{X}_n$ de $n=49$ observations a une variance $\text{Var}(\bar{X}_{49}) = 2$.
Quelle est la variance $\sigma^2$ de la population d'origine ?
\end{exercicebox}

\begin{exercicebox}[Exercice 6 : Retrouver $n$ (via $S_n$)]
La somme $S_n$ d'observations i.i.d. a une variance $\text{Var}(S_n) = 300$. La variance de la population est $\sigma^2 = 12$.
Quelle est la taille de l'échantillon $n$ ?
\end{exercicebox}

% --- Section 2 : Calculs de probabilité (Forme $\bar{X}_n$) ---

\begin{exercicebox}[Exercice 7 : CLT pour $\bar{X}_n$ (Queue Droite)]
Soit $X_i$ i.i.d. avec $\mu=100$ et $\sigma=15$. Soit $n=36$.
Calculez $P(\bar{X}_{36} > 105)$.
\end{exercicebox}

\begin{exercicebox}[Exercice 8 : CLT pour $\bar{X}_n$ (Queue Gauche)]
Soit $X_i$ i.i.d. avec $\mu=50$ et $\sigma=8$. Soit $n=64$.
Calculez $P(\bar{X}_{64} < 48)$.
\end{exercicebox}

\begin{exercicebox}[Exercice 9 : CLT pour $\bar{X}_n$ (Intervalle)]
Soit $X_i$ i.i.d. avec $\mu=20$ et $\sigma=5$. Soit $n=100$.
Calculez $P(19 \le \bar{X}_{100} \le 21.25)$.
\end{exercicebox}

\begin{exercicebox}[Exercice 10 : Application $\bar{X}_n$ (Bouteilles)]
Une machine remplit des bouteilles avec $\mu=500$ ml et $\sigma=6$ ml. On prend $n=36$ bouteilles.
Calculez $P(\bar{X}_{36} > 501.5)$.
\end{exercicebox}

\begin{exercicebox}[Exercice 11 : Application $\bar{X}_n$ (Tailles)]
La taille des individus a $\mu=175$ cm et $\sigma=8$ cm. On prend $n=64$ individus.
Calculez $P(\bar{X}_{64} < 173)$.
\end{exercicebox}

\begin{exercicebox}[Exercice 12 : Application $\bar{X}_n$ (Notes)]
Les notes à un examen ont $\mu=70$ et $\sigma=12$. Une classe de $n=36$ étudiants est un échantillon.
Calculez $P(\bar{X}_{36} < 67)$.
\end{exercicebox}

% --- Section 3 : Calculs de probabilité (Forme $S_n$) ---

\begin{exercicebox}[Exercice 13 : CLT pour $S_n$ (Queue Droite)]
Soit $X_i$ i.i.d. avec $\mu=10$ et $\sigma=2$. Soit $n=100$.
Calculez $P(S_{100} > 1020)$.
\end{exercicebox}

\begin{exercicebox}[Exercice 14 : CLT pour $S_n$ (Queue Gauche)]
Soit $X_i$ i.i.d. avec $\mu=5$ et $\sigma=4$. Soit $n=64$.
Calculez $P(S_{64} \le 304)$.
\end{exercicebox}

\begin{exercicebox}[Exercice 15 : CLT pour $S_n$ (Intervalle)]
Soit $X_i$ i.i.d. avec $\mu=2$ et $\sigma=3$. Soit $n=36$.
Calculez $P(S_{36} > 90)$.
\end{exercicebox}

\begin{exercicebox}[Exercice 16 : Application $S_n$ (Ascenseur)]
Un ascenseur transporte $n=49$ personnes. Poids : $\mu=70$kg, $\sigma=14$kg.
Calculez $P(S_{49} > 3528)$.
\end{exercicebox}

\begin{exercicebox}[Exercice 17 : Application $S_n$ (Rendement)]
Le rendement quotidien $X_i$ a $\mu=0.001$ et $\sigma=0.01$. Soit $n=100$.
Calculez $P(S_{100} > 0.2)$.
\end{exercicebox}

\begin{exercicebox}[Exercice 18 : Application $S_n$ (Rendement)]
En utilisant les données de l'exercice 17, calculez $P(S_{100} < 0)$.
\end{exercicebox}

% --- Section 4 : Problèmes Inverses (CLT) ---

\begin{exercicebox}[Exercice 19 : Inverse (Trouver $c$ pour $\bar{X}_n$)]
Soit $X_i$ i.i.d. avec $\mu=50$ et $\sigma=10$. Soit $n=100$.
Trouvez la valeur $c$ telle que $P(\bar{X}_{100} \le c) \approx 0.8413$.
\end{exercicebox}

\begin{exercicebox}[Exercice 20 : Inverse (Trouver $c$ pour $S_n$)]
Soit $X_i$ i.i.d. avec $\mu=10$ et $\sigma=3$. Soit $n=36$.
Trouvez la valeur $c$ telle que $P(S_{36} \le c) \approx 0.0013$.
\end{exercicebox}

\begin{exercicebox}[Exercice 21 : Inverse (Taille d'échantillon $n$)]
Une population a $\mu=0$ et $\sigma=10$.
Quelle taille d'échantillon $n$ faut-il pour que $P(|\bar{X}_n| \le 1) \ge 0.95$ ?
(Indice : $P(-1.96 \le Z \le 1.96) = 0.95$).
\end{exercicebox}

\begin{exercicebox}[Exercice 22 : Inverse (Taille d'échantillon $n$)]
Une population a $\mu=100$ et $\sigma=20$.
Quelle taille d'échantillon $n$ faut-il pour que $P(\bar{X}_n \ge 102) \le 0.0228$ ?
\end{exercicebox}

% --- Section 5 : Proportions (Application de Bernoulli) ---

\begin{exercicebox}[Exercice 23 : Paramètres (Proportion)]
On sonde $n=400$ personnes. La vraie proportion $p$ est $0.25$. On modélise $X_i \sim \text{Bern}(p)$.
\begin{enumerate}
    \item Calculez $\mu = E[X_i]$ et $\sigma^2 = \text{Var}(X_i)$.
    \item Calculez $E[\hat{p}]$ et $\text{Var}(\hat{p})$ (où $\hat{p} = \bar{X}_n$).
\end{enumerate}
\end{exercicebox}

\begin{exercicebox}[Exercice 24 : Calcul (Proportion)]
On sonde $n=100$ personnes. La vraie proportion est $p=0.5$.
Calculez la probabilité que la proportion observée $\hat{p}$ soit supérieure à 0.6, $P(\hat{p} > 0.6)$.
\end{exercicebox}

\begin{exercicebox}[Exercice 25 : Marge d'Erreur (Proportion)]
Un sondage sur $n=1000$ personnes donne un résultat $\hat{p} = 0.54$.
Calculez la marge d'erreur à 95\% (c'est-à-dire $1.96 \times SE_{\hat{p}}$).
\end{exercicebox}


\subsection{Corrections des Exercices}

\begin{correctionbox}[Correction Exercice 1 : Paramètres de $\bar{X}_n$]
$E[\bar{X}_{25}] = \mu = 50$.
$\text{Var}(\bar{X}_{25}) = \frac{\sigma^2}{n} = \frac{100}{25} = 4$.
\end{correctionbox}

\begin{correctionbox}[Correction Exercice 2 : Paramètres de $S_n$]
$E[S_{16}] = n\mu = 16 \times 10 = 160$.
$\text{Var}(S_{16}) = n\sigma^2 = 16 \times 3^2 = 16 \times 9 = 144$.
$\sigma_{S_{16}} = \sqrt{\text{Var}(S_{16})} = \sqrt{144} = 12$.
\end{correctionbox}

\begin{correctionbox}[Correction Exercice 3 : Paramètres de $\bar{X}_n$ (bis)]
$E[\bar{X}_{400}] = \mu = 70$.
$\sigma_{\bar{X}_{400}} = \frac{\sigma}{\sqrt{n}} = \frac{10}{\sqrt{400}} = \frac{10}{20} = 0.5$.
\end{correctionbox}

\begin{correctionbox}[Correction Exercice 4 : Paramètres de $S_n$ (bis)]
$E[S_{64}] = n\mu = 64 \times 0.5 = 32$.
$\text{Var}(S_{64}) = n\sigma^2 = 64 \times 0.01 = 0.64$.
\end{correctionbox}

\begin{correctionbox}[Correction Exercice 5 : Retrouver $\sigma^2$ (via $\bar{X}_n$)]
$\text{Var}(\bar{X}_n) = \frac{\sigma^2}{n} \implies \sigma^2 = n \cdot \text{Var}(\bar{X}_n)$.
$\sigma^2 = 49 \times 2 = 98$.
\end{correctionbox}

\begin{correctionbox}[Correction Exercice 6 : Retrouver $n$ (via $S_n$)]
$\text{Var}(S_n) = n\sigma^2 \implies n = \frac{\text{Var}(S_n)}{\sigma^2}$.
$n = \frac{300}{12} = 25$.
\end{correctionbox}

\begin{correctionbox}[Correction Exercice 7 : CLT pour $\bar{X}_n$ (Queue Droite)]
$\mu=100, \sigma=15, n=36$. $\bar{X}_{36} \approx \mathcal{N}(\mu, \sigma^2/n)$.
$\mu_{\bar{X}} = 100$. $\sigma_{\bar{X}} = \frac{15}{\sqrt{36}} = \frac{15}{6} = 2.5$.
On cherche $P(\bar{X}_{36} > 105)$.
$Z = \frac{105 - 100}{2.5} = \frac{5}{2.5} = 2$.
$P(Z > 2) = 1 - \Phi(2) \approx 1 - 0.9772 = 0.0228$.
\end{correctionbox}

\begin{correctionbox}[Correction Exercice 8 : CLT pour $\bar{X}_n$ (Queue Gauche)]
$\mu=50, \sigma=8, n=64$. $\bar{X}_{64} \approx \mathcal{N}(\mu, \sigma^2/n)$.
$\mu_{\bar{X}} = 50$. $\sigma_{\bar{X}} = \frac{8}{\sqrt{64}} = \frac{8}{8} = 1$.
On cherche $P(\bar{X}_{64} < 48)$.
$Z = \frac{48 - 50}{1} = -2$.
$P(Z < -2) = \Phi(-2) = 1 - \Phi(2) \approx 1 - 0.9772 = 0.0228$.
\end{correctionbox}

\begin{correctionbox}[Correction Exercice 9 : CLT pour $\bar{X}_n$ (Intervalle)]
$\mu=20, \sigma=5, n=100$. $\bar{X}_{100} \approx \mathcal{N}(\mu, \sigma^2/n)$.
$\mu_{\bar{X}} = 20$. $\sigma_{\bar{X}} = \frac{5}{\sqrt{100}} = \frac{5}{10} = 0.5$.
On cherche $P(19 \le \bar{X}_{100} \le 21.25)$.
$Z_1 = \frac{19 - 20}{0.5} = -2$.
$Z_2 = \frac{21.25 - 20}{0.5} = \frac{1.25}{0.5} = 2.5$.
$P(-2 \le Z \le 2.5) = \Phi(2.5) - \Phi(-2) = \Phi(2.5) - (1 - \Phi(2))$.
$P \approx 0.9938 - (1 - 0.9772) = 0.9938 - 0.0228 = 0.9710$.
\end{correctionbox}

\begin{correctionbox}[Correction Exercice 10 : Application $\bar{X}_n$ (Bouteilles)]
$\mu=500, \sigma=6, n=36$. $\bar{X}_{36} \approx \mathcal{N}(\mu, \sigma^2/n)$.
$\mu_{\bar{X}} = 500$. $\sigma_{\bar{X}} = \frac{6}{\sqrt{36}} = \frac{6}{6} = 1$.
On cherche $P(\bar{X}_{36} > 501.5)$.
$Z = \frac{501.5 - 500}{1} = 1.5$.
$P(Z > 1.5) = 1 - \Phi(1.5) \approx 1 - 0.9332 = 0.0668$.
\end{correctionbox}

\begin{correctionbox}[Correction Exercice 11 : Application $\bar{X}_n$ (Tailles)]
$\mu=175, \sigma=8, n=64$. $\bar{X}_{64} \approx \mathcal{N}(\mu, \sigma^2/n)$.
$\mu_{\bar{X}} = 175$. $\sigma_{\bar{X}} = \frac{8}{\sqrt{64}} = \frac{8}{8} = 1$.
On cherche $P(\bar{X}_{64} < 173)$.
$Z = \frac{173 - 175}{1} = -2$.
$P(Z < -2) = \Phi(-2) = 1 - \Phi(2) \approx 1 - 0.9772 = 0.0228$.
\end{correctionbox}

\begin{correctionbox}[Correction Exercice 12 : Application $\bar{X}_n$ (Notes)]
$\mu=70, \sigma=12, n=36$. $\bar{X}_{36} \approx \mathcal{N}(\mu, \sigma^2/n)$.
$\mu_{\bar{X}} = 70$. $\sigma_{\bar{X}} = \frac{12}{\sqrt{36}} = \frac{12}{6} = 2$.
On cherche $P(\bar{X}_{36} < 67)$.
$Z = \frac{67 - 70}{2} = -1.5$.
$P(Z < -1.5) = \Phi(-1.5) = 1 - \Phi(1.5) \approx 1 - 0.9332 = 0.0668$.
\end{correctionbox}

\begin{correctionbox}[Correction Exercice 13 : CLT pour $S_n$ (Queue Droite)]
$\mu=10, \sigma=2, n=100$. $S_{100} \approx \mathcal{N}(n\mu, n\sigma^2)$.
$E[S_{100}] = 100 \times 10 = 1000$.
$\sigma_{S_n} = \sigma\sqrt{n} = 2 \times \sqrt{100} = 2 \times 10 = 20$.
On cherche $P(S_{100} > 1020)$.
$Z = \frac{1020 - 1000}{20} = \frac{20}{20} = 1$.
$P(Z > 1) = 1 - \Phi(1) \approx 1 - 0.8413 = 0.1587$.
\end{correctionbox}

\begin{correctionbox}[Correction Exercice 14 : CLT pour $S_n$ (Queue Gauche)]
$\mu=5, \sigma=4, n=64$. $S_{64} \approx \mathcal{N}(n\mu, n\sigma^2)$.
$E[S_{64}] = 64 \times 5 = 320$.
$\sigma_{S_n} = \sigma\sqrt{n} = 4 \times \sqrt{64} = 4 \times 8 = 32$.
On cherche $P(S_{64} \le 304)$.
$Z = \frac{304 - 320}{32} = \frac{-16}{32} = -0.5$.
$P(Z \le -0.5) = \Phi(-0.5) = 1 - \Phi(0.5)$. (Valeur non fournie).
\end{correctionbox}

\begin{correctionbox}[Correction Exercice 15 : CLT pour $S_n$ (Intervalle)]
$\mu=2, \sigma=3, n=36$. $S_{36} \approx \mathcal{N}(n\mu, n\sigma^2)$.
$E[S_{36}] = 36 \times 2 = 72$.
$\sigma_{S_n} = \sigma\sqrt{n} = 3 \times \sqrt{36} = 3 \times 6 = 18$.
On cherche $P(S_{36} > 90)$.
$Z = \frac{90 - 72}{18} = \frac{18}{18} = 1$.
$P(Z > 1) = 1 - \Phi(1) \approx 1 - 0.8413 = 0.1587$.
\end{correctionbox}

\begin{correctionbox}[Correction Exercice 16 : Application $S_n$ (Ascenseur)]
$\mu=70, \sigma=14, n=49$. $S_{49} \approx \mathcal{N}(n\mu, n\sigma^2)$.
$E[S_{49}] = 49 \times 70 = 3430$.
$\sigma_{S_n} = \sigma\sqrt{n} = 14 \times \sqrt{49} = 14 \times 7 = 98$.
On cherche $P(S_{49} > 3528)$.
$Z = \frac{3528 - 3430}{98} = \frac{98}{98} = 1$.
$P(Z > 1) = 1 - \Phi(1) \approx 1 - 0.8413 = 0.1587$.
\end{correctionbox}

\begin{correctionbox}[Correction Exercice 17 : Application $S_n$ (Rendement)]
$\mu=0.001, \sigma=0.01, n=100$. $S_{100} \approx \mathcal{N}(n\mu, n\sigma^2)$.
$E[S_{100}] = 100 \times 0.001 = 0.1$.
$\sigma_{S_n} = \sigma\sqrt{n} = 0.01 \times \sqrt{100} = 0.01 \times 10 = 0.1$.
On cherche $P(S_{100} > 0.2)$.
$Z = \frac{0.2 - 0.1}{0.1} = \frac{0.1}{0.1} = 1$.
$P(Z > 1) = 1 - \Phi(1) \approx 1 - 0.8413 = 0.1587$.
\end{correctionbox}

\begin{correctionbox}[Correction Exercice 18 : Application $S_n$ (Rendement)]
On utilise les mêmes paramètres : $S_{100} \approx \mathcal{N}(0.1, 0.1^2)$.
On cherche $P(S_{100} < 0)$.
$Z = \frac{0 - 0.1}{0.1} = -1$.
$P(Z < -1) = \Phi(-1) = 1 - \Phi(1) \approx 1 - 0.8413 = 0.1587$.
\end{correctionbox}

\begin{correctionbox}[Correction Exercice 19 : Inverse (Trouver $c$ pour $\bar{X}_n$)]
$\mu=50, \sigma=10, n=100$. $\bar{X}_{100} \approx \mathcal{N}(\mu, \sigma^2/n)$.
$\mu_{\bar{X}} = 50$. $\sigma_{\bar{X}} = \frac{10}{\sqrt{100}} = 1$.
On cherche $c$ tel que $P(\bar{X}_{100} \le c) \approx 0.8413$.
$P(Z \le \frac{c - 50}{1}) = 0.8413$.
D'après la table, le $z$ correspondant est $1.0$.
$\frac{c - 50}{1} = 1 \implies c = 51$.
\end{correctionbox}

\begin{correctionbox}[Correction Exercice 20 : Inverse (Trouver $c$ pour $S_n$)]
$\mu=10, \sigma=3, n=36$. $S_{36} \approx \mathcal{N}(n\mu, n\sigma^2)$.
$E[S_{36}] = 36 \times 10 = 360$.
$\sigma_{S_n} = \sigma\sqrt{n} = 3 \times \sqrt{36} = 18$.
On cherche $c$ tel que $P(S_{36} \le c) \approx 0.0013$.
$P(Z \le \frac{c - 360}{18}) = 0.0013$.
On sait $\Phi(3) \approx 0.9987$, donc $\Phi(-3) = 1 - 0.9987 = 0.0013$.
Le $z$ correspondant est $-3.0$.
$\frac{c - 360}{18} = -3 \implies c = 360 - 3(18) = 360 - 54 = 306$.
\end{correctionbox}

\begin{correctionbox}[Correction Exercice 21 : Inverse (Taille d'échantillon $n$)]
$\mu=0, \sigma=10$. On veut $P(|\bar{X}_n| \le 1) \ge 0.95$.
$P(-1 \le \bar{X}_n \le 1) \ge 0.95$.
Standardisation : $\sigma_{\bar{X}} = \frac{10}{\sqrt{n}}$.
$P\left( \frac{-1 - 0}{10/\sqrt{n}} \le Z \le \frac{1 - 0}{10/\sqrt{n}} \right) \ge 0.95$.
$P\left( \frac{-\sqrt{n}}{10} \le Z \le \frac{\sqrt{n}}{10} \right) \ge 0.95$.
On sait $P(-1.96 \le Z \le 1.96) = 0.95$.
On doit donc avoir $\frac{\sqrt{n}}{10} \ge 1.96$.
$\sqrt{n} \ge 19.6 \implies n \ge (19.6)^2 = 384.16$.
Il faut $n = 385$ au minimum.
\end{correctionbox}

\begin{correctionbox}[Correction Exercice 22 : Inverse (Taille d'échantillon $n$)]
$\mu=100, \sigma=20$. On veut $P(\bar{X}_n \ge 102) \le 0.0228$.
$P(Z \ge z) \le 0.0228$. On sait $P(Z \ge 2) = 1 - \Phi(2) \approx 0.0228$.
Donc on a besoin que notre Z-score soit $\ge 2$.
$Z = \frac{102 - 100}{20 / \sqrt{n}} = \frac{2}{20 / \sqrt{n}} = \frac{2\sqrt{n}}{20} = \frac{\sqrt{n}}{10}$.
On pose $Z \ge 2 \implies \frac{\sqrt{n}}{10} \ge 2$.
$\sqrt{n} \ge 20 \implies n \ge 400$.
Il faut $n = 400$ au minimum.
\end{correctionbox}

\begin{correctionbox}[Correction Exercice 23 : Paramètres (Proportion)]
$X_i \sim \text{Bern}(p)$ avec $p=0.25$.
1.  $\mu = E[X_i] = p = 0.25$.
    $\sigma^2 = \text{Var}(X_i) = p(1-p) = 0.25 \times 0.75 = 0.1875$.
2.  $\hat{p} = \bar{X}_n$. $n=400$.
    $E[\hat{p}] = \mu = 0.25$.
    $\text{Var}(\hat{p}) = \frac{\sigma^2}{n} = \frac{0.1875}{400} \approx 0.00046875$.
\end{correctionbox}

\begin{correctionbox}[Correction Exercice 24 : Calcul (Proportion)]
$p=0.5, n=100$. $\hat{p} \approx \mathcal{N}(p, \frac{p(1-p)}{n})$.
$E[\hat{p}] = 0.5$.
$\text{Var}(\hat{p}) = \frac{0.5 \times 0.5}{100} = \frac{0.25}{100} = 0.0025$.
$\sigma_{\hat{p}} = \sqrt{0.0025} = 0.05$.
On cherche $P(\hat{p} > 0.6)$.
$Z = \frac{0.6 - 0.5}{0.05} = \frac{0.1}{0.05} = 2$.
$P(Z > 2) = 1 - \Phi(2) \approx 1 - 0.9772 = 0.0228$.
\end{correctionbox}

\begin{correctionbox}[Correction Exercice 25 : Marge d'Erreur (Proportion)]
$n=1000, \hat{p}=0.54$.
On estime l'erreur standard $SE = \sqrt{\frac{\hat{p}(1-\hat{p})}{n}}$.
$SE = \sqrt{\frac{0.54 \times (1 - 0.54)}{1000}} = \sqrt{\frac{0.2484}{1000}} \approx 0.01576$.
La marge d'erreur à 95\% est $ME = 1.96 \times SE$.
$ME = 1.96 \times 0.01576 \approx 0.0309$.
(Soit $\pm 3.09\%$).
\end{correctionbox}

\subsection{Exercices Python}

Ces exercices appliquent le Théorème Central Limite (TCL) aux données financières. Nous considérerons l'ensemble des rendements journaliers de Google (GOOG) sur une longue période comme notre \textbf{population} (dont nous connaissons le "vrai" $\mu$ et $\sigma^2$). Nous simulerons ensuite un \textbf{échantillonnage} (tirer $n$ jours au hasard) pour voir comment la \textbf{distribution d'échantillonnage de la moyenne} ($\bar{X}_n$) se comporte.

\begin{codecell}
!pip install yfinance
import yfinance as yf
import pandas as pd
import numpy as np
import matplotlib.pyplot as plt
from scipy.stats import norm

# Definir le ticker et une longue periode pour notre "population"
ticker = "GOOG"
start_date = "2010-01-01"
end_date = "2024-12-31"

# Telecharger les prix de cloture ajustes
data = yf.download(ticker, start=start_date, end=end_date)["Adj Close"]

# Calculer les rendements journaliers (notre population X)
returns = data.pct_change().dropna()

# Calculer les "vrais" parametres de la population
population_mean = returns.mean()
population_var = returns.var()
population_std = returns.std()

print(f"--- Parametres de la Population (GOOG 2010-2024) ---")
print(f"Mu (moyenne) = {population_mean:.6f}")
print(f"Sigma^2 (variance) = {population_var:.6f}")
print(f"Sigma (ecart-type) = {population_std:.6f}")
print(f"N total (population) = {len(returns)}")
\end{codecell}

\begin{exercicebox}[Exercice 1 : Distribution de la Population vs. TCL]
La population des rendements journaliers $X_i$ n'est pas parfaitement normale. La distribution de $\bar{X}_n$, en revanche, devrait l'être.

\textbf{Votre tâche :}
\begin{enumerate}
    \item \textbf{(Plot)} Tracer l'histogramme de la \textbf{population} (la série \texttt{returns} complète).
    \item (Conclusion) La distribution de la population ressemble-t-elle à une loi normale parfaite ? (Regardez les queues).
\end{enumerate}
\end{exercicebox}

\begin{exercicebox}[Exercice 2 : Paramètres de la Distribution d'Échantillonnage (Théorie)]
Le TCL prédit les paramètres de la distribution de $\bar{X}_n$.
Supposons que nous prenions des échantillons de taille $n=30$.

\textbf{Votre tâche :}
\begin{enumerate}
    \item Calculer l'espérance \textbf{théorique} de la moyenne d'échantillon, $E[\bar{X}_{30}]$.
    \item Calculer la variance \textbf{théorique} de la moyenne d'échantillon, $\text{Var}(\bar{X}_{30}) = \sigma^2 / n$.
    \item Calculer l'erreur standard \textbf{théorique} (l'écart-type) de la moyenne d'échantillon, $\sigma_{\bar{X}_{30}} = \sigma / \sqrt{n}$.
    \item (Utilisez les $\mu$ et $\sigma^2$ de la population calculés dans la cellule de setup).
\end{enumerate}
\end{exercicebox}

\begin{exercicebox}[Exercice 3 : Simulation de la Distribution d'Échantillonnage]
Vérifions le TCL par simulation. Nous allons générer $k=1000$ échantillons de taille $n=30$ et calculer la moyenne de chacun.

\textbf{Votre tâche :}
\begin{enumerate}
    \item Créer une liste (ou un array) vide \texttt{sample\_means}.
    \item Boucler $k=1000$ fois :
        \begin{itemize}
            \item Tirer un échantillon de $n=30$ rendements de la \texttt{returns} (avec \texttt{np.random.choice(..., size=30, replace=True)}).
            \item Calculer la moyenne de cet échantillon.
            \item Ajouter cette moyenne à votre liste \texttt{sample\_means}.
        \end{itemize}
    \item Vous avez maintenant 1000 valeurs de $\bar{X}_{30}$.
\end{enumerate}
\end{exercicebox}

\begin{exercicebox}[Exercice 4 : Vérification des Paramètres (Empirique vs. Théorie)]
Utilisons les 1000 moyennes d'échantillon (\texttt{sample\_means}) de l'exercice 3.

\textbf{Votre tâche :}
\begin{enumerate}
    \item Calculer la moyenne \textbf{empirique} des 1000 moyennes d'échantillon.
    \item Comparer ce résultat à l'espérance \textbf{théorique} $E[\bar{X}_{30}]$ de l'exercice 2.
    \item Calculer la variance \textbf{empirique} des 1000 moyennes d'échantillon.
    \item Comparer ce résultat à la variance \textbf{théorique} $\text{Var}(\bar{X}_{30})$ de l'exercice 2.
\end{enumerate}
\end{exercicebox}

\begin{exercicebox}[Exercice 5 : Visualisation du TCL (Plot)]
C'est la visualisation la plus importante. Nous allons superposer la distribution empirique des moyennes (de l'Ex 3) et la distribution normale théorique (de l'Ex 2).

\textbf{Votre tâche :}
\begin{enumerate}
    \item \textbf{(Plot)} Tracer l'histogramme des 1000 \texttt{sample\_means} (de l'Ex 3). Assurez-vous d'utiliser \texttt{density=True}.
    \item \textbf{(Plot)} Sur le \textbf{même} graphique, tracer la PDF de la loi normale \textbf{théorique} prédite par le TCL.
    \item (Indice : $X \sim \mathcal{N}(\mu, \sigma^2/n)$. Utilisez \texttt{scipy.stats.norm.pdf()} avec \texttt{loc = population\_mean} et \texttt{scale = (population\_std / np.sqrt(30))}).
    \item (Conclusion) La prédiction du TCL correspond-elle à la simulation ?
\end{enumerate}
\end{exercicebox}

\begin{exercicebox}[Exercice 6 : L'Effet de la Taille $n$ (Plot)]
La convergence est plus rapide lorsque $n$ est grand.

\textbf{Votre tâche :}
\begin{enumerate}
    \item Répéter l'exercice 3, mais pour $n=5$ (créer 1000 \texttt{sample\_means\_5}).
    \item Répéter l'exercice 3, mais pour $n=100$ (créer 1000 \texttt{sample\_means\_100}).
    \item \textbf{(Plot)} Créer trois histogrammes côte à côte (\texttt{plt.subplot}) pour $n=5$, $n=30$ (de l'Ex 3), et $n=100$.
    \item (Conclusion) Que constatez-vous à propos de la variance (largeur de la cloche) de la distribution de $\bar{X}_n$ lorsque $n$ augmente ?
\end{enumerate}
\end{exercicebox}

\begin{exercicebox}[Exercice 7 : Application du TCL (Calcul de Probabilité pour $\bar{X}_n$)]
Utilisons le TCL pour répondre à une question pratique.
Quelle est la probabilité que le rendement \textbf{moyen} sur un mois de trading ($n=21$ jours) soit positif ?

\textbf{Votre tâche :}
\begin{enumerate}
    \item On cherche $P(\bar{X}_{21} > 0)$.
    \item Identifier $\mu$ et $\sigma$ (de la population).
    \item Calculer l'erreur standard $\sigma_{\bar{X}_{21}} = \sigma / \sqrt{21}$.
    \item Standardiser la valeur $0$ : $Z = (0 - \mu) / \sigma_{\bar{X}_{21}}$.
    \item Calculer la probabilité $P(Z > z) = 1 - \Phi(z)$ en utilisant \texttt{scipy.stats.norm.cdf()}.
\end{enumerate}
\end{exercicebox}

\begin{exercicebox}[Exercice 8 : Application du TCL (Calcul de Probabilité pour $S_n$)]
Quelle est la probabilité que le rendement \textbf{total} (la somme) sur une année ($n=252$ jours) soit supérieur à 10\% ?

\textbf{Votre tâche :}
\begin{enumerate}
    \item On cherche $P(S_{252} > 0.10)$.
    \item Le TCL dit $S_n \approx \mathcal{N}(n\mu, n\sigma^2)$.
    \item Calculer l'espérance de la somme : $E[S_{252}] = n\mu = 252 \times \mu$.
    \item Calculer l'écart-type de la somme : $\sigma_{S_{252}} = \sigma \sqrt{n} = \sigma \times \sqrt{252}$.
    \item Standardiser la valeur $0.10$ : $Z = (0.10 - E[S_{252}]) / \sigma_{S_{252}}$.
    \item Calculer la probabilité $P(Z > z) = 1 - \Phi(z)$.
\end{enumerate}
\end{exercicebox}

\begin{exercicebox}[Exercice 9 : TCL pour les Proportions (Binomiale)]
Le TCL s'applique aussi aux proportions (qui sont des moyennes de Bernoulli). Soit $p$ la probabilité qu'un jour soit un "jour de hausse" (rendement > 0).

\textbf{Votre tâche :}
\begin{enumerate}
    \item Estimer la "vraie" proportion $p$ (notre $\mu$) en calculant la proportion de jours de hausse dans toute la population \texttt{returns}.
    \item On sonde $n=100$ jours. Quelle est la probabilité que notre sondage ($\hat{p} = \bar{X}_{100}$) montre une majorité de jours de baisse ($\hat{p} < 0.5$) ?
    \item L'erreur standard pour une proportion est $\sigma_{\hat{p}} = \sqrt{p(1-p) / n}$.
    \item Standardiser 0.5 : $Z = (0.5 - p) / \sigma_{\hat{p}}$.
    \item Calculer la probabilité $P(Z < z) = \Phi(z)$.
\end{enumerate}
\end{exercicebox}

\begin{exercicebox}[Exercice 10 : Marge d'Erreur (Intervalle de Confiance)]
C'est l'application la plus courante du TCL dans les médias.
Quelle est la "marge d'erreur" à 95\% pour notre estimation de $p$ (proportion de jours de hausse) si l'on utilise un échantillon de $n=1000$ jours ?

\textbf{Votre tâche :}
\begin{enumerate}
    \item Utiliser le $p$ (proportion de la population) estimé à l'exercice 9.
    \item Calculer l'erreur standard pour $n=1000$ : $SE = \sqrt{p(1-p) / 1000}$.
    \item La marge d'erreur à 95\% est $ME = 1.96 \times SE$. (Car $P(-1.96 \le Z \le 1.96) \approx 0.95$).
    \item (Conclusion) Interpréter le résultat : "Notre estimation de la proportion de jours de hausse sera correcte à $\pm$ [ME] près, 95\% du temps."
\end{enumerate}
\end{exercicebox}
\newpage

\section{Le Théorème Central Limite (TCL)}

\subsection{Introduction : L'omniprésence de la loi normale}

Dans la section précédente, la Loi des Grands Nombres (LLN) nous a donné une garantie fondamentale : la moyenne d'échantillon $\bar{X}_n$ converge vers la vraie moyenne $\mu$ lorsque $n$ devient grand.
$$ \bar{X}_n \xrightarrow{p.s.} \mu $$
La LLN nous dit \textbf{où} la moyenne d'échantillon converge (vers la constante $\mu$), mais elle ne nous dit rien sur la \textit{forme} de la distribution de $\bar{X}_n$ autour de $\mu$ pour un $n$ grand, mais fini.

Le \textbf{Théorème Central Limite (TCL)} comble cette lacune. Il décrit la \textit{manière} dont $\bar{X}_n$ converge, en nous donnant la forme de sa distribution. C'est sans doute le théorème le plus important des statistiques.

\begin{intuitionbox}[L'Idée Fondamentale]
Intuitivement, ce résultat affirme qu'une \textbf{somme} d'un grand nombre de variables aléatoires indépendantes et identiquement distribuées (i.i.d.) tend, le plus souvent, à suivre une \textbf{loi normale} (aussi appelée loi de Laplace-Gauss ou "courbe en cloche").

Ce théorème et ses généralisations offrent une explication à l'omniprésence de la loi normale dans la nature. De nombreux phénomènes (la taille d'un individu, l'erreur de mesure d'un instrument, le bruit de fond d'un signal) sont le résultat de l'addition d'un très grand nombre de petites perturbations aléatoires. Le TCL nous dit que le résultat de cette somme sera, inévitablement, distribué selon une loi normale.
\end{intuitionbox}

\subsection{L'illustration : la somme des "Pile ou Face"}

Prenons l'exemple le plus simple pour illustrer ce phénomène : le jeu de "pile ou face".



\begin{examplebox}[Distribution de la Somme de $n$ Lancers]
Soit $X_i$ le résultat du $i$-ème lancer, avec $X_i = 1$ pour "Face" (probabilité 0,5) et $X_i = 0$ pour "Pile" (probabilité 0,5). La distribution d'origine (pour $n=1$) n'est pas du tout une courbe en cloche : c'est une distribution discrète avec deux bâtons de même hauteur.

Considérons la \textbf{somme} $S_n = X_1 + X_2 + \dots + X_n$, qui représente le nombre total de "Face" obtenus en $n$ lancers.

\begin{itemize}
    \item \textbf{Pour $n=1$ :} La distribution de $S_1$ est :
    \begin{itemize}
        \item Valeurs de la somme : \{0, 1\}
        \item Fréquences : \{0.5, 0.5\}
    \end{itemize}
    
    \item \textbf{Pour $n=2$ :} Les sommes possibles sont \{0, 1, 2\}. La distribution de $S_2$ est :
    \begin{itemize}
        \item Valeurs de la somme : \{0, 1, 2\}
        \item Fréquences : \{0.25, 0.5, 0.25\} (elle forme un triangle).
    \end{itemize}
    
    \item \textbf{Pour $n=3$ :} Les sommes possibles sont \{0, 1, 2, 3\}. La distribution de $S_3$ est :
    \begin{itemize}
        \item Valeurs de la somme : \{0, 1, 2, 3\}
        \item Fréquences : \{0.125, 0.375, 0.375, 0.125\}
    \end{itemize}
\end{itemize}

\begin{center}
\begin{tikzpicture}
  \begin{axis}[
    width=0.8\textwidth,
    height=0.5\textwidth,
    xlabel={Valeurs de la somme},
    title={Fonction de fréquence pour des tirages à pile ou face},
    grid=both,
    grid style={line width=.1pt, draw=gray!30},
    major grid style={line width=.2pt,draw=gray!50},
    domain=0:4,
    samples=100,
    enlargelimits=false,
    axis lines=middle,
    xmin=0, xmax=4,
    ymin=0, ymax=0.6
  ]
  % n=1
  \addplot [thick, color=blue, fill=blue!20, fill opacity=0.5] coordinates {(0,0.5) (1,0.5)} \closedcycle;
  % n=2
  \addplot [thick, color=green, fill=green!20, fill opacity=0.5] coordinates {(0,0.25) (1,0.5) (2,0.25)} \closedcycle;
  % n=3
  \addplot [thick, color=red, fill=red!20, fill opacity=0.5] coordinates {(0,0.125) (1,0.375) (2,0.375) (3,0.125)} \closedcycle;
  % n=4 (suggéré)
  \addplot [thick, color=black, fill=black!20, fill opacity=0.5] coordinates {(0,0.0625) (1,0.25) (2,0.375) (3,0.25) (4,0.0625)} \closedcycle;
  
  \legend{$n=1$,$n=2$,$n=3$,$n=4$}
  \end{axis}
\end{tikzpicture}
\end{center}

Graphiquement, on constate que plus le nombre de tirages $n$ augmente (par exemple, jusqu'à $n=12$), plus la courbe de fréquence (qui reste discrète) se rapproche d'une courbe en cloche symétrique, caractéristique de la loi normale.
\end{examplebox}

\subsection{Distribution de la population vs. Distribution d'échantillonnage}

Le point le plus remarquable du TCL est qu'il fonctionne \textit{quelle que soit} la distribution de départ.

\begin{intuitionbox}[Population vs. Échantillonnage]
Imaginez deux univers de distributions :

\begin{itemize}
    \item \textbf{1. La Distribution de la Population ($X_i$) :} C'est la loi de nos variables $X_i$ individuelles. Elle peut avoir \textbf{n'importe quelle forme} (par exemple, une distribution bimodale, asymétrique, ou uniforme). Cette distribution a une "vraie" moyenne $\mu$ et un "vrai" écart-type $\sigma$.
    
    \item \textbf{2. La Distribution d'Échantillonnage ($\bar{X}_n$) :} C'est la distribution de la \textit{moyenne} $\bar{X}_n = (X_1 + \dots + X_n)/n$, calculée sur des échantillons de taille $n$. C'est la distribution de "toutes les moyennes d'échantillon possibles".
\end{itemize}

Le TCL énonce la relation magique entre les deux :

\textbf{Quelle que soit la forme de la distribution de la population, plus la taille de l'échantillon $n$ croît, plus la distribution d'échantillonnage de la moyenne $\bar{X}_n$ est proche d'une loi normale (gaussienne).}

De plus, les paramètres de cette loi normale sont :
\begin{itemize}
    \item \textbf{Moyenne :} La distribution de $\bar{X}_n$ est centrée sur la même moyenne $\mu$ que la population.
    \item \textbf{Écart-type :} La distribution de $\bar{X}_n$ est beaucoup plus resserrée. Son écart-type (appelé "erreur standard") est $\sigma_{\bar{X}} = \frac{\sigma}{\sqrt{n}}$.
\end{itemize}
Cette dispersion $\sigma/\sqrt{n}$ qui tend vers 0 est la manifestation de la Loi des Grands Nombres. Le TCL précise que la \textit{forme} de cette convergence est gaussienne.
\end{intuitionbox}

\subsection{Énoncé formel du Théorème Central Limite}

Pour énoncer le théorème formellement, nous devons d'abord définir les propriétés de la somme $S_n$ et de la moyenne $\bar{X}_n$.

Soit $X_1, \dots, X_n$ des variables aléatoires i.i.d. avec $E[X_i] = \mu$ et $\text{Var}(X_i) = \sigma^2$.

\begin{itemize}
    \item \textbf{La Somme $S_n = \sum X_i$} :
    \begin{itemize}
        \item Espérance : $E[S_n] = E[\sum X_i] = \sum E[X_i] = n\mu$
        \item Variance : $\text{Var}(S_n) = \text{Var}(\sum X_i) = \sum \text{Var}(X_i) = n\sigma^2$
        \item Écart-type : $\sigma_{S_n} = \sqrt{n\sigma^2} = \sigma\sqrt{n}$
    \end{itemize}
    
    \item \textbf{La Moyenne $\bar{X}_n = S_n / n$} :
    \begin{itemize}
        \item Espérance : $E[\bar{X}_n] = E[S_n / n] = \frac{1}{n} E[S_n] = \frac{1}{n} (n\mu) = \mu$
        \item Variance : $\text{Var}(\bar{X}_n) = \text{Var}(S_n / n) = \frac{1}{n^2} \text{Var}(S_n) = \frac{1}{n^2} (n\sigma^2) = \frac{\sigma^2}{n}$
        \item Écart-type : $\sigma_{\bar{X}_n} = \sqrt{\sigma^2 / n} = \frac{\sigma}{\sqrt{n}}$
    \end{itemize}
\end{itemize}

Nous voyons que la distribution de $S_n$ s'étale (variance $\to \infty$) tandis que celle de $\bar{X}_n$ se contracte (variance $\to 0$). Pour étudier la \textit{forme} de la convergence, nous créons une variable "stable" en la centrant (soustrayant la moyenne) et en la réduisant (divisant par l'écart-type). C'est la variable $Z_n$.

\begin{theorembox}[Théorème Central Limite (Lindeberg-Lévy)]
Soit $X_1, X_2, \dots, X_n$ une suite de variables aléatoires \textbf{i.i.d.} (indépendantes et identiquement distribuées) suivant la même loi $D$.
Supposons que l'\textbf{espérance $\mu$} et l'\textbf{écart-type $\sigma$} de cette loi $D$ existent, sont finis, et $\sigma \neq 0$.

Considérons la variable aléatoire standardisée $Z_n$ :
$$ Z_n = \frac{S_n - E[S_n]}{\sigma_{S_n}} = \frac{S_n - n\mu}{\sigma\sqrt{n}} $$
Cette variable est équivalente à la moyenne standardisée :
$$ Z_n = \frac{\bar{X}_n - E[\bar{X}_n]}{\sigma_{\bar{X}_n}} = \frac{\bar{X}_n - \mu}{\sigma / \sqrt{n}} $$
(Pour tout $n$, $Z_n$ est une variable centrée-réduite : $E[Z_n] = 0$ et $\text{Var}(Z_n) = 1$).

Alors, la suite de variables aléatoires $Z_1, Z_2, \dots, Z_n, \dots$ \textbf{converge en loi} vers une variable aléatoire $Z$ qui suit la \textbf{loi normale centrée réduite $N(0, 1)$}, lorsque $n$ tend vers l'infini.

Cela signifie que si $\Phi$ est la fonction de répartition de la loi $N(0, 1)$, alors pour tout réel $z$ :
$$ \lim_{n \to \infty} P(Z_n \le z) = \lim_{n \to \infty} P\left( \frac{\bar{X}_n - \mu}{\sigma/\sqrt{n}} \le z \right) = \Phi(z) $$
\end{theorembox}

\subsection{Applications Pratiques du TCL}

Le TCL n'est pas seulement une curiosité mathématique ; c'est le fondement de l'inférence statistique. Voici comment l'appliquer concrètement pour résoudre des problèmes.

\begin{examplebox}[La taille des individus]
\textbf{Contexte :} La taille des individus dans une population suit une courbe en cloche. Pourquoi ? Car elle est la \textbf{somme} de milliers de petites influences (gènes, nutrition, etc.). Le TCL s'applique.

\textbf{Données :} Supposons que dans une population, la taille $X$ des individus ait une espérance $\mu = 175$ cm et un écart-type $\sigma = 8$ cm. (Note : la loi de $X$ n'est pas forcément normale, même si en pratique elle l'est).

\textbf{Problème :} On prélève un échantillon aléatoire de $n=64$ individus. Quelle est la probabilité que la \textbf{moyenne de cet échantillon} ($\bar{X}_{64}$) soit supérieure à 177 cm ?

\textbf{Solution :}
\begin{enumerate}
    \item \textbf{Identifier les paramètres :}
    \begin{itemize}
        \item Moyenne de la population : $\mu = 175$ cm
        \item Écart-type de la population : $\sigma = 8$ cm
        \item Taille de l'échantillon : $n = 64$
    \end{itemize}
    
    \item \textbf{Appliquer le TCL :}
    Puisque $n=64$ est grand (généralement $n \ge 30$ est suffisant), le TCL s'applique. La distribution d'échantillonnage de la moyenne $\bar{X}_n$ suit approximativement une loi normale.
    $$ \bar{X}_n \approx N\left(\mu, \frac{\sigma^2}{n}\right) $$
    
    \item \textbf{Calculer les paramètres de la loi normale de $\bar{X}_n$ :}
    \begin{itemize}
        \item Espérance de $\bar{X}_n$ : $E[\bar{X}_n] = \mu = 175$ cm.
        \item Écart-type de $\bar{X}_n$ (appelé "Erreur Standard") :
        $$ \sigma_{\bar{X}_n} = \frac{\sigma}{\sqrt{n}} = \frac{8}{\sqrt{64}} = \frac{8}{8} = 1 \text{ cm} $$
    \end{itemize}
    Donc, $\bar{X}_{64} \approx N(175, 1^2)$.
    
    \item \textbf{Standardiser (Calculer le Z-score) :}
    Nous cherchons $P(\bar{X}_{64} > 177)$. Nous transformons cette valeur en un score $Z$ pour utiliser la loi normale centrée réduite $N(0, 1)$.
    $$ Z = \frac{\bar{X}_n - \mu}{\sigma_{\bar{X}_n}} = \frac{177 - 175}{1} = 2 $$
    
    \item \textbf{Trouver la probabilité :}
    Chercher $P(\bar{X}_{64} > 177)$ revient à chercher $P(Z > 2)$.
    En utilisant la table de la loi normale (ou une calculatrice) :
    $$ P(Z > 2) = 1 - P(Z \le 2) = 1 - \Phi(2) $$
    Sachant que $\Phi(2) \approx 0.9772$,
    $$ P(Z > 2) = 1 - 0.9772 = 0.0228 $$
\end{enumerate}
\textbf{Conclusion :} Il y a environ 2.28\% de chances qu'un échantillon de 64 personnes ait une taille moyenne supérieure à 177 cm.
\end{examplebox}

\begin{examplebox}[Remplissage de bouteilles]
\textbf{Contexte :} Une machine remplit des bouteilles de soda. Le volume versé $X_i$ fluctue légèrement. La loi de $X_i$ est inconnue.

\textbf{Données :} La machine est réglée pour verser en moyenne $\mu = 500$ ml. L'écart-type du processus est connu et vaut $\sigma = 6$ ml. Pour un contrôle, on prélève un échantillon de $n=36$ bouteilles.

\textbf{Problème :} On considère que la machine est déréglée si la moyenne de l'échantillon $\bar{X}_{36}$ est inférieure à 498 ml. Quelle est la probabilité d'une "fausse alarme" (c'est-à-dire, la machine fonctionne bien à $\mu=500$, mais l'échantillon a une moyenne $\bar{X}_{36} < 498$) ?

\textbf{Solution :}
\begin{enumerate}
    \item \textbf{Identifier les paramètres :}
    $\mu = 500$ ml, $\sigma = 6$ ml, $n = 36$.
    
    \item \textbf{Appliquer le TCL :}
    $n=36 \ge 30$, donc le TCL s'applique.
    $$ \bar{X}_{36} \approx N\left(\mu, \frac{\sigma^2}{n}\right) $$
    
    \item \textbf{Calculer les paramètres de $\bar{X}_{36}$ :}
    \begin{itemize}
        \item Espérance : $E[\bar{X}_{36}] = \mu = 500$ ml.
        \item Erreur Standard : $\sigma_{\bar{X}} = \frac{\sigma}{\sqrt{n}} = \frac{6}{\sqrt{36}} = \frac{6}{6} = 1$ ml.
    \end{itemize}
    Donc, $\bar{X}_{36} \approx N(500, 1^2)$.
    
    \item \textbf{Standardiser (Calculer le Z-score) :}
    Nous cherchons la probabilité $P(\bar{X}_{36} < 498)$.
    $$ Z = \frac{\bar{X}_n - \mu}{\sigma_{\bar{X}_n}} = \frac{498 - 500}{1} = -2 $$
    
    \item \textbf{Trouver la probabilité :}
    Chercher $P(\bar{X}_{36} < 498)$ revient à chercher $P(Z < -2)$.
    $$ P(Z < -2) = \Phi(-2) $$
    Par symétrie de la loi normale, $\Phi(-z) = 1 - \Phi(z)$.
    $$ P(Z < -2) = 1 - \Phi(2) = 1 - 0.9772 = 0.0228 $$
\end{enumerate}
\textbf{Conclusion :} Il y a 2.28\% de chances d'avoir une fausse alarme, c'est-à-dire de croire à tort que la machine est déréglée alors qu'elle fonctionne normalement.
\end{examplebox}

\begin{examplebox}[Rendement d'un portefeuille (sur la Somme)]
\textbf{Contexte :} Le rendement quotidien $X_i$ d'un actif est très volatile. On s'intéresse au rendement annuel \textbf{total}, qui est la \textbf{somme} des rendements quotidiens.

\textbf{Données :} Supposons que le rendement quotidien $X_i$ ait une espérance $\mu = 0.04\%$ et un écart-type $\sigma = 1\%$. (La loi de $X_i$ est inconnue, mais $\mu$ et $\sigma$ existent). Il y a $n=252$ jours de trading dans l'année.

\textbf{Problème :} Quelle est la probabilité que le rendement annuel total $S_{252} = X_1 + \dots + X_{252}$ soit négatif (inférieur à 0) ?

\textbf{Solution :}
\begin{enumerate}
    \item \textbf{Identifier les paramètres (pour une seule v.a. $X_i$) :}
    $\mu = 0.0004$, $\sigma = 0.01$, $n = 252$.
    
    \item \textbf{Appliquer le TCL (pour la somme $S_n$) :}
    $n=252$ est grand. Le TCL s'applique à la somme $S_n$.
    $$ S_n \approx N\left(n\mu, n\sigma^2\right) $$
    
    \item \textbf{Calculer les paramètres de la loi normale de $S_{252}$ :}
    \begin{itemize}
        \item Espérance de $S_{252}$ : $E[S_n] = n\mu = 252 \times 0.0004 = 0.1008$ (soit 10.08\%).
        \item Variance de $S_{252}$ : $\text{Var}(S_n) = n\sigma^2 = 252 \times (0.01)^2 = 252 \times 0.0001 = 0.0252$.
        \item Écart-type de $S_{252}$ : $\sigma_{S_n} = \sqrt{n\sigma^2} = \sqrt{0.0252} \approx 0.1587$ (soit 15.87\%).
    \end{itemize}
    Donc, $S_{252} \approx N(0.1008, 0.1587^2)$.
    
    \item \textbf{Standardiser (Calculer le Z-score) :}
    Nous cherchons $P(S_{252} < 0)$.
    $$ Z = \frac{S_n - E[S_n]}{\sigma_{S_n}} = \frac{0 - 0.1008}{0.1587} \approx -0.635 $$
    
    \item \textbf{Trouver la probabilité :}
    Chercher $P(S_{252} < 0)$ revient à chercher $P(Z < -0.635)$.
    $$ P(Z < -0.635) = \Phi(-0.635) = 1 - \Phi(0.635) $$
    En interpolant dans la table, $\Phi(0.635) \approx 0.7373$.
    $$ P(Z < -0.635) \approx 1 - 0.7373 = 0.2627 $$
\end{enumerate}
\textbf{Conclusion :} Malgré une espérance de rendement quotidien positive, il y a environ 26.3\% de chances que le rendement annuel total soit négatif.
\end{examplebox}

\begin{examplebox}[Estimation d'une proportion (Marge d'erreur)]
\textbf{Contexte :} On veut estimer la proportion $p$ de votants qui approuvent un candidat. On modélise chaque personne $i$ par une variable de Bernoulli $X_i$ (1 si "oui", 0 si "non").
L'espérance de la population est $\mu = E[X_i] = p$.
La variance de la population est $\sigma^2 = \text{Var}(X_i) = p(1-p)$.
Le résultat du sondage est la moyenne d'échantillon $\bar{X}_n = \hat{p}$ (la proportion observée).

\textbf{Données :} On sonde $n=1000$ personnes. Le résultat est que 540 personnes disent "oui". Donc $\hat{p} = 540/1000 = 0.54$.

\textbf{Problème :} Calculer l'intervalle de confiance à 95\% pour la vraie proportion $p$ (la fameuse "marge d'erreur").

\textbf{Solution :}
\begin{enumerate}
    \item \textbf{Appliquer le TCL :}
    $n=1000$ est grand. Le TCL nous dit que la proportion d'échantillon $\hat{p} = \bar{X}_n$ suit une loi normale :
    $$ \hat{p} \approx N\left(p, \frac{p(1-p)}{n}\right) $$
    
    \item \textbf{Formule de l'Intervalle de Confiance :}
    Un intervalle de confiance à 95\% est centré sur notre estimation $\hat{p}$ et s'étend de $\pm 1.96$ erreurs standard (car $P(-1.96 \le Z \le 1.96) = 0.95$).
    $$ I.C._{95\%} = \left[ \hat{p} - 1.96 \cdot \sigma_{\hat{p}} \ ; \ \hat{p} + 1.96 \cdot \sigma_{\hat{p}} \right] $$
    où $\sigma_{\hat{p}} = \sqrt{p(1-p)/n}$.
    
    \item \textbf{Estimer l'Erreur Standard :}
    Problème : nous ne connaissons pas $p$ (c'est ce que nous cherchons !). Nous ne pouvons donc pas calculer $\sigma_{\hat{p}}$.
    \textbf{Solution :} Nous l'estimons en utilisant notre meilleur estimateur pour $p$, qui est $\hat{p} = 0.54$.
    $$ \text{Erreur Standard Estimée (SE)} = \sqrt{\frac{\hat{p}(1-\hat{p})}{n}} $$
    $$ SE = \sqrt{\frac{0.54 \times (1 - 0.54)}{1000}} = \sqrt{\frac{0.54 \times 0.46}{1000}} = \sqrt{\frac{0.2484}{1000}} \approx \sqrt{0.0002484} \approx 0.01576 $$
    
    \item \textbf{Calculer la Marge d'Erreur :}
    La marge d'erreur (ME) est la demi-largeur de l'intervalle.
    $$ ME = 1.96 \times SE = 1.96 \times 0.01576 \approx 0.0309 $$
    
    \item \textbf{Construire l'Intervalle :}
    $$ I.C._{95\%} = [ 0.54 - 0.0309 \ ; \ 0.54 + 0.0309 ] = [ 0.5091 \ ; \ 0.5709 ] $$
\end{enumerate}
\textbf{Conclusion :} Avec 54\% d'intentions de vote sur un échantillon de 1000 personnes, nous sommes confiants à 95\% que la vraie proportion $p$ dans la population se situe entre 50.9\% et 57.1\%. La marge d'erreur du sondage est de $\pm 3.1\%$.
\end{examplebox}

% \newpage

\section{Appendice A: Séries de Taylor et Maclaurin}

\begin{definitionbox}[Séries de Taylor et Maclaurin]
Si une fonction $f$ est indéfiniment dérivable au voisinage d'un point $a$, sa \textbf{série de Taylor} centrée en $a$ est définie par :
$$ f(x) = \sum_{k=0}^{\infty} \frac{f^{(k)}(a)}{k!} (x-a)^k $$
où $f^{(k)}(a)$ est la $k$-ième dérivée de $f$ évaluée en $a$.
\newline
\newline
Dans le cas particulier où $\mathbf{a=0}$, la série est appelée une \textbf{série de Maclaurin}. C'est la forme la plus courante, car elle approxime les fonctions autour de l'origine.
\end{definitionbox}

\subsection{Construction pas à pas d'une série de Taylor}

\begin{intuitionbox}[La logique de la correspondance des dérivées]
L'objectif fondamental d'une série de Taylor est de construire un polynôme, $P(x)$, qui soit une "copie conforme" d'une fonction $f(x)$ autour d'un point $a$. Pour ce faire, on force le polynôme à avoir exactement les mêmes propriétés locales que la fonction : même valeur, même pente, même courbure, etc. Cela se traduit mathématiquement par une exigence : \textbf{la n-ième dérivée du polynôme en $a$ doit être égale à la n-ième dérivée de la fonction en $a$}, et ce pour tous les ordres $n$.

Prenons l'exemple de $f(x) = e^x$ et construisons sa série de Maclaurin (centrée en $a=0$), où $f^{(k)}(0)=1$ pour tout $k$.

\begin{enumerate}
    \item \textbf{Ordre 0 : Faire correspondre la valeur}
    \newline
    \textbf{Objectif :} Le polynôme $P_0(x)$ doit avoir la même valeur que $f(x)$ en $x=0$. On veut $P_0(0) = f(0)$.
    \newline
    \textbf{Solution :} On choisit le polynôme le plus simple, une constante : $P_0(x) = f(0)$. Pour $e^x$, $f(0)=1$, donc $\mathbf{P_0(x) = 1}$.
    \newline
    \textbf{Vérification :} $P_0(0) = 1$. L'objectif est atteint.

    \item \textbf{Ordre 1 : Faire correspondre la première dérivée}
    \newline
    \textbf{Objectif :} On veut un nouveau polynôme $P_1(x)$ qui préserve la correspondance précédente ($P_1(0) = f(0)$) ET qui a la même pente, c'est-à-dire $P_1'(0) = f'(0)$.
    \newline
    \textbf{Solution :} On ajoute un terme en $x$ à notre polynôme précédent : $P_1(x) = P_0(x) + c_1 x = 1 + c_1 x$.
    \newline
    \textbf{Vérification :}
    \begin{itemize}
        \item $P_1(0) = 1 + c_1(0) = 1$. La valeur correspond toujours, car le nouveau terme s'annule en 0.
        \item On dérive : $P_1'(x) = c_1$. Pour que les pentes correspondent en 0, il faut $P_1'(0) = c_1 = f'(0)$. Comme $f'(0)=1$, on doit choisir $\mathbf{c_1=1}$.
    \end{itemize}
    Notre polynôme est maintenant $\mathbf{P_1(x) = 1+x}$.

    \item \textbf{Ordre 2 : Faire correspondre la deuxième dérivée}
    \newline
    \textbf{Objectif :} On veut $P_2(x)$ tel que $P_2(0)=f(0)$, $P_2'(0)=f'(0)$ ET $P_2''(0)=f''(0)$.
    \newline
    \textbf{Solution :} On ajoute un terme en $x^2$ : $P_2(x) = P_1(x) + c_2 x^2 = 1 + x + c_2 x^2$.
    \newline
    \textbf{Vérification :}
    \begin{itemize}
        \item Les dérivées d'ordre 0 et 1 en $x=0$ ne sont pas affectées, car la dérivée de $c_2x^2$ (soit $2c_2x$) et le terme lui-même s'annulent en 0. Les objectifs précédents sont préservés.
        \item On dérive deux fois : $P_2'(x) = 1 + 2c_2x$ et $P_2''(x) = 2c_2$.
        \item Pour que les courbures correspondent, il faut $P_2''(0) = 2c_2 = f''(0)$. Comme $f''(0)=1$, on doit choisir $\mathbf{c_2 = 1/2}$.
    \end{itemize}
    Notre polynôme est $\mathbf{P_2(x) = 1+x+\frac{1}{2}x^2}$.

    \item \textbf{Le schéma général : L'importance de la factorielle}
    \newline
    Pour faire correspondre la $k$-ième dérivée, on ajoute un terme $c_k x^k$.
    \newline
    Quand on dérive $c_k x^k$ exactement $k$ fois, on obtient $c_k \times k!$.
    \newline
    Toutes les dérivées d'ordre inférieur s'annulent en $x=0$. On doit donc avoir :
    $$ P_k^{(k)}(0) = c_k \cdot k! = f^{(k)}(0) $$
    Cela nous donne la règle pour trouver chaque coefficient :
    $$ c_k = \frac{f^{(k)}(0)}{k!} $$
    C'est précisément le coefficient qui apparaît dans la formule de Taylor, et il est choisi pour cette unique raison : forcer la $k$-ième dérivée du polynôme à correspondre parfaitement à celle de la fonction au point de développement.
\end{enumerate}

\tcblower
\centering
\begin{tikzpicture}
    \begin{axis}[
        xlabel={$x$},
        ylabel={$y$},
        xmin=-2, xmax=2,
        ymin=-0.5, ymax=4,
        axis lines=middle,
        legend style={at={(0.05,0.95)}, anchor=north west, font=\small},
        grid=major,
        samples=150,
        domain=-2:2,
        height=9cm,
        width=\linewidth-1cm,
        tick label style={font=\tiny}
    ]
    
    \addplot[black, dashed, ultra thick] {exp(x)};
    \addlegendentry{$e^x$}

    \addplot[red, thick] {1};
    \addlegendentry{$P_0(x)=1$}

    \addplot[blue, thick] {1+x};
    \addlegendentry{$P_1(x)=1+x$}

    \addplot[green!70!black, thick] {1+x+x^2/2};
    \addlegendentry{$P_2(x)=1+x+\frac{x^2}{2!}$}

    \addplot[orange, thick] {1+x+x^2/2+x^3/6};
    \addlegendentry{$P_3(x)=1+x+\frac{x^2}{2!}+\frac{x^3}{3!}$}

    \end{axis}
\end{tikzpicture}
\par\small\textit{Visualisation de la construction progressive de la série de Maclaurin pour $e^x$.}
\end{intuitionbox}

\subsection{Intuition de la série de Taylor en un point quelconque $a$}

\begin{intuitionbox}[Construire une approximation loin de l'origine]
La série de Maclaurin est puissante, mais elle nous contraint à approximer une fonction uniquement autour de $x=0$. Que faire si l'on s'intéresse au comportement d'une fonction ailleurs, par exemple $f(x)=\ln(x)$ autour de $x=1$ (puisque $\ln(0)$ n'est pas défini) ? C'est là qu'intervient la série de Taylor générale.

L'objectif reste le même : construire un polynôme $P(x)$ qui est une "copie conforme" de $f(x)$ au point $a$. Pour cela, on force les dérivées du polynôme à correspondre à celles de la fonction en ce point $a$. La seule différence est que notre "variable" de base n'est plus $x$, mais l'écart par rapport au centre, c'est-à-dire $(x-a)$.

Prenons l'exemple de $f(x) = \ln(x)$ et construisons sa série centrée en $\mathbf{a=1}$.

\begin{enumerate}
    \item \textbf{Ordre 0 : Faire correspondre la valeur}
    \newline
    \textbf{Objectif :} $P_0(a) = f(a)$.
    \newline
    \textbf{Solution :} On calcule $f(1) = \ln(1) = 0$. Le polynôme est la constante $\mathbf{P_0(x) = 0}$.

    \item \textbf{Ordre 1 : Faire correspondre la pente}
    \newline
    \textbf{Objectif :} $P_1(a) = f(a)$ et $P_1'(a) = f'(a)$.
    \newline
    \textbf{Solution :} On ajoute un terme proportionnel à l'écart $(x-a)$ : $P_1(x) = f(a) + c_1 (x-a)$.
    \newline
    \textbf{Vérification :}
    \begin{itemize}
        \item $P_1(1) = 0 + c_1(1-1) = 0$. La valeur correspond.
        \item On dérive : $P_1'(x) = c_1$. On veut $P_1'(1) = c_1 = f'(1)$.
        \item La dérivée de $f(x)=\ln(x)$ est $f'(x) = 1/x$, donc $f'(1)=1$. On doit choisir $\mathbf{c_1=1}$.
    \end{itemize}
    Notre polynôme est $\mathbf{P_1(x) = (x-1)}$. C'est la tangente à $\ln(x)$ en $x=1$.

    \item \textbf{Ordre 2 : Faire correspondre la courbure}
    \newline
    \textbf{Objectif :} Les dérivées jusqu'à l'ordre 2 doivent correspondre en $a=1$.
    \newline
    \textbf{Solution :} On ajoute un terme en $(x-a)^2$ : $P_2(x) = (x-1) + c_2 (x-1)^2$.
    \newline
    \textbf{Vérification :}
    \begin{itemize}
        \item Les correspondances d'ordre 0 et 1 sont préservées.
        \item On dérive deux fois : $P_2'(x) = 1 + 2c_2(x-1)$ et $P_2''(x) = 2c_2$.
        \item On veut $P_2''(1) = 2c_2 = f''(1)$.
        \item La dérivée seconde de $f(x)$ est $f''(x) = -1/x^2$, donc $f''(1)=-1$. On choisit $\mathbf{c_2 = -1/2}$.
    \end{itemize}
    Notre polynôme est $\mathbf{P_2(x) = (x-1) - \frac{1}{2}(x-1)^2}$.

    \item \textbf{Le schéma général}
    \newline
    Le coefficient $c_k$ du terme $(x-a)^k$ est choisi pour faire correspondre la $k$-ième dérivée. La dérivation de $c_k(x-a)^k$ $k$ fois donne $c_k \cdot k!$. On impose donc $c_k \cdot k! = f^{(k)}(a)$, ce qui mène directement à la formule générale $c_k = \frac{f^{(k)}(a)}{k!}$.
\end{enumerate}

\tcblower
\centering
\begin{tikzpicture}
    \begin{axis}[
        xlabel={$x$},
        ylabel={$y$},
        xmin=-0.5, xmax=2.5,
        ymin=-2, ymax=1,
        axis lines=middle,
        legend style={at={(0.05,0.05)}, anchor=south west, font=\small},
        grid=major,
        samples=150,
        domain=0.01:2.5,
        height=9cm,
        width=\linewidth-1cm,
        tick label style={font=\tiny}
    ]
    
    \addplot[black, dashed, ultra thick] {ln(x)};
    \addlegendentry{$\ln(x)$}

    \addplot[red, thick, domain=-0.5:2.5] {0};
    \addlegendentry{$P_0(x)=0$}

    \addplot[blue, thick, domain=-0.5:2.5] {x-1};
    \addlegendentry{$P_1(x)=(x-1)$}

    \addplot[green!70!black, thick, domain=-0.5:2.5] {(x-1) - 0.5*(x-1)^2};
    \addlegendentry{$P_2(x)=(x-1)-\frac{(x-1)^2}{2}$}

    \end{axis}
\end{tikzpicture}
\par\small\textit{Approximation de $\ln(x)$ autour de $a=1$. Le polynôme "colle" à la fonction près de $x=1$.}
\end{intuitionbox}


\subsection{La Fonction Exponentielle ($e^x$)}

\begin{theorembox}[Série de Maclaurin pour $e^x$]
Pour tout nombre réel $x$, la fonction exponentielle peut s'écrire :
$$ e^x = \sum_{k=0}^{\infty} \frac{x^k}{k!} = 1 + x + \frac{x^2}{2!} + \frac{x^3}{3!} + \frac{x^4}{4!} + \cdots $$
\end{theorembox}

\begin{intuitionbox}[Visualiser la Croissance Exponentielle]
La fonction exponentielle est unique car elle est sa propre dérivée. Cela signifie que toutes ses informations locales (valeur, pente, courbure) en $a=0$ sont égales à \textbf{1}. La série pour $e^x$ est donc le polynôme le plus « pur », où chaque terme $x^k$ est simplement normalisé par $k!$. Le graphique ci-dessous montre comment les polynômes de Taylor convergent rapidement vers la véritable courbe exponentielle, illustrant sa croissance puissante.

\tcblower

\centering
\begin{tikzpicture}
    \begin{axis}[
        xlabel={$x$},
        ylabel={$y$},
        xmin=-3, xmax=3,
        ymin=-1, ymax=9,
        axis lines=middle,
        legend style={at={(0.05,0.95)}, anchor=north west, font=\small},
        grid=major,
        samples=150,
        domain=-3:3,
        height=9cm,
        width=\linewidth-1cm,
        tick label style={font=\tiny}
    ]
    
    \addplot[black, dashed, ultra thick] {exp(x)};
    \addlegendentry{$e^x$}

    \addplot[red, thick] {1+x};
    \addlegendentry{$T_1(x)$}

    \addplot[blue, thick] {1+x+x^2/2};
    \addlegendentry{$T_2(x)$}

    \addplot[green!70!black, thick] {1+x+x^2/2+x^3/6};
    \addlegendentry{$T_3(x)$}

    \addplot[orange, thick] {1+x+x^2/2+x^3/6+x^4/24};
    \addlegendentry{$T_4(x)$}

    \end{axis}
\end{tikzpicture}
\par\small\textit{Approximation de $e^x$ par ses polynômes de Maclaurin.}
\end{intuitionbox}

\begin{proofbox}
Soit $f(x) = e^x$. Pour tout entier $k \ge 0$, la $k$-ième dérivée est $f^{(k)}(x) = e^x$. En évaluant en $a=0$, on obtient $f^{(k)}(0) = e^0 = 1$ pour tout $k$. En appliquant la formule de Maclaurin :
$$ e^x = \sum_{k=0}^{\infty} \frac{f^{(k)}(0)}{k!} x^k = \sum_{k=0}^{\infty} \frac{1}{k!} x^k = 1 + x + \frac{x^2}{2} + \frac{x^3}{6} + \cdots $$
\end{proofbox}


\subsection{La Fonction Sinus ($\sin(x)$)}

\begin{theorembox}[Série de Maclaurin pour $\sin(x)$]
Pour tout nombre réel $x$ :
$$ \sin(x) = \sum_{k=0}^{\infty} (-1)^k \frac{x^{2k+1}}{(2k+1)!} = x - \frac{x^3}{3!} + \frac{x^5}{5!} - \frac{x^7}{7!} + \cdots $$
\end{theorembox}

\begin{intuitionbox}[Visualiser l'Oscillation du Sinus]
La série du sinus reflète ses propriétés fondamentales. En tant que fonction \textbf{impaire} ($ \sin(-x) = -\sin(x) $), son développement ne contient que des puissances \textbf{impaires} de $x$. Les signes alternés capturent sa nature oscillatoire. Le graphique ci-dessous montre comment l'ajout de termes permet au polynôme d'« épouser » la courbe du sinus sur un plus grand nombre de périodes.

\tcblower

\centering
\begin{tikzpicture}
    \begin{axis}[
        xlabel={$x$},
        ylabel={$y$},
        xmin=-2*pi, xmax=2*pi,
        ymin=-2.0, ymax=2.0,
        axis lines=middle,
        legend style={at={(0.5,1.15)}, anchor=south, font=\small, column sep=5pt},
        legend columns=3,
        grid=major,
        samples=200,
        domain=-2*pi:2*pi,
        height=9cm,
        width=\linewidth-1cm,
        tick label style={font=\tiny}
    ]
    \addplot[black, dashed, ultra thick] {sin(deg(x))};
    \addlegendentry{$\sin(x)$}
    \addplot[red, thick] {x};
    \addlegendentry{$T_1(x)$}
    \addplot[blue, thick] {x - (x^3)/6};
    \addlegendentry{$T_3(x)$}
    \addplot[green!70!black, thick] {x - (x^3)/6 + (x^5)/120};
    \addlegendentry{$T_5(x)$}
    \addplot[orange, thick] {x - (x^3)/6 + (x^5)/120 - (x^7)/5040};
    \addlegendentry{$T_7(x)$}
    \end{axis}
\end{tikzpicture}
\par\small\textit{Approximation de $\sin(x)$ par ses polynômes de Maclaurin.}
\end{intuitionbox}

\begin{proofbox}
Soit $f(x) = \sin(x)$. Les dérivées en $a=0$ suivent un cycle $(0, 1, 0, -1, \dots)$. Seuls les termes d'ordre impair ($2k+1$) sont non nuls, avec des valeurs de $(-1)^k$, ce qui donne la formule.
\end{proofbox}


\subsection{La Fonction Cosinus ($\cos(x)$)}

\begin{theorembox}[Série de Maclaurin pour $\cos(x)$]
Pour tout nombre réel $x$ :
$$ \cos(x) = \sum_{k=0}^{\infty} (-1)^k \frac{x^{2k}}{(2k)!} = 1 - \frac{x^2}{2!} + \frac{x^4}{4!} - \frac{x^6}{6!} + \cdots $$
\end{theorembox}

\begin{intuitionbox}[Visualiser la Symétrie du Cosinus]
En tant que fonction \textbf{paire} ($ \cos(-x) = \cos(x) $), la série du cosinus ne contient, de manière appropriée, que des puissances \textbf{paires} de $x$. Elle commence à 1 (son maximum) puis oscille, un comportement capturé par les signes alternés.

\tcblower

\centering
\begin{tikzpicture}
    \begin{axis}[
        xlabel={$x$},
        ylabel={$y$},
        xmin=-2*pi, xmax=2*pi,
        ymin=-2.0, ymax=2.0,
        axis lines=middle,
        legend style={at={(0.5,1.15)}, anchor=south, font=\small, column sep=5pt},
        legend columns=3,
        grid=major,
        samples=200,
        domain=-2*pi:2*pi,
        height=9cm,
        width=\linewidth-1cm,
        tick label style={font=\tiny}
    ]
    \addplot[black, dashed, ultra thick] {cos(deg(x))};
    \addlegendentry{$\cos(x)$}
    \addplot[red, thick] {1};
    \addlegendentry{$T_0(x)$}
    \addplot[blue, thick] {1 - x^2/2};
    \addlegendentry{$T_2(x)$}
    \addplot[green!70!black, thick] {1 - x^2/2 + x^4/24};
    \addlegendentry{$T_4(x)$}
    \addplot[orange, thick] {1 - x^2/2 + x^4/24 - x^6/720};
    \addlegendentry{$T_6(x)$}
    \end{axis}
\end{tikzpicture}
\par\small\textit{Approximation de $\cos(x)$ par ses polynômes de Maclaurin.}
\end{intuitionbox}

\begin{proofbox}
Soit $g(x) = \cos(x)$. Les dérivées en $a=0$ suivent un cycle $(1, 0, -1, 0, \dots)$. Seuls les termes d'ordre pair ($2k$) sont non nuls, avec des valeurs de $(-1)^k$, ce qui donne la formule.
\end{proofbox}


\subsection{Le Logarithme Népérien ($\ln(1+x)$)}

\begin{theorembox}[Série de Maclaurin pour $\ln(1+x)$]
Pour $|x| < 1$ :
$$ \ln(1+x) = \sum_{k=1}^{\infty} (-1)^{k-1} \frac{x^k}{k} = x - \frac{x^2}{2} + \frac{x^3}{3} - \frac{x^4}{4} + \cdots $$
\end{theorembox}

\begin{intuitionbox}[Visualiser l'Approximation Logarithmique]
Cette série est essentielle pour approximer les logarithmes près de 1. Contrairement aux fonctions précédentes, elle ne converge que pour $|x|<1$. Le graphique montre que l'approximation est excellente près de $x=0$ mais diverge rapidement lorsque $x$ s'approche de la frontière de convergence à $x=1$.

\tcblower

\centering
\begin{tikzpicture}
    \begin{axis}[
        xlabel={$x$},
        ylabel={$y$},
        xmin=-1.2, xmax=1.2,
        ymin=-4, ymax=2,
        axis lines=middle,
        legend style={at={(0.05,0.95)}, anchor=north west, font=\small},
        grid=major,
        samples=150,
        domain=-0.99:1, % Domain restricted for ln
        height=9cm,
        width=\linewidth-1cm,
        tick label style={font=\tiny}
    ]
    \addplot[black, dashed, ultra thick] {ln(1+x)};
    \addlegendentry{$\ln(1+x)$}
    
    \addplot[red, thick, domain=-1.2:1.2] {x};
    \addlegendentry{$T_1(x)$}

    \addplot[blue, thick, domain=-1.2:1.2] {x - x^2/2};
    \addlegendentry{$T_2(x)$}

    \addplot[green!70!black, thick, domain=-1.2:1.2] {x - x^2/2 + x^3/3};
    \addlegendentry{$T_3(x)$}
    
    \addplot[orange, thick, domain=-1.2:1.2] {x - x^2/2 + x^3/3 - x^4/4};
    \addlegendentry{$T_4(x)$}

    \end{axis}
\end{tikzpicture}
\par\small\textit{Approximation de $\ln(1+x)$ par ses polynômes de Maclaurin.}
\end{intuitionbox}

\begin{proofbox}
Soit $f(x) = \ln(1+x)$. Pour $k \ge 1$, la $k$-ième dérivée en $a=0$ est $f^{(k)}(0) = (-1)^{k-1} (k-1)!$. En substituant cela dans la formule de Maclaurin, le $(k-1)!$ au numérateur annule partiellement le $k!$ au dénominateur, laissant un $k$ en bas.
\end{proofbox}

\subsection{La Série Géométrique ($\frac{1}{1-x}$)}

\begin{theorembox}[Série de Maclaurin pour $\frac{1}{1-x}$]
Pour $|x| < 1$ :
$$ \frac{1}{1-x} = \sum_{k=0}^{\infty} x^k = 1 + x + x^2 + x^3 + \cdots $$
\end{theorembox}

\begin{intuitionbox}[Le Fondement de Nombreuses Séries]
Cette série, connue sous le nom de série géométrique, est l'un des développements en série de puissances les plus fondamentaux. Elle converge uniquement lorsque la valeur absolue de $x$ est inférieure à 1. Chaque coefficient est simplement 1, ce qui en fait la série de Maclaurin la plus simple. De nombreuses autres séries, comme celle de $\ln(1+x)$ ou de $\arctan(x)$, peuvent être dérivées de celle-ci par intégration ou substitution.
\end{intuitionbox}

\begin{proofbox}
Soit $f(x) = (1-x)^{-1}$. Les dérivées successives sont $f'(x) = 1(1-x)^{-2}$, $f''(x) = 2(1-x)^{-3}$, $f'''(x) = 6(1-x)^{-4}$, et ainsi de suite. La formule générale pour la $k$-ième dérivée est $f^{(k)}(x) = k!(1-x)^{-(k+1)}$. En évaluant en $a=0$, on obtient $f^{(k)}(0) = k!$. En substituant dans la formule de Maclaurin :
$$ \frac{1}{1-x} = \sum_{k=0}^{\infty} \frac{f^{(k)}(0)}{k!} x^k = \sum_{k=0}^{\infty} \frac{k!}{k!} x^k = \sum_{k=0}^{\infty} x^k $$
\end{proofbox}

\subsection{Exercices}

\begin{exercicebox}[Termes de base]
Trouvez les quatre premiers termes non nuls de la série de Maclaurin pour $f(x) = \cos(2x)$.
\end{exercicebox}

\begin{correctionbox}
On utilise la série connue de $\cos(u) = 1 - \frac{u^2}{2!} + \frac{u^4}{4!} - \frac{u^6}{6!} + \cdots$.
En substituant $u = 2x$, on obtient :
$$ \cos(2x) = 1 - \frac{(2x)^2}{2!} + \frac{(2x)^4}{4!} - \frac{(2x)^6}{6!} + \cdots $$
$$ \cos(2x) = 1 - \frac{4x^2}{2} + \frac{16x^4}{24} - \frac{64x^6}{5040} + \cdots $$
$$ \cos(2x) = 1 - 2x^2 + \frac{2}{3}x^4 - \frac{4}{315}x^6 + \cdots $$
Les quatre premiers termes sont $1$, $-2x^2$, $\frac{2}{3}x^4$ et $-\frac{4}{315}x^6$.
\end{correctionbox}

\begin{exercicebox}[Dérivation de séries]
Utilisez la série de Maclaurin de $\sin(x)$ pour trouver la série de Maclaurin de $\cos(x)$.
\end{exercicebox}

\begin{correctionbox}
On sait que $\frac{d}{dx}(\sin(x)) = \cos(x)$. On peut dériver la série de $\sin(x)$ terme à terme :
$$ \sin(x) = x - \frac{x^3}{3!} + \frac{x^5}{5!} - \frac{x^7}{7!} + \cdots $$
$$ \frac{d}{dx} \sin(x) = \frac{d}{dx} \left( x - \frac{x^3}{6} + \frac{x^5}{120} - \cdots \right) $$
$$ \cos(x) = 1 - \frac{3x^2}{6} + \frac{5x^4}{120} - \cdots = 1 - \frac{x^2}{2} + \frac{x^4}{24} - \cdots = 1 - \frac{x^2}{2!} + \frac{x^4}{4!} - \cdots $$
On retrouve bien la série de Maclaurin pour $\cos(x)$.
\end{correctionbox}

\begin{exercicebox}[Intégration de séries]
Trouvez la série de Maclaurin pour $\arctan(x)$ en intégrant la série de $\frac{1}{1+x^2}$.
\end{exercicebox}

\begin{correctionbox}
On part de la série géométrique $\frac{1}{1-u} = \sum_{k=0}^{\infty} u^k$. En posant $u = -x^2$, on obtient :
$$ \frac{1}{1+x^2} = \sum_{k=0}^{\infty} (-x^2)^k = \sum_{k=0}^{\infty} (-1)^k x^{2k} = 1 - x^2 + x^4 - x^6 + \cdots $$
Puisque $\int \frac{1}{1+x^2} dx = \arctan(x)$, on intègre la série terme à terme :
$$ \arctan(x) = \int (1 - x^2 + x^4 - \cdots) dx = C + x - \frac{x^3}{3} + \frac{x^5}{5} - \cdots $$
Comme $\arctan(0) = 0$, la constante d'intégration $C$ est nulle.
$$ \arctan(x) = \sum_{k=0}^{\infty} (-1)^k \frac{x^{2k+1}}{2k+1} $$
\end{correctionbox}

\begin{exercicebox}[Approximation de valeur]
Utilisez les trois premiers termes non nuls de la série de Maclaurin de $e^x$ pour approximer la valeur de $\sqrt{e}$.
\end{exercicebox}

\begin{correctionbox}
On veut approximer $\sqrt{e} = e^{0.5}$. La série est $e^x \approx 1 + x + \frac{x^2}{2!}$.
En posant $x=0.5$ :
$$ e^{0.5} \approx 1 + 0.5 + \frac{(0.5)^2}{2} = 1 + 0.5 + \frac{0.25}{2} = 1.5 + 0.125 = 1.625 $$
La valeur réelle est $e^{0.5} \approx 1.6487$. L'approximation est raisonnablement proche.
\end{correctionbox}

\begin{exercicebox}[Série de Taylor non centrée en 0]
Trouvez la série de Taylor pour $f(x) = \ln(x)$ centrée en $a=1$.
\end{exercicebox}

\begin{correctionbox}
On calcule les dérivées de $f(x)=\ln(x)$ et on les évalue en $a=1$.
$f(x) = \ln(x) \implies f(1) = 0$
$f'(x) = 1/x \implies f'(1) = 1$
$f''(x) = -1/x^2 \implies f''(1) = -1$
$f'''(x) = 2/x^3 \implies f'''(1) = 2$
$f^{(k)}(x) = (-1)^{k-1}(k-1)!/x^k \implies f^{(k)}(1) = (-1)^{k-1}(k-1)!$ pour $k \ge 1$.
La série de Taylor est :
$$ \ln(x) = \sum_{k=1}^{\infty} \frac{(-1)^{k-1}(k-1)!}{k!} (x-1)^k = \sum_{k=1}^{\infty} \frac{(-1)^{k-1}}{k} (x-1)^k $$
\end{correctionbox}

\begin{exercicebox}[Identifier une fonction]
Quelle fonction est représentée par la série de Maclaurin $ \sum_{k=0}^{\infty} \frac{(-1)^k x^{2k}}{(2k)!} $ ?
\end{exercicebox}

\begin{correctionbox}
Cette série est $1 - \frac{x^2}{2!} + \frac{x^4}{4!} - \frac{x^6}{6!} + \cdots$. Il s'agit de la série de Maclaurin de la fonction $\cos(x)$.
\end{correctionbox}

\begin{exercicebox}[Série binomiale]
Trouvez les trois premiers termes de la série de Maclaurin pour $f(x) = \sqrt{1+x}$.
\end{exercicebox}

\begin{correctionbox}
On utilise la série binomiale $(1+x)^\alpha$ avec $\alpha=1/2$.
$$ (1+x)^{1/2} = 1 + \alpha x + \frac{\alpha(\alpha-1)}{2!}x^2 + \cdots $$
$$ (1+x)^{1/2} = 1 + \frac{1}{2}x + \frac{\frac{1}{2}(\frac{1}{2}-1)}{2}x^2 + \cdots $$
$$ (1+x)^{1/2} = 1 + \frac{1}{2}x + \frac{\frac{1}{2}(-\frac{1}{2})}{2}x^2 + \cdots = 1 + \frac{1}{2}x - \frac{1}{8}x^2 + \cdots $$
\end{correctionbox}

\begin{exercicebox}[Multiplication de séries]
Trouvez les termes jusqu'à $x^3$ pour la série de Maclaurin de $f(x) = e^x \sin(x)$.
\end{exercicebox}

\begin{correctionbox}
On multiplie les développements de $e^x$ et $\sin(x)$ :
$$ e^x \sin(x) = \left(1 + x + \frac{x^2}{2} + \frac{x^3}{6} + \cdots\right) \left(x - \frac{x^3}{6} + \cdots\right) $$
On collecte les termes par puissance croissante :
\begin{itemize}
    \item Terme en $x$ : $1 \cdot x = x$
    \item Terme en $x^2$ : $x \cdot x = x^2$
    \item Terme en $x^3$ : $1 \cdot (-\frac{x^3}{6}) + \frac{x^2}{2} \cdot x = -\frac{x^3}{6} + \frac{x^3}{2} = \frac{2x^3}{6} = \frac{x^3}{3}$
\end{itemize}
Donc, $e^x \sin(x) = x + x^2 + \frac{x^3}{3} + \cdots$.
\end{correctionbox}

\begin{exercicebox}[Calcul de limite]
Évaluez la limite suivante en utilisant les séries de Maclaurin : $ \lim_{x \to 0} \frac{\sin(x) - x}{x^3} $.
\end{exercicebox}

\begin{correctionbox}
On remplace $\sin(x)$ par son développement :
$$ \lim_{x \to 0} \frac{(x - \frac{x^3}{3!} + \frac{x^5}{5!} - \cdots) - x}{x^3} $$
$$ = \lim_{x \to 0} \frac{-\frac{x^3}{6} + \frac{x^5}{120} - \cdots}{x^3} $$
$$ = \lim_{x \to 0} \left(-\frac{1}{6} + \frac{x^2}{120} - \cdots\right) = -\frac{1}{6} $$
\end{correctionbox}

\begin{exercicebox}[Fonction hyperbolique]
Trouvez la série de Maclaurin pour le cosinus hyperbolique, $\cosh(x) = \frac{e^x + e^{-x}}{2}$.
\end{exercicebox}

\begin{correctionbox}
On utilise les séries de $e^x$ et $e^{-x}$ :
$e^x = 1 + x + \frac{x^2}{2!} + \frac{x^3}{3!} + \cdots$
$e^{-x} = 1 - x + \frac{x^2}{2!} - \frac{x^3}{3!} + \cdots$
En les additionnant, les termes de puissance impaire s'annulent :
$e^x + e^{-x} = 2 + 2\frac{x^2}{2!} + 2\frac{x^4}{4!} + \cdots$
En divisant par 2 :
$$ \cosh(x) = 1 + \frac{x^2}{2!} + \frac{x^4}{4!} + \frac{x^6}{6!} + \cdots = \sum_{k=0}^{\infty} \frac{x^{2k}}{(2k)!} $$
\end{correctionbox}

\begin{exercicebox}[Coefficient de Taylor]
Soit $f(x) = \frac{x^2}{1+x^3}$. Trouvez la valeur de la 8ème dérivée en zéro, $f^{(8)}(0)$.
\end{exercicebox}

\begin{correctionbox}
On sait que le coefficient du terme $x^k$ dans une série de Maclaurin est $\frac{f^{(k)}(0)}{k!}$.
On développe $f(x)$ :
$$ f(x) = x^2 \cdot \frac{1}{1-(-x^3)} = x^2 \sum_{n=0}^{\infty} (-x^3)^n = x^2 \sum_{n=0}^{\infty} (-1)^n x^{3n} = \sum_{n=0}^{\infty} (-1)^n x^{3n+2} $$
On cherche le terme en $x^8$. On doit avoir $3n+2=8$, ce qui donne $3n=6$, soit $n=2$.
Le coefficient de $x^8$ est donc $(-1)^2 = 1$.
On a alors $\frac{f^{(8)}(0)}{8!} = 1$, ce qui implique $f^{(8)}(0) = 8! = 40320$.
\end{correctionbox}

\begin{exercicebox}[Approximation d'intégrale]
Estimez la valeur de $\int_0^1 \sin(x^2) dx$ en utilisant les deux premiers termes non nuls de la série de Maclaurin de la fonction à intégrer.
\end{exercicebox}

\begin{correctionbox}
On part de $\sin(u) = u - \frac{u^3}{6} + \cdots$. On pose $u=x^2$ :
$$ \sin(x^2) = x^2 - \frac{(x^2)^3}{6} + \cdots = x^2 - \frac{x^6}{6} + \cdots $$
On intègre ce polynôme de 0 à 1 :
$$ \int_0^1 \left(x^2 - \frac{x^6}{6}\right) dx = \left[ \frac{x^3}{3} - \frac{x^7}{42} \right]_0^1 $$
$$ = \left(\frac{1}{3} - \frac{1}{42}\right) - 0 = \frac{14 - 1}{42} = \frac{13}{42} \approx 0.3095 $$
\end{correctionbox}

\begin{exercicebox}[Un autre centre]
Trouvez les trois premiers termes de la série de Taylor pour $f(x) = \frac{1}{x}$ centrée en $a=2$.
\end{exercicebox}

\begin{correctionbox}
On calcule les dérivées et on les évalue en $a=2$.
$f(x) = x^{-1} \implies f(2) = 1/2$
$f'(x) = -x^{-2} \implies f'(2) = -1/4$
$f''(x) = 2x^{-3} \implies f''(2) = 2/8 = 1/4$
La série commence par :
$$ f(x) \approx f(2) + f'(2)(x-2) + \frac{f''(2)}{2!}(x-2)^2 $$
$$ f(x) \approx \frac{1}{2} - \frac{1}{4}(x-2) + \frac{1/4}{2}(x-2)^2 = \frac{1}{2} - \frac{1}{4}(x-2) + \frac{1}{8}(x-2)^2 $$
\end{correctionbox}

\begin{exercicebox}[Combinaison de séries]
Trouvez le terme en $x^4$ du développement de Maclaurin de $f(x) = \ln(1-x^2)$.
\end{exercicebox}

\begin{correctionbox}
On utilise la série de $\ln(1+u) = u - \frac{u^2}{2} + \frac{u^3}{3} - \frac{u^4}{4} + \cdots$.
On substitue $u = -x^2$ :
$$ \ln(1-x^2) = (-x^2) - \frac{(-x^2)^2}{2} + \frac{(-x^2)^3}{3} - \frac{(-x^2)^4}{4} + \cdots $$
$$ = -x^2 - \frac{x^4}{2} - \frac{x^6}{3} - \frac{x^8}{4} - \cdots $$
Le terme en $x^4$ est $-\frac{x^4}{2}$.
\end{correctionbox}

\begin{exercicebox}[Application en physique]
En relativité restreinte, l'énergie cinétique d'une particule est $K = mc^2(\gamma - 1)$, où $\gamma = (1-v^2/c^2)^{-1/2}$. Montrez que pour des vitesses faibles ($v \ll c$), cette formule se réduit à la formule classique $K \approx \frac{1}{2}mv^2$.
\end{exercicebox}

\begin{correctionbox}
On utilise le développement binomial $(1+x)^\alpha$ avec $x = -v^2/c^2$ et $\alpha = -1/2$.
$$ \gamma = \left(1 - \frac{v^2}{c^2}\right)^{-1/2} \approx 1 + \alpha x = 1 + \left(-\frac{1}{2}\right)\left(-\frac{v^2}{c^2}\right) = 1 + \frac{1}{2}\frac{v^2}{c^2} $$
On substitue ce résultat dans la formule de l'énergie :
$$ K = mc^2(\gamma - 1) \approx mc^2 \left( \left(1 + \frac{1}{2}\frac{v^2}{c^2}\right) - 1 \right) $$
$$ K \approx mc^2 \left( \frac{1}{2}\frac{v^2}{c^2} \right) = \frac{1}{2}mv^2 $$
On retrouve bien l'énergie cinétique classique comme approximation de premier ordre.
\end{correctionbox}

\section{Exemples de Code}

\begin{examplebox}[Simulation d'un lancer de dé]
On peut simuler $n$ lancers d'un dé équilibré à 6 faces en utilisant la bibliothèque \texttt{random} de Python.

\begin{codecell}
import random
import numpy
\end{codecell}

\begin{outputcell}
>> "vamonos"
\end{outputcell}

\end{examplebox}

\begin{definitionbox}[Variable Aléatoire]
Une variable aléatoire est une fonction qui associe un nombre réel à chaque résultat possible d'une expérience aléatoire.
\end{definitionbox}

\begin{codecell}
import math
import random

def poisson_knuth(lmbda: float) -> int:
  """
  Simule une variable aleatoire suivant une loi de Poisson()
  en utilisant l algorithme de Knuth.
  """
  L = math.exp(-lmbda)
  k = 0
  p = 1.0

  while p > L:
  k += 1
  p *= random.random()

  return k - 1
\end{codecell}



\end{document}