\documentclass{article}

% --- GESTION DES MARGES DE PAGE ---
\usepackage[a4paper, top=2.5cm, bottom=2.5cm, left=3cm, right=3cm]{geometry}

% --- PRÉAMBULE STANDARD ---
\usepackage[utf8]{inputenc}
\usepackage[T1]{fontenc}
\usepackage{lmodern}
\usepackage[french]{babel}
\usepackage{xcolor}
\usepackage{tcolorbox}
\usepackage{listings}
\usepackage{amsmath}
\usepackage{graphicx} % Requis pour inclure des images
\usepackage{amssymb} % Pour les symboles mathématiques comme \subseteq
\usepackage{sectsty} % Pour le style des sections
\usepackage{etoolbox} % Pour les conditions
\usepackage[dvipsnames]{xcolor}
\usepackage{pgfplots} % Le package principal pour les graphiques
\usepackage{enumitem}

\setlist[itemize,1]{label=$\cdot$}

\usetikzlibrary{
    matrix, 
    patterns.meta,
    calc,
    positioning,
    decorations.pathreplacing,
    trees,
    backgrounds
}

% Redéfinir le symbole pour le premier niveau de la liste
\renewcommand{\labelitemi}{$\cdot$}
\renewcommand{\labelitemii}{$\circ$}
\renewcommand{\labelitemiii}{$\ast$}

% Styles de hachures (inchangés)
\tikzset{
  red_hatch/.style={
    pattern={Lines[angle=45, line width=0.8pt, distance=4pt]}, 
    pattern color=red
  },
  blue_hatch/.style={
    pattern={Lines[angle=-45, line width=0.8pt, distance=4pt]}, 
    pattern color=blue
  },
  purple_hatch/.style={
    pattern={Lines[angle=45, line width=0.8pt, distance=4pt]}, 
    pattern color=red,
    postaction={
      pattern={Lines[angle=-45, line width=0.8pt, distance=4pt]}, 
      pattern color=blue
    }
  }
}

% --- BIBLIOTHÈQUES TCOLORBOX ---
\tcbuselibrary{listings, skins, breakable}

% --- GESTION DES LIENS HYPERTEXTE ---
\usepackage[colorlinks=true, linkcolor=black, urlcolor=blue]{hyperref}

% --- SOULIGNER LES TITRES ET SOUS-TITRES ---
\sectionfont{\underline}
\subsectionfont{\underline}

% --- PAGE DE GARDE AMÉLIORÉE ---
\makeatletter
\renewcommand{\maketitle}{%
\begin{titlepage}
\centering
\vspace*{\stretch{1.5}}
{\Huge \bfseries Mes Notes de Lecture\par}
\vspace{0.4cm}
\rule{0.8\linewidth}{0.4pt}
\vspace{1cm}
{\LARGE \bfseries Introduction à la Probabilité\par}
\vspace*{\stretch{2.5}}
{\Large \scshape Lou Brunet\par}
\vspace{0.5cm}
{\large \today\par}
\vspace*{\stretch{1}}
\end{titlepage}
}
\makeatother

% --- DÉFINITION DES COULEURS STYLE VS CODE ---
\definecolor{vscodeBlue}{HTML}{569CD6}
\definecolor{vscodeOrange}{HTML}{CE9178}
\definecolor{vscodeGreen}{HTML}{6A9955}
\definecolor{vscodePurple}{HTML}{C586C0}
\definecolor{vscodeGray}{HTML}{9B9B9B}
\definecolor{codeBackground}{HTML}{1E1E1E}

% --- CONFIGURATION DU STYLE LISTINGS ---
\lstdefinestyle{vscode}{
    language=Python,
    backgroundcolor=\color{codeBackground},
    basicstyle=\ttfamily\small\color{white},
    keywordstyle=\color{vscodeBlue},
    stringstyle=\color{vscodeOrange},
    commentstyle=\color{vscodeGreen},
    numberstyle=\tiny\color{vscodeGray},
    otherkeywords={self, True, False, None},
    keywordstyle=[2]\color{vscodePurple},
    showstringspaces=false,
    breaklines=true,
    frame=none,
    tabsize=4
}

% --- STYLE DE L'OUTPUT ---
\lstdefinestyle{outputstyle}{
    basicstyle=\ttfamily\small\color{white},
    numbers=left,
    numberstyle=\color{white}\def\thelstnumber{>>},
    numbersep=5pt,
    breaklines=true,
    frame=none
}

% --- DÉFINITION DES CELLULES DE CODE ET OUTPUT ---
\newtcblisting{codecell}{
  arc=2mm, boxrule=0.8pt, colframe=vscodeBlue, colback=codeBackground,
  coltitle=white, fonttitle=\bfseries, title=Code Python,
  listing only, listing options={style=vscode, basicstyle=\ttfamily\footnotesize\color{white}},
  left=6mm, right=6mm, top=3mm, bottom=3mm, boxsep=4mm
}
\newtcblisting{outputcell}{
  arc=2mm, boxrule=0.8pt, colframe=black!95, colback=black!85,
  fonttitle=\bfseries\color{white}, title=Output, coltitle=black!95,
  listing only, listing options={style=outputstyle, basicstyle=\ttfamily\footnotesize\color{white}},
  left=6mm, right=6mm, top=3mm, bottom=3mm, boxsep=4mm
}

% --- DÉFINITION DES COULEURS POUR DÉF/THÉO/PREUVE/INTUITION/EXEMPLE ---
\definecolor{defColor}{HTML}{1b1f3a}       % Bleu nuit pour les définitions
\definecolor{theoColor}{HTML}{53354a}      % Violet aubergine pour les théorèmes
\definecolor{proofColor}{HTML}{a64942}     % Rouge brique pour les preuves
\definecolor{intuitionColor}{HTML}{16A085} % Turquoise pour l'intuition
\definecolor{exampleColor}{HTML}{4a6982}   % Bleu ardoise pour les exemples
\definecolor{exoColor}{HTML}{1E8449}       % Vert
\definecolor{corrColor}{HTML}{7F8C8D}      % Gris

% --- DÉFINITION D'UN STYLE DE BASE COMMUN ---
\tcbset{
    baseboxstyle/.style={
        arc=2mm, boxrule=0.8pt,
        fontupper=\color{black}, left=6mm, right=6mm, top=3mm, bottom=3mm,
        boxsep=4mm, breakable
    }
}

% --- DÉFINITION DES CELLULES THÉORIQUES AVEC FOND DE COULEUR ---
\newtcolorbox{definitionbox}[1][]{
  baseboxstyle, colframe=defColor, colback=defColor!10, coltitle=defColor,
  fonttitle=\bfseries\color{white},
  title=Définition\ifstrempty{#1}{}{ : #1}
}
\newtcolorbox{theorembox}[1][]{
  baseboxstyle, colframe=theoColor, colback=theoColor!10, coltitle=theoColor,
  fonttitle=\bfseries\color{white},
  title=Théorème\ifstrempty{#1}{}{ : #1}
}
\newtcolorbox{proofbox}[1][]{
  baseboxstyle, colframe=proofColor, colback=proofColor!10, coltitle=proofColor,
  fonttitle=\bfseries\color{white},
  title=Preuve\ifstrempty{#1}{}{ : #1}
}
\newtcolorbox{intuitionbox}[1][]{
  baseboxstyle, colframe=intuitionColor, colback=intuitionColor!10, coltitle=intuitionColor,
  fonttitle=\bfseries\color{white},
  title=Intuition\ifstrempty{#1}{}{ : #1}
}
\newtcolorbox{examplebox}[1][]{
  baseboxstyle, colframe=exampleColor, colback=exampleColor!10, coltitle=exampleColor,
  fonttitle=\bfseries\color{white},
  title=Exemple\ifstrempty{#1}{}{ : #1}
}

% --- DÉFINITION DES CELLULES EXERCICE (GRIS) ET CORRECTION (VERT) ---
\newtcolorbox{exercicebox}[1][]{
    baseboxstyle, colframe=corrColor, colback=corrColor!10, coltitle=corrColor,
    fonttitle=\bfseries\color{white},
    title=Exercice\ifstrempty{#1}{}{ : #1}
}

\newtcolorbox{correctionbox}[1][]{
    baseboxstyle, colframe=exoColor, colback=exoColor!10, coltitle=exoColor,
    fonttitle=\bfseries\color{white},
    title=Correction\ifstrempty{#1}{}{ : #1}
}

% --- DÉFINITION D'UNE CELLULE remarque ---
\definecolor{remarqueColor}{HTML}{D35400} 

\newtcolorbox{remarquebox}[1][]{
  baseboxstyle, colframe=remarqueColor, colback=remarqueColor!10, coltitle=remarqueColor,
  fonttitle=\bfseries\color{white},
  title=Remarque\ifstrempty{#1}{}{ : #1}
}

% =============================================
% --- CORPS DU DOCUMENT ---
% =============================================
% =============================================
% --- CORPS DU DOCUMENT ---
% =============================================
\begin{document}

\maketitle

\newpage
\tableofcontents 
\newpage

\section{Préface}

\noindent À l'origine de ce projet se trouve une démarche purement personnelle : la volonté de compiler, au sein d'un support numérique unique, les notes et les concepts clés issus de mes lectures universitaires. Progressivement, cette idée a mûri pour répondre à un besoin plus large : celui de disposer d'un outil de référence agile, permettant de revisiter rapidement une notion oubliée. L'ambition est d'y retrouver non seulement une définition rigoureuse, mais également l'intuition qui la sous-tend et un exemple concret pour l'ancrer durablement. C'est cet impératif d'accessibilité et de clarté qui a guidé l'évolution de ce document. Cette démarche s'inscrit dans un contexte où de nombreux domaines de pointe, de l'apprentissage profond à la finance quantitative ou à la physique théorique, exigent la maîtrise d'un socle mathématique dense. L'algèbre linéaire, la théorie des probabilités ou l'analyse en constituent les piliers. Or, il m'a semblé qu'il manquait, notamment dans le paysage francophone, un support synthétique pour se réapproprier efficacement ces fondements. Tel est donc l'esprit de ce document : offrir une passerelle vers des concepts essentiels, en espérant qu'elle s'avérera un allié précieux pour le lecteur.

\newpage
\section{Probabilités et Dénombrement}

\subsection{Concepts fondamentaux}

\begin{intuitionbox}[Nécessité d'un Cadre Formel]
Avant de calculer des probabilités, il est crucial de définir les règles du jeu :
\newline
\textbf{Qu'est-ce qui peut arriver ?}
\newline
On définit l'ensemble de tous les résultats possibles de l'expérience.
\newline
\textbf{À quoi s'intéresse-t-on ?} 
\newline
On identifie les sous-ensembles de résultats spécifiques qui nous intéressent.
\newline
Ces deux idées nous conduisent aux notions d'Univers et d'Événement, qui sont les piliers de toute théorie des probabilités.
\end{intuitionbox}

\begin{definitionbox}[Concepts Fondamentaux]
\textbf{Univers (ou Espace Échantillon), $S$ :} 
\newline
L'ensemble de tous les résultats possibles d'une expérience aléatoire.
\newline
\textbf{Événement, $A$ :} 
\newline
Un sous-ensemble de l'univers ($A \subseteq S$). C'est un ensemble de résultats auxquels on s'intéresse.
\end{definitionbox}

\begin{examplebox}[Univers et Événement]
Pour l'expérience du "lancer d'un dé à six faces" :
\newline
L'\textbf{univers} est $S = \{1, 2, 3, 4, 5, 6\}$.
"Obtenir un nombre impair" est un événement, représenté par le sous-ensemble $A = \{1, 3, 5\}$.
\end{examplebox}

\subsection{Définition Naïve de la Probabilité}

\begin{definitionbox}[Probabilité Naïve]
Pour une expérience où chaque issue dans un espace échantillon fini $S$ est équiprobable, la probabilité d'un événement $A$ est le rapport du nombre d'issues favorables à $A$ sur le nombre total d'issues :
$$ P(A) = \frac{\text{Nombre d'issues favorables}}{\text{Nombre total d'issues}} = \frac{|A|}{|S|} $$
\end{definitionbox}

\begin{examplebox}[Applications de la définition naïve]
\begin{enumerate}
    \item \textbf{Lancer une pièce équilibrée :}
    L'espace échantillon est $S = \{\text{Pile, Face}\}$, donc $|S| = 2$.
    Si l'événement $A$ est "obtenir Pile", alors $A = \{\text{Pile}\}$ et $|A| = 1$.
    La probabilité est $P(A) = \frac{1}{2}$.

    \item \textbf{Lancer un dé à six faces non pipé :}
    L'espace échantillon est $S = \{1, 2, 3, 4, 5, 6\}$, donc $|S| = 6$.
    Si l'événement $B$ est "obtenir un nombre pair", alors $B = \{2, 4, 6\}$ et $|B| = 3$.
    La probabilité est $P(B) = \frac{3}{6} = \frac{1}{2}$.

    \item \textbf{Tirer une carte d'un jeu de 52 cartes :}
    L'espace échantillon $S$ contient 52 cartes, donc $|S| = 52$.
    Si l'événement $C$ est "tirer un Roi", il y a 4 Rois dans le jeu, donc $|C| = 4$.
    La probabilité est $P(C) = \frac{4}{52} = \frac{1}{13}$.
\end{enumerate}
\end{examplebox}

\subsection{Permutations (Arrangements)}

\begin{definitionbox}[Permutation de $k$ objets parmi $n$]
Le nombre de façons d'arranger $k$ objets choisis parmi $n$ objets distincts (où l'ordre compte et il n'y a pas de répétition) est noté $P(n, k)$ ou $A_n^k$ et est défini par :
$$ P(n, k) = \frac{n!}{(n-k)!} $$
où $n!$ est la factorielle de $n$, et par convention $0! = 1$.
\end{definitionbox}

\begin{intuitionbox}[Permutations de $k$ parmi $n$]
Pour placer $k$ objets dans un ordre spécifique en les choisissant parmi $n$ objets disponibles, on a $n$ choix pour la première position, $(n-1)$ choix pour la deuxième, ..., et $(n-k+1)$ choix pour la $k$-ième position. Cela donne $n \times (n-1) \times \cdots \times (n-k+1)$ arrangements. Ce produit contient $k$ termes. Il est égal à $\frac{n!}{(n-k)!}$, car cela revient à diviser la suite complète $n!$ par les facteurs non utilisés $(n-k) \times (n-k-1) \times \cdots \times 1$.
\end{intuitionbox}

\begin{examplebox}[Permutations de $k$ parmi $n$]
\textbf{Podium d'une course :} Une course réunit 8 coureurs. Combien y a-t-il de podiums (1er, 2e, 3e) possibles ? \\
On cherche le nombre de façons d'ordonner 3 coureurs parmi 8 : $P(8, 3)$. 
$$ P(8, 3) = \frac{8!}{(8-3)!} = \frac{8!}{5!} = 8 \times 7 \times 6 = 336 $$
Il y a 336 podiums possibles.
\end{examplebox}

\subsection{Le Coefficient Binomial}

\begin{theorembox}[Formule du Coefficient Binomial]
Le nombre de façons de choisir $k$ objets parmi un ensemble de $n$ objets distincts (sans remise et sans ordre) est donné par le coefficient binomial :
$$ \binom{n}{k} = \frac{n!}{k!(n-k)!} $$
\end{theorembox}

\begin{intuitionbox}

L’idée est de relier $\binom{n}{k}$ à quelque chose de plus facile à compter : les \textbf{permutations} de $k$ objets parmi $n$, c’est-à-dire les listes ordonnées.  
On sait qu’il y en a :
\[
P(n,k) = \frac{n!}{(n-k)!}.
\]

D’un autre côté, on peut construire chaque permutation en deux étapes :
\begin{enumerate}
    \item Choisir un \textbf{sous-ensemble} de $k$ objets (sans ordre), il y a $\binom{n}{k}$ façons de le faire.
    \item Ordonner ces $k$ objets, il y a $k!$ façons de le faire.
\end{enumerate}
Donc, le nombre total de permutations est aussi $\binom{n}{k} \cdot k!$.

\medskip

\noindent En égalisant les deux expressions :
\[
\binom{n}{k} \cdot k! = \frac{n!}{(n-k)!}
\quad\Longrightarrow\quad
\binom{n}{k} = \frac{n!}{k!(n-k)!}.
\]

\medskip

\noindent Pour rendre cela concret, voici le cas $\binom{5}{3}$.  
Il y a 10 sous-ensembles de 3 éléments parmi $\{a,b,c,d,e\}$. Chacun donne lieu à $3! = 6$ permutations.  
Le tableau ci-dessous montre \textbf{toutes les 60 permutations}, regroupées par sous-ensemble :

\begin{center}
\small
\renewcommand{\arraystretch}{0.9}
\setlength{\tabcolsep}{2pt}
\begin{tabular}{|c|c|c|c|c|c|c|c|c|c|}
\hline
\textbf{$\{a,b,c\}$} & \textbf{$\{a,b,d\}$} & \textbf{$\{a,b,e\}$} & \textbf{$\{a,c,d\}$} & \textbf{$\{a,c,e\}$} & \textbf{$\{a,d,e\}$} & \textbf{$\{b,c,d\}$} & \textbf{$\{b,c,e\}$} & \textbf{$\{b,d,e\}$} & \textbf{$\{c,d,e\}$} \\
\hline
$abc$ & $abd$ & $abe$ & $acd$ & $ace$ & $ade$ & $bcd$ & $bce$ & $bde$ & $cde$ \\
\hline
$acb$ & $adb$ & $aeb$ & $adc$ & $aec$ & $aed$ & $bdc$ & $bec$ & $bed$ & $ced$ \\
\hline
$bac$ & $bad$ & $bae$ & $cad$ & $cae$ & $dae$ & $cbd$ & $ceb$ & $dbe$ & $dce$ \\
\hline
$bca$ & $bda$ & $bea$ & $cda$ & $cea$ & $dea$ & $cdb$ & $ceb$ & $deb$ & $dec$ \\
\hline
$cab$ & $dab$ & $eab$ & $dac$ & $eac$ & $ead$ & $dbc$ & $ebc$ & $edb$ & $ecd$ \\
\hline
$cba$ & $dba$ & $eba$ & $dca$ & $eca$ & $eda$ & $dcb$ & $ebc$ & $edb$ & $edc$ \\
\hline
\end{tabular}
\end{center}

\smallskip

Chaque colonne correspond à \textbf{un seul et même choix non ordonné} (par exemple $\{a,b,c\}$), mais à 6 listes différentes selon l’ordre.  
Ainsi, pour obtenir le nombre de \textit{choix non ordonnés}, on divise le nombre total de listes ($60$) par le nombre d’ordres par groupe ($6$) :
\[
\binom{5}{3} = \frac{60}{6} = 10.
\]

\medskip

\noindent C’est exactement ce que fait la formule :
\[
\binom{n}{k} = \frac{\text{nombre de permutations de } k \text{ parmi } n}{k!} = \frac{n!}{k!(n-k)!}.
\]

\end{intuitionbox}

\begin{examplebox}[Utilisation du Coefficient Binomial]
    \textbf{Comité d'étudiants :} De combien de manières peut-on former un comité de 3 étudiants à partir d'une classe de 10 ? L'ordre ne compte pas.
    $$ \binom{10}{3} = \frac{10!}{3!(10-3)!} = \frac{10 \times 9 \times 8}{3 \times 2 \times 1} = 120 \text{ comités possibles.} $$
\end{examplebox}

\subsection{Identité de Vandermonde}

\begin{theorembox}[Identité de Vandermonde]
Cette identité offre une relation remarquable entre les coefficients binomiaux. Pour des entiers non négatifs $m, n$ et $k$, on a :
$$ \binom{m+n}{k} = \sum_{j=0}^{k} \binom{m}{j} \binom{n}{k-j} $$
\end{theorembox}

\begin{intuitionbox}
C'est le "principe du diviser pour régner". Imaginez que vous devez choisir un comité de $k$ personnes à partir d'un groupe contenant $m$ hommes et $n$ femmes.
Le côté gauche, $\binom{m+n}{k}$, compte directement le nombre total de comités possibles.
Le côté droit arrive au même résultat en additionnant toutes les compositions possibles du comité : choisir 0 homme et $k$ femmes, PLUS 1 homme et $k-1$ femmes, PLUS 2 hommes et $k-2$ femmes, etc., jusqu'à choisir $k$ hommes et 0 femme. La somme de toutes ces possibilités doit être égale au total.
\end{intuitionbox}

\begin{examplebox}[Application de l'Identité de Vandermonde]
On veut former un comité de 3 personnes ($k=3$) à partir d'un groupe de 5 hommes ($m=5$) et 4 femmes ($n=4$).
\vspace{0.3cm}
\noindent\textbf{Méthode directe (côté gauche) :} \\
On choisit 3 personnes parmi les $5+4=9$ au total.
$$ \binom{9}{3} = \frac{9 \times 8 \times 7}{3 \times 2 \times 1} = 84 $$
\vspace{0.3cm}
\noindent\textbf{Méthode par cas (côté droit) :} \\
La somme est $\binom{5}{0}\binom{4}{3} + \binom{5}{1}\binom{4}{2} + \binom{5}{2}\binom{4}{1} + \binom{5}{3}\binom{4}{0} = 84$. Les deux méthodes donnent bien le même résultat.
\end{examplebox}

\subsection{Bose-Einstein (Étoiles et Bâtons)}

\begin{theorembox}[Combinaisons avec répétition]
Le nombre de façons de distribuer $k$ objets indiscernables dans $n$ boîtes discernables (ou de choisir $k$ objets parmi $n$ avec remise, où l'ordre ne compte pas) est donné par la formule :
$$ \binom{n+k-1}{k} = \binom{n+k-1}{n-1} $$
\end{theorembox}

\begin{intuitionbox}[Étoiles et Bâtons]
Imaginez que les $k$ objets sont des étoiles ($\star$) et que nous avons besoin de $n-1$ bâtons ($|$) pour les séparer en $n$ groupes. Par exemple, pour distribuer $k=7$ étoiles dans $n=4$ boîtes, une configuration possible serait :
$$ \star\star\star \mid \star \mid \mid \star\star\star $$
Cela correspond à 3 objets dans la première boîte, 1 dans la deuxième, 0 dans la troisième et 3 dans la quatrième.
Le problème revient à trouver le nombre de façons d'arranger ces $k$ étoiles et $n-1$ bâtons. Nous avons un total de $n+k-1$ positions, et nous devons choisir les $k$ positions pour les étoiles (ou les $n-1$ positions pour les bâtons). Le nombre de manières de le faire est précisément $\binom{n+k-1}{k}$.
\end{intuitionbox}

\begin{examplebox}[Distribution de biens identiques]
De combien de manières peut-on distribuer 10 croissants identiques à 4 enfants ?
\newline
Ici, $k=10$ (les croissants, objets indiscernables) et $n=4$ (les enfants, boîtes discernables).
Le nombre de distributions possibles est :
$$ \binom{4+10-1}{10} = \binom{13}{10} = \binom{13}{3} = \frac{13 \times 12 \times 11}{3 \times 2 \times 1} = 13 \times 2 \times 11 = 286 $$
Il y a 286 façons de distribuer les croissants.
\end{examplebox}

\subsection{Principe d'Inclusion-Exclusion}

\begin{theorembox}[Principe d'Inclusion-Exclusion pour 3 ensembles]
Pour trois ensembles finis $A$, $B$ et $C$, le nombre d'éléments dans leur union est donné par :
$$ |A \cup B \cup C| = |A| + |B| + |C| - |A \cap B| - |A \cap C| - |B \cap C| + |A \cap B \cap C| $$
\end{theorembox}

\begin{intuitionbox}[Visualisation avec 3 ensembles]
Le principe d'inclusion-exclusion permet de compter le nombre d'éléments dans une union d'ensembles sans double-comptage. Pour comprendre intuitivement pourquoi on ajoute et soustrait alternativement, considérons trois ensembles $A$, $B$ et $C$ :

\begin{center}
\begin{tikzpicture}[set/.style = {draw,
    circle,
    minimum size = 6cm,
    fill=Rhodamine,
    opacity = 0.4,
    text opacity = 1}]
 
\node (A) [set] {$A$};
\node (B) at (60:4cm) [set] {$B$};
\node (C) at (0:4cm) [set] {$C$};
 
\node at (barycentric cs:A=1,B=1) [left] {$X$};
\node at (barycentric cs:A=1,C=1) [below] {$Y$};
\node at (barycentric cs:B=1,C=1) [right] {$Z$};
\node at (barycentric cs:A=1,B=1,C=1) [] {$T$};
 
\end{tikzpicture}
\end{center}

\textbf{Le problème :} Si on additionne simplement $|A| + |B| + |C|$, on compte certaines zones plusieurs fois :
\begin{itemize}
    \item Les intersections deux à deux ($X$, $Y$, $Z$) sont comptées \textbf{deux fois}
    \item L'intersection triple ($T$) est comptée \textbf{trois fois}
\end{itemize}

\textbf{La solution :} On corrige en soustrayant les intersections deux à deux, mais alors l'intersection triple est comptée :
\begin{itemize}
    \item $+3$ fois dans la somme initiale
    \item $-3$ fois dans la soustraction des intersections deux à deux (car elle appartient à chacune)
    \item Donc $0$ fois au total ! Il faut la rajouter.
\end{itemize}

D'où la formule : $|A \cup B \cup C| = |A| + |B| + |C| - |A \cap B| - |A \cap C| - |B \cap C| + |A \cap B \cap C|$
\end{intuitionbox}

\begin{theorembox}[Principe d'Inclusion-Exclusion généralisé]
Pour $n$ ensembles finis $A_1, A_2, \dots, A_n$, on a :
\begin{align*}
|A_1 \cup A_2 \cup \cdots \cup A_n| = & \sum_{i=1}^n |A_i| \\
& - \sum_{1 \leq i < j \leq n} |A_i \cap A_j| \\
& + \sum_{1 \leq i < j < k \leq n} |A_i \cap A_j \cap A_k| \\
& - \cdots \\
& + (-1)^{n+1} |A_1 \cap A_2 \cap \cdots \cap A_n|
\end{align*}
Ce qui s'écrit plus compactement :
$$ \left| \bigcup_{i=1}^n A_i \right| = \sum_{k=1}^n (-1)^{k+1} \sum_{1 \leq i_1 < i_2 < \cdots < i_k \leq n} |A_{i_1} \cap A_{i_2} \cap \cdots \cap A_{i_k}| $$
\end{theorembox}

\begin{intuitionbox}[Généralisation]
La logique reste la même que pour trois ensembles, mais l'argument clé est de prouver que chaque élément est compté \textbf{exactement une fois}, peu importe le nombre d'ensembles auxquels il appartient.

Supposons qu'un élément $x$ est membre d'exactement $k$ ensembles parmi les $n$ ensembles $A_1, \ldots, A_n$. Analysons combien de fois $x$ est compté dans la formule :
\begin{itemize}
    \item \textbf{Première somme ($\sum |A_i|$)} : $x$ est dans $k$ ensembles, donc il est ajouté $k$ fois. Le nombre de fois est $\binom{k}{1}$.
    
    \item \textbf{Deuxième somme ($-\sum |A_i \cap A_j|$)} : On soustrait $x$ pour chaque paire d'ensembles auxquels il appartient. Il y a $\binom{k}{2}$ telles paires.
    
    \item \textbf{Troisième somme ($+\sum |A_i \cap A_j \cap A_k|$)} : On ajoute de nouveau $x$ pour chaque triplet d'ensembles auxquels il appartient. Il y en a $\binom{k}{3}$.
    
    \item \textbf{Et ainsi de suite...}
\end{itemize}

Au total, l'élément $x$ est compté :
$$ \binom{k}{1} - \binom{k}{2} + \binom{k}{3} - \cdots + (-1)^{k-1}\binom{k}{k} \text{ fois.} $$

Pour voir que cette somme vaut exactement 1, rappelons une identité fondamentale issue du binôme de Newton :
$$ (1-1)^k = \sum_{j=0}^{k} (-1)^j \binom{k}{j} = \binom{k}{0} - \binom{k}{1} + \binom{k}{2} - \cdots + (-1)^k \binom{k}{k} = 0 $$

En réarrangeant cette équation, sachant que $\binom{k}{0}=1$ :
$$ \binom{k}{0} = \binom{k}{1} - \binom{k}{2} + \binom{k}{3} - \cdots - (-1)^{k}\binom{k}{k} $$
$$ 1 = \binom{k}{1} - \binom{k}{2} + \binom{k}{3} - \cdots + (-1)^{k-1}\binom{k}{k} $$

Cela prouve que n'importe quel élément, qu'il soit dans un seul ensemble ($k=1$) ou dans plusieurs ($k>1$), contribue précisément pour 1 au décompte final. Le principe d'inclusion-exclusion est donc une méthode infaillible pour corriger les comptages multiples de manière systématique.
\end{intuitionbox}


\begin{examplebox}[Application probabiliste]
On lance trois dés équilibrés. Quelle est la probabilité d'obtenir au moins un 6 ?

\vspace{0.3cm}
\noindent\textbf{Solution avec inclusion-exclusion :}

Soit $A$ = "le premier dé montre 6", $B$ = "le deuxième dé montre 6", $C$ = "le troisième dé montre 6".

On veut $P(A \cup B \cup C)$.

\begin{align*}
P(A \cup B \cup C) &= P(A) + P(B) + P(C) \\
&\quad - P(A \cap B) - P(A \cap C) - P(B \cap C) \\
&\quad + P(A \cap B \cap C) \\
&= \frac{1}{6} + \frac{1}{6} + \frac{1}{6} - \frac{1}{36} - \frac{1}{36} - \frac{1}{36} + \frac{1}{216} \\
&= \frac{3}{6} - \frac{3}{36} + \frac{1}{216} = \frac{1}{2} - \frac{1}{12} + \frac{1}{216} \\
&= \frac{108 - 18 + 1}{216} = \frac{91}{216} \approx 0.421
\end{align*}

\vspace{0.3cm}
\noindent\textbf{Vérification par la méthode complémentaire :}
La probabilité de n'obtenir aucun 6 est $\left(\frac{5}{6}\right)^3 = \frac{125}{216}$, donc la probabilité d'au moins un 6 est $1 - \frac{125}{216} = \frac{91}{216}$.
\end{examplebox}

\subsection{Exercices}

\begin{exercicebox}[Définition naïve de la probabilité]
On tire une carte d'un jeu de 52 cartes bien battu. Quelle est la probabilité de tirer une carte qui est soit un cœur, soit un Roi ?
\end{exercicebox}

\begin{correctionbox}
Soit $C$ l'événement "tirer un cœur" et $R$ l'événement "tirer un Roi".
Il y a 13 cœurs et 4 Rois. Cependant, le Roi de cœur est compté dans les deux ensembles.
On utilise le principe d'inclusion-exclusion pour les probabilités : $P(C \cup R) = P(C) + P(R) - P(C \cap R)$.
$P(C) = \frac{13}{52}$.
$P(R) = \frac{4}{52}$.
$P(C \cap R)$ (probabilité de tirer le Roi de cœur) = $\frac{1}{52}$.
$P(C \cup R) = \frac{13}{52} + \frac{4}{52} - \frac{1}{52} = \frac{16}{52} = \frac{4}{13}$.
\end{correctionbox}

\begin{exercicebox}[Permutations]
Cinq livres différents (A, B, C, D, E) doivent être rangés sur une étagère.
\begin{enumerate}
    \item Combien de rangements différents sont possibles ?
    \item Si les livres A et B doivent être côte à côte, combien de rangements sont possibles ?
\end{enumerate}
\end{exercicebox}

\begin{correctionbox}
1. Le nombre de façons de ranger 5 objets distincts est une permutation de 5, soit $5!$.
$5! = 5 \times 4 \times 3 \times 2 \times 1 = 120$. Il y a 120 rangements possibles.

2. On peut traiter les livres A et B comme un seul "bloc". Nous avons donc 4 "objets" à ranger : (AB), C, D, E. Il y a $4!$ façons de les ranger.
De plus, à l'intérieur du bloc (AB), les livres peuvent être dans l'ordre AB ou BA, soit $2!$ façons.
Le nombre total de rangements est donc $4! \times 2! = 24 \times 2 = 48$.
\end{correctionbox}

\begin{exercicebox}[Coefficient Binomial]
Une pizzeria propose 12 garnitures différentes. Un client souhaite commander une pizza avec exactement 3 garnitures. Combien de pizzas différentes peut-il composer ?
\end{exercicebox}

\begin{correctionbox}
L'ordre des garnitures ne compte pas, il s'agit donc de choisir 3 garnitures parmi 12. On utilise le coefficient binomial :
$$ \binom{12}{3} = \frac{12!}{3!(12-3)!} = \frac{12 \times 11 \times 10}{3 \times 2 \times 1} = 2 \times 11 \times 10 = 220 $$
Il peut composer 220 pizzas différentes.
\end{correctionbox}

\begin{exercicebox}[Principe d'Inclusion-Exclusion]
Dans une classe de 30 élèves, 15 étudient l'espagnol, 12 l'allemand et 5 étudient les deux langues. Combien d'élèves n'étudient aucune de ces deux langues ?
\end{exercicebox}

\begin{correctionbox}
Soit $E$ le nombre d'élèves étudiant l'espagnol et $A$ le nombre d'élèves étudiant l'allemand.
On cherche le nombre d'élèves qui étudient au moins une langue : $|E \cup A| = |E| + |A| - |E \cap A|$.
$|E \cup A| = 15 + 12 - 5 = 22$.
22 élèves étudient au moins une des deux langues.
Le nombre d'élèves qui n'en étudient aucune est le total moins ce nombre : $30 - 22 = 8$.
\end{correctionbox}

\begin{exercicebox}[Étoiles et Bâtons]
Un investisseur souhaite répartir 8 milliers d'euros (indiscernables) dans 4 fonds d'investissement différents. De combien de manières peut-il le faire ?
\end{exercicebox}

\begin{correctionbox}
C'est un problème de combinaisons avec répétition. On distribue $k=8$ objets (milliers d'euros) dans $n=4$ boîtes (fonds).
On utilise la formule "étoiles et bâtons" :
$$ \binom{n+k-1}{k} = \binom{4+8-1}{8} = \binom{11}{8} = \binom{11}{3} $$
$$ \binom{11}{3} = \frac{11 \times 10 \times 9}{3 \times 2 \times 1} = 11 \times 5 \times 3 = 165 $$
Il y a 165 manières de répartir l'investissement.
\end{correctionbox}

\begin{exercicebox}[Identité de Vandermonde]
Une équipe de 4 personnes doit être formée à partir d'un groupe de 6 physiciens et 5 chimistes. De combien de manières peut-on former l'équipe si elle doit contenir exactement 2 physiciens et 2 chimistes ? Vérifiez le résultat en utilisant l'identité de Vandermonde comme raisonnement.
\end{exercicebox}

\begin{correctionbox}
On doit choisir 2 physiciens parmi 6 ET 2 chimistes parmi 5. Le nombre de manières est le produit des combinaisons :
$$ \binom{6}{2} \binom{5}{2} = \left(\frac{6 \times 5}{2}\right) \times \left(\frac{5 \times 4}{2}\right) = 15 \times 10 = 150 $$
Ceci est un terme de la somme de l'identité de Vandermonde. Le nombre total de comités de 4 personnes parmi 11 ($m=6, n=5, k=4$) serait $\binom{11}{4}$. La somme $\sum_{j=0}^{4} \binom{6}{j} \binom{5}{4-j}$ décompose ce total selon le nombre de physiciens ($j$). Notre calcul correspond au cas $j=2$.
\end{correctionbox}

\begin{exercicebox}[Dénombrement et Probabilité]
On forme un mot de 3 lettres en utilisant les lettres A, B, C, D, E, sans répétition.
\begin{enumerate}
    \item Combien de mots peut-on former ?
    \item Si on choisit un de ces mots au hasard, quelle est la probabilité qu'il contienne la lettre A ?
\end{enumerate}
\end{exercicebox}

\begin{correctionbox}
1. On arrange 3 lettres parmi 5, l'ordre compte. C'est une permutation de 3 parmi 5 :
$P(5, 3) = \frac{5!}{(5-3)!} = 5 \times 4 \times 3 = 60$ mots.

2. Pour calculer la probabilité, on compte le nombre de mots favorables.
Un mot contenant A peut avoir A en 1ère, 2ème ou 3ème position.
Si A est en 1ère position, il reste 2 positions à remplir avec 4 lettres : $P(4, 2) = 4 \times 3 = 12$ mots.
C'est la même chose si A est en 2ème ou 3ème position.
Nombre de mots avec A = $3 \times 12 = 36$.
Probabilité = $\frac{\text{Favorables}}{\text{Total}} = \frac{36}{60} = \frac{3}{5}$.
\end{correctionbox}

\begin{exercicebox}[Principe d'Inclusion-Exclusion à 3 ensembles]
Sur 100 étudiants, 40 suivent le cours de maths, 30 celui de physique et 25 celui de chimie. 10 suivent maths et physique, 8 physique et chimie, 7 chimie et maths. Enfin, 3 suivent les trois cours. Combien d'étudiants ne suivent aucun de ces trois cours ?
\end{exercicebox}

\begin{correctionbox}
Soit M, P, C les ensembles d'étudiants. On cherche le nombre d'étudiants suivant au moins un cours, $|M \cup P \cup C|$ :
$|M \cup P \cup C| = |M| + |P| + |C| - (|M \cap P| + |P \cap C| + |C \cap M|) + |M \cap P \cap C|$
$|M \cup P \cup C| = 40 + 30 + 25 - (10 + 8 + 7) + 3 = 95 - 25 + 3 = 73$.
73 étudiants suivent au moins un cours.
Le nombre d'étudiants n'en suivant aucun est $100 - 73 = 27$.
\end{correctionbox}

\begin{exercicebox}[Coefficients binomiaux et chemins]
Sur une grille, combien de chemins mènent du point (0,0) au point (4,3) en se déplaçant uniquement vers la droite (D) ou vers le haut (H) ?
\end{exercicebox}

\begin{correctionbox}
Pour aller de (0,0) à (4,3), tout chemin doit être composé d'exactement 4 déplacements vers la droite (D) et 3 déplacements vers le haut (H). La longueur totale du chemin est de $4+3=7$ pas.
Le problème revient à trouver le nombre de séquences de 7 lettres contenant 4 'D' et 3 'H'.
C'est équivalent à choisir les 4 positions pour les 'D' parmi les 7 positions totales :
$$ \binom{7}{4} = \frac{7!}{4!3!} = \frac{7 \times 6 \times 5}{3 \times 2 \times 1} = 35 $$
Il y a 35 chemins possibles.
\end{correctionbox}

\begin{exercicebox}[Probabilité et Permutations circulaires]
Six personnes, dont Alice et Bob, s'assoient au hasard autour d'une table ronde. Quelle est la probabilité qu'Alice et Bob soient assis l'un à côté de l'autre ?
\end{exercicebox}

\begin{correctionbox}
Le nombre total d'arrangements de $n$ personnes autour d'une table ronde est $(n-1)!$. Ici, $(6-1)! = 5! = 120$ arrangements.
Pour les cas favorables, on traite Alice et Bob comme un seul bloc. On a donc 5 "entités" à placer, ce qui donne $(5-1)! = 4! = 24$ arrangements.
À l'intérieur du bloc, Alice et Bob peuvent être assis de 2 façons (Alice à gauche de Bob, ou Bob à gauche d'Alice).
Nombre de cas favorables = $24 \times 2 = 48$.
Probabilité = $\frac{\text{Favorables}}{\text{Total}} = \frac{48}{120} = \frac{2}{5}$.
\end{correctionbox}
\newpage

\section{Probabilité conditionnelle}

\begin{intuitionbox}[Question Fondamentale]
La probabilité conditionnelle est le concept qui répond à la question fondamentale : comment devons-nous mettre à jour nos croyances à la lumière des nouvelles informations que nous observons ?
\end{intuitionbox}

\subsection{Définition de la Probabilité Conditionnelle}

\begin{definitionbox}[Probabilité Conditionnelle]
Si $A$ et $B$ sont deux événements avec $P(B) > 0$, alors la probabilité conditionnelle de $A$ sachant $B$, notée $P(A|B)$, est définie comme :
$$P(A|B) = \frac{P(A \cap B)}{P(B)}$$
\end{definitionbox}

\begin{intuitionbox}
Imaginez que l'ensemble de tous les résultats possibles est un grand terrain. Savoir que l'événement $B$ s'est produit, c'est comme si on vous disait que le résultat se trouve dans une zone spécifique de ce terrain. La probabilité conditionnelle $P(A|B)$ ne s'intéresse plus au terrain entier, mais seulement à la proportion de la zone $B$ qui est également occupée par $A$. On "zoome" sur le monde où $B$ est vrai, et on recalcule les probabilités dans ce nouveau monde plus petit.
\end{intuitionbox}

\subsection{Règle du Produit (Intersection de deux événements)}

\begin{theorembox}[Probabilité de l'intersection de deux événements]
Pour tous événements $A$ et $B$ avec des probabilités positives, nous avons :
$$P(A \cap B) = P(A)P(B|A) = P(B)P(A|B)$$
Cela découle directement de la définition de la probabilité conditionnelle.
\end{theorembox}

\begin{intuitionbox}
Pour que deux événements se produisent, le premier doit se produire, PUIS le second doit se produire, sachant que le premier a eu lieu. Cette formule exprime mathématiquement cette idée séquentielle.
\end{intuitionbox}

\begin{examplebox}
Quelle est la probabilité de tirer deux As d'un jeu de 52 cartes sans remise ?
Soit $A$ l'événement "le premier tirage est un As", avec $P(A) = \frac{4}{52}$. Soit $B$ l'événement "le deuxième tirage est un As". Nous cherchons $P(A \cap B)$, que l'on calcule avec la formule $P(A \cap B) = P(A) \times P(B|A)$. La probabilité $P(B|A)$ correspond à tirer un As sachant que la première carte était un As. Il reste alors 51 cartes, dont 3 As. Donc, $P(B|A) = \frac{3}{51}$. Finalement, la probabilité de l'intersection est $P(A \cap B) = \frac{4}{52} \times \frac{3}{51} = \frac{12}{2652} \approx 0.0045$.
\end{examplebox}

\subsection{Règle de la Chaîne (Intersection de n événements)}

\begin{theorembox}[Probabilité de l'intersection de n événements]
Pour tous événements $A_1, \dots, A_n$ avec $P(A_1 \cap A_2 \cap \dots \cap A_{n-1}) > 0$, nous avons :
$$P(A_1 \cap \dots \cap A_n) = P(A_1)P(A_2|A_1)P(A_3|A_1 \cap A_2) \cdots P(A_n|A_1 \cap \dots \cap A_{n-1})$$
\end{theorembox}

\begin{intuitionbox}
Ceci est une généralisation de l'idée précédente, souvent appelée "règle de la chaîne" (chain rule). Pour qu'une séquence d'événements se produise, chaque événement doit se réaliser tour à tour, en tenant compte de tous les événements précédents qui se sont déjà produits.
\end{intuitionbox}

\begin{examplebox}
On tire 3 cartes sans remise. Quelle est la probabilité d'obtenir la séquence Roi, Dame, Valet ?
La probabilité de tirer un Roi en premier ($A_1$) est $P(A_1) = \frac{4}{52}$.
Ensuite, la probabilité de tirer une Dame ($A_2$) sachant qu'un Roi a été tiré est $P(A_2|A_1) = \frac{4}{51}$.
Enfin, la probabilité de tirer un Valet ($A_3$) sachant qu'un Roi et une Dame ont été tirés est $P(A_3|A_1 \cap A_2) = \frac{4}{50}$.
La probabilité totale de la séquence est donc le produit de ces probabilités : $P(A_1 \cap A_2 \cap A_3) = \frac{4}{52} \times \frac{4}{51} \times \frac{4}{50} \approx 0.00048$.
\end{examplebox}

\subsection{Règle de Bayes}

\begin{theorembox}[Règle de Bayes]
$$P(A|B) = \frac{P(B|A)P(A)}{P(B)}$$
\end{theorembox}

\begin{intuitionbox}
La règle de Bayes est la formule pour "inverser" une probabilité conditionnelle. Souvent, il est facile de connaître la probabilité d'un effet étant donné une cause ($P(\text{symptôme}|\text{maladie})$), mais ce qui nous intéresse vraiment, c'est la probabilité de la cause étant donné l'effet observé ($P(\text{maladie}|\text{symptôme})$). La règle de Bayes nous permet de faire ce retournement en utilisant notre connaissance initiale de la probabilité de la cause ($P(\text{maladie})$). C'est le fondement mathématique de la mise à jour de nos croyances.
\end{intuitionbox}

\begin{examplebox}[Dépistage médical]
Une maladie touche 1\% de la population ($P(M) = 0.01$). Un test de dépistage est fiable à 95\% : il est positif pour 95\% des malades ($P(T|M)=0.95$) et négatif pour 95\% des non-malades, ce qui implique un taux de faux positifs de $P(T|\neg M) = 0.05$.
Une personne est testée positive. Quelle est la probabilité qu'elle soit réellement malade, $P(M|T)$ ?
On cherche $P(M|T) = \frac{P(T|M)P(M)}{P(T)}$.
D'abord, on calcule $P(T)$ avec la formule des probabilités totales :
$P(T) = P(T|M)P(M) + P(T|\neg M)P(\neg M) = (0.95 \times 0.01) + (0.05 \times 0.99) = 0.0095 + 0.0495 = 0.059$.
Ensuite, on applique la règle de Bayes : $P(M|T) = \frac{0.95 \times 0.01}{0.059} \approx 0.161$.
Malgré un test positif, il n'y a que 16.1\% de chance que la personne soit malade.
\end{examplebox}

\subsection{Formule des Probabilités Totales}

\begin{theorembox}[Formule des probabilités totales]
Soit $A_1, \dots, A_n$ une partition de l'espace échantillon $S$ (c'est-à-dire que les $A_i$ sont des événements disjoints et leur union est $S$), avec $P(A_i) > 0$ pour tout $i$. Alors pour tout événement $B$ :
$$P(B) = \sum_{i=1}^{n} P(B|A_i)P(A_i)$$
\end{theorembox}

\begin{intuitionbox}
C'est une stratégie de "diviser pour régner". Pour calculer la probabilité totale d'un événement $B$, on peut décomposer le monde en plusieurs scénarios mutuellement exclusifs (la partition $A_i$). On calcule ensuite la probabilité de $B$ dans chacun de ces scénarios ($P(B|A_i)$), on pondère chaque résultat par la probabilité du scénario en question ($P(A_i)$), et on additionne le tout.

\begin{center}
\begin{tikzpicture}
% 1. Dessiner le grand rectangle et les lignes verticales de partition
\draw (0,0) rectangle (12,7);

% 3. Dessiner une grande ellipse pour la forme B
\filldraw[
    fill=gray!30, % Remplissage gris clair
    thick % Trait épais pour le contour
] (6, 3.5) ellipse (5.5cm and 2.5cm); % Centre (6,3.5), rayon x=5.5cm, rayon y=2.5cm

\foreach \x in {2,4,6,8,10} {
    \draw (\x,0) -- (\x,7);
}

% 2. Placer les étiquettes A_1, A_2, ... en bas
\foreach \i [evaluate=\i as \xpos using \i*2-1] in {1,...,6} {
    \node at (\xpos, -0.5) {$A_{\i}$};
}

% 4. Placer l'étiquette pour l'ensemble B
\node at (11, 6) {$B$}; % Ajusté pour être au-dessus de l'ellipse

% 5. Placer les étiquettes pour les intersections B ∩ A_i, toutes au même niveau
\node at (1.2, 3.5) {$B \cap A_1$};
\node at (3, 3.5) {$B \cap A_2$};
\node at (5, 3.5) {$B \cap A_3$};
\node at (7, 3.5) {$B \cap A_4$};
\node at (9, 3.5) {$B \cap A_5$};
\node at (10.8, 3.5) {$B \cap A_6$};
\end{tikzpicture}
\end{center}
\end{intuitionbox}

\begin{examplebox}
Une usine possède trois machines, M1, M2, et M3, qui produisent respectivement 50\%, 30\% et 20\% des articles. Leurs taux de production défectueuse sont de 4\%, 2\% et 5\%. Quelle est la probabilité qu'un article choisi au hasard soit défectueux ?
Soit $D$ l'événement "l'article est défectueux". Les machines forment une partition avec $P(M1)=0.5$, $P(M2)=0.3$, et $P(M3)=0.2$. Les probabilités conditionnelles de défaut sont $P(D|M1)=0.04$, $P(D|M2)=0.02$, et $P(D|M3)=0.05$.
En appliquant la formule, on obtient :
$P(D) = P(D|M1)P(M1) + P(D|M2)P(M2) + P(D|M3)P(M3) = (0.04 \times 0.5) + (0.02 \times 0.3) + (0.05 \times 0.2) = 0.02 + 0.006 + 0.01 = 0.036$.
La probabilité qu'un article soit défectueux est de 3.6\%.
\end{examplebox}

\begin{proofbox}[Démonstration de la formule des probabilités totales]
Puisque les $A_i$ forment une partition de $S$, on peut décomposer $B$ comme :
$$B = (B \cap A_1) \cup (B \cap A_2) \cup \cdots \cup (B \cap A_n)$$
Comme les $A_i$ sont disjoints, les événements $(B \cap A_i)$ le sont aussi. On peut donc sommer leurs probabilités :
$$P(B) = P(B \cap A_1) + P(B \cap A_2) + \cdots + P(B \cap A_n)$$
En appliquant le théorème de l'intersection des probabilités à chaque terme, on obtient :
$$P(B) = P(B|A_1)P(A_1) + P(B|A_2)P(A_2) + \cdots + P(B|A_n) = \sum_{i=1}^{n} P(B|A_i)P(A_i)$$
\end{proofbox}

\subsection{Règle de Bayes avec Conditionnement Additionnel}

\begin{theorembox}[Règle de Bayes avec conditionnement additionnel]
À condition que $P(A \cap E) > 0$ et $P(B \cap E) > 0$, nous avons :
$$P(A|B, E) = \frac{P(B|A, E)P(A|E)}{P(B|E)}$$
\end{theorembox}

\begin{intuitionbox}
Cette formule est simplement la règle de Bayes standard, mais appliquée à l'intérieur d'un univers que l'on a déjà "rétréci".

Imaginez que vous recevez une information \textbf{E} qui élimine une grande partie des possibilités. C'est votre nouveau point de départ, votre monde est plus petit. Toutes les probabilités que vous calculez désormais sont relatives à ce monde restreint.

Dans ce nouveau monde, vous recevez une autre information, l'évidence \textbf{B}. La règle de Bayes conditionnelle vous permet alors de mettre à jour votre croyance sur un événement \textbf{A}, en utilisant exactement la même logique que la règle de Bayes classique, mais en vous assurant que chaque calcul reste confiné à l'intérieur des frontières de l'univers défini par \textbf{E}.
\end{intuitionbox}

\subsection{Formule des Probabilités Totales avec Conditionnement Additionnel}

\begin{theorembox}[Formule des probabilités totales avec conditionnement additionnel]
Soit $A_1, \dots, A_n$ une partition de $S$. À condition que $P(A_i \cap E) > 0$ pour tout $i$, nous avons :
$$P(B|E) = \sum_{i=1}^{n} P(B|A_i, E)P(A_i|E)$$
\end{theorembox}

\begin{intuitionbox}
\begin{center}
\begin{tikzpicture}
  % Matrice principale, nommée "m"
  \matrix (m) [
    matrix of nodes,
    row sep = -\pgflinewidth,
    column sep = -\pgflinewidth,
    nodes={
      rectangle, draw=black, anchor=center,
      text height=4ex, text depth=0.5ex, minimum width=4em, fill=intuitionColor!10
    }
  ]
  {
    | |              & | |              & |[red_hatch]|    & | |              & | |              & | |            \\
    |[red_hatch]|    & |[purple_hatch]| & |[purple_hatch]| & | |              & |[red_hatch]|    & |[red_hatch]|  \\
    |[red_hatch]|    & |[blue_hatch]|   & |[red_hatch]|    & |[red_hatch]|    & |[red_hatch]|    & | |            \\
  };

  % --- DÉLIMITATION DES COLONNES AVEC ACCOLADES ---
  \draw [decorate, decoration={brace, amplitude=5pt, raise=4mm}]
    (m-1-1.north west) -- (m-1-2.north east) 
    node [midway, yshift=8mm, font=\bfseries] {A1};
    
  \draw [decorate, decoration={brace, amplitude=5pt, raise=4mm}]
    (m-1-3.north west) -- (m-1-4.north east) 
    node [midway, yshift=8mm, font=\bfseries] {A2};
    
  \draw [decorate, decoration={brace, amplitude=5pt, raise=4mm}]
    (m-1-5.north west) -- (m-1-6.north east) 
    node [midway, yshift=8mm, font=\bfseries] {A3};
\end{tikzpicture}
\end{center}
Imaginez que le graphique ci-dessus représente la carte d'un trésor. La carte est partitionnée en trois grandes régions : \textbf{A1}, \textbf{A2}, et \textbf{A3}. Sur cette carte, on a identifié deux types de terrains : une \textbf{zone marécageuse} (événement E, hachures rouges) qui s'étend sur \textbf{10 parcelles}, et une \textbf{zone près d'un vieux chêne} (événement B, hachures bleues) qui couvre \textbf{3 parcelles}.

On vous donne un premier indice : "Le trésor est dans la zone marécageuse (E)". Votre univers de recherche se réduit instantanément à ces 10 parcelles rouges. Puis, on vous donne un second indice : "Le trésor est aussi près d'un chêne (B)". Votre recherche se concentre alors sur les parcelles qui sont à la fois marécageuses et proches d'un chêne (les cases violettes, $B \cap E$).

La question est : "Sachant que le trésor est dans une parcelle violette, quelle est la probabilité qu'il se trouve dans la région A2 ?". On cherche donc $P(A_2 | B, E)$. La règle de Bayes nous permet de le calculer.

\textbf{Calcul des termes nécessaires :} D'abord, nous devons évaluer les probabilités à l'intérieur du "monde marécageux" (sachant E).

La \textbf{vraisemblance} est $P(B|A_2, E)$. En se limitant aux 4 parcelles marécageuses de la région A2, une seule est aussi près d'un chêne. Donc, $P(B|A_2, E) = 1/4$.

La \textbf{probabilité a priori} est $P(A_2|E)$. Sur les 10 parcelles marécageuses, 4 sont dans la région A2. Donc, $P(A_2|E) = 4/10$.

L'\textbf{évidence}, $P(B|E)$, est la probabilité de trouver un chêne dans l'ensemble de la zone marécageuse. On peut la calculer avec la formule des probabilités totales :
$$P(B|E) = P(B|A_1, E)P(A_1|E) + P(B|A_2, E)P(A_2|E) + P(B|A_3, E)P(A_3|E)$$
$$P(B|E) = (\frac{1}{3} \times \frac{3}{10}) + (\frac{1}{4} \times \frac{4}{10}) + (0 \times \frac{3}{10}) = \frac{1}{10} + \frac{1}{10} = \frac{2}{10}$$

\textbf{Application de la règle de Bayes :} Maintenant, nous assemblons le tout.
$$P(A_2|B, E) = \frac{P(B|A_2, E)P(A_2|E)}{P(B|E)} = \frac{(1/4) \times (4/10)}{2/10} = \frac{1/10}{2/10} = \frac{1}{2}$$
L'intuition confirme le calcul : sachant que le trésor est sur une parcelle violette, et qu'il n'y en a que deux (une en A1, une en A2), il y a bien une chance sur deux qu'il se trouve dans la région A2.
\end{intuitionbox}

\subsection{Indépendance de Deux Événements}

\begin{definitionbox}[Indépendance de deux événements]
Les événements $A$ et $B$ sont indépendants si :
$$P(A \cap B) = P(A)P(B)$$
Si $P(A) > 0$ et $P(B) > 0$, cela est équivalent à :
$$P(A|B) = P(A)$$
\end{definitionbox}

\begin{intuitionbox}
L'indépendance est l'absence d'information. Si deux événements sont indépendants, apprendre que l'un s'est produit ne change absolument rien à la probabilité de l'autre. Savoir qu'il pleut à Tokyo ($B$) ne modifie pas la probabilité que vous obteniez pile en lançant une pièce ($A$).
\end{intuitionbox}

\subsection{Indépendance Conditionnelle}

\begin{definitionbox}[Indépendance Conditionnelle]
Les événements $A$ et $B$ sont dits conditionnellement indépendants étant donné $E$ si :
$$P(A \cap B | E) = P(A|E)P(B|E)$$
\end{definitionbox}

\begin{intuitionbox}
L'indépendance peut apparaître ou disparaître quand on observe un autre événement. Par exemple, vos notes en maths ($A$) et en physique ($B$) ne sont probablement pas indépendantes. Mais si l'on sait que vous avez beaucoup travaillé ($E$), alors vos notes en maths et en physique pourraient devenir indépendantes. L'information "vous avez beaucoup travaillé" explique la corrélation ; une fois qu'on la connaît, connaître votre note en maths n'apporte plus d'information sur votre note en physique.
\end{intuitionbox}

\subsection{Le Problème de Monty Hall}

\begin{remarquebox}[Le problème de Monty Hall]
Imaginez que vous êtes à un jeu télévisé. Face à vous se trouvent trois portes fermées. Derrière l'une d'elles se trouve une voiture, et derrière les deux autres, des chèvres.
\begin{enumerate}
    \item Vous choisissez une porte (disons, la porte n°1).
    \item L'animateur, qui sait où se trouve la voiture, ouvre une autre porte (par exemple, la n°3) derrière laquelle se trouve une chèvre.
    \item Il vous demande alors : "Voulez-vous conserver votre choix initial (porte n°1) ou changer pour l'autre porte restante (la n°2) ?"
\end{enumerate}
\textbf{Question :} Avez-vous intérêt à changer de porte ? Votre probabilité de gagner la voiture est-elle plus grande si vous changez, si vous ne changez pas, ou est-elle la même dans les deux cas ?
\end{remarquebox}

\begin{correctionbox}[Solution du problème de Monty Hall]
La réponse est sans équivoque : il faut \textbf{toujours changer de porte}. Cette stratégie fait passer la probabilité de gagner de $1/3$ à $2/3$. L'intuition et la preuve ci-dessous détaillent ce résultat surprenant.
\end{correctionbox}

\begin{intuitionbox}[Le secret : l'information de l'animateur]
L'erreur commune est de supposer qu'il reste deux portes avec une chance égale de $1/2$. Cela ignore une information capitale : le choix de l'animateur n'est \textbf{pas aléatoire}. Il sait où se trouve la voiture et ouvrira toujours une porte perdante.

Le raisonnement correct se déroule en deux temps. D'abord, votre choix initial a $\mathbf{1/3}$ de chance d'être correct. Cela implique qu'il y a $\mathbf{2/3}$ de chance que la voiture soit derrière l'une des \textit{deux autres portes}. Ensuite, lorsque l'animateur ouvre l'une de ces deux portes, il ne fait que vous montrer où la voiture n'est \textit{pas} dans cet ensemble. La probabilité de $2/3$ se \textbf{concentre} alors entièrement sur la seule porte qu'il a laissée fermée. Changer de porte revient à miser sur cette probabilité de $2/3$.
\end{intuitionbox}

\begin{proofbox}[Preuve par l'arbre de décision]
L'analyse de la meilleure stratégie peut être visualisée à l'aide de l'arbre de décision ci-dessous. Il décompose le problème en deux scénarios initiaux : avoir choisi la bonne porte (probabilité $1/3$) ou une mauvaise porte (probabilité $2/3$).

\vspace{0.5cm}
\begin{center}
\begin{tikzpicture}[
  grow=right,
  level distance=4.5cm,
  level 1/.style={sibling distance=3cm},
  level 2/.style={sibling distance=2.5cm},
  edge from parent/.style={draw, -latex},
  % --- Définition des styles pour les cadres ---
  porte_style/.style={rectangle, rounded corners, draw=black, fill=gray!20, thick, inner sep=4pt, text width=2.5cm, align=center},
  gain_style/.style={rectangle, rounded corners, draw=green!60!black, fill=green!20, thick, inner sep=4pt},
  perte_style/.style={rectangle, rounded corners, draw=red!60!black, fill=red!20, thick, inner sep=4pt}
]

\node {S}
    % --- Branche du haut ---
    child {
        node[porte_style] {Bonne porte}
        child {
            node[gain_style] {Gain}
            edge from parent
            node[above, sloped] {$1/2$}
        }
        child {
            node[perte_style] {Perte}
            edge from parent
            node[below, sloped] {$1/2$}
        }
        edge from parent
        node[above, sloped] {1/3}
    }
    % --- Branche du bas ---
    child {
        node[porte_style] {Mauvaise porte}
        child {
            node[gain_style] {Gain}
            edge from parent
            node[above, sloped] {1}
        }
        edge from parent
        node[below, sloped] {2/3}
    };
\end{tikzpicture}
\end{center}
\vspace{0.5cm}

\noindent\textbf{Analyse de l'arbre :}

\vspace{0.3cm}
\noindent\textbf{Branche du bas (cas le plus probable) :}
\newline
Avec une probabilité de $\mathbf{2/3}$, votre choix initial se porte sur une "Mauvaise porte". L'animateur est alors obligé de révéler l'autre porte perdante. La seule porte restante est donc la bonne. L'arbre montre que cela mène à un "Gain" avec une probabilité de $\mathbf{1}$. Ce chemin correspond au résultat de la stratégie \textbf{"Changer"}.

\vspace{0.3cm}
\noindent\textbf{Branche du haut (cas le moins probable) :}
\newline
Avec une probabilité de $\mathbf{1/3}$, vous avez choisi la "Bonne porte" du premier coup. L'arbre se divise alors en deux issues équiprobables ($1/2$ chacune). L'issue "Gain" correspond à la stratégie \textbf{"Garder"} votre choix initial, tandis que l'issue "Perte" correspond à la stratégie \textbf{"Changer"} pour la porte perdante restante.

\vspace{0.3cm}
\noindent\textbf{Conclusion :}
\newline
Pour évaluer la meilleure stratégie, il suffit de sommer les probabilités de gain. La \textbf{probabilité de gain en changeant} est de $\mathbf{2/3}$, car vous gagnez uniquement si votre choix initial était mauvais (branche du bas). La \textbf{probabilité de gain en gardant} est de $\mathbf{1/3}$, car vous gagnez uniquement si votre choix initial était bon (branche "Gain" du haut). La stratégie optimale est donc bien de toujours changer de porte.
\end{proofbox}
\newpage
\section{Variables Aléatoires Discrètes}

\subsection{Variable Aléatoire}

\begin{definitionbox}[Variable Aléatoire]
Étant donné une expérience avec un univers $S$, une variable aléatoire est une fonction de l'univers $S$ vers les nombres réels $\mathbb{R}$.
\end{definitionbox}

\begin{intuitionbox}
Une variable aléatoire est une manière de traduire les résultats d'une expérience en nombres. Au lieu de travailler avec des concepts comme "Pile" ou "Face", on leur assigne des valeurs numériques (par exemple, 1 pour Pile, 0 pour Face). Cela nous permet d'utiliser toute la puissance des outils mathématiques (fonctions, calculs, etc.) pour analyser le hasard. C'est un pont entre le monde concret des événements et le monde abstrait des nombres.
\end{intuitionbox}

\begin{examplebox}
On lance deux dés. L'univers $S$ est l'ensemble des 36 paires de résultats, comme $(1,1), (1,2), \dots, (6,6)$. On peut définir une variable aléatoire $X$ comme étant la \textbf{somme des deux dés}.
Pour le résultat $(2, 5)$, la valeur de la variable aléatoire est $X(2, 5) = 2 + 5 = 7$.
\end{examplebox}

\subsection{Variable Aléatoire Discrète}

\begin{definitionbox}[Variable Aléatoire Discrète]
Une variable aléatoire $X$ est dite discrète s'il existe une liste finie ou infinie dénombrable de valeurs $a_1, a_2, \dots$ telle que $P(X=a_j \text{ pour un certain } j) = 1$.
\end{definitionbox}

\begin{intuitionbox}
Une variable aléatoire est "discrète" si on peut lister (compter) toutes les valeurs qu'elle peut prendre, même si cette liste est infinie. Pensez aux "sauts" d'une valeur à l'autre, sans possibilité de prendre une valeur intermédiaire. C'est comme monter un escalier : on peut être sur la marche 1, 2 ou 3, mais jamais sur la marche 2.5. Le nombre de têtes en 10 lancers, le résultat d'un dé, le nombre d'emails que vous recevez en une heure sont des exemples. À l'opposé, une variable continue pourrait être la taille exacte d'une personne, qui peut prendre n'importe quelle valeur dans un intervalle.
\end{intuitionbox}

\subsection{Fonction de Masse (PMF)}

\begin{definitionbox}[Probability Mass Function (PMF)]
La fonction de masse (PMF) d'une variable aléatoire discrète $X$ est la fonction $P_X$ donnée par $P_X(x) = P(X=x)$.
\end{definitionbox}

\begin{intuitionbox}
La PMF est la "carte d'identité" probabiliste d'une variable aléatoire discrète. Pour chaque valeur que la variable peut prendre, la PMF nous donne la probabilité exacte associée à cette valeur. C'est comme si chaque résultat possible avait une "étiquette de prix" qui indique sa chance de se produire. La somme de toutes ces probabilités doit bien sûr valoir 1.
\end{intuitionbox}

\begin{examplebox}
Soit $X$ le résultat d'un lancer de dé équilibré. La variable $X$ peut prendre les valeurs $\{1, 2, 3, 4, 5, 6\}$.
La PMF de $X$ est la fonction qui assigne $1/6$ à chaque valeur :
$P(X=1) = 1/6$, $P(X=2) = 1/6$, ..., $P(X=6) = 1/6$.
Pour toute autre valeur $x$ (par exemple $x=2.5$ ou $x=7$), $P(X=x) = 0$.
\end{examplebox}

\subsection{Distribution de Bernoulli}

\begin{definitionbox}[Distribution de Bernoulli]
Une variable aléatoire $X$ suit la distribution de Bernoulli avec paramètre $p$ si $P(X=1) = p$ et $P(X=0) = 1-p$, où $0 < p < 1$. On note cela $X \sim \text{Bern}(p)$.
\end{definitionbox}

\begin{intuitionbox}
La distribution de Bernoulli est le modèle le plus simple pour une expérience aléatoire avec seulement deux issues : "succès" (codé par 1) et "échec" (codé par 0). C'est la brique de base de nombreuses autres distributions. Pensez à un unique lancer de pièce (Pile/Face), un unique tir au but (Marqué/Manqué), ou la réponse à une question par oui/non. Le paramètre $p$ est simplement la probabilité du "succès".
\end{intuitionbox}

\subsection{Variable Aléatoire Indicatrice}

\begin{definitionbox}[Variable Aléatoire Indicatrice]
La variable aléatoire indicatrice d'un événement $A$ est la variable aléatoire qui vaut 1 si $A$ se produit et 0 sinon. Nous la noterons $I_A$. Notez que $I_A \sim \text{Bern}(p)$ avec $p=P(A)$.
\end{definitionbox}

\begin{intuitionbox}
Une variable indicatrice est un interrupteur. Elle est sur "ON" (valeur 1) si un événement qui nous intéresse se produit, et sur "OFF" (valeur 0) sinon. C'est un outil extrêmement puissant car il transforme les questions sur les probabilités des événements en questions sur les espérances des variables aléatoires, ce qui simplifie souvent les calculs.
\end{intuitionbox}

\subsection{Distribution Binomiale}

\begin{theorembox}[PMF Binomiale]
Si $X \sim \text{Bin}(n, p)$, alors la PMF de $X$ est :
$$ P(X=k) = \binom{n}{k} p^k (1-p)^{n-k} $$
pour $k = 0, 1, \dots, n$.
\end{theorembox}

\begin{intuitionbox}
La distribution binomiale répond à la question : "Si je répète $n$ fois la même expérience de Bernoulli (qui a une probabilité de succès $p$), quelle est la probabilité d'obtenir exactement $k$ succès ?"
La formule est construite logiquement en multipliant trois composantes. D'abord, $\mathbf{p^k}$ représente la probabilité d'obtenir $k$ succès. Ensuite, $\mathbf{(1-p)^{n-k}}$ est la probabilité que les $n-k$ échecs restants se produisent. Finalement, comme les $k$ succès peuvent apparaître n'importe où parmi les $n$ essais, on multiplie par $\mathbf{\binom{n}{k}}$, qui compte le nombre de manières distinctes de placer ces succès.
\end{intuitionbox}

\begin{examplebox}
On lance une pièce équilibrée 10 fois ($n=10$, $p=0.5$). Quelle est la probabilité d'obtenir exactement 6 Piles ($k=6$) ?
$$ P(X=6) = \binom{10}{6} (0.5)^6 (1-0.5)^{10-6} = \frac{10!}{6!4!} (0.5)^{10} = 210 \times (0.5)^{10} \approx 0.205 $$
Il y a environ 20.5\% de chance d'obtenir exactement 6 Piles.
\end{examplebox}

\subsection{Distribution Hypergéométrique}

\begin{theorembox}[PMF Hypergéométrique]
Si $X \sim \text{HG}(w, b, m)$, alors la PMF de $X$ est :
$$ P(X=k) = \frac{\binom{w}{k} \binom{b}{m-k}}{\binom{w+b}{m}} $$
\end{theorembox}

\begin{intuitionbox}
La distribution hypergéométrique est la "cousine" de la binomiale pour les tirages \textbf{sans remise}. Imaginez une urne avec des boules de deux couleurs (par exemple, $w$ blanches et $b$ noires). Vous tirez $m$ boules d'un coup. Quelle est la probabilité que vous ayez exactement $k$ boules blanches ?
La formule est un simple ratio issu du dénombrement. Le \textbf{dénominateur}, $\binom{w+b}{m}$, compte le nombre total de façons de tirer $m$ boules parmi toutes celles disponibles. Le \textbf{numérateur} compte les issues favorables : c'est le produit du nombre de façons de choisir $k$ blanches parmi les $w$ ($\binom{w}{k}$) ET de choisir les $m-k$ boules restantes parmi les noires ($\binom{b}{m-k}$). La différence clé avec la loi binomiale est que les tirages ne sont pas indépendants.
\end{intuitionbox}

\begin{examplebox}
Un comité de 5 personnes est choisi au hasard parmi un groupe de 8 hommes et 10 femmes. Quelle est la probabilité que le comité soit composé de 2 hommes et 3 femmes ?
Ici, on tire 5 personnes ($m=5$) d'une population de 18 personnes. On s'intéresse au nombre d'hommes ($k=2$) parmi les 8 disponibles ($w=8$). Le reste du comité sera composé de femmes ($b=10$).
$$ P(X=2) = \frac{\binom{8}{2} \binom{10}{3}}{\binom{18}{5}} = \frac{28 \times 120}{8568} \approx 0.392 $$
Il y a environ 39.2\% de chance que le comité ait exactement cette composition.
\end{examplebox}

\subsection{Fonction de Répartition (CDF)}

\begin{definitionbox}[Cumulative Distribution Function (CDF)]
La fonction de répartition (CDF) d'une variable aléatoire $X$ est la fonction $F_X$ donnée par $F_X(x) = P(X \le x)$.
\end{definitionbox}

\begin{intuitionbox}
Alors que la PMF répond à la question "Quelle est la probabilité d'obtenir \textit{exactement} $x$ ?", la CDF répond à la question "Quelle est la probabilité d'obtenir \textit{au plus} $x$ ?". C'est une fonction cumulative : pour une valeur $x$ donnée, elle additionne les probabilités de tous les résultats inférieurs ou égaux à $x$.
La CDF a toujours une forme d'escalier pour les variables discrètes. Elle commence à 0 (très loin à gauche) et monte par "sauts" à chaque valeur possible de la variable, pour finalement atteindre 1 (très loin à droite). La hauteur de chaque saut correspond à la valeur de la PMF à ce point.
\end{intuitionbox}

\begin{examplebox}
Reprenons le lancer d'un dé équilibré ($X$). Calculons quelques valeurs de la CDF, notée $F(x)$.
\newline
$F(0.5) = P(X \le 0.5) = 0$
\newline
$F(1) = P(X \le 1) = P(X=1) = 1/6$
\newline
$F(1.5) = P(X \le 1.5) = P(X=1) = 1/6$
\newline
$F(2) = P(X \le 2) = P(X=1) + P(X=2) = 2/6$
\newline
$F(5.9) = P(X \le 5.9) = P(X=1) + \dots + P(X=5) = 5/6$
\newline
$F(6) = P(X \le 6) = 1$
\newline
$F(100) = P(X \le 100) = 1$
\end{examplebox}

\subsection{Exercices}

\begin{exercicebox}[PMF d'un dé spécial]
Un dé à 4 faces (tétraèdre) est truqué. La probabilité d'obtenir un certain nombre est proportionnelle à ce nombre. Soit $X$ la variable aléatoire du résultat d'un lancer.
\begin{enumerate}
    \item Déterminez la fonction de masse (PMF) de $X$.
    \item Calculez $P(X \ge 3)$.
\end{enumerate}
\end{exercicebox}

\begin{correctionbox}
1. Les résultats possibles sont $\{1, 2, 3, 4\}$. La probabilité est proportionnelle au résultat, donc $P(X=k) = c \cdot k$ pour une constante $c$.
La somme des probabilités doit valoir 1 :
$$ \sum_{k=1}^{4} P(X=k) = c \cdot 1 + c \cdot 2 + c \cdot 3 + c \cdot 4 = 10c = 1 \implies c = \frac{1}{10} $$
La PMF est donc : $P(X=1)=1/10$, $P(X=2)=2/10$, $P(X=3)=3/10$, $P(X=4)=4/10$.

2. On calcule $P(X \ge 3) = P(X=3) + P(X=4) = \frac{3}{10} + \frac{4}{10} = \frac{7}{10}$.
\end{correctionbox}

\begin{exercicebox}[Loi Binomiale : Tirs au but]
Un footballeur a une probabilité de $0.8$ de marquer un penalty. Il tire 5 penaltys. Soit $X$ le nombre de penaltys marqués.
\begin{enumerate}
    \item Quelle est la distribution de $X$ ?
    \item Quelle est la probabilité qu'il marque exactement 4 penaltys ?
\end{enumerate}
\end{exercicebox}

\begin{correctionbox}
1. Les tirs sont des épreuves de Bernoulli indépendantes et répétées avec la même probabilité de succès. $X$ suit donc une loi binomiale : $X \sim \text{Bin}(n=5, p=0.8)$.

2. On cherche $P(X=4)$. On applique la formule de la PMF binomiale :
$$ P(X=4) = \binom{5}{4} (0.8)^4 (1-0.8)^{5-4} = 5 \times (0.8)^4 \times (0.2)^1 = 5 \times 0.4096 \times 0.2 = 0.4096 $$
La probabilité est de 40.96\%.
\end{correctionbox}

\begin{exercicebox}[Loi Hypergéométrique : Contrôle qualité]
Une boîte contient 20 ampoules, dont 5 sont défectueuses. On prélève 4 ampoules au hasard sans remise pour les tester. Soit $X$ le nombre d'ampoules défectueuses dans l'échantillon.
\begin{enumerate}
    \item Quelle est la distribution de $X$ ?
    \item Quelle est la probabilité de ne trouver aucune ampoule défectueuse ?
\end{enumerate}
\end{exercicebox}

\begin{correctionbox}
1. Il s'agit d'un tirage sans remise d'une population finie contenant deux types d'objets. $X$ suit donc une loi hypergéométrique : $X \sim \text{HG}(w=5, b=15, m=4)$.

2. On cherche $P(X=0)$. On applique la formule de la PMF hypergéométrique :
$$ P(X=0) = \frac{\binom{5}{0} \binom{15}{4}}{\binom{20}{4}} = \frac{1 \times \frac{15 \times 14 \times 13 \times 12}{4 \times 3 \times 2 \times 1}}{\frac{20 \times 19 \times 18 \times 17}{4 \times 3 \times 2 \times 1}} = \frac{1365}{4845} \approx 0.2817 $$
La probabilité est d'environ 28.17\%.
\end{correctionbox}

\begin{exercicebox}[CDF]
En utilisant la PMF du dé truqué de l'exercice 1, déterminez et tracez la fonction de répartition (CDF) de $X$.
\end{exercicebox}

\begin{correctionbox}
La PMF était $P(X=1)=0.1$, $P(X=2)=0.2$, $P(X=3)=0.3$, $P(X=4)=0.4$.
La CDF, $F(x)=P(X \le x)$, se calcule par accumulation :
\begin{itemize}
    \item Pour $x < 1$, $F(x) = 0$.
    \item Pour $1 \le x < 2$, $F(x) = P(X=1) = 0.1$.
    \item Pour $2 \le x < 3$, $F(x) = P(X \le 2) = 0.1 + 0.2 = 0.3$.
    \item Pour $3 \le x < 4$, $F(x) = P(X \le 3) = 0.3 + 0.3 = 0.6$.
    \item Pour $x \ge 4$, $F(x) = P(X \le 4) = 0.6 + 0.4 = 1$.
\end{itemize}
C'est une fonction en escalier qui saute aux points 1, 2, 3 et 4.
\end{correctionbox}

\begin{exercicebox}[Binomiale : "Au moins un"]
Un système de sécurité a 4 composants identiques. Chaque composant a une probabilité de 0.05 de tomber en panne dans l'année. Les pannes sont indépendantes. Quelle est la probabilité qu'au moins un composant tombe en panne dans l'année ?
\end{exercicebox}

\begin{correctionbox}
Soit $X$ le nombre de composants en panne. $X \sim \text{Bin}(n=4, p=0.05)$.
Calculer $P(X \ge 1)$ directement serait long ($P(X=1)+P(X=2)+...$). Il est plus simple de passer par l'événement complémentaire : "aucun composant ne tombe en panne".
$$ P(X \ge 1) = 1 - P(X=0) $$
$$ P(X=0) = \binom{4}{0} (0.05)^0 (0.95)^4 = 1 \times 1 \times (0.95)^4 \approx 0.8145 $$
$$ P(X \ge 1) = 1 - 0.8145 = 0.1855 $$
La probabilité est d'environ 18.55\%.
\end{correctionbox}

\begin{exercicebox}[Hypergéométrique : Main de poker]
Quelle est la probabilité de recevoir exactement 2 Rois dans une main de 5 cartes tirées d'un jeu standard de 52 cartes ?
\end{exercicebox}

\begin{correctionbox}
Soit $X$ le nombre de Rois dans la main. C'est un tirage sans remise. Il y a 4 Rois et 48 autres cartes dans le jeu. $X \sim \text{HG}(w=4, b=48, m=5)$.
On cherche $P(X=2)$ :
$$ P(X=2) = \frac{\binom{4}{2} \binom{48}{3}}{\binom{52}{5}} = \frac{6 \times 17296}{2598960} = \frac{103776}{2598960} \approx 0.0399 $$
La probabilité est d'environ 3.99\%.
\end{correctionbox}

\begin{exercicebox}[Identifier la distribution 1]
Une usine produit des vis. 2\% des vis sont défectueuses. Vous achetez une boîte de 100 vis. Modélisez le nombre de vis défectueuses dans votre boîte.
\end{exercicebox}

\begin{correctionbox}
Chaque vis peut être vue comme une épreuve de Bernoulli (défectueuse ou non). Puisque le nombre total de vis produites par l'usine est très grand par rapport à la taille de l'échantillon (100), on peut considérer les tirages comme étant indépendants et avec remise. La situation est donc modélisée par une loi binomiale : $X \sim \text{Bin}(n=100, p=0.02)$.
\end{correctionbox}

\begin{exercicebox}[Identifier la distribution 2]
Une classe contient 12 filles et 10 garçons. On choisit une équipe de 4 élèves au hasard pour un projet. Modélisez le nombre de filles dans l'équipe.
\end{exercicebox}

\begin{correctionbox}
Le choix se fait sans remise à partir d'une petite population finie (22 élèves). Les choix ne sont pas indépendants. La situation est donc modélisée par une loi hypergéométrique : $X \sim \text{HG}(w=12, b=10, m=4)$.
\end{correctionbox}

\begin{exercicebox}[Variable Indicatrice]
On lance deux dés. Soit $A$ l'événement "la somme des dés est 7". Définissez la variable aléatoire indicatrice $I_A$ et donnez sa distribution.
\end{exercicebox}

\begin{correctionbox}
La variable indicatrice $I_A$ est définie comme :
$I_A = 1$ si la somme est 7.
$I_A = 0$ si la somme n'est pas 7.

Pour trouver sa distribution, il faut calculer $p = P(A)$. Les paires qui donnent une somme de 7 sont (1,6), (2,5), (3,4), (4,3), (5,2), (6,1). Il y en a 6 sur 36 résultats possibles.
Donc, $p = P(A) = 6/36 = 1/6$.

La distribution de $I_A$ est une distribution de Bernoulli : $I_A \sim \text{Bern}(p=1/6)$.
\end{correctionbox}

\begin{exercicebox}[Binomiale : Quiz]
Un étudiant répond au hasard à un QCM de 10 questions. Chaque question a 4 choix de réponse, dont un seul est correct. Quelle est la probabilité qu'il ait au moins 3 bonnes réponses ?
\end{exercicebox}

\begin{correctionbox}
Soit $X$ le nombre de bonnes réponses. Chaque question est une épreuve de Bernoulli avec une probabilité de succès $p = 1/4 = 0.25$. Donc, $X \sim \text{Bin}(n=10, p=0.25)$.
On cherche $P(X \ge 3)$. On utilise le complémentaire : $P(X \ge 3) = 1 - P(X < 3) = 1 - (P(X=0) + P(X=1) + P(X=2))$.

$P(X=0) = \binom{10}{0}(0.25)^0(0.75)^{10} \approx 0.0563$
$P(X=1) = \binom{10}{1}(0.25)^1(0.75)^9 \approx 0.1877$
$P(X=2) = \binom{10}{2}(0.25)^2(0.75)^8 \approx 0.2816$

$P(X < 3) \approx 0.0563 + 0.1877 + 0.2816 = 0.5256$
$P(X \ge 3) \approx 1 - 0.5256 = 0.4744$

La probabilité d'avoir au moins 3 bonnes réponses est d'environ 47.44\%.
\end{correctionbox}
\newpage
\section{Distributions Multivariées et Concepts Associés}

\subsection{Distributions Jointes et Marginales}

Jusqu'à présent, nous avons étudié les variables aléatoires isolément. Nous allons maintenant examiner comment analyser les relations entre *plusieurs* variables aléatoires.

\begin{definitionbox}[Distribution Jointe (Cas Discret)]
Pour deux variables aléatoires discrètes $X$ et $Y$, la \textbf{distribution jointe} (ou loi jointe) spécifie la probabilité de chaque paire d'issues. La fonction de masse de probabilité jointe (joint PMF) est :
$$P(X=x, Y=y)$$
Si $X$ prend ses valeurs dans un ensemble $S$ et $Y$ dans un ensemble $T$, alors la somme de toutes les probabilités jointes est égale à 1 :
$$\sum_{x \in S} \sum_{y \in T} P(X=x, Y=y) = 1$$
\end{definitionbox}

Cette loi jointe est la "carte" complète de toutes les issues possibles.

\begin{intuitionbox}
La distribution jointe est la "carte" complète de toutes les issues possibles. Elle répond à la question : "Quelle est la probabilité que $X$ prenne cette valeur ET que $Y$ prenne cette autre valeur en même temps ?". Si vous imaginez un tableau à double entrée pour $X$ et $Y$, la loi jointe est l'ensemble de toutes les probabilités à l'intérieur du tableau.
\end{intuitionbox}

Cette "carte" complète contient toutes les informations. Si nous ne nous intéressons qu'à une seule variable, nous pouvons la "réduire" en calculant sa distribution marginale.

\begin{definitionbox}[Distribution Marginale]
À partir de la distribution jointe, on peut obtenir la distribution \textbf{marginale} (ou loi marginale) de chaque variable. Pour obtenir la probabilité que $X$ prenne une valeur $x$, on somme sur toutes les valeurs possibles de $Y$ :
$$P(X=x) = \sum_{y \in T} P(X=x, Y=y)$$
\end{definitionbox}

Visuellement, cela correspond à "écraser" le tableau de probabilités sur un seul de ses axes.

\begin{intuitionbox}
Les distributions marginales sont les "ombres" ou "projections" de la carte jointe sur un seul axe. Si la loi jointe est un tableau, les lois marginales sont les totaux de chaque ligne et de chaque colonne, que l'on écrirait "dans la marge" du tableau. Elles nous disent la probabilité d'une issue pour $X$ sans se soucier de ce qu'il advient de $Y$.
\end{intuitionbox}

L'exemple le plus simple est le lancer de deux dés.

\begin{examplebox}[Lois jointe et marginale]
On lance un dé rouge ($X$) et un dé bleu ($Y$). Il y a 36 issues, chacune avec une probabilité de 1/36.
\textbf{Loi jointe} : $P(X=x, Y=y) = 1/36$ pour tout $x, y \in \{1, \dots, 6\}$.
Par exemple, $P(X=2, Y=5) = 1/36$.

\textbf{Loi marginale} de $X$ : Cherchons $P(X=2)$. C'est la probabilité d'obtenir 2 sur le dé rouge, quel que soit le résultat du bleu.
$$P(X=2) = \sum_{y=1}^6 P(X=2, Y=y)$$
$$P(X=2) = P(X=2,Y=1) + \dots + P(X=2,Y=6)$$
$$P(X=2) = \frac{1}{36} + \frac{1}{36} + \frac{1}{36} + \frac{1}{36} + \frac{1}{36} + \frac{1}{36} = \frac{6}{36} = \frac{1}{6}$$
Ceci est bien la loi d'un seul dé.
\end{examplebox}

\subsection{Espérance d'une fonction de deux variables}

Maintenant que nous avons la loi jointe (la carte des probabilités), nous pouvons l'utiliser pour calculer l'espérance de n'importe quelle fonction qui dépend des deux variables, $g(X,Y)$.

\begin{definitionbox}[Espérance d'une fonction $g(X,Y)$]
L'espérance d'une fonction $g(X,Y)$ de deux variables aléatoires discrètes $X$ et $Y$ est une généralisation du théorème de transfert (LOTUS) :
$$E[g(X,Y)] = \sum_{x \in S} \sum_{y \in T} g(x,y) P(X=x, Y=y)$$
\end{definitionbox}

C'est la moyenne de $g$, pondérée par les probabilités jointes.

\begin{intuitionbox}
C'est la valeur moyenne attendue de la fonction $g$. Pour la calculer, on prend chaque résultat possible de $g(x,y)$, on le pondère par la probabilité que cette combinaison $(x,y)$ se produise (donnée par la loi jointe), et on somme le tout.
\end{intuitionbox}

Le cas le plus important de $g(X,Y)$ est la somme $X+Y$.

\begin{examplebox}{Espérance de $E[X+Y]$}
Avec nos deux dés, calculons l'espérance de la somme $S = X + Y$.  
La fonction est $g(X,Y) = X + Y$.

\[
E[X+Y] = \sum_{x=1}^6 \sum_{y=1}^6 (x+y)\, P(X=x, Y=y)
\]
\[
E[X+Y] = \sum_{x=1}^6 \sum_{y=1}^6 (x+y)\, \frac{1}{36}
\]

Plutôt que de faire ce long calcul, on peut utiliser la linéarité de l'espérance (qui est un cas particulier de ce théorème) :

\[
E[X+Y] = E[X] + E[Y] = 3.5 + 3.5 = 7.
\]
\end{examplebox}

\subsection{Covariance et Corrélation}

La linéarité $E[X+Y] = E[X] + E[Y]$ est un outil puissant. Mais l'espérance ne nous dit rien sur la *relation* entre $X$ et $Y$. Pour cela, nous introduisons la covariance.

\begin{definitionbox}[Covariance]
La \textbf{covariance} entre deux variables aléatoires $X$ et $Y$, avec pour moyennes respectives $\mu_X$ et $\mu_Y$, mesure la façon dont elles varient ensemble.
$$\text{Cov}(X,Y) = E[(X - \mu_X)(Y - \mu_Y)]$$
\end{definitionbox}

Elle mesure la direction de leur relation.

\begin{intuitionbox}
La covariance est positive si les variables ont tendance à "bouger" dans la même direction (quand $X$ est au-dessus de sa moyenne, $Y$ a tendance à l'être aussi). Elle est négative si elles bougent en sens opposé (quand $X$ est au-dessus de sa moyenne, $Y$ a tendance à être en dessous). Si elle est nulle, il n'y a pas de tendance linéaire entre elles.
\end{intuitionbox}

La définition $E[(X - \mu_X)(Y - \mu_Y)]$ est bonne pour l'intuition, mais difficile à calculer. Une formule alternative est presque toujours utilisée.

\begin{theorembox}[Formule de calcul de la covariance]
Une formule computationnelle plus simple pour la covariance est :
$$\text{Cov}(X,Y) = E[XY] - E[X]E[Y]$$
\end{theorembox}

La preuve est une simple expansion algébrique.

\begin{proofbox}
Soit $\mu_X = E[X]$ et $\mu_Y = E[Y]$. On part de la définition :
\begin{align*}
\text{Cov}(X,Y) &= E[(X - \mu_X)(Y - \mu_Y)] \\
&= E[XY - X\mu_Y - Y\mu_X + \mu_X\mu_Y] \quad \text{(On développe)} \\
&= E[XY] - E[X\mu_Y] - E[Y\mu_X] + E[\mu_X\mu_Y] \quad \text{(Par linéarité)} \\
&= E[XY] - \mu_Y E[X] - \mu_X E[Y] + \mu_X\mu_Y \quad \text{(Les moyennes sont des constantes)} \\
&= E[XY] - \mu_Y \mu_X - \mu_X \mu_Y + \mu_X\mu_Y \\
&= E[XY] - \mu_X \mu_Y \\
&= E[XY] - E[X]E[Y]
\end{align*}
\end{proofbox}

Voyons cette formule en action.

\begin{examplebox}[Calcul de covariance]
\textbf{Cas 1 : Dés indépendants}. $X$ et $Y$ sont les résultats de deux dés. $E[X]=3.5, E[Y]=3.5$.
Calculons $E[XY]$. Puisqu'ils sont indépendants, $E[XY] = E[X]E[Y] = 3.5 \times 3.5 = 12.25$.
$\text{Cov}(X,Y) = E[XY] - E[X]E[Y] = 12.25 - 12.25 = 0$.
La covariance est nulle, ce qui est attendu pour des variables indépendantes.

\textbf{Cas 2 : Variables dépendantes}. Soit $X$ un lancer de dé, et $Y = 2X$.
$E[X] = 3.5$. $E[Y] = E[2X] = 2E[X] = 7$.
$E[XY] = E[X \cdot 2X] = E[2X^2] = 2 E[X^2]$.
On sait que $E[X^2] = \frac{1^2+...+6^2}{6} = 91/6$.
$E[XY] = 2(91/6) = 91/3$.
$\text{Cov}(X,Y) = E[XY] - E[X]E[Y] = \frac{91}{3} - (3.5)(7) = \frac{91}{3} - 24.5 = 30.33... - 24.5 \approx 5.833$.
La covariance est positive, ce qui est logique : si $X$ est grand, $Y$ l'est aussi.
\end{examplebox}

La covariance est un bon indicateur de la direction de la relation, mais sa magnitude est difficile à interpréter. Pour cela, nous la normalisons.

\begin{definitionbox}[Corrélation]
La \textbf{corrélation} (ou coefficient de corrélation de Pearson, $r$) est une version normalisée de la covariance, qui se situe toujours entre -1 et 1.
$$\text{Corr}(X,Y) = r = \frac{\text{Cov}(X,Y)}{\sqrt{\text{Var}(X)\text{Var}(Y)}} = \frac{\text{Cov}(X,Y)}{\sigma_X \sigma_Y}$$
\end{definitionbox}

La corrélation résout le problème des unités.

\begin{intuitionbox}
Le problème de la covariance est qu'elle dépend des unités de $X$ et $Y$ (par ex., $\text{kg} \cdot \text{cm}$). Si vous changez les unités (grammes et mètres), la valeur de la covariance change, même si la relation est identique. La corrélation résout ce problème : elle est sans unité. Un coefficient de +1 indique une relation linéaire positive parfaite, -1 une relation linéaire négative parfaite, et 0 une absence de relation linéaire.
\end{intuitionbox}

Cette normalisation se comprend mieux en voyant la corrélation comme une covariance de variables standardisées.

\begin{intuitionbox}[Interprétation de la formule]
On peut voir la corrélation de Pearson comme un processus en 3 étapes :
\begin{enumerate}
    \item \textbf{Centrer les variables :} On calcule l'écart de chaque valeur à sa moyenne ($x_i - \bar{x}$ et $y_i - \bar{y}$). Cela élimine "l'effet de base" (ex: une personne de 180cm vs 170cm ; la moyenne change mais les écarts relatifs restent les mêmes).
    \item \textbf{Normaliser les variables :} On divise chaque écart par l'écart-type de sa variable ($z_{xi} = (x_i - \bar{x})/\sigma_X$ et $z_{yi} = (y_i - \bar{y})/\sigma_Y$). Ces nouvelles variables $Z_X$ et $Z_Y$ sont \textbf{standardisées} : elles ont une moyenne de 0, un écart-type de 1, et sont sans unité.
    \item \textbf{Calculer la covariance des variables standardisées :} La corrélation n'est rien d'autre que la covariance de ces deux nouvelles variables standardisées : $r = \text{Cov}(Z_X, Z_Y)$.
\end{enumerate}
Parce que les deux variables sont maintenant sur la même échelle (écart-type de 1), leur covariance (la corrélation) ne peut pas dépasser 1 en valeur absolue.
\end{intuitionbox}

Reprenons notre exemple de dépendance parfaite :

\begin{examplebox}[Calcul de corrélation]
Reprenons l'exemple $Y=2X$, où $X$ est un lancer de dé.
On a $\text{Cov}(X,Y) = 5.833... = 35/6$.
$\text{Var}(X) = E[X^2] - E[X]^2 = 91/6 - (3.5)^2 = 35/12$.
$\text{Var}(Y) = \text{Var(2X)} = 2^2 \text{Var}(X) = 4(35/12) = 35/3$.
$\sigma_X \sigma_Y = \sqrt{35/12} \cdot \sqrt{35/3} = \sqrt{(35 \cdot 35) / (12 \cdot 3)} = \sqrt{35^2 / 36} = 35/6$.
$$\text{Corr}(X,Y) = \frac{\text{Cov}(X,Y)}{\sigma_X \sigma_Y} = \frac{35/6}{35/6} = 1$$
La corrélation est de 1, ce qui est parfait : $Y$ est une fonction linéaire parfaite de $X$.
\end{examplebox}

\subsection{Linéarité de la Covariance}

Tout comme l'espérance, la covariance possède d'importantes propriétés de linéarité qui simplifient les calculs.

\begin{definitionbox}[Linéarité de la Covariance]
Pour des variables aléatoires $X, Y, Z$ et des constantes $a, b, c$ :
\begin{align*}
\text{Cov}(aX + bY + c, Z) &= a\text{Cov}(X, Z) + b\text{Cov}(Y, Z) \\
\text{Cov}(X, aY + bZ + c) &= a\text{Cov}(X, Y) + b\text{Cov}(X, Z)
\end{align*}
La covariance est linéaire pour chaque argument (elle est \textbf{bilinéaire}). Les constantes additives disparaissent.
\end{definitionbox}

\subsection{Résultats sur la Corrélation}

La propriété la plus importante de la corrélation, qui découle de sa normalisation, est qu'elle est bornée.

\begin{theorembox}[Bornes du Coefficient de Corrélation de Pearson]
Pour toutes variables aléatoires $X$ et $Y$, le coefficient de corrélation $\text{Corr}(X,Y)$ est borné :
$$-1 \le \text{Corr}(X,Y) \le 1$$
De plus, si $\text{Corr}(X,Y) = \pm 1$, alors il existe des constantes $a$ et $b$ telles que $Y = aX + b$, indiquant une relation linéaire parfaite.
\end{theorembox}

La preuve de ces bornes repose sur le fait que la variance est toujours positive.

\begin{proofbox}[Démonstration des Bornes de la Corrélation]
La preuve repose sur le fait que la variance d'une variable aléatoire est toujours positive ou nulle.

\textbf{Étape 1 : Variables Standardisées}
On définit les versions standardisées de $X$ et $Y$ :
$$X^* = \frac{X - \mu_X}{\sigma_X} \quad ; \quad Y^* = \frac{Y - \mu_Y}{\sigma_Y}$$
Par construction, $E[X^*]=E[Y^*]=0$ et $\text{Var}(X^*) = \text{Var}(Y^*) = 1$.

\textbf{Étape 2 : Covariance des variables standardisées}
Calculons la covariance de $X^*$ et $Y^*$, qui est, par définition, la corrélation de $X$ et $Y$.
\begin{align*}
\text{Cov}(X^*, Y^*) &= \text{Cov}\left( \frac{X - \mu_X}{\sigma_X}, \frac{Y - \mu_Y}{\sigma_Y} \right) \\
&= \frac{1}{\sigma_X \sigma_Y} \text{Cov}(X - \mu_X, Y - \mu_Y) \\
&= \frac{1}{\sigma_X \sigma_Y} \text{Cov}(X, Y) \\
&= \text{Corr}(X,Y)
\end{align*}

\textbf{Étape 3 : Variance de la somme et de la différence}
Considérons la variance de la somme et de la différence de ces variables standardisées.
$$\text{Var}(X^* + Y^*) = \text{Var}(X^*) + \text{Var}(Y^*) + 2\text{Cov}(X^*, Y^*)$$
$$\text{Var}(X^* + Y^*) = 1 + 1 + 2\text{Corr}(X,Y) = 2 + 2\text{Corr}(X,Y)$$
De même :
$$\text{Var}(X^* - Y^*) = \text{Var}(X^*) + \text{Var}(Y^*) - 2\text{Cov}(X^*, Y^*)$$
$$\text{Var}(X^* - Y^*) = 1 + 1 - 2\text{Corr}(X,Y) = 2 - 2\text{Corr}(X,Y)$$

\textbf{Étape 4 : La variance est toujours $\ge 0$}
La variance d'une variable aléatoire ne peut pas être négative.
$$\text{Var}(X^* + Y^*) \ge 0 \implies 2 + 2\text{Corr}(X,Y) \ge 0 \implies \text{Corr}(X,Y) \ge -1$$
$$\text{Var}(X^* - Y^*) \ge 0 \implies 2 - 2\text{Corr}(X,Y) \ge 0 \implies \text{Corr}(X,Y) \le 1$$
Ceci nous donne le résultat final :
$$-1 \le \text{Corr}(X,Y) \le 1$$
\end{proofbox}

\subsection{Standardisation et Non-Corrélation}

Le processus de 'standardisation' utilisé dans la preuve de la corrélation et dans l'intuition est un concept fondamental en soi.

\begin{definitionbox}[Variable Centrée Réduite]
Soit $X$ une variable aléatoire avec :
\begin{itemize}
    \item moyenne $\mu_X = E[X]$
    \item écart-type $\sigma_X = \sqrt{\operatorname{Var}(X)} > 0$
\end{itemize}
On définit sa version \textbf{centrée réduite} (standardisée) $Z$ par :
$$Z = \frac{X - \mu_X}{\sigma_X}$$
Alors, $Z$ a les propriétés suivantes :
\begin{enumerate}
    \item \textbf{Centrée (moyenne nulle)} :
    \begin{align*}
    E[Z] &= E\left[\frac{X - \mu_X}{\sigma_X}\right] \\
         &= \frac{1}{\sigma_X} E[X - \mu_X] \quad (\text{par linéarité, } \sigma_X \text{ est une constante}) \\
         &= \frac{1}{\sigma_X} (E[X] - E[\mu_X]) \\
         &= \frac{1}{\sigma_X} (E[X] - \mu_X) \quad (\text{car } \mu_X \text{ est une constante}) \\
         &= \frac{\mu_X - \mu_X}{\sigma_X} = 0
    \end{align*}
    \item \textbf{Réduite (écart-type égal à 1)} :
    \begin{align*}
    \operatorname{Var}(Z) &= \operatorname{Var}\left(\frac{X - \mu_X}{\sigma_X}\right) \\
         &= \left(\frac{1}{\sigma_X}\right)^2 \operatorname{Var}(X - \mu_X) \quad (\text{propriété } \operatorname{Var}(aY) = a^2 \operatorname{Var}(Y)) \\
         &= \frac{1}{\sigma_X^2} \operatorname{Var}(X) \quad (\text{propriété } \operatorname{Var}(Y+b) = \operatorname{Var}(Y)) \\
         &= \frac{1}{\sigma_X^2} \cdot \sigma_X^2 = 1
    \end{align*}
    L'écart-type est donc $\sigma_Z = \sqrt{\operatorname{Var}(Z)} = \sqrt{1} = 1$.
\end{enumerate}
\end{definitionbox}

Cette transformation permet de comparer des variables sur des échelles différentes.

\begin{intuitionbox}[Que signifie centrer-réduire ?]
Standardiser une variable se fait en deux temps, comme le montre la formule $Z = \frac{X - \mu_X}{\sigma_X}$ :
\begin{enumerate}
    \item \textbf{Centrer ($X - \mu_X$)} : C'est la première étape. On soustrait la moyenne $\mu_X$. Cela revient à "déplacer" la distribution pour que son centre de gravité (sa moyenne) soit maintenant à 0. On ne regarde plus les valeurs brutes $X$, mais leurs \textbf{écarts} par rapport à la moyenne. (Propriété 1 : $E[Z]=0$)
    \item \textbf{Réduire ($... / \sigma_X$)} : C'est la deuxième étape. On divise ces écarts par l'écart-type $\sigma_X$. Cela revient à changer d'unité de mesure. L'ancienne unité (kg, cm, points...) est remplacée par une nouvelle unité universelle : "le nombre d'écarts-types". (Propriété 2 : $\text{Var}(Z)=1$)
\end{enumerate}
Au final, une variable $Z$ avec une valeur de 1.5 signifie "cette observation est 1.5 écarts-types au-dessus de la moyenne de sa distribution d'origine", peu importe ce que $X$ mesurait.
\end{intuitionbox}

\begin{intuitionbox}[Analogie simple]
Imaginons 2 élèves :
\begin{itemize}
    \item Alice a des notes entre 80 et 100 (moyenne 90, écart-type 5).
    \item Bob a des notes entre 0 et 20 (moyenne 10, écart-type 4).
\end{itemize}
Comparer leurs notes brutes n'a pas de sens. Mais si on les standardise, on peut se demander : "quand Alice est 1 écart-type au-dessus de sa moyenne (une note de 95), Bob est-il aussi 1 écart-type au-dessus de sa propre moyenne (une note de 14) ?". La standardisation permet cette comparaison.
\end{intuitionbox}

\begin{examplebox}[Centrer-réduire un dé]
Pour un lancer de dé $X$, on a $\mu_X = 3.5$ et $\sigma_X = \sqrt{35/12} \approx 1.708$.
Si on obtient $X=6$ : $Z = (6 - 3.5) / 1.708 \approx 1.46$.
Si on obtient $X=1$ : $Z = (1 - 3.5) / 1.708 \approx -1.46$.
Obtenir 6 est à 1.46 écarts-types au-dessus de la moyenne.
\end{examplebox}

Maintenant, formalisons le concept d'une covariance nulle.

\begin{definitionbox}[Variables Non Corréelées]
On dit que deux variables aléatoires $X$ et $Y$ sont \textbf{non corrélées} si leur covariance est nulle :
$$\text{Cov}(X,Y) = 0$$
Cela est équivalent à dire que $E[XY] = E[X]E[Y]$.
\end{definitionbox}

Il est crucial de ne pas confondre "non corrélées" et "indépendantes".

\begin{intuitionbox}
"Non corrélées" signifie qu'il n'y a \textbf{pas de relation linéaire} entre les variables. C'est plus faible que l'indépendance. Si $X$ et $Y$ sont indépendantes, elles sont forcément non corrélées. Mais l'inverse n'est pas vrai : $X$ et $Y$ peuvent être non corrélées (Cov=0) mais quand même dépendantes (par exemple si $Y=X^2$ pour un $X$ centré).
\end{intuitionbox}

\subsection{Variance d'une Somme de Variables Aléatoires}

Nous pouvons maintenant combiner nos connaissances de la variance et de la covariance pour répondre à une question cruciale : quelle est la variance d'une somme de variables, $X+Y$ ?

\begin{theorembox}[Formules pour la variance d'une somme de deux variables]
Pour deux variables aléatoires $X$ et $Y$ :
\begin{align*}
\text{Var}(X+Y) &= \text{Var}(X) + \text{Var}(Y) + 2\text{Cov}(X,Y) \\
\end{align*}
\end{theorembox}

La preuve découle de la définition de la variance et de la linéarité de l'espérance.

\begin{proofbox}
Soit $\mu_X = E[X]$ et $\mu_Y = E[Y]$.
\begin{align*}
\text{Var}(X+Y) &= E\left[ ((X+Y) - E[X+Y])^2 \right] \\
&= E\left[ ((X+Y) - (\mu_X + \mu_Y))^2 \right] \quad \text{(Par linéarité de E)} \\
&= E\left[ ((X - \mu_X) + (Y - \mu_Y))^2 \right] \quad \text{(On regroupe les termes)} \\
\text{Posons } A = (X - \mu_X) \text{ et } B = (Y - \mu_Y). \\
&= E[ (A + B)^2 ] = E[ A^2 + 2AB + B^2 ] \\
&= E[A^2] + 2E[AB] + E[B^2] \quad \text{(Par linéarité de E)} \\
\text{Or, par définition :} \\
E[A^2] &= E[(X-\mu_X)^2] = \text{Var}(X) \\
E[B^2] &= E[(Y-\mu_Y)^2] = \text{Var}(Y) \\
E[AB] &= E[(X-\mu_X)(Y-\mu_Y)] = \text{Cov}(X,Y) \\
\text{Donc, } \text{Var}(X+Y) &= \text{Var}(X) + \text{Var}(Y) + 2\text{Cov}(X,Y)
\end{align*}
\end{proofbox}

Cette formule est fondamentale en finance et en ingénierie.

\begin{intuitionbox}
La "volatilité" (variance) d'une somme n'est pas juste la somme des volatilités. Il faut ajouter le terme d'interaction (covariance).
Si $\text{Cov}(X,Y) > 0$ (elles bougent ensemble), la somme est \textbf{plus} volatile que la somme des parties.
Si $\text{Cov}(X,Y) < 0$ (elles bougent en sens inverse), elles s'amortissent mutuellement. La somme est \textbf{moins} volatile. C'est le principe de la diversification en finance.
\end{intuitionbox}

Cela mène à un corollaire très important lorsque la covariance est nulle.

\begin{theorembox}[Cas Particulier : Variables Non Corréelées]
Si $X$ et $Y$ sont non corrélées (Cov=0), la formule se simplifie :
$$\text{Var}(X+Y) = \text{Var}(X) + \text{Var}(Y)$$
\end{theorembox}

\begin{proofbox}
Cela découle directement du théorème précédent. Si $X$ et $Y$ sont non corrélées, alors $\text{Cov}(X,Y) = 0$.
Le terme $2\text{Cov}(X,Y)$ dans la formule générale $\text{Var}(X) + \text{Var}(Y) + 2\text{Cov}(X,Y)$ devient nul, laissant :
$$\text{Var}(X+Y) = \text{Var}(X) + \text{Var}(Y)$$
\end{proofbox}

C'est le cas pour nos dés indépendants.

\begin{examplebox}[Variance d'une somme de dés]
Soit $S = X+Y$ la somme de deux dés indépendants.
Puisqu'ils sont indépendants, ils sont non corrélés ($\text{Cov}(X,Y)=0$).
On sait $\text{Var}(X) = 35/12$ et $\text{Var}(Y) = 35/12$.
$$\text{Var}(S) = \text{Var}(X) + \text{Var}(Y) = \frac{35}{12} + \frac{35}{12} = \frac{70}{12} = \frac{35}{6} \approx 5.833$$
C'est bien plus simple que de calculer $E[S^2]$ et $E[S]$.
\end{examplebox}

On peut généraliser cette formule à $N$ variables.

\begin{theorembox}[Variance d'une somme de N variables]
La formule générale pour la somme de $N$ variables aléatoires $X_1, \dots, X_n$ est :
$$\text{Var}\left(\sum_{i=1}^n X_i\right) = \sum_{i=1}^n \text{Var}(X_i) + \sum_{i \neq j} \text{Cov}(X_i, X_j)$$
\end{theorembox}

\begin{proofbox}
On utilise la propriété $\text{Var}(S) = \text{Cov}(S, S)$ et la bilinéarité de la covariance.
Soit $S = \sum_{i=1}^n X_i$.
\begin{align*}
\text{Var}(S) &= \text{Cov}(S, S) = \text{Cov}\left(\sum_{i=1}^n X_i, \sum_{j=1}^n X_j\right) \\
&= \sum_{i=1}^n \sum_{j=1}^n \text{Cov}(X_i, X_j) \quad \text{(Par bilinéarité)}
\end{align*}
On peut séparer cette double somme en deux parties : le cas où $i=j$ et le cas où $i \neq j$.
$$ \text{Var}(S) = \sum_{i=1}^n \text{Cov}(X_i, X_i) + \sum_{i \neq j} \text{Cov}(X_i, X_j) $$
Puisque $\text{Cov}(X_i, X_i) = E[(X_i - \mu_i)(X_i - \mu_i)] = E[(X_i - \mu_i)^2] = \text{Var}(X_i)$, on obtient :
$$ \text{Var}(S) = \sum_{i=1}^n \text{Var}(X_i) + \sum_{i \neq j} \text{Cov}(X_i, X_j) $$
\end{proofbox}

Cette formule est au cœur de la théorie moderne du portefeuille.

\begin{intuitionbox}
La variance totale d'un système (comme un portefeuille d'actions) est la somme de toutes les variances individuelles ("risques propres") plus la somme de \textbf{toutes} les paires de covariances ("risques d'interaction"). Dans un grand portefeuille, le nombre de termes de covariance (environ $n^2$) est bien plus grand que le nombre de termes de variance ($n$), donc le risque total est dominé par la façon dont les actifs interagissent.
\end{intuitionbox}

\subsection{Théorème sur la somme de lois de Poisson}

Terminons avec un théorème très utile qui combine les idées d'indépendance et de somme de variables aléatoires pour une distribution spécifique.

\begin{theorembox}[La Somme de v.a. de Poisson Indépendantes est Poisson]
Soit $X_1, \dots, X_k$ une séquence de variables aléatoires de Poisson indépendantes, avec des paramètres respectifs $\lambda_1, \dots, \lambda_k$.
$$X_i \sim \text{Poisson}(\lambda_i) \quad \text{pour } i=1, \dots, k$$
Alors leur somme $Y = X_1 + \dots + X_k$ suit également une loi de Poisson, dont le paramètre est la somme des paramètres :
$$Y \sim \text{Poisson}(\lambda_1 + \dots + \lambda_k)$$
\end{theorembox}

La preuve pour $k=2$ (qui se généralise par récurrence) utilise l'indépendance et la formule du binôme de Newton.

\begin{proofbox}[Preuve pour la somme de deux v.a.]
Soit $X \sim \text{Poisson}(\lambda_1)$ et $Y \sim \text{Poisson}(\lambda_2)$, indépendantes.
Soit $S = X+Y$. Nous cherchons $P(S=k)$.
Pour que $S=k$, il faut que $X=j$ et $Y=k-j$, pour toutes les valeurs possibles de $j$ (de $0$ à $k$).
$$ P(S=k) = \sum_{j=0}^k P(X=j, Y=k-j) $$
Par indépendance, $P(X=j, Y=k-j) = P(X=j)P(Y=k-j)$.
\begin{align*}
P(S=k) &= \sum_{j=0}^k \left( \frac{e^{-\lambda_1}\lambda_1^j}{j!} \right) \left( \frac{e^{-\lambda_2}\lambda_2^{k-j}}{(k-j)!} \right) \\
&= e^{-(\lambda_1 + \lambda_2)} \sum_{j=0}^k \frac{\lambda_1^j \lambda_2^{k-j}}{j!(k-j)!}
\end{align*}
On multiplie et on divise par $k!$ pour faire apparaître le coefficient binomial :
\begin{align*}
P(S=k) &= \frac{e^{-(\lambda_1 + \lambda_2)}}{k!} \sum_{j=0}^k \frac{k!}{j!(k-j)!} \lambda_1^j \lambda_2^{k-j} \\
&= \frac{e^{-(\lambda_1 + \lambda_2)}}{k!} \sum_{j=0}^k \binom{k}{j} \lambda_1^j \lambda_2^{k-j}
\end{align*}
La somme est l'expansion du binôme de Newton pour $(\lambda_1 + \lambda_2)^k$.
$$ P(S=k) = \frac{e^{-(\lambda_1 + \lambda_2)} (\lambda_1 + \lambda_2)^k}{k!} $$
C'est la PMF d'une loi $\text{Poisson}(\lambda_1 + \lambda_2)$.
\end{proofbox}

Ce résultat est très intuitif :

\begin{intuitionbox}
Si des événements rares se produisent indépendamment à des taux constants, le nombre total d'événements se produisant est aussi un événement rare se produisant au taux total. Si les emails arrivent à $\lambda_1=5$/heure et les appels à $\lambda_2=10$/heure, les "communications totales" arrivent simplement à $\lambda = 5+10 = 15$/heure.
\end{intuitionbox}

\begin{examplebox}[Centre d'appels]
Un centre d'appels reçoit des appels "Ventes" selon $X_1 \sim \text{Poisson}(10 \text{ appels/heure})$ et des appels "Support" selon $X_2 \sim \text{Poisson}(15 \text{ appels/heure})$. Les deux types d'appels sont indépendants.
Le nombre total d'appels $Y = X_1 + X_2$ suit une loi $Y \sim \text{Poisson}(10+15=25 \text{ appels/heure})$.
La probabilité de recevoir exactement 20 appels en une heure est :
$$P(Y=20) = \frac{e^{-25} 25^{20}}{20!}$$
\end{examplebox}
\newpage
\section{La Loi Normale (ou Gaussienne)}

\subsection{Introduction et Fonction de Densité (PDF)}

Après les lois discrètes et les lois continues de base (Uniforme, Exponentielle), nous abordons la distribution la plus célèbre et la plus utilisée en probabilités et statistiques.

\begin{definitionbox}[Loi Normale]
Une variable aléatoire continue $X$ suit une \textbf{loi normale} (ou loi de Gauss) de paramètres $\mu$ (l'espérance) et $\sigma^2$ (la variance), notée $X \sim \mathcal{N}(\mu, \sigma^2)$, si sa fonction de densité de probabilité (PDF) est donnée par :
$$ f(x; \mu, \sigma) = \frac{1}{\sigma \sqrt{2\pi}} e^{ -\frac{1}{2} \left( \frac{x-\mu}{\sigma} \right)^2 } $$
pour tout $x \in (-\infty, \infty)$, où $\sigma > 0$.
\end{definitionbox}

Cette formule, bien qu'imposante, décrit une forme très familière : la courbe en cloche.

\begin{intuitionbox}[La Courbe en Cloche]
La loi normale est sans doute la distribution la plus importante en probabilités et statistiques. Pourquoi ? Parce qu'elle modélise remarquablement bien de nombreux phénomènes naturels et processus aléatoires où les valeurs tendent à se regrouper autour d'une moyenne, avec des écarts symétriques devenant de plus en plus rares à mesure qu'on s'éloigne de cette moyenne. Pensez à la taille des individus dans une population, aux erreurs de mesure répétées, ou même aux notes d'un grand groupe d'étudiants à un examen bien conçu. 

Sa densité a une forme caractéristique de \textbf{cloche symétrique} :
\begin{itemize}
    \item \textbf{Le Centre ($\mu$)} : Le paramètre $\mu$ représente l'\textbf{espérance} (la moyenne) de la distribution. C'est le centre de symétrie de la courbe, là où la cloche atteint son \textbf{sommet}. C'est la valeur la plus probable (le mode) et aussi la valeur qui coupe la distribution en deux moitiés égales (la médiane). Changer $\mu$ \textit{translate} la cloche horizontalement sans changer sa forme.
    \item \textbf{La Dispersion ($\sigma$)} : Le paramètre $\sigma$ est l'\textbf{écart-type} ($\sigma^2$ est la variance). Il mesure la \textbf{dispersion} des valeurs autour de la moyenne $\mu$. Géométriquement, $\sigma$ contrôle la \textbf{largeur} de la cloche.
        \begin{itemize}
            \item Un \textit{petit} $\sigma$ signifie que les données sont très concentrées autour de la moyenne, donnant une cloche \textbf{étroite et pointue}.
            \item Un \textit{grand} $\sigma$ signifie que les données sont plus étalées, donnant une cloche \textbf{large et aplatie}.
        \end{itemize}
    Les points d'inflexion de la courbe (là où la courbure change de sens) se situent exactement à $\mu \pm \sigma$.
\end{itemize}

\tcblower
\centering
\begin{tikzpicture}
    \begin{axis}[
        title={La Courbe en Cloche (PDF de la Loi Normale)},
        xlabel={$x$},
        ylabel={$f(x)$},
        axis lines=middle,
        no markers,
        samples=100,
        domain=-4:4,
        height=8cm,
        width=\linewidth-1cm,
        tick label style={font=\tiny},
        legend style={at={(0.5,-0.15)}, anchor=north, font=\small},
        legend columns=2
    ]
    % N(0, 1)
    \addplot [blue, ultra thick] {1/(sqrt(2*pi))*exp(-x^2/2)};
    \addlegendentry{$\mu=0, \sigma=1$};
    % N(0, 0.25) => sigma=0.5
    \addplot [red, ultra thick] {1/(0.5*sqrt(2*pi))*exp(-x^2/(2*0.5^2))};
    \addlegendentry{$\mu=0, \sigma=0.5$ (étroite)};
    % N(1, 2.25) => sigma=1.5
    \addplot [green!70!black, ultra thick] {1/(1.5*sqrt(2*pi))*exp(-(x-1)^2/(2*1.5^2))};
    \addlegendentry{$\mu=1, \sigma=1.5$ (large, décalée)};

    \draw [dashed] (axis cs:0,0) -- (axis cs:0, {1/(sqrt(2*pi))}) node[above, font=\tiny] {pic à $\mu=0$}; % Ligne pour mu=0
    \draw [dashed] (axis cs:1,0) -- (axis cs:1, {1/(1.5*sqrt(2*pi))}) node[above right, font=\tiny] {pic à $\mu=1$}; % Ligne pour mu=1
    \end{axis}
\end{tikzpicture}
\par\small\textit{Influence de $\mu$ (position) et $\sigma$ (largeur) sur la forme de la cloche.}
\end{intuitionbox}

Mais d'où vient cette formule spécifique ? Il existe une dérivation fascinante à partir d'hypothèses fondamentales sur les erreurs aléatoires (argument d'Herschel-Maxwell).

\begin{proofbox}[Dérivation de la Densité Normale à partir des Principes Fondamentaux]

\textbf{Contexte Visuel :} Imaginons un nuage de points dispersés autour d'une cible à l'origine $(0,0)$, comme des impacts de fléchettes. Le graphique ci-dessous illustre cette dispersion. On s'intéresse à la probabilité de tomber dans une petite zone, comme $dA$, autour d'un point $(x, y)$.

\begin{center}
\begin{tikzpicture}
\begin{axis}[
    axis lines=middle, % Axes qui passent par l'origine (0,0)
    xlabel=$x$,       % Étiquette axe X
    ylabel=$y$,       % Étiquette axe Y
    axis line style={magenta}, % Couleur des axes
    xlabel style={anchor=west, magenta}, % Style de l'étiquette X
    ylabel style={anchor=south, magenta}, % Style de l'étiquette Y
    xmin=-3.5, xmax=3.5, % Limites du graphique
    ymin=-3.5, ymax=3.5,
    tick label style={font=\tiny} % Police plus petite pour les graduations
]

% Ajout des points du nuage. Ce sont des coordonnées approximatives.
\addplot [only marks, mark=*, cyan, mark size=1.5pt]
coordinates {
    (0.2, 0.1) (-0.5, 0.2) (0.1, -0.3) (0.5, -0.8) (-0.3, -1.2) (0,0.1)
    (1.5, 1.5) (0.8, 0.8) (2.0, -0.5) (2.8, -1.4) (2.5, -0.2)
    (-1.8, -1.3) (-2.5, 0.5) (-1.5, 0.3) (-2.2, -0.8) (-0.8, 0.5)
    (0.5, 1.2) (0.7, 2.8) (0.2, 3.2) (-0.5, 1.5) (-1.0, -2.0)
    (1.8, -1.0) (1.0, -1.5) (0.3, -2.5) (-1.8, -2.8) (1.2, 0.5)
    (-1.2, -1.5) (-0.8, 1.1)
};

% --- Annotations ---

% Points et boîte pour 'dA'
\addplot [only marks, mark=*, red, mark size=1.5pt] coordinates {(1.3, 2.0)};
\draw [red, thick] (axis cs:1.1, 1.8) rectangle (axis cs:1.5, 2.2);
\node [red, above] at (axis cs:1.3, 2.2) {$dA$};

% Points et boîte pour 'dB'
\addplot [only marks, mark=*, cyan, mark size=1.5pt] 
coordinates {(-1.2, 2.0) (-1.3, 2.3) (-1.0, 2.2) (-1.1, 1.9)};
\draw [blue, thick] (axis cs:-1.8, 1.2) rectangle (axis cs:-0.8, 2.8);
\node [blue, above] at (axis cs:-1.3, 2.8) {$dB$};

\end{axis}
\end{tikzpicture}
\end{center}

\textbf{Objectif :} Expliquer comment arriver à la formule mathématique de la courbe en cloche (densité de probabilité normale) en partant de principes fondamentaux sur les erreurs aléatoires.

\textbf{1. Le Point de Départ : Densité et Aire $dA$}
Dans une distribution continue, la probabilité de tomber \textit{exactement} sur un point $(x, y)$ est nulle. On ne peut donc pas parler de "probabilité d'un point". On parle de la probabilité de tomber \textit{dans une petite zone}, comme un rectangle $dA = dx \cdot dy$ autour du point $(x, y)$.
Cette probabilité, notée $P(\text{dans } dA)$, est \textit{proportionnelle} à l'aire de la zone $dA$. La \textit{constante de proportionnalité} est la \textbf{fonction de densité de probabilité} $p(x, y)$ évaluée en ce point. En d'autres termes, la densité $p(x, y)$ \textit{représente} localement la concentration de probabilité. Ainsi, la probabilité de tomber dans la zone $dA$ est approximativement :
$$ P(\text{dans } dA) \approx p(x, y) \cdot dA $$

\textbf{2. Les Hypothèses Fondamentales}
On pose deux hypothèses sur la nature de ces erreurs (représentées par la densité $p(x, y)$) :
\begin{enumerate}
    \item \textbf{Indépendance des axes :} L'erreur horizontale ($x$) est indépendante de l'erreur verticale ($y$). Cela implique que la densité jointe $p(x, y)$ peut s'écrire comme le produit de la densité marginale sur $x$, notée $f(x)$, et de la densité marginale sur $y$, notée $f(y)$. Donc, $p(x, y) = f(x) \cdot f(y)$.
    \item \textbf{Symétrie de rotation (Isotropie) :} La densité ne dépend que de la distance $r = \sqrt{x^2 + y^2}$ au centre, pas de l'angle. Il existe donc une fonction $\phi(r)$ telle que la densité en $(x,y)$ est $p(x, y) = \phi(\sqrt{x^2 + y^2})$.
\end{enumerate}

\textbf{3. L'Équation Fonctionnelle}
En égalant les deux expressions pour la même densité $p(x, y)$ (à une constante près), on obtient :
$$ f(x) \cdot f(y) = \phi(\sqrt{x^2 + y^2}) $$
Pour $y=0$, on a $f(x) \cdot f(0) = \phi(x)$. Posons $f(0) = \lambda$. Alors $\phi(x) = \lambda f(x)$.
L'équation devient :
$$ f(x) \cdot f(y) = \lambda f(\sqrt{x^2 + y^2}) $$

\textbf{4. Résolution de l'Équation Fonctionnelle}
Posons $g(x) = f(x)/\lambda$, avec $g(0)=1$. L'équation se simplifie en :
$$ g(x) g(y) = g(\sqrt{x^2 + y^2}) $$
Posons $g(x) = h(x^2)$. L'équation devient $h(x^2)h(y^2) = h(x^2+y^2)$. Avec $a=x^2$ et $b=y^2$, on a :
$$ h(a) h(b) = h(a+b) $$
La solution continue de cette équation de Cauchy est $h(a) = e^{Aa}$ pour une constante $A$.
Retour aux fonctions : $g(x) = h(x^2) = e^{Ax^2}$. $f(x) = \lambda g(x) = \lambda e^{Ax^2}$.
Comme la densité doit diminuer loin du centre, $A$ doit être négative. Posons $A = -k$ avec $k>0$.
$$ f(x) = \lambda e^{-k x^2} $$

\textbf{5. Normalisation et Identification des Paramètres}
\begin{enumerate}
    \item \textbf{Condition $\int f(x) dx = 1$} : L'intégrale Gaussienne $\int_{-\infty}^{\infty} e^{-k x^2} \, \mathrm{d}x = \sqrt{\frac{\pi}{k}}$.
    Donc, $\int_{-\infty}^{\infty} f(x) dx = \lambda \sqrt{\frac{\pi}{k}} = 1 \implies \lambda = \sqrt{\frac{k}{\pi}}$.
    \item \textbf{Lien avec la Variance ($\sigma^2$)} : Pour une distribution centrée, $\sigma^2 = E[X^2] = \int x^2 f(x) dx$.
    $$ \sigma^2 = \int_{-\infty}^{\infty} x^2 \left( \sqrt{\frac{k}{\pi}} e^{-k x^2} \right) \, \mathrm{d}x = \sqrt{\frac{k}{\pi}} \left( \frac{1}{2k} \sqrt{\frac{\pi}{k}} \right) = \frac{1}{2k} $$
    Donc, $k = \frac{1}{2\sigma^2}$.
    \item \textbf{Substitution Finale :} Remplaçons $k$ dans $\lambda$ et $f(x)$.
    $$ \lambda = \sqrt{\frac{1/(2\sigma^2)}{\pi}} = \frac{1}{\sigma\sqrt{2\pi}} $$
    $$ f(x) = \frac{1}{\sigma\sqrt{2\pi}} e^{-\frac{1}{2\sigma^2} x^2} = \frac{1}{\sigma\sqrt{2\pi}} e^{ -\frac{x^2}{2\sigma^2} } $$
    \item \textbf{Généralisation (Moyenne $\mu$)} : Pour centrer la distribution sur $\mu$, on remplace $x$ par $(x-\mu)$ dans l'exposant :
    $$ f(x; \mu, \sigma) = \frac{1}{\sigma \sqrt{2\pi}} e^{ -\frac{(x-\mu)^2}{2\sigma^2} } $$
\end{enumerate}
C'est la fonction de densité de la loi normale $\mathcal{N}(\mu, \sigma^2)$.
\end{proofbox}

\subsection{La Loi Normale Centrée Réduite $\mathcal{N}(0, 1)$}

Avant d'explorer les propriétés de la loi normale générale, concentrons-nous sur son cas le plus simple et le plus fondamental.

\begin{definitionbox}[Loi Normale Standard (ou Centrée Réduite)]
Un cas particulier extraordinairement utile est la loi normale avec une moyenne $\mu=0$ et une variance $\sigma^2=1$ (donc $\sigma=1$). On l'appelle la \textbf{loi normale standard} ou \textbf{centrée réduite}, et on la note souvent $Z$. Sa PDF est traditionnellement notée $\phi(z)$ :
$$ \phi(z) = \frac{1}{\sqrt{2\pi}} e^{-z^2/2} $$
Sa fonction de répartition (CDF), qui donne $P(Z \le z)$, est notée $\Phi(z)$ :
$$ \Phi(z) = P(Z \le z) = \int_{-\infty}^z \frac{1}{\sqrt{2\pi}} e^{-t^2/2} \, \mathrm{d}t $$
\end{definitionbox}

Pourquoi cette version standard est-elle si importante ? Elle sert de référence universelle.

\begin{intuitionbox}[La Référence Universelle et le Changement d'Unités]
Pourquoi cette loi $\mathcal{N}(0, 1)$ est-elle si centrale ? Imaginez que vous ayez des mesures en degrés Celsius ($\mathcal{N}(\mu_C, \sigma_C^2)$) et d'autres en degrés Fahrenheit ($\mathcal{N}(\mu_F, \sigma_F^2)$). Comment les comparer ? La loi normale standard fournit un \textbf{système d'unités universel}.

Toute variable normale $X \sim \mathcal{N}(\mu, \sigma^2)$ peut être transformée ("standardisée") en une variable $Z \sim \mathcal{N}(0, 1)$ par un simple changement d'échelle et de position : $Z = (X-\mu)/\sigma$. 

Cela signifie qu'au lieu de devoir calculer des aires (probabilités) pour une infinité de courbes en cloche différentes (une pour chaque paire $\mu, \sigma$), on peut tout ramener à \textbf{une seule courbe de référence}, $\mathcal{N}(0, 1)$. Les aires sous cette courbe standard ($\Phi(z)$) ont été calculées une fois pour toutes et sont disponibles dans des tables ou des logiciels. On n'a plus qu'à convertir notre problème dans cette "langue" standard, trouver la probabilité, et interpréter le résultat.
\end{intuitionbox}

La notation est très standardisée pour cette loi.

\begin{remarquebox}[Notation $\phi$ et $\Phi$]
Les symboles $\phi$ (phi minuscule) pour la PDF et $\Phi$ (phi majuscule) pour la CDF de la loi normale standard sont quasi universels. Il est important de ne pas les confondre. $\phi(z)$ est la \textit{hauteur} de la courbe en $z$, tandis que $\Phi(z)$ est l'\textit{aire} sous la courbe à gauche de $z$.
\end{remarquebox}

Un détail technique important concerne le calcul de $\Phi(z)$.

\begin{remarquebox}[Absence de Primitive Simple]
L'intégrale $\int e^{-t^2/2} \, \mathrm{d}t$, nécessaire pour calculer $\Phi(z)$, n'a \textbf{pas d'expression analytique} en termes de fonctions élémentaires (polynômes, exponentielles, log, sin, cos...). C'est une fonction spéciale, connue sous le nom de \textbf{fonction d'erreur} (liée à $\Phi$ par une transformation simple). C'est la raison pour laquelle on dépend de tables ou de calculs numériques pour obtenir les valeurs de $\Phi(z)$. Heureusement, ces outils sont omniprésents aujourd'hui.
\end{remarquebox}

\subsection{Standardisation : Le Score Z}

Formalisons cette transformation clé qui relie toute loi normale à la loi standard.

\begin{theorembox}[Standardisation d'une Variable Normale]
Si $X \sim \mathcal{N}(\mu, \sigma^2)$, alors la variable $Z$ définie par :
$$ Z = \frac{X - \mu}{\sigma} $$
suit la loi normale standard, $Z \sim \mathcal{N}(0, 1)$.
\end{theorembox}

La preuve formelle utilise un changement de variable dans la fonction de répartition.

\begin{proofbox}
Soit $F_X(x)$ la CDF de $X$ et $F_Z(z)$ la CDF de $Z$. Nous voulons montrer que $F_Z(z) = \Phi(z)$.
\begin{align*}
F_Z(z) &= P(Z \le z) \\
&= P\left( \frac{X-\mu}{\sigma} \le z \right) \\
&= P(X - \mu \le z\sigma) \\
&= P(X \le \mu + z\sigma) \\
&= F_X(\mu + z\sigma)
\end{align*}
Par définition de la CDF de $X$ :
$$ F_X(x) = \int_{-\infty}^x \frac{1}{\sigma \sqrt{2\pi}} e^{ -\frac{1}{2} \left( \frac{t-\mu}{\sigma} \right)^2 } \, dt $$
Donc,
$$ F_Z(z) = \int_{-\infty}^{\mu + z\sigma} \frac{1}{\sigma \sqrt{2\pi}} e^{ -\frac{1}{2} \left( \frac{t-\mu}{\sigma} \right)^2 } \, dt $$
Effectuons le changement de variable $u = (t-\mu)/\sigma$. Alors $t = \mu + u\sigma$ et $dt = \sigma du$.
Les bornes d'intégration changent :
\begin{itemize}
    \item Quand $t \to -\infty$, $u \to -\infty$.
    \item Quand $t = \mu + z\sigma$, $u = ((\mu + z\sigma)-\mu)/\sigma = z$.
\end{itemize}
L'intégrale devient :
$$ F_Z(z) = \int_{-\infty}^{z} \frac{1}{\sigma \sqrt{2\pi}} e^{ -\frac{1}{2} u^2 } (\sigma du) $$
$$ F_Z(z) = \int_{-\infty}^{z} \frac{1}{\sqrt{2\pi}} e^{ -u^2/2 } \, du $$
C'est exactement la définition de $\Phi(z)$, la CDF de la loi normale standard. Ainsi, $Z \sim \mathcal{N}(0, 1)$.
\end{proofbox}

Cette transformation a une interprétation très concrète.

\begin{intuitionbox}[Mesurer en "Unités d'Écart-Type"]
Transformer $X$ en $Z$ s'appelle \textbf{standardiser} la variable. Le résultat, $z = \frac{x-\mu}{\sigma}$, est appelé le \textbf{Score Z} (ou cote Z). Ce score Z est une mesure \textit{sans unité} qui indique \textbf{à combien d'écarts-types} une valeur observée $x$ se situe par rapport à la moyenne $\mu$ de sa distribution.
\begin{itemize}
    \item $z = 0$ : $x$ est exactement à la moyenne ($\mathbf{x = \mu}$).
    \item $z = +1$ : $x$ est un écart-type \textit{au-dessus} de la moyenne ($\mathbf{x = \mu + \sigma}$).
    \item $z = -2$ : $x$ est deux écarts-types \textit{en dessous} de la moyenne ($\mathbf{x = \mu - 2\sigma}$).
\end{itemize}
Cette transformation est extrêmement utile pour :
\begin{enumerate}
    \item \textbf{Comparer des valeurs} issues de distributions normales différentes. Un score Z de +1.5 a toujours la même signification relative, que l'on parle de QI, de taille, ou de température.
    \item \textbf{Calculer des probabilités} en utilisant la table unique de la loi $\mathcal{N}(0, 1)$.
\end{enumerate}
\end{intuitionbox}

Un exemple classique est la comparaison de notes.

\begin{examplebox}[Comparaison de Performances]
Un étudiant A obtient 80 points à un examen où la moyenne est $\mu_A=70$ et l'écart-type $\sigma_A=5$. Un étudiant B obtient 85 points à un autre examen où $\mu_B=75$ et $\sigma_B=10$. Qui a le mieux réussi relativement à son groupe ?

Calculons les Z-scores :
$$ Z_A = \frac{80 - 70}{5} = \frac{10}{5} = +2.0 $$
$$ Z_B = \frac{85 - 75}{10} = \frac{10}{10} = +1.0 $$
L'étudiant A a un score Z plus élevé (+2.0 contre +1.0), ce qui signifie qu'il se situe plus d'écarts-types au-dessus de la moyenne de son groupe que l'étudiant B. L'étudiant A a donc relativement mieux réussi.
\end{examplebox}

\subsection{Propriétés Importantes de la Loi Normale}

La loi normale possède des propriétés de stabilité remarquables sous certaines transformations.

\begin{theorembox}[Stabilité par Transformation Linéaire]
Si $X \sim \mathcal{N}(\mu, \sigma^2)$ et $Y = aX + b$ (avec $a \neq 0$), alors $Y$ suit aussi une loi normale :
$$ Y \sim \mathcal{N}(a\mu + b, \, (a\sigma)^2) $$
L'espérance est transformée linéairement ($E[aX+b] = aE[X]+b$), et la variance est multipliée par $a^2$ ($\text{Var}(aX+b) = a^2\text{Var}(X)$).
\end{theorembox}

\begin{proofbox}
Nous utilisons le fait que si $X \sim \mathcal{N}(\mu, \sigma^2)$, alors $Z = (X-\mu)/\sigma \sim \mathcal{N}(0,1)$.
Exprimons $X$ en fonction de $Z$ : $X = \mu + \sigma Z$.
Substituons cela dans l'expression de $Y$:
$$ Y = a(\mu + \sigma Z) + b = (a\mu + b) + (a\sigma)Z $$
Posons $\mu_Y = a\mu + b$ et $\sigma_Y = |a|\sigma$. Alors $Y = \mu_Y + \sigma_Y Z$ (si $a>0$) ou $Y = \mu_Y - \sigma_Y Z$ (si $a<0$).
Dans les deux cas, $Y$ est une transformation linéaire d'une variable normale standard $Z$.
La CDF de $Y$ peut être exprimée en termes de la CDF $\Phi$ de $Z$.
Si $a>0$ :
$$ P(Y \le y) = P(\mu_Y + a\sigma Z \le y) = P(a\sigma Z \le y - \mu_Y) = P\left( Z \le \frac{y - \mu_Y}{a\sigma} \right) = \Phi\left(\frac{y - \mu_Y}{a\sigma}\right) $$
C'est la CDF d'une loi $\mathcal{N}(\mu_Y, (a\sigma)^2)$.
Le cas $a<0$ est similaire et mène au même résultat pour la distribution (la variance dépend de $a^2$).
Ainsi, $Y \sim \mathcal{N}(a\mu + b, (a\sigma)^2)$.
\end{proofbox}

Cette propriété est très utile pour les changements d'unités.

\begin{examplebox}[Changement d'Unités]
Si la température en Celsius $T_C$ suit $\mathcal{N}(20, 5^2)$, quelle est la loi de la température en Fahrenheit $T_F = \frac{9}{5}T_C + 32$ ?

$a = 9/5$, $b=32$.

Nouvelle moyenne : $E[T_F] = \frac{9}{5}(20) + 32 = 36 + 32 = 68$.

Nouvel écart-type : $\sigma_{T_F} = |a|\sigma_{T_C} = \frac{9}{5}(5) = 9$. Nouvelle variance : $\sigma_{T_F}^2 = 9^2 = 81$.

Donc, $T_F \sim \mathcal{N}(68, 9^2)$.
\end{examplebox}

Une autre propriété cruciale concerne la somme de variables normales indépendantes.

\begin{theorembox}[Stabilité par Addition (Indépendance)]
Si $X \sim \mathcal{N}(\mu_X, \sigma_X^2)$ et $Y \sim \mathcal{N}(\mu_Y, \sigma_Y^2)$ sont des variables aléatoires \textbf{indépendantes}, alors leur somme $S = X + Y$ suit aussi une loi normale :
$$ S \sim \mathcal{N}(\mu_X + \mu_Y, \, \sigma_X^2 + \sigma_Y^2) $$
Les moyennes s'ajoutent, et (grâce à l'indépendance) les variances s'ajoutent.
\end{theorembox}

La preuve formelle de ce théorème est plus avancée et utilise généralement les fonctions caractéristiques ou les fonctions génératrices des moments.

\begin{proofbox}[Idée de la preuve (via Fonctions Caractéristiques)]
La fonction caractéristique $\varphi_X(t)$ d'une variable aléatoire $X$ est définie comme $\varphi_X(t) = E[e^{itX}]$.
Pour une loi normale $X \sim \mathcal{N}(\mu, \sigma^2)$, sa fonction caractéristique est $\varphi_X(t) = e^{i\mu t - \frac{1}{2}\sigma^2 t^2}$.
Si $X$ et $Y$ sont indépendantes, la fonction caractéristique de leur somme $S=X+Y$ est le produit de leurs fonctions caractéristiques : $\varphi_S(t) = \varphi_X(t) \varphi_Y(t)$.
\begin{align*}
\varphi_S(t) &= \left( e^{i\mu_X t - \frac{1}{2}\sigma_X^2 t^2} \right) \left( e^{i\mu_Y t - \frac{1}{2}\sigma_Y^2 t^2} \right) \\
&= e^{i(\mu_X + \mu_Y)t - \frac{1}{2}(\sigma_X^2 + \sigma_Y^2)t^2}
\end{align*}
On reconnaît ici la fonction caractéristique d'une loi normale avec pour moyenne $\mu_X + \mu_Y$ et pour variance $\sigma_X^2 + \sigma_Y^2$. Comme la fonction caractéristique détermine de manière unique la distribution, on conclut que $S \sim \mathcal{N}(\mu_X + \mu_Y, \sigma_X^2 + \sigma_Y^2)$.
\end{proofbox}

Il est essentiel de se souvenir de la condition d'indépendance pour l'addition des variances.

\begin{remarquebox}[Attention à l'Indépendance]
La propriété d'addition des variances ($\sigma_S^2 = \sigma_X^2 + \sigma_Y^2$) est cruciale et ne tient \textbf{que si $X$ et $Y$ sont indépendantes}. Si elles ne le sont pas, la variance de la somme inclut un terme de covariance : $\text{Var}(X+Y) = \text{Var}(X) + \text{Var}(Y) + 2\text{Cov}(X, Y)$. Cependant, la somme de variables normales (même dépendantes) reste normale (si elles sont conjointement normales).
\end{remarquebox}

Appliquons ce théorème à un exemple concret.

\begin{examplebox}[Poids Total]
Le poids d'une pomme suit $\mathcal{N}(150g, 10^2)$. Le poids d'une orange suit $\mathcal{N}(200g, 15^2)$. On suppose les poids indépendants. Quel est la loi du poids total d'une pomme et d'une orange ?

Soit $P$ le poids de la pomme, $O$ celui de l'orange. $T = P+O$.

$E[T] = E[P] + E[O] = 150 + 200 = 350g$.

$\text{Var}(T) = \text{Var}(P) + \text{Var}(O) = 10^2 + 15^2 = 100 + 225 = 325$.

Donc, $T \sim \mathcal{N}(350, 325)$. L'écart-type du poids total est $\sqrt{325} \approx 18.03g$.
\end{examplebox}

\subsection{La Règle Empirique (68-95-99.7)}

Une conséquence directe des aires sous la courbe normale standard est une règle approximative très utile.

\begin{theorembox}[Règle Empirique]
Pour toute variable $X \sim \mathcal{N}(\mu, \sigma^2)$ :
\begin{itemize}
    \item $P(\mu - \sigma \le X \le \mu + \sigma) \approx 0.6827$ (Environ \textbf{68\%} des valeurs dans $\mu \pm \sigma$).
    \item $P(\mu - 2\sigma \le X \le \mu + 2\sigma) \approx 0.9545$ (Environ \textbf{95\%} des valeurs dans $\mu \pm 2\sigma$).
    \item $P(\mu - 3\sigma \le X \le \mu + 3\sigma) \approx 0.9973$ (Environ \textbf{99.7\%} des valeurs dans $\mu \pm 3\sigma$).
\end{itemize}
\end{theorembox}

\begin{proofbox}[Dérivation à partir de $\Phi(z)$]
Ces valeurs sont obtenues en calculant les aires sous la PDF de la loi normale standard $\mathcal{N}(0, 1)$ entre les Z-scores correspondants.
\begin{itemize}
    \item $P(-1 \le Z \le 1) = \Phi(1) - \Phi(-1) = \Phi(1) - (1 - \Phi(1)) = 2\Phi(1) - 1$.
    Avec $\Phi(1) \approx 0.8413$, on obtient $2(0.8413) - 1 \approx 0.6826$.
    \item $P(-2 \le Z \le 2) = \Phi(2) - \Phi(-2) = 2\Phi(2) - 1$.
    Avec $\Phi(2) \approx 0.9772$, on obtient $2(0.9772) - 1 \approx 0.9544$.
    \item $P(-3 \le Z \le 3) = \Phi(3) - \Phi(-3) = 2\Phi(3) - 1$.
    Avec $\Phi(3) \approx 0.99865$, on obtient $2(0.99865) - 1 \approx 0.9973$.
\end{itemize}
Ces valeurs sont souvent arrondies à 68%, 95%, et 99.7% pour faciliter la mémorisation.
\end{proofbox}

Cette règle fournit des repères très pratiques.

\begin{intuitionbox}[Repères Essentiels sur la Cloche]
Cette règle, dérivée directement des aires sous la courbe $\mathcal{N}(0, 1)$ entre $z=\pm 1$, $z=\pm 2$ et $z=\pm 3$, fournit des repères extrêmement utiles pour interpréter l'écart-type $\sigma$. Elle nous dit où se trouve la grande majorité des données.


Une observation qui tombe en dehors de l'intervalle $\mu \pm 3\sigma$ est très inhabituelle (elle n'a que 0.3% de chances de se produire). C'est souvent considéré comme une \textit{valeur aberrante} (outlier) potentielle.
\end{intuitionbox}

\subsection{Calcul de Probabilités Normales}

En pratique, pour calculer une probabilité $P(a \le X \le b)$ pour une loi $\mathcal{N}(\mu, \sigma^2)$, on utilise systématiquement la standardisation.

\begin{examplebox}[Utilisation du Z-score]
Supposons que le QI d'une population suit $\mathcal{N}(100, 15^2)$. Quelle est la probabilité $P(X > 130)$ ?

1.  \textbf{Standardiser :} $z = \frac{130 - 100}{15} = 2$. On cherche $P(Z > 2)$.
2.  \textbf{Utiliser la CDF Standard :} $P(Z > 2) = 1 - P(Z \le 2) = 1 - \Phi(2)$.
3.  \textbf{Chercher dans la table / Calculer :} $\Phi(2) \approx 0.9772$.
4.  \textbf{Résultat :} $P(X > 130) = 1 - 0.9772 = 0.0228$. Environ 2.3% de la population a un QI supérieur à 130.
\end{examplebox}

Pour les intervalles, on utilise la propriété $P(a \le Z \le b) = \Phi(b) - \Phi(a)$.

\begin{examplebox}[Probabilité entre deux valeurs]
Quelle est la probabilité $P(85 \le X \le 115)$ ? ($\mu=100, \sigma=15$)

1.  \textbf{Standardiser :} $z_1 = \frac{85 - 100}{15} = -1$, $z_2 = \frac{115 - 100}{15} = 1$. On cherche $P(-1 \le Z \le 1)$.
2.  \textbf{Utiliser la CDF Standard :} $P(-1 \le Z \le 1) = \Phi(1) - \Phi(-1)$.
3.  \textbf{Utiliser la symétrie :} $\Phi(-z) = 1 - \Phi(z)$. Donc $\Phi(-1) = 1 - \Phi(1)$.
    $P(-1 \le Z \le 1) = \Phi(1) - (1 - \Phi(1)) = 2\Phi(1) - 1$.
4.  \textbf{Chercher dans la table / Calculer :} $\Phi(1) \approx 0.8413$.
5.  \textbf{Résultat :} $P(85 \le X \le 115) \approx 2(0.8413) - 1 = 1.6826 - 1 = 0.6826$. (On retrouve la règle des 68% !)
\end{examplebox}

On peut aussi inverser le processus : trouver la valeur $x$ correspondant à une probabilité donnée.

\begin{examplebox}[Trouver une valeur pour une probabilité donnée (Problème Inverse)]
Quel est le QI minimum requis pour être dans le top 10\% de la population ? ($\mu=100, \sigma=15$).

1.  \textbf{Trouver le Z-score correspondant :} On cherche $x$ tel que $P(X > x) = 0.10$. Cela équivaut à $P(Z > z) = 0.10$, où $z = (x-100)/15$.
    Si $P(Z > z) = 0.10$, alors $P(Z \le z) = \Phi(z) = 1 - 0.10 = 0.90$.
2.  \textbf{Chercher dans la table inverse / Calculer :} On cherche la valeur $z$ pour laquelle l'aire à gauche est 0.90 (le 90ème percentile). On trouve $z \approx 1.28$.
3.  \textbf{Convertir en X :} On utilise la relation $z = (x-\mu)/\sigma$ pour trouver $x$:
    $1.28 = \frac{x - 100}{15}$
    $x = 100 + 1.28 \times 15 = 100 + 19.2 = 119.2$.
    Il faut un QI d'environ 119.2 pour être dans le top 10\%.
\end{examplebox}
\newpage
\section{La Loi Normale (ou Gaussienne)}

\subsection{Introduction et Fonction de Densité (PDF)}

Après les lois discrètes et les lois continues de base (Uniforme, Exponentielle), nous abordons la distribution la plus célèbre et la plus utilisée en probabilités et statistiques.

\begin{definitionbox}[Loi Normale]
Une variable aléatoire continue $X$ suit une \textbf{loi normale} (ou loi de Gauss) de paramètres $\mu$ (l'espérance) et $\sigma^2$ (la variance), notée $X \sim \mathcal{N}(\mu, \sigma^2)$, si sa fonction de densité de probabilité (PDF) est donnée par :
$$ f(x; \mu, \sigma) = \frac{1}{\sigma \sqrt{2\pi}} e^{ -\frac{1}{2} \left( \frac{x-\mu}{\sigma} \right)^2 } $$
pour tout $x \in (-\infty, \infty)$, où $\sigma > 0$.
\end{definitionbox}

Cette formule, bien qu'imposante, décrit une forme très familière : la courbe en cloche.

\begin{intuitionbox}[La Courbe en Cloche]
La loi normale est sans doute la distribution la plus importante en probabilités et statistiques. Pourquoi ? Parce qu'elle modélise remarquablement bien de nombreux phénomènes naturels et processus aléatoires où les valeurs tendent à se regrouper autour d'une moyenne, avec des écarts symétriques devenant de plus en plus rares à mesure qu'on s'éloigne de cette moyenne. Pensez à la taille des individus dans une population, aux erreurs de mesure répétées, ou même aux notes d'un grand groupe d'étudiants à un examen bien conçu. 

Sa densité a une forme caractéristique de \textbf{cloche symétrique} :
\begin{itemize}
    \item \textbf{Le Centre ($\mu$)} : Le paramètre $\mu$ représente l'\textbf{espérance} (la moyenne) de la distribution. C'est le centre de symétrie de la courbe, là où la cloche atteint son \textbf{sommet}. C'est la valeur la plus probable (le mode) et aussi la valeur qui coupe la distribution en deux moitiés égales (la médiane). Changer $\mu$ \textit{translate} la cloche horizontalement sans changer sa forme.
    \item \textbf{La Dispersion ($\sigma$)} : Le paramètre $\sigma$ est l'\textbf{écart-type} ($\sigma^2$ est la variance). Il mesure la \textbf{dispersion} des valeurs autour de la moyenne $\mu$. Géométriquement, $\sigma$ contrôle la \textbf{largeur} de la cloche.
        \begin{itemize}
            \item Un \textit{petit} $\sigma$ signifie que les données sont très concentrées autour de la moyenne, donnant une cloche \textbf{étroite et pointue}.
            \item Un \textit{grand} $\sigma$ signifie que les données sont plus étalées, donnant une cloche \textbf{large et aplatie}.
        \end{itemize}
    Les points d'inflexion de la courbe (là où la courbure change de sens) se situent exactement à $\mu \pm \sigma$.
\end{itemize}

\tcblower
\centering
\begin{tikzpicture}
    \begin{axis}[
        title={La Courbe en Cloche (PDF de la Loi Normale)},
        xlabel={$x$},
        ylabel={$f(x)$},
        axis lines=middle,
        no markers,
        samples=100,
        domain=-4:4,
        height=8cm,
        width=\linewidth-1cm,
        tick label style={font=\tiny},
        legend style={at={(0.5,-0.15)}, anchor=north, font=\small},
        legend columns=2
    ]
    % N(0, 1)
    \addplot [blue, ultra thick] {1/(sqrt(2*pi))*exp(-x^2/2)};
    \addlegendentry{$\mu=0, \sigma=1$};
    % N(0, 0.25) => sigma=0.5
    \addplot [red, ultra thick] {1/(0.5*sqrt(2*pi))*exp(-x^2/(2*0.5^2))};
    \addlegendentry{$\mu=0, \sigma=0.5$ (étroite)};
    % N(1, 2.25) => sigma=1.5
    \addplot [green!70!black, ultra thick] {1/(1.5*sqrt(2*pi))*exp(-(x-1)^2/(2*1.5^2))};
    \addlegendentry{$\mu=1, \sigma=1.5$ (large, décalée)};

    \draw [dashed] (axis cs:0,0) -- (axis cs:0, {1/(sqrt(2*pi))}) node[above, font=\tiny] {pic à $\mu=0$}; % Ligne pour mu=0
    \draw [dashed] (axis cs:1,0) -- (axis cs:1, {1/(1.5*sqrt(2*pi))}) node[above right, font=\tiny] {pic à $\mu=1$}; % Ligne pour mu=1
    \end{axis}
\end{tikzpicture}
\par\small\textit{Influence de $\mu$ (position) et $\sigma$ (largeur) sur la forme de la cloche.}
\end{intuitionbox}

Mais d'où vient cette formule spécifique ? Il existe une dérivation fascinante à partir d'hypothèses fondamentales sur les erreurs aléatoires (argument d'Herschel-Maxwell).

\begin{proofbox}[Dérivation de la Densité Normale à partir des Principes Fondamentaux]

\textbf{Contexte Visuel :} Imaginons un nuage de points dispersés autour d'une cible à l'origine $(0,0)$, comme des impacts de fléchettes. Le graphique ci-dessous illustre cette dispersion. On s'intéresse à la probabilité de tomber dans une petite zone, comme $dA$, autour d'un point $(x, y)$.

\begin{center}
\begin{tikzpicture}
\begin{axis}[
    axis lines=middle, % Axes qui passent par l'origine (0,0)
    xlabel=$x$,       % Étiquette axe X
    ylabel=$y$,       % Étiquette axe Y
    axis line style={magenta}, % Couleur des axes
    xlabel style={anchor=west, magenta}, % Style de l'étiquette X
    ylabel style={anchor=south, magenta}, % Style de l'étiquette Y
    xmin=-3.5, xmax=3.5, % Limites du graphique
    ymin=-3.5, ymax=3.5,
    tick label style={font=\tiny} % Police plus petite pour les graduations
]

% Ajout des points du nuage. Ce sont des coordonnées approximatives.
\addplot [only marks, mark=*, cyan, mark size=1.5pt]
coordinates {
    (0.2, 0.1) (-0.5, 0.2) (0.1, -0.3) (0.5, -0.8) (-0.3, -1.2) (0,0.1)
    (1.5, 1.5) (0.8, 0.8) (2.0, -0.5) (2.8, -1.4) (2.5, -0.2)
    (-1.8, -1.3) (-2.5, 0.5) (-1.5, 0.3) (-2.2, -0.8) (-0.8, 0.5)
    (0.5, 1.2) (0.7, 2.8) (0.2, 3.2) (-0.5, 1.5) (-1.0, -2.0)
    (1.8, -1.0) (1.0, -1.5) (0.3, -2.5) (-1.8, -2.8) (1.2, 0.5)
    (-1.2, -1.5) (-0.8, 1.1)
};

% --- Annotations ---

% Points et boîte pour 'dA'
\addplot [only marks, mark=*, red, mark size=1.5pt] coordinates {(1.3, 2.0)};
\draw [red, thick] (axis cs:1.1, 1.8) rectangle (axis cs:1.5, 2.2);
\node [red, above] at (axis cs:1.3, 2.2) {$dA$};

% Points et boîte pour 'dB'
\addplot [only marks, mark=*, cyan, mark size=1.5pt] 
coordinates {(-1.2, 2.0) (-1.3, 2.3) (-1.0, 2.2) (-1.1, 1.9)};
\draw [blue, thick] (axis cs:-1.8, 1.2) rectangle (axis cs:-0.8, 2.8);
\node [blue, above] at (axis cs:-1.3, 2.8) {$dB$};

\end{axis}
\end{tikzpicture}
\end{center}

\textbf{Objectif :} Expliquer comment arriver à la formule mathématique de la courbe en cloche (densité de probabilité normale) en partant de principes fondamentaux sur les erreurs aléatoires.

\textbf{1. Le Point de Départ : Densité et Aire $dA$}
Dans une distribution continue, la probabilité de tomber \textit{exactement} sur un point $(x, y)$ est nulle. On ne peut donc pas parler de "probabilité d'un point". On parle de la probabilité de tomber \textit{dans une petite zone}, comme un rectangle $dA = dx \cdot dy$ autour du point $(x, y)$.
Cette probabilité, notée $P(\text{dans } dA)$, est \textit{proportionnelle} à l'aire de la zone $dA$. La \textit{constante de proportionnalité} est la \textbf{fonction de densité de probabilité} $p(x, y)$ évaluée en ce point. En d'autres termes, la densité $p(x, y)$ \textit{représente} localement la concentration de probabilité. Ainsi, la probabilité de tomber dans la zone $dA$ est approximativement :
$$ P(\text{dans } dA) \approx p(x, y) \cdot dA $$

\textbf{2. Les Hypothèses Fondamentales}
On pose deux hypothèses sur la nature de ces erreurs (représentées par la densité $p(x, y)$) :
\begin{enumerate}
    \item \textbf{Indépendance des axes :} L'erreur horizontale ($x$) est indépendante de l'erreur verticale ($y$). Cela implique que la densité jointe $p(x, y)$ peut s'écrire comme le produit de la densité marginale sur $x$, notée $f(x)$, et de la densité marginale sur $y$, notée $f(y)$. Donc, $p(x, y) = f(x) \cdot f(y)$.
    \item \textbf{Symétrie de rotation (Isotropie) :} La densité ne dépend que de la distance $r = \sqrt{x^2 + y^2}$ au centre, pas de l'angle. Il existe donc une fonction $\phi(r)$ telle que la densité en $(x,y)$ est $p(x, y) = \phi(\sqrt{x^2 + y^2})$.
\end{enumerate}

\textbf{3. L'Équation Fonctionnelle}
En égalant les deux expressions pour la même densité $p(x, y)$ (à une constante près), on obtient :
$$ f(x) \cdot f(y) = \phi(\sqrt{x^2 + y^2}) $$
Pour $y=0$, on a $f(x) \cdot f(0) = \phi(x)$. Posons $f(0) = \lambda$. Alors $\phi(x) = \lambda f(x)$.
L'équation devient :
$$ f(x) \cdot f(y) = \lambda f(\sqrt{x^2 + y^2}) $$

\textbf{4. Résolution de l'Équation Fonctionnelle}
Posons $g(x) = f(x)/\lambda$, avec $g(0)=1$. L'équation se simplifie en :
$$ g(x) g(y) = g(\sqrt{x^2 + y^2}) $$
Posons $g(x) = h(x^2)$. L'équation devient $h(x^2)h(y^2) = h(x^2+y^2)$. Avec $a=x^2$ et $b=y^2$, on a :
$$ h(a) h(b) = h(a+b) $$
La solution continue de cette équation de Cauchy est $h(a) = e^{Aa}$ pour une constante $A$.
Retour aux fonctions : $g(x) = h(x^2) = e^{Ax^2}$. $f(x) = \lambda g(x) = \lambda e^{Ax^2}$.
Comme la densité doit diminuer loin du centre, $A$ doit être négative. Posons $A = -k$ avec $k>0$.
$$ f(x) = \lambda e^{-k x^2} $$

\textbf{5. Normalisation et Identification des Paramètres}
\begin{enumerate}
    \item \textbf{Condition $\int f(x) dx = 1$} : L'intégrale Gaussienne $\int_{-\infty}^{\infty} e^{-k x^2} \, \mathrm{d}x = \sqrt{\frac{\pi}{k}}$.
    Donc, $\int_{-\infty}^{\infty} f(x) dx = \lambda \sqrt{\frac{\pi}{k}} = 1 \implies \lambda = \sqrt{\frac{k}{\pi}}$.
    \item \textbf{Lien avec la Variance ($\sigma^2$)} : Pour une distribution centrée, $\sigma^2 = E[X^2] = \int x^2 f(x) dx$.
    $$ \sigma^2 = \int_{-\infty}^{\infty} x^2 \left( \sqrt{\frac{k}{\pi}} e^{-k x^2} \right) \, \mathrm{d}x = \sqrt{\frac{k}{\pi}} \left( \frac{1}{2k} \sqrt{\frac{\pi}{k}} \right) = \frac{1}{2k} $$
    Donc, $k = \frac{1}{2\sigma^2}$.
    \item \textbf{Substitution Finale :} Remplaçons $k$ dans $\lambda$ et $f(x)$.
    $$ \lambda = \sqrt{\frac{1/(2\sigma^2)}{\pi}} = \frac{1}{\sigma\sqrt{2\pi}} $$
    $$ f(x) = \frac{1}{\sigma\sqrt{2\pi}} e^{-\frac{1}{2\sigma^2} x^2} = \frac{1}{\sigma\sqrt{2\pi}} e^{ -\frac{x^2}{2\sigma^2} } $$
    \item \textbf{Généralisation (Moyenne $\mu$)} : Pour centrer la distribution sur $\mu$, on remplace $x$ par $(x-\mu)$ dans l'exposant :
    $$ f(x; \mu, \sigma) = \frac{1}{\sigma \sqrt{2\pi}} e^{ -\frac{(x-\mu)^2}{2\sigma^2} } $$
\end{enumerate}
C'est la fonction de densité de la loi normale $\mathcal{N}(\mu, \sigma^2)$.
\end{proofbox}

\subsection{La Loi Normale Centrée Réduite $\mathcal{N}(0, 1)$}

Avant d'explorer les propriétés de la loi normale générale, concentrons-nous sur son cas le plus simple et le plus fondamental.

\begin{definitionbox}[Loi Normale Standard (ou Centrée Réduite)]
Un cas particulier extraordinairement utile est la loi normale avec une moyenne $\mu=0$ et une variance $\sigma^2=1$ (donc $\sigma=1$). On l'appelle la \textbf{loi normale standard} ou \textbf{centrée réduite}, et on la note souvent $Z$. Sa PDF est traditionnellement notée $\phi(z)$ :
$$ \phi(z) = \frac{1}{\sqrt{2\pi}} e^{-z^2/2} $$
Sa fonction de répartition (CDF), qui donne $P(Z \le z)$, est notée $\Phi(z)$ :
$$ \Phi(z) = P(Z \le z) = \int_{-\infty}^z \frac{1}{\sqrt{2\pi}} e^{-t^2/2} \, \mathrm{d}t $$
\end{definitionbox}

Pourquoi cette version standard est-elle si importante ? Elle sert de référence universelle.

\begin{intuitionbox}[La Référence Universelle et le Changement d'Unités]
Pourquoi cette loi $\mathcal{N}(0, 1)$ est-elle si centrale ? Imaginez que vous ayez des mesures en degrés Celsius ($\mathcal{N}(\mu_C, \sigma_C^2)$) et d'autres en degrés Fahrenheit ($\mathcal{N}(\mu_F, \sigma_F^2)$). Comment les comparer ? La loi normale standard fournit un \textbf{système d'unités universel}.

Toute variable normale $X \sim \mathcal{N}(\mu, \sigma^2)$ peut être transformée ("standardisée") en une variable $Z \sim \mathcal{N}(0, 1)$ par un simple changement d'échelle et de position : $Z = (X-\mu)/\sigma$. 

Cela signifie qu'au lieu de devoir calculer des aires (probabilités) pour une infinité de courbes en cloche différentes (une pour chaque paire $\mu, \sigma$), on peut tout ramener à \textbf{une seule courbe de référence}, $\mathcal{N}(0, 1)$. Les aires sous cette courbe standard ($\Phi(z)$) ont été calculées une fois pour toutes et sont disponibles dans des tables ou des logiciels. On n'a plus qu'à convertir notre problème dans cette "langue" standard, trouver la probabilité, et interpréter le résultat.
\end{intuitionbox}

La notation est très standardisée pour cette loi.

\begin{remarquebox}[Notation $\phi$ et $\Phi$]
Les symboles $\phi$ (phi minuscule) pour la PDF et $\Phi$ (phi majuscule) pour la CDF de la loi normale standard sont quasi universels. Il est important de ne pas les confondre. $\phi(z)$ est la \textit{hauteur} de la courbe en $z$, tandis que $\Phi(z)$ est l'\textit{aire} sous la courbe à gauche de $z$.
\end{remarquebox}

Un détail technique important concerne le calcul de $\Phi(z)$.

\begin{remarquebox}[Absence de Primitive Simple]
L'intégrale $\int e^{-t^2/2} \, \mathrm{d}t$, nécessaire pour calculer $\Phi(z)$, n'a \textbf{pas d'expression analytique} en termes de fonctions élémentaires (polynômes, exponentielles, log, sin, cos...). C'est une fonction spéciale, connue sous le nom de \textbf{fonction d'erreur} (liée à $\Phi$ par une transformation simple). C'est la raison pour laquelle on dépend de tables ou de calculs numériques pour obtenir les valeurs de $\Phi(z)$. Heureusement, ces outils sont omniprésents aujourd'hui.
\end{remarquebox}

\subsection{Standardisation : Le Score Z}

Formalisons cette transformation clé qui relie toute loi normale à la loi standard.

\begin{theorembox}[Standardisation d'une Variable Normale]
Si $X \sim \mathcal{N}(\mu, \sigma^2)$, alors la variable $Z$ définie par :
$$ Z = \frac{X - \mu}{\sigma} $$
suit la loi normale standard, $Z \sim \mathcal{N}(0, 1)$.
\end{theorembox}

La preuve formelle utilise un changement de variable dans la fonction de répartition.

\begin{proofbox}
Soit $F_X(x)$ la CDF de $X$ et $F_Z(z)$ la CDF de $Z$. Nous voulons montrer que $F_Z(z) = \Phi(z)$.
\begin{align*}
F_Z(z) &= P(Z \le z) \\
&= P\left( \frac{X-\mu}{\sigma} \le z \right) \\
&= P(X - \mu \le z\sigma) \\
&= P(X \le \mu + z\sigma) \\
&= F_X(\mu + z\sigma)
\end{align*}
Par définition de la CDF de $X$ :
$$ F_X(x) = \int_{-\infty}^x \frac{1}{\sigma \sqrt{2\pi}} e^{ -\frac{1}{2} \left( \frac{t-\mu}{\sigma} \right)^2 } \, dt $$
Donc,
$$ F_Z(z) = \int_{-\infty}^{\mu + z\sigma} \frac{1}{\sigma \sqrt{2\pi}} e^{ -\frac{1}{2} \left( \frac{t-\mu}{\sigma} \right)^2 } \, dt $$
Effectuons le changement de variable $u = (t-\mu)/\sigma$. Alors $t = \mu + u\sigma$ et $dt = \sigma du$.
Les bornes d'intégration changent :
\begin{itemize}
    \item Quand $t \to -\infty$, $u \to -\infty$.
    \item Quand $t = \mu + z\sigma$, $u = ((\mu + z\sigma)-\mu)/\sigma = z$.
\end{itemize}
L'intégrale devient :
$$ F_Z(z) = \int_{-\infty}^{z} \frac{1}{\sigma \sqrt{2\pi}} e^{ -\frac{1}{2} u^2 } (\sigma du) $$
$$ F_Z(z) = \int_{-\infty}^{z} \frac{1}{\sqrt{2\pi}} e^{ -u^2/2 } \, du $$
C'est exactement la définition de $\Phi(z)$, la CDF de la loi normale standard. Ainsi, $Z \sim \mathcal{N}(0, 1)$.
\end{proofbox}

Cette transformation a une interprétation très concrète.

\begin{intuitionbox}[Mesurer en "Unités d'Écart-Type"]
Transformer $X$ en $Z$ s'appelle \textbf{standardiser} la variable. Le résultat, $z = \frac{x-\mu}{\sigma}$, est appelé le \textbf{Score Z} (ou cote Z). Ce score Z est une mesure \textit{sans unité} qui indique \textbf{à combien d'écarts-types} une valeur observée $x$ se situe par rapport à la moyenne $\mu$ de sa distribution.
\begin{itemize}
    \item $z = 0$ : $x$ est exactement à la moyenne ($\mathbf{x = \mu}$).
    \item $z = +1$ : $x$ est un écart-type \textit{au-dessus} de la moyenne ($\mathbf{x = \mu + \sigma}$).
    \item $z = -2$ : $x$ est deux écarts-types \textit{en dessous} de la moyenne ($\mathbf{x = \mu - 2\sigma}$).
\end{itemize}
Cette transformation est extrêmement utile pour :
\begin{enumerate}
    \item \textbf{Comparer des valeurs} issues de distributions normales différentes. Un score Z de +1.5 a toujours la même signification relative, que l'on parle de QI, de taille, ou de température.
    \item \textbf{Calculer des probabilités} en utilisant la table unique de la loi $\mathcal{N}(0, 1)$.
\end{enumerate}
\end{intuitionbox}

Un exemple classique est la comparaison de notes.

\begin{examplebox}[Comparaison de Performances]
Un étudiant A obtient 80 points à un examen où la moyenne est $\mu_A=70$ et l'écart-type $\sigma_A=5$. Un étudiant B obtient 85 points à un autre examen où $\mu_B=75$ et $\sigma_B=10$. Qui a le mieux réussi relativement à son groupe ?

Calculons les Z-scores :
$$ Z_A = \frac{80 - 70}{5} = \frac{10}{5} = +2.0 $$
$$ Z_B = \frac{85 - 75}{10} = \frac{10}{10} = +1.0 $$
L'étudiant A a un score Z plus élevé (+2.0 contre +1.0), ce qui signifie qu'il se situe plus d'écarts-types au-dessus de la moyenne de son groupe que l'étudiant B. L'étudiant A a donc relativement mieux réussi.
\end{examplebox}

\subsection{Propriétés Importantes de la Loi Normale}

La loi normale possède des propriétés de stabilité remarquables sous certaines transformations.

\begin{theorembox}[Stabilité par Transformation Linéaire]
Si $X \sim \mathcal{N}(\mu, \sigma^2)$ et $Y = aX + b$ (avec $a \neq 0$), alors $Y$ suit aussi une loi normale :
$$ Y \sim \mathcal{N}(a\mu + b, \, (a\sigma)^2) $$
L'espérance est transformée linéairement ($E[aX+b] = aE[X]+b$), et la variance est multipliée par $a^2$ ($\text{Var}(aX+b) = a^2\text{Var}(X)$).
\end{theorembox}

\begin{proofbox}
Nous utilisons le fait que si $X \sim \mathcal{N}(\mu, \sigma^2)$, alors $Z = (X-\mu)/\sigma \sim \mathcal{N}(0,1)$.
Exprimons $X$ en fonction de $Z$ : $X = \mu + \sigma Z$.
Substituons cela dans l'expression de $Y$:
$$ Y = a(\mu + \sigma Z) + b = (a\mu + b) + (a\sigma)Z $$
Posons $\mu_Y = a\mu + b$ et $\sigma_Y = |a|\sigma$. Alors $Y = \mu_Y + \sigma_Y Z$ (si $a>0$) ou $Y = \mu_Y - \sigma_Y Z$ (si $a<0$).
Dans les deux cas, $Y$ est une transformation linéaire d'une variable normale standard $Z$.
La CDF de $Y$ peut être exprimée en termes de la CDF $\Phi$ de $Z$.
Si $a>0$ :
$$ P(Y \le y) = P(\mu_Y + a\sigma Z \le y) = P(a\sigma Z \le y - \mu_Y) = P\left( Z \le \frac{y - \mu_Y}{a\sigma} \right) = \Phi\left(\frac{y - \mu_Y}{a\sigma}\right) $$
C'est la CDF d'une loi $\mathcal{N}(\mu_Y, (a\sigma)^2)$.
Le cas $a<0$ est similaire et mène au même résultat pour la distribution (la variance dépend de $a^2$).
Ainsi, $Y \sim \mathcal{N}(a\mu + b, (a\sigma)^2)$.
\end{proofbox}

Cette propriété est très utile pour les changements d'unités.

\begin{examplebox}[Changement d'Unités]
Si la température en Celsius $T_C$ suit $\mathcal{N}(20, 5^2)$, quelle est la loi de la température en Fahrenheit $T_F = \frac{9}{5}T_C + 32$ ?

$a = 9/5$, $b=32$.

Nouvelle moyenne : $E[T_F] = \frac{9}{5}(20) + 32 = 36 + 32 = 68$.

Nouvel écart-type : $\sigma_{T_F} = |a|\sigma_{T_C} = \frac{9}{5}(5) = 9$. Nouvelle variance : $\sigma_{T_F}^2 = 9^2 = 81$.

Donc, $T_F \sim \mathcal{N}(68, 9^2)$.
\end{examplebox}

Une autre propriété cruciale concerne la somme de variables normales indépendantes.

\begin{theorembox}[Stabilité par Addition (Indépendance)]
Si $X \sim \mathcal{N}(\mu_X, \sigma_X^2)$ et $Y \sim \mathcal{N}(\mu_Y, \sigma_Y^2)$ sont des variables aléatoires \textbf{indépendantes}, alors leur somme $S = X + Y$ suit aussi une loi normale :
$$ S \sim \mathcal{N}(\mu_X + \mu_Y, \, \sigma_X^2 + \sigma_Y^2) $$
Les moyennes s'ajoutent, et (grâce à l'indépendance) les variances s'ajoutent.
\end{theorembox}

La preuve formelle de ce théorème est plus avancée et utilise généralement les fonctions caractéristiques ou les fonctions génératrices des moments.

\begin{proofbox}[Idée de la preuve (via Fonctions Caractéristiques)]
La fonction caractéristique $\varphi_X(t)$ d'une variable aléatoire $X$ est définie comme $\varphi_X(t) = E[e^{itX}]$.
Pour une loi normale $X \sim \mathcal{N}(\mu, \sigma^2)$, sa fonction caractéristique est $\varphi_X(t) = e^{i\mu t - \frac{1}{2}\sigma^2 t^2}$.
Si $X$ et $Y$ sont indépendantes, la fonction caractéristique de leur somme $S=X+Y$ est le produit de leurs fonctions caractéristiques : $\varphi_S(t) = \varphi_X(t) \varphi_Y(t)$.
\begin{align*}
\varphi_S(t) &= \left( e^{i\mu_X t - \frac{1}{2}\sigma_X^2 t^2} \right) \left( e^{i\mu_Y t - \frac{1}{2}\sigma_Y^2 t^2} \right) \\
&= e^{i(\mu_X + \mu_Y)t - \frac{1}{2}(\sigma_X^2 + \sigma_Y^2)t^2}
\end{align*}
On reconnaît ici la fonction caractéristique d'une loi normale avec pour moyenne $\mu_X + \mu_Y$ et pour variance $\sigma_X^2 + \sigma_Y^2$. Comme la fonction caractéristique détermine de manière unique la distribution, on conclut que $S \sim \mathcal{N}(\mu_X + \mu_Y, \sigma_X^2 + \sigma_Y^2)$.
\end{proofbox}

Il est essentiel de se souvenir de la condition d'indépendance pour l'addition des variances.

\begin{remarquebox}[Attention à l'Indépendance]
La propriété d'addition des variances ($\sigma_S^2 = \sigma_X^2 + \sigma_Y^2$) est cruciale et ne tient \textbf{que si $X$ et $Y$ sont indépendantes}. Si elles ne le sont pas, la variance de la somme inclut un terme de covariance : $\text{Var}(X+Y) = \text{Var}(X) + \text{Var}(Y) + 2\text{Cov}(X, Y)$. Cependant, la somme de variables normales (même dépendantes) reste normale (si elles sont conjointement normales).
\end{remarquebox}

Appliquons ce théorème à un exemple concret.

\begin{examplebox}[Poids Total]
Le poids d'une pomme suit $\mathcal{N}(150g, 10^2)$. Le poids d'une orange suit $\mathcal{N}(200g, 15^2)$. On suppose les poids indépendants. Quel est la loi du poids total d'une pomme et d'une orange ?

Soit $P$ le poids de la pomme, $O$ celui de l'orange. $T = P+O$.

$E[T] = E[P] + E[O] = 150 + 200 = 350g$.

$\text{Var}(T) = \text{Var}(P) + \text{Var}(O) = 10^2 + 15^2 = 100 + 225 = 325$.

Donc, $T \sim \mathcal{N}(350, 325)$. L'écart-type du poids total est $\sqrt{325} \approx 18.03g$.
\end{examplebox}

\subsection{La Règle Empirique (68-95-99.7)}

Une conséquence directe des aires sous la courbe normale standard est une règle approximative très utile.

\begin{theorembox}[Règle Empirique]
Pour toute variable $X \sim \mathcal{N}(\mu, \sigma^2)$ :
\begin{itemize}
    \item $P(\mu - \sigma \le X \le \mu + \sigma) \approx 0.6827$ (Environ \textbf{68\%} des valeurs dans $\mu \pm \sigma$).
    \item $P(\mu - 2\sigma \le X \le \mu + 2\sigma) \approx 0.9545$ (Environ \textbf{95\%} des valeurs dans $\mu \pm 2\sigma$).
    \item $P(\mu - 3\sigma \le X \le \mu + 3\sigma) \approx 0.9973$ (Environ \textbf{99.7\%} des valeurs dans $\mu \pm 3\sigma$).
\end{itemize}
\end{theorembox}

\begin{proofbox}[Dérivation à partir de $\Phi(z)$]
Ces valeurs sont obtenues en calculant les aires sous la PDF de la loi normale standard $\mathcal{N}(0, 1)$ entre les Z-scores correspondants.
\begin{itemize}
    \item $P(-1 \le Z \le 1) = \Phi(1) - \Phi(-1) = \Phi(1) - (1 - \Phi(1)) = 2\Phi(1) - 1$.
    Avec $\Phi(1) \approx 0.8413$, on obtient $2(0.8413) - 1 \approx 0.6826$.
    \item $P(-2 \le Z \le 2) = \Phi(2) - \Phi(-2) = 2\Phi(2) - 1$.
    Avec $\Phi(2) \approx 0.9772$, on obtient $2(0.9772) - 1 \approx 0.9544$.
    \item $P(-3 \le Z \le 3) = \Phi(3) - \Phi(-3) = 2\Phi(3) - 1$.
    Avec $\Phi(3) \approx 0.99865$, on obtient $2(0.99865) - 1 \approx 0.9973$.
\end{itemize}
Ces valeurs sont souvent arrondies à 68%, 95%, et 99.7% pour faciliter la mémorisation.
\end{proofbox}

Cette règle fournit des repères très pratiques.

\begin{intuitionbox}[Repères Essentiels sur la Cloche]
Cette règle, dérivée directement des aires sous la courbe $\mathcal{N}(0, 1)$ entre $z=\pm 1$, $z=\pm 2$ et $z=\pm 3$, fournit des repères extrêmement utiles pour interpréter l'écart-type $\sigma$. Elle nous dit où se trouve la grande majorité des données.


Une observation qui tombe en dehors de l'intervalle $\mu \pm 3\sigma$ est très inhabituelle (elle n'a que 0.3% de chances de se produire). C'est souvent considéré comme une \textit{valeur aberrante} (outlier) potentielle.
\end{intuitionbox}

\subsection{Calcul de Probabilités Normales}

En pratique, pour calculer une probabilité $P(a \le X \le b)$ pour une loi $\mathcal{N}(\mu, \sigma^2)$, on utilise systématiquement la standardisation.

\begin{examplebox}[Utilisation du Z-score]
Supposons que le QI d'une population suit $\mathcal{N}(100, 15^2)$. Quelle est la probabilité $P(X > 130)$ ?

1.  \textbf{Standardiser :} $z = \frac{130 - 100}{15} = 2$. On cherche $P(Z > 2)$.
2.  \textbf{Utiliser la CDF Standard :} $P(Z > 2) = 1 - P(Z \le 2) = 1 - \Phi(2)$.
3.  \textbf{Chercher dans la table / Calculer :} $\Phi(2) \approx 0.9772$.
4.  \textbf{Résultat :} $P(X > 130) = 1 - 0.9772 = 0.0228$. Environ 2.3% de la population a un QI supérieur à 130.
\end{examplebox}

Pour les intervalles, on utilise la propriété $P(a \le Z \le b) = \Phi(b) - \Phi(a)$.

\begin{examplebox}[Probabilité entre deux valeurs]
Quelle est la probabilité $P(85 \le X \le 115)$ ? ($\mu=100, \sigma=15$)

1.  \textbf{Standardiser :} $z_1 = \frac{85 - 100}{15} = -1$, $z_2 = \frac{115 - 100}{15} = 1$. On cherche $P(-1 \le Z \le 1)$.
2.  \textbf{Utiliser la CDF Standard :} $P(-1 \le Z \le 1) = \Phi(1) - \Phi(-1)$.
3.  \textbf{Utiliser la symétrie :} $\Phi(-z) = 1 - \Phi(z)$. Donc $\Phi(-1) = 1 - \Phi(1)$.
    $P(-1 \le Z \le 1) = \Phi(1) - (1 - \Phi(1)) = 2\Phi(1) - 1$.
4.  \textbf{Chercher dans la table / Calculer :} $\Phi(1) \approx 0.8413$.
5.  \textbf{Résultat :} $P(85 \le X \le 115) \approx 2(0.8413) - 1 = 1.6826 - 1 = 0.6826$. (On retrouve la règle des 68% !)
\end{examplebox}

On peut aussi inverser le processus : trouver la valeur $x$ correspondant à une probabilité donnée.

\begin{examplebox}[Trouver une valeur pour une probabilité donnée (Problème Inverse)]
Quel est le QI minimum requis pour être dans le top 10\% de la population ? ($\mu=100, \sigma=15$).

1.  \textbf{Trouver le Z-score correspondant :} On cherche $x$ tel que $P(X > x) = 0.10$. Cela équivaut à $P(Z > z) = 0.10$, où $z = (x-100)/15$.
    Si $P(Z > z) = 0.10$, alors $P(Z \le z) = \Phi(z) = 1 - 0.10 = 0.90$.
2.  \textbf{Chercher dans la table inverse / Calculer :} On cherche la valeur $z$ pour laquelle l'aire à gauche est 0.90 (le 90ème percentile). On trouve $z \approx 1.28$.
3.  \textbf{Convertir en X :} On utilise la relation $z = (x-\mu)/\sigma$ pour trouver $x$:
    $1.28 = \frac{x - 100}{15}$
    $x = 100 + 1.28 \times 15 = 100 + 19.2 = 119.2$.
    Il faut un QI d'environ 119.2 pour être dans le top 10\%.
\end{examplebox}

\subsection{Exercices}

% --- PDF, CDF et Loi Normale Standard ---

\begin{exercicebox}[Exercice 1 : Concepts de Base $\Phi(z)$]
Soit $Z \sim \mathcal{N}(0, 1)$ la loi normale standard. Sa CDF est $\Phi(z)$.
Exprimez les probabilités suivantes en termes de $\Phi(z)$ :
\begin{enumerate}
    \item $P(Z \le 1.5)$
    \item $P(Z > 1)$
    \item $P(Z \le -1.5)$ (Indice : utilisez la symétrie $\Phi(-z) = 1 - \Phi(z)$)
    \item $P(-1.5 \le Z \le 1.5)$
\end{enumerate}
\end{exercicebox}

\begin{exercicebox}[Exercice 2 : Utilisation d'une Table $\Phi(z)$]
En utilisant une table ou une calculatrice pour la loi $\mathcal{N}(0, 1)$, on sait que $\Phi(1) \approx 0.8413$, $\Phi(1.96) \approx 0.975$ et $\Phi(2) \approx 0.9772$.
Calculez :
\begin{enumerate}
    \item $P(Z > 1)$
    \item $P(Z \le -2)$
    \item $P(-1.96 \le Z \le 1.96)$
\end{enumerate}
\end{exercicebox}

\begin{exercicebox}[Exercice 3 : Propriétés de la PDF $\phi(z)$]
Soit $\phi(z)$ la PDF de la loi $\mathcal{N}(0, 1)$.
\begin{enumerate}
    \item Quelle est la valeur de $\phi(0)$ ? (Le pic de la courbe).
    \item Que vaut $\phi(z)$ par rapport à $\phi(-z)$ ?
    \item Que vaut $\int_{-\infty}^{\infty} \phi(z) \, dz$ ?
\end{enumerate}
\end{exercicebox}

% --- Standardisation (Z-score) et Calcul de Probabilités ---

\begin{exercicebox}[Exercice 4 : Calcul de Z-scores]
Une variable aléatoire $X$ suit une loi normale $\mathcal{N}(\mu=50, \sigma^2=100)$. Notez que $\sigma=10$.
Calculez le Z-score pour les valeurs suivantes de $X$ :
\begin{enumerate}
    \item $x = 60$
    \item $x = 50$
    \item $x = 35$
\end{enumerate}
\end{exercicebox}

\begin{exercicebox}[Exercice 5 : Calcul de Probabilité (Général)]
La taille des hommes adultes dans un pays suit une loi normale $\mathcal{N}(175 \text{ cm}, 7^2 \text{ cm}^2)$.
Soit $X$ la taille d'un homme choisi au hasard. Calculez :
\begin{enumerate}
    \item $P(X \le 182 \text{ cm})$ (Indice : Standardisez $x=182$ et utilisez $\Phi(1) \approx 0.8413$)
    \item $P(X > 168 \text{ cm})$
\end{enumerate}
\end{exercicebox}

\begin{exercicebox}[Exercice 6 : Calcul de Probabilité (Intervalle)]
Les scores à un test de QI suivent une loi normale $\mathcal{N}(100, 15^2)$.
Quelle est la probabilité qu'une personne choisie au hasard ait un QI compris entre 85 et 115 ?
(Indice : Standardisez les deux bornes).
\end{exercicebox}

\begin{exercicebox}[Exercice 7 : Calcul de Probabilité (Queue Extrême)]
En utilisant la même loi $\mathcal{N}(100, 15^2)$ pour le QI :
Quelle est la probabilité qu'une personne ait un QI supérieur à 130 ?
(Indice : Utilisez $\Phi(2) \approx 0.9772$).
\end{exercicebox}

% --- Problèmes Inverses (Trouver x) ---

\begin{exercicebox}[Exercice 8 : Problème Inverse (Z-score)]
Soit $Z \sim \mathcal{N}(0, 1)$. Trouvez la valeur $z$ telle que :
(Utilisez $\Phi(1.28) \approx 0.90$ et $\Phi(1.645) \approx 0.95$)
\begin{enumerate}
    \item $P(Z \le z) = 0.90$
    \item $P(Z > z) = 0.05$ (Indice : si $P(Z>z)=0.05$, que vaut $P(Z \le z)$ ?)
    \item $P(Z \le z) = 0.10$ (Indice : utilisez la symétrie)
\end{enumerate}
\end{exercicebox}

\begin{exercicebox}[Exercice 9 : Problème Inverse (Général)]
Les scores au test $\mathcal{N}(100, 15^2)$ sont utilisés pour sélectionner des candidats. Seul le top 5\% des scores est accepté.
Quel est le score minimum requis pour être accepté ?
(Indice : Utilisez $z \approx 1.645$ pour le top 5\%).
\end{exercicebox}

\begin{exercicebox}[Exercice 10 : Problème Inverse (Intervalle Central)]
Soit $Z \sim \mathcal{N}(0, 1)$. Trouvez la valeur $z$ telle que $P(-z \le Z \le z) = 0.95$.
(Indice : si 95\% est au centre, combien reste-t-il dans chaque queue ? Utilisez $\Phi(1.96) \approx 0.975$).
\end{exercicebox}

\begin{exercicebox}[Exercice 11 : Problème Inverse (Général)]
La durée de vie d'une batterie suit $\mathcal{N}(500 \text{ heures}, 50^2 \text{ heures}^2)$.
Le fabricant veut offrir une garantie. Il ne veut remplacer que 2.5\% des batteries.
Quelle durée de garantie (en heures) doit-il proposer ?
(Indice : $P(Z \le -1.96) \approx 0.025$).
\end{exercicebox}

% --- Règle Empirique (68-95-99.7) ---

\begin{exercicebox}[Exercice 12 : Règle Empirique (Application)]
Le poids de paquets de café suit $\mathcal{N}(250g, 5^2g^2)$.
En utilisant la règle empirique (68-95-99.7), donnez un intervalle qui contient :
\begin{enumerate}
    \item Environ 68\% des poids.
    \item Environ 95\% des poids.
    \item Environ 99.7\% des poids.
\end{enumerate}
\end{exercicebox}

\begin{exercicebox}[Exercice 13 : Règle Empirique (Probabilité)]
En utilisant la situation de l'exercice 12 ($\mathcal{N}(250, 5^2)$) et la règle empirique :
\begin{enumerate}
    \item Estimez $P(245 \le X \le 255)$.
    \item Estimez $P(X \le 240)$. (Indice : L'intervalle $\mu \pm 2\sigma$ est [240, 260] et contient 95\%. Utilisez la symétrie).
\end{enumerate}
\end{exercicebox}

% --- Propriétés (Transformations Linéaires et Sommes) ---

\begin{exercicebox}[Exercice 14 : Transformation Linéaire (Celsius -> Fahrenheit)]
La température $T_C$ à midi en été dans une ville suit $\mathcal{N}(25, 3^2)$ (en degrés Celsius).
On convertit la température en Fahrenheit : $T_F = 1.8 \times T_C + 32$.
Quelle est la loi de $T_F$ ? (Donnez sa moyenne et sa variance).
\end{exercicebox}

\begin{exercicebox}[Exercice 15 : Transformation Linéaire (Z-score)]
Soit $X \sim \mathcal{N}(\mu, \sigma^2)$. Soit $Y = aX+b$.
Trouvez $a$ et $b$ (en fonction de $\mu$ et $\sigma$) tels que $Y \sim \mathcal{N}(0, 1)$.
\end{exercicebox}

\begin{exercicebox}[Exercice 16 : Somme de Normales Indépendantes]
Soit $X \sim \mathcal{N}(10, 3^2)$ et $Y \sim \mathcal{N}(20, 4^2)$. $X$ et $Y$ sont indépendantes.
Soit $S = X + Y$.
\begin{enumerate}
    \item Quelle est la loi de $S$ ? (Donnez sa moyenne et sa variance).
    \item Quel est l'écart-type de $S$ ?
\end{enumerate}
\end{exercicebox}

\begin{exercicebox}[Exercice 17 : Différence de Normales Indépendantes]
En utilisant $X$ et $Y$ de l'exercice 16, soit $D = Y - X$.
\begin{enumerate}
    \item Quelle est la loi de $D$ ? (Donnez sa moyenne et sa variance).
    \item Quel est l'écart-type de $D$ ? (Comparez-le à celui de $S$).
\end{enumerate}
\end{exercicebox}

\begin{exercicebox}[Exercice 18 : Application (Somme)]
Le poids d'une boîte vide $B$ suit $\mathcal{N}(100g, 5^2)$. Le poids du contenu $C$ suit $\mathcal{N}(800g, 10^2)$. $B$ et $C$ sont indépendants.
Soit $T = B+C$ le poids total.
\begin{enumerate}
    \item Quelle est la loi de $T$ ?
    \item Calculez $P(T > 925g)$. (Utilisez $\Phi(2) \approx 0.9772$).
\end{enumerate}
\end{exercicebox}

\begin{exercicebox}[Exercice 19 : Moyenne d'un Échantillon (Avancé)]
Soient $X_1, X_2, X_3, X_4$ quatre observations indépendantes de la loi $\mathcal{N}(10, 4^2)$.
Soit $\bar{X} = \frac{X_1 + X_2 + X_3 + X_4}{4}$ la moyenne de l'échantillon.
\begin{enumerate}
    \item Soit $S = X_1+X_2+X_3+X_4$. Quelle est la loi de $S$ ?
    \item En utilisant la transformation linéaire $\bar{X} = \frac{1}{4}S$, quelle est la loi de $\bar{X}$ ?
\end{enumerate}
\end{exercicebox}

\begin{exercicebox}[Exercice 20 : Comparaison (Différence)]
Alice et Bob passent un examen. Les notes d'Alice $A$ suivent $\mathcal{N}(80, 5^2)$. Les notes de Bob $B$ suivent $\mathcal{N}(78, 3^2)$. On suppose leurs notes indépendantes.
Quelle est la probabilité que Bob ait une meilleure note qu'Alice ?
(Indice : Calculez $P(B > A)$, ce qui est équivalent à $P(B - A > 0)$).
\end{exercicebox}

\subsection{Corrections des Exercices}

% --- Corrections : PDF, CDF et Loi Normale Standard ---

\begin{correctionbox}[Correction Exercice 1 : Concepts de Base $\Phi(z)$]
1.  $P(Z \le 1.5) = \Phi(1.5)$.
2.  $P(Z > 1) = 1 - P(Z \le 1) = 1 - \Phi(1)$.
3.  $P(Z \le -1.5) = 1 - P(Z \le 1.5) = 1 - \Phi(1.5)$.
4.  $P(-1.5 \le Z \le 1.5) = P(Z \le 1.5) - P(Z \le -1.5) = \Phi(1.5) - (1 - \Phi(1.5)) = 2\Phi(1.5) - 1$.
\end{correctionbox}

\begin{correctionbox}[Correction Exercice 2 : Utilisation d'une Table $\Phi(z)$]
Données : $\Phi(1) \approx 0.8413$, $\Phi(1.96) \approx 0.975$, $\Phi(2) \approx 0.9772$.
1.  $P(Z > 1) = 1 - \Phi(1) \approx 1 - 0.8413 = 0.1587$.
2.  $P(Z \le -2) = 1 - \Phi(2) \approx 1 - 0.9772 = 0.0228$.
3.  $P(-1.96 \le Z \le 1.96) = \Phi(1.96) - \Phi(-1.96) = \Phi(1.96) - (1 - \Phi(1.96))$
    $= 2\Phi(1.96) - 1 \approx 2(0.975) - 1 = 1.95 - 1 = 0.95$.
    (C'est l'intervalle de confiance à 95\%).
\end{correctionbox}

\begin{correctionbox}[Correction Exercice 3 : Propriétés de la PDF $\phi(z)$]
$\phi(z) = \frac{1}{\sqrt{2\pi}} e^{-z^2/2}$.
1.  $\phi(0) = \frac{1}{\sqrt{2\pi}} e^{0} = \frac{1}{\sqrt{2\pi}} \approx 0.3989$.
2.  Puisque $z^2 = (-z)^2$, on a $\phi(z) = \phi(-z)$. La fonction est paire (symétrique par rapport à l'axe y).
3.  Par définition d'une PDF, l'aire totale sous la courbe doit être 1. $\int_{-\infty}^{\infty} \phi(z) \, dz = 1$.
\end{correctionbox}

% --- Corrections : Standardisation (Z-score) et Calcul de Probabilités ---

\begin{correctionbox}[Correction Exercice 4 : Calcul de Z-scores]
$X \sim \mathcal{N}(\mu=50, \sigma^2=100) \implies \sigma=10$.
$Z = \frac{X - \mu}{\sigma}$.
1.  $x = 60 \implies z = (60 - 50) / 10 = 10 / 10 = 1$.
2.  $x = 50 \implies z = (50 - 50) / 10 = 0 / 10 = 0$.
3.  $x = 35 \implies z = (35 - 50) / 10 = -15 / 10 = -1.5$.
\end{correctionbox}

\begin{correctionbox}[Correction Exercice 5 : Calcul de Probabilité (Général)]
$X \sim \mathcal{N}(175, 7^2)$. $\mu=175, \sigma=7$.
1.  $P(X \le 182) = P\left(Z \le \frac{182 - 175}{7}\right) = P(Z \le \frac{7}{7}) = P(Z \le 1) = \Phi(1) \approx 0.8413$.
2.  $P(X > 168) = P\left(Z > \frac{168 - 175}{7}\right) = P(Z > \frac{-7}{7}) = P(Z > -1)$.
    Par symétrie, $P(Z > -1) = P(Z < 1) = \Phi(1) \approx 0.8413$.
\end{correctionbox}

\begin{correctionbox}[Correction Exercice 6 : Calcul de Probabilité (Intervalle)]
$X \sim \mathcal{N}(100, 15^2)$. $\mu=100, \sigma=15$.
On cherche $P(85 \le X \le 115)$.
$z_1 = (85 - 100) / 15 = -15 / 15 = -1$.
$z_2 = (115 - 100) / 15 = 15 / 15 = 1$.
$P(-1 \le Z \le 1) = \Phi(1) - \Phi(-1) = \Phi(1) - (1 - \Phi(1)) = 2\Phi(1) - 1$.
En utilisant $\Phi(1) \approx 0.8413$, $P \approx 2(0.8413) - 1 = 1.6826 - 1 = 0.6826$.
(On retrouve la règle des 68\%).
\end{correctionbox}

\begin{correctionbox}[Correction Exercice 7 : Calcul de Probabilité (Queue Extrême)]
$X \sim \mathcal{N}(100, 15^2)$.
On cherche $P(X > 130)$.
$z = (130 - 100) / 15 = 30 / 15 = 2$.
$P(X > 130) = P(Z > 2) = 1 - P(Z \le 2) = 1 - \Phi(2)$.
En utilisant $\Phi(2) \approx 0.9772$, $P \approx 1 - 0.9772 = 0.0228$.
\end{correctionbox}

% --- Corrections : Problèmes Inverses (Trouver x) ---

\begin{correctionbox}[Correction Exercice 8 : Problème Inverse (Z-score)]
1.  $P(Z \le z) = 0.90 \implies z = \Phi^{-1}(0.90) \approx 1.28$.
2.  $P(Z > z) = 0.05 \implies P(Z \le z) = 1 - 0.05 = 0.95$.
    $z = \Phi^{-1}(0.95) \approx 1.645$.
3.  $P(Z \le z) = 0.10$. C'est dans la queue gauche. Par symétrie, $z = - \Phi^{-1}(1 - 0.10) = - \Phi^{-1}(0.90)$.
    $z \approx -1.28$.
\end{correctionbox}

\begin{correctionbox}[Correction Exercice 9 : Problème Inverse (Général)]
$X \sim \mathcal{N}(100, 15^2)$. On cherche $x$ tel que $P(X > x) = 0.05$.
1.  Trouver le Z-score : $P(Z > z) = 0.05 \implies P(Z \le z) = 0.95 \implies z \approx 1.645$.
2.  Convertir en $x$ : $z = (x-\mu)/\sigma \implies x = \mu + z\sigma$.
    $x = 100 + (1.645)(15) = 100 + 24.675 = 124.675$.
    Le score minimum est d'environ 125.
\end{correctionbox}

\begin{correctionbox}[Correction Exercice 10 : Problème Inverse (Intervalle Central)]
$P(-z \le Z \le z) = 0.95$.
Si 95\% est au centre, il reste $1 - 0.95 = 0.05$ (ou 5\%) dans les deux queues.
Par symétrie, chaque queue a $0.05 / 2 = 0.025$.
La probabilité à gauche de $z$ est $P(Z \le z) = 0.95 + 0.025 = 0.975$.
On cherche $z = \Phi^{-1}(0.975)$.
En utilisant l'indice, $z \approx 1.96$.
\end{correctionbox}

\begin{correctionbox}[Correction Exercice 11 : Problème Inverse (Général)]
$X \sim \mathcal{N}(500, 50^2)$. On cherche $x$ tel que $P(X \le x) = 0.025$.
1.  Trouver le Z-score : $P(Z \le z) = 0.025$. C'est la queue gauche.
    En utilisant l'indice $P(Z \le -1.96) \approx 0.025$, on a $z \approx -1.96$.
2.  Convertir en $x$ : $x = \mu + z\sigma$.
    $x = 500 + (-1.96)(50) = 500 - 98 = 402$.
    Le fabricant doit proposer une garantie de 402 heures.
\end{correctionbox}

% --- Corrections : Règle Empirique (68-95-99.7) ---

\begin{correctionbox}[Correction Exercice 12 : Règle Empirique (Application)]
$X \sim \mathcal{N}(\mu=250, \sigma=5)$.
1.  68\% $\implies \mu \pm 1\sigma = 250 \pm 5 \implies [245, 255]$.
2.  95\% $\implies \mu \pm 2\sigma = 250 \pm 2(5) = 250 \pm 10 \implies [240, 260]$.
3.  99.7\% $\implies \mu \pm 3\sigma = 250 \pm 3(5) = 250 \pm 15 \implies [235, 265]$.
\end{correctionbox}

\begin{correctionbox}[Correction Exercice 13 : Règle Empirique (Probabilité)]
1.  $P(245 \le X \le 255)$ est l'intervalle $\mu \pm 1\sigma$.
    La probabilité est d'environ 68\% ou 0.68.
2.  L'intervalle $\mu \pm 2\sigma$ est $[240, 260]$ et contient 95\% des données.
    Il reste $100\% - 95\% = 5\%$ dans les deux queues (i.e., $P(X < 240) + P(X > 260) = 0.05$).
    Par symétrie, la queue gauche $P(X < 240)$ est $0.05 / 2 = 0.025$.
    La probabilité est d'environ 2.5\% ou 0.025.
\end{correctionbox}

% --- Corrections : Propriétés (Transformations Linéaires et Sommes) ---

\begin{correctionbox}[Correction Exercice 14 : Transformation Linéaire]
$T_C \sim \mathcal{N}(25, 3^2)$. $T_F = a T_C + b$ avec $a=1.8$ et $b=32$.
Loi de $T_F$ : $T_F \sim \mathcal{N}(a\mu + b, (a\sigma)^2)$.
Moyenne : $E[T_F] = 1.8(25) + 32 = 45 + 32 = 77$.
Variance : $\text{Var}(T_F) = (1.8)^2 \text{Var}(T_C) = (1.8)^2 (3^2) = (1.8 \times 3)^2 = (5.4)^2 = 29.16$.
Donc, $T_F \sim \mathcal{N}(77, 29.16)$.
\end{correctionbox}

\begin{correctionbox}[Correction Exercice 15 : Transformation Linéaire (Z-score)]
On veut $Y = aX+b \sim \mathcal{N}(0, 1)$.
$E[Y] = aE[X] + b = a\mu + b$. On veut $a\mu + b = 0$.
$\text{Var}(Y) = a^2 \text{Var}(X) = a^2 \sigma^2$. On veut $a^2 \sigma^2 = 1$.
De $\text{Var}(Y)=1 \implies a^2 = 1/\sigma^2 \implies a = 1/\sigma$ (en supposant $a>0$).
De $E[Y]=0 \implies (1/\sigma)\mu + b = 0 \implies b = -\mu/\sigma$.
Les constantes sont $a = 1/\sigma$ et $b = -\mu/\sigma$. (C'est la définition de la standardisation).
\end{correctionbox}

\begin{correctionbox}[Correction Exercice 16 : Somme de Normales Indépendantes]
$X \sim \mathcal{N}(10, 9)$ et $Y \sim \mathcal{N}(20, 16)$. $S = X+Y$.
1.  La somme de normales indépendantes est une normale.
    $E[S] = E[X] + E[Y] = 10 + 20 = 30$.
    $\text{Var}(S) = \text{Var}(X) + \text{Var}(Y) = 9 + 16 = 25$.
    Donc, $S \sim \mathcal{N}(30, 25)$.
2.  $\text{Var}(S) = 25 \implies \sigma_S = \sqrt{25} = 5$.
\end{correctionbox}

\begin{correctionbox}[Correction Exercice 17 : Différence de Normales Indépendantes]
$D = Y - X$.
1.  La différence est aussi une normale.
    $E[D] = E[Y] - E[X] = 20 - 10 = 10$.
    $\text{Var}(D) = \text{Var}(Y + (-1)X) = \text{Var}(Y) + (-1)^2 \text{Var}(X) = \text{Var}(Y) + \text{Var}(X)$.
    $\text{Var}(D) = 16 + 9 = 25$.
    Donc, $D \sim \mathcal{N}(10, 25)$.
2.  $\sigma_D = \sqrt{25} = 5$. (Identique à $\sigma_S$. La variance s'additionne toujours).
\end{correctionbox}

\begin{correctionbox}[Correction Exercice 18 : Application (Somme)]
$B \sim \mathcal{N}(100, 25)$, $C \sim \mathcal{N}(800, 100)$. $T = B+C$.
1.  $E[T] = E[B] + E[C] = 100 + 800 = 900$.
    $\text{Var}(T) = \text{Var}(B) + \text{Var}(C) = 25 + 100 = 125$.
    $T \sim \mathcal{N}(900, 125)$.
2.  $P(T > 925)$. $\sigma_T = \sqrt{125} = \sqrt{25 \times 5} = 5\sqrt{5} \approx 11.18$.
    $z = (925 - 900) / \sqrt{125} = 25 / (5\sqrt{5}) = 5/\sqrt{5} = \sqrt{5} \approx 2.236$.
    $P(T > 925) = P(Z > 2.236) = 1 - \Phi(2.236) \approx 1 - 0.9873 = 0.0127$.
    (Note : L'indice $\Phi(2) \approx 0.9772$ semble être une approximation pour un $z$ de 2, qui n'est pas le bon $z$ ici).
\end{correctionbox}

\begin{correctionbox}[Correction Exercice 19 : Moyenne d'un Échantillon (Avancé)]
$X_i \sim \mathcal{N}(10, 16)$ (indép.). $\bar{X} = \frac{1}{4} S$ où $S = X_1+X_2+X_3+X_4$.
1.  $S$ est une somme de normales indépendantes.
    $E[S] = E[X_1] + \dots + E[X_4] = 4 \times 10 = 40$.
    $\text{Var}(S) = \text{Var}(X_1) + \dots + \text{Var}(X_4) = 4 \times 16 = 64$.
    $S \sim \mathcal{N}(40, 64)$.
2.  $\bar{X}$ est une transformation linéaire de $S$.
    $E[\bar{X}] = E[\frac{1}{4}S] = \frac{1}{4}E[S] = \frac{1}{4}(40) = 10$.
    $\text{Var}(\bar{X}) = \text{Var}(\frac{1}{4}S) = (\frac{1}{4})^2 \text{Var}(S) = \frac{1}{16}(64) = 4$.
    $\bar{X} \sim \mathcal{N}(10, 4)$.
\end{correctionbox}

\begin{correctionbox}[Correction Exercice 20 : Comparaison (Différence)]
$A \sim \mathcal{N}(80, 25)$, $B \sim \mathcal{N}(78, 9)$. Indép.
On cherche $P(B > A)$, ce qui est $P(B - A > 0)$.
Soit $D = B - A$. $D$ suit une loi normale.
$E[D] = E[B] - E[A] = 78 - 80 = -2$.
$\text{Var}(D) = \text{Var}(B) + \text{Var}(A) = 9 + 25 = 34$.
Donc $D \sim \mathcal{N}(-2, 34)$. $\sigma_D = \sqrt{34} \approx 5.83$.
On cherche $P(D > 0)$.
$z = (0 - (-2)) / \sqrt{34} = 2 / \sqrt{34} \approx 0.342$.
$P(D > 0) = P(Z > 0.342) = 1 - \Phi(0.342) \approx 1 - 0.6338 = 0.3662$.
Il y a environ 36.6\% de chance que Bob ait une meilleure note.
\end{correctionbox}

\subsection{Exercices Pratiques (Python)}

L'une des applications les plus célèbres de la loi normale est la modélisation des rendements financiers. Bien que ce modèle ne soit pas parfait (les krachs boursiers sont plus fréquents que ne le prédit la loi normale), il constitue la pierre angulaire de la finance moderne.

Nous allons supposer que les \textbf{rendements logarithmiques} quotidiens d'un actif financier (comme l'indice S\&P 500) suivent une loi normale $X \sim \mathcal{N}(\mu, \sigma^2)$.

\begin{itemize}
    \item $\mu$ est le rendement moyen quotidien (souvent proche de zéro).
    \item $\sigma$ est la volatilité quotidienne (l'écart-type du rendement).
\end{itemize}

Nous utiliserons \texttt{yfinance} pour obtenir des données réelles, \texttt{numpy} pour les calculs, et \texttt{scipy.stats} pour les fonctions $\Phi$ (CDF) et $\Phi^{-1}$ (PPF).

\begin{codecell}
# Cellule d'installation et d'importation
pip install numpy pandas yfinance scipy
\end{codecell}

\begin{codecell}
import numpy as np
import pandas as pd
import yfinance as yf
from scipy.stats import norm
\end{codecell}

\begin{exercicebox}[Exercice 1 : Modélisation des Rendements du S\&P 500]
Notre première étape est d'obtenir les données de l'indice S\&P 500 (ticker : GSPC) et d'estimer les paramètres $\mu$ et $\sigma$ de notre modèle normal.

\textbf{Votre tâche :}
\begin{enumerate}
    \item Télécharger les données du GSPC des 5 dernières années.
    \item Calculer les rendements logarithmiques quotidiens. La formule est $R = \log(P_t / P_{t-1})$. (Indice : utilisez \texttt{np.log(data['Close'] / data['Close'].shift(1))}).
    \item Estimer $\mu$ (la moyenne) et $\sigma$ (l'écart-type) de ces rendements.
    \item Afficher $\mu$ et $\sigma$. Vous avez maintenant votre modèle $X \sim \mathcal{N}(\mu, \sigma^2)$ !
\end{enumerate}


\begin{codecell}
import numpy as np
import yfinance as yf

ticker = "GSPC"
data = yf.download(ticker, period='5y')

# 1. Calculer les rendements log (log(P_t / P_{t-1}))
# Indice : utilisez .shift(1) pour P_{t-1}
# log_returns = ...
log_returns = log_returns.dropna() # On enleve la premiere valeur (NaN)

# 2. Estimer mu (moyenne) et sigma (ecart-type)
# mu = ...
# sigma = ...

# 3. Afficher les parametres
# print(f"Modele N(mu={mu:.6f}, sigma={sigma:.6f})")
\end{codecell}
\end{exercicebox}

\begin{exercicebox}[Exercice 2 : Calcul de Probabilité (Z-score)]
Utilisons notre modèle $\mathcal{N}(\mu, \sigma^2)$ de l'exercice 1. Un "krach" pourrait être défini comme une baisse de plus de 3\% en une seule journée.

Quelle est la probabilité que cela se produise, selon notre modèle ?

\textbf{Votre tâche :}
\begin{enumerate}
    \item Définir la valeur $x$ d'une baisse de 3\% (en rendement log) : $x = \log(0.97)$.
    \item Standardiser $x$ pour obtenir le Z-score : $z = (x - \mu) / \sigma$.
    \item Utiliser \texttt{scipy.stats.norm.cdf(z)} (qui est $\Phi(z)$) pour trouver la probabilité $P(X \le x)$.
\end{enumerate}

\begin{codecell}
from scipy.stats import norm

# Supposons que mu et sigma sont definis (de l'Ex 1)
# mu = ... (copiez votre valeur de l'Ex 1)
# sigma = ... (copiez votre valeur de l'Ex 1)

# 1. Definir x pour une baisse de 3%
# x = np.log(...)

# 2. Standardiser x pour obtenir le Z-score
# z_score = ...

# 3. Utiliser norm.cdf(z) pour trouver P(X <= x)
# probability = ...

# print(f"Probabilite d'une baisse > 3% : {probability:.8f}")
\end{codecell}
\end{exercicebox}

\begin{exercicebox}[Exercice 3 : Problème Inverse (Value at Risk - VaR)]
Le "Value at Risk" (VaR) est un concept financier qui répond à la question : "Quel est le montant minimum que je peux m'attendre à perdre avec une probabilité $p$ ?"

Calculons le 5\% VaR quotidien. C'est la valeur $x$ (rendement) telle que $P(X \le x) = 0.05$.

\textbf{Votre tâche :}
\begin{enumerate}
    \item Trouver le Z-score $z$ qui correspond au 5ème percentile (probabilité 0.05).
    (Indice : utilisez \texttt{scipy.stats.norm.ppf(0.05)}, qui est $\Phi^{-1}(0.05)$).
    \item "Dé-standardiser" ce Z-score pour trouver la valeur $x$ : $x = \mu + z \cdot \sigma$.
    \item Interpréter le résultat (convertir $x$ en pourcentage : $np.exp(x) - 1$).
\end{enumerate}

\begin{codecell}
from scipy.stats import norm

# Supposons que mu et sigma sont definis (de l'Ex 1)
# mu = ...
# sigma = ...

probabilite = 0.05

# 1. Trouver le Z-score pour 5% (Indice: norm.ppf)
# z_score_var = ...

# 2. De-standardiser pour trouver x (x = mu + z*sigma)
# x_var = ...

# 3. Convertir en pourcentage (Indice: np.exp(x_var) - 1)
# percent_loss = ...

# print(f"Le 5% VaR est une perte de {abs(percent_loss):.2f}%")
\end{codecell}
\end{exercicebox}

\begin{exercicebox}[Exercice 4 : Règle Empirique (68-95-99.7)]
Vérifions à quel point la règle empirique (68-95-99.7) s'applique à nos données réelles de \texttt{log\_returns}.

\textbf{Votre tâche :}
\begin{enumerate}
    \item Définir les intervalles $1\sigma$, $2\sigma$, et $3\sigma$ autour de la moyenne $\mu$.
    \item Calculer le pourcentage réel de rendements (dans \texttt{log\_returns}) qui tombent dans chacun de ces trois intervalles.
    \item Comparer ces pourcentages empiriques aux valeurs théoriques (68.3\%, 95.4\%, 99.7\%).
\end{enumerate}

\begin{codecell}
# Supposons que mu, sigma, et log_returns sont definis

# 1. Definir les bornes
borne_1s_inf = mu - 1 * sigma
borne_1s_sup = mu + 1 * sigma
# ... faire de meme pour 2s et 3s ...
borne_2s_inf = ...
borne_2s_sup = ...
borne_3s_inf = ...
borne_3s_sup = ...

# 2. Compter le pourcentage de 'log_returns' dans chaque intervalle
# Indice: ((log_returns > borne_inf) & (log_returns < borne_sup)).mean()
# within_1s = ...
# within_2s = ...
# within_3s = ...

# print(f"Empirique 1-sigma: {within_1s:.4f} (Theorique: 0.6827)")
# print(f"Empirique 2-sigma: {within_2s:.4f} (Theorique: 0.9545)")
# print(f"Empirique 3-sigma: {within_3s:.4f} (Theorique: 0.9973)")
\end{codecell}
\end{exercicebox}

\begin{exercicebox}[Exercice 5 : Stabilité par Addition (Portfolio Simple)]
Un portfolio est composé de 50\% de S\&P 500 (GSPC) et 50\% d'Or (GC=F).
Nous allons modéliser la loi du rendement de ce portfolio, $P = 0.5 X_S + 0.5 X_G$.

Nous utiliserons les théorèmes $E[aX+bY] = aE[X]+bE[Y]$ et, en supposant (pour cet exercice) l'indépendance : $\text{Var}(aX+bY) = a^2\text{Var}(X) + b^2\text{Var}(Y)$.

\textbf{Votre tâche :}
\begin{enumerate}
    \item Télécharger les données de l'Or (\texttt{`GC=F`}) et calculer $\mu_G$ et $\sigma_G^2$ (variance) de ses rendements log.
    \item Récupérer $\mu_S$ et $\sigma_S^2$ (variance) du S\&P 500 de l'exercice 1.
    \item Calculer la moyenne du portfolio $\mu_P = 0.5\mu_S + 0.5\mu_G$.
    \item Calculer la variance du portfolio $\sigma_P^2 = (0.5)^2\sigma_S^2 + (0.5)^2\sigma_G^2$.
    \item Afficher l'écart-type $\sigma_P$ et le comparer à $\sigma_S$ et $\sigma_G$.
\end{enumerate}

\begin{codecell}
# mu_S et sigma_S de l'Ex 1
# mu_S = ...
# var_S = sigma_S**2

# 1. Obtenir les donnees pour l'Or ('GC=F') et calculer mu_G, var_G
# gold_data = ...
# gold_returns = ...
# mu_G = ...
# var_G = ...

# 2. Poids
w_S = 0.5
w_G = 0.5

# 3. Calculer mu_P (moyenne du portfolio)
# mu_P = ...

# 4. Calculer var_P (en supposant l'independance)
# var_P = ...
# sigma_P = np.sqrt(var_P)

# print(f"Volatilite S&P 500: {sigma_S:.6f}")
# print(f"Volatilite Or: {np.sqrt(var_G):.6f}")
# print(f"Volatilite Portfolio (indep): {sigma_P:.6f}")
\end{codecell}
\end{exercicebox}
\newpage

\section{Moments d'une distribution}

\subsection{Définitions fondamentales des moments}

Après avoir défini l'espérance ($\mu$) et la variance ($\sigma^2$), qui sont les moments d'ordre 1 et 2, nous pouvons généraliser cette idée pour capturer des informations plus subtiles sur la forme d'une distribution.

\begin{definitionbox}[Types de Moments]
Soit $X$ une variable aléatoire ayant une espérance $\mu$ et une variance $\sigma^2$. Pour tout entier positif $m$, on définit les moments suivants :
\begin{itemize}
    \item \textbf{$m$-ième moment (non centré)} : $E[X^m]$.
    \item \textbf{$m$-ième moment centré} : $E[(X - \mu)^m]$.
    \item \textbf{$m$-ième moment standardisé} : $E\left[\left(\frac{X - \mu}{\sigma}\right)^m\right]$.
\end{itemize}
Les moments centrés et standardisés permettent d'étudier les propriétés de la distribution indépendamment de sa position ($\mu$) et de son échelle ($\sigma$).
\end{definitionbox}

\subsection{Asymétrie (Skewness)}

Le premier moment nous donne la tendance centrale. Le deuxième moment (la variance) nous donne la dispersion. Le troisième moment, lui, va nous renseigner sur la \textit{symétrie} de la distribution.

\begin{definitionbox}[Asymétrie (Skewness)]
L'\textbf{asymétrie} (ou \textit{skewness}) d'une variable aléatoire $X$ de moyenne $\mu$ et d'écart-type $\sigma$ est définie comme le \textbf{troisième moment standardisé} :
$$ \text{Skew}(X) = E\left[ \left( \frac{X - \mu}{\sigma} \right)^3 \right]. $$
\end{definitionbox}

\begin{intuitionbox}[Comprendre la Formule du Skewness]
Pour une variable aléatoire $X$ de moyenne $\mu$ et d'écart-type $\sigma$, le \textbf{skewness} est défini comme :
\[
\text{Skew}(X) = \frac{E[(X - \mu)^3]}{\sigma^3}
\]

\medskip

\textbf{Logique du numérateur : le moment centré d'ordre 3}
\begin{itemize}
    \item Le terme $(X - \mu)^3$ est le \textbf{cube de l'écart à la moyenne}
    \item Contrairement à $(X - \mu)^2$ (toujours positif), le cube \textbf{conserve le signe} de l'écart
    \item Il pondère différemment les observations à gauche et à droite de la moyenne
\end{itemize}

\medskip

% --- MODIFIÉ : Tableau supprimé et fusionné dans la liste ---
\textbf{Interprétation intuitive}
\begin{itemize}
    \item \textbf{Skewness = 0 (Symétrique)} : La distribution est symétrique. Les écarts positifs et négatifs s'annulent. Typiquement : Moyenne = Médiane = Mode.
    \item \textbf{Skewness > 0 (Queue à droite)} : La distribution présente une queue longue à droite. Les grandes valeurs positives sont amplifiées par le cube. Les valeurs extrêmes tirent la moyenne vers la droite.
    \item \textbf{Skewness < 0 (Queue à gauche)} : La distribution présente une queue longue à gauche. Les écarts négatifs dominent. Les valeurs extrêmes tirent la moyenne vers la gauche.
\end{itemize}
% --- FIN MODIFICATION ---

\medskip

\textbf{Pourquoi $\sigma^3$ au dénominateur ?}
\begin{itemize}
    \item Le moment d'ordre 3 est homogène à des unités au cube
    \item On divise par $\sigma^3$ pour obtenir un coefficient \textbf{sans dimension}
    \item Permet la comparaison entre distributions de différentes échelles
\end{itemize}
\end{intuitionbox}

\begin{remarquebox}[Pourquoi Standardiser ?]
En standardisant d'abord ($\frac{X-\mu}{\sigma}$), la définition de $\text{Skew}(X)$ ne dépend ni de la position ($\mu$) ni de l'échelle ($\sigma$) de la distribution, ce qui est raisonnable puisque ces informations sont déjà fournies par la moyenne et l'écart-type. De plus, cette standardisation garantit que l'asymétrie est invariante par changement d'unité de mesure (par exemple, passer des pouces aux mètres n'affecte pas la valeur de l'asymétrie).
\end{remarquebox}

\subsection{Propriétés de symétrie}

Le skewness est une mesure numérique de l'asymétrie. Mais nous pouvons aussi définir la symétrie de manière formelle.

\begin{definitionbox}[Symétrie d'une Variable Aléatoire]
On dit qu'une variable aléatoire $X$ a une distribution \textbf{symétrique} autour de $\mu$ si la variable $X - \mu$ a la même distribution que $\mu - X$. On dit aussi que $X$ est symétrique ou que sa distribution est symétrique. Ces trois formulations ont le même sens.
\end{definitionbox}

\begin{theorembox}[Symétrie en Termes de Fonction de Densité]
Soit $X$ une variable aléatoire continue de fonction de densité de probabilité (PDF) $f$. Alors, $X$ est symétrique autour de $\mu$ si et seulement si :
$$ f(x) = f(2\mu - x) \quad \text{pour tout } x. $$
\end{theorembox}

\begin{proofbox}[Preuve du Théorème de Symétrie]
Soit $F$ la fonction de répartition (CDF) de $X$. Si la symétrie tient, alors :
$$ F(x) = P(X \le x) = P(X - \mu \le x - \mu) = P(\mu - X \le x - \mu) = P(X \ge 2\mu - x) = 1 - F(2\mu - x). $$

En prenant la dérivée des deux côtés par rapport à $x$, on obtient :
$$ f(x) = \frac{d}{dx}F(x) = \frac{d}{dx}[1 - F(2\mu - x)] = f(2\mu - x). $$

Cela démontre que la condition $f(x) = f(2\mu - x)$ est nécessaire et suffisante pour la symétrie.
\end{proofbox}

\subsection{Aplatissement (Kurtosis)}

Après l'asymétrie (ordre 3), le moment d'ordre 4 nous informe sur "l'épaisseur" des queues de la distribution, c'est-à-dire la probabilité d'obtenir des valeurs très éloignées de la moyenne.

\begin{definitionbox}[Kurtosis (Aplatissement)]
Pour une variable aléatoire $X$ de moyenne $\mu$ et d'écart-type $\sigma$, le \textbf{kurtosis} est défini comme le \textbf{quatrième moment standardisé} :
$$ \text{Kurtosis}(X) = E\left[ \left( \frac{X - \mu}{\sigma} \right)^4 \right]. $$

Dans la pratique, on utilise plus souvent le \textbf{kurtosis excessif} (ou excès de kurtosis), défini comme :
$$ \text{Excess Kurtosis}(X) = E\left[ \left( \frac{X - \mu}{\sigma} \right)^4 \right] - 3. $$
La soustraction de 3 fait en sorte que le kurtosis d'une loi normale soit égal à 0.
\end{definitionbox}

\begin{intuitionbox}[Comprendre la Kurtosis]
Pour une variable aléatoire $X$, le \textbf{kurtosis} est défini comme :
\[
\text{Kurt}(X) = \frac{E[(X - \mu)^4]}{\sigma^4}
\]
et l'\textbf{excess kurtosis} (kurtosis excédentaire) comme : $\text{Excess Kurtosis} = \text{Kurt}(X) - 3$.

\medskip

\textbf{Pourquoi le moment d'ordre 4 ?}
\begin{itemize}
    \item Comme la variance, on utilise une puissance paire (pas d'effet de signe)
    \item La puissance 4 \textbf{amplifie énormément les écarts extrêmes}
    \item Mesure le \textbf{poids des queues} et la \textbf{concentration autour de la moyenne}
\end{itemize}

\medskip

% --- MODIFIÉ : Tableau supprimé et fusionné dans la liste ---
\textbf{Interprétation intuitive (basée sur l'Excess Kurtosis)}
\begin{itemize}
    \item \textbf{Leptokurtique (Excess Kurtosis > 0)} : Kurtosis total > 3. Distribution pointue avec des queues épaisses. Les événements extrêmes sont plus probables que pour une loi normale.
    \item \textbf{Mésocurtique (Excess Kurtosis = 0)} : Kurtosis total = 3. C'est la référence (loi normale).
    \item \textbf{Platykurtique (Excess Kurtosis < 0)} : Kurtosis total < 3. Distribution aplatie avec des queues légères et un centre large. Les événements extrêmes sont moins probables.
\end{itemize}
% --- FIN MODIFICATION ---

\medskip

\textbf{Application en finance}
\begin{itemize}
    \item Les rendements financiers ont souvent un excès de kurtosis positif
    \item Indique une probabilité plus élevée d'événements extrêmes que la loi normale
    \item Justifie le "vol smile" dans les options
\end{itemize}

\medskip

\textbf{Pourquoi $\sigma^4$ au dénominateur ?}
\begin{itemize}
    \item Le moment d'ordre 4 est homogène à des unités$^4$
    \item On divise par $\sigma^4$ pour un coefficient \textbf{sans dimension}
\end{itemize}
\end{intuitionbox}

\subsection{Exemples de distributions}

Pour bien fixer les idées, comparons le skewness et le kurtosis de plusieurs distributions classiques. Notez que dans les graphiques suivants, le "Kurtosis" affiché est l'\textit{excess kurtosis} (centré à 0).

\begin{examplebox}[La Distribution Normale (Mésokurtique)]

\begin{center}
\begin{tikzpicture}
  \begin{axis}[
    width=0.8\textwidth,
    height=0.5\textwidth,
    xlabel={$x$},
    title={Densité Normale ($\mu=0$, $\sigma=1$)},
    grid=both,
    grid style={line width=.1pt, draw=gray!30},
    major grid style={line width=.2pt,draw=gray!50},
    domain=-4:4,
    samples=100,
    enlargelimits=false,
    axis lines=middle,
    xmin=-4, xmax=4,
    ymin=0, ymax=0.55
  ]
 
  % Courbe de densité
  \addplot [thick, color=blue, fill=blue!20, fill opacity=0.5] 
    {1/(sqrt(2*pi))*exp(-x^2/2)} \closedcycle;
 
  % Ligne de la moyenne
  \addplot [red, dashed, thick] coordinates {(0,0) (0,0.4)};
  \node[red, fill=white, font=\scriptsize, rounded corners, inner sep=2pt, opacity=0.8] at (axis cs:0.5,0.35) {Moyenne};
 
  % Boîte de texte avec moments seulement
  \node [draw=black, fill=white, rounded corners, font=\scriptsize, align=left, anchor=north east, opacity=0.8] 
    at (axis description cs:0.95,0.95) { % MODIFIÉ
    \textbf{Moments :}\\
    • Moyenne = 0.00\\
    • Variance = 1.00\\
    • Skewness = 0.00\\
    • Kurtosis = 0.00
    };
    
  \end{axis}
\end{tikzpicture}
\end{center}

La distribution normale est l'archétype de la courbe en cloche. Imaginez une cible : la majorité des flèches touchent le centre, et plus on s'éloigne du centre, moins il y a de chances d'être touché. C'est une distribution parfaitement symétrique, ce qui se traduit par un \textbf{skewness nul (0.00)}. Son pic est ni trop pointu, ni trop plat : c'est notre point de référence, on dit qu'elle est \textbf{mésokurtique}, d'où son kurtosis de \textbf{0.00}. C'est la base de nombreuses analyses statistiques car elle modélise naturellement beaucoup de phénomènes.

\end{examplebox}

\begin{examplebox}[La Distribution Exponentielle (Asymétrique à Droite)]

\begin{center}
\begin{tikzpicture}
  \begin{axis}[
    width=0.8\textwidth,
    height=0.5\textwidth,
    xlabel={$x$},
    title={Densité Exponentielle ($\lambda=1$)},
    grid=both,
    grid style={line width=.1pt, draw=gray!30},
    major grid style={line width=.2pt,draw=gray!50},
    domain=0:6,
    samples=100,
    enlargelimits=false,
    axis lines=middle,
    xmin=0, xmax=6,
    ymin=0, ymax=1.1
  ]
 
  % Courbe de densité exponentielle
  \addplot [thick, color=blue, fill=blue!20, fill opacity=0.5] 
    {exp(-x)} \closedcycle;
 
  % Ligne de la moyenne
  \addplot [red, dashed, thick] coordinates {(1,0) (1,0.37)};
  \node[red, fill=white, font=\scriptsize, rounded corners, inner sep=2pt, opacity=0.8] at (axis cs:1.5,0.3) {Moyenne};
 
  % Boîte de texte avec moments seulement
  \node [draw=black, fill=white, rounded corners, font=\scriptsize, align=left, anchor=north east, opacity=0.8] 
    at (axis description cs:0.95,0.95) { % MODIFIÉ
    \textbf{Moments :}\\
    • Moyenne = 1.00\\
    • Variance = 1.00\\
    • Skewness = 2.00\\
    • Kurtosis = 6.00
    };
    
  \end{axis}
\end{tikzpicture}
\end{center}

Imaginez le temps d'attente avant un événement rare, comme un appel téléphonique. La plupart du temps, l'appel arrive vite, mais il peut parfois y avoir de longues attentes. C'est exactement ce que modélise la distribution exponentielle : un pic à gauche et une longue queue à droite. Cela se traduit par un \textbf{skewness positif élevé (2.00)}, indiquant une asymétrie marquée. Elle est aussi \textbf{leptokurtique} (\textbf{kurtosis = 6.00}) : son pic est pointu, et la longue queue droite signifie qu'il y a une probabilité non négligeable de valeurs extrêmes.

\end{examplebox}

\begin{examplebox}[La Distribution Uniforme (Platykurtique)]

\begin{center}
\begin{tikzpicture}
  \begin{axis}[
    width=0.8\textwidth,
    height=0.5\textwidth,
    xlabel={$x$},
    title={Densité Uniforme ($a=0$, $b=2$)},
    grid=both,
    grid style={line width=.1pt, draw=gray!30},
    major grid style={line width=.2pt,draw=gray!50},
    domain=-0.5:2.5,
    samples=100,
    enlargelimits=false,
    axis lines=middle,
    ymin=0,
    ymax=.75,
    xmin=-1, xmax=4
  ]
 
  % Courbe de densité uniforme
  \addplot [thick, color=blue, fill=blue!20, fill opacity=0.5, const plot] 
    coordinates {(-0.5,0) (0,0) (0,0.5) (2,0.5) (2,0) (2.5,0)};
 
  % Ligne de la moyenne
  \addplot [red, dashed, thick] coordinates {(1,0) (1,0.5)};
  \node[red, fill=white, font=\scriptsize, rounded corners, inner sep=2pt, opacity=0.8] at (axis cs:1.3,0.4) {Moyenne};
 
  % Boîte de texte avec moments seulement
  \node [draw=black, fill=white, rounded corners, font=\scriptsize, align=left, anchor=north east, opacity=0.8] 
    at (axis description cs:0.95,0.95) { % MODIFIÉ
    \textbf{Moments :}\\
    • Moyenne = 1.00\\
    • Variance = 0.33\\
    • Skewness = 0.00\\
    • Kurtosis = -1.20
    };
    
  \end{axis}
\end{tikzpicture}
\end{center}

La distribution uniforme, c'est le "tirage au sort parfait" : chaque valeur sur un intervalle a la même chance d'être tirée. Visuellement, c'est un rectangle, donc aucune valeur n'est privilégiée. Elle est symétrique (\textbf{skewness = 0.00}), mais contrairement à la normale, elle est "plate", sans pic central. Cela se traduit par un \textbf{kurtosis négatif (-1.20)}, ce qui signifie qu'elle est \textbf{platykurtique}. Elle est donc très différente des distributions avec un pic central comme la normale.

\end{examplebox}

\begin{examplebox}[La Distribution Log-Normale (Fortement Leptokurtique)]

\begin{center}
\begin{tikzpicture}
  \begin{axis}[
    width=0.8\textwidth,
    height=0.5\textwidth,
    xlabel={$x$},
    title={Densité Log-Normale ($\sigma=0.7$)},
    grid=both,
    grid style={line width=.1pt, draw=gray!30},
    major grid style={line width=.2pt,draw=gray!50},
    domain=0:6,
    samples=100,
    enlargelimits=false,
    axis lines=middle,
    xmin=0, xmax=6,
    ymin=0, ymax=1
  ]
 
  % Courbe de densité log-normale
  \addplot [thick, color=blue, fill=blue!20, fill opacity=0.5] 
    {1/(x*0.7*sqrt(2*pi))*exp(-(ln(x))^2/(2*0.7^2))};
 
  % Ligne de la moyenne
  \addplot [red, dashed, thick] coordinates {(1.28,0) (1.28,0.4)};
  \node[red, fill=white, font=\scriptsize, rounded corners, inner sep=2pt, opacity=0.8] at (axis cs:1.8,0.5) {Moyenne};
 
  % Boîte de texte avec moments seulement
  \node [draw=black, fill=white, rounded corners, font=\scriptsize, align=left, anchor=north east, opacity=0.8] 
    at (axis description cs:0.95,0.95) { % MODIFIÉ
    \textbf{Moments :}\\
    • Moyenne = 1.28\\
    • Variance = 1.03\\
    • Skewness = 2.89\\
    • Kurtosis = 20.78
    };
    
  \end{axis}
\end{tikzpicture}
\end{center}

La log-normale est une distribution très asymétrique. Imaginez la richesse d'une population : la majorité est modeste, mais il existe une petite proportion de très riches, ce qui "étire" la droite de la courbe. Cela donne un \textbf{skewness très élevé (2.89)}. Elle est extrêmement \textbf{leptokurtique} (\textbf{kurtosis = 20.78}) : un pic très aigu et une queue droite très lourde. Cela signifie qu'il y a un risque élevé de valeurs extrêmement grandes, ce qui la rend très utile pour modéliser des phénomènes avec de rares événements extrêmes.

\end{examplebox}

Nous avons défini les moments d'une \textit{distribution} (moments de population), tels que $\mu = E[X]$ ou $\sigma^2 = E[(X-\mu)^2]$. Ce sont des valeurs théoriques, la "vérité" sous-jacente.

En pratique, nous ne connaissons presque jamais cette "vérité". Nous ne disposons que de données. Notre but est d'utiliser ces données pour \textit{estimer} les moments de la population.

\subsection{Moments d'échantillon (Sample Moments)}

\begin{definitionbox}[Moments d'Échantillon]
Soit $X_1, X_2, \dots, X_n$ un échantillon de $n$ observations.
\begin{itemize}
    \item La \textbf{moyenne d'échantillon} (notre "meilleure estimation" de $\mu$) est :
    $$ \bar{X} = \frac{1}{n} \sum_{i=1}^n X_i $$
    \item La \textbf{variance d'échantillon (non biaisée)} (notre "meilleure estimation" de $\sigma^2$) est :
    $$ s^2 = \frac{1}{n-1} \sum_{i=1}^n (X_i - \bar{X})^2 $$
\end{itemize}
De même, on peut calculer un \textit{skewness d'échantillon} et un \textit{kurtosis d'échantillon} en utilisant $\bar{X}$ et $s$, qui seront nos estimations du vrai skewness et du vrai kurtosis de la population.
\end{definitionbox}

\begin{examplebox}[Application : Contrôle Qualité ]
Imaginez une usine qui produit des sacs de sucre de 1kg.
\begin{itemize}
    \item \textbf{Population :} L'infinité de tous les sacs de sucre que la machine produira.
    \item \textbf{Moment de population (inconnu) :} Le poids moyen \textit{réel} $\mu$ que la machine verse, et la variance \textit{réelle} $\sigma^2$ (sa constance).
    \item \textbf{Problème :} Nous ne pouvons pas peser tous les sacs !
    \item \textbf{Solution :} Nous prélevons un \textbf{échantillon} de $n=10$ sacs.
    
    Nous les pesons : $\{ 1002g, 998g, 1001g, 995g, 1003g, 1000g, 997g, 1005g, 999g, 1000g \}$.
    
    \item \textbf{Calcul des moments d'échantillon :}
    \begin{itemize}
        \item $\bar{X} = (1002 + 998 + \dots + 1000) / 10 = 1000g$.
        \item $s^2 = \frac{1}{10-1} \left( (1002-1000)^2 + (998-1000)^2 + \dots \right) = 7.33 g^2$.
    \end{itemize}
    \item \textbf{Conclusion :} Notre meilleure estimation est que la machine est bien réglée sur $\mu = 1000g$. L'écart-type de notre échantillon est $s = \sqrt{7.33} \approx 2.7g$. Nous pouvons utiliser cela pour affirmer, par exemple, que 95\% des sacs se situent probablement entre $1000 \pm 2s$ (si la distribution est normale).
\end{itemize}
\end{examplebox}

\begin{remarquebox}[L'Intuition du "$n-1$"]
Pourquoi diviser par $n-1$ pour la variance ? C'est la \textbf{correction de Bessel}.

Imaginez un échantillon de 1 seule personne ($n=1$). Sa taille est 170cm.
\begin{itemize}
    \item Quelle est la moyenne de l'échantillon ? $\bar{X} = 170$ cm.
    \item Quelle est la variance de l'échantillon ? $\sum (X_i - \bar{X})^2 = (170 - 170)^2 = 0$.
    \item Si on divisait par $n=1$, on estimerait que la variance de la population est 0. C'est absurde ! Cela voudrait dire que tout le monde mesure 170cm.
\end{itemize}
En divisant par $n-1$ (donc $1-1=0$), la formule devient $0/0$ (indéfinie), ce qui nous dit à juste titre : "Je ne peux pas estimer la dispersion avec une seule personne."

\textbf{Intuition plus générale :} Nous "perdons un degré de liberté". Pour calculer la variance, nous avons besoin de connaître la moyenne. Mais nous ne connaissons pas la vraie moyenne $\mu$. Nous devons donc utiliser $\bar{X}$, une \textit{estimation}. Le fait d'utiliser une estimation calculée \textit{à partir de ce même échantillon} introduit un léger biais (nos données sont, par définition, centrées sur $\bar{X}$). Diviser par $n-1$ au lieu de $n$ "gonfle" légèrement le résultat pour compenser ce biais.
\end{remarquebox}

\subsection{Fonctions génératrices des moments (MGF)}

\begin{definitionbox}[Fonction Génératrice des Moments (MGF)]
La \textbf{fonction génératrice des moments} (MGF) d'une variable aléatoire $X$, notée $M_X(t)$, est définie comme :
$$ M_X(t) = E[e^{tX}] $$
\end{definitionbox}

\begin{intuitionbox}[L'ADN, le Code-Barres, ou le Fichier .zip]
Ce concept est abstrait, alors utilisons des analogies :

\textbf{Analogie 1 : L'ADN ou l'Empreinte Digitale}
\begin{itemize}
    \item La MGF est l'**empreinte digitale unique** d'une distribution.
    \item Elle "compresse" \textit{toutes} les informations sur votre distribution (moyenne, variance, skewness, kurtosis, etc.) en une seule, unique fonction.
    \item Si deux distributions ont la même MGF, elles sont identiques. C'est la \textbf{propriété d'unicité}.
\end{itemize}

\textbf{Analogie 2 : Le Code-Barres}
\begin{itemize}
    \item Pensez à une distribution (ex: Loi Normale) comme à un produit au supermarché.
    \item La MGF, $M_X(t)$, est son **code-barres unique**.
    \item Le processus de "génération de moments" (que nous verrons ci-dessous) est le \textbf{scanner}.
    \item En scannant le code-barres ($M_X(t)$), vous pouvez obtenir n'importe quelle information :
        \item Scan 1 ($M_X'(0)$) $\to$ vous donne le prix ($E[X]$).
        \item Scan 2 ($M_X''(0)$) $\to$ vous donne le poids ($E[X^2]$).
        \item Scan 3 ($M_X'''(0)$) $\to$ vous donne le pays d'origine ($E[X^3]$).
\end{itemize}

\textbf{Pourquoi $e^{tX}$ ?}
La "magie" vient du développement en série de Taylor de $e^x$:
$$ e^{tX} = 1 + (tX) + \frac{(tX)^2}{2!} + \frac{(tX)^3}{3!} + \dots $$
Quand on prend l'espérance, $E[\cdot]$, les puissances de $X$ (c'est-à-dire $X, X^2, X^3\dots$) apparaissent. Ce sont les moments ! La MGF "stocke" tous ces moments en les organisant comme coefficients d'un polynôme infini en $t$.
\end{intuitionbox}

\subsection{Génération des moments via les MGF}

\begin{theorembox}[Moments par Dérivation]
Si la MGF $M_X(t)$ existe, alors le $m$-ième moment non centré $E[X^m]$ est la $m$-ième dérivée de $M_X(t)$, évaluée en $t=0$ :
$$ E[X^m] = \frac{d^m}{dt^m} M_X(t) \bigg|_{t=0} = M_X^{(m)}(0) $$
\end{theorembox}

\begin{examplebox}[Application : La Loi de Poisson]
Une loi de Poisson modélise le nombre d'événements (ex: appels à un centre d'appels) par heure. Soit $X \sim \text{Poisson}(\lambda)$, où $\lambda$ est le nombre moyen d'appels.

La MGF (l'ADN) d'une loi de Poisson est (on l'admet) :
$$ M_X(t) = e^{\lambda(e^t - 1)} $$

Utilisons notre "scanner" (les dérivées) pour trouver les moments.

\textbf{1. Trouver la Moyenne $E[X]$ :}
On dérive une fois (règle de la chaîne) :
$$ M_X'(t) = \frac{d}{dt} \left( e^{\lambda(e^t - 1)} \right) = \underbrace{e^{\lambda(e^t - 1)}}_{\text{répète}} \cdot \underbrace{(\lambda e^t)}_{\text{dérivée interne}} $$
Maintenant, on évalue en $t=0$ :
$$ E[X] = M_X'(0) = e^{\lambda(e^0 - 1)} \cdot (\lambda e^0) = e^{\lambda(1 - 1)} \cdot (\lambda \cdot 1) = e^0 \cdot \lambda = 1 \cdot \lambda = \lambda $$
\textbf{Résultat :} La moyenne est $\lambda$, ce qui est la définition même du paramètre de la loi de Poisson. Parfait.

\textbf{2. Trouver $E[X^2]$ (pour la variance) :}
On dérive une seconde fois (règle du produit sur $M_X'(t) = (\lambda e^t) \cdot (e^{\lambda(e^t - 1)})$) :
$$ M_X''(t) = \underbrace{(\lambda e^t)}_{\text{dérivée de u}} \cdot \underbrace{(e^{\lambda(e^t - 1)})}_{\text{v}} + \underbrace{(\lambda e^t)}_{\text{u}} \cdot \underbrace{(e^{\lambda(e^t - 1)} \cdot \lambda e^t)}_{\text{dérivée de v}} $$
Maintenant, on évalue en $t=0$ (tous les $e^0$ deviennent 1) :
$$ E[X^2] = M_X''(0) = (\lambda \cdot 1) \cdot (e^{\lambda(1-1)}) + (\lambda \cdot 1) \cdot (e^{\lambda(1-1)} \cdot \lambda \cdot 1) $$
$$ E[X^2] = (\lambda) \cdot (e^0) + (\lambda) \cdot (e^0 \cdot \lambda) = \lambda \cdot 1 + \lambda \cdot (1 \cdot \lambda) = \lambda + \lambda^2 $$

\textbf{3. Trouver la Variance $\text{Var}(X)$ :}
$\text{Var}(X) = E[X^2] - (E[X])^2 = (\lambda + \lambda^2) - (\lambda)^2 = \lambda$
\textbf{Résultat :} Nous avons prouvé par les MGF que pour une loi de Poisson, $\text{Moyenne} = \text{Variance} = \lambda$. C'est une propriété fondamentale de cette loi.
\end{examplebox}

\subsection{Sommes de variables aléatoires indépendantes via les MGF}

C'est la super-puissance des MGF.

\begin{theorembox}[MGF d'une Somme]
Soient $X$ et $Y$ deux variables aléatoires \textbf{indépendantes}. Soit $S = X + Y$. Alors la MGF de $S$ est le produit des MGF individuelles :
$$ M_S(t) = M_{X+Y}(t) = M_X(t) \cdot M_Y(t) $$
\end{theorembox}

\begin{intuitionbox}[La Magie de l'Exponentielle]
Pourquoi est-ce vrai ? $M_{X+Y}(t) = E[e^{t(X+Y)}] = E[e^{tX} \cdot e^{tY}]$.
Parce que $X$ et $Y$ sont indépendantes, $E[f(X)g(Y)] = E[f(X)]E[g(Y)]$.
Donc, $E[e^{tX} \cdot e^{tY}] = E[e^{tX}] \cdot E[e^{tY}] = M_X(t) \cdot M_Y(t)$.

Les MGF transforment une opération analytiquement horrible (la "convolution" de densités) en une simple multiplication algébrique.
\end{intuitionbox}

\begin{examplebox}[Application : Portefeuille d'Actifs ou Tailles Humaines]
C'est l'un des théorèmes les plus importants des statistiques.
\textbf{Problème :} Soit $X$ la taille d'un homme, $X \sim N(\mu_X, \sigma_X^2)$. Soit $Y$ la taille d'une femme, $Y \sim N(\mu_Y, \sigma_Y^2)$. Si on les choisit au hasard, quelle est la loi de la somme de leurs tailles $S = X+Y$ ?

\begin{enumerate}
    \item \textbf{ADN de $X$} : La MGF d'une loi Normale $N(\mu, \sigma^2)$ est $M(t) = \exp(\mu t + \frac{1}{2}\sigma^2 t^2)$.
    \item \textbf{ADN de $X$ et $Y$} :
    $M_X(t) = \exp(\mu_X t + \frac{1}{2}\sigma_X^2 t^2)$
    $M_Y(t) = \exp(\mu_Y t + \frac{1}{2}\sigma_Y^2 t^2)$
    
    \item \textbf{ADN de $S = X+Y$} (on multiplie) :
    $M_S(t) = M_X(t) \cdot M_Y(t) = \exp(\mu_X t + \frac{1}{2}\sigma_X^2 t^2) \cdot \exp(\mu_Y t + \frac{1}{2}\sigma_Y^2 t^2)$
    
    \item \textbf{Simplification} (en additionnant les exposants) :
    $M_S(t) = \exp\left( (\mu_X t + \mu_Y t) + (\frac{1}{2}\sigma_X^2 t^2 + \frac{1}{2}\sigma_Y^2 t^2) \right)$
    $M_S(t) = \exp\left( (\mu_X + \mu_Y)t + \frac{1}{2}(\sigma_X^2 + \sigma_Y^2)t^2 \right)$
    
    \item \textbf{Conclusion (par Unicité)} :
    Regardez cet ADN ! C'est l'ADN d'une loi Normale !
    Le nouveau $\mu$ est $(\mu_X + \mu_Y)$.
    La nouvelle $\sigma^2$ est $(\sigma_X^2 + \sigma_Y^2)$.
\end{enumerate}

\textbf{Résultat :} Nous avons prouvé que \textbf{la somme de deux Normales indépendantes est une nouvelle Normale}.
Si $X \sim N(175cm, 7^2)$ et $Y \sim N(165cm, 6^2)$, alors $S \sim N(340cm, 7^2 + 6^2 = 85)$.
Notez que les écarts-types \textit{ne s'additionnent pas} ($\sqrt{85} \approx 9.2 \ne 7+6$). Ce sont les variances qui s'additionnent.
\end{examplebox}
\newpage

\section{Les Lois des Grands Nombres (LLN)}

Dans la section précédente, nous avons fait une distinction cruciale entre les \textbf{moments de population} (les "vraies" valeurs théoriques, inconnues, comme $\mu$ et $\sigma^2$) et les \textbf{moments d'échantillon} (nos estimations calculées à partir des données, comme $\bar{X}$ et $s^2$).

Par exemple, nous avons défini la moyenne d'échantillon $\bar{X} = \frac{1}{n} \sum X_i$ comme notre "meilleure estimation" de la moyenne de population $\mu$. Mais qu'est-ce qui nous garantit que cette estimation est "bonne" ? Qu'est-ce qui nous assure que si nous collections plus de données (en augmentant $n$), notre $\bar{X}$ se rapprocherait de $\mu$ ?

La réponse à cette question fondamentale est fournie par les \textbf{Lois des Grands Nombres (LLN)}. Elles forment le pont théorique entre les probabilités (la théorie) et les statistiques (la pratique).

\begin{intuitionbox}[L'Idée Fondamentale : L'Exemple du Dé]
Supposons que nous voulons connaître la valeur moyenne d'un lancer de dé équilibré.
\begin{itemize}
    \item \textbf{Moment de Population :} Nous savons par la théorie que $\mu = E[X] = \frac{1+2+3+4+5+6}{6} = 3.5$.
    
    \item \textbf{Moments d'Échantillon :} Nous n'avons pas cette information, alors nous lançons le dé.
    \begin{itemize}
        \item $n=2$ lancers : On obtient (2, 6). $\bar{X}_2 = (2+6)/2 = 4.0$. (Assez loin de 3.5)
        \item $n=10$ lancers : On obtient (1, 6, 3, 3, 5, 2, 4, 1, 6, 4). $\bar{X}_{10} = 3.5$. (Pile dessus !)
        \item $n=100$ lancers : On obtiendra $\bar{X}_{100} \approx 3.48$ (par exemple).
        \item $n=1,000,000$ lancers : On obtiendra $\bar{X}_{1,000,000} \approx 3.5001$ (par exemple).
    \end{itemize}
\end{itemize}
La Loi des Grands Nombres formalise cette intuition : à mesure que $n \to \infty$, la moyenne de notre échantillon $\bar{X}_n$ \textbf{converge} vers la vraie moyenne $\mu$.

La distinction entre les lois "Faible" et "Forte" réside dans la \textit{manière} dont nous définissons cette convergence.
\end{intuitionbox}


\subsection{L'Inégalité de Chebyshev}

Avant de prouver la Loi Faible, nous avons besoin d'un outil fondamental qui relie la variance d'une variable à la probabilité qu'elle s'éloigne de sa moyenne. C'est l'Inégalité de Chebyshev.

Sa puissance réside dans son universalité : elle s'applique à \textit{n'importe quelle} distribution, à condition qu'elle ait une moyenne et une variance finies.

\begin{theorembox}[Inégalité de Chebyshev]
Soit $Y$ une variable aléatoire avec une espérance finie $\mu = E[Y]$ et une variance finie $\sigma^2 = \text{Var}(Y)$.

Alors, pour tout nombre réel $k > 0$ :
$$ P(|Y - \mu| \ge k) \le \frac{\text{Var}(Y)}{k^2} = \frac{\sigma^2}{k^2} $$
\end{theorembox}

% --- DÉBUT DE L'AJOUT DE LA PREUVE ---
\begin{proofbox}[Preuve de l'Inégalité de Chebyshev]
Nous présentons la preuve pour une variable aléatoire continue $Y$ de densité $f(y)$. La preuve pour le cas discret est similaire en remplaçant les intégrales par des sommes.

\begin{enumerate}
    \item Par définition, la variance $\sigma^2$ est $E[(Y - \mu)^2]$ :
    $$ \sigma^2 = E[(Y - \mu)^2] = \int_{-\infty}^{\infty} (y - \mu)^2 f(y) dy $$
    
    \item Nous pouvons scinder cette intégrale en deux parties : la région où $Y$ est proche de $\mu$ ($|y - \mu| < k$) et la région où $Y$ est loin de $\mu$ ($|y - \mu| \ge k$) :
    $$ \sigma^2 = \int_{|y - \mu| < k} (y - \mu)^2 f(y) dy + \int_{|y - \mu| \ge k} (y - \mu)^2 f(y) dy $$
    
    \item L'intégrande $(y - \mu)^2 f(y)$ est toujours non-négative (un carré fois une densité). Par conséquent, la première intégrale est $\ge 0$. En la supprimant, nous ne pouvons que diminuer la valeur totale :
    $$ \sigma^2 \ge \int_{|y - \mu| \ge k} (y - \mu)^2 f(y) dy $$
    
    \item Maintenant, concentrons-nous sur la région d'intégration : $|y - \mu| \ge k$. Dans cette région, par définition, nous avons $(y - \mu)^2 \ge k^2$.
    
    \item Nous pouvons remplacer $(y - \mu)^2$ par $k^2$ dans l'intégrale. Puisque nous remplaçons un terme par quelque chose de plus petit ou égal, la valeur de l'intégrale diminue (ou reste égale) :
    $$ \sigma^2 \ge \int_{|y - \mu| \ge k} k^2 f(y) dy $$
    
    \item $k^2$ est une constante, nous pouvons la sortir de l'intégrale :
    $$ \sigma^2 \ge k^2 \int_{|y - \mu| \ge k} f(y) dy $$
    
    \item Par définition, l'intégrale de la densité $f(y)$ sur la région $|y - \mu| \ge k$ n'est autre que la probabilité $P(|Y - \mu| \ge k)$.
    $$ \sigma^2 \ge k^2 \cdot P(|Y - \mu| \ge k) $$
    
    \item En réarrangeant les termes (puisque $k > 0$, $k^2 > 0$), nous obtenons l'inégalité désirée :
    $$ P(|Y - \mu| \ge k) \le \frac{\sigma^2}{k^2} $$
\end{enumerate}
Cette preuve est un cas particulier de l'Inégalité de Markov (appliquée à la variable aléatoire non-négative $X = (Y-\mu)^2$ et à la constante $a = k^2$).
\end{proofbox}
% --- FIN DE L'AJOUT DE LA PREUVE ---

\begin{intuitionbox}[Comprendre l'Inégalité de Chebyshev]
Cette formule peut être lue comme suit :

\textbf{"La probabilité de s'écarter de la moyenne ($\mu$) d'au moins $k$ est bornée par la variance divisée par $k^2$."}

\begin{itemize}
    \item \textbf{Le rôle de la variance ($\sigma^2$) :} Si la variance est grande, la borne supérieure est élevée. L'inégalité nous dit "il est possible que la variable s'éloigne", ce qui est logique pour une grande dispersion. Si la variance est faible, la borne est basse, ce qui force la probabilité d'être loin à être faible.
    \item \textbf{Le rôle de l'écart ($k$) :} Le terme $k^2$ au dénominateur est crucial. Il signifie que la probabilité de s'écarter de la moyenne diminue \textit{quadratiquement} avec la distance $k$. Être très loin est (relativement) très improbable.
\end{itemize}
\end{intuitionbox}

\begin{examplebox}[Une Borne Universelle]
Exprimons l'inégalité en termes d'écarts-types (en posant $k = c \cdot \sigma$) :
$$ P(|Y - \mu| \ge c\sigma) \le \frac{\sigma^2}{(c\sigma)^2} = \frac{1}{c^2} $$

\begin{itemize}
    \item \textbf{Pour $c=2$ :} $P(|Y - \mu| \ge 2\sigma) \le \frac{1}{4} = 25\%$.
    Peu importe la distribution (symétrique, asymétrique, bizarre...), la probabilité d'être à 2 écarts-types ou plus de la moyenne est \textbf{au maximum} de 25\%. (Pour une loi normale, cette probabilité est bien plus faible, $\approx 4.55\%$).
    
    \item \textbf{Pour $c=3$ :} $P(|Y - \mu| \ge 3\sigma) \le \frac{1}{9} \approx 11.1\%$.
    La probabilité d'être à 3 écarts-types ou plus est au maximum de 11.1\%. (Pour une loi normale, c'est $\approx 0.27\%$).
\end{itemize}
Chebyshev fournit une borne "garantie", bien que souvent non optimale. Elle est l'outil parfait pour la preuve qui suit.
\end{examplebox}


\subsection{La Loi Faible des Grands Nombres (LFGN / WLLN)}

La loi faible stipule que la probabilité que notre moyenne d'échantillon s'écarte de la vraie moyenne de plus qu'une petite quantité $\epsilon$ tend vers zéro. C'est une \textbf{convergence en probabilité}.

\begin{definitionbox}[Convergence en Probabilité]
On dit qu'une suite de variables aléatoires $Y_n$ converge en probabilité vers une constante $c$, noté $Y_n \xrightarrow{P} c$, si pour tout $\epsilon > 0$ (aussi petit soit-il) :
$$\lim_{n \to \infty} P(|Y_n - c| > \epsilon) = 0$$
\end{definitionbox}

\begin{intuitionbox}[Comprendre la Convergence en Probabilité]
La définition $P(|\bar{X}_n - \mu| > \epsilon) \to 0$ signifie :
\begin{itemize}
    \item $\epsilon$ est votre \textbf{marge d'erreur} acceptable (ex: 0.01).
    \item $|\bar{X}_n - \mu|$ est l'erreur réelle de votre estimation.
    \item $P(\dots)$ est la probabilité que votre erreur \textbf{dépasse} votre marge.
    \item $\lim_{n \to \infty} (\dots) = 0$ signifie : "Si vous prenez un échantillon $n$ suffisamment grand, la probabilité de faire une erreur plus grande que $\epsilon$ devient négligeable."
\end{itemize}
C'est une affirmation sur ce qui se passe pour un $n$ fixe et très grand.
\end{intuitionbox}

\begin{theorembox}[Loi Faible des Grands Nombres (Khinchine)]
Soit $X_1, X_2, \dots, X_n$ une suite de variables aléatoires \textbf{i.i.d.} (indépendantes et identiquement distribuées) avec une espérance finie $E[X_i] = \mu$.
Soit $\bar{X}_n = \frac{1}{n} \sum_{i=1}^n X_i$ la moyenne d'échantillon.

Alors, $\bar{X}_n$ converge en probabilité vers $\mu$ :
$$\bar{X}_n \xrightarrow{P} \mu$$
\end{theorembox}

\begin{proofbox}[Preuve (simplifiée) via l'Inégalité de Chebyshev]
La loi faible de Khinchine ne nécessite qu'une moyenne finie. Cependant, si nous ajoutons la condition que la \textbf{variance $\sigma^2$ est aussi finie}, la preuve devient très simple.

Elle repose directement sur l'Inégalité de Chebyshev, que nous venons de voir. Nous l'appliquons à la variable aléatoire $Y = \bar{X}_n$.

\begin{enumerate}
    \item Identifions les termes pour l'inégalité $P(|Y - E[Y]| \ge k) \le \frac{\text{Var}(Y)}{k^2}$:
    \begin{itemize}
        \item Notre variable est $Y = \bar{X}_n$.
        \item Son espérance est $E[Y] = E[\bar{X}_n] = \mu$.
        \item Sa variance est $\text{Var}(Y) = \text{Var}(\bar{X}_n) = \frac{\sigma^2}{n}$.
        \item Notre écart $k$ est la marge d'erreur $\epsilon$.
    \end{itemize}

    \item (Rappel du calcul de la variance de $\bar{X}_n$) :
    Puisque les $X_i$ sont i.i.d., $\text{Var}(\bar{X}_n) = \text{Var}\left(\frac{1}{n}\sum X_i\right) = \frac{1}{n^2}\sum \text{Var}(X_i) = \frac{1}{n^2}(n\sigma^2) = \frac{\sigma^2}{n}$.

    \item Appliquons l'inégalité de Chebyshev avec ces termes :
    $$ P(|\bar{X}_n - \mu| \ge \epsilon) \le \frac{\text{Var}(\bar{X}_n)}{\epsilon^2} = \frac{\sigma^2 / n}{\epsilon^2} = \frac{\sigma^2}{n \epsilon^2} $$
    
    \item Prenons maintenant la limite quand $n \to \infty$ :
    $$ \lim_{n \to \infty} P(|\bar{X}_n - \mu| \ge \epsilon) \le \lim_{n \to \infty} \frac{\sigma^2}{n \epsilon^2} $$
    
    \item Puisque $\sigma^2$ et $\epsilon^2$ sont des constantes finies, le terme de droite $\frac{\text{constante}}{n}$ tend vers 0.
    
    \item Comme une probabilité ne peut pas être négative, nous avons :
    $$ \lim_{n \to \infty} P(|\bar{X}_n - \mu| > \epsilon) = 0 $$
\end{enumerate}
C'est exactement la définition de la convergence en probabilité.
\end{proofbox}

\subsection{La Loi Forte des Grands Nombres (LFGN / SLLN)}

La loi forte est une affirmation beaucoup plus puissante. Elle ne dit pas seulement qu'un "gros" écart est improbable pour un $n$ "grand" ; elle dit que la probabilité que la suite $\bar{X}_n$ \textit{ne converge pas} vers $\mu$ est nulle. C'est une \textbf{convergence presque sûre}.

\begin{definitionbox}[Convergence Presque Sûre]
On dit qu'une suite de variables aléatoires $Y_n$ converge presque sûrement vers une constante $c$, noté $Y_n \xrightarrow{p.s.} c$, si :
$$P\left( \lim_{n \to \infty} Y_n = c \right) = 1$$
\end{definitionbox}

\begin{theorembox}[Loi Forte des Grands Nombres (Kolmogorov)]
Soit $X_1, X_2, \dots, X_n$ une suite de variables aléatoires \textbf{i.i.d.} avec une espérance finie $E[X_i] = \mu$.
Alors, $\bar{X}_n$ converge presque sûrement vers $\mu$ :
$$\bar{X}_n \xrightarrow{p.s.} \mu$$
\end{theorembox}

\begin{remarquebox}[Forte implique Faible]
La convergence "presque sûre" (SLLN) est une condition plus stricte que la convergence "en probabilité" (WLLN). Si une suite converge presque sûrement, elle converge aussi en probabilité. L'inverse n'est pas toujours vrai.
\end{remarquebox}

\subsection{Différence : Faible vs. Forte}

\begin{intuitionbox}[Faible vs. Forte : L'Analogie du Casino]
Soit $\bar{X}_n$ votre gain moyen par partie après avoir joué $n$ fois à la roulette. La vraie moyenne (l'avantage de la maison) est $\mu = -0.053$ (pour une roulette américaine).

\begin{itemize}
    \item \textbf{Loi Faible (WLLN) :} "Si vous prévoyez de jouer $n = 1 \text{ million}$ de parties ce soir. La probabilité qu'à la fin de votre millionième partie, votre moyenne $\bar{X}_{1,000,000}$ soit loin de $-0.053$ (par exemple, que vous soyez gagnant, $\bar{X}_n > 0$) est infinitésimale."
    \item C'est une affirmation sur la distribution de $\bar{X}_n$ \textbf{à un point fixe $n$ (très grand)}. Elle n'exclut pas la possibilité théorique (mais improbable) que si vous continuiez à jouer, votre moyenne $\bar{X}_n$ puisse à nouveau diverger follement avant de reconverger plus tard.

    \item \textbf{Loi Forte (SLLN) :} "Si vous jouez à la roulette \textit{pour l'éternité}, en regardant la séquence de vos moyennes $\bar{X}_1, \bar{X}_2, \bar{X}_3, \dots, \bar{X}_n, \dots$."
    \item "La probabilité que cette \textbf{séquence entière} ne converge pas exactement vers $\mu = -0.053$ est de 0."
    \item C'est une affirmation sur la \textbf{trajectoire complète}. Elle dit que, avec une probabilité de 1, la trajectoire de $\bar{X}_n$ va s'approcher de $\mu$ et \textbf{ne plus s'en écarter} de manière significative.
\end{itemize}

En résumé :
\begin{itemize}
    \item \textbf{Faible :} Pour $n$ assez grand, un écart est \textbf{improbable}.
    \item \textbf{Forte :} La \textbf{trajectoire} converge vers $\mu$ (avec une probabilité de 1).
\end{itemize}
\end{intuitionbox}

\subsection{Application : La Méthode de Monte-Carlo}

La Loi Forte des Grands Nombres est le moteur de l'une des techniques de calcul les plus puissantes : la simulation de Monte-Carlo. Elle nous permet d'estimer des quantités complexes (comme des intégrales) en utilisant le hasard.

\begin{examplebox}[Estimer la valeur de $\pi$]

\textbf{Problème :} Comment calculer $\pi$ sans formule géométrique ?

\textbf{Méthode (Statistique) :}
\begin{enumerate}
    \item Imaginez un carré de côté 1 (de $(0,0)$ à $(1,1)$). Son aire est $A_{\text{carré}} = 1$.
    \item Imaginez un quart de cercle de rayon $r=1$ inscrit dans ce carré. Son aire est $A_{\text{cercle}} = \frac{1}{4}\pi r^2 = \frac{\pi}{4}$.
    \item Le \textit{ratio} des aires est $\frac{A_{\text{cercle}}}{A_{\text{carré}}} = \frac{\pi / 4}{1} = \frac{\pi}{4}$.
\end{enumerate}

\textbf{Simulation :}
\begin{enumerate}
    \item Nous allons "lancer des fléchettes" au hasard sur ce carré $n$ fois.
    \item Pour ce faire, nous générons $n$ paires de nombres aléatoires $(X_i, Y_i)$, où $X_i \sim U(0, 1)$ et $Y_i \sim U(0, 1)$.
    \item Pour chaque point $i$, nous vérifions s'il a atterri \textbf{dans le cercle}. La condition est $X_i^2 + Y_i^2 \le 1$.
    \item Nous définissons une nouvelle variable aléatoire $Z_i$ (de Bernoulli) :
$$ Z_i = \begin{cases} 1 & \text{si } X_i^2 + Y_i^2 \le 1 \quad \text{(le point est dans le cercle)} \\ 0 & \text{sinon} \end{cases} $$
\end{enumerate}

\textbf{Application de la LLN :}
\begin{itemize}
    \item Quelle est la "vraie moyenne" $\mu$ de cette variable $Z_i$ ?
    \item $\mu = E[Z_i] = 1 \cdot P(Z_i=1) + 0 \cdot P(Z_i=0) = P(Z_i=1)$.
    \item $P(Z_i=1)$ est la probabilité qu'un point aléatoire tombe dans le cercle. Puisque les points sont uniformes, cette probabilité est simplement le ratio des aires !
    \item Donc, la vraie moyenne (inconnue) est $\mu = \frac{A_{\text{cercle}}}{A_{\text{carré}}} = \frac{\pi}{4}$.
    
    \item Comment estimer $\mu$ ? Nous utilisons la moyenne d'échantillon $\bar{Z}_n$ :
    $$\bar{Z}_n = \frac{1}{n} \sum_{i=1}^n Z_i = \frac{\text{Nombre de points dans le cercle}}{n}$$
    
    \item Par la \textbf{Loi Forte des Grands Nombres}, nous avons la garantie que :
    $$\bar{Z}_n \xrightarrow{p.s.} \mu = \frac{\pi}{4}$$
\end{itemize}

\textbf{Conclusion :}
Pour estimer $\pi$, il suffit de calculer $\bar{Z}_n$ (une simple proportion) et de la multiplier par 4.
$$\pi \approx 4 \cdot \bar{Z}_n$$
Plus notre nombre de simulations $n$ est grand, plus la SLLN nous garantit que notre estimation sera proche de la vraie valeur de $\pi$.
\end{examplebox}
\newpage

\section{Appendice A: Séries de Taylor et Maclaurin}

\begin{definitionbox}[Séries de Taylor et Maclaurin]
Si une fonction $f$ est indéfiniment dérivable au voisinage d'un point $a$, sa \textbf{série de Taylor} centrée en $a$ est définie par :
$$ f(x) = \sum_{k=0}^{\infty} \frac{f^{(k)}(a)}{k!} (x-a)^k $$
où $f^{(k)}(a)$ est la $k$-ième dérivée de $f$ évaluée en $a$.
\newline
\newline
Dans le cas particulier où $\mathbf{a=0}$, la série est appelée une \textbf{série de Maclaurin}. C'est la forme la plus courante, car elle approxime les fonctions autour de l'origine.
\end{definitionbox}

\subsection{Construction pas à pas d'une série de Taylor}

\begin{intuitionbox}[La logique de la correspondance des dérivées]
L'objectif fondamental d'une série de Taylor est de construire un polynôme, $P(x)$, qui soit une "copie conforme" d'une fonction $f(x)$ autour d'un point $a$. Pour ce faire, on force le polynôme à avoir exactement les mêmes propriétés locales que la fonction : même valeur, même pente, même courbure, etc. Cela se traduit mathématiquement par une exigence : \textbf{la n-ième dérivée du polynôme en $a$ doit être égale à la n-ième dérivée de la fonction en $a$}, et ce pour tous les ordres $n$.

Prenons l'exemple de $f(x) = e^x$ et construisons sa série de Maclaurin (centrée en $a=0$), où $f^{(k)}(0)=1$ pour tout $k$.

\begin{enumerate}
    \item \textbf{Ordre 0 : Faire correspondre la valeur}
    \newline
    \textbf{Objectif :} Le polynôme $P_0(x)$ doit avoir la même valeur que $f(x)$ en $x=0$. On veut $P_0(0) = f(0)$.
    \newline
    \textbf{Solution :} On choisit le polynôme le plus simple, une constante : $P_0(x) = f(0)$. Pour $e^x$, $f(0)=1$, donc $\mathbf{P_0(x) = 1}$.
    \newline
    \textbf{Vérification :} $P_0(0) = 1$. L'objectif est atteint.

    \item \textbf{Ordre 1 : Faire correspondre la première dérivée}
    \newline
    \textbf{Objectif :} On veut un nouveau polynôme $P_1(x)$ qui préserve la correspondance précédente ($P_1(0) = f(0)$) ET qui a la même pente, c'est-à-dire $P_1'(0) = f'(0)$.
    \newline
    \textbf{Solution :} On ajoute un terme en $x$ à notre polynôme précédent : $P_1(x) = P_0(x) + c_1 x = 1 + c_1 x$.
    \newline
    \textbf{Vérification :}
    \begin{itemize}
        \item $P_1(0) = 1 + c_1(0) = 1$. La valeur correspond toujours, car le nouveau terme s'annule en 0.
        \item On dérive : $P_1'(x) = c_1$. Pour que les pentes correspondent en 0, il faut $P_1'(0) = c_1 = f'(0)$. Comme $f'(0)=1$, on doit choisir $\mathbf{c_1=1}$.
    \end{itemize}
    Notre polynôme est maintenant $\mathbf{P_1(x) = 1+x}$.

    \item \textbf{Ordre 2 : Faire correspondre la deuxième dérivée}
    \newline
    \textbf{Objectif :} On veut $P_2(x)$ tel que $P_2(0)=f(0)$, $P_2'(0)=f'(0)$ ET $P_2''(0)=f''(0)$.
    \newline
    \textbf{Solution :} On ajoute un terme en $x^2$ : $P_2(x) = P_1(x) + c_2 x^2 = 1 + x + c_2 x^2$.
    \newline
    \textbf{Vérification :}
    \begin{itemize}
        \item Les dérivées d'ordre 0 et 1 en $x=0$ ne sont pas affectées, car la dérivée de $c_2x^2$ (soit $2c_2x$) et le terme lui-même s'annulent en 0. Les objectifs précédents sont préservés.
        \item On dérive deux fois : $P_2'(x) = 1 + 2c_2x$ et $P_2''(x) = 2c_2$.
        \item Pour que les courbures correspondent, il faut $P_2''(0) = 2c_2 = f''(0)$. Comme $f''(0)=1$, on doit choisir $\mathbf{c_2 = 1/2}$.
    \end{itemize}
    Notre polynôme est $\mathbf{P_2(x) = 1+x+\frac{1}{2}x^2}$.

    \item \textbf{Le schéma général : L'importance de la factorielle}
    \newline
    Pour faire correspondre la $k$-ième dérivée, on ajoute un terme $c_k x^k$.
    \newline
    Quand on dérive $c_k x^k$ exactement $k$ fois, on obtient $c_k \times k!$.
    \newline
    Toutes les dérivées d'ordre inférieur s'annulent en $x=0$. On doit donc avoir :
    $$ P_k^{(k)}(0) = c_k \cdot k! = f^{(k)}(0) $$
    Cela nous donne la règle pour trouver chaque coefficient :
    $$ c_k = \frac{f^{(k)}(0)}{k!} $$
    C'est précisément le coefficient qui apparaît dans la formule de Taylor, et il est choisi pour cette unique raison : forcer la $k$-ième dérivée du polynôme à correspondre parfaitement à celle de la fonction au point de développement.
\end{enumerate}

\tcblower
\centering
\begin{tikzpicture}
    \begin{axis}[
        xlabel={$x$},
        ylabel={$y$},
        xmin=-2, xmax=2,
        ymin=-0.5, ymax=4,
        axis lines=middle,
        legend style={at={(0.05,0.95)}, anchor=north west, font=\small},
        grid=major,
        samples=150,
        domain=-2:2,
        height=9cm,
        width=\linewidth-1cm,
        tick label style={font=\tiny}
    ]
    
    \addplot[black, dashed, ultra thick] {exp(x)};
    \addlegendentry{$e^x$}

    \addplot[red, thick] {1};
    \addlegendentry{$P_0(x)=1$}

    \addplot[blue, thick] {1+x};
    \addlegendentry{$P_1(x)=1+x$}

    \addplot[green!70!black, thick] {1+x+x^2/2};
    \addlegendentry{$P_2(x)=1+x+\frac{x^2}{2!}$}

    \addplot[orange, thick] {1+x+x^2/2+x^3/6};
    \addlegendentry{$P_3(x)=1+x+\frac{x^2}{2!}+\frac{x^3}{3!}$}

    \end{axis}
\end{tikzpicture}
\par\small\textit{Visualisation de la construction progressive de la série de Maclaurin pour $e^x$.}
\end{intuitionbox}

\subsection{Intuition de la série de Taylor en un point quelconque $a$}

\begin{intuitionbox}[Construire une approximation loin de l'origine]
La série de Maclaurin est puissante, mais elle nous contraint à approximer une fonction uniquement autour de $x=0$. Que faire si l'on s'intéresse au comportement d'une fonction ailleurs, par exemple $f(x)=\ln(x)$ autour de $x=1$ (puisque $\ln(0)$ n'est pas défini) ? C'est là qu'intervient la série de Taylor générale.

L'objectif reste le même : construire un polynôme $P(x)$ qui est une "copie conforme" de $f(x)$ au point $a$. Pour cela, on force les dérivées du polynôme à correspondre à celles de la fonction en ce point $a$. La seule différence est que notre "variable" de base n'est plus $x$, mais l'écart par rapport au centre, c'est-à-dire $(x-a)$.

Prenons l'exemple de $f(x) = \ln(x)$ et construisons sa série centrée en $\mathbf{a=1}$.

\begin{enumerate}
    \item \textbf{Ordre 0 : Faire correspondre la valeur}
    \newline
    \textbf{Objectif :} $P_0(a) = f(a)$.
    \newline
    \textbf{Solution :} On calcule $f(1) = \ln(1) = 0$. Le polynôme est la constante $\mathbf{P_0(x) = 0}$.

    \item \textbf{Ordre 1 : Faire correspondre la pente}
    \newline
    \textbf{Objectif :} $P_1(a) = f(a)$ et $P_1'(a) = f'(a)$.
    \newline
    \textbf{Solution :} On ajoute un terme proportionnel à l'écart $(x-a)$ : $P_1(x) = f(a) + c_1 (x-a)$.
    \newline
    \textbf{Vérification :}
    \begin{itemize}
        \item $P_1(1) = 0 + c_1(1-1) = 0$. La valeur correspond.
        \item On dérive : $P_1'(x) = c_1$. On veut $P_1'(1) = c_1 = f'(1)$.
        \item La dérivée de $f(x)=\ln(x)$ est $f'(x) = 1/x$, donc $f'(1)=1$. On doit choisir $\mathbf{c_1=1}$.
    \end{itemize}
    Notre polynôme est $\mathbf{P_1(x) = (x-1)}$. C'est la tangente à $\ln(x)$ en $x=1$.

    \item \textbf{Ordre 2 : Faire correspondre la courbure}
    \newline
    \textbf{Objectif :} Les dérivées jusqu'à l'ordre 2 doivent correspondre en $a=1$.
    \newline
    \textbf{Solution :} On ajoute un terme en $(x-a)^2$ : $P_2(x) = (x-1) + c_2 (x-1)^2$.
    \newline
    \textbf{Vérification :}
    \begin{itemize}
        \item Les correspondances d'ordre 0 et 1 sont préservées.
        \item On dérive deux fois : $P_2'(x) = 1 + 2c_2(x-1)$ et $P_2''(x) = 2c_2$.
        \item On veut $P_2''(1) = 2c_2 = f''(1)$.
        \item La dérivée seconde de $f(x)$ est $f''(x) = -1/x^2$, donc $f''(1)=-1$. On choisit $\mathbf{c_2 = -1/2}$.
    \end{itemize}
    Notre polynôme est $\mathbf{P_2(x) = (x-1) - \frac{1}{2}(x-1)^2}$.

    \item \textbf{Le schéma général}
    \newline
    Le coefficient $c_k$ du terme $(x-a)^k$ est choisi pour faire correspondre la $k$-ième dérivée. La dérivation de $c_k(x-a)^k$ $k$ fois donne $c_k \cdot k!$. On impose donc $c_k \cdot k! = f^{(k)}(a)$, ce qui mène directement à la formule générale $c_k = \frac{f^{(k)}(a)}{k!}$.
\end{enumerate}

\tcblower
\centering
\begin{tikzpicture}
    \begin{axis}[
        xlabel={$x$},
        ylabel={$y$},
        xmin=-0.5, xmax=2.5,
        ymin=-2, ymax=1,
        axis lines=middle,
        legend style={at={(0.05,0.05)}, anchor=south west, font=\small},
        grid=major,
        samples=150,
        domain=0.01:2.5,
        height=9cm,
        width=\linewidth-1cm,
        tick label style={font=\tiny}
    ]
    
    \addplot[black, dashed, ultra thick] {ln(x)};
    \addlegendentry{$\ln(x)$}

    \addplot[red, thick, domain=-0.5:2.5] {0};
    \addlegendentry{$P_0(x)=0$}

    \addplot[blue, thick, domain=-0.5:2.5] {x-1};
    \addlegendentry{$P_1(x)=(x-1)$}

    \addplot[green!70!black, thick, domain=-0.5:2.5] {(x-1) - 0.5*(x-1)^2};
    \addlegendentry{$P_2(x)=(x-1)-\frac{(x-1)^2}{2}$}

    \end{axis}
\end{tikzpicture}
\par\small\textit{Approximation de $\ln(x)$ autour de $a=1$. Le polynôme "colle" à la fonction près de $x=1$.}
\end{intuitionbox}


\subsection{La Fonction Exponentielle ($e^x$)}

\begin{theorembox}[Série de Maclaurin pour $e^x$]
Pour tout nombre réel $x$, la fonction exponentielle peut s'écrire :
$$ e^x = \sum_{k=0}^{\infty} \frac{x^k}{k!} = 1 + x + \frac{x^2}{2!} + \frac{x^3}{3!} + \frac{x^4}{4!} + \cdots $$
\end{theorembox}

\begin{intuitionbox}[Visualiser la Croissance Exponentielle]
La fonction exponentielle est unique car elle est sa propre dérivée. Cela signifie que toutes ses informations locales (valeur, pente, courbure) en $a=0$ sont égales à \textbf{1}. La série pour $e^x$ est donc le polynôme le plus « pur », où chaque terme $x^k$ est simplement normalisé par $k!$. Le graphique ci-dessous montre comment les polynômes de Taylor convergent rapidement vers la véritable courbe exponentielle, illustrant sa croissance puissante.

\tcblower

\centering
\begin{tikzpicture}
    \begin{axis}[
        xlabel={$x$},
        ylabel={$y$},
        xmin=-3, xmax=3,
        ymin=-1, ymax=9,
        axis lines=middle,
        legend style={at={(0.05,0.95)}, anchor=north west, font=\small},
        grid=major,
        samples=150,
        domain=-3:3,
        height=9cm,
        width=\linewidth-1cm,
        tick label style={font=\tiny}
    ]
    
    \addplot[black, dashed, ultra thick] {exp(x)};
    \addlegendentry{$e^x$}

    \addplot[red, thick] {1+x};
    \addlegendentry{$T_1(x)$}

    \addplot[blue, thick] {1+x+x^2/2};
    \addlegendentry{$T_2(x)$}

    \addplot[green!70!black, thick] {1+x+x^2/2+x^3/6};
    \addlegendentry{$T_3(x)$}

    \addplot[orange, thick] {1+x+x^2/2+x^3/6+x^4/24};
    \addlegendentry{$T_4(x)$}

    \end{axis}
\end{tikzpicture}
\par\small\textit{Approximation de $e^x$ par ses polynômes de Maclaurin.}
\end{intuitionbox}

\begin{proofbox}
Soit $f(x) = e^x$. Pour tout entier $k \ge 0$, la $k$-ième dérivée est $f^{(k)}(x) = e^x$. En évaluant en $a=0$, on obtient $f^{(k)}(0) = e^0 = 1$ pour tout $k$. En appliquant la formule de Maclaurin :
$$ e^x = \sum_{k=0}^{\infty} \frac{f^{(k)}(0)}{k!} x^k = \sum_{k=0}^{\infty} \frac{1}{k!} x^k = 1 + x + \frac{x^2}{2} + \frac{x^3}{6} + \cdots $$
\end{proofbox}


\subsection{La Fonction Sinus ($\sin(x)$)}

\begin{theorembox}[Série de Maclaurin pour $\sin(x)$]
Pour tout nombre réel $x$ :
$$ \sin(x) = \sum_{k=0}^{\infty} (-1)^k \frac{x^{2k+1}}{(2k+1)!} = x - \frac{x^3}{3!} + \frac{x^5}{5!} - \frac{x^7}{7!} + \cdots $$
\end{theorembox}

\begin{intuitionbox}[Visualiser l'Oscillation du Sinus]
La série du sinus reflète ses propriétés fondamentales. En tant que fonction \textbf{impaire} ($ \sin(-x) = -\sin(x) $), son développement ne contient que des puissances \textbf{impaires} de $x$. Les signes alternés capturent sa nature oscillatoire. Le graphique ci-dessous montre comment l'ajout de termes permet au polynôme d'« épouser » la courbe du sinus sur un plus grand nombre de périodes.

\tcblower

\centering
\begin{tikzpicture}
    \begin{axis}[
        xlabel={$x$},
        ylabel={$y$},
        xmin=-2*pi, xmax=2*pi,
        ymin=-2.0, ymax=2.0,
        axis lines=middle,
        legend style={at={(0.5,1.15)}, anchor=south, font=\small, column sep=5pt},
        legend columns=3,
        grid=major,
        samples=200,
        domain=-2*pi:2*pi,
        height=9cm,
        width=\linewidth-1cm,
        tick label style={font=\tiny}
    ]
    \addplot[black, dashed, ultra thick] {sin(deg(x))};
    \addlegendentry{$\sin(x)$}
    \addplot[red, thick] {x};
    \addlegendentry{$T_1(x)$}
    \addplot[blue, thick] {x - (x^3)/6};
    \addlegendentry{$T_3(x)$}
    \addplot[green!70!black, thick] {x - (x^3)/6 + (x^5)/120};
    \addlegendentry{$T_5(x)$}
    \addplot[orange, thick] {x - (x^3)/6 + (x^5)/120 - (x^7)/5040};
    \addlegendentry{$T_7(x)$}
    \end{axis}
\end{tikzpicture}
\par\small\textit{Approximation de $\sin(x)$ par ses polynômes de Maclaurin.}
\end{intuitionbox}

\begin{proofbox}
Soit $f(x) = \sin(x)$. Les dérivées en $a=0$ suivent un cycle $(0, 1, 0, -1, \dots)$. Seuls les termes d'ordre impair ($2k+1$) sont non nuls, avec des valeurs de $(-1)^k$, ce qui donne la formule.
\end{proofbox}


\subsection{La Fonction Cosinus ($\cos(x)$)}

\begin{theorembox}[Série de Maclaurin pour $\cos(x)$]
Pour tout nombre réel $x$ :
$$ \cos(x) = \sum_{k=0}^{\infty} (-1)^k \frac{x^{2k}}{(2k)!} = 1 - \frac{x^2}{2!} + \frac{x^4}{4!} - \frac{x^6}{6!} + \cdots $$
\end{theorembox}

\begin{intuitionbox}[Visualiser la Symétrie du Cosinus]
En tant que fonction \textbf{paire} ($ \cos(-x) = \cos(x) $), la série du cosinus ne contient, de manière appropriée, que des puissances \textbf{paires} de $x$. Elle commence à 1 (son maximum) puis oscille, un comportement capturé par les signes alternés.

\tcblower

\centering
\begin{tikzpicture}
    \begin{axis}[
        xlabel={$x$},
        ylabel={$y$},
        xmin=-2*pi, xmax=2*pi,
        ymin=-2.0, ymax=2.0,
        axis lines=middle,
        legend style={at={(0.5,1.15)}, anchor=south, font=\small, column sep=5pt},
        legend columns=3,
        grid=major,
        samples=200,
        domain=-2*pi:2*pi,
        height=9cm,
        width=\linewidth-1cm,
        tick label style={font=\tiny}
    ]
    \addplot[black, dashed, ultra thick] {cos(deg(x))};
    \addlegendentry{$\cos(x)$}
    \addplot[red, thick] {1};
    \addlegendentry{$T_0(x)$}
    \addplot[blue, thick] {1 - x^2/2};
    \addlegendentry{$T_2(x)$}
    \addplot[green!70!black, thick] {1 - x^2/2 + x^4/24};
    \addlegendentry{$T_4(x)$}
    \addplot[orange, thick] {1 - x^2/2 + x^4/24 - x^6/720};
    \addlegendentry{$T_6(x)$}
    \end{axis}
\end{tikzpicture}
\par\small\textit{Approximation de $\cos(x)$ par ses polynômes de Maclaurin.}
\end{intuitionbox}

\begin{proofbox}
Soit $g(x) = \cos(x)$. Les dérivées en $a=0$ suivent un cycle $(1, 0, -1, 0, \dots)$. Seuls les termes d'ordre pair ($2k$) sont non nuls, avec des valeurs de $(-1)^k$, ce qui donne la formule.
\end{proofbox}


\subsection{Le Logarithme Népérien ($\ln(1+x)$)}

\begin{theorembox}[Série de Maclaurin pour $\ln(1+x)$]
Pour $|x| < 1$ :
$$ \ln(1+x) = \sum_{k=1}^{\infty} (-1)^{k-1} \frac{x^k}{k} = x - \frac{x^2}{2} + \frac{x^3}{3} - \frac{x^4}{4} + \cdots $$
\end{theorembox}

\begin{intuitionbox}[Visualiser l'Approximation Logarithmique]
Cette série est essentielle pour approximer les logarithmes près de 1. Contrairement aux fonctions précédentes, elle ne converge que pour $|x|<1$. Le graphique montre que l'approximation est excellente près de $x=0$ mais diverge rapidement lorsque $x$ s'approche de la frontière de convergence à $x=1$.

\tcblower

\centering
\begin{tikzpicture}
    \begin{axis}[
        xlabel={$x$},
        ylabel={$y$},
        xmin=-1.2, xmax=1.2,
        ymin=-4, ymax=2,
        axis lines=middle,
        legend style={at={(0.05,0.95)}, anchor=north west, font=\small},
        grid=major,
        samples=150,
        domain=-0.99:1, % Domain restricted for ln
        height=9cm,
        width=\linewidth-1cm,
        tick label style={font=\tiny}
    ]
    \addplot[black, dashed, ultra thick] {ln(1+x)};
    \addlegendentry{$\ln(1+x)$}
    
    \addplot[red, thick, domain=-1.2:1.2] {x};
    \addlegendentry{$T_1(x)$}

    \addplot[blue, thick, domain=-1.2:1.2] {x - x^2/2};
    \addlegendentry{$T_2(x)$}

    \addplot[green!70!black, thick, domain=-1.2:1.2] {x - x^2/2 + x^3/3};
    \addlegendentry{$T_3(x)$}
    
    \addplot[orange, thick, domain=-1.2:1.2] {x - x^2/2 + x^3/3 - x^4/4};
    \addlegendentry{$T_4(x)$}

    \end{axis}
\end{tikzpicture}
\par\small\textit{Approximation de $\ln(1+x)$ par ses polynômes de Maclaurin.}
\end{intuitionbox}

\begin{proofbox}
Soit $f(x) = \ln(1+x)$. Pour $k \ge 1$, la $k$-ième dérivée en $a=0$ est $f^{(k)}(0) = (-1)^{k-1} (k-1)!$. En substituant cela dans la formule de Maclaurin, le $(k-1)!$ au numérateur annule partiellement le $k!$ au dénominateur, laissant un $k$ en bas.
\end{proofbox}

\subsection{La Série Géométrique ($\frac{1}{1-x}$)}

\begin{theorembox}[Série de Maclaurin pour $\frac{1}{1-x}$]
Pour $|x| < 1$ :
$$ \frac{1}{1-x} = \sum_{k=0}^{\infty} x^k = 1 + x + x^2 + x^3 + \cdots $$
\end{theorembox}

\begin{intuitionbox}[Le Fondement de Nombreuses Séries]
Cette série, connue sous le nom de série géométrique, est l'un des développements en série de puissances les plus fondamentaux. Elle converge uniquement lorsque la valeur absolue de $x$ est inférieure à 1. Chaque coefficient est simplement 1, ce qui en fait la série de Maclaurin la plus simple. De nombreuses autres séries, comme celle de $\ln(1+x)$ ou de $\arctan(x)$, peuvent être dérivées de celle-ci par intégration ou substitution.
\end{intuitionbox}

\begin{proofbox}
Soit $f(x) = (1-x)^{-1}$. Les dérivées successives sont $f'(x) = 1(1-x)^{-2}$, $f''(x) = 2(1-x)^{-3}$, $f'''(x) = 6(1-x)^{-4}$, et ainsi de suite. La formule générale pour la $k$-ième dérivée est $f^{(k)}(x) = k!(1-x)^{-(k+1)}$. En évaluant en $a=0$, on obtient $f^{(k)}(0) = k!$. En substituant dans la formule de Maclaurin :
$$ \frac{1}{1-x} = \sum_{k=0}^{\infty} \frac{f^{(k)}(0)}{k!} x^k = \sum_{k=0}^{\infty} \frac{k!}{k!} x^k = \sum_{k=0}^{\infty} x^k $$
\end{proofbox}

\subsection{Exercices}

\begin{exercicebox}[Termes de base]
Trouvez les quatre premiers termes non nuls de la série de Maclaurin pour $f(x) = \cos(2x)$.
\end{exercicebox}

\begin{correctionbox}
On utilise la série connue de $\cos(u) = 1 - \frac{u^2}{2!} + \frac{u^4}{4!} - \frac{u^6}{6!} + \cdots$.
En substituant $u = 2x$, on obtient :
$$ \cos(2x) = 1 - \frac{(2x)^2}{2!} + \frac{(2x)^4}{4!} - \frac{(2x)^6}{6!} + \cdots $$
$$ \cos(2x) = 1 - \frac{4x^2}{2} + \frac{16x^4}{24} - \frac{64x^6}{5040} + \cdots $$
$$ \cos(2x) = 1 - 2x^2 + \frac{2}{3}x^4 - \frac{4}{315}x^6 + \cdots $$
Les quatre premiers termes sont $1$, $-2x^2$, $\frac{2}{3}x^4$ et $-\frac{4}{315}x^6$.
\end{correctionbox}

\begin{exercicebox}[Dérivation de séries]
Utilisez la série de Maclaurin de $\sin(x)$ pour trouver la série de Maclaurin de $\cos(x)$.
\end{exercicebox}

\begin{correctionbox}
On sait que $\frac{d}{dx}(\sin(x)) = \cos(x)$. On peut dériver la série de $\sin(x)$ terme à terme :
$$ \sin(x) = x - \frac{x^3}{3!} + \frac{x^5}{5!} - \frac{x^7}{7!} + \cdots $$
$$ \frac{d}{dx} \sin(x) = \frac{d}{dx} \left( x - \frac{x^3}{6} + \frac{x^5}{120} - \cdots \right) $$
$$ \cos(x) = 1 - \frac{3x^2}{6} + \frac{5x^4}{120} - \cdots = 1 - \frac{x^2}{2} + \frac{x^4}{24} - \cdots = 1 - \frac{x^2}{2!} + \frac{x^4}{4!} - \cdots $$
On retrouve bien la série de Maclaurin pour $\cos(x)$.
\end{correctionbox}

\begin{exercicebox}[Intégration de séries]
Trouvez la série de Maclaurin pour $\arctan(x)$ en intégrant la série de $\frac{1}{1+x^2}$.
\end{exercicebox}

\begin{correctionbox}
On part de la série géométrique $\frac{1}{1-u} = \sum_{k=0}^{\infty} u^k$. En posant $u = -x^2$, on obtient :
$$ \frac{1}{1+x^2} = \sum_{k=0}^{\infty} (-x^2)^k = \sum_{k=0}^{\infty} (-1)^k x^{2k} = 1 - x^2 + x^4 - x^6 + \cdots $$
Puisque $\int \frac{1}{1+x^2} dx = \arctan(x)$, on intègre la série terme à terme :
$$ \arctan(x) = \int (1 - x^2 + x^4 - \cdots) dx = C + x - \frac{x^3}{3} + \frac{x^5}{5} - \cdots $$
Comme $\arctan(0) = 0$, la constante d'intégration $C$ est nulle.
$$ \arctan(x) = \sum_{k=0}^{\infty} (-1)^k \frac{x^{2k+1}}{2k+1} $$
\end{correctionbox}

\begin{exercicebox}[Approximation de valeur]
Utilisez les trois premiers termes non nuls de la série de Maclaurin de $e^x$ pour approximer la valeur de $\sqrt{e}$.
\end{exercicebox}

\begin{correctionbox}
On veut approximer $\sqrt{e} = e^{0.5}$. La série est $e^x \approx 1 + x + \frac{x^2}{2!}$.
En posant $x=0.5$ :
$$ e^{0.5} \approx 1 + 0.5 + \frac{(0.5)^2}{2} = 1 + 0.5 + \frac{0.25}{2} = 1.5 + 0.125 = 1.625 $$
La valeur réelle est $e^{0.5} \approx 1.6487$. L'approximation est raisonnablement proche.
\end{correctionbox}

\begin{exercicebox}[Série de Taylor non centrée en 0]
Trouvez la série de Taylor pour $f(x) = \ln(x)$ centrée en $a=1$.
\end{exercicebox}

\begin{correctionbox}
On calcule les dérivées de $f(x)=\ln(x)$ et on les évalue en $a=1$.
$f(x) = \ln(x) \implies f(1) = 0$
$f'(x) = 1/x \implies f'(1) = 1$
$f''(x) = -1/x^2 \implies f''(1) = -1$
$f'''(x) = 2/x^3 \implies f'''(1) = 2$
$f^{(k)}(x) = (-1)^{k-1}(k-1)!/x^k \implies f^{(k)}(1) = (-1)^{k-1}(k-1)!$ pour $k \ge 1$.
La série de Taylor est :
$$ \ln(x) = \sum_{k=1}^{\infty} \frac{(-1)^{k-1}(k-1)!}{k!} (x-1)^k = \sum_{k=1}^{\infty} \frac{(-1)^{k-1}}{k} (x-1)^k $$
\end{correctionbox}

\begin{exercicebox}[Identifier une fonction]
Quelle fonction est représentée par la série de Maclaurin $ \sum_{k=0}^{\infty} \frac{(-1)^k x^{2k}}{(2k)!} $ ?
\end{exercicebox}

\begin{correctionbox}
Cette série est $1 - \frac{x^2}{2!} + \frac{x^4}{4!} - \frac{x^6}{6!} + \cdots$. Il s'agit de la série de Maclaurin de la fonction $\cos(x)$.
\end{correctionbox}

\begin{exercicebox}[Série binomiale]
Trouvez les trois premiers termes de la série de Maclaurin pour $f(x) = \sqrt{1+x}$.
\end{exercicebox}

\begin{correctionbox}
On utilise la série binomiale $(1+x)^\alpha$ avec $\alpha=1/2$.
$$ (1+x)^{1/2} = 1 + \alpha x + \frac{\alpha(\alpha-1)}{2!}x^2 + \cdots $$
$$ (1+x)^{1/2} = 1 + \frac{1}{2}x + \frac{\frac{1}{2}(\frac{1}{2}-1)}{2}x^2 + \cdots $$
$$ (1+x)^{1/2} = 1 + \frac{1}{2}x + \frac{\frac{1}{2}(-\frac{1}{2})}{2}x^2 + \cdots = 1 + \frac{1}{2}x - \frac{1}{8}x^2 + \cdots $$
\end{correctionbox}

\begin{exercicebox}[Multiplication de séries]
Trouvez les termes jusqu'à $x^3$ pour la série de Maclaurin de $f(x) = e^x \sin(x)$.
\end{exercicebox}

\begin{correctionbox}
On multiplie les développements de $e^x$ et $\sin(x)$ :
$$ e^x \sin(x) = \left(1 + x + \frac{x^2}{2} + \frac{x^3}{6} + \cdots\right) \left(x - \frac{x^3}{6} + \cdots\right) $$
On collecte les termes par puissance croissante :
\begin{itemize}
    \item Terme en $x$ : $1 \cdot x = x$
    \item Terme en $x^2$ : $x \cdot x = x^2$
    \item Terme en $x^3$ : $1 \cdot (-\frac{x^3}{6}) + \frac{x^2}{2} \cdot x = -\frac{x^3}{6} + \frac{x^3}{2} = \frac{2x^3}{6} = \frac{x^3}{3}$
\end{itemize}
Donc, $e^x \sin(x) = x + x^2 + \frac{x^3}{3} + \cdots$.
\end{correctionbox}

\begin{exercicebox}[Calcul de limite]
Évaluez la limite suivante en utilisant les séries de Maclaurin : $ \lim_{x \to 0} \frac{\sin(x) - x}{x^3} $.
\end{exercicebox}

\begin{correctionbox}
On remplace $\sin(x)$ par son développement :
$$ \lim_{x \to 0} \frac{(x - \frac{x^3}{3!} + \frac{x^5}{5!} - \cdots) - x}{x^3} $$
$$ = \lim_{x \to 0} \frac{-\frac{x^3}{6} + \frac{x^5}{120} - \cdots}{x^3} $$
$$ = \lim_{x \to 0} \left(-\frac{1}{6} + \frac{x^2}{120} - \cdots\right) = -\frac{1}{6} $$
\end{correctionbox}

\begin{exercicebox}[Fonction hyperbolique]
Trouvez la série de Maclaurin pour le cosinus hyperbolique, $\cosh(x) = \frac{e^x + e^{-x}}{2}$.
\end{exercicebox}

\begin{correctionbox}
On utilise les séries de $e^x$ et $e^{-x}$ :
$e^x = 1 + x + \frac{x^2}{2!} + \frac{x^3}{3!} + \cdots$
$e^{-x} = 1 - x + \frac{x^2}{2!} - \frac{x^3}{3!} + \cdots$
En les additionnant, les termes de puissance impaire s'annulent :
$e^x + e^{-x} = 2 + 2\frac{x^2}{2!} + 2\frac{x^4}{4!} + \cdots$
En divisant par 2 :
$$ \cosh(x) = 1 + \frac{x^2}{2!} + \frac{x^4}{4!} + \frac{x^6}{6!} + \cdots = \sum_{k=0}^{\infty} \frac{x^{2k}}{(2k)!} $$
\end{correctionbox}

\begin{exercicebox}[Coefficient de Taylor]
Soit $f(x) = \frac{x^2}{1+x^3}$. Trouvez la valeur de la 8ème dérivée en zéro, $f^{(8)}(0)$.
\end{exercicebox}

\begin{correctionbox}
On sait que le coefficient du terme $x^k$ dans une série de Maclaurin est $\frac{f^{(k)}(0)}{k!}$.
On développe $f(x)$ :
$$ f(x) = x^2 \cdot \frac{1}{1-(-x^3)} = x^2 \sum_{n=0}^{\infty} (-x^3)^n = x^2 \sum_{n=0}^{\infty} (-1)^n x^{3n} = \sum_{n=0}^{\infty} (-1)^n x^{3n+2} $$
On cherche le terme en $x^8$. On doit avoir $3n+2=8$, ce qui donne $3n=6$, soit $n=2$.
Le coefficient de $x^8$ est donc $(-1)^2 = 1$.
On a alors $\frac{f^{(8)}(0)}{8!} = 1$, ce qui implique $f^{(8)}(0) = 8! = 40320$.
\end{correctionbox}

\begin{exercicebox}[Approximation d'intégrale]
Estimez la valeur de $\int_0^1 \sin(x^2) dx$ en utilisant les deux premiers termes non nuls de la série de Maclaurin de la fonction à intégrer.
\end{exercicebox}

\begin{correctionbox}
On part de $\sin(u) = u - \frac{u^3}{6} + \cdots$. On pose $u=x^2$ :
$$ \sin(x^2) = x^2 - \frac{(x^2)^3}{6} + \cdots = x^2 - \frac{x^6}{6} + \cdots $$
On intègre ce polynôme de 0 à 1 :
$$ \int_0^1 \left(x^2 - \frac{x^6}{6}\right) dx = \left[ \frac{x^3}{3} - \frac{x^7}{42} \right]_0^1 $$
$$ = \left(\frac{1}{3} - \frac{1}{42}\right) - 0 = \frac{14 - 1}{42} = \frac{13}{42} \approx 0.3095 $$
\end{correctionbox}

\begin{exercicebox}[Un autre centre]
Trouvez les trois premiers termes de la série de Taylor pour $f(x) = \frac{1}{x}$ centrée en $a=2$.
\end{exercicebox}

\begin{correctionbox}
On calcule les dérivées et on les évalue en $a=2$.
$f(x) = x^{-1} \implies f(2) = 1/2$
$f'(x) = -x^{-2} \implies f'(2) = -1/4$
$f''(x) = 2x^{-3} \implies f''(2) = 2/8 = 1/4$
La série commence par :
$$ f(x) \approx f(2) + f'(2)(x-2) + \frac{f''(2)}{2!}(x-2)^2 $$
$$ f(x) \approx \frac{1}{2} - \frac{1}{4}(x-2) + \frac{1/4}{2}(x-2)^2 = \frac{1}{2} - \frac{1}{4}(x-2) + \frac{1}{8}(x-2)^2 $$
\end{correctionbox}

\begin{exercicebox}[Combinaison de séries]
Trouvez le terme en $x^4$ du développement de Maclaurin de $f(x) = \ln(1-x^2)$.
\end{exercicebox}

\begin{correctionbox}
On utilise la série de $\ln(1+u) = u - \frac{u^2}{2} + \frac{u^3}{3} - \frac{u^4}{4} + \cdots$.
On substitue $u = -x^2$ :
$$ \ln(1-x^2) = (-x^2) - \frac{(-x^2)^2}{2} + \frac{(-x^2)^3}{3} - \frac{(-x^2)^4}{4} + \cdots $$
$$ = -x^2 - \frac{x^4}{2} - \frac{x^6}{3} - \frac{x^8}{4} - \cdots $$
Le terme en $x^4$ est $-\frac{x^4}{2}$.
\end{correctionbox}

\begin{exercicebox}[Application en physique]
En relativité restreinte, l'énergie cinétique d'une particule est $K = mc^2(\gamma - 1)$, où $\gamma = (1-v^2/c^2)^{-1/2}$. Montrez que pour des vitesses faibles ($v \ll c$), cette formule se réduit à la formule classique $K \approx \frac{1}{2}mv^2$.
\end{exercicebox}

\begin{correctionbox}
On utilise le développement binomial $(1+x)^\alpha$ avec $x = -v^2/c^2$ et $\alpha = -1/2$.
$$ \gamma = \left(1 - \frac{v^2}{c^2}\right)^{-1/2} \approx 1 + \alpha x = 1 + \left(-\frac{1}{2}\right)\left(-\frac{v^2}{c^2}\right) = 1 + \frac{1}{2}\frac{v^2}{c^2} $$
On substitue ce résultat dans la formule de l'énergie :
$$ K = mc^2(\gamma - 1) \approx mc^2 \left( \left(1 + \frac{1}{2}\frac{v^2}{c^2}\right) - 1 \right) $$
$$ K \approx mc^2 \left( \frac{1}{2}\frac{v^2}{c^2} \right) = \frac{1}{2}mv^2 $$
On retrouve bien l'énergie cinétique classique comme approximation de premier ordre.
\end{correctionbox}
\section{Tests}

\begin{examplebox}[Simulation d'un lancer de dé]
On peut simuler $n$ lancers d'un dé équilibré à 6 faces en utilisant la bibliothèque \texttt{random} de Python.

\begin{codecell}
import random
import numpy
\end{codecell}

\begin{outputcell}
>> "vamonos"
\end{outputcell}

\end{examplebox}

\begin{definitionbox}[Variable Aléatoire]
Une variable aléatoire est une fonction qui associe un nombre réel à chaque résultat possible d'une expérience aléatoire.
\end{definitionbox}

\begin{codecell}
import math
import random

def poisson_knuth(lmbda: float) -> int:
  """
  Simule une variable aleatoire suivant une loi de Poisson()
  en utilisant l algorithme de Knuth.
  """
  L = math.exp(-lmbda)
  k = 0
  p = 1.0

  while p > L:
  k += 1
  p *= random.random()

  return k - 1
\end{codecell}


\end{document}