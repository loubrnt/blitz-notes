\documentclass{article}

% --- GESTION DES MARGES DE PAGE ---
\usepackage[a4paper, top=2.5cm, bottom=2.5cm, left=3cm, right=3cm]{geometry}

% --- PRÉAMBULE STANDARD ---
\usepackage[utf8]{inputenc}
\usepackage[T1]{fontenc}
\usepackage{lmodern}
\usepackage[french]{babel}
\usepackage{parskip} % NOUVEAU : Supprime l'indentation et ajoute espace inter-paragraphe
\usepackage{xcolor}
\usepackage{tcolorbox}
\usepackage{listings}
\usepackage{amsmath}
\usepackage{graphicx} % Requis pour inclure des images
\usepackage{amssymb} % Pour les symboles mathématiques comme \subseteq
\usepackage{sectsty} % Pour le style des sections
\usepackage{etoolbox} % Pour les conditions
\usepackage[dvipsnames]{xcolor}
\usepackage{pgfplots} % Le package principal pour les graphiques
\usepackage{enumitem}
\usepackage{diagbox}

\setlist[itemize,1]{label=$\cdot$}

\usetikzlibrary{
    matrix, 
    patterns.meta,
    calc,
    positioning,
    decorations.pathreplacing,
    trees,
    backgrounds
}

% Redéfinir le symbole pour le premier niveau de la liste
\renewcommand{\labelitemi}{$\cdot$}
\renewcommand{\labelitemii}{$\circ$}
\renewcommand{\labelitemiii}{$\ast$}

% Styles de hachures (inchangés)
\tikzset{
  red_hatch/.style={
    pattern={Lines[angle=45, line width=0.8pt, distance=4pt]}, 
    pattern color=red
  },
  blue_hatch/.style={
    pattern={Lines[angle=-45, line width=0.8pt, distance=4pt]}, 
    pattern color=blue
  },
  purple_hatch/.style={
    pattern={Lines[angle=45, line width=0.8pt, distance=4pt]}, 
    pattern color=red,
    postaction={
      pattern={Lines[angle=-45, line width=0.8pt, distance=4pt]}, 
      pattern color=blue
    }
  }
}

% --- BIBLIOTHÈQUES TCOLORBOX ---
\tcbuselibrary{listings, skins, breakable}

% --- GESTION DES LIENS HYPERTEXTE ---
\usepackage[colorlinks=true, linkcolor=black, urlcolor=blue]{hyperref}

% --- SOULIGNER LES TITRES ET SOUS-TITRES ---
\sectionfont{\underline}
\subsectionfont{\underline}
\subsubsectionfont{\underline}

% --- PAGE DE GARDE AMÉLIORÉE ---
\makeatletter
\renewcommand{\maketitle}{%
\begin{titlepage}
\centering
\vspace*{\stretch{1.5}}
{\Huge \bfseries Mes Notes de Lecture\par}
\vspace{0.4cm}
\rule{0.8\linewidth}{0.4pt}
\vspace{1cm}
{\LARGE \bfseries Introduction à la Probabilité\par}
\vspace*{\stretch{2.5}}
{\Large \scshape Lou Brunet\par}
\vspace{0.5cm}
{\large \today\par}
\vspace*{\stretch{1}}
\end{titlepage}
}
\makeatother

% ==================================================================
% --- MODIFIÉ : DÉFINITION DES COULEURS STYLE VS CODE (LIGHT) ---
% ==================================================================
\definecolor{vscodeBlue}{HTML}{569CD6}
\definecolor{vscodeOrange}{HTML}{CE9178}
\definecolor{vscodeGreen}{HTML}{6A9955}
\definecolor{vscodePurple}{HTML}{C586C0}
\definecolor{vscodeGray}{HTML}{9B9B9B}
\definecolor{codeBackground}{HTML}{F8F8F8} % Fond gris très clair
\definecolor{codeText}{HTML}{242424}       % Texte principal (presque noir)
\definecolor{codeGray}{HTML}{A0A0A0}       % Numéros de ligne (gris moyen)

% ==================================================================
% --- MODIFIÉ : CONFIGURATION DU STYLE LISTINGS (LIGHT) ---
% ==================================================================
\lstdefinestyle{vscode}{
    language=Python,
    backgroundcolor=\color{codeBackground},     % Utilise le nouveau fond F8F8F8
    basicstyle=\ttfamily\small\color{codeText}, % Texte principal noir
    keywordstyle=\color{vscodeBlue},
    stringstyle=\color{vscodeOrange},
    commentstyle=\color{vscodeGreen},
    numberstyle=\tiny\color{codeGray},         % Numéros de ligne en gris
    otherkeywords={self, True, False, None},
    keywordstyle=[2]\color{vscodePurple},
    showstringspaces=false,
    breaklines=true,
    frame=none,
    tabsize=4
}

% --- STYLE DE L'OUTPUT ---
\lstdefinestyle{outputstyle}{
    basicstyle=\ttfamily\small\color{codeText}, % Texte principal noir
    breaklines=true,
    frame=none
}

% ==================================================================
% --- MODIFIÉ : DÉFINITION DES CELLULES DE CODE ET OUTPUT (LIGHT) ---
% ==================================================================
% Elles utilisent maintenant le même style "sidebar" que les autres boîtes.

\newtcblisting{codecell}{
  skin=enhanced, % Pour la bordure latérale
  arc=0mm,       % Coins carrés
  boxrule=0pt,   % Pas de cadre
  colback=codeBackground, % Fond clair (F8F8F8)
  borderline west={2pt}{0pt}{vscodeBlue}, % Barre latérale bleue
  fonttitle=\bfseries\color{vscodeBlue}, % Titre en bleu
  listing only,
  listing options={style=vscode, basicstyle=\ttfamily\footnotesize\color{codeText}}, % TEXTE NOIR
  left=3mm, right=3mm, top=2mm, bottom=2mm, % Padding (identique à sidebarstyle)
  boxsep=0mm, % (identique à sidebarstyle)
  breakable  % (identique à sidebarstyle)
}
\newtcblisting{outputcell}{
  skin=enhanced, % Pour la bordure latérale
  arc=0mm,       % Coins carrés
  boxrule=0pt,   % Pas de cadre
  colback=black!5, % Fond gris très clair (Output)
  borderline west={2pt}{0pt}{corrColor}, % Barre latérale grise
  fonttitle=\bfseries\color{corrColor}, % Titre en gris
  listing only,
  listing options={style=outputstyle, basicstyle=\ttfamily\footnotesize\color{codeText}}, % TEXTE NOIR
  left=3mm, right=3mm, top=2mm, bottom=2mm, % Padding (identique à sidebarstyle)
  boxsep=0mm, % (identique à sidebarstyle)
  breakable  % (identique à sidebarstyle)
}

% --- DÉFINITION DES COULEURS POUR DÉF/THÉO/PREUVE/INTUITION/EXEMPLE ---
\definecolor{defColor}{HTML}{1b1f3a}       % Bleu nuit pour les définitions
\definecolor{theoColor}{HTML}{53354a}      % Violet aubergine pour les théorèmes
\definecolor{proofColor}{HTML}{a64942}     % Rouge brique pour les preuves
\definecolor{intuitionColor}{HTML}{16A085} % Turquoise pour l'intuition
\definecolor{exampleColor}{HTML}{4a6982}   % Bleu ardoise pour les exemples
\definecolor{exoColor}{HTML}{1E8449}       % Vert
\definecolor{corrColor}{HTML}{7F8C8D}      % Gris
\definecolor{remarqueColor}{HTML}{D35400} 

% ==================================================================
% --- DÉFINITION DU STYLE DE BASE (MARGES RÉDUITES) ---
% ==================================================================

% --- STYLE "ORGANIQUE" AVEC BARRE LATÉRALE (Pour TOUTES les boîtes) ---
\tcbset{
    sidebarstyle/.style={
        skin=enhanced, % Nécessaire pour les bordures partielles
        arc=0mm,       % Coins carrés
        boxrule=0pt,   % Pas de cadre
        colback=white, % Fond blanc
        colframe=white,
        coltitle=white,
        fontupper=\color{black},
        left=3mm, right=3mm, top=2mm, bottom=2mm, % Padding interne réduit
        boxsep=0mm, % TRÈS IMPORTANT
        breakable
    }
}

% ==================================================================
% --- DÉFINITION DE TOUTES LES CELLULES (TEXTE) ---
% ==================================================================

% --- CELLULES "SIDEBAR" (Pour le contenu théorique) ---
\newtcolorbox{definitionbox}[1][]{
  sidebarstyle,
  borderline west={2pt}{0pt}{defColor}, % Barre latérale gauche
  fonttitle=\bfseries\color{defColor},  % Titre en couleur (sans fond)
  title=Définition\ifstrempty{#1}{}{ : #1}
}
\newtcolorbox{theorembox}[1][]{
  sidebarstyle,
  borderline west={2pt}{0pt}{theoColor},
  fonttitle=\bfseries\color{theoColor},
  title=Théorème\ifstrempty{#1}{}{ : #1}
}
\newtcolorbox{proofbox}[1][]{
  sidebarstyle,
  borderline west={2pt}{0pt}{proofColor},
  fonttitle=\bfseries\color{proofColor},
  title=Preuve\ifstrempty{#1}{}{ : #1}
}
\newtcolorbox{intuitionbox}[1][]{
  sidebarstyle,
  borderline west={2pt}{0pt}{intuitionColor},
  fonttitle=\bfseries\color{intuitionColor},
  title=Intuition\ifstrempty{#1}{}{ : #1}
}
\newtcolorbox{remarquebox}[1][]{
  sidebarstyle,
  borderline west={2pt}{0pt}{remarqueColor},
  fonttitle=\bfseries\color{remarqueColor},
  title=Remarque\ifstrempty{#1}{}{ : #1}
}


% --- CELLULES "SIDEBAR" (Pour les applications) ---
\newtcolorbox{examplebox}[1][]{
  sidebarstyle,
  borderline west={2pt}{0pt}{exampleColor},
  fonttitle=\bfseries\color{exampleColor},
  title=Exemple\ifstrempty{#1}{}{ : #1}
}
\newtcolorbox{exercicebox}[1][]{
    sidebarstyle,
    borderline west={2pt}{0pt}{corrColor},
    fonttitle=\bfseries\color{corrColor},
    title=#1 % <-- On utilise directement l'argument fourni
}
\newtcolorbox{correctionbox}[1][]{
    sidebarstyle,
    borderline west={2pt}{0pt}{exoColor},
    fonttitle=\bfseries\color{exoColor},
    title=#1
}


% =============================================
% --- CORPS DU DOCUMENT ---
% =============================================
\begin{document}

\maketitle

\newpage
\phantomsection
\addcontentsline{toc}{section}{Sommaire}
\tableofcontents 
\newpage

\newpage
\section{Probabilités et Dénombrement}

\subsection{Concepts fondamentaux}

Avant de pouvoir calculer des probabilités, il est essentiel d'établir un vocabulaire commun pour décrire les expériences aléatoires.

\begin{intuitionbox}[Nécessité d'un Cadre Formel]
Avant de calculer des probabilités, il est crucial de définir les règles du jeu :

\textbf{Qu'est-ce qui peut arriver ?}

On définit l'ensemble de tous les résultats possibles de l'expérience.

\textbf{À quoi s'intéresse-t-on ?} 

On identifie les sous-ensembles de résultats spécifiques qui nous intéressent.

Ces deux idées nous conduisent aux notions d'Univers et d'Événement, qui sont les piliers de toute théorie des probabilités.
\end{intuitionbox}

Cette intuition se traduit formellement par deux définitions clés :

\begin{definitionbox}[Concepts Fondamentaux]
\textbf{Univers (ou Espace Échantillon), $S$ :} 

L'ensemble de tous les résultats possibles d'une expérience aléatoire.

\textbf{Événement, $A$ :} 

Un sous-ensemble de l'univers ($A \subseteq S$). C'est un ensemble de résultats auxquels on s'intéresse.
\end{definitionbox}

Un exemple simple permet de solidifier ces concepts :

\begin{examplebox}[Univers et Événement]
Pour l'expérience du "lancer d'un dé à six faces" :

L'\textbf{univers} est $S = \{1, 2, 3, 4, 5, 6\}$.

"Obtenir un nombre impair" est un événement, représenté par le sous-ensemble $A = \{1, 3, 5\}$.
\end{examplebox}

\subsection{Définition Formelle (Axiomatique) de la Probabilité}

Maintenant que nous avons défini les événements, nous pouvons définir formellement les règles que doit suivre une fonction de probabilité. L'image que vous avez fournie contient cette définition axiomatique fondamentale.

\begin{definitionbox}[Fonction de Probabilité (Axiomes)]
Une \textbf{fonction de probabilité} $P$ sur un univers (espace échantillon) fini $S$ est une fonction qui associe à chaque événement $A \subseteq S$ un nombre $P(A) \in [0,1]$ tel que :
\begin{enumerate}
    \item $P(S) = 1$, et
    \item $P(A \cup B) = P(A) + P(B)$ si $A$ et $B$ sont disjoints (c'est-à-dire $A \cap B = \emptyset$).
\end{enumerate}
Le nombre $P(A)$ est appelé la \textbf{probabilité} que $A$ se produise.
\end{definitionbox}

Cette définition est très générale. Le premier cas que nous allons étudier, la "définition naïve", est l'application la plus simple de ces axiomes, valable dans le cas particulier de l'équiprobabilité.

\subsection{Définition Naïve de la Probabilité}

Pour de nombreuses expériences simples, comme lancer un dé non pipé, chaque résultat possible est "équiprobable". Cette hypothèse est la base de la première définition formelle de la probabilité.

\begin{definitionbox}[Probabilité Naïve]
Pour une expérience où chaque issue dans un espace échantillon fini $S$ est équiprobable, la probabilité d'un événement $A$ est le rapport du nombre d'issues favorables à $A$ sur le nombre total d'issues :
$$ P(A) = \frac{\text{Nombre d'issues favorables}}{\text{Nombre total d'issues}} = \frac{|A|}{|S|} $$
\end{definitionbox}

Appliquons cette formule à quelques cas classiques :

\begin{examplebox}[Applications de la définition naïve]
\begin{enumerate}
    \item \textbf{Lancer une pièce équilibrée :}
    L'espace échantillon est $S = \{\text{Pile, Face}\}$, donc $|S| = 2$.
    Si l'événement $A$ est "obtenir Pile", alors $A = \{\text{Pile}\}$ et $|A| = 1$.
    La probabilité est $P(A) = \frac{1}{2}$.

    \item \textbf{Lancer un dé à six faces non pipé :}
    L'espace échantillon est $S = \{1, 2, 3, 4, 5, 6\}$, donc $|S| = 6$.
    Si l'événement $B$ est "obtenir un nombre pair", alors $B = \{2, 4, 6\}$ et $|B| = 3$.
    La probabilité est $P(B) = \frac{3}{6} = \frac{1}{2}$.

    \item \textbf{Tirer une carte d'un jeu de 52 cartes :}
    L'espace échantillon $S$ contient 52 cartes, donc $|S| = 52$.
    Si l'événement $C$ est "tirer un Roi", il y a 4 Rois dans le jeu, donc $|C| = 4$.
    La probabilité est $P(C) = \frac{4}{52} = \frac{1}{13}$.
\end{enumerate}
\end{examplebox}

\subsection{Permutations (Arrangements)}

Le dénombrement, qui est l'art de compter les tailles $|A|$ et $|S|$, est fondamental pour appliquer la définition naïve. Le premier outil que nous verrons est la permutation, qui compte les arrangements \textbf{ordonnés}.

\begin{definitionbox}[Permutation de $k$ objets parmi $n$]
Le nombre de façons d'arranger $k$ objets choisis parmi $n$ objets distincts (où l'ordre compte et il n'y a pas de répétition) est noté $P(n, k)$ ou $A_n^k$ et est défini par :
$$ P(n, k) = \frac{n!}{(n-k)!} $$
où $n!$ est la factorielle de $n$, et par convention $0! = 1$.
\end{definitionbox}

Cette formule peut sembler abstraite, mais elle provient d'un raisonnement logique simple par "cases" :

\begin{intuitionbox}[Permutations de $k$ parmi $n$]
Pour placer $k$ objets dans un ordre spécifique en les choisissant parmi $n$ objets disponibles, on a $n$ choix pour la première position, $(n-1)$ choix pour la deuxième, ..., et $(n-k+1)$ choix pour la $k$-ième position. Cela donne $n \times (n-1) \times \cdots \times (n-k+1)$ arrangements. Ce produit contient $k$ termes. Il est égal à $\frac{n!}{(n-k)!}$, car cela revient à diviser la suite complète $n!$ par les facteurs non utilisés $(n-k) \times (n-k-1) \times \cdots \times 1$.
\end{intuitionbox}

Voyons une application classique de ce principe :

\begin{examplebox}[Permutations de $k$ parmi $n$]
\textbf{Podium d'une course :} Une course réunit 8 coureurs. Combien y a-t-il de podiums (1er, 2e, 3e) possibles ?

On cherche le nombre de façons d'ordonner 3 coureurs parmi 8 : $P(8, 3)$. 
$$ P(8, 3) = \frac{8!}{(8-3)!} = \frac{8!}{5!} = 8 \times 7 \times 6 = 336 $$
Il y a 336 podiums possibles.
\end{examplebox}

\subsection{Le Coefficient Binomial}

Que se passe-t-il si l'ordre ne compte pas ? Au lieu de compter des podiums, nous voulons compter des comités. C'est le rôle du coefficient binomial.

\begin{theorembox}[Formule du Coefficient Binomial]
Le nombre de façons de choisir $k$ objets parmi un ensemble de $n$ objets distincts (sans remise et sans ordre) est donné par le coefficient binomial :
$$ \binom{n}{k} = \frac{n!}{k!(n-k)!} $$
\end{theorembox}

% NOUVEAU :
La preuve de cette formule repose sur un argument combinatoire élégant : nous allons compter la même chose (les permutations) de deux façons différentes.
% FIN NOUVEAU

\begin{proofbox}
Considérons le nombre de permutations de $k$ objets parmi $n$, noté $P(n,k)$.
\begin{enumerate}
    \item \textbf{Méthode 1 :} Par définition (vue ci-dessus), nous savons que $P(n,k) = \frac{n!}{(n-k)!}$.
    
    \item \textbf{Méthode 2 :} Nous pouvons construire une telle permutation en deux étapes successives :
    \begin{itemize}
        \item D'abord, \textbf{choisir un sous-ensemble} de $k$ objets parmi $n$ (l'ordre ne compte pas). C'est le nombre que nous cherchons, notons-le $\binom{n}{k}$.
        \item Ensuite, \textbf{ordonner} ces $k$ objets choisis. Il y a $k!$ façons de les arranger.
    \end{itemize}
    Le nombre total de permutations est donc le produit de ces étapes : $P(n,k) = \binom{n}{k} \times k!$.
\end{enumerate}
En égalisant les deux méthodes, on obtient :
\[ \binom{n}{k} \cdot k! = \frac{n!}{(n-k)!} \]
En divisant par $k!$, on trouve bien la formule :
\[ \binom{n}{k} = \frac{n!}{k!(n-k)!} \]
\end{proofbox}

% NOUVEAU :
L'intuition visuelle derrière cette preuve est de voir comment chaque "choix" (une colonne du tableau) génère $k!$ "ordres" (les lignes de cette colonne).
% FIN NOUVEAU

\begin{intuitionbox}
Pour rendre cela concret, voici le cas $\binom{5}{3}$.  
Il y a 10 sous-ensembles de 3 éléments parmi $\{a,b,c,d,e\}$. Chacun donne lieu à $3! = 6$ permutations.  
Le tableau ci-dessous montre \textbf{toutes les 60 permutations}, regroupées par sous-ensemble :

\vspace{3mm}

\begin{center}
\small
\renewcommand{\arraystretch}{0.9}
\setlength{\tabcolsep}{2pt}
\begin{tabular}{|c|c|c|c|c|c|c|c|c|c|}
\hline
\textbf{$\{a,b,c\}$} & \textbf{$\{a,b,d\}$} & \textbf{$\{a,b,e\}$} & \textbf{$\{a,c,d\}$} & \textbf{$\{a,c,e\}$} & \textbf{$\{a,d,e\}$} & \textbf{$\{b,c,d\}$} & \textbf{$\{b,c,e\}$} & \textbf{$\{b,d,e\}$} & \textbf{$\{c,d,e\}$} \\
\hline
$abc$ & $abd$ & $abe$ & $acd$ & $ace$ & $ade$ & $bcd$ & $bce$ & $bde$ & $cde$ \\
\hline
$acb$ & $adb$ & $aeb$ & $adc$ & $aec$ & $aed$ & $bdc$ & $bec$ & $bed$ & $ced$ \\
\hline
$bac$ & $bad$ & $bae$ & $cad$ & $cae$ & $dae$ & $cbd$ & $ceb$ & $dbe$ & $dce$ \\
\hline
$bca$ & $bda$ & $bea$ & $cda$ & $cea$ & $dea$ & $cdb$ & $ceb$ & $deb$ & $dec$ \\
\hline
$cab$ & $dab$ & $eab$ & $dac$ & $eac$ & $ead$ & $dbc$ & $ebc$ & $edb$ & $ecd$ \\
\hline
$cba$ & $dba$ & $eba$ & $dca$ & $eca$ & $eda$ & $dcb$ & $ebc$ & $edb$ & $edc$ \\
\hline
\end{tabular}
\end{center}

\vspace{3mm}

\smallskip

Chaque colonne correspond à \textbf{un seul et même choix non ordonné} (par exemple $\{a,b,c\}$), mais à 6 listes différentes selon l’ordre.  
Ainsi, pour obtenir le nombre de \textit{choix non ordonnés}, on divise le nombre total de listes ($60$) par le nombre d’ordres par groupe ($6$) :
\[
\binom{5}{3} = \frac{60}{6} = 10.
\]
\end{intuitionbox}

L'application la plus directe est le tirage d'un groupe où l'ordre n'importe pas :

\begin{examplebox}[Utilisation du Coefficient Binomial]
    \textbf{Comité d'étudiants :} De combien de manières peut-on former un comité de 3 étudiants à partir d'une classe de 10 ? L'ordre ne compte pas.
    $$ \binom{10}{3} = \frac{10!}{3!(10-3)!} = \frac{10 \times 9 \times 8}{3 \times 2 \times 1} = 120 \text{ comités possibles.} $$
\end{examplebox}

\subsection{Identité de Vandermonde}

Les coefficients binomiaux obéissent à de nombreuses identités. L'identité de Vandermonde est l'une des plus utiles, car elle montre comment décomposer un problème de comptage complexe en sous-problèmes.

\begin{theorembox}[Identité de Vandermonde]
Cette identité offre une relation remarquable entre les coefficients binomiaux. Pour des entiers non négatifs $m, n$ et $k$, on a :
$$ \binom{m+n}{k} = \sum_{j=0}^{k} \binom{m}{j} \binom{n}{k-j} $$
\end{theorembox}

% NOUVEAU :
La preuve la plus intuitive est une "preuve par l'histoire" (proof by story), qui consiste à trouver un scénario de dénombrement que les deux côtés de l'équation résolvent.
% FIN NOUVEAU

\begin{proofbox}[Preuve combinatoire]
Imaginons un groupe composé de $m$ hommes et $n$ femmes. Nous souhaitons former un comité de $k$ personnes. Nous allons compter le nombre de comités possibles de deux façons.

\textbf{Côté gauche : $\binom{m+n}{k}$}
Le groupe total contient $m+n$ personnes. Le nombre de façons de choisir un comité de $k$ personnes parmi ce total est, par définition, $\binom{m+n}{k}$.

\textbf{Côté droit : $\sum_{j=0}^{k} \binom{m}{j} \binom{n}{k-j}$}
Nous pouvons compter le même nombre en conditionnant sur le nombre d'hommes (noté $j$) dans le comité. Un comité de $k$ personnes doit contenir $j$ hommes ET $k-j$ femmes, où $j$ peut aller de $0$ à $k$.
\begin{itemize}
    \item Pour $j=0$ : Choisir 0 homme ($\binom{m}{0}$) ET $k$ femmes ($\binom{n}{k}$).
    \item Pour $j=1$ : Choisir 1 homme ($\binom{m}{1}$) ET $k-1$ femmes ($\binom{n}{k-1}$).
    \item ...
    \item Pour $j=k$ : Choisir $k$ hommes ($\binom{m}{k}$) ET 0 femme ($\binom{n}{0}$).
\end{itemize}
Puisque ces cas (0 homme, 1 homme, etc.) sont mutuellement exclusifs, le nombre total de comités est la somme de toutes ces possibilités :
\[ \sum_{j=0}^{k} \binom{m}{j} \binom{n}{k-j} \]
Puisque les deux côtés comptent exactement la même chose (le nombre total de comités), ils doivent être égaux.
\end{proofbox}

Vérifions cette identité avec un exemple numérique concret, en reprenant l'analogie du comité :

\begin{examplebox}[Application de l'Identité de Vandermonde]
On veut former un comité de 3 personnes ($k=3$) à partir d'un groupe de 5 hommes ($m=5$) et 4 femmes ($n=4$).

\textbf{Méthode directe (côté gauche) :}
On choisit 3 personnes parmi les $5+4=9$ au total.
$$ \binom{9}{3} = \frac{9 \times 8 \times 7}{3 \times 2 \times 1} = 84 $$

\textbf{Méthode par cas (côté droit) :}
La somme est $\binom{5}{0}\binom{4}{3} + \binom{5}{1}\binom{4}{2} + \binom{5}{2}\binom{4}{1} + \binom{5}{3}\binom{4}{0} = 84$. Les deux méthodes donnent bien le même résultat.
\end{examplebox}

\newpage

\subsection{Bose-Einstein (Étoiles et Bâtons)}

Jusqu'à présent, nous avons supposé un "tirage sans remise". La statistique de Bose-Einstein, ou plus visuellement la méthode des "étoiles et bâtons", s'attaque au problème du \textbf{tirage avec remise} où l'ordre ne compte pas.

\begin{theorembox}[Combinaisons avec répétition]
Le nombre de façons de distribuer $k$ objets indiscernables dans $n$ boîtes discernables (ou de choisir $k$ objets parmi $n$ avec remise, où l'ordre ne compte pas) est donné par la formule :
$$ \binom{n+k-1}{k} = \binom{n+k-1}{n-1} $$
\end{theorembox}

% NOUVEAU :
La preuve de cette formule est l'un des résultats les plus élégants du dénombrement. L'astuce consiste à transformer le problème de distribution en un problème d'arrangement de symboles.
% FIN NOUVEAU

\begin{proofbox}[Par les "Étoiles et Bâtons"]
Nous cherchons à distribuer $k$ objets indiscernables ($\star$) dans $n$ boîtes discernables.
Nous pouvons représenter n'importe quelle distribution comme une séquence de symboles. Nous avons besoin de $k$ étoiles (les objets) et de $n-1$ bâtons ($|$) pour servir de séparateurs entre les $n$ boîtes.

Par exemple, pour distribuer $k=7$ étoiles dans $n=4$ boîtes, la séquence :
$$ \star\star\star \mid \star \mid \mid \star\star\star $$
correspond à : 3 étoiles dans la boîte 1, 1 étoile dans la boîte 2, 0 étoile dans la boîte 3 (l'espace entre deux bâtons), et 3 étoiles dans la boîte 4.

Chaque arrangement unique de ces symboles correspond à une distribution unique. Le problème revient donc à trouver le nombre de façons d'arranger ces $k$ étoiles et ces $n-1$ bâtons.

Nous avons un total de $n+k-1$ positions à remplir. Le nombre de façons de le faire est simplement le nombre de manières de choisir les $k$ positions pour les étoiles (les autres positions étant automatiquement remplies par des bâtons).
C'est exactement :
$$ \binom{n+k-1}{k} $$
(Ce qui est aussi égal à $\binom{n+k-1}{n-1}$, le nombre de façons de choisir les positions des $n-1$ bâtons).
\end{proofbox}

C'est la méthode parfaite pour tout problème de distribution d'objets identiques :

\begin{examplebox}[Distribution de biens identiques]
De combien de manières peut-on distribuer 10 croissants identiques à 4 enfants ?

Ici, $k=10$ (les croissants, objets indiscernables) et $n=4$ (les enfants, boîtes discernables).
Le nombre de distributions possibles est :
$$ \binom{4+10-1}{10} = \binom{13}{10} = \binom{13}{3} = \frac{13 \times 12 \times 11}{3 \times 2 \times 1} = 13 \times 2 \times 11 = 286 $$
Il y a 286 façons de distribuer les croissants.
\end{examplebox}

\subsection{Principe d'Inclusion-Exclusion}

Comment compter le nombre d'éléments dans l'union de plusieurs ensembles ? Si on additionne simplement leurs tailles, on compte les intersections plusieurs fois. Le principe d'inclusion-exclusion corrige systématiquement ce sur-comptage.

\begin{theorembox}[Principe d'Inclusion-Exclusion pour 3 ensembles]
Pour trois ensembles finis $A$, $B$ et $C$, le nombre d'éléments dans leur union est donné par :
$$ |A \cup B \cup C| = |A| + |B| + |C| - |A \cap B| - |A \cap C| - |B \cap C| + |A \cap B \cap C| $$
\end{theorembox}

% NOUVEAU :
La preuve pour 3 ensembles se fait en appliquant la formule pour 2 ensembles de manière répétée.
% FIN NOUVEAU

\begin{proofbox}
Nous utilisons la formule pour deux ensembles, $|X \cup Y| = |X| + |Y| - |X \cap Y|$, de manière imbriquée.
Posons $X = A \cup B$ et $Y = C$.
\begin{align*}
|A \cup B \cup C| &= |(A \cup B) \cup C| \\
&= |A \cup B| + |C| - |(A \cup B) \cap C|
\end{align*}
Nous devons maintenant développer les deux termes compliqués :
\begin{enumerate}
    \item $|A \cup B| = |A| + |B| - |A \cap B|$
    \item Par distributivité de l'intersection sur l'union, $(A \cup B) \cap C = (A \cap C) \cup (B \cap C)$.
\end{enumerate}
Appliquons la formule pour 2 ensembles à ce deuxième terme :
\[ |(A \cap C) \cup (B \cap C)| = |A \cap C| + |B \cap C| - |(A \cap C) \cap (B \cap C)| \]
Ce qui se simplifie en $|A \cap C| + |B \cap C| - |A \cap B \cap C|$.

Finalement, en substituant tout dans l'équation de départ :
\begin{align*}
|A \cup B \cup C| &= \underbrace{(|A| + |B| - |A \cap B|)}_{|A \cup B|} + |C| \\
                 &\quad - \underbrace{(|A \cap C| + |B \cap C| - |A \cap B \cap C|)}_{|(A \cup B) \cap C|}
\end{align*}
En réarrangeant les termes, on obtient la formule voulue :
\[ |A| + |B| + |C| - |A \cap B| - |A \cap C| - |B \cap C| + |A \cap B \cap C| \]
\end{proofbox}

La formule devient évidente lorsque l'on utilise un diagramme de Venn pour visualiser le sur-comptage et sa correction.

\begin{intuitionbox}[Visualisation avec 3 ensembles]
Le principe d'inclusion-exclusion permet de compter le nombre d'éléments dans une union d'ensembles sans double-comptage. Pour comprendre intuitivement pourquoi on ajoute et soustrait alternativement, considérons trois ensembles $A$, $B$ et $C$ :

\begin{center}
\begin{tikzpicture}[set/.style = {draw,
    circle,
    minimum size = 6cm,
    fill=Rhodamine,
    opacity = 0.4,
    text opacity = 1}]
 
\node (A) [set] {$A$};
\node (B) at (60:4cm) [set] {$B$};
\node (C) at (0:4cm) [set] {$C$};
 
\node at (barycentric cs:A=1,B=1) [left] {$X$};
\node at (barycentric cs:A=1,C=1) [below] {$Y$};
\node at (barycentric cs:B=1,C=1) [right] {$Z$};
\node at (barycentric cs:A=1,B=1,C=1) [] {$T$};
 
\end{tikzpicture}
\end{center}

\textbf{Le problème :} Si on additionne simplement $|A| + |B| + |C|$, on compte certaines zones plusieurs fois :
\begin{itemize}
    \item Les intersections deux à deux ($X$, $Y$, $Z$) sont comptées \textbf{deux fois}
    \item L'intersection triple ($T$) est comptée \textbf{trois fois}
\end{itemize}

\textbf{La solution :} On corrige en soustrayant les intersections deux à deux, mais alors l'intersection triple est comptée :
\begin{itemize}
    \item $+3$ fois dans la somme initiale
    \item $-3$ fois dans la soustraction des intersections deux à deux (car elle appartient à chacune)
    \item Donc $0$ fois au total ! Il faut la rajouter.
\end{itemize}

D'où la formule : $|A \cup B \cup C| = |A| + |B| + |C| - |A \cap B| - |A \cap C| - |B \cap C| + |A \cap B \cap C|$
\end{intuitionbox}

Ce que nous avons fait visuellement pour 3 ensembles peut être généralisé par récurrence à $n$ ensembles. La formule générale suit le même principe d'alternance des signes :

\begin{theorembox}[Principe d'Inclusion-Exclusion généralisé]
Pour $n$ ensembles finis $A_1, A_2, \dots, A_n$, on a :
\begin{align*}
|A_1 \cup A_2 \cup \cdots \cup A_n| = & \sum_{i=1}^n |A_i| \\
& - \sum_{1 \leq i < j \leq n} |A_i \cap A_j| \\
& + \sum_{1 \leq i < j < k \leq n} |A_i \cap A_j \cap A_k| \\
& - \cdots \\
& + (-1)^{n+1} |A_1 \cap A_2 \cap \cdots \cap A_n|
\end{align*}
Ce qui s'écrit plus compactement :
$$ \left| \bigcup_{i=1}^n A_i \right| = \sum_{k=1}^n (-1)^{k+1} \sum_{1 \leq i_1 < i_2 < \cdots < i_k \leq n} |A_{i_1} \cap A_{i_2} \cap \cdots \cap A_{i_k}| $$
\end{theorembox}

% NOUVEAU :
La preuve formelle que cette formule gigantesque fonctionne est fascinante. Il suffit de montrer que n'importe quel élément $x$ de l'union, peu importe à combien d'ensembles il appartient, est compté \textbf{exactement une fois} au final.
% FIN NOUVEAU

\begin{proofbox}[Preuve par comptage d'un élément]
Considérons un élément $x$ qui appartient à exactement $k$ ensembles parmi les $n$ ensembles $A_1, \ldots, A_n$ (où $k \ge 1$). Nous devons montrer que $x$ est compté exactement 1 fois par la formule.

Analysons combien de fois $x$ est compté dans chaque somme de la formule :
\begin{itemize}
    \item \textbf{Première somme ($\sum |A_i|$)} : $x$ est dans $k$ ensembles, donc il est ajouté $k$ fois. Le nombre de fois est $\binom{k}{1}$.
    
    \item \textbf{Deuxième somme ($-\sum |A_i \cap A_j|$)} : $x$ est compté (et soustrait) pour chaque \textit{paire} d'ensembles auxquels il appartient. Comme il appartient à $k$ ensembles, il y a $\binom{k}{2}$ telles paires.
    
    \item \textbf{Troisième somme ($+\sum |A_i \cap A_j \cap A_k|$)} : $x$ est ajouté pour chaque \textit{triplet} d'ensembles auxquels il appartient. Il y en a $\binom{k}{3}$.
    
    \item \textbf{Et ainsi de suite...}
\end{itemize}
Au total, l'élément $x$ est compté :
$$ \text{Total} = \binom{k}{1} - \binom{k}{2} + \binom{k}{3} - \cdots + (-1)^{k-1}\binom{k}{k} \text{ fois.} $$
Pour évaluer cette somme, rappelons l'identité fondamentale du binôme de Newton :
$$ (1 + x)^k = \sum_{j=0}^{k} \binom{k}{j} x^j = \binom{k}{0} + \binom{k}{1}x + \binom{k}{2}x^2 + \cdots $$
Si nous posons $x = -1$, nous obtenons :
$$ (1-1)^k = 0 = \binom{k}{0} - \binom{k}{1} + \binom{k}{2} - \binom{k}{3} + \cdots + (-1)^k\binom{k}{k} $$
Sachant que $\binom{k}{0} = 1$, on a :
$$ 0 = 1 - \left( \binom{k}{1} - \binom{k}{2} + \binom{k}{3} - \cdots + (-1)^{k-1}\binom{k}{k} \right) $$
En réarrangeant, on trouve :
$$ 1 = \binom{k}{1} - \binom{k}{2} + \binom{k}{3} - \cdots + (-1)^{k-1}\binom{k}{k} $$
Cela prouve que n'importe quel élément de l'union est compté exactement une fois.
\end{proofbox}

Ce principe est très utile en probabilité, car il permet de calculer $P(A \cup B \cup \dots)$ en se basant sur les probabilités des intersections, qui sont souvent plus faciles à trouver.

\begin{examplebox}[Application probabiliste]
On lance trois dés équilibrés. Quelle est la probabilité d'obtenir au moins un 6 ?

\textbf{Solution avec inclusion-exclusion :}

Soit $A$ = "le premier dé montre 6", $B$ = "le deuxième dé montre 6", $C$ = "le troisième dé montre 6".

On veut $P(A \cup B \cup C)$.

\begin{align*}
P(A \cup B \cup C) &= P(A) + P(B) + P(C) \\
&\quad - P(A \cap B) - P(A \cap C) - P(B \cap C) \\
&\quad + P(A \cap B \cap C) \\
&= \frac{1}{6} + \frac{1}{6} + \frac{1}{6} - \frac{1}{36} - \frac{1}{36} - \frac{1}{36} + \frac{1}{216} \\
&= \frac{3}{6} - \frac{3}{36} + \frac{1}{216} = \frac{1}{2} - \frac{1}{12} + \frac{1}{216} \\
&= \frac{108 - 18 + 1}{216} = \frac{91}{216} \approx 0.421
\end{align*}

\textbf{Vérification par la méthode complémentaire :}

La probabilité de n'obtenir aucun 6 est $\left(\frac{5}{6}\right)^3 = \frac{125}{216}$, donc la probabilité d'au moins un 6 est $1 - \frac{125}{216} = \frac{91}{216}$.
\end{examplebox}
\section{Probabilités et Dénombrement}

\subsection{Concepts fondamentaux}

\begin{intuitionbox}[Nécessité d'un Cadre Formel]
Avant de calculer des probabilités, il est crucial de définir les règles du jeu :
\newline
\textbf{Qu'est-ce qui peut arriver ?}
\newline
On définit l'ensemble de tous les résultats possibles de l'expérience.
\newline
\textbf{À quoi s'intéresse-t-on ?} 
\newline
On identifie les sous-ensembles de résultats spécifiques qui nous intéressent.
\newline
Ces deux idées nous conduisent aux notions d'Univers et d'Événement, qui sont les piliers de toute théorie des probabilités.
\end{intuitionbox}

\begin{definitionbox}[Concepts Fondamentaux]
\textbf{Univers (ou Espace Échantillon), $S$ :} 
\newline
L'ensemble de tous les résultats possibles d'une expérience aléatoire.
\newline
\textbf{Événement, $A$ :} 
\newline
Un sous-ensemble de l'univers ($A \subseteq S$). C'est un ensemble de résultats auxquels on s'intéresse.
\end{definitionbox}

\begin{examplebox}[Univers et Événement]
Pour l'expérience du "lancer d'un dé à six faces" :
\newline
L'\textbf{univers} est $S = \{1, 2, 3, 4, 5, 6\}$.
"Obtenir un nombre impair" est un événement, représenté par le sous-ensemble $A = \{1, 3, 5\}$.
\end{examplebox}

\subsection{Définition Naïve de la Probabilité}

\begin{definitionbox}[Probabilité Naïve]
Pour une expérience où chaque issue dans un espace échantillon fini $S$ est équiprobable, la probabilité d'un événement $A$ est le rapport du nombre d'issues favorables à $A$ sur le nombre total d'issues :
$$ P(A) = \frac{\text{Nombre d'issues favorables}}{\text{Nombre total d'issues}} = \frac{|A|}{|S|} $$
\end{definitionbox}

\begin{examplebox}[Applications de la définition naïve]
\begin{enumerate}
    \item \textbf{Lancer une pièce équilibrée :}
    L'espace échantillon est $S = \{\text{Pile, Face}\}$, donc $|S| = 2$.
    Si l'événement $A$ est "obtenir Pile", alors $A = \{\text{Pile}\}$ et $|A| = 1$.
    La probabilité est $P(A) = \frac{1}{2}$.

    \item \textbf{Lancer un dé à six faces non pipé :}
    L'espace échantillon est $S = \{1, 2, 3, 4, 5, 6\}$, donc $|S| = 6$.
    Si l'événement $B$ est "obtenir un nombre pair", alors $B = \{2, 4, 6\}$ et $|B| = 3$.
    La probabilité est $P(B) = \frac{3}{6} = \frac{1}{2}$.

    \item \textbf{Tirer une carte d'un jeu de 52 cartes :}
    L'espace échantillon $S$ contient 52 cartes, donc $|S| = 52$.
    Si l'événement $C$ est "tirer un Roi", il y a 4 Rois dans le jeu, donc $|C| = 4$.
    La probabilité est $P(C) = \frac{4}{52} = \frac{1}{13}$.
\end{enumerate}
\end{examplebox}

\subsection{Permutations (Arrangements)}

\begin{definitionbox}[Permutation de $k$ objets parmi $n$]
Le nombre de façons d'arranger $k$ objets choisis parmi $n$ objets distincts (où l'ordre compte et il n'y a pas de répétition) est noté $P(n, k)$ ou $A_n^k$ et est défini par :
$$ P(n, k) = \frac{n!}{(n-k)!} $$
où $n!$ est la factorielle de $n$, et par convention $0! = 1$.
\end{definitionbox}

\begin{intuitionbox}[Permutations de $k$ parmi $n$]
Pour placer $k$ objets dans un ordre spécifique en les choisissant parmi $n$ objets disponibles, on a $n$ choix pour la première position, $(n-1)$ choix pour la deuxième, ..., et $(n-k+1)$ choix pour la $k$-ième position. Cela donne $n \times (n-1) \times \cdots \times (n-k+1)$ arrangements. Ce produit contient $k$ termes. Il est égal à $\frac{n!}{(n-k)!}$, car cela revient à diviser la suite complète $n!$ par les facteurs non utilisés $(n-k) \times (n-k-1) \times \cdots \times 1$.
\end{intuitionbox}

\begin{examplebox}[Permutations de $k$ parmi $n$]
\textbf{Podium d'une course :} Une course réunit 8 coureurs. Combien y a-t-il de podiums (1er, 2e, 3e) possibles ? \\
On cherche le nombre de façons d'ordonner 3 coureurs parmi 8 : $P(8, 3)$. 
$$ P(8, 3) = \frac{8!}{(8-3)!} = \frac{8!}{5!} = 8 \times 7 \times 6 = 336 $$
Il y a 336 podiums possibles.
\end{examplebox}

\subsection{Le Coefficient Binomial}

\begin{theorembox}[Formule du Coefficient Binomial]
Le nombre de façons de choisir $k$ objets parmi un ensemble de $n$ objets distincts (sans remise et sans ordre) est donné par le coefficient binomial :
$$ \binom{n}{k} = \frac{n!}{k!(n-k)!} $$
\end{theorembox}

\begin{intuitionbox}

L’idée est de relier $\binom{n}{k}$ à quelque chose de plus facile à compter : les \textbf{permutations} de $k$ objets parmi $n$, c’est-à-dire les listes ordonnées.  
On sait qu’il y en a :
\[
P(n,k) = \frac{n!}{(n-k)!}.
\]

D’un autre côté, on peut construire chaque permutation en deux étapes :
\begin{enumerate}
    \item Choisir un \textbf{sous-ensemble} de $k$ objets (sans ordre), il y a $\binom{n}{k}$ façons de le faire.
    \item Ordonner ces $k$ objets, il y a $k!$ façons de le faire.
\end{enumerate}
Donc, le nombre total de permutations est aussi $\binom{n}{k} \cdot k!$.

\medskip

\noindent En égalisant les deux expressions :
\[
\binom{n}{k} \cdot k! = \frac{n!}{(n-k)!}
\quad\Longrightarrow\quad
\binom{n}{k} = \frac{n!}{k!(n-k)!}.
\]

\medskip

\noindent Pour rendre cela concret, voici le cas $\binom{5}{3}$.  
Il y a 10 sous-ensembles de 3 éléments parmi $\{a,b,c,d,e\}$. Chacun donne lieu à $3! = 6$ permutations.  
Le tableau ci-dessous montre \textbf{toutes les 60 permutations}, regroupées par sous-ensemble :

\begin{center}
\small
\renewcommand{\arraystretch}{0.9}
\setlength{\tabcolsep}{2pt}
\begin{tabular}{|c|c|c|c|c|c|c|c|c|c|}
\hline
\textbf{$\{a,b,c\}$} & \textbf{$\{a,b,d\}$} & \textbf{$\{a,b,e\}$} & \textbf{$\{a,c,d\}$} & \textbf{$\{a,c,e\}$} & \textbf{$\{a,d,e\}$} & \textbf{$\{b,c,d\}$} & \textbf{$\{b,c,e\}$} & \textbf{$\{b,d,e\}$} & \textbf{$\{c,d,e\}$} \\
\hline
$abc$ & $abd$ & $abe$ & $acd$ & $ace$ & $ade$ & $bcd$ & $bce$ & $bde$ & $cde$ \\
\hline
$acb$ & $adb$ & $aeb$ & $adc$ & $aec$ & $aed$ & $bdc$ & $bec$ & $bed$ & $ced$ \\
\hline
$bac$ & $bad$ & $bae$ & $cad$ & $cae$ & $dae$ & $cbd$ & $ceb$ & $dbe$ & $dce$ \\
\hline
$bca$ & $bda$ & $bea$ & $cda$ & $cea$ & $dea$ & $cdb$ & $ceb$ & $deb$ & $dec$ \\
\hline
$cab$ & $dab$ & $eab$ & $dac$ & $eac$ & $ead$ & $dbc$ & $ebc$ & $edb$ & $ecd$ \\
\hline
$cba$ & $dba$ & $eba$ & $dca$ & $eca$ & $eda$ & $dcb$ & $ebc$ & $edb$ & $edc$ \\
\hline
\end{tabular}
\end{center}

\smallskip

Chaque colonne correspond à \textbf{un seul et même choix non ordonné} (par exemple $\{a,b,c\}$), mais à 6 listes différentes selon l’ordre.  
Ainsi, pour obtenir le nombre de \textit{choix non ordonnés}, on divise le nombre total de listes ($60$) par le nombre d’ordres par groupe ($6$) :
\[
\binom{5}{3} = \frac{60}{6} = 10.
\]

\medskip

\noindent C’est exactement ce que fait la formule :
\[
\binom{n}{k} = \frac{\text{nombre de permutations de } k \text{ parmi } n}{k!} = \frac{n!}{k!(n-k)!}.
\]

\end{intuitionbox}

\begin{examplebox}[Utilisation du Coefficient Binomial]
    \textbf{Comité d'étudiants :} De combien de manières peut-on former un comité de 3 étudiants à partir d'une classe de 10 ? L'ordre ne compte pas.
    $$ \binom{10}{3} = \frac{10!}{3!(10-3)!} = \frac{10 \times 9 \times 8}{3 \times 2 \times 1} = 120 \text{ comités possibles.} $$
\end{examplebox}

\subsection{Identité de Vandermonde}

\begin{theorembox}[Identité de Vandermonde]
Cette identité offre une relation remarquable entre les coefficients binomiaux. Pour des entiers non négatifs $m, n$ et $k$, on a :
$$ \binom{m+n}{k} = \sum_{j=0}^{k} \binom{m}{j} \binom{n}{k-j} $$
\end{theorembox}

\begin{intuitionbox}
C'est le "principe du diviser pour régner". Imaginez que vous devez choisir un comité de $k$ personnes à partir d'un groupe contenant $m$ hommes et $n$ femmes.
Le côté gauche, $\binom{m+n}{k}$, compte directement le nombre total de comités possibles.
Le côté droit arrive au même résultat en additionnant toutes les compositions possibles du comité : choisir 0 homme et $k$ femmes, PLUS 1 homme et $k-1$ femmes, PLUS 2 hommes et $k-2$ femmes, etc., jusqu'à choisir $k$ hommes et 0 femme. La somme de toutes ces possibilités doit être égale au total.
\end{intuitionbox}

\begin{examplebox}[Application de l'Identité de Vandermonde]
On veut former un comité de 3 personnes ($k=3$) à partir d'un groupe de 5 hommes ($m=5$) et 4 femmes ($n=4$).
\vspace{0.3cm}
\noindent\textbf{Méthode directe (côté gauche) :} \\
On choisit 3 personnes parmi les $5+4=9$ au total.
$$ \binom{9}{3} = \frac{9 \times 8 \times 7}{3 \times 2 \times 1} = 84 $$
\vspace{0.3cm}
\noindent\textbf{Méthode par cas (côté droit) :} \\
La somme est $\binom{5}{0}\binom{4}{3} + \binom{5}{1}\binom{4}{2} + \binom{5}{2}\binom{4}{1} + \binom{5}{3}\binom{4}{0} = 84$. Les deux méthodes donnent bien le même résultat.
\end{examplebox}

\subsection{Bose-Einstein (Étoiles et Bâtons)}

\begin{theorembox}[Combinaisons avec répétition]
Le nombre de façons de distribuer $k$ objets indiscernables dans $n$ boîtes discernables (ou de choisir $k$ objets parmi $n$ avec remise, où l'ordre ne compte pas) est donné par la formule :
$$ \binom{n+k-1}{k} = \binom{n+k-1}{n-1} $$
\end{theorembox}

\begin{intuitionbox}[Étoiles et Bâtons]
Imaginez que les $k$ objets sont des étoiles ($\star$) et que nous avons besoin de $n-1$ bâtons ($|$) pour les séparer en $n$ groupes. Par exemple, pour distribuer $k=7$ étoiles dans $n=4$ boîtes, une configuration possible serait :
$$ \star\star\star \mid \star \mid \mid \star\star\star $$
Cela correspond à 3 objets dans la première boîte, 1 dans la deuxième, 0 dans la troisième et 3 dans la quatrième.
Le problème revient à trouver le nombre de façons d'arranger ces $k$ étoiles et $n-1$ bâtons. Nous avons un total de $n+k-1$ positions, et nous devons choisir les $k$ positions pour les étoiles (ou les $n-1$ positions pour les bâtons). Le nombre de manières de le faire est précisément $\binom{n+k-1}{k}$.
\end{intuitionbox}

\begin{examplebox}[Distribution de biens identiques]
De combien de manières peut-on distribuer 10 croissants identiques à 4 enfants ?
\newline
Ici, $k=10$ (les croissants, objets indiscernables) et $n=4$ (les enfants, boîtes discernables).
Le nombre de distributions possibles est :
$$ \binom{4+10-1}{10} = \binom{13}{10} = \binom{13}{3} = \frac{13 \times 12 \times 11}{3 \times 2 \times 1} = 13 \times 2 \times 11 = 286 $$
Il y a 286 façons de distribuer les croissants.
\end{examplebox}

\subsection{Principe d'Inclusion-Exclusion}

\begin{theorembox}[Principe d'Inclusion-Exclusion pour 3 ensembles]
Pour trois ensembles finis $A$, $B$ et $C$, le nombre d'éléments dans leur union est donné par :
$$ |A \cup B \cup C| = |A| + |B| + |C| - |A \cap B| - |A \cap C| - |B \cap C| + |A \cap B \cap C| $$
\end{theorembox}

\begin{intuitionbox}[Visualisation avec 3 ensembles]
Le principe d'inclusion-exclusion permet de compter le nombre d'éléments dans une union d'ensembles sans double-comptage. Pour comprendre intuitivement pourquoi on ajoute et soustrait alternativement, considérons trois ensembles $A$, $B$ et $C$ :

\begin{center}
\begin{tikzpicture}[set/.style = {draw,
    circle,
    minimum size = 6cm,
    fill=Rhodamine,
    opacity = 0.4,
    text opacity = 1}]
 
\node (A) [set] {$A$};
\node (B) at (60:4cm) [set] {$B$};
\node (C) at (0:4cm) [set] {$C$};
 
\node at (barycentric cs:A=1,B=1) [left] {$X$};
\node at (barycentric cs:A=1,C=1) [below] {$Y$};
\node at (barycentric cs:B=1,C=1) [right] {$Z$};
\node at (barycentric cs:A=1,B=1,C=1) [] {$T$};
 
\end{tikzpicture}
\end{center}

\textbf{Le problème :} Si on additionne simplement $|A| + |B| + |C|$, on compte certaines zones plusieurs fois :
\begin{itemize}
    \item Les intersections deux à deux ($X$, $Y$, $Z$) sont comptées \textbf{deux fois}
    \item L'intersection triple ($T$) est comptée \textbf{trois fois}
\end{itemize}

\textbf{La solution :} On corrige en soustrayant les intersections deux à deux, mais alors l'intersection triple est comptée :
\begin{itemize}
    \item $+3$ fois dans la somme initiale
    \item $-3$ fois dans la soustraction des intersections deux à deux (car elle appartient à chacune)
    \item Donc $0$ fois au total ! Il faut la rajouter.
\end{itemize}

D'où la formule : $|A \cup B \cup C| = |A| + |B| + |C| - |A \cap B| - |A \cap C| - |B \cap C| + |A \cap B \cap C|$
\end{intuitionbox}

\begin{theorembox}[Principe d'Inclusion-Exclusion généralisé]
Pour $n$ ensembles finis $A_1, A_2, \dots, A_n$, on a :
\begin{align*}
|A_1 \cup A_2 \cup \cdots \cup A_n| = & \sum_{i=1}^n |A_i| \\
& - \sum_{1 \leq i < j \leq n} |A_i \cap A_j| \\
& + \sum_{1 \leq i < j < k \leq n} |A_i \cap A_j \cap A_k| \\
& - \cdots \\
& + (-1)^{n+1} |A_1 \cap A_2 \cap \cdots \cap A_n|
\end{align*}
Ce qui s'écrit plus compactement :
$$ \left| \bigcup_{i=1}^n A_i \right| = \sum_{k=1}^n (-1)^{k+1} \sum_{1 \leq i_1 < i_2 < \cdots < i_k \leq n} |A_{i_1} \cap A_{i_2} \cap \cdots \cap A_{i_k}| $$
\end{theorembox}

\begin{intuitionbox}[Généralisation]
La logique reste la même que pour trois ensembles, mais l'argument clé est de prouver que chaque élément est compté \textbf{exactement une fois}, peu importe le nombre d'ensembles auxquels il appartient.

Supposons qu'un élément $x$ est membre d'exactement $k$ ensembles parmi les $n$ ensembles $A_1, \ldots, A_n$. Analysons combien de fois $x$ est compté dans la formule :
\begin{itemize}
    \item \textbf{Première somme ($\sum |A_i|$)} : $x$ est dans $k$ ensembles, donc il est ajouté $k$ fois. Le nombre de fois est $\binom{k}{1}$.
    
    \item \textbf{Deuxième somme ($-\sum |A_i \cap A_j|$)} : On soustrait $x$ pour chaque paire d'ensembles auxquels il appartient. Il y a $\binom{k}{2}$ telles paires.
    
    \item \textbf{Troisième somme ($+\sum |A_i \cap A_j \cap A_k|$)} : On ajoute de nouveau $x$ pour chaque triplet d'ensembles auxquels il appartient. Il y en a $\binom{k}{3}$.
    
    \item \textbf{Et ainsi de suite...}
\end{itemize}

Au total, l'élément $x$ est compté :
$$ \binom{k}{1} - \binom{k}{2} + \binom{k}{3} - \cdots + (-1)^{k-1}\binom{k}{k} \text{ fois.} $$

Pour voir que cette somme vaut exactement 1, rappelons une identité fondamentale issue du binôme de Newton :
$$ (1-1)^k = \sum_{j=0}^{k} (-1)^j \binom{k}{j} = \binom{k}{0} - \binom{k}{1} + \binom{k}{2} - \cdots + (-1)^k \binom{k}{k} = 0 $$

En réarrangeant cette équation, sachant que $\binom{k}{0}=1$ :
$$ \binom{k}{0} = \binom{k}{1} - \binom{k}{2} + \binom{k}{3} - \cdots - (-1)^{k}\binom{k}{k} $$
$$ 1 = \binom{k}{1} - \binom{k}{2} + \binom{k}{3} - \cdots + (-1)^{k-1}\binom{k}{k} $$

Cela prouve que n'importe quel élément, qu'il soit dans un seul ensemble ($k=1$) ou dans plusieurs ($k>1$), contribue précisément pour 1 au décompte final. Le principe d'inclusion-exclusion est donc une méthode infaillible pour corriger les comptages multiples de manière systématique.
\end{intuitionbox}


\begin{examplebox}[Application probabiliste]
On lance trois dés équilibrés. Quelle est la probabilité d'obtenir au moins un 6 ?

\vspace{0.3cm}
\noindent\textbf{Solution avec inclusion-exclusion :}

Soit $A$ = "le premier dé montre 6", $B$ = "le deuxième dé montre 6", $C$ = "le troisième dé montre 6".

On veut $P(A \cup B \cup C)$.

\begin{align*}
P(A \cup B \cup C) &= P(A) + P(B) + P(C) \\
&\quad - P(A \cap B) - P(A \cap C) - P(B \cap C) \\
&\quad + P(A \cap B \cap C) \\
&= \frac{1}{6} + \frac{1}{6} + \frac{1}{6} - \frac{1}{36} - \frac{1}{36} - \frac{1}{36} + \frac{1}{216} \\
&= \frac{3}{6} - \frac{3}{36} + \frac{1}{216} = \frac{1}{2} - \frac{1}{12} + \frac{1}{216} \\
&= \frac{108 - 18 + 1}{216} = \frac{91}{216} \approx 0.421
\end{align*}

\vspace{0.3cm}
\noindent\textbf{Vérification par la méthode complémentaire :}
La probabilité de n'obtenir aucun 6 est $\left(\frac{5}{6}\right)^3 = \frac{125}{216}$, donc la probabilité d'au moins un 6 est $1 - \frac{125}{216} = \frac{91}{216}$.
\end{examplebox}

\subsection{Exercices}

\begin{exercicebox}[Définition naïve de la probabilité]
On tire une carte d'un jeu de 52 cartes bien battu. Quelle est la probabilité de tirer une carte qui est soit un cœur, soit un Roi ?
\end{exercicebox}

\begin{correctionbox}
Soit $C$ l'événement "tirer un cœur" et $R$ l'événement "tirer un Roi".
Il y a 13 cœurs et 4 Rois. Cependant, le Roi de cœur est compté dans les deux ensembles.
On utilise le principe d'inclusion-exclusion pour les probabilités : $P(C \cup R) = P(C) + P(R) - P(C \cap R)$.
$P(C) = \frac{13}{52}$.
$P(R) = \frac{4}{52}$.
$P(C \cap R)$ (probabilité de tirer le Roi de cœur) = $\frac{1}{52}$.
$P(C \cup R) = \frac{13}{52} + \frac{4}{52} - \frac{1}{52} = \frac{16}{52} = \frac{4}{13}$.
\end{correctionbox}

\begin{exercicebox}[Permutations]
Cinq livres différents (A, B, C, D, E) doivent être rangés sur une étagère.
\begin{enumerate}
    \item Combien de rangements différents sont possibles ?
    \item Si les livres A et B doivent être côte à côte, combien de rangements sont possibles ?
\end{enumerate}
\end{exercicebox}

\begin{correctionbox}
1. Le nombre de façons de ranger 5 objets distincts est une permutation de 5, soit $5!$.
$5! = 5 \times 4 \times 3 \times 2 \times 1 = 120$. Il y a 120 rangements possibles.

2. On peut traiter les livres A et B comme un seul "bloc". Nous avons donc 4 "objets" à ranger : (AB), C, D, E. Il y a $4!$ façons de les ranger.
De plus, à l'intérieur du bloc (AB), les livres peuvent être dans l'ordre AB ou BA, soit $2!$ façons.
Le nombre total de rangements est donc $4! \times 2! = 24 \times 2 = 48$.
\end{correctionbox}

\begin{exercicebox}[Coefficient Binomial]
Une pizzeria propose 12 garnitures différentes. Un client souhaite commander une pizza avec exactement 3 garnitures. Combien de pizzas différentes peut-il composer ?
\end{exercicebox}

\begin{correctionbox}
L'ordre des garnitures ne compte pas, il s'agit donc de choisir 3 garnitures parmi 12. On utilise le coefficient binomial :
$$ \binom{12}{3} = \frac{12!}{3!(12-3)!} = \frac{12 \times 11 \times 10}{3 \times 2 \times 1} = 2 \times 11 \times 10 = 220 $$
Il peut composer 220 pizzas différentes.
\end{correctionbox}

\begin{exercicebox}[Principe d'Inclusion-Exclusion]
Dans une classe de 30 élèves, 15 étudient l'espagnol, 12 l'allemand et 5 étudient les deux langues. Combien d'élèves n'étudient aucune de ces deux langues ?
\end{exercicebox}

\begin{correctionbox}
Soit $E$ le nombre d'élèves étudiant l'espagnol et $A$ le nombre d'élèves étudiant l'allemand.
On cherche le nombre d'élèves qui étudient au moins une langue : $|E \cup A| = |E| + |A| - |E \cap A|$.
$|E \cup A| = 15 + 12 - 5 = 22$.
22 élèves étudient au moins une des deux langues.
Le nombre d'élèves qui n'en étudient aucune est le total moins ce nombre : $30 - 22 = 8$.
\end{correctionbox}

\begin{exercicebox}[Étoiles et Bâtons]
Un investisseur souhaite répartir 8 milliers d'euros (indiscernables) dans 4 fonds d'investissement différents. De combien de manières peut-il le faire ?
\end{exercicebox}

\begin{correctionbox}
C'est un problème de combinaisons avec répétition. On distribue $k=8$ objets (milliers d'euros) dans $n=4$ boîtes (fonds).
On utilise la formule "étoiles et bâtons" :
$$ \binom{n+k-1}{k} = \binom{4+8-1}{8} = \binom{11}{8} = \binom{11}{3} $$
$$ \binom{11}{3} = \frac{11 \times 10 \times 9}{3 \times 2 \times 1} = 11 \times 5 \times 3 = 165 $$
Il y a 165 manières de répartir l'investissement.
\end{correctionbox}

\begin{exercicebox}[Identité de Vandermonde]
Une équipe de 4 personnes doit être formée à partir d'un groupe de 6 physiciens et 5 chimistes. De combien de manières peut-on former l'équipe si elle doit contenir exactement 2 physiciens et 2 chimistes ? Vérifiez le résultat en utilisant l'identité de Vandermonde comme raisonnement.
\end{exercicebox}

\begin{correctionbox}
On doit choisir 2 physiciens parmi 6 ET 2 chimistes parmi 5. Le nombre de manières est le produit des combinaisons :
$$ \binom{6}{2} \binom{5}{2} = \left(\frac{6 \times 5}{2}\right) \times \left(\frac{5 \times 4}{2}\right) = 15 \times 10 = 150 $$
Ceci est un terme de la somme de l'identité de Vandermonde. Le nombre total de comités de 4 personnes parmi 11 ($m=6, n=5, k=4$) serait $\binom{11}{4}$. La somme $\sum_{j=0}^{4} \binom{6}{j} \binom{5}{4-j}$ décompose ce total selon le nombre de physiciens ($j$). Notre calcul correspond au cas $j=2$.
\end{correctionbox}

\begin{exercicebox}[Dénombrement et Probabilité]
On forme un mot de 3 lettres en utilisant les lettres A, B, C, D, E, sans répétition.
\begin{enumerate}
    \item Combien de mots peut-on former ?
    \item Si on choisit un de ces mots au hasard, quelle est la probabilité qu'il contienne la lettre A ?
\end{enumerate}
\end{exercicebox}

\begin{correctionbox}
1. On arrange 3 lettres parmi 5, l'ordre compte. C'est une permutation de 3 parmi 5 :
$P(5, 3) = \frac{5!}{(5-3)!} = 5 \times 4 \times 3 = 60$ mots.

2. Pour calculer la probabilité, on compte le nombre de mots favorables.
Un mot contenant A peut avoir A en 1ère, 2ème ou 3ème position.
Si A est en 1ère position, il reste 2 positions à remplir avec 4 lettres : $P(4, 2) = 4 \times 3 = 12$ mots.
C'est la même chose si A est en 2ème ou 3ème position.
Nombre de mots avec A = $3 \times 12 = 36$.
Probabilité = $\frac{\text{Favorables}}{\text{Total}} = \frac{36}{60} = \frac{3}{5}$.
\end{correctionbox}

\begin{exercicebox}[Principe d'Inclusion-Exclusion à 3 ensembles]
Sur 100 étudiants, 40 suivent le cours de maths, 30 celui de physique et 25 celui de chimie. 10 suivent maths et physique, 8 physique et chimie, 7 chimie et maths. Enfin, 3 suivent les trois cours. Combien d'étudiants ne suivent aucun de ces trois cours ?
\end{exercicebox}

\begin{correctionbox}
Soit M, P, C les ensembles d'étudiants. On cherche le nombre d'étudiants suivant au moins un cours, $|M \cup P \cup C|$ :
$|M \cup P \cup C| = |M| + |P| + |C| - (|M \cap P| + |P \cap C| + |C \cap M|) + |M \cap P \cap C|$
$|M \cup P \cup C| = 40 + 30 + 25 - (10 + 8 + 7) + 3 = 95 - 25 + 3 = 73$.
73 étudiants suivent au moins un cours.
Le nombre d'étudiants n'en suivant aucun est $100 - 73 = 27$.
\end{correctionbox}

\begin{exercicebox}[Coefficients binomiaux et chemins]
Sur une grille, combien de chemins mènent du point (0,0) au point (4,3) en se déplaçant uniquement vers la droite (D) ou vers le haut (H) ?
\end{exercicebox}

\begin{correctionbox}
Pour aller de (0,0) à (4,3), tout chemin doit être composé d'exactement 4 déplacements vers la droite (D) et 3 déplacements vers le haut (H). La longueur totale du chemin est de $4+3=7$ pas.
Le problème revient à trouver le nombre de séquences de 7 lettres contenant 4 'D' et 3 'H'.
C'est équivalent à choisir les 4 positions pour les 'D' parmi les 7 positions totales :
$$ \binom{7}{4} = \frac{7!}{4!3!} = \frac{7 \times 6 \times 5}{3 \times 2 \times 1} = 35 $$
Il y a 35 chemins possibles.
\end{correctionbox}

\begin{exercicebox}[Probabilité et Permutations circulaires]
Six personnes, dont Alice et Bob, s'assoient au hasard autour d'une table ronde. Quelle est la probabilité qu'Alice et Bob soient assis l'un à côté de l'autre ?
\end{exercicebox}

\begin{correctionbox}
Le nombre total d'arrangements de $n$ personnes autour d'une table ronde est $(n-1)!$. Ici, $(6-1)! = 5! = 120$ arrangements.
Pour les cas favorables, on traite Alice et Bob comme un seul bloc. On a donc 5 "entités" à placer, ce qui donne $(5-1)! = 4! = 24$ arrangements.
À l'intérieur du bloc, Alice et Bob peuvent être assis de 2 façons (Alice à gauche de Bob, ou Bob à gauche d'Alice).
Nombre de cas favorables = $24 \times 2 = 48$.
Probabilité = $\frac{\text{Favorables}}{\text{Total}} = \frac{48}{120} = \frac{2}{5}$.
\end{correctionbox}
\newpage

\section{Probabilité conditionnelle}

\begin{intuitionbox}[Question Fondamentale]
La probabilité conditionnelle est le concept qui répond à la question fondamentale : comment devons-nous mettre à jour nos croyances à la lumière des nouvelles informations que nous observons ?
\end{intuitionbox}

\subsection{Définition de la Probabilité Conditionnelle}

\begin{definitionbox}[Probabilité Conditionnelle]
Si $A$ et $B$ sont deux événements avec $P(B) > 0$, alors la probabilité conditionnelle de $A$ sachant $B$, notée $P(A|B)$, est définie comme :
$$P(A|B) = \frac{P(A \cap B)}{P(B)}$$
\end{definitionbox}

\begin{intuitionbox}
Imaginez que l'ensemble de tous les résultats possibles est un grand terrain. Savoir que l'événement $B$ s'est produit, c'est comme si on vous disait que le résultat se trouve dans une zone spécifique de ce terrain. La probabilité conditionnelle $P(A|B)$ ne s'intéresse plus au terrain entier, mais seulement à la proportion de la zone $B$ qui est également occupée par $A$. On "zoome" sur le monde où $B$ est vrai, et on recalcule les probabilités dans ce nouveau monde plus petit.
\end{intuitionbox}

\subsection{Règle du Produit (Intersection de deux événements)}

\begin{theorembox}[Probabilité de l'intersection de deux événements]
Pour tous événements $A$ et $B$ avec des probabilités positives, nous avons :
$$P(A \cap B) = P(A)P(B|A) = P(B)P(A|B)$$
Cela découle directement de la définition de la probabilité conditionnelle.
\end{theorembox}

\begin{intuitionbox}
Pour que deux événements se produisent, le premier doit se produire, PUIS le second doit se produire, sachant que le premier a eu lieu. Cette formule exprime mathématiquement cette idée séquentielle.
\end{intuitionbox}

\begin{examplebox}
Quelle est la probabilité de tirer deux As d'un jeu de 52 cartes sans remise ?
Soit $A$ l'événement "le premier tirage est un As", avec $P(A) = \frac{4}{52}$. Soit $B$ l'événement "le deuxième tirage est un As". Nous cherchons $P(A \cap B)$, que l'on calcule avec la formule $P(A \cap B) = P(A) \times P(B|A)$. La probabilité $P(B|A)$ correspond à tirer un As sachant que la première carte était un As. Il reste alors 51 cartes, dont 3 As. Donc, $P(B|A) = \frac{3}{51}$. Finalement, la probabilité de l'intersection est $P(A \cap B) = \frac{4}{52} \times \frac{3}{51} = \frac{12}{2652} \approx 0.0045$.
\end{examplebox}

\subsection{Règle de la Chaîne (Intersection de n événements)}

\begin{theorembox}[Probabilité de l'intersection de n événements]
Pour tous événements $A_1, \dots, A_n$ avec $P(A_1 \cap A_2 \cap \dots \cap A_{n-1}) > 0$, nous avons :
$$P(A_1 \cap \dots \cap A_n) = P(A_1)P(A_2|A_1)P(A_3|A_1 \cap A_2) \cdots P(A_n|A_1 \cap \dots \cap A_{n-1})$$
\end{theorembox}

\begin{intuitionbox}
Ceci est une généralisation de l'idée précédente, souvent appelée "règle de la chaîne" (chain rule). Pour qu'une séquence d'événements se produise, chaque événement doit se réaliser tour à tour, en tenant compte de tous les événements précédents qui se sont déjà produits.
\end{intuitionbox}

\begin{examplebox}
On tire 3 cartes sans remise. Quelle est la probabilité d'obtenir la séquence Roi, Dame, Valet ?
La probabilité de tirer un Roi en premier ($A_1$) est $P(A_1) = \frac{4}{52}$.
Ensuite, la probabilité de tirer une Dame ($A_2$) sachant qu'un Roi a été tiré est $P(A_2|A_1) = \frac{4}{51}$.
Enfin, la probabilité de tirer un Valet ($A_3$) sachant qu'un Roi et une Dame ont été tirés est $P(A_3|A_1 \cap A_2) = \frac{4}{50}$.
La probabilité totale de la séquence est donc le produit de ces probabilités : $P(A_1 \cap A_2 \cap A_3) = \frac{4}{52} \times \frac{4}{51} \times \frac{4}{50} \approx 0.00048$.
\end{examplebox}

\subsection{Règle de Bayes}

\begin{theorembox}[Règle de Bayes]
$$P(A|B) = \frac{P(B|A)P(A)}{P(B)}$$
\end{theorembox}

\begin{intuitionbox}
La règle de Bayes est la formule pour "inverser" une probabilité conditionnelle. Souvent, il est facile de connaître la probabilité d'un effet étant donné une cause ($P(\text{symptôme}|\text{maladie})$), mais ce qui nous intéresse vraiment, c'est la probabilité de la cause étant donné l'effet observé ($P(\text{maladie}|\text{symptôme})$). La règle de Bayes nous permet de faire ce retournement en utilisant notre connaissance initiale de la probabilité de la cause ($P(\text{maladie})$). C'est le fondement mathématique de la mise à jour de nos croyances.
\end{intuitionbox}

\begin{examplebox}[Dépistage médical]
Une maladie touche 1\% de la population ($P(M) = 0.01$). Un test de dépistage est fiable à 95\% : il est positif pour 95\% des malades ($P(T|M)=0.95$) et négatif pour 95\% des non-malades, ce qui implique un taux de faux positifs de $P(T|\neg M) = 0.05$.
Une personne est testée positive. Quelle est la probabilité qu'elle soit réellement malade, $P(M|T)$ ?
On cherche $P(M|T) = \frac{P(T|M)P(M)}{P(T)}$.
D'abord, on calcule $P(T)$ avec la formule des probabilités totales :
$P(T) = P(T|M)P(M) + P(T|\neg M)P(\neg M) = (0.95 \times 0.01) + (0.05 \times 0.99) = 0.0095 + 0.0495 = 0.059$.
Ensuite, on applique la règle de Bayes : $P(M|T) = \frac{0.95 \times 0.01}{0.059} \approx 0.161$.
Malgré un test positif, il n'y a que 16.1\% de chance que la personne soit malade.
\end{examplebox}

\subsection{Formule des Probabilités Totales}

\begin{theorembox}[Formule des probabilités totales]
Soit $A_1, \dots, A_n$ une partition de l'espace échantillon $S$ (c'est-à-dire que les $A_i$ sont des événements disjoints et leur union est $S$), avec $P(A_i) > 0$ pour tout $i$. Alors pour tout événement $B$ :
$$P(B) = \sum_{i=1}^{n} P(B|A_i)P(A_i)$$
\end{theorembox}

\begin{intuitionbox}
C'est une stratégie de "diviser pour régner". Pour calculer la probabilité totale d'un événement $B$, on peut décomposer le monde en plusieurs scénarios mutuellement exclusifs (la partition $A_i$). On calcule ensuite la probabilité de $B$ dans chacun de ces scénarios ($P(B|A_i)$), on pondère chaque résultat par la probabilité du scénario en question ($P(A_i)$), et on additionne le tout.

\begin{center}
\begin{tikzpicture}
% 1. Dessiner le grand rectangle et les lignes verticales de partition
\draw (0,0) rectangle (12,7);

% 3. Dessiner une grande ellipse pour la forme B
\filldraw[
    fill=gray!30, % Remplissage gris clair
    thick % Trait épais pour le contour
] (6, 3.5) ellipse (5.5cm and 2.5cm); % Centre (6,3.5), rayon x=5.5cm, rayon y=2.5cm

\foreach \x in {2,4,6,8,10} {
    \draw (\x,0) -- (\x,7);
}

% 2. Placer les étiquettes A_1, A_2, ... en bas
\foreach \i [evaluate=\i as \xpos using \i*2-1] in {1,...,6} {
    \node at (\xpos, -0.5) {$A_{\i}$};
}

% 4. Placer l'étiquette pour l'ensemble B
\node at (11, 6) {$B$}; % Ajusté pour être au-dessus de l'ellipse

% 5. Placer les étiquettes pour les intersections B ∩ A_i, toutes au même niveau
\node at (1.2, 3.5) {$B \cap A_1$};
\node at (3, 3.5) {$B \cap A_2$};
\node at (5, 3.5) {$B \cap A_3$};
\node at (7, 3.5) {$B \cap A_4$};
\node at (9, 3.5) {$B \cap A_5$};
\node at (10.8, 3.5) {$B \cap A_6$};
\end{tikzpicture}
\end{center}
\end{intuitionbox}

\begin{examplebox}
Une usine possède trois machines, M1, M2, et M3, qui produisent respectivement 50\%, 30\% et 20\% des articles. Leurs taux de production défectueuse sont de 4\%, 2\% et 5\%. Quelle est la probabilité qu'un article choisi au hasard soit défectueux ?
Soit $D$ l'événement "l'article est défectueux". Les machines forment une partition avec $P(M1)=0.5$, $P(M2)=0.3$, et $P(M3)=0.2$. Les probabilités conditionnelles de défaut sont $P(D|M1)=0.04$, $P(D|M2)=0.02$, et $P(D|M3)=0.05$.
En appliquant la formule, on obtient :
$P(D) = P(D|M1)P(M1) + P(D|M2)P(M2) + P(D|M3)P(M3) = (0.04 \times 0.5) + (0.02 \times 0.3) + (0.05 \times 0.2) = 0.02 + 0.006 + 0.01 = 0.036$.
La probabilité qu'un article soit défectueux est de 3.6\%.
\end{examplebox}

\begin{proofbox}[Démonstration de la formule des probabilités totales]
Puisque les $A_i$ forment une partition de $S$, on peut décomposer $B$ comme :
$$B = (B \cap A_1) \cup (B \cap A_2) \cup \cdots \cup (B \cap A_n)$$
Comme les $A_i$ sont disjoints, les événements $(B \cap A_i)$ le sont aussi. On peut donc sommer leurs probabilités :
$$P(B) = P(B \cap A_1) + P(B \cap A_2) + \cdots + P(B \cap A_n)$$
En appliquant le théorème de l'intersection des probabilités à chaque terme, on obtient :
$$P(B) = P(B|A_1)P(A_1) + P(B|A_2)P(A_2) + \cdots + P(B|A_n) = \sum_{i=1}^{n} P(B|A_i)P(A_i)$$
\end{proofbox}

\subsection{Règle de Bayes avec Conditionnement Additionnel}

\begin{theorembox}[Règle de Bayes avec conditionnement additionnel]
À condition que $P(A \cap E) > 0$ et $P(B \cap E) > 0$, nous avons :
$$P(A|B, E) = \frac{P(B|A, E)P(A|E)}{P(B|E)}$$
\end{theorembox}

\begin{intuitionbox}
Cette formule est simplement la règle de Bayes standard, mais appliquée à l'intérieur d'un univers que l'on a déjà "rétréci".

Imaginez que vous recevez une information \textbf{E} qui élimine une grande partie des possibilités. C'est votre nouveau point de départ, votre monde est plus petit. Toutes les probabilités que vous calculez désormais sont relatives à ce monde restreint.

Dans ce nouveau monde, vous recevez une autre information, l'évidence \textbf{B}. La règle de Bayes conditionnelle vous permet alors de mettre à jour votre croyance sur un événement \textbf{A}, en utilisant exactement la même logique que la règle de Bayes classique, mais en vous assurant que chaque calcul reste confiné à l'intérieur des frontières de l'univers défini par \textbf{E}.
\end{intuitionbox}

\subsection{Formule des Probabilités Totales avec Conditionnement Additionnel}

\begin{theorembox}[Formule des probabilités totales avec conditionnement additionnel]
Soit $A_1, \dots, A_n$ une partition de $S$. À condition que $P(A_i \cap E) > 0$ pour tout $i$, nous avons :
$$P(B|E) = \sum_{i=1}^{n} P(B|A_i, E)P(A_i|E)$$
\end{theorembox}

\begin{intuitionbox}
\begin{center}
\begin{tikzpicture}
  % Matrice principale, nommée "m"
  \matrix (m) [
    matrix of nodes,
    row sep = -\pgflinewidth,
    column sep = -\pgflinewidth,
    nodes={
      rectangle, draw=black, anchor=center,
      text height=4ex, text depth=0.5ex, minimum width=4em, fill=intuitionColor!10
    }
  ]
  {
    | |              & | |              & |[red_hatch]|    & | |              & | |              & | |            \\
    |[red_hatch]|    & |[purple_hatch]| & |[purple_hatch]| & | |              & |[red_hatch]|    & |[red_hatch]|  \\
    |[red_hatch]|    & |[blue_hatch]|   & |[red_hatch]|    & |[red_hatch]|    & |[red_hatch]|    & | |            \\
  };

  % --- DÉLIMITATION DES COLONNES AVEC ACCOLADES ---
  \draw [decorate, decoration={brace, amplitude=5pt, raise=4mm}]
    (m-1-1.north west) -- (m-1-2.north east) 
    node [midway, yshift=8mm, font=\bfseries] {A1};
    
  \draw [decorate, decoration={brace, amplitude=5pt, raise=4mm}]
    (m-1-3.north west) -- (m-1-4.north east) 
    node [midway, yshift=8mm, font=\bfseries] {A2};
    
  \draw [decorate, decoration={brace, amplitude=5pt, raise=4mm}]
    (m-1-5.north west) -- (m-1-6.north east) 
    node [midway, yshift=8mm, font=\bfseries] {A3};
\end{tikzpicture}
\end{center}
Imaginez que le graphique ci-dessus représente la carte d'un trésor. La carte est partitionnée en trois grandes régions : \textbf{A1}, \textbf{A2}, et \textbf{A3}. Sur cette carte, on a identifié deux types de terrains : une \textbf{zone marécageuse} (événement E, hachures rouges) qui s'étend sur \textbf{10 parcelles}, et une \textbf{zone près d'un vieux chêne} (événement B, hachures bleues) qui couvre \textbf{3 parcelles}.

On vous donne un premier indice : "Le trésor est dans la zone marécageuse (E)". Votre univers de recherche se réduit instantanément à ces 10 parcelles rouges. Puis, on vous donne un second indice : "Le trésor est aussi près d'un chêne (B)". Votre recherche se concentre alors sur les parcelles qui sont à la fois marécageuses et proches d'un chêne (les cases violettes, $B \cap E$).

La question est : "Sachant que le trésor est dans une parcelle violette, quelle est la probabilité qu'il se trouve dans la région A2 ?". On cherche donc $P(A_2 | B, E)$. La règle de Bayes nous permet de le calculer.

\textbf{Calcul des termes nécessaires :} D'abord, nous devons évaluer les probabilités à l'intérieur du "monde marécageux" (sachant E).

La \textbf{vraisemblance} est $P(B|A_2, E)$. En se limitant aux 4 parcelles marécageuses de la région A2, une seule est aussi près d'un chêne. Donc, $P(B|A_2, E) = 1/4$.

La \textbf{probabilité a priori} est $P(A_2|E)$. Sur les 10 parcelles marécageuses, 4 sont dans la région A2. Donc, $P(A_2|E) = 4/10$.

L'\textbf{évidence}, $P(B|E)$, est la probabilité de trouver un chêne dans l'ensemble de la zone marécageuse. On peut la calculer avec la formule des probabilités totales :
$$P(B|E) = P(B|A_1, E)P(A_1|E) + P(B|A_2, E)P(A_2|E) + P(B|A_3, E)P(A_3|E)$$
$$P(B|E) = (\frac{1}{3} \times \frac{3}{10}) + (\frac{1}{4} \times \frac{4}{10}) + (0 \times \frac{3}{10}) = \frac{1}{10} + \frac{1}{10} = \frac{2}{10}$$

\textbf{Application de la règle de Bayes :} Maintenant, nous assemblons le tout.
$$P(A_2|B, E) = \frac{P(B|A_2, E)P(A_2|E)}{P(B|E)} = \frac{(1/4) \times (4/10)}{2/10} = \frac{1/10}{2/10} = \frac{1}{2}$$
L'intuition confirme le calcul : sachant que le trésor est sur une parcelle violette, et qu'il n'y en a que deux (une en A1, une en A2), il y a bien une chance sur deux qu'il se trouve dans la région A2.
\end{intuitionbox}

\subsection{Indépendance de Deux Événements}

\begin{definitionbox}[Indépendance de deux événements]
Les événements $A$ et $B$ sont indépendants si :
$$P(A \cap B) = P(A)P(B)$$
Si $P(A) > 0$ et $P(B) > 0$, cela est équivalent à :
$$P(A|B) = P(A)$$
\end{definitionbox}

\begin{intuitionbox}
L'indépendance est l'absence d'information. Si deux événements sont indépendants, apprendre que l'un s'est produit ne change absolument rien à la probabilité de l'autre. Savoir qu'il pleut à Tokyo ($B$) ne modifie pas la probabilité que vous obteniez pile en lançant une pièce ($A$).
\end{intuitionbox}

\subsection{Indépendance Conditionnelle}

\begin{definitionbox}[Indépendance Conditionnelle]
Les événements $A$ et $B$ sont dits conditionnellement indépendants étant donné $E$ si :
$$P(A \cap B | E) = P(A|E)P(B|E)$$
\end{definitionbox}

\begin{intuitionbox}
L'indépendance peut apparaître ou disparaître quand on observe un autre événement. Par exemple, vos notes en maths ($A$) et en physique ($B$) ne sont probablement pas indépendantes. Mais si l'on sait que vous avez beaucoup travaillé ($E$), alors vos notes en maths et en physique pourraient devenir indépendantes. L'information "vous avez beaucoup travaillé" explique la corrélation ; une fois qu'on la connaît, connaître votre note en maths n'apporte plus d'information sur votre note en physique.
\end{intuitionbox}

\subsection{Le Problème de Monty Hall}

\begin{remarquebox}[Le problème de Monty Hall]
Imaginez que vous êtes à un jeu télévisé. Face à vous se trouvent trois portes fermées. Derrière l'une d'elles se trouve une voiture, et derrière les deux autres, des chèvres.
\begin{enumerate}
    \item Vous choisissez une porte (disons, la porte n°1).
    \item L'animateur, qui sait où se trouve la voiture, ouvre une autre porte (par exemple, la n°3) derrière laquelle se trouve une chèvre.
    \item Il vous demande alors : "Voulez-vous conserver votre choix initial (porte n°1) ou changer pour l'autre porte restante (la n°2) ?"
\end{enumerate}
\textbf{Question :} Avez-vous intérêt à changer de porte ? Votre probabilité de gagner la voiture est-elle plus grande si vous changez, si vous ne changez pas, ou est-elle la même dans les deux cas ?
\end{remarquebox}

\begin{correctionbox}[Solution du problème de Monty Hall]
La réponse est sans équivoque : il faut \textbf{toujours changer de porte}. Cette stratégie fait passer la probabilité de gagner de $1/3$ à $2/3$. L'intuition et la preuve ci-dessous détaillent ce résultat surprenant.
\end{correctionbox}

\begin{intuitionbox}[Le secret : l'information de l'animateur]
L'erreur commune est de supposer qu'il reste deux portes avec une chance égale de $1/2$. Cela ignore une information capitale : le choix de l'animateur n'est \textbf{pas aléatoire}. Il sait où se trouve la voiture et ouvrira toujours une porte perdante.

Le raisonnement correct se déroule en deux temps. D'abord, votre choix initial a $\mathbf{1/3}$ de chance d'être correct. Cela implique qu'il y a $\mathbf{2/3}$ de chance que la voiture soit derrière l'une des \textit{deux autres portes}. Ensuite, lorsque l'animateur ouvre l'une de ces deux portes, il ne fait que vous montrer où la voiture n'est \textit{pas} dans cet ensemble. La probabilité de $2/3$ se \textbf{concentre} alors entièrement sur la seule porte qu'il a laissée fermée. Changer de porte revient à miser sur cette probabilité de $2/3$.
\end{intuitionbox}

\begin{proofbox}[Preuve par l'arbre de décision]
L'analyse de la meilleure stratégie peut être visualisée à l'aide de l'arbre de décision ci-dessous. Il décompose le problème en deux scénarios initiaux : avoir choisi la bonne porte (probabilité $1/3$) ou une mauvaise porte (probabilité $2/3$).

\vspace{0.5cm}
\begin{center}
\begin{tikzpicture}[
  grow=right,
  level distance=4.5cm,
  level 1/.style={sibling distance=3cm},
  level 2/.style={sibling distance=2.5cm},
  edge from parent/.style={draw, -latex},
  % --- Définition des styles pour les cadres ---
  porte_style/.style={rectangle, rounded corners, draw=black, fill=gray!20, thick, inner sep=4pt, text width=2.5cm, align=center},
  gain_style/.style={rectangle, rounded corners, draw=green!60!black, fill=green!20, thick, inner sep=4pt},
  perte_style/.style={rectangle, rounded corners, draw=red!60!black, fill=red!20, thick, inner sep=4pt}
]

\node {S}
    % --- Branche du haut ---
    child {
        node[porte_style] {Bonne porte}
        child {
            node[gain_style] {Gain}
            edge from parent
            node[above, sloped] {$1/2$}
        }
        child {
            node[perte_style] {Perte}
            edge from parent
            node[below, sloped] {$1/2$}
        }
        edge from parent
        node[above, sloped] {1/3}
    }
    % --- Branche du bas ---
    child {
        node[porte_style] {Mauvaise porte}
        child {
            node[gain_style] {Gain}
            edge from parent
            node[above, sloped] {1}
        }
        edge from parent
        node[below, sloped] {2/3}
    };
\end{tikzpicture}
\end{center}
\vspace{0.5cm}

\noindent\textbf{Analyse de l'arbre :}

\vspace{0.3cm}
\noindent\textbf{Branche du bas (cas le plus probable) :}
\newline
Avec une probabilité de $\mathbf{2/3}$, votre choix initial se porte sur une "Mauvaise porte". L'animateur est alors obligé de révéler l'autre porte perdante. La seule porte restante est donc la bonne. L'arbre montre que cela mène à un "Gain" avec une probabilité de $\mathbf{1}$. Ce chemin correspond au résultat de la stratégie \textbf{"Changer"}.

\vspace{0.3cm}
\noindent\textbf{Branche du haut (cas le moins probable) :}
\newline
Avec une probabilité de $\mathbf{1/3}$, vous avez choisi la "Bonne porte" du premier coup. L'arbre se divise alors en deux issues équiprobables ($1/2$ chacune). L'issue "Gain" correspond à la stratégie \textbf{"Garder"} votre choix initial, tandis que l'issue "Perte" correspond à la stratégie \textbf{"Changer"} pour la porte perdante restante.

\vspace{0.3cm}
\noindent\textbf{Conclusion :}
\newline
Pour évaluer la meilleure stratégie, il suffit de sommer les probabilités de gain. La \textbf{probabilité de gain en changeant} est de $\mathbf{2/3}$, car vous gagnez uniquement si votre choix initial était mauvais (branche du bas). La \textbf{probabilité de gain en gardant} est de $\mathbf{1/3}$, car vous gagnez uniquement si votre choix initial était bon (branche "Gain" du haut). La stratégie optimale est donc bien de toujours changer de porte.
\end{proofbox}
\newpage

\section{Variables Aléatoires Continues}

\subsection{Fonction de Densité de Probabilité (PDF)}

Nous passons maintenant aux variables aléatoires qui peuvent prendre n'importe quelle valeur dans un intervalle, comme la taille d'une personne ou le temps d'attente exact. Pour ces variables, la notion de PMF n'a plus de sens, car la probabilité d'obtenir une valeur *exacte* est nulle. Nous introduisons donc le concept de densité.

\begin{definitionbox}[Fonction de Densité de Probabilité (PDF)]
Soit $X$ une variable aléatoire continue. Une fonction $f$ est une \textbf{fonction de densité de probabilité} (Probability Density Function, ou PDF) de $X$ si, pour tout $x$ :
\begin{enumerate}
    \item $f(x) \ge 0$, pour tout $-\infty < x < \infty$
    \item $\int_{-\infty}^{\infty} f(x) \, \mathrm{d}x = 1$ (l'aire totale sous la courbe vaut 1)
\end{enumerate}
\end{definitionbox}

Il est crucial de comprendre que $f(x)$ n'est *pas* une probabilité.

\begin{intuitionbox}
Dans le cas discret, la PMF donnait une "masse" de probabilité à chaque point. Dans le cas continu, la probabilité en un point exact est nulle ($P(X=x)=0$). La PDF, $f(x)$, n'est \textbf{pas} une probabilité.

Il faut voir $f(x)$ comme une \textbf{densité} : elle décrit la "concentration" de probabilité autour de $x$. Pour obtenir une probabilité (une "masse"), il faut intégrer cette densité sur un intervalle. La probabilité que $X$ tombe dans un intervalle $[a, b]$ est l'aire sous la courbe de la PDF entre $a$ et $b$ :
$$ P(a \le X \le b) = \int_a^b f(x) \, \mathrm{d}x $$
\end{intuitionbox}

Cette distinction est fondamentale.

\begin{remarquebox}[PDF vs Probabilité]
Une erreur fréquente est de confondre la valeur $f(x)$ avec $P(X=x)$. Pour une variable continue, $P(X=x)$ est \textbf{toujours zéro}. La PDF $f(x)$ peut être supérieure à 1 (contrairement à une probabilité), tant que l'aire totale sous la courbe reste égale à 1. Pensez-y comme à une densité de population : elle peut être très élevée en un point, mais la "population" (probabilité) exacte en ce point infinitésimal est nulle.
\end{remarquebox}

Vérifions un exemple simple.

\begin{examplebox}[Une PDF simple]
Soit $X$ une v.a. avec la PDF $f(x) = 2x$ pour $x \in [0, 1]$, et $f(x)=0$ sinon.
\begin{enumerate}
    \item Est-ce une PDF valide ?
    
    (1) $f(x) \ge 0$ pour tout $x$ dans $[0, 1]$.
    
    (2) $\int_{-\infty}^{\infty} f(x) \, \mathrm{d}x = \int_0^1 2x \, \mathrm{d}x = [x^2]_0^1 = 1-0 = 1$.
    
    Oui, c'est une PDF valide.
    \item Quelle est la probabilité $P(X \le 0.5)$ ?
    $$ P(X \le 0.5) = \int_0^{0.5} 2x \, \mathrm{d}x = [x^2]_0^{0.5} = (0.5)^2 - 0 = 0.25 $$
\end{enumerate}
\end{examplebox}

\subsection{Fonction de Répartition (CDF)}

Comme dans le cas discret, nous pouvons définir une fonction qui accumule la probabilité. Pour le cas continu, cette accumulation se fait via une intégrale.

\begin{definitionbox}[Fonction de Répartition Continue (CDF)]
Soit $X$ une variable aléatoire continue. La \textbf{fonction de répartition} (Cumulative Distribution Function, ou CDF) de $X$ est la fonction $F$ définie par :
$$ F(x) = P(X \le x) = \int_{-\infty}^x f(t) \, \mathrm{d}t $$
Pour être une CDF valide, la fonction $F$ doit respecter les propriétés suivantes :
\begin{enumerate}
    \item $\lim_{x \to \infty} F(x) = 1$
    \item $\lim_{x \to -\infty} F(x) = 0$
    \item $F$ est continue et non décroissante.
\end{enumerate}
\end{definitionbox}

La CDF est l'intégrale de la PDF, et inversement, la PDF est la dérivée de la CDF.

\begin{intuitionbox}
La CDF est "l'accumulateur" de probabilité. Elle part de 0 (à $-\infty$) et "accumule" l'aire sous la PDF à mesure qu'on avance sur l'axe des $x$, pour finalement atteindre 1 (à $+\infty$).

Le lien fondamental est que la PDF est la dérivée de la CDF (par le théorème fondamental de l'analyse) :
$$ f(x) = F'(x) $$
Cela signifie que la valeur de la PDF $f(x)$ représente le \textbf{taux d'accumulation} de la probabilité au point $x$.
\end{intuitionbox}

La CDF est souvent le moyen le plus simple de calculer des probabilités sur des intervalles.

\begin{remarquebox}[Calcul de Probabilités via la CDF]
La CDF est très pratique pour calculer des probabilités sur des intervalles :
$$ P(a < X \le b) = F(b) - F(a) $$
Pour les variables continues, les inégalités strictes ou larges ne changent rien ($P(X=a)=0$).
\end{remarquebox}

Calculons la CDF de notre exemple précédent.

\begin{examplebox}[CDF de l'exemple précédent]
Pour $f(x) = 2x$ sur $[0, 1]$, la CDF $F(x)$ est :
\begin{itemize}
    \item Si $x < 0$ : $F(x) = \int_{-\infty}^x 0 \, \mathrm{d}t = 0$.
    \item Si $0 \le x \le 1$ : $F(x) = \int_{-\infty}^0 f(t) \mathrm{d}t + \int_0^x 2t \, \mathrm{d}t = 0 + [t^2]_0^x = x^2$.
    \item Si $x > 1$ : $F(x) = \int_{-\infty}^1 f(t) \mathrm{d}t + \int_1^x 0 \, \mathrm{d}t = \int_0^1 2t \, \mathrm{d}t = 1$.
\end{itemize}
Donc, $F(x) = \begin{cases} 0 & \text{si } x < 0 \\ x^2 & \text{si } 0 \le x \le 1 \\ 1 & \text{si } x > 1 \end{cases}$
\end{examplebox}

\subsection{Espérance et Variance (Cas Continu)}

Les concepts d'espérance et de variance s'étendent naturellement au cas continu, en remplaçant les sommes par des intégrales.

\begin{definitionbox}[Espérance et Variance (Cas Continu)]
Pour une variable aléatoire $X$ de fonction de densité $f$ :

L'\textbf{espérance} de $X$ est le centre de gravité de la densité :
$$ E[X] = \int_{-\infty}^{\infty} x f(x) \, \mathrm{d}x $$
La \textbf{variance} de $X$ est l'espérance du carré de l'écart à la moyenne :
$$ \text{Var}(X) = E[(X - E[X])^2] = \int_{-\infty}^{\infty} (x - E[X])^2 f(x) \, \mathrm{d}x $$
\end{definitionbox}

Comme dans le cas discret, une formule alternative existe pour la variance.

\begin{theorembox}[Formule de calcul de la Variance]
Une formule plus simple pour le calcul de la variance est :
$$ \text{Var}(X) = E[X^2] - (E[X])^2 $$
où $E[X^2] = \int_{-\infty}^{\infty} x^2 f(x) \, \mathrm{d}x$. (Ceci est une application de LOTUS).
\end{theorembox}

La preuve est identique à celle du cas discret, en utilisant la linéarité de l'espérance.

\begin{proofbox}
Soit $\mu = E(X)$. On part de la définition de la variance :
\begin{align*}
\text{Var}(X) &= E[ (X - \mu)^2 ] \\
&= E[ X^2 - 2X\mu + \mu^2 ] \quad \text{(On développe le carré)} \\
&= E(X^2) - E(2\mu X) + E(\mu^2) \quad \text{(Par linéarité de l'espérance, qui s'applique aussi au cas continu)} \\
&= E(X^2) - 2\mu E(X) + \mu^2 \quad \text{(Car $2\mu$ et $\mu^2$ sont des constantes)} \\
&= E(X^2) - 2\mu(\mu) + \mu^2 \quad \text{(Car $E(X) = \mu$)} \\
&= E(X^2) - 2\mu^2 + \mu^2 \\
&= E(X^2) - \mu^2 = E(X^2) - [E(X)]^2
\end{align*}
\end{proofbox}

Le calcul de $E[X^2]$ (et plus généralement de $E[g(X)]$) repose sur le théorème de transfert, adapté au cas continu.

\begin{theorembox}[Théorème de Transfert (LOTUS)]
Si $X$ est une v.a. continue de densité $f(x)$, et $g$ une fonction, alors :
$$ E[g(X)] = \int_{-\infty}^{\infty} g(x) f(x) \, \mathrm{d}x $$
\end{theorembox}

La preuve formelle est plus avancée, mais l'idée est analogue au cas discret : on pondère chaque valeur $g(x)$ par la densité de probabilité $f(x)$ au voisinage de $x$.

\begin{proofbox}[Idée de la preuve]
La preuve formelle repose sur la théorie de la mesure ou sur un argument de changement de variable pour l'intégrale, en passant par la fonction de répartition de $Y=g(X)$. Intuitivement, pour un petit intervalle $dx$ autour de $x$, la "masse" de probabilité est $f(x)dx$. Cette masse correspond à une valeur $g(x)$ pour la nouvelle variable. L'espérance est la somme (intégrale) de ces valeurs pondérées par leur masse : $\int g(x) f(x)dx$.
\end{proofbox}

La propriété la plus importante de l'espérance reste valide.

\begin{remarquebox}[Linéarité de l'Espérance]
Comme dans le cas discret, l'espérance reste linéaire pour les variables continues :
$E[aX+bY] = aE[X]+bE[Y]$.
\end{remarquebox}

Calculons l'espérance et la variance pour notre exemple.

\begin{examplebox}[Espérance et Variance de l'exemple précédent]
Pour $f(x) = 2x$ sur $[0, 1]$ :

$E[X] = \int_0^1 x \cdot (2x) \, \mathrm{d}x = \int_0^1 2x^2 \, \mathrm{d}x = \left[ \frac{2x^3}{3} \right]_0^1 = \frac{2}{3}$.

$E[X^2] = \int_0^1 x^2 \cdot (2x) \, \mathrm{d}x = \int_0^1 2x^3 \, \mathrm{d}x = \left[ \frac{2x^4}{4} \right]_0^1 = \frac{1}{2}$.

$\text{Var}(X) = E[X^2] - (E[X])^2 = \frac{1}{2} - \left(\frac{2}{3}\right)^2 = \frac{1}{2} - \frac{4}{9} = \frac{9-8}{18} = \frac{1}{18}$.
\end{examplebox}

\subsection{Loi Uniforme}

La loi continue la plus simple est celle où la densité est constante sur un intervalle.

\begin{definitionbox}[Loi Uniforme]
Une variable aléatoire $X$ est \textbf{uniformément distribuée} sur un intervalle $[a, b]$ si sa densité est une constante sur cet intervalle. Pour que l'aire totale soit 1, cette constante doit être $\frac{1}{b-a}$.
$$ f(x) = \begin{cases} \frac{1}{b-a} & \text{pour } x \in [a, b] \\ 0 & \text{sinon} \end{cases} $$
On note cela $X \sim \text{Unif}(a, b)$.
\end{definitionbox}

C'est le modèle du "hasard pur" sur un segment.

\begin{intuitionbox}
C'est la distribution du "hasard pur" dans un intervalle borné. La probabilité de tomber dans un sous-intervalle ne dépend que de la \textbf{longueur} de ce sous-intervalle, pas de sa position (tant qu'il est dans $[a, b]$). 
\end{intuitionbox}

Les propriétés de cette loi sont faciles à dériver par intégration directe.

\begin{theorembox}[Propriétés de la Loi Uniforme]
Si $X \sim \text{Unif}(a, b)$ :
\begin{itemize}
    \item \textbf{CDF :} $F(x) = \frac{x-a}{b-a}$ pour $x \in [a, b]$.
    \item \textbf{Espérance :} $E[X] = \frac{a+b}{2}$ (le point milieu de l'intervalle).
    \item \textbf{Variance :} $\text{Var}(X) = \frac{(b-a)^2}{12}$.
\end{itemize}
\end{theorembox}

\begin{proofbox}[Dérivation des propriétés]
Soit $f(x) = \frac{1}{b-a}$ pour $x \in [a, b]$ et $0$ sinon.

\textbf{CDF :} Pour $x \in [a, b]$,
$$ F(x) = \int_{-\infty}^x f(t) \, dt = \int_a^x \frac{1}{b-a} \, dt = \frac{1}{b-a} [t]_a^x = \frac{x-a}{b-a} $$
(Pour $x<a$, $F(x)=0$. Pour $x>b$, $F(x)=1$.)

\textbf{Espérance :}
$$ E[X] = \int_a^b x \frac{1}{b-a} \, dx = \frac{1}{b-a} \left[ \frac{x^2}{2} \right]_a^b = \frac{1}{b-a} \frac{b^2 - a^2}{2} = \frac{(b-a)(b+a)}{2(b-a)} = \frac{a+b}{2} $$

\textbf{Variance :} D'abord, calculons $E[X^2]$.
$$ E[X^2] = \int_a^b x^2 \frac{1}{b-a} \, dx = \frac{1}{b-a} \left[ \frac{x^3}{3} \right]_a^b = \frac{1}{b-a} \frac{b^3 - a^3}{3} $$
En utilisant $b^3 - a^3 = (b-a)(b^2 + ab + a^2)$, on obtient $E[X^2] = \frac{b^2 + ab + a^2}{3}$.
Maintenant, appliquons la formule $\text{Var}(X) = E[X^2] - (E[X])^2$ :
\begin{align*}
\text{Var}(X) &= \frac{b^2 + ab + a^2}{3} - \left(\frac{a+b}{2}\right)^2 \\
&= \frac{b^2 + ab + a^2}{3} - \frac{a^2 + 2ab + b^2}{4} \\
&= \frac{4(b^2 + ab + a^2) - 3(a^2 + 2ab + b^2)}{12} \\
&= \frac{4b^2 + 4ab + 4a^2 - 3a^2 - 6ab - 3b^2}{12} \\
&= \frac{b^2 - 2ab + a^2}{12} = \frac{(b-a)^2}{12}
\end{align*}
\end{proofbox}

\subsection{Loi Exponentielle}

Passons à une loi fondamentale pour modéliser les temps d'attente.

\begin{definitionbox}[Loi Exponentielle]
Une variable aléatoire $X$ suit une \textbf{loi exponentielle} de paramètre $\lambda > 0$ si sa fonction de densité a la forme :
$$ f(x) = \begin{cases} \lambda e^{-\lambda x} & \text{pour } x \ge 0 \\ 0 & \text{sinon} \end{cases} $$
On note $X \sim \text{Exp}(\lambda)$.
\end{definitionbox}

Cette loi est intimement liée au processus de Poisson.

\begin{intuitionbox}[Lien entre les lois de Poisson et Exponentielle]
La loi exponentielle modélise le temps d'attente \textit{avant} le prochain événement dans un processus de Poisson.

Posons la question : « Si je commence à observer maintenant, combien de temps $T$ vais-je devoir attendre avant de voir le prochain événement ? »
\begin{enumerate}
    \item Dans un processus de Poisson de taux $\lambda$, le nombre d'événements $N(t)$ dans un intervalle de temps $t$ suit une loi de Poisson de paramètre $\lambda t$ :
    $$ P(N(t)=k) = \frac{(\lambda t)^k e^{-\lambda t}}{k!} $$
    \item La probabilité de ne voir \textbf{aucun} événement ($k=0$) pendant une durée $t$ est :
    $$ P(N(t)=0) = \frac{(\lambda t)^0 e^{-\lambda t}}{0!} = e^{-\lambda t} $$
    \item Mais ne voir aucun événement pendant un temps $t$, c'est exactement dire que le temps d'attente $T$ du premier événement est \textit{plus grand} que $t$.
    $$ P(T > t) = P(N(t)=0) = e^{-\lambda t} $$
    \item À partir de là, on déduit la fonction de répartition (CDF) de $T$ :
    $$ F_T(t) = P(T \le t) = 1 - P(T > t) = 1 - e^{-\lambda t} \quad (\text{pour } t \ge 0) $$
    \item En dérivant la CDF pour obtenir la densité (PDF) :
    $$ f_T(t) = F_T'(t) = \frac{d}{dt}(1 - e^{-\lambda t}) = -(-\lambda e^{-\lambda t}) = \lambda e^{-\lambda t} $$
\end{enumerate}
C'est exactement la densité de la loi exponentielle de paramètre $\lambda$.
\end{intuitionbox}

Cette loi possède des propriétés remarquables.

\begin{theorembox}[Propriétés de la Loi Exponentielle]
Si $X \sim \text{Exp}(\lambda)$ :
\begin{itemize}
    \item \textbf{CDF :} $F(x) = 1 - e^{-\lambda x}$ pour $x \ge 0$.
    \item \textbf{Espérance :} $E[X] = \frac{1}{\lambda}$.
    \item \textbf{Variance :} $\text{Var}(X) = \frac{1}{\lambda^2}$.
    \item \textbf{Propriété de non-mémoire :} Pour $s, t \ge 0$, $P(X > s+t \mid X > s) = P(X > t)$.
\end{itemize}
\end{theorembox}

Les preuves de l'espérance et de la variance nécessitent une intégration par parties. La preuve de la non-mémoire est plus directe.

\begin{proofbox}[Dérivation des propriétés]
Soit $f(x) = \lambda e^{-\lambda x}$ pour $x \ge 0$.

\textbf{CDF :} A été dérivée dans l'intuition ci-dessus.
$$ F(x) = \int_0^x \lambda e^{-\lambda t} \, dt = [ -e^{-\lambda t} ]_0^x = -e^{-\lambda x} - (-e^0) = 1 - e^{-\lambda x} $$

\textbf{Espérance :} On utilise l'intégration par parties ($\int u dv = uv - \int v du$) avec $u=x$ et $dv=\lambda e^{-\lambda x}dx$. Alors $du=dx$ et $v=-e^{-\lambda x}$.
\begin{align*}
E[X] &= \int_0^\infty x (\lambda e^{-\lambda x}) \, dx \\
&= \left[ x (-e^{-\lambda x}) \right]_0^\infty - \int_0^\infty (-e^{-\lambda x}) \, dx \\
&= (0 - 0) + \int_0^\infty e^{-\lambda x} \, dx \quad \text{(car } \lim_{x\to\infty} -xe^{-\lambda x} = 0 \text{)} \\
&= \left[ -\frac{1}{\lambda} e^{-\lambda x} \right]_0^\infty = (0) - (-\frac{1}{\lambda} e^0) = \frac{1}{\lambda}
\end{align*}

\textbf{Variance :} D'abord $E[X^2]$. Intégration par parties avec $u=x^2$, $dv=\lambda e^{-\lambda x}dx$. $du=2xdx$, $v=-e^{-\lambda x}$.
\begin{align*}
E[X^2] &= \int_0^\infty x^2 (\lambda e^{-\lambda x}) \, dx \\
&= \left[ x^2 (-e^{-\lambda x}) \right]_0^\infty - \int_0^\infty (-e^{-\lambda x}) (2x \, dx) \\
&= 0 + \int_0^\infty 2x e^{-\lambda x} \, dx \\
&= \frac{2}{\lambda} \int_0^\infty x (\lambda e^{-\lambda x}) \, dx \quad \text{(On fait apparaître } E[X] \text{)} \\
&= \frac{2}{\lambda} E[X] = \frac{2}{\lambda} \left( \frac{1}{\lambda} \right) = \frac{2}{\lambda^2}
\end{align*}
Donc, $\text{Var}(X) = E[X^2] - (E[X])^2 = \frac{2}{\lambda^2} - \left(\frac{1}{\lambda}\right)^2 = \frac{1}{\lambda^2}$.

\textbf{Propriété de non-mémoire :}
Rappelons que $P(X>t) = e^{-\lambda t}$.
\begin{align*}
P(X > s+t \mid X > s) &= \frac{P(X > s+t \text{ et } X > s)}{P(X > s)} \\
&= \frac{P(X > s+t)}{P(X > s)} \quad \text{(car si } X>s+t \text{, alors } X>s \text{)} \\
&= \frac{e^{-\lambda(s+t)}}{e^{-\lambda s}} = \frac{e^{-\lambda s} e^{-\lambda t}}{e^{-\lambda s}} = e^{-\lambda t} \\
&= P(X > t)
\end{align*}
\end{proofbox}

Le paramètre $\lambda$ a une interprétation concrète.

\begin{remarquebox}[Interprétation du paramètre $\lambda$]
Le paramètre $\lambda$ représente le \textbf{taux} moyen d'occurrence des événements dans le processus de Poisson sous-jacent (par exemple, nombre moyen d'appels par minute). L'espérance $1/\lambda$ est alors le \textbf{temps moyen entre les événements}.
\end{remarquebox}

La propriété de non-mémoire est unique à la loi exponentielle (dans le cas continu).

\begin{intuitionbox}[La Propriété de Non-Mémoire]
C'est la propriété la plus contre-intuitive et la plus importante de la loi exponentielle. Elle signifie que le processus "oublie" le passé. Si vous attendez un bus qui arrive selon un processus de Poisson (et donc le temps d'attente suit une loi exponentielle), et que vous avez déjà attendu 5 minutes ($X>5$), la probabilité que vous deviez attendre encore au moins 2 minutes ($X>5+2$) est la même que si vous veniez juste d'arriver à l'arrêt et deviez attendre au moins 2 minutes ($X>2$). L'information "j'ai déjà attendu 5 minutes" est inutile pour prédire l'attente future.
\end{intuitionbox}

\subsection{Distributions Conjointes (Cas Continu)}

Comme pour le cas discret, nous pouvons définir des lois conjointes pour plusieurs variables aléatoires continues.

\begin{definitionbox}[Fonction de Densité Conjointe]
Pour des variables aléatoires continues $X$ et $Y$, la \textbf{fonction de densité conjointe} $f(x, y)$ décrit la densité de probabilité sur le plan $(x, y)$. Elle doit respecter :
\begin{enumerate}
    \item $f(x, y) \ge 0$, pour tous $x, y$.
    \item $\int_{-\infty}^{\infty} \int_{-\infty}^{\infty} f(x, y) \, \mathrm{d}x \, \mathrm{d}y = 1$.
\end{enumerate}
\end{definitionbox}

Ici, la probabilité est associée à un volume sous la surface de densité.

\begin{intuitionbox}[Volume = Probabilité]
La probabilité que le couple $(X, Y)$ tombe dans une région $A$ du plan $xy$ est le \textbf{volume} sous la surface $z=f(x,y)$ au-dessus de cette région $A$.
$$ P((X, Y) \in A) = \iint_A f(x, y) \, \mathrm{d}A $$

\end{intuitionbox}

On retrouve les lois marginales en intégrant (en "écrasant" le volume).

\begin{definitionbox}[Densités Marginales]
On peut retrouver les densités individuelles (marginales) en "écrasant" le volume 3D sur un seul axe. Pour obtenir la PDF de $X$ seul, on intègre $f(x,y)$ sur toutes les valeurs possibles de $y$ :
$$ f_X(x) = \int_{-\infty}^{\infty} f(x, y) \, \mathrm{d}y $$
$$ f_Y(y) = \int_{-\infty}^{\infty} f(x, y) \, \mathrm{d}x $$
\end{definitionbox}

La CDF conjointe accumule ce volume.

\begin{definitionbox}[CDF Conjointe]
La \textbf{fonction de répartition conjointe} (CDF) est :
$$ F(x, y) = P(X \le x, Y \le y) = \int_{-\infty}^y \int_{-\infty}^x f(s, t) \, \mathrm{d}s \, \mathrm{d}t $$
Elle représente le volume "au sud-ouest" du point $(x, y)$.
\end{definitionbox}

\subsection{Espérance, Indépendance et Covariance (Cas Conjoint)}

Les concepts clés s'étendent naturellement au cas conjoint continu.

\begin{theorembox}[LOTUS pour les v.a. conjointes]
Si $X$ et $Y$ ont une densité conjointe $f(x, y)$, et $g(x, y)$ est une fonction :
$$ E[g(X, Y)] = \int_{-\infty}^{\infty} \int_{-\infty}^{\infty} g(x, y) f(x, y) \, \mathrm{d}x \, \mathrm{d}y $$
\end{theorembox}

La preuve est analogue à celle de LOTUS 1D, mais en dimension supérieure.

\begin{proofbox}[Idée de la preuve]
Comme pour LOTUS 1D, la preuve rigoureuse utilise des arguments de théorie de la mesure. L'intuition est que pour un petit rectangle $dx dy$ autour de $(x,y)$, la "masse" de probabilité est $f(x,y)dx dy$. Cette masse correspond à la valeur $g(x,y)$. L'espérance est la somme (double intégrale) de ces valeurs $g(x,y)$ pondérées par leur masse $f(x,y)dx dy$.
\end{proofbox}

La condition d'indépendance s'exprime via la factorisation de la densité.

\begin{definitionbox}[Indépendance et Densité]
Les variables aléatoires continues $X$ et $Y$ sont \textbf{indépendantes} si et seulement si leur densité conjointe est le produit de leurs densités marginales :
$$ f(x, y) = f_X(x) f_Y(y), \quad \text{pour tous } x, y $$
\end{definitionbox}

Cela signifie que le profil selon $x$ ne dépend pas de $y$.

\begin{intuitionbox}
Intuitivement, l'indépendance signifie que le "profil" de la densité en $x$ ne change pas quelle que soit la valeur de $y$ (et vice-versa). La surface de densité $z=f(x,y)$ peut être "séparée" en une fonction de $x$ multipliée par une fonction de $y$.
\end{intuitionbox}

La covariance se définit et se calcule de manière similaire.

\begin{definitionbox}[Covariance (cas continu)]
La \textbf{covariance} de $X$ et $Y$ mesure leur variation linéaire commune :
$$ \text{Cov}(X, Y) = E[(X - \mu_X)(Y - \mu_Y)] $$
$$ = \int_{-\infty}^{\infty} \int_{-\infty}^{\infty} (x - \mu_X)(y - \mu_Y) f(x, y) \, \mathrm{d}x \, \mathrm{d}y $$
\end{definitionbox}

La formule de calcul reste la même.

\begin{theorembox}[Formule de calcul de la Covariance]
Une formule plus simple pour le calcul de la covariance est :
$$ \text{Cov}(X, Y) = E[XY] - E[X]E[Y] $$
où $E[XY]$ est calculé via LOTUS : $E[XY] = \iint xy f(x, y) \, \mathrm{d}x \mathrm{d}y$.
\end{theorembox}

La preuve est identique à celle du cas discret.

\begin{proofbox}
La preuve est identique à celle vue pour les variables discrètes, car elle ne repose que sur la linéarité de l'espérance, qui est vraie aussi dans le cas continu.
Soit $\mu_X = E[X]$ et $\mu_Y = E[Y]$.
\begin{align*}
\text{Cov}(X,Y) &= E[(X - \mu_X)(Y - \mu_Y)] \\
&= E[XY - X\mu_Y - Y\mu_X + \mu_X\mu_Y] \\
&= E[XY] - E[X\mu_Y] - E[Y\mu_X] + E[\mu_X\mu_Y] \\
&= E[XY] - \mu_Y E[X] - \mu_X E[Y] + \mu_X\mu_Y \\
&= E[XY] - \mu_Y \mu_X - \mu_X \mu_Y + \mu_X\mu_Y \\
&= E[XY] - E[X]E[Y]
\end{align*}
\end{proofbox}

La relation entre indépendance et covariance reste la même.

\begin{remarquebox}[Indépendance et Covariance]
Si $X$ et $Y$ sont indépendantes, alors $\text{Cov}(X, Y) = 0$. Cependant, la réciproque n'est \textbf{pas} toujours vraie pour les variables aléatoires en général (bien qu'elle le soit dans certains cas importants comme pour les variables gaussiennes). Une covariance nulle signifie seulement une absence de \textit{relation linéaire}, mais il peut exister d'autres formes de dépendance.
\end{remarquebox}


\subsection{Espérance d'une variable aléatoire continue}

Lorsque la variable aléatoire $X$ est continue, sa distribution est décrite par une fonction de densité de probabilité (PDF), $f(x)$. L'espérance est définie de manière analogue, en remplaçant la somme par une intégrale.

\begin{definitionbox}[Espérance (cas continu)]
L'espérance (ou valeur attendue) d'une variable aléatoire continue $X$ avec une fonction de densité $f(x)$ est définie par :
$$E(X) = \int_{-\infty}^{\infty} x f(x) \, dx$$
L'intégrale doit être absolument convergente, c'est-à-dire $\int_{-\infty}^{\infty} |x| f(x) \, dx < \infty$.
\end{definitionbox}

\begin{intuitionbox}
L'intuition du \textbf{centre de gravité} est toujours valable. Si la fonction de densité $f(x)$ représente la répartition de la masse sur une tige (l'axe des $x$), alors $E(X)$ est le point d'équilibre où la tige tiendrait en balance.
\end{intuitionbox}

\begin{examplebox}[Loi uniforme]
Soit $X \sim \mathcal{U}(a, b)$. Sa densité est $f(x) = \frac{1}{b-a}$ pour $x \in [a, b]$, et 0 ailleurs.
\begin{align*}
E(X) &= \int_{-\infty}^{\infty} x f(x) \, dx = \int_{a}^{b} x \left( \frac{1}{b-a} \right) \, dx \\
&= \frac{1}{b-a} \left[ \frac{x^2}{2} \right]_a^b = \frac{1}{b-a} \left( \frac{b^2 - a^2}{2} \right) \\
&= \frac{1}{b-a} \frac{(b-a)(b+a)}{2} = \frac{a+b}{2}
\end{align*}
L'espérance est le point milieu de l'intervalle, ce qui est intuitivement correct.
\end{examplebox}

\subsection{Linéarité de l'espérance}

Le calcul de l'espérance deviendrait très fastidieux si nous devions toujours utiliser la définition. Heureusement, l'espérance possède une propriété fondamentale qui simplifie énormément les calculs.

\begin{theorembox}[Linéarité de l'espérance]
Pour toutes variables aléatoires $X$ et $Y$ (discrètes ou continues), et pour toute constante $c$, on a :
\begin{align*}
E(X+Y) &= E(X) + E(Y) \\
E(cX) &= cE(X)
\end{align*}
Cette propriété est extrêmement puissante car elle ne requiert pas que $X$ et $Y$ soient indépendantes.
\end{theorembox}

\begin{proofbox}
La première propriété est directe.
\begin{itemize}
    \item \textbf{Cas continu :} $E(cX) = \int (cx) f(x) dx = c \int x f(x) dx = cE(X)$
\end{itemize}
Pour la seconde, $E(X+Y) = E(X) + E(Y)$ :

\textbf{Cas continu :} La preuve est identique en remplaçant les sommes par des intégrales et la PMF jointe par la PDF jointe $f(x, y)$:
\begin{align*}
E(X+Y) &= \int_{-\infty}^{\infty} \int_{-\infty}^{\infty} (x+y) f(x, y) \, dx \, dy \\
&= \int_{-\infty}^{\infty} \int_{-\infty}^{\infty} x f(x, y) \, dx \, dy + \int_{-\infty}^{\infty} \int_{-\infty}^{\infty} y f(x, y) \, dx \, dy \\
&= \int_{-\infty}^{\infty} x \left( \int_{-\infty}^{\infty} f(x, y) \, dy \right) \, dx + \int_{-\infty}^{\infty} y \left( \int_{-\infty}^{\infty} f(x, y) \, dx \right) \, dy
\end{align*}
Les intégrales internes sont les densités marginales $f_X(x) = \int f(x, y) dy$ et $f_Y(y) = \int f(x, y) dx$.
$$E(X+Y) = \int_{-\infty}^{\infty} x f_X(x) \, dx + \int_{-\infty}^{\infty} y f_Y(y) \, dy = E(X) + E(Y)$$
Notez que l'indépendance n'a jamais été requise pour cette preuve.
\end{proofbox}

\begin{intuitionbox}
Cette propriété formalise une idée très simple : "la moyenne d'une somme est la somme des moyennes". Si vous jouez à deux jeux de hasard, votre gain moyen total est simplement la somme de ce que vous gagnez en moyenne à chaque jeu, que les jeux soient liés ou non.
\end{intuitionbox}

\subsection{Loi du statisticien inconscient (LOTUS)}

Souvent, nous ne sommes pas intéressés par l'espérance de $X$ elle-même, mais par l'espérance d'une fonction de $X$, par exemple $E(X^2)$ ou $E(e^X)$.

\begin{theorembox}[Théorème de Transfert (LOTUS)]
Si $X$ est une variable aléatoire continue et $g(x)$ est une fonction de $\mathbb{R}$ dans $\mathbb{R}$, alors l'espérance de la variable aléatoire $g(X)$ est donnée par :
\begin{itemize}
    \item \textbf{Cas continu :} $E[g(X)] = \int_{-\infty}^{\infty} g(x) f_X(x) \, dx$
\end{itemize}
Ce théorème est utile car il évite d'avoir à trouver la distribution (PDF) de $g(X)$.
\end{theorembox}

\begin{proofbox}
La preuve pour le cas continu est plus technique (utilisant un changement de variable) et est omise.
\end{proofbox}

Ce théorème justifie son nom : c'est ce que l'on ferait "inconsciemment".

\begin{intuitionbox}
Pour trouver la valeur moyenne d'une fonction d'une variable aléatoire, vous n'avez pas besoin de déterminer d'abord la distribution de cette fonction. Vous pouvez simplement prendre chaque valeur possible du résultat original, lui appliquer la fonction, et pondérer ce nouveau résultat par la densité du résultat original.
\end{intuitionbox}

\begin{examplebox}[Calcul de $E(X^2)$ pour une loi uniforme (continu)]
Soit $X \sim \mathcal{U}(0, 1)$. Sa densité est $f(x)=1$ sur $[0, 1]$. Calculons l'espérance de $Y=X^2$. La fonction est $g(x)=x^2$.
\begin{align*}
E(X^2) &= \int_{-\infty}^{\infty} g(x) f(x) \, dx = \int_{0}^{1} x^2 \cdot 1 \, dx \\
&= \left[ \frac{x^3}{3} \right]_0^1 = \frac{1}{3}
\end{align*}
\end{examplebox}

\subsection{Variance}

L'espérance nous donne le centre d'une distribution, mais elle ne dit rien sur sa "largeur" ou sa "dispersion". C'est le rôle de la variance.

\begin{definitionbox}[Variance et écart-type]
La \textbf{variance} d'une variable aléatoire $X$ mesure la dispersion de sa distribution autour de son espérance $\mu = E(X)$. Elle est définie par :
$$\text{Var}(X) = E\left[ (X - \mu)^2 \right]$$
Concrètement, cela se traduit par (en utilisant LOTUS avec $g(x)=(x-\mu)^2$) :
\begin{itemize}
    \item \textbf{Cas continu :} $\text{Var}(X) = \int_{-\infty}^{\infty} (x - \mu)^2 f(x) \, dx$
\end{itemize}
La racine carrée de la variance est appelée l' \textbf{écart-type} :
$$\text{SD}(X) = \sqrt{\text{Var}(X)}$$
\end{definitionbox}

L'idée est de mesurer l'écart quadratique moyen à l'espérance.

\begin{intuitionbox}
La variance est la "distance carrée moyenne à la moyenne". On prend l'écart de chaque valeur par rapport à la moyenne, on le met au carré (pour que les écarts positifs et négatifs ne s'annulent pas), puis on en calcule la moyenne. L'écart-type est souvent plus interprétable car il ramène cette mesure de dispersion dans les mêmes unités que la variable aléatoire elle-même.
\end{intuitionbox}

La définition $E[(X-\mu)^2]$ est excellente pour l'interprétation, mais pénible pour le calcul. Une formule alternative est presque toujours utilisée.

\begin{theorembox}[Formule de calcul de la variance]
Pour toute variable aléatoire $X$ (discrète ou continue), une formule plus pratique pour le calcul de la variance est :
$$\text{Var}(X) = E(X^2) - [E(X)]^2$$
\end{theorembox}

La preuve est une simple expansion algébrique utilisant la linéarité de l'espérance.

\begin{proofbox}
Soit $\mu = E(X)$. On part de la définition de la variance :
\begin{align*}
\text{Var}(X) &= E[ (X - \mu)^2 ] \\
&= E[ X^2 - 2X\mu + \mu^2 ] \quad \text{(On développe le carré)} \\
&= E(X^2) - E(2\mu X) + E(\mu^2) \quad \text{(Par linéarité de l'espérance)} \\
&= E(X^2) - 2\mu E(X) + \mu^2 \quad \text{(Car $2\mu$ et $\mu^2$ sont des constantes)} \\
&= E(X^2) - 2\mu(\mu) + \mu^2 \quad \text{(Car $E(X) = \mu$)} \\
&= E(X^2) - 2\mu^2 + \mu^2 \\
&= E(X^2) - \mu^2 = E(X^2) - [E(X)]^2
\end{align*}
\end{proofbox}

\begin{examplebox}[Variance de la loi uniforme]
Calculons la variance de $X \sim \mathcal{U}(a, b)$.
Nous avons trouvé $E(X) = \frac{a+b}{2}$.
Nous devons d'abord calculer $E(X^2)$ en utilisant LOTUS :
\begin{align*}
E(X^2) &= \int_a^b x^2 f(x) \, dx = \int_a^b x^2 \left( \frac{1}{b-a} \right) \, dx \\
&= \frac{1}{b-a} \left[ \frac{x^3}{3} \right]_a^b = \frac{1}{b-a} \left( \frac{b^3 - a^3}{3} \right) \\
&= \frac{1}{b-a} \frac{(b-a)(a^2+ab+b^2)}{3} = \frac{a^2+ab+b^2}{3}
\end{align*}
On utilise maintenant la formule de calcul $\text{Var}(X) = E(X^2) - [E(X)]^2$ :
\begin{align*}
\text{Var}(X) &= \frac{a^2+ab+b^2}{3} - \left( \frac{a+b}{2} \right)^2 \\
&= \frac{a^2+ab+b^2}{3} - \frac{a^2+2ab+b^2}{4} \\
&= \frac{4(a^2+ab+b^2) - 3(a^2+2ab+b^2)}{12} \\
&= \frac{4a^2+4ab+4b^2 - 3a^2-6ab-3b^2}{12} \\
&= \frac{a^2-2ab+b^2}{12} = \frac{(b-a)^2}{12}
\end{align*}


\end{examplebox}

\subsection{Changement de Variable pour les Variables Aléatoires Continues}

Un problème courant en probabilité est de déterminer la densité d'une variable aléatoire $Y = g(X)$, où $X$ est une variable aléatoire continue de densité connue $f_X(x)$ et $g$ est une fonction. Cette technique est fondamentale pour transformer des distributions.

\begin{theorembox}[Changement de Variable - Cas Monotone]
Soit $X$ une variable aléatoire continue de densité $f_X(x)$ supportée sur un intervalle $I$. Soit $g : I \to J$ une fonction strictement monotone (croissante ou décroissante) et dérivable, avec $g'(x) \ne 0$ sur $I$. Alors $Y = g(X)$ est une variable aléatoire continue de densité :
$$ f_Y(y) = f_X(g^{-1}(y)) \cdot \left| \frac{d}{dy} g^{-1}(y) \right| $$
pour $y \in J$, et $f_Y(y) = 0$ ailleurs.
\end{theorembox}

\begin{proofbox}[Preuve du Théorème]
La preuve repose sur la relation entre la CDF de $Y$ et celle de $X$.

\textbf{Cas 1 : $g$ strictement croissante.}
$$ F_Y(y) = P(Y \le y) = P(g(X) \le y) = P(X \le g^{-1}(y)) = F_X(g^{-1}(y)) $$
En dérivant par rapport à $y$ :
$$ f_Y(y) = \frac{d}{dy} F_X(g^{-1}(y)) = f_X(g^{-1}(y)) \cdot \frac{d}{dy} g^{-1}(y) $$
Comme $g$ est croissante, $g^{-1}$ est croissante, donc $\frac{d}{dy} g^{-1}(y) > 0$. On a donc :
$$ f_Y(y) = f_X(g^{-1}(y)) \cdot \left| \frac{d}{dy} g^{-1}(y) \right| $$

\textbf{Cas 2 : $g$ strictement décroissante.}
$$ F_Y(y) = P(Y \le y) = P(g(X) \le y) = P(X \ge g^{-1}(y)) = 1 - F_X(g^{-1}(y)) $$
En dérivant par rapport à $y$ :
$$ f_Y(y) = -f_X(g^{-1}(y)) \cdot \frac{d}{dy} g^{-1}(y) $$
Comme $g$ est décroissante, $g^{-1}$ est décroissante, donc $\frac{d}{dy} g^{-1}(y) < 0$. On a donc :
$$ f_Y(y) = f_X(g^{-1}(y)) \cdot \left| \frac{d}{dy} g^{-1}(y) \right| $$
\end{proofbox}




\begin{examplebox}[Changement d'unité : distance entre véhicules sur une autoroute]
\textbf{Contexte:} Supposons que l’on modélise le trafic sur une portion isolée d’autoroute en zone rurale. Les véhicules passent rarement, de manière indépendante les uns des autres, à des instants aléatoires. Dans ce cadre, le nombre de voitures observées sur un intervalle de temps fixé suit une \textbf{loi de Poisson}, ce qui implique naturellement que le \textbf{temps entre deux passages successifs} suit une loi exponentielle. Par extension, si l’on suppose une vitesse constante moyenne, alors la \textbf{distance entre deux véhicules consécutifs} peut également être modélisée par une loi exponentielle.

On s’intéresse précisément à cette distance \( X \), exprimée en \textbf{mètres}. On suppose qu’elle suit une loi exponentielle de paramètre \( \lambda = 0.001 \), ce qui correspond à une distance moyenne de \( 1/\lambda = 1000 \) mètres entre deux voitures — une situation réaliste pour une très faible densité de circulation.

Ainsi, la densité de \( X \) est :
\[
f_X(x) = 
\begin{cases}
0.001 \, e^{-0.001x} & x \geq 0 \\
0 & x < 0
\end{cases}
\]

Un ingénieur souhaite maintenant exprimer cette distance en \textbf{kilomètres} pour des raisons pratiques (rapports, comparaison avec d'autres données). On définit donc :
\[
Y = \frac{X}{1000}
\]
Notre objectif est de déterminer la densité \( f_Y(y) \) de \( Y \) en suivant rigoureusement la méthode de changement de variable présentée dans le théorème précédent.

\textbf{Étape 1 : Identifier la transformation}  
La fonction \( g \) est définie par :
\[
g(x) = \frac{x}{1000}
\]
C’est une fonction \textbf{strictement croissante} et dérivable sur \( [0, \infty[ \), donc le théorème s’applique.

\textbf{Étape 2 : Calculer la réciproque}  
\[
y = \frac{x}{1000} \quad \Rightarrow \quad x = 1000y \quad \Rightarrow \quad g^{-1}(y) = 1000y
\]

\textbf{Étape 3 : Calculer la dérivée de la réciproque}  
\[
\frac{d}{dy} g^{-1}(y) = \frac{d}{dy}(1000y) = 1000
\]
Donc :
\[
\left| \frac{d}{dy} g^{-1}(y) \right| = 1000
\]

\textbf{Étape 4 : Appliquer la formule du changement de variable}  
D’après le théorème :
\[
f_Y(y) = f_X(g^{-1}(y)) \cdot \left| \frac{d}{dy} g^{-1}(y) \right| = f_X(1000y) \cdot 1000
\]

Substituons \( f_X(1000y) \) :
\[
f_X(1000y) = 0.001 \, e^{-0.001 \cdot (1000y)} = 0.001 \, e^{-y}
\]
Donc :
\[
f_Y(y) = 0.001 \, e^{-y} \cdot 1000 = e^{-y}, \quad y \geq 0
\]

\textbf{Résultat final :}
\[
f_Y(y) = 
\begin{cases}
e^{-y} & y \geq 0 \\
0 & y < 0
\end{cases}
\]
Ce qui signifie que \( Y \sim \mathcal{E}(1) \), une loi exponentielle de paramètre \( 1 \), dont l’espérance est \( \mathbb{E}[Y] = 1 \) kilomètre.

\textbf{Vérification probabiliste :}  
Calculons la probabilité que la distance entre deux véhicules soit inférieure à 500 mètres.

- En mètres : \( P(X < 500) = 1 - e^{-0.001 \times 500} = 1 - e^{-0.5} \approx 0.3935 \)
- En kilomètres : \( 500 \, \text{m} = 0.5 \, \text{km} \), donc \( P(Y < 0.5) = 1 - e^{-1 \times 0.5} = 1 - e^{-0.5} \approx 0.3935 \)

Les probabilités sont identiques, confirmant la validité du changement de variable.

\textbf{Conclusion :}  
Le passage d’une unité à une autre (ici, mètres → kilomètres) modifie la forme apparente de la densité, mais respecte la structure probabiliste sous-jacente. Le facteur jacobien \( \left| \frac{d}{dy} g^{-1}(y) \right| = 1000 \) compense exactement la compression de l’échelle, garantissant que la nouvelle densité intègre bien à 1 et que toutes les probabilités restent cohérentes avec la réalité physique du phénomène modélisé.
\end{examplebox}
\newpage
\section{La Loi Gamma}

\subsection{Petite promenade avant le grand plongeon}

Avant que les équations n’apparaissent, laissez-nous quelques instants de flânerie.  
Imaginez-vous debout (ou assis(e), selon l’humeur) à un arrêt de bus dont l’horaire est… on va dire « créativement imprévisible ». Les véhicules arrivent sans régularité apparente, mais avec une propriété étonnante : peu importe le temps déjà écoulé depuis le dernier bus, la probabilité qu’un nouveau bus arrive dans la minute qui suit reste la même. Cette absence de mémoire est la signature d’une loi exponentielle.

Maintenant, si vous décidez de ne \textbf{pas} prendre le premier bus qui se présente, ni le deuxième, mais d’attendre patiemment d’en voir \textbf{n} passer avant de lever le pouce, vous venez de créer — sans vous en douter — une variable aléatoire \textbf{Gamma}. Le temps total que vous passerez sur le trottoir est la \textbf{some} de \textbf{n} durées exponentielles indépendantes. Et cette somme a une forme, une densité, des moments… bref, une personnalité propre : la \textbf{loi Gamma}.

\subsection{Intuition « assis(e) à l’arrêt de bus »}

\begin{intuitionbox}[Pourquoi la somme ?]
\begin{itemize}
  \item Chaque bus arrive \emph{sans horaire fixe} ; l’écart entre deux bus est modelé par une loi exponentielle $\mathcal{E}(\lambda)$.
  \item Tu \textbf{ne bouges pas} : tu démarres ton chronomètre quand le premier bus arrive et tu l’arrêtes quand le \emph{n-ième} bus passe.
  \item Le temps total que tu vas passer assis(e) est donc
        \[ Y = X_1 + X_2 + \dots + X_n \]
        où les $X_i$ sont les durées \emph{entre} bus, indépendantes et de même paramètre $\lambda$.
\end{itemize}
\smallskip
\underline{Point-clé} : une exponentielle modélise \emph{un} temps d’attente ; une Gamma modélise la \emph{some} de plusieurs de ces temps.
\end{intuitionbox}

\subsection{La loi Gamma en deux lignes}

\begin{definitionbox}[Loi Gamma]
Une variable aléatoire $Y$ suit une loi Gamma de paramètres de forme $k > 0$ et de taux $\lambda > 0$, notée
\[ Y \sim \text{Gamma}(k, \lambda), \]
si sa densité est
\[ f_Y(y) = \frac{\lambda^k y^{k-1} e^{-\lambda y}}{\Gamma(k)}, \qquad y > 0. \]
Pour $k$ entier on a $\Gamma(k) = (k-1)!$ ; c’est le cas qui nous intéresse ici ($k = n$).
\end{definitionbox}

\subsection{Preuve par récurrence « bus après bus »}

Commençons par observer les premiers cas concrets avant d'établir le passage général de $n$ à $n+1$.

\begin{proofbox}[Étape 0 – $n = 1$ (un seul bus)]
Tu \textbf{vois le premier bus}.  
Le \textbf{temps d’attente} est :
\[
X_1 \sim \text{Exp}(\lambda)
\]
Sa \textbf{densité} est :
\[
f_1(y) = \lambda e^{-\lambda y}
\]
\textbf{Remarque} : c’est \textbf{déjà} une \textbf{Gamma$(1, \lambda)$}.
\end{proofbox}

\begin{proofbox}[Étape 1 – $n = 2$ (deux bus)]
Tu \textbf{ne pars toujours pas}.  
Tu \textbf{attends le deuxième bus}.

Le \textbf{temps entre le 1er et le 2e} est :
\[
X_2 \sim \text{Exp}(\lambda), \quad \text{indépendant de } X_1
\]
Tu veux la \textbf{densité} de :
\[
Y_2 = X_1 + X_2
\]

\subsection*{On \textbf{convoit} les deux densités :}
\[
f_2(y) = \int_0^y f_{X_1}(x) f_{X_2}(y - x) \, dx
= \int_0^y \lambda e^{-\lambda x} \cdot \lambda e^{-\lambda (y - x)} \, dx
\]

\[
= \lambda^2 e^{-\lambda y} \int_0^y dx
= \lambda^2 y e^{-\lambda y}
\]
\textbf{C'est exactement} la densité d’une \textbf{Gamma$(2, \lambda)$}.
\end{proofbox}

\begin{proofbox}[Étape 2 – $n = 3$ (trois bus)]
Tu \textbf{attends encore}.  
Le \textbf{temps entre le 2e et le 3e} est :
\[
X_3 \sim \text{Exp}(\lambda), \quad \text{indépendant}
\]
Tu veux la \textbf{densité} de :
\[
Y_3 = Y_2 + X_3
\]

\subsection*{On \textbf{convoit} encore :}
\[
f_3(y) = \int_0^y f_{Y_2}(x) f_{X_3}(y - x) \, dx
= \int_0^y \lambda^2 x e^{-\lambda x} \cdot \lambda e^{-\lambda (y - x)} \, dx
\]

\[
= \lambda^3 e^{-\lambda y} \int_0^y x \, dx
= \lambda^3 e^{-\lambda y} \cdot \frac{y^2}{2}
= \frac{\lambda^3 y^2 e^{-\lambda y}}{2!}
\]
\textbf{C'est exactement} la densité d’une \textbf{Gamma$(3, \lambda)$}.
\end{proofbox}

\begin{proofbox}[Étape 3 – Tu vois la forme qui sort]
À \textbf{chaque fois} que tu \textbf{ajoutes un bus}, tu \textbf{convois} avec une \textbf{exponentielle}, et tu \textbf{fais apparaître} :
\begin{itemize}
  \item \textbf{$y^{n-1}$} au numérateur
  \item \textbf{$(n-1)!$} au dénominateur
  \item \textbf{$\lambda^n e^{-\lambda y}$} devant
\end{itemize}
\end{proofbox}

\begin{proofbox}[Étape 4 – Généralisation $n \to n+1$]
\subsection*{Supposons que :}
\[
f_n(y) = \frac{\lambda^n y^{n-1} e^{-\lambda y}}{(n-1)!}
\]
\subsection*{Alors :}
\[
f_{n+1}(y) = \int_0^y f_n(x) f_{X_{n+1}}(y - x) \, dx
= \int_0^y \frac{\lambda^n x^{n-1} e^{-\lambda x}}{(n-1)!} \cdot \lambda e^{-\lambda (y - x)} \, dx
\]

\[
= \frac{\lambda^{n+1} e^{-\lambda y}}{(n-1)!} \int_0^y x^{n-1} \, dx
= \frac{\lambda^{n+1} e^{-\lambda y}}{(n-1)!} \cdot \frac{y^n}{n}
= \frac{\lambda^{n+1} y^n e^{-\lambda y}}{n!}
\]
\textbf{C'est exactement} la densité d’une \textbf{Gamma$(n+1, \lambda)$}.

\smallskip
\emph{Conséquence} : par récurrence, $X_1+\dots+X_n \sim \text{Gamma}(n, \lambda)$ pour tout $n\geq 1$.
\end{proofbox}

\subsection{Règles mnémotechniques}

\begin{remarquebox}[Formules « directes » pour $n$ entier]
Si $Y \sim \text{Gamma}(n, \lambda)$ avec $n$ entier :
\begin{itemize}
  \item Espérance : $\mathbb{E}[Y] = \dfrac{n}{\lambda}$ \quad (somme de $n$ moyennes $1/\lambda$)
  \item Variance : $\text{Var}(Y) = \dfrac{n}{\lambda^2}$ \quad (somme de $n$ variances $1/\lambda^2$)
\end{itemize}
\underline{Rappel visuel} : « $n$ bus = $n$ fois l'espérance d'un seul trajet ».
\end{remarquebox}

\subsection{Lien avec le processus de Poisson (facultatif mais éclairant)}

\begin{intuitionbox}[Du bus au comptage]
Les instants d'arrivée des bus forment un \emph{processus de Poisson} d'intensité $\lambda$.  
La variable $Y$ ci-dessus n'est autre que la date du $n$-ième événement ;  
sa densité Gamma traduit le fait qu'"attendre $n$ événements" prend \emph{en moyenne} $n/\lambda$ unités de temps, avec une dispersion croissante avec $n$.
\end{intuitionbox}
\input{sections/section5a.tex}

\newpage

\section{Transformations de Variable}

Un problème fondamental en théorie des probabilités est de déterminer la distribution d'une variable aléatoire transformée \( Y = g(X) \), lorsque \( X \) est une variable continue de densité connue \( f_X(x) \). Cette section présente les méthodes pour obtenir la densité de \( Y \), selon que la fonction \( g \) est monotone ou non.

\subsection{Cas où \( g \) est strictement monotone}

Lorsque \( g \) est strictement monotone (croissante ou décroissante) et dérivable sur le support de \( X \), on peut appliquer une formule directe basée sur la réciproque de \( g \).

\begin{theorembox}[Changement de Variable - Cas Monotone]
Soit \( X \) une variable aléatoire continue de densité \( f_X(x) \), supportée sur un intervalle \( I \). Soit \( g : I \to J \) une fonction strictement monotone et dérivable, avec \( g'(x) \ne 0 \) sur \( I \). Alors \( Y = g(X) \) est une variable aléatoire continue de densité :
$$ f_Y(y) = f_X(g^{-1}(y)) \cdot \left| \frac{d}{dy} g^{-1}(y) \right| $$
pour \( y \in J \), et \( f_Y(y) = 0 \) ailleurs.
\end{theorembox}

\begin{proofbox}[Preuve du Théorème]
On distingue deux cas selon la monotonie de \( g \).

\textbf{Cas 1 : \( g \) strictement croissante.}
$$ F_Y(y) = P(Y \le y) = P(g(X) \le y) = P(X \le g^{-1}(y)) = F_X(g^{-1}(y)) $$
En dérivant par rapport à \( y \) :
$$ f_Y(y) = \frac{d}{dy} F_X(g^{-1}(y)) = f_X(g^{-1}(y)) \cdot \frac{d}{dy} g^{-1}(y) $$
Comme \( \frac{d}{dy} g^{-1}(y) > 0 \), on a :
$$ f_Y(y) = f_X(g^{-1}(y)) \cdot \left| \frac{d}{dy} g^{-1}(y) \right| $$

\textbf{Cas 2 : \( g \) strictement décroissante.}
$$ F_Y(y) = P(Y \le y) = P(g(X) \le y) = P(X \ge g^{-1}(y)) = 1 - F_X(g^{-1}(y)) $$
Dérivons :
$$ f_Y(y) = -f_X(g^{-1}(y)) \cdot \frac{d}{dy} g^{-1}(y) $$
Puisque \( \frac{d}{dy} g^{-1}(y) < 0 \), la valeur absolue donne :
$$ f_Y(y) = f_X(g^{-1}(y)) \cdot \left| \frac{d}{dy} g^{-1}(y) \right| $$
\end{proofbox}

\begin{examplebox}[Changement d'unité : distance entre véhicules]
Soit \( X \sim \text{Exp}(0.001) \), exprimée en mètres. On pose \( Y = X / 1000 \), en kilomètres. Alors :
- \( g(x) = x/1000 \), strictement croissante.
- \( g^{-1}(y) = 1000y \)
- \( \left| \frac{d}{dy} g^{-1}(y) \right| = 1000 \)

Ainsi :
\[
f_Y(y) = f_X(1000y) \cdot 1000 = (0.001 e^{-0.001 \cdot 1000y}) \cdot 1000 = e^{-y}, \quad y \geq 0
\]
Donc \( Y \sim \text{Exp}(1) \).
\end{examplebox}

\subsection{Cas où \( g \) n'est pas monotone}

Lorsque \( g \) n’est pas monotone, l’inverse global \( g^{-1} \) n’existe pas. On doit alors partitionner le support de \( X \) en sous-intervalles sur lesquels \( g \) est monotone, puis sommer les contributions de chaque branche.

\begin{theorembox}[Changement de Variable - Cas Général]
Soit \( X \) de densité \( f_X(x) \), et \( Y = g(X) \). Supposons que \( g \) soit différentiable et que l'on puisse diviser le support de \( X \) en \( n \) intervalles disjoints \( I_1, \dots, I_n \), sur chacun desquels \( g \) est strictement monotone. Soit \( x_k = h_k(y) \) la réciproque de \( g \) sur \( I_k \). Alors :
$$ f_Y(y) = \sum_{k=1}^n f_X(h_k(y)) \cdot \left| \frac{d}{dy} h_k(y) \right| $$
pour tout \( y \) tel que \( g(x) = y \) admette des solutions dans ces intervalles.
\end{theorembox}

\begin{examplebox}[Transformation non monotone : $ Y = \sqrt{|X|} $]

\textbf{Énoncé.} Soit $ X \sim \mathcal{N}(0, 1) $, de densité :
\[
f_X(x) = \frac{1}{\sqrt{2\pi}} e^{-\frac{x^2}{2}}, \quad x \in \mathbb{R}.
\]
On considère $ Y = \sqrt{|X|} $. Déterminons la densité de $ Y $.

\textbf{Analyse de la transformation.}

\textit{Support de $ Y $ :} Puisque $ |X| \geq 0 $, on a $ Y = \sqrt{|X|} \geq 0 $. Le support de $ Y $ est donc $ [0, \infty) $.

\textit{Non-monotonie :} La fonction $ g(x) = \sqrt{|x|} $ n'est pas monotone sur $ \mathbb{R} $. Pour une valeur donnée $ y > 0 $, l'équation $ y = \sqrt{|x|} $ admet \textbf{deux solutions} :
\begin{itemize}
  \item $ x_1 = y^2 $ (quand $ X > 0 $),
  \item $ x_2 = -y^2 $ (quand $ X < 0 $).
\end{itemize}
Cette transformation est dite \textbf{deux-pour-un} : deux valeurs distinctes de $ X $ produisent la même valeur de $ Y $.

\textbf{Application de la formule générale.}

On partitionne le support de $ X $ en deux intervalles sur lesquels $ g $ est monotone :
\begin{itemize}
  \item $ A_1 = (0, \infty) $ : sur cet intervalle, $ g(x) = \sqrt{x} $ est strictement croissante,
  \item $ A_2 = (-\infty, 0) $ : sur cet intervalle, $ g(x) = \sqrt{-x} $ est strictement décroissante.
\end{itemize}

Pour $ y > 0 $, on applique la formule :
\[
f_Y(y) = \sum_{k=1}^{2} f_X(x_k) \left| \frac{dx_k}{dy} \right|,
\]
où $ x_1 = y^2 $ et $ x_2 = -y^2 $ sont les inverses sur chaque branche.

\textit{Calcul des dérivées :}
\begin{itemize}
  \item $ \frac{dx_1}{dy} = \frac{d}{dy}(y^2) = 2y $, donc $ \left| \frac{dx_1}{dy} \right| = 2y $ (car $ y > 0 $),
  \item $ \frac{dx_2}{dy} = \frac{d}{dy}(-y^2) = -2y $, donc $ \left| \frac{dx_2}{dy} \right| = 2y $.
\end{itemize}

\textit{Évaluation de $ f_X $ :}
\begin{itemize}
  \item $ f_X(y^2) = \frac{1}{\sqrt{2\pi}} e^{-\frac{(y^2)^2}{2}} = \frac{1}{\sqrt{2\pi}} e^{-\frac{y^4}{2}} $,
  \item $ f_X(-y^2) = \frac{1}{\sqrt{2\pi}} e^{-\frac{(-y^2)^2}{2}} = \frac{1}{\sqrt{2\pi}} e^{-\frac{y^4}{2}} $.
\end{itemize}
Les deux termes sont identiques par symétrie de la loi normale.

\textit{Densité de $ Y $ :}
\[
f_Y(y) = f_X(y^2) \cdot 2y + f_X(-y^2) \cdot 2y = 2 \cdot \frac{1}{\sqrt{2\pi}} e^{-\frac{y^4}{2}} \cdot 2y = \frac{4y}{\sqrt{2\pi}} e^{-\frac{y^4}{2}}, \quad y > 0.
\]

\textbf{Justification de l'addition des contributions.}

Considérons un petit intervalle $ [y, y + dy] $. La probabilité que $ Y $ tombe dans cet intervalle provient de deux événements disjoints :
\begin{itemize}
  \item $ X $ proche de $ y^2 $ (branche positive),
  \item $ X $ proche de $ -y^2 $ (branche négative).
\end{itemize}
On a donc :
\[
P(Y \in [y, y + dy]) = P(X \in [y^2, (y + dy)^2]) + P(X \in [-(y + dy)^2, -y^2]).
\]
En utilisant les densités :
\[
f_Y(y) \, dy \approx f_X(y^2) \, dx_1 + f_X(-y^2) \, dx_2,
\]
où $ dx_1 $ et $ dx_2 $ sont les longueurs des intervalles correspondants dans l'espace de $ X $. Cela conduit à sommer les contributions.

\textbf{Conclusion.}

La densité de $ Y = \sqrt{|X|} $ est :
\[
f_Y(y) = \frac{4y}{\sqrt{2\pi}} e^{-\frac{y^4}{2}}, \quad y > 0.
\]
Elle résulte de la somme de deux contributions égales, reflétant la symétrie de $ X $ autour de zéro.

\end{examplebox}
\input{sections/section6a.tex}
\newpage
\section{La Fonction Delta de Dirac}

\subsection{Introduction et Définition}

Après avoir étudié des distributions continues classiques comme la loi normale et log-normale, nous abordons un objet mathématique singulier : la \textit{fonction delta de Dirac}, notée $\delta(x)$. Bien qu'elle ne soit pas une fonction au sens classique, elle est fondamentale en probabilités.

\begin{definitionbox}[Fonction Delta de Dirac]
La \textbf{fonction delta de Dirac} $\delta(x)$ est une \textit{distribution} (ou fonction généralisée) caractérisée par deux propriétés :
\begin{enumerate}
    \item $\delta(x) = 0$ pour tout $x \neq 0$,
    \item $\int_{-\infty}^{+\infty} \delta(x) \, dx = 1$.
\end{enumerate}

Intuitivement, $\delta(x)$ représente une « masse ponctuelle » d'intensité infinie localisée en $x = 0$, mais d'aire totale égale à 1.
\end{definitionbox}

\begin{remarquebox}[Paradoxe Apparent]
Ces deux propriétés semblent contradictoires : comment une fonction nulle partout sauf en un point peut-elle avoir une intégrale égale à 1 ?

La résolution du paradoxe vient du fait que $\delta(x)$ n'est \textbf{pas une fonction ordinaire}, mais une \textit{distribution} au sens de Laurent Schwartz. Elle ne prend pas de « valeur » en un point, mais agit sur d'autres fonctions via l'intégration. On dit que $\delta$ est une \textbf{mesure de Dirac} concentrée en 0.
\end{remarquebox}

\subsection{Intuition : De la Masse Diffuse à la Masse Ponctuelle}

La delta de Dirac peut être comprise comme la limite d'une suite de fonctions de plus en plus « piquées ».

\begin{intuitionbox}[Approximation par des Fonctions Régulières]
\textbf{1. Idée de Base} :

Imaginons une distribution de masse sur la droite réelle. Au lieu d'être étalée, nous voulons concentrer toute la masse en un seul point (l'origine), tout en préservant la masse totale.

\textbf{2. Construction par Limite (Gaussiennes)} :

Considérons la famille de gaussiennes centrées en 0 :
$$ \delta_\epsilon(x) = \frac{1}{\epsilon\sqrt{2\pi}} \exp\left(-\frac{x^2}{2\epsilon^2}\right) $$

\begin{itemize}
    \item Pour tout $\epsilon > 0$, $\delta_\epsilon(x)$ est une densité de probabilité normale $\mathcal{N}(0, \epsilon^2)$.
    \item $\int_{-\infty}^{+\infty} \delta_\epsilon(x) \, dx = 1$ (aire sous la courbe = 1).
    \item Quand $\epsilon \to 0$ :
    \begin{itemize}
        \item La hauteur du pic augmente : $\delta_\epsilon(0) = \frac{1}{\epsilon\sqrt{2\pi}} \to +\infty$.
        \item La largeur diminue : la variance $\epsilon^2 \to 0$.
        \item Pour tout $x \neq 0$, $\delta_\epsilon(x) \to 0$ (la queue s'aplatit).
        \item L'aire totale reste 1.
    \end{itemize}
\end{itemize}

On écrit symboliquement :
$$ \delta(x) = \lim_{\epsilon \to 0} \delta_\epsilon(x) $$

\textbf{3. Autres Approximations Possibles} :

On peut obtenir $\delta(x)$ comme limite de nombreuses familles de fonctions :
\begin{itemize}
    \item \textbf{Rectangles} : $\delta_\epsilon(x) = \begin{cases} \frac{1}{2\epsilon} & \text{si } |x| < \epsilon \\ 0 & \text{sinon} \end{cases}$
    \item \textbf{Lorentziennes} : $\delta_\epsilon(x) = \frac{1}{\pi} \cdot \frac{\epsilon}{\epsilon^2 + x^2}$
    \item \textbf{Sinus cardinal} : $\delta_\epsilon(x) = \frac{1}{\pi x} \sin\left(\frac{x}{\epsilon}\right)$ (avec régularisation)
\end{itemize}

Toutes ces fonctions « tendent » vers $\delta(x)$ au sens des distributions.

\tcblower
\centering
\begin{tikzpicture}
    \begin{axis}[
        title={Approximation de $\delta(x)$ par des Gaussiennes},
        xlabel={$x$},
        ylabel={$\delta_\epsilon(x)$},
        axis lines=middle,
        no markers,
        samples=200,
        domain=-3:3,
        ymin=0, ymax=5,
        height=8cm,
        width=\linewidth-1cm,
        tick label style={font=\tiny},
        legend style={at={(0.5,-0.15)}, anchor=north, font=\small},
        legend columns=3
    ]
    % Gaussienne epsilon = 1
    \addplot [blue, thick] {1/(1*sqrt(2*pi))*exp(-(x^2)/(2*1^2))};
    \addlegendentry{$\epsilon=1.0$};
    % Gaussienne epsilon = 0.5
    \addplot [red, thick] {1/(0.5*sqrt(2*pi))*exp(-(x^2)/(2*0.5^2))};
    \addlegendentry{$\epsilon=0.5$};
    % Gaussienne epsilon = 0.2
    \addplot [green!70!black, very thick] {1/(0.2*sqrt(2*pi))*exp(-(x^2)/(2*0.2^2))};
    \addlegendentry{$\epsilon=0.2$ (plus piquée)};
    \end{axis}
\end{tikzpicture}
\par\small\textit{Quand $\epsilon \to 0$, la hauteur $\to \infty$, la largeur $\to 0$, mais l'aire reste $= 1$.}
\end{intuitionbox}

\subsection{Propriété Fondamentale : La Propriété de Filtrage}

La définition vraiment opérationnelle de $\delta(x)$ repose sur son action sur les fonctions continues.

\begin{theorembox}[Propriété de Filtrage (Sifting Property)]
Pour toute fonction $f$ continue en $x = 0$ :
$$ \int_{-\infty}^{+\infty} f(x) \, \delta(x) \, dx = f(0) $$

Plus généralement, $\delta$ décalé en $a$ filtre la valeur en $a$ :
$$ \int_{-\infty}^{+\infty} f(x) \, \delta(x - a) \, dx = f(a) $$
\end{theorembox}

\begin{proofbox}[Justification Intuitive]
Utilisons l'approximation gaussienne $\delta_\epsilon(x) = \frac{1}{\epsilon\sqrt{2\pi}} e^{-x^2/(2\epsilon^2)}$.

On a :
$$ \int_{-\infty}^{+\infty} f(x) \, \delta_\epsilon(x) \, dx = \mathbb{E}[f(X)] \quad \text{où } X \sim \mathcal{N}(0, \epsilon^2) $$

Quand $\epsilon \to 0$, la loi de $X$ se concentre entièrement en $x = 0$. Donc :
$$ \lim_{\epsilon \to 0} \mathbb{E}[f(X)] = f(0) \quad \text{(par continuité de } f \text{)} $$

Pour $\delta(x - a)$, le raisonnement est identique en centrant la gaussienne en $a$.
\end{proofbox}

\begin{remarquebox}[Interprétation Probabiliste]
En théorie des probabilités, $\delta(x - a)$ est la « densité » (au sens des distributions) d'une variable aléatoire \textbf{dégénérée} qui prend la valeur $a$ avec probabilité 1.

Si $X = a$ presque sûrement, alors pour toute fonction mesurable $f$ :
$$ \mathbb{E}[f(X)] = f(a) = \int_{-\infty}^{+\infty} f(x) \, \delta(x - a) \, dx $$

Ainsi, $\delta(x - a)$ généralise la notion de densité de probabilité aux masses ponctuelles.
\end{remarquebox}

\begin{examplebox}[Loi de Durée de Vie d’un Dispositif — Mélange Discret/Continu]
\textbf{Contexte} : Une entreprise produit des dispositifs dont la durée de vie $X$ (en années) n’est pas toujours continue. En effet, environ 2\% des dispositifs sont défectueux dès le départ : leur durée de vie est donc exactement $0$ an. Pour les 98\% restants, la durée de vie suit une loi exponentielle de paramètre $\lambda = 2$ an$^{-1}$.

\textbf{Objectif} : Déterminer la \textbf{densité généralisée} de $X$, qui mélange une masse de Dirac en 0 et une composante continue sur $\mathbb{R}_+^*$.

\textbf{Modélisation} :
\begin{itemize}
    \item $\mathbb{P}(X = 0) = 0.02$ : masse ponctuelle en 0.
    \item Pour $x > 0$, $X \mid X > 0 \sim \text{Exp}(2)$, donc :
    \[
    f_{X \mid X > 0}(x) = 2 \mathrm{e}^{-2x}, \quad x > 0.
    \]
\end{itemize}

\textbf{Densité généralisée de $X$} :
On écrit la densité $f_X(x)$ comme un \textbf{mélange} d’une masse de Dirac et d’une densité continue :
\[
f_X(x) = 0.02 \cdot \delta(x) + 0.98 \cdot 2 \mathrm{e}^{-2x} \cdot \mathbb{I}_{x > 0}.
\]

\textbf{Rôle de l’indicatrice $\mathbb{I}_{x > 0}$} :  
La fonction indicatrice
\[
\mathbb{I}_{x > 0} = 
\begin{cases}
1 & \text{si } x > 0,\\[2mm]
0 & \text{sinon}
\end{cases}
\]
force la partie exponentielle à être \textbf{nulle sur $\mathbb{R}_-$}.  
Sans elle, $2\mathrm{e}^{-2x}$ serait positive pour \textbf{tous} les réels, y compris les $x<0$, ce qui \textbf{n’a aucun sens} pour une durée de vie.  
Avec $\mathbb{I}_{x > 0}$ on a bien
\[
f_X(x)=0 \quad \text{pour tout } x<0,
\]
ce qui garantit que la durée de vie ne peut pas être négative.

\textbf{Vérification} :
\begin{itemize}
    \item Masse totale :
    \[
    \int_{-\infty}^{+\infty} f_X(x)\,\mathrm{d}x 
    = 0.02 \underbrace{\int_{-\infty}^{+\infty}\delta(x)\,\mathrm{d}x}_{=1} 
    + 0.98 \underbrace{\int_{0}^{+\infty} 2\mathrm{e}^{-2x}\,\mathrm{d}x}_{=1} 
    = 0.02 + 0.98 = 1.
    \]
    \item Espérance (via la propriété de filtrage) :
    \[
    \mathbb{E}[X] 
    = \int_{-\infty}^{+\infty} x f_X(x)\,\mathrm{d}x 
    = 0.02 \cdot 0 + 0.98 \int_{0}^{+\infty} x \cdot 2\mathrm{e}^{-2x}\,\mathrm{d}x 
    = 0.98 \cdot \tfrac{1}{2} = 0.49 \text{ an}.
    \]
\end{itemize}

\textbf{Fonction de répartition $F_X(x)=\mathbb{P}(X\le x)$} :

Pour $x<0$ : aucune masse négative $\Rightarrow F_X(x)=0$.

Pour $x\ge 0$ :
\begin{align*}
F_X(x)
&= \mathbb{P}(X=0) + \mathbb{P}(0<X\le x) \\[2mm]
&= 0.02 + 0.98\int_{0}^{x} 2\mathrm{e}^{-2t}\,\mathrm{d}t \\[2mm]
&= 0.02 + 0.98\Bigl(1-\mathrm{e}^{-2x}\Bigr) \\[2mm]
&= 1 - 0.98\,\mathrm{e}^{-2x}.
\end{align*}

Ainsi
\[
F_X(x)=
\begin{cases}
0 & \text{si } x<0,\\[2mm]
1-0.98\,\mathrm{e}^{-2x} & \text{si } x\ge 0.
\end{cases}
\]

\textbf{Vérification} :  
$\displaystyle\lim_{x\to\infty}F_X(x)=1$ et $F_X(0)=0.02$, qui est bien la masse ponctuelle en $0$.

\vspace{5mm}


\textbf{Interprétation} : La fonction delta de Dirac permet de \textbf{modéliser explicitement} la masse de probabilité ponctuelle en $x=0$, tandis que l’indicatrice $\mathbb{I}_{x>0}$ assure que la composante continue reste confinée aux durées de vie strictement positives.

\end{examplebox}
\newpage
\section{La Loi Normale (ou Gaussienne)}

\subsection{Introduction et Fonction de Densité (PDF)}

Après les lois discrètes et les lois continues de base (Uniforme, Exponentielle), nous abordons la distribution la plus célèbre et la plus utilisée en probabilités et statistiques.

\begin{definitionbox}[Loi Normale]
Une variable aléatoire continue $X$ suit une \textbf{loi normale} (ou loi de Gauss) de paramètres $\mu$ (l'espérance) et $\sigma^2$ (la variance), notée $X \sim \mathcal{N}(\mu, \sigma^2)$, si sa fonction de densité de probabilité (PDF) est donnée par :
$$ f(x; \mu, \sigma) = \frac{1}{\sigma \sqrt{2\pi}} e^{ -\frac{1}{2} \left( \frac{x-\mu}{\sigma} \right)^2 } $$
pour tout $x \in (-\infty, \infty)$, où $\sigma > 0$.
\end{definitionbox}

Cette formule, bien qu'imposante, décrit une forme très familière : la courbe en cloche.

\begin{intuitionbox}[La Courbe en Cloche]
La loi normale est sans doute la distribution la plus importante en probabilités et statistiques. Pourquoi ? Parce qu'elle modélise remarquablement bien de nombreux phénomènes naturels et processus aléatoires où les valeurs tendent à se regrouper autour d'une moyenne, avec des écarts symétriques devenant de plus en plus rares à mesure qu'on s'éloigne de cette moyenne. Pensez à la taille des individus dans une population, aux erreurs de mesure répétées, ou même aux notes d'un grand groupe d'étudiants à un examen bien conçu. 

Sa densité a une forme caractéristique de \textbf{cloche symétrique} :
\begin{itemize}
    \item \textbf{Le Centre ($\mu$)} : Le paramètre $\mu$ représente l'\textbf{espérance} (la moyenne) de la distribution. C'est le centre de symétrie de la courbe, là où la cloche atteint son \textbf{sommet}. C'est la valeur la plus probable (le mode) et aussi la valeur qui coupe la distribution en deux moitiés égales (la médiane). Changer $\mu$ \textit{translate} la cloche horizontalement sans changer sa forme.
    \item \textbf{La Dispersion ($\sigma$)} : Le paramètre $\sigma$ est l'\textbf{écart-type} ($\sigma^2$ est la variance). Il mesure la \textbf{dispersion} des valeurs autour de la moyenne $\mu$. Géométriquement, $\sigma$ contrôle la \textbf{largeur} de la cloche.
        \begin{itemize}
            \item Un \textit{petit} $\sigma$ signifie que les données sont très concentrées autour de la moyenne, donnant une cloche \textbf{étroite et pointue}.
            \item Un \textit{grand} $\sigma$ signifie que les données sont plus étalées, donnant une cloche \textbf{large et aplatie}.
        \end{itemize}
    Les points d'inflexion de la courbe (là où la courbure change de sens) se situent exactement à $\mu \pm \sigma$.
\end{itemize}

\tcblower
\centering
\begin{tikzpicture}
    \begin{axis}[
        title={La Courbe en Cloche (PDF de la Loi Normale)},
        xlabel={$x$},
        ylabel={$f(x)$},
        axis lines=middle,
        no markers,
        samples=100,
        domain=-4:4,
        height=8cm,
        width=\linewidth-1cm,
        tick label style={font=\tiny},
        legend style={at={(0.5,-0.15)}, anchor=north, font=\small},
        legend columns=2
    ]
    % N(0, 1)
    \addplot [blue, ultra thick] {1/(sqrt(2*pi))*exp(-x^2/2)};
    \addlegendentry{$\mu=0, \sigma=1$};
    % N(0, 0.25) => sigma=0.5
    \addplot [red, ultra thick] {1/(0.5*sqrt(2*pi))*exp(-x^2/(2*0.5^2))};
    \addlegendentry{$\mu=0, \sigma=0.5$ (étroite)};
    % N(1, 2.25) => sigma=1.5
    \addplot [green!70!black, ultra thick] {1/(1.5*sqrt(2*pi))*exp(-(x-1)^2/(2*1.5^2))};
    \addlegendentry{$\mu=1, \sigma=1.5$ (large, décalée)};

    \draw [dashed] (axis cs:0,0) -- (axis cs:0, {1/(sqrt(2*pi))}) node[above, font=\tiny] {pic à $\mu=0$}; % Ligne pour mu=0
    \draw [dashed] (axis cs:1,0) -- (axis cs:1, {1/(1.5*sqrt(2*pi))}) node[above right, font=\tiny] {pic à $\mu=1$}; % Ligne pour mu=1
    \end{axis}
\end{tikzpicture}
\par\small\textit{Influence de $\mu$ (position) et $\sigma$ (largeur) sur la forme de la cloche.}
\end{intuitionbox}

Mais d'où vient cette formule spécifique ? Il existe une dérivation fascinante à partir d'hypothèses fondamentales sur les erreurs aléatoires (argument d'Herschel-Maxwell).

\begin{proofbox}[Dérivation de la Densité Normale à partir des Principes Fondamentaux]

\textbf{Contexte Visuel :} Imaginons un nuage de points dispersés autour d'une cible à l'origine $(0,0)$, comme des impacts de fléchettes. Le graphique ci-dessous illustre cette dispersion. On s'intéresse à la probabilité de tomber dans une petite zone, comme $dA$, autour d'un point $(x, y)$.

\begin{center}
\begin{tikzpicture}
\begin{axis}[
    axis lines=middle, % Axes qui passent par l'origine (0,0)
    xlabel=$x$,       % Étiquette axe X
    ylabel=$y$,       % Étiquette axe Y
    axis line style={magenta}, % Couleur des axes
    xlabel style={anchor=west, magenta}, % Style de l'étiquette X
    ylabel style={anchor=south, magenta}, % Style de l'étiquette Y
    xmin=-3.5, xmax=3.5, % Limites du graphique
    ymin=-3.5, ymax=3.5,
    tick label style={font=\tiny} % Police plus petite pour les graduations
]

% Ajout des points du nuage. Ce sont des coordonnées approximatives.
\addplot [only marks, mark=*, cyan, mark size=1.5pt]
coordinates {
    (0.2, 0.1) (-0.5, 0.2) (0.1, -0.3) (0.5, -0.8) (-0.3, -1.2) (0,0.1)
    (1.5, 1.5) (0.8, 0.8) (2.0, -0.5) (2.8, -1.4) (2.5, -0.2)
    (-1.8, -1.3) (-2.5, 0.5) (-1.5, 0.3) (-2.2, -0.8) (-0.8, 0.5)
    (0.5, 1.2) (0.7, 2.8) (0.2, 3.2) (-0.5, 1.5) (-1.0, -2.0)
    (1.8, -1.0) (1.0, -1.5) (0.3, -2.5) (-1.8, -2.8) (1.2, 0.5)
    (-1.2, -1.5) (-0.8, 1.1)
};

% --- Annotations ---

% Points et boîte pour 'dA'
\addplot [only marks, mark=*, red, mark size=1.5pt] coordinates {(1.3, 2.0)};
\draw [red, thick] (axis cs:1.1, 1.8) rectangle (axis cs:1.5, 2.2);
\node [red, above] at (axis cs:1.3, 2.2) {$dA$};

% Points et boîte pour 'dB'
\addplot [only marks, mark=*, cyan, mark size=1.5pt] 
coordinates {(-1.2, 2.0) (-1.3, 2.3) (-1.0, 2.2) (-1.1, 1.9)};
\draw [blue, thick] (axis cs:-1.8, 1.2) rectangle (axis cs:-0.8, 2.8);
\node [blue, above] at (axis cs:-1.3, 2.8) {$dB$};

\end{axis}
\end{tikzpicture}
\end{center}

\textbf{Objectif :} Expliquer comment arriver à la formule mathématique de la courbe en cloche (densité de probabilité normale) en partant de principes fondamentaux sur les erreurs aléatoires.

\textbf{1. Le Point de Départ : Densité et Aire $dA$}
Dans une distribution continue, la probabilité de tomber \textit{exactement} sur un point $(x, y)$ est nulle. On ne peut donc pas parler de "probabilité d'un point". On parle de la probabilité de tomber \textit{dans une petite zone}, comme un rectangle $dA = dx \cdot dy$ autour du point $(x, y)$.
Cette probabilité, notée $P(\text{dans } dA)$, est \textit{proportionnelle} à l'aire de la zone $dA$. La \textit{constante de proportionnalité} est la \textbf{fonction de densité de probabilité} $p(x, y)$ évaluée en ce point. En d'autres termes, la densité $p(x, y)$ \textit{représente} localement la concentration de probabilité. Ainsi, la probabilité de tomber dans la zone $dA$ est approximativement :
$$ P(\text{dans } dA) \approx p(x, y) \cdot dA $$

\textbf{2. Les Hypothèses Fondamentales}
On pose deux hypothèses sur la nature de ces erreurs (représentées par la densité $p(x, y)$) :
\begin{enumerate}
    \item \textbf{Indépendance des axes :} L'erreur horizontale ($x$) est indépendante de l'erreur verticale ($y$). Cela implique que la densité jointe $p(x, y)$ peut s'écrire comme le produit de la densité marginale sur $x$, notée $f(x)$, et de la densité marginale sur $y$, notée $f(y)$. Donc, $p(x, y) = f(x) \cdot f(y)$.
    \item \textbf{Symétrie de rotation (Isotropie) :} La densité ne dépend que de la distance $r = \sqrt{x^2 + y^2}$ au centre, pas de l'angle. Il existe donc une fonction $\phi(r)$ telle que la densité en $(x,y)$ est $p(x, y) = \phi(\sqrt{x^2 + y^2})$.
\end{enumerate}

\textbf{3. L'Équation Fonctionnelle}
En égalant les deux expressions pour la même densité $p(x, y)$ (à une constante près), on obtient :
$$ f(x) \cdot f(y) = \phi(\sqrt{x^2 + y^2}) $$
Pour $y=0$, on a $f(x) \cdot f(0) = \phi(x)$. Posons $f(0) = \lambda$. Alors $\phi(x) = \lambda f(x)$.
L'équation devient :
$$ f(x) \cdot f(y) = \lambda f(\sqrt{x^2 + y^2}) $$

\textbf{4. Résolution de l'Équation Fonctionnelle}
Posons $g(x) = f(x)/\lambda$, avec $g(0)=1$. L'équation se simplifie en :
$$ g(x) g(y) = g(\sqrt{x^2 + y^2}) $$
Posons $g(x) = h(x^2)$. L'équation devient $h(x^2)h(y^2) = h(x^2+y^2)$. Avec $a=x^2$ et $b=y^2$, on a :
$$ h(a) h(b) = h(a+b) $$
La solution continue de cette équation de Cauchy est $h(a) = e^{Aa}$ pour une constante $A$.
Retour aux fonctions : $g(x) = h(x^2) = e^{Ax^2}$. $f(x) = \lambda g(x) = \lambda e^{Ax^2}$.
Comme la densité doit diminuer loin du centre, $A$ doit être négative. Posons $A = -k$ avec $k>0$.
$$ f(x) = \lambda e^{-k x^2} $$

\textbf{5. Normalisation et Identification des Paramètres}
\begin{enumerate}
    \item \textbf{Condition $\int f(x) dx = 1$} : L'intégrale Gaussienne $\int_{-\infty}^{\infty} e^{-k x^2} \, \mathrm{d}x = \sqrt{\frac{\pi}{k}}$.
    Donc, $\int_{-\infty}^{\infty} f(x) dx = \lambda \sqrt{\frac{\pi}{k}} = 1 \implies \lambda = \sqrt{\frac{k}{\pi}}$.
    \item \textbf{Lien avec la Variance ($\sigma^2$)} : Pour une distribution centrée, $\sigma^2 = E[X^2] = \int x^2 f(x) dx$.
    $$ \sigma^2 = \int_{-\infty}^{\infty} x^2 \left( \sqrt{\frac{k}{\pi}} e^{-k x^2} \right) \, \mathrm{d}x = \sqrt{\frac{k}{\pi}} \left( \frac{1}{2k} \sqrt{\frac{\pi}{k}} \right) = \frac{1}{2k} $$
    Donc, $k = \frac{1}{2\sigma^2}$.
    \item \textbf{Substitution Finale :} Remplaçons $k$ dans $\lambda$ et $f(x)$.
    $$ \lambda = \sqrt{\frac{1/(2\sigma^2)}{\pi}} = \frac{1}{\sigma\sqrt{2\pi}} $$
    $$ f(x) = \frac{1}{\sigma\sqrt{2\pi}} e^{-\frac{1}{2\sigma^2} x^2} = \frac{1}{\sigma\sqrt{2\pi}} e^{ -\frac{x^2}{2\sigma^2} } $$
    \item \textbf{Généralisation (Moyenne $\mu$)} : Pour centrer la distribution sur $\mu$, on remplace $x$ par $(x-\mu)$ dans l'exposant :
    $$ f(x; \mu, \sigma) = \frac{1}{\sigma \sqrt{2\pi}} e^{ -\frac{(x-\mu)^2}{2\sigma^2} } $$
\end{enumerate}
C'est la fonction de densité de la loi normale $\mathcal{N}(\mu, \sigma^2)$.
\end{proofbox}

\subsection{La Loi Normale Centrée Réduite $\mathcal{N}(0, 1)$}

Avant d'explorer les propriétés de la loi normale générale, concentrons-nous sur son cas le plus simple et le plus fondamental.

\begin{definitionbox}[Loi Normale Standard (ou Centrée Réduite)]
Un cas particulier extraordinairement utile est la loi normale avec une moyenne $\mu=0$ et une variance $\sigma^2=1$ (donc $\sigma=1$). On l'appelle la \textbf{loi normale standard} ou \textbf{centrée réduite}, et on la note souvent $Z$. Sa PDF est traditionnellement notée $\phi(z)$ :
$$ \phi(z) = \frac{1}{\sqrt{2\pi}} e^{-z^2/2} $$
Sa fonction de répartition (CDF), qui donne $P(Z \le z)$, est notée $\Phi(z)$ :
$$ \Phi(z) = P(Z \le z) = \int_{-\infty}^z \frac{1}{\sqrt{2\pi}} e^{-t^2/2} \, \mathrm{d}t $$
\end{definitionbox}

Pourquoi cette version standard est-elle si importante ? Elle sert de référence universelle.

\begin{intuitionbox}[La Référence Universelle et le Changement d'Unités]
Pourquoi cette loi $\mathcal{N}(0, 1)$ est-elle si centrale ? Imaginez que vous ayez des mesures en degrés Celsius ($\mathcal{N}(\mu_C, \sigma_C^2)$) et d'autres en degrés Fahrenheit ($\mathcal{N}(\mu_F, \sigma_F^2)$). Comment les comparer ? La loi normale standard fournit un \textbf{système d'unités universel}.

Toute variable normale $X \sim \mathcal{N}(\mu, \sigma^2)$ peut être transformée ("standardisée") en une variable $Z \sim \mathcal{N}(0, 1)$ par un simple changement d'échelle et de position : $Z = (X-\mu)/\sigma$. 

Cela signifie qu'au lieu de devoir calculer des aires (probabilités) pour une infinité de courbes en cloche différentes (une pour chaque paire $\mu, \sigma$), on peut tout ramener à \textbf{une seule courbe de référence}, $\mathcal{N}(0, 1)$. Les aires sous cette courbe standard ($\Phi(z)$) ont été calculées une fois pour toutes et sont disponibles dans des tables ou des logiciels. On n'a plus qu'à convertir notre problème dans cette "langue" standard, trouver la probabilité, et interpréter le résultat.
\end{intuitionbox}

La notation est très standardisée pour cette loi.

\begin{remarquebox}[Notation $\phi$ et $\Phi$]
Les symboles $\phi$ (phi minuscule) pour la PDF et $\Phi$ (phi majuscule) pour la CDF de la loi normale standard sont quasi universels. Il est important de ne pas les confondre. $\phi(z)$ est la \textit{hauteur} de la courbe en $z$, tandis que $\Phi(z)$ est l'\textit{aire} sous la courbe à gauche de $z$.
\end{remarquebox}

Un détail technique important concerne le calcul de $\Phi(z)$.

\begin{remarquebox}[Absence de Primitive Simple]
L'intégrale $\int e^{-t^2/2} \, \mathrm{d}t$, nécessaire pour calculer $\Phi(z)$, n'a \textbf{pas d'expression analytique} en termes de fonctions élémentaires (polynômes, exponentielles, log, sin, cos...). C'est une fonction spéciale, connue sous le nom de \textbf{fonction d'erreur} (liée à $\Phi$ par une transformation simple). C'est la raison pour laquelle on dépend de tables ou de calculs numériques pour obtenir les valeurs de $\Phi(z)$. Heureusement, ces outils sont omniprésents aujourd'hui.
\end{remarquebox}

\subsection{Standardisation : Le Score Z}

Formalisons cette transformation clé qui relie toute loi normale à la loi standard.

\begin{theorembox}[Standardisation d'une Variable Normale]
Si $X \sim \mathcal{N}(\mu, \sigma^2)$, alors la variable $Z$ définie par :
$$ Z = \frac{X - \mu}{\sigma} $$
suit la loi normale standard, $Z \sim \mathcal{N}(0, 1)$.
\end{theorembox}

La preuve formelle utilise un changement de variable dans la fonction de répartition.

\begin{proofbox}
Soit $F_X(x)$ la CDF de $X$ et $F_Z(z)$ la CDF de $Z$. Nous voulons montrer que $F_Z(z) = \Phi(z)$.
\begin{align*}
F_Z(z) &= P(Z \le z) \\
&= P\left( \frac{X-\mu}{\sigma} \le z \right) \\
&= P(X - \mu \le z\sigma) \\
&= P(X \le \mu + z\sigma) \\
&= F_X(\mu + z\sigma)
\end{align*}
Par définition de la CDF de $X$ :
$$ F_X(x) = \int_{-\infty}^x \frac{1}{\sigma \sqrt{2\pi}} e^{ -\frac{1}{2} \left( \frac{t-\mu}{\sigma} \right)^2 } \, dt $$
Donc,
$$ F_Z(z) = \int_{-\infty}^{\mu + z\sigma} \frac{1}{\sigma \sqrt{2\pi}} e^{ -\frac{1}{2} \left( \frac{t-\mu}{\sigma} \right)^2 } \, dt $$
Effectuons le changement de variable $u = (t-\mu)/\sigma$. Alors $t = \mu + u\sigma$ et $dt = \sigma du$.
Les bornes d'intégration changent :
\begin{itemize}
    \item Quand $t \to -\infty$, $u \to -\infty$.
    \item Quand $t = \mu + z\sigma$, $u = ((\mu + z\sigma)-\mu)/\sigma = z$.
\end{itemize}
L'intégrale devient :
$$ F_Z(z) = \int_{-\infty}^{z} \frac{1}{\sigma \sqrt{2\pi}} e^{ -\frac{1}{2} u^2 } (\sigma du) $$
$$ F_Z(z) = \int_{-\infty}^{z} \frac{1}{\sqrt{2\pi}} e^{ -u^2/2 } \, du $$
C'est exactement la définition de $\Phi(z)$, la CDF de la loi normale standard. Ainsi, $Z \sim \mathcal{N}(0, 1)$.
\end{proofbox}

Cette transformation a une interprétation très concrète.

\begin{intuitionbox}[Mesurer en "Unités d'Écart-Type"]
Transformer $X$ en $Z$ s'appelle \textbf{standardiser} la variable. Le résultat, $z = \frac{x-\mu}{\sigma}$, est appelé le \textbf{Score Z} (ou cote Z). Ce score Z est une mesure \textit{sans unité} qui indique \textbf{à combien d'écarts-types} une valeur observée $x$ se situe par rapport à la moyenne $\mu$ de sa distribution.
\begin{itemize}
    \item $z = 0$ : $x$ est exactement à la moyenne ($\mathbf{x = \mu}$).
    \item $z = +1$ : $x$ est un écart-type \textit{au-dessus} de la moyenne ($\mathbf{x = \mu + \sigma}$).
    \item $z = -2$ : $x$ est deux écarts-types \textit{en dessous} de la moyenne ($\mathbf{x = \mu - 2\sigma}$).
\end{itemize}
Cette transformation est extrêmement utile pour :
\begin{enumerate}
    \item \textbf{Comparer des valeurs} issues de distributions normales différentes. Un score Z de +1.5 a toujours la même signification relative, que l'on parle de QI, de taille, ou de température.
    \item \textbf{Calculer des probabilités} en utilisant la table unique de la loi $\mathcal{N}(0, 1)$.
\end{enumerate}
\end{intuitionbox}

Un exemple classique est la comparaison de notes.

\begin{examplebox}[Comparaison de Performances]
Un étudiant A obtient 80 points à un examen où la moyenne est $\mu_A=70$ et l'écart-type $\sigma_A=5$. Un étudiant B obtient 85 points à un autre examen où $\mu_B=75$ et $\sigma_B=10$. Qui a le mieux réussi relativement à son groupe ?

Calculons les Z-scores :
$$ Z_A = \frac{80 - 70}{5} = \frac{10}{5} = +2.0 $$
$$ Z_B = \frac{85 - 75}{10} = \frac{10}{10} = +1.0 $$
L'étudiant A a un score Z plus élevé (+2.0 contre +1.0), ce qui signifie qu'il se situe plus d'écarts-types au-dessus de la moyenne de son groupe que l'étudiant B. L'étudiant A a donc relativement mieux réussi.
\end{examplebox}

\subsection{Propriétés Importantes de la Loi Normale}

La loi normale possède des propriétés de stabilité remarquables sous certaines transformations.

\begin{theorembox}[Stabilité par Transformation Linéaire]
Si $X \sim \mathcal{N}(\mu, \sigma^2)$ et $Y = aX + b$ (avec $a \neq 0$), alors $Y$ suit aussi une loi normale :
$$ Y \sim \mathcal{N}(a\mu + b, \, (a\sigma)^2) $$
L'espérance est transformée linéairement ($E[aX+b] = aE[X]+b$), et la variance est multipliée par $a^2$ ($\text{Var}(aX+b) = a^2\text{Var}(X)$).
\end{theorembox}

\begin{proofbox}
Nous utilisons le fait que si $X \sim \mathcal{N}(\mu, \sigma^2)$, alors $Z = (X-\mu)/\sigma \sim \mathcal{N}(0,1)$.
Exprimons $X$ en fonction de $Z$ : $X = \mu + \sigma Z$.
Substituons cela dans l'expression de $Y$:
$$ Y = a(\mu + \sigma Z) + b = (a\mu + b) + (a\sigma)Z $$
Posons $\mu_Y = a\mu + b$ et $\sigma_Y = |a|\sigma$. Alors $Y = \mu_Y + \sigma_Y Z$ (si $a>0$) ou $Y = \mu_Y - \sigma_Y Z$ (si $a<0$).
Dans les deux cas, $Y$ est une transformation linéaire d'une variable normale standard $Z$.
La CDF de $Y$ peut être exprimée en termes de la CDF $\Phi$ de $Z$.
Si $a>0$ :
$$ P(Y \le y) = P(\mu_Y + a\sigma Z \le y) = P(a\sigma Z \le y - \mu_Y) = P\left( Z \le \frac{y - \mu_Y}{a\sigma} \right) = \Phi\left(\frac{y - \mu_Y}{a\sigma}\right) $$
C'est la CDF d'une loi $\mathcal{N}(\mu_Y, (a\sigma)^2)$.
Le cas $a<0$ est similaire et mène au même résultat pour la distribution (la variance dépend de $a^2$).
Ainsi, $Y \sim \mathcal{N}(a\mu + b, (a\sigma)^2)$.
\end{proofbox}

Cette propriété est très utile pour les changements d'unités.

\begin{examplebox}[Changement d'Unités]
Si la température en Celsius $T_C$ suit $\mathcal{N}(20, 5^2)$, quelle est la loi de la température en Fahrenheit $T_F = \frac{9}{5}T_C + 32$ ?

$a = 9/5$, $b=32$.

Nouvelle moyenne : $E[T_F] = \frac{9}{5}(20) + 32 = 36 + 32 = 68$.

Nouvel écart-type : $\sigma_{T_F} = |a|\sigma_{T_C} = \frac{9}{5}(5) = 9$. Nouvelle variance : $\sigma_{T_F}^2 = 9^2 = 81$.

Donc, $T_F \sim \mathcal{N}(68, 9^2)$.
\end{examplebox}

Une autre propriété cruciale concerne la somme de variables normales indépendantes.

\begin{theorembox}[Stabilité par Addition (Indépendance)]
Si $X \sim \mathcal{N}(\mu_X, \sigma_X^2)$ et $Y \sim \mathcal{N}(\mu_Y, \sigma_Y^2)$ sont des variables aléatoires \textbf{indépendantes}, alors leur somme $S = X + Y$ suit aussi une loi normale :
$$ S \sim \mathcal{N}(\mu_X + \mu_Y, \, \sigma_X^2 + \sigma_Y^2) $$
Les moyennes s'ajoutent, et (grâce à l'indépendance) les variances s'ajoutent.
\end{theorembox}

La preuve formelle de ce théorème est plus avancée et utilise généralement les fonctions caractéristiques ou les fonctions génératrices des moments.

\begin{proofbox}[Idée de la preuve (via Fonctions Caractéristiques)]
La fonction caractéristique $\varphi_X(t)$ d'une variable aléatoire $X$ est définie comme $\varphi_X(t) = E[e^{itX}]$.
Pour une loi normale $X \sim \mathcal{N}(\mu, \sigma^2)$, sa fonction caractéristique est $\varphi_X(t) = e^{i\mu t - \frac{1}{2}\sigma^2 t^2}$.
Si $X$ et $Y$ sont indépendantes, la fonction caractéristique de leur somme $S=X+Y$ est le produit de leurs fonctions caractéristiques : $\varphi_S(t) = \varphi_X(t) \varphi_Y(t)$.
\begin{align*}
\varphi_S(t) &= \left( e^{i\mu_X t - \frac{1}{2}\sigma_X^2 t^2} \right) \left( e^{i\mu_Y t - \frac{1}{2}\sigma_Y^2 t^2} \right) \\
&= e^{i(\mu_X + \mu_Y)t - \frac{1}{2}(\sigma_X^2 + \sigma_Y^2)t^2}
\end{align*}
On reconnaît ici la fonction caractéristique d'une loi normale avec pour moyenne $\mu_X + \mu_Y$ et pour variance $\sigma_X^2 + \sigma_Y^2$. Comme la fonction caractéristique détermine de manière unique la distribution, on conclut que $S \sim \mathcal{N}(\mu_X + \mu_Y, \sigma_X^2 + \sigma_Y^2)$.
\end{proofbox}

Il est essentiel de se souvenir de la condition d'indépendance pour l'addition des variances.

\begin{remarquebox}[Attention à l'Indépendance]
La propriété d'addition des variances ($\sigma_S^2 = \sigma_X^2 + \sigma_Y^2$) est cruciale et ne tient \textbf{que si $X$ et $Y$ sont indépendantes}. Si elles ne le sont pas, la variance de la somme inclut un terme de covariance : $\text{Var}(X+Y) = \text{Var}(X) + \text{Var}(Y) + 2\text{Cov}(X, Y)$. Cependant, la somme de variables normales (même dépendantes) reste normale (si elles sont conjointement normales).
\end{remarquebox}

Appliquons ce théorème à un exemple concret.

\begin{examplebox}[Poids Total]
Le poids d'une pomme suit $\mathcal{N}(150g, 10^2)$. Le poids d'une orange suit $\mathcal{N}(200g, 15^2)$. On suppose les poids indépendants. Quel est la loi du poids total d'une pomme et d'une orange ?

Soit $P$ le poids de la pomme, $O$ celui de l'orange. $T = P+O$.

$E[T] = E[P] + E[O] = 150 + 200 = 350g$.

$\text{Var}(T) = \text{Var}(P) + \text{Var}(O) = 10^2 + 15^2 = 100 + 225 = 325$.

Donc, $T \sim \mathcal{N}(350, 325)$. L'écart-type du poids total est $\sqrt{325} \approx 18.03g$.
\end{examplebox}

\subsection{La Règle Empirique (68-95-99.7)}

Une conséquence directe des aires sous la courbe normale standard est une règle approximative très utile.

\begin{theorembox}[Règle Empirique]
Pour toute variable $X \sim \mathcal{N}(\mu, \sigma^2)$ :
\begin{itemize}
    \item $P(\mu - \sigma \le X \le \mu + \sigma) \approx 0.6827$ (Environ \textbf{68\%} des valeurs dans $\mu \pm \sigma$).
    \item $P(\mu - 2\sigma \le X \le \mu + 2\sigma) \approx 0.9545$ (Environ \textbf{95\%} des valeurs dans $\mu \pm 2\sigma$).
    \item $P(\mu - 3\sigma \le X \le \mu + 3\sigma) \approx 0.9973$ (Environ \textbf{99.7\%} des valeurs dans $\mu \pm 3\sigma$).
\end{itemize}
\end{theorembox}

\begin{proofbox}[Dérivation à partir de $\Phi(z)$]
Ces valeurs sont obtenues en calculant les aires sous la PDF de la loi normale standard $\mathcal{N}(0, 1)$ entre les Z-scores correspondants.
\begin{itemize}
    \item $P(-1 \le Z \le 1) = \Phi(1) - \Phi(-1) = \Phi(1) - (1 - \Phi(1)) = 2\Phi(1) - 1$.
    Avec $\Phi(1) \approx 0.8413$, on obtient $2(0.8413) - 1 \approx 0.6826$.
    \item $P(-2 \le Z \le 2) = \Phi(2) - \Phi(-2) = 2\Phi(2) - 1$.
    Avec $\Phi(2) \approx 0.9772$, on obtient $2(0.9772) - 1 \approx 0.9544$.
    \item $P(-3 \le Z \le 3) = \Phi(3) - \Phi(-3) = 2\Phi(3) - 1$.
    Avec $\Phi(3) \approx 0.99865$, on obtient $2(0.99865) - 1 \approx 0.9973$.
\end{itemize}
Ces valeurs sont souvent arrondies à 68%, 95%, et 99.7% pour faciliter la mémorisation.
\end{proofbox}

Cette règle fournit des repères très pratiques.

\begin{intuitionbox}[Repères Essentiels sur la Cloche]
Cette règle, dérivée directement des aires sous la courbe $\mathcal{N}(0, 1)$ entre $z=\pm 1$, $z=\pm 2$ et $z=\pm 3$, fournit des repères extrêmement utiles pour interpréter l'écart-type $\sigma$. Elle nous dit où se trouve la grande majorité des données.


Une observation qui tombe en dehors de l'intervalle $\mu \pm 3\sigma$ est très inhabituelle (elle n'a que 0.3% de chances de se produire). C'est souvent considéré comme une \textit{valeur aberrante} (outlier) potentielle.
\end{intuitionbox}

\subsection{Calcul de Probabilités Normales}

En pratique, pour calculer une probabilité $P(a \le X \le b)$ pour une loi $\mathcal{N}(\mu, \sigma^2)$, on utilise systématiquement la standardisation.

\begin{examplebox}[Utilisation du Z-score]
Supposons que le QI d'une population suit $\mathcal{N}(100, 15^2)$. Quelle est la probabilité $P(X > 130)$ ?

1.  \textbf{Standardiser :} $z = \frac{130 - 100}{15} = 2$. On cherche $P(Z > 2)$.
2.  \textbf{Utiliser la CDF Standard :} $P(Z > 2) = 1 - P(Z \le 2) = 1 - \Phi(2)$.
3.  \textbf{Chercher dans la table / Calculer :} $\Phi(2) \approx 0.9772$.
4.  \textbf{Résultat :} $P(X > 130) = 1 - 0.9772 = 0.0228$. Environ 2.3% de la population a un QI supérieur à 130.
\end{examplebox}

Pour les intervalles, on utilise la propriété $P(a \le Z \le b) = \Phi(b) - \Phi(a)$.

\begin{examplebox}[Probabilité entre deux valeurs]
Quelle est la probabilité $P(85 \le X \le 115)$ ? ($\mu=100, \sigma=15$)

1.  \textbf{Standardiser :} $z_1 = \frac{85 - 100}{15} = -1$, $z_2 = \frac{115 - 100}{15} = 1$. On cherche $P(-1 \le Z \le 1)$.
2.  \textbf{Utiliser la CDF Standard :} $P(-1 \le Z \le 1) = \Phi(1) - \Phi(-1)$.
3.  \textbf{Utiliser la symétrie :} $\Phi(-z) = 1 - \Phi(z)$. Donc $\Phi(-1) = 1 - \Phi(1)$.
    $P(-1 \le Z \le 1) = \Phi(1) - (1 - \Phi(1)) = 2\Phi(1) - 1$.
4.  \textbf{Chercher dans la table / Calculer :} $\Phi(1) \approx 0.8413$.
5.  \textbf{Résultat :} $P(85 \le X \le 115) \approx 2(0.8413) - 1 = 1.6826 - 1 = 0.6826$. (On retrouve la règle des 68% !)
\end{examplebox}

On peut aussi inverser le processus : trouver la valeur $x$ correspondant à une probabilité donnée.

\begin{examplebox}[Trouver une valeur pour une probabilité donnée (Problème Inverse)]
Quel est le QI minimum requis pour être dans le top 10\% de la population ? ($\mu=100, \sigma=15$).

1.  \textbf{Trouver le Z-score correspondant :} On cherche $x$ tel que $P(X > x) = 0.10$. Cela équivaut à $P(Z > z) = 0.10$, où $z = (x-100)/15$.
    Si $P(Z > z) = 0.10$, alors $P(Z \le z) = \Phi(z) = 1 - 0.10 = 0.90$.
2.  \textbf{Chercher dans la table inverse / Calculer :} On cherche la valeur $z$ pour laquelle l'aire à gauche est 0.90 (le 90ème percentile). On trouve $z \approx 1.28$.
3.  \textbf{Convertir en X :} On utilise la relation $z = (x-\mu)/\sigma$ pour trouver $x$:
    $1.28 = \frac{x - 100}{15}$
    $x = 100 + 1.28 \times 15 = 100 + 19.2 = 119.2$.
    Il faut un QI d'environ 119.2 pour être dans le top 10\%.
\end{examplebox}

\subsection{Exercices}

% --- PDF, CDF et Loi Normale Standard ---

\begin{exercicebox}[Exercice 1 : Concepts de Base $\Phi(z)$]
Soit $Z \sim \mathcal{N}(0, 1)$ la loi normale standard. Sa CDF est $\Phi(z)$.
Exprimez les probabilités suivantes en termes de $\Phi(z)$ :
\begin{enumerate}
    \item $P(Z \le 1.5)$
    \item $P(Z > 1)$
    \item $P(Z \le -1.5)$ (Indice : utilisez la symétrie $\Phi(-z) = 1 - \Phi(z)$)
    \item $P(-1.5 \le Z \le 1.5)$
\end{enumerate}
\end{exercicebox}

\begin{exercicebox}[Exercice 2 : Utilisation d'une Table $\Phi(z)$]
En utilisant une table ou une calculatrice pour la loi $\mathcal{N}(0, 1)$, on sait que $\Phi(1) \approx 0.8413$, $\Phi(1.96) \approx 0.975$ et $\Phi(2) \approx 0.9772$.
Calculez :
\begin{enumerate}
    \item $P(Z > 1)$
    \item $P(Z \le -2)$
    \item $P(-1.96 \le Z \le 1.96)$
\end{enumerate}
\end{exercicebox}

\begin{exercicebox}[Exercice 3 : Propriétés de la PDF $\phi(z)$]
Soit $\phi(z)$ la PDF de la loi $\mathcal{N}(0, 1)$.
\begin{enumerate}
    \item Quelle est la valeur de $\phi(0)$ ? (Le pic de la courbe).
    \item Que vaut $\phi(z)$ par rapport à $\phi(-z)$ ?
    \item Que vaut $\int_{-\infty}^{\infty} \phi(z) \, dz$ ?
\end{enumerate}
\end{exercicebox}

% --- Standardisation (Z-score) et Calcul de Probabilités ---

\begin{exercicebox}[Exercice 4 : Calcul de Z-scores]
Une variable aléatoire $X$ suit une loi normale $\mathcal{N}(\mu=50, \sigma^2=100)$. Notez que $\sigma=10$.
Calculez le Z-score pour les valeurs suivantes de $X$ :
\begin{enumerate}
    \item $x = 60$
    \item $x = 50$
    \item $x = 35$
\end{enumerate}
\end{exercicebox}

\begin{exercicebox}[Exercice 5 : Calcul de Probabilité (Général)]
La taille des hommes adultes dans un pays suit une loi normale $\mathcal{N}(175 \text{ cm}, 7^2 \text{ cm}^2)$.
Soit $X$ la taille d'un homme choisi au hasard. Calculez :
\begin{enumerate}
    \item $P(X \le 182 \text{ cm})$ (Indice : Standardisez $x=182$ et utilisez $\Phi(1) \approx 0.8413$)
    \item $P(X > 168 \text{ cm})$
\end{enumerate}
\end{exercicebox}

\begin{exercicebox}[Exercice 6 : Calcul de Probabilité (Intervalle)]
Les scores à un test de QI suivent une loi normale $\mathcal{N}(100, 15^2)$.
Quelle est la probabilité qu'une personne choisie au hasard ait un QI compris entre 85 et 115 ?
(Indice : Standardisez les deux bornes).
\end{exercicebox}

\begin{exercicebox}[Exercice 7 : Calcul de Probabilité (Queue Extrême)]
En utilisant la même loi $\mathcal{N}(100, 15^2)$ pour le QI :
Quelle est la probabilité qu'une personne ait un QI supérieur à 130 ?
(Indice : Utilisez $\Phi(2) \approx 0.9772$).
\end{exercicebox}

% --- Problèmes Inverses (Trouver x) ---

\begin{exercicebox}[Exercice 8 : Problème Inverse (Z-score)]
Soit $Z \sim \mathcal{N}(0, 1)$. Trouvez la valeur $z$ telle que :
(Utilisez $\Phi(1.28) \approx 0.90$ et $\Phi(1.645) \approx 0.95$)
\begin{enumerate}
    \item $P(Z \le z) = 0.90$
    \item $P(Z > z) = 0.05$ (Indice : si $P(Z>z)=0.05$, que vaut $P(Z \le z)$ ?)
    \item $P(Z \le z) = 0.10$ (Indice : utilisez la symétrie)
\end{enumerate}
\end{exercicebox}

\begin{exercicebox}[Exercice 9 : Problème Inverse (Général)]
Les scores au test $\mathcal{N}(100, 15^2)$ sont utilisés pour sélectionner des candidats. Seul le top 5\% des scores est accepté.
Quel est le score minimum requis pour être accepté ?
(Indice : Utilisez $z \approx 1.645$ pour le top 5\%).
\end{exercicebox}

\begin{exercicebox}[Exercice 10 : Problème Inverse (Intervalle Central)]
Soit $Z \sim \mathcal{N}(0, 1)$. Trouvez la valeur $z$ telle que $P(-z \le Z \le z) = 0.95$.
(Indice : si 95\% est au centre, combien reste-t-il dans chaque queue ? Utilisez $\Phi(1.96) \approx 0.975$).
\end{exercicebox}

\begin{exercicebox}[Exercice 11 : Problème Inverse (Général)]
La durée de vie d'une batterie suit $\mathcal{N}(500 \text{ heures}, 50^2 \text{ heures}^2)$.
Le fabricant veut offrir une garantie. Il ne veut remplacer que 2.5\% des batteries.
Quelle durée de garantie (en heures) doit-il proposer ?
(Indice : $P(Z \le -1.96) \approx 0.025$).
\end{exercicebox}

% --- Règle Empirique (68-95-99.7) ---

\begin{exercicebox}[Exercice 12 : Règle Empirique (Application)]
Le poids de paquets de café suit $\mathcal{N}(250g, 5^2g^2)$.
En utilisant la règle empirique (68-95-99.7), donnez un intervalle qui contient :
\begin{enumerate}
    \item Environ 68\% des poids.
    \item Environ 95\% des poids.
    \item Environ 99.7\% des poids.
\end{enumerate}
\end{exercicebox}

\begin{exercicebox}[Exercice 13 : Règle Empirique (Probabilité)]
En utilisant la situation de l'exercice 12 ($\mathcal{N}(250, 5^2)$) et la règle empirique :
\begin{enumerate}
    \item Estimez $P(245 \le X \le 255)$.
    \item Estimez $P(X \le 240)$. (Indice : L'intervalle $\mu \pm 2\sigma$ est [240, 260] et contient 95\%. Utilisez la symétrie).
\end{enumerate}
\end{exercicebox}

% --- Propriétés (Transformations Linéaires et Sommes) ---

\begin{exercicebox}[Exercice 14 : Transformation Linéaire (Celsius -> Fahrenheit)]
La température $T_C$ à midi en été dans une ville suit $\mathcal{N}(25, 3^2)$ (en degrés Celsius).
On convertit la température en Fahrenheit : $T_F = 1.8 \times T_C + 32$.
Quelle est la loi de $T_F$ ? (Donnez sa moyenne et sa variance).
\end{exercicebox}

\begin{exercicebox}[Exercice 15 : Transformation Linéaire (Z-score)]
Soit $X \sim \mathcal{N}(\mu, \sigma^2)$. Soit $Y = aX+b$.
Trouvez $a$ et $b$ (en fonction de $\mu$ et $\sigma$) tels que $Y \sim \mathcal{N}(0, 1)$.
\end{exercicebox}

\begin{exercicebox}[Exercice 16 : Somme de Normales Indépendantes]
Soit $X \sim \mathcal{N}(10, 3^2)$ et $Y \sim \mathcal{N}(20, 4^2)$. $X$ et $Y$ sont indépendantes.
Soit $S = X + Y$.
\begin{enumerate}
    \item Quelle est la loi de $S$ ? (Donnez sa moyenne et sa variance).
    \item Quel est l'écart-type de $S$ ?
\end{enumerate}
\end{exercicebox}

\begin{exercicebox}[Exercice 17 : Différence de Normales Indépendantes]
En utilisant $X$ et $Y$ de l'exercice 16, soit $D = Y - X$.
\begin{enumerate}
    \item Quelle est la loi de $D$ ? (Donnez sa moyenne et sa variance).
    \item Quel est l'écart-type de $D$ ? (Comparez-le à celui de $S$).
\end{enumerate}
\end{exercicebox}

\begin{exercicebox}[Exercice 18 : Application (Somme)]
Le poids d'une boîte vide $B$ suit $\mathcal{N}(100g, 5^2)$. Le poids du contenu $C$ suit $\mathcal{N}(800g, 10^2)$. $B$ et $C$ sont indépendants.
Soit $T = B+C$ le poids total.
\begin{enumerate}
    \item Quelle est la loi de $T$ ?
    \item Calculez $P(T > 925g)$. (Utilisez $\Phi(2) \approx 0.9772$).
\end{enumerate}
\end{exercicebox}

\begin{exercicebox}[Exercice 19 : Moyenne d'un Échantillon (Avancé)]
Soient $X_1, X_2, X_3, X_4$ quatre observations indépendantes de la loi $\mathcal{N}(10, 4^2)$.
Soit $\bar{X} = \frac{X_1 + X_2 + X_3 + X_4}{4}$ la moyenne de l'échantillon.
\begin{enumerate}
    \item Soit $S = X_1+X_2+X_3+X_4$. Quelle est la loi de $S$ ?
    \item En utilisant la transformation linéaire $\bar{X} = \frac{1}{4}S$, quelle est la loi de $\bar{X}$ ?
\end{enumerate}
\end{exercicebox}

\begin{exercicebox}[Exercice 20 : Comparaison (Différence)]
Alice et Bob passent un examen. Les notes d'Alice $A$ suivent $\mathcal{N}(80, 5^2)$. Les notes de Bob $B$ suivent $\mathcal{N}(78, 3^2)$. On suppose leurs notes indépendantes.
Quelle est la probabilité que Bob ait une meilleure note qu'Alice ?
(Indice : Calculez $P(B > A)$, ce qui est équivalent à $P(B - A > 0)$).
\end{exercicebox}

\subsection{Corrections des Exercices}

% --- Corrections : PDF, CDF et Loi Normale Standard ---

\begin{correctionbox}[Correction Exercice 1 : Concepts de Base $\Phi(z)$]
1.  $P(Z \le 1.5) = \Phi(1.5)$.
2.  $P(Z > 1) = 1 - P(Z \le 1) = 1 - \Phi(1)$.
3.  $P(Z \le -1.5) = 1 - P(Z \le 1.5) = 1 - \Phi(1.5)$.
4.  $P(-1.5 \le Z \le 1.5) = P(Z \le 1.5) - P(Z \le -1.5) = \Phi(1.5) - (1 - \Phi(1.5)) = 2\Phi(1.5) - 1$.
\end{correctionbox}

\begin{correctionbox}[Correction Exercice 2 : Utilisation d'une Table $\Phi(z)$]
Données : $\Phi(1) \approx 0.8413$, $\Phi(1.96) \approx 0.975$, $\Phi(2) \approx 0.9772$.
1.  $P(Z > 1) = 1 - \Phi(1) \approx 1 - 0.8413 = 0.1587$.
2.  $P(Z \le -2) = 1 - \Phi(2) \approx 1 - 0.9772 = 0.0228$.
3.  $P(-1.96 \le Z \le 1.96) = \Phi(1.96) - \Phi(-1.96) = \Phi(1.96) - (1 - \Phi(1.96))$
    $= 2\Phi(1.96) - 1 \approx 2(0.975) - 1 = 1.95 - 1 = 0.95$.
    (C'est l'intervalle de confiance à 95\%).
\end{correctionbox}

\begin{correctionbox}[Correction Exercice 3 : Propriétés de la PDF $\phi(z)$]
$\phi(z) = \frac{1}{\sqrt{2\pi}} e^{-z^2/2}$.
1.  $\phi(0) = \frac{1}{\sqrt{2\pi}} e^{0} = \frac{1}{\sqrt{2\pi}} \approx 0.3989$.
2.  Puisque $z^2 = (-z)^2$, on a $\phi(z) = \phi(-z)$. La fonction est paire (symétrique par rapport à l'axe y).
3.  Par définition d'une PDF, l'aire totale sous la courbe doit être 1. $\int_{-\infty}^{\infty} \phi(z) \, dz = 1$.
\end{correctionbox}

% --- Corrections : Standardisation (Z-score) et Calcul de Probabilités ---

\begin{correctionbox}[Correction Exercice 4 : Calcul de Z-scores]
$X \sim \mathcal{N}(\mu=50, \sigma^2=100) \implies \sigma=10$.
$Z = \frac{X - \mu}{\sigma}$.
1.  $x = 60 \implies z = (60 - 50) / 10 = 10 / 10 = 1$.
2.  $x = 50 \implies z = (50 - 50) / 10 = 0 / 10 = 0$.
3.  $x = 35 \implies z = (35 - 50) / 10 = -15 / 10 = -1.5$.
\end{correctionbox}

\begin{correctionbox}[Correction Exercice 5 : Calcul de Probabilité (Général)]
$X \sim \mathcal{N}(175, 7^2)$. $\mu=175, \sigma=7$.
1.  $P(X \le 182) = P\left(Z \le \frac{182 - 175}{7}\right) = P(Z \le \frac{7}{7}) = P(Z \le 1) = \Phi(1) \approx 0.8413$.
2.  $P(X > 168) = P\left(Z > \frac{168 - 175}{7}\right) = P(Z > \frac{-7}{7}) = P(Z > -1)$.
    Par symétrie, $P(Z > -1) = P(Z < 1) = \Phi(1) \approx 0.8413$.
\end{correctionbox}

\begin{correctionbox}[Correction Exercice 6 : Calcul de Probabilité (Intervalle)]
$X \sim \mathcal{N}(100, 15^2)$. $\mu=100, \sigma=15$.
On cherche $P(85 \le X \le 115)$.
$z_1 = (85 - 100) / 15 = -15 / 15 = -1$.
$z_2 = (115 - 100) / 15 = 15 / 15 = 1$.
$P(-1 \le Z \le 1) = \Phi(1) - \Phi(-1) = \Phi(1) - (1 - \Phi(1)) = 2\Phi(1) - 1$.
En utilisant $\Phi(1) \approx 0.8413$, $P \approx 2(0.8413) - 1 = 1.6826 - 1 = 0.6826$.
(On retrouve la règle des 68\%).
\end{correctionbox}

\begin{correctionbox}[Correction Exercice 7 : Calcul de Probabilité (Queue Extrême)]
$X \sim \mathcal{N}(100, 15^2)$.
On cherche $P(X > 130)$.
$z = (130 - 100) / 15 = 30 / 15 = 2$.
$P(X > 130) = P(Z > 2) = 1 - P(Z \le 2) = 1 - \Phi(2)$.
En utilisant $\Phi(2) \approx 0.9772$, $P \approx 1 - 0.9772 = 0.0228$.
\end{correctionbox}

% --- Corrections : Problèmes Inverses (Trouver x) ---

\begin{correctionbox}[Correction Exercice 8 : Problème Inverse (Z-score)]
1.  $P(Z \le z) = 0.90 \implies z = \Phi^{-1}(0.90) \approx 1.28$.
2.  $P(Z > z) = 0.05 \implies P(Z \le z) = 1 - 0.05 = 0.95$.
    $z = \Phi^{-1}(0.95) \approx 1.645$.
3.  $P(Z \le z) = 0.10$. C'est dans la queue gauche. Par symétrie, $z = - \Phi^{-1}(1 - 0.10) = - \Phi^{-1}(0.90)$.
    $z \approx -1.28$.
\end{correctionbox}

\begin{correctionbox}[Correction Exercice 9 : Problème Inverse (Général)]
$X \sim \mathcal{N}(100, 15^2)$. On cherche $x$ tel que $P(X > x) = 0.05$.
1.  Trouver le Z-score : $P(Z > z) = 0.05 \implies P(Z \le z) = 0.95 \implies z \approx 1.645$.
2.  Convertir en $x$ : $z = (x-\mu)/\sigma \implies x = \mu + z\sigma$.
    $x = 100 + (1.645)(15) = 100 + 24.675 = 124.675$.
    Le score minimum est d'environ 125.
\end{correctionbox}

\begin{correctionbox}[Correction Exercice 10 : Problème Inverse (Intervalle Central)]
$P(-z \le Z \le z) = 0.95$.
Si 95\% est au centre, il reste $1 - 0.95 = 0.05$ (ou 5\%) dans les deux queues.
Par symétrie, chaque queue a $0.05 / 2 = 0.025$.
La probabilité à gauche de $z$ est $P(Z \le z) = 0.95 + 0.025 = 0.975$.
On cherche $z = \Phi^{-1}(0.975)$.
En utilisant l'indice, $z \approx 1.96$.
\end{correctionbox}

\begin{correctionbox}[Correction Exercice 11 : Problème Inverse (Général)]
$X \sim \mathcal{N}(500, 50^2)$. On cherche $x$ tel que $P(X \le x) = 0.025$.
1.  Trouver le Z-score : $P(Z \le z) = 0.025$. C'est la queue gauche.
    En utilisant l'indice $P(Z \le -1.96) \approx 0.025$, on a $z \approx -1.96$.
2.  Convertir en $x$ : $x = \mu + z\sigma$.
    $x = 500 + (-1.96)(50) = 500 - 98 = 402$.
    Le fabricant doit proposer une garantie de 402 heures.
\end{correctionbox}

% --- Corrections : Règle Empirique (68-95-99.7) ---

\begin{correctionbox}[Correction Exercice 12 : Règle Empirique (Application)]
$X \sim \mathcal{N}(\mu=250, \sigma=5)$.
1.  68\% $\implies \mu \pm 1\sigma = 250 \pm 5 \implies [245, 255]$.
2.  95\% $\implies \mu \pm 2\sigma = 250 \pm 2(5) = 250 \pm 10 \implies [240, 260]$.
3.  99.7\% $\implies \mu \pm 3\sigma = 250 \pm 3(5) = 250 \pm 15 \implies [235, 265]$.
\end{correctionbox}

\begin{correctionbox}[Correction Exercice 13 : Règle Empirique (Probabilité)]
1.  $P(245 \le X \le 255)$ est l'intervalle $\mu \pm 1\sigma$.
    La probabilité est d'environ 68\% ou 0.68.
2.  L'intervalle $\mu \pm 2\sigma$ est $[240, 260]$ et contient 95\% des données.
    Il reste $100\% - 95\% = 5\%$ dans les deux queues (i.e., $P(X < 240) + P(X > 260) = 0.05$).
    Par symétrie, la queue gauche $P(X < 240)$ est $0.05 / 2 = 0.025$.
    La probabilité est d'environ 2.5\% ou 0.025.
\end{correctionbox}

% --- Corrections : Propriétés (Transformations Linéaires et Sommes) ---

\begin{correctionbox}[Correction Exercice 14 : Transformation Linéaire]
$T_C \sim \mathcal{N}(25, 3^2)$. $T_F = a T_C + b$ avec $a=1.8$ et $b=32$.
Loi de $T_F$ : $T_F \sim \mathcal{N}(a\mu + b, (a\sigma)^2)$.
Moyenne : $E[T_F] = 1.8(25) + 32 = 45 + 32 = 77$.
Variance : $\text{Var}(T_F) = (1.8)^2 \text{Var}(T_C) = (1.8)^2 (3^2) = (1.8 \times 3)^2 = (5.4)^2 = 29.16$.
Donc, $T_F \sim \mathcal{N}(77, 29.16)$.
\end{correctionbox}

\begin{correctionbox}[Correction Exercice 15 : Transformation Linéaire (Z-score)]
On veut $Y = aX+b \sim \mathcal{N}(0, 1)$.
$E[Y] = aE[X] + b = a\mu + b$. On veut $a\mu + b = 0$.
$\text{Var}(Y) = a^2 \text{Var}(X) = a^2 \sigma^2$. On veut $a^2 \sigma^2 = 1$.
De $\text{Var}(Y)=1 \implies a^2 = 1/\sigma^2 \implies a = 1/\sigma$ (en supposant $a>0$).
De $E[Y]=0 \implies (1/\sigma)\mu + b = 0 \implies b = -\mu/\sigma$.
Les constantes sont $a = 1/\sigma$ et $b = -\mu/\sigma$. (C'est la définition de la standardisation).
\end{correctionbox}

\begin{correctionbox}[Correction Exercice 16 : Somme de Normales Indépendantes]
$X \sim \mathcal{N}(10, 9)$ et $Y \sim \mathcal{N}(20, 16)$. $S = X+Y$.
1.  La somme de normales indépendantes est une normale.
    $E[S] = E[X] + E[Y] = 10 + 20 = 30$.
    $\text{Var}(S) = \text{Var}(X) + \text{Var}(Y) = 9 + 16 = 25$.
    Donc, $S \sim \mathcal{N}(30, 25)$.
2.  $\text{Var}(S) = 25 \implies \sigma_S = \sqrt{25} = 5$.
\end{correctionbox}

\begin{correctionbox}[Correction Exercice 17 : Différence de Normales Indépendantes]
$D = Y - X$.
1.  La différence est aussi une normale.
    $E[D] = E[Y] - E[X] = 20 - 10 = 10$.
    $\text{Var}(D) = \text{Var}(Y + (-1)X) = \text{Var}(Y) + (-1)^2 \text{Var}(X) = \text{Var}(Y) + \text{Var}(X)$.
    $\text{Var}(D) = 16 + 9 = 25$.
    Donc, $D \sim \mathcal{N}(10, 25)$.
2.  $\sigma_D = \sqrt{25} = 5$. (Identique à $\sigma_S$. La variance s'additionne toujours).
\end{correctionbox}

\begin{correctionbox}[Correction Exercice 18 : Application (Somme)]
$B \sim \mathcal{N}(100, 25)$, $C \sim \mathcal{N}(800, 100)$. $T = B+C$.
1.  $E[T] = E[B] + E[C] = 100 + 800 = 900$.
    $\text{Var}(T) = \text{Var}(B) + \text{Var}(C) = 25 + 100 = 125$.
    $T \sim \mathcal{N}(900, 125)$.
2.  $P(T > 925)$. $\sigma_T = \sqrt{125} = \sqrt{25 \times 5} = 5\sqrt{5} \approx 11.18$.
    $z = (925 - 900) / \sqrt{125} = 25 / (5\sqrt{5}) = 5/\sqrt{5} = \sqrt{5} \approx 2.236$.
    $P(T > 925) = P(Z > 2.236) = 1 - \Phi(2.236) \approx 1 - 0.9873 = 0.0127$.
    (Note : L'indice $\Phi(2) \approx 0.9772$ semble être une approximation pour un $z$ de 2, qui n'est pas le bon $z$ ici).
\end{correctionbox}

\begin{correctionbox}[Correction Exercice 19 : Moyenne d'un Échantillon (Avancé)]
$X_i \sim \mathcal{N}(10, 16)$ (indép.). $\bar{X} = \frac{1}{4} S$ où $S = X_1+X_2+X_3+X_4$.
1.  $S$ est une somme de normales indépendantes.
    $E[S] = E[X_1] + \dots + E[X_4] = 4 \times 10 = 40$.
    $\text{Var}(S) = \text{Var}(X_1) + \dots + \text{Var}(X_4) = 4 \times 16 = 64$.
    $S \sim \mathcal{N}(40, 64)$.
2.  $\bar{X}$ est une transformation linéaire de $S$.
    $E[\bar{X}] = E[\frac{1}{4}S] = \frac{1}{4}E[S] = \frac{1}{4}(40) = 10$.
    $\text{Var}(\bar{X}) = \text{Var}(\frac{1}{4}S) = (\frac{1}{4})^2 \text{Var}(S) = \frac{1}{16}(64) = 4$.
    $\bar{X} \sim \mathcal{N}(10, 4)$.
\end{correctionbox}

\begin{correctionbox}[Correction Exercice 20 : Comparaison (Différence)]
$A \sim \mathcal{N}(80, 25)$, $B \sim \mathcal{N}(78, 9)$. Indép.
On cherche $P(B > A)$, ce qui est $P(B - A > 0)$.
Soit $D = B - A$. $D$ suit une loi normale.
$E[D] = E[B] - E[A] = 78 - 80 = -2$.
$\text{Var}(D) = \text{Var}(B) + \text{Var}(A) = 9 + 25 = 34$.
Donc $D \sim \mathcal{N}(-2, 34)$. $\sigma_D = \sqrt{34} \approx 5.83$.
On cherche $P(D > 0)$.
$z = (0 - (-2)) / \sqrt{34} = 2 / \sqrt{34} \approx 0.342$.
$P(D > 0) = P(Z > 0.342) = 1 - \Phi(0.342) \approx 1 - 0.6338 = 0.3662$.
Il y a environ 36.6\% de chance que Bob ait une meilleure note.
\end{correctionbox}

\subsection{Exercices Pratiques (Python)}

L'une des applications les plus célèbres de la loi normale est la modélisation des rendements financiers. Bien que ce modèle ne soit pas parfait (les krachs boursiers sont plus fréquents que ne le prédit la loi normale), il constitue la pierre angulaire de la finance moderne.

Nous allons supposer que les \textbf{rendements logarithmiques} quotidiens d'un actif financier (comme l'indice S\&P 500) suivent une loi normale $X \sim \mathcal{N}(\mu, \sigma^2)$.

\begin{itemize}
    \item $\mu$ est le rendement moyen quotidien (souvent proche de zéro).
    \item $\sigma$ est la volatilité quotidienne (l'écart-type du rendement).
\end{itemize}

Nous utiliserons \texttt{yfinance} pour obtenir des données réelles, \texttt{numpy} pour les calculs, et \texttt{scipy.stats} pour les fonctions $\Phi$ (CDF) et $\Phi^{-1}$ (PPF).

\begin{codecell}
# Cellule d'installation et d'importation
pip install numpy pandas yfinance scipy
\end{codecell}

\begin{codecell}
import numpy as np
import pandas as pd
import yfinance as yf
from scipy.stats import norm
\end{codecell}

\begin{exercicebox}[Exercice 1 : Modélisation des Rendements du S\&P 500]
Notre première étape est d'obtenir les données de l'indice S\&P 500 (ticker : GSPC) et d'estimer les paramètres $\mu$ et $\sigma$ de notre modèle normal.

\textbf{Votre tâche :}
\begin{enumerate}
    \item Télécharger les données du GSPC des 5 dernières années.
    \item Calculer les rendements logarithmiques quotidiens. La formule est $R = \log(P_t / P_{t-1})$. (Indice : utilisez \texttt{np.log(data['Close'] / data['Close'].shift(1))}).
    \item Estimer $\mu$ (la moyenne) et $\sigma$ (l'écart-type) de ces rendements.
    \item Afficher $\mu$ et $\sigma$. Vous avez maintenant votre modèle $X \sim \mathcal{N}(\mu, \sigma^2)$ !
\end{enumerate}


\begin{codecell}
import numpy as np
import yfinance as yf

ticker = "GSPC"
data = yf.download(ticker, period='5y')

# 1. Calculer les rendements log (log(P_t / P_{t-1}))
# Indice : utilisez .shift(1) pour P_{t-1}
# log_returns = ...
log_returns = log_returns.dropna() # On enleve la premiere valeur (NaN)

# 2. Estimer mu (moyenne) et sigma (ecart-type)
# mu = ...
# sigma = ...

# 3. Afficher les parametres
# print(f"Modele N(mu={mu:.6f}, sigma={sigma:.6f})")
\end{codecell}
\end{exercicebox}

\begin{exercicebox}[Exercice 2 : Calcul de Probabilité (Z-score)]
Utilisons notre modèle $\mathcal{N}(\mu, \sigma^2)$ de l'exercice 1. Un "krach" pourrait être défini comme une baisse de plus de 3\% en une seule journée.

Quelle est la probabilité que cela se produise, selon notre modèle ?

\textbf{Votre tâche :}
\begin{enumerate}
    \item Définir la valeur $x$ d'une baisse de 3\% (en rendement log) : $x = \log(0.97)$.
    \item Standardiser $x$ pour obtenir le Z-score : $z = (x - \mu) / \sigma$.
    \item Utiliser \texttt{scipy.stats.norm.cdf(z)} (qui est $\Phi(z)$) pour trouver la probabilité $P(X \le x)$.
\end{enumerate}

\begin{codecell}
from scipy.stats import norm

# Supposons que mu et sigma sont definis (de l'Ex 1)
# mu = ... (copiez votre valeur de l'Ex 1)
# sigma = ... (copiez votre valeur de l'Ex 1)

# 1. Definir x pour une baisse de 3%
# x = np.log(...)

# 2. Standardiser x pour obtenir le Z-score
# z_score = ...

# 3. Utiliser norm.cdf(z) pour trouver P(X <= x)
# probability = ...

# print(f"Probabilite d'une baisse > 3% : {probability:.8f}")
\end{codecell}
\end{exercicebox}

\begin{exercicebox}[Exercice 3 : Problème Inverse (Value at Risk - VaR)]
Le "Value at Risk" (VaR) est un concept financier qui répond à la question : "Quel est le montant minimum que je peux m'attendre à perdre avec une probabilité $p$ ?"

Calculons le 5\% VaR quotidien. C'est la valeur $x$ (rendement) telle que $P(X \le x) = 0.05$.

\textbf{Votre tâche :}
\begin{enumerate}
    \item Trouver le Z-score $z$ qui correspond au 5ème percentile (probabilité 0.05).
    (Indice : utilisez \texttt{scipy.stats.norm.ppf(0.05)}, qui est $\Phi^{-1}(0.05)$).
    \item "Dé-standardiser" ce Z-score pour trouver la valeur $x$ : $x = \mu + z \cdot \sigma$.
    \item Interpréter le résultat (convertir $x$ en pourcentage : $np.exp(x) - 1$).
\end{enumerate}

\begin{codecell}
from scipy.stats import norm

# Supposons que mu et sigma sont definis (de l'Ex 1)
# mu = ...
# sigma = ...

probabilite = 0.05

# 1. Trouver le Z-score pour 5% (Indice: norm.ppf)
# z_score_var = ...

# 2. De-standardiser pour trouver x (x = mu + z*sigma)
# x_var = ...

# 3. Convertir en pourcentage (Indice: np.exp(x_var) - 1)
# percent_loss = ...

# print(f"Le 5% VaR est une perte de {abs(percent_loss):.2f}%")
\end{codecell}
\end{exercicebox}

\begin{exercicebox}[Exercice 4 : Règle Empirique (68-95-99.7)]
Vérifions à quel point la règle empirique (68-95-99.7) s'applique à nos données réelles de \texttt{log\_returns}.

\textbf{Votre tâche :}
\begin{enumerate}
    \item Définir les intervalles $1\sigma$, $2\sigma$, et $3\sigma$ autour de la moyenne $\mu$.
    \item Calculer le pourcentage réel de rendements (dans \texttt{log\_returns}) qui tombent dans chacun de ces trois intervalles.
    \item Comparer ces pourcentages empiriques aux valeurs théoriques (68.3\%, 95.4\%, 99.7\%).
\end{enumerate}

\begin{codecell}
# Supposons que mu, sigma, et log_returns sont definis

# 1. Definir les bornes
borne_1s_inf = mu - 1 * sigma
borne_1s_sup = mu + 1 * sigma
# ... faire de meme pour 2s et 3s ...
borne_2s_inf = ...
borne_2s_sup = ...
borne_3s_inf = ...
borne_3s_sup = ...

# 2. Compter le pourcentage de 'log_returns' dans chaque intervalle
# Indice: ((log_returns > borne_inf) & (log_returns < borne_sup)).mean()
# within_1s = ...
# within_2s = ...
# within_3s = ...

# print(f"Empirique 1-sigma: {within_1s:.4f} (Theorique: 0.6827)")
# print(f"Empirique 2-sigma: {within_2s:.4f} (Theorique: 0.9545)")
# print(f"Empirique 3-sigma: {within_3s:.4f} (Theorique: 0.9973)")
\end{codecell}
\end{exercicebox}

\begin{exercicebox}[Exercice 5 : Stabilité par Addition (Portfolio Simple)]
Un portfolio est composé de 50\% de S\&P 500 (GSPC) et 50\% d'Or (GC=F).
Nous allons modéliser la loi du rendement de ce portfolio, $P = 0.5 X_S + 0.5 X_G$.

Nous utiliserons les théorèmes $E[aX+bY] = aE[X]+bE[Y]$ et, en supposant (pour cet exercice) l'indépendance : $\text{Var}(aX+bY) = a^2\text{Var}(X) + b^2\text{Var}(Y)$.

\textbf{Votre tâche :}
\begin{enumerate}
    \item Télécharger les données de l'Or (\texttt{`GC=F`}) et calculer $\mu_G$ et $\sigma_G^2$ (variance) de ses rendements log.
    \item Récupérer $\mu_S$ et $\sigma_S^2$ (variance) du S\&P 500 de l'exercice 1.
    \item Calculer la moyenne du portfolio $\mu_P = 0.5\mu_S + 0.5\mu_G$.
    \item Calculer la variance du portfolio $\sigma_P^2 = (0.5)^2\sigma_S^2 + (0.5)^2\sigma_G^2$.
    \item Afficher l'écart-type $\sigma_P$ et le comparer à $\sigma_S$ et $\sigma_G$.
\end{enumerate}

\begin{codecell}
# mu_S et sigma_S de l'Ex 1
# mu_S = ...
# var_S = sigma_S**2

# 1. Obtenir les donnees pour l'Or ('GC=F') et calculer mu_G, var_G
# gold_data = ...
# gold_returns = ...
# mu_G = ...
# var_G = ...

# 2. Poids
w_S = 0.5
w_G = 0.5

# 3. Calculer mu_P (moyenne du portfolio)
# mu_P = ...

# 4. Calculer var_P (en supposant l'independance)
# var_P = ...
# sigma_P = np.sqrt(var_P)

# print(f"Volatilite S&P 500: {sigma_S:.6f}")
# print(f"Volatilite Or: {np.sqrt(var_G):.6f}")
# print(f"Volatilite Portfolio (indep): {sigma_P:.6f}")
\end{codecell}
\end{exercicebox}
\newpage

\section{Moments d'une distribution}

\subsection{Définitions fondamentales des moments}

Après avoir défini l'espérance ($\mu$) et la variance ($\sigma^2$), qui sont les moments d'ordre 1 et 2, nous pouvons généraliser cette idée pour capturer des informations plus subtiles sur la forme d'une distribution.

\begin{definitionbox}[Types de Moments]
Soit $X$ une variable aléatoire ayant une espérance $\mu$ et une variance $\sigma^2$. Pour tout entier positif $m$, on définit les moments suivants :
\begin{itemize}
    \item \textbf{$m$-ième moment (non centré)} : $E[X^m]$.
    \item \textbf{$m$-ième moment centré} : $E[(X - \mu)^m]$.
    \item \textbf{$m$-ième moment standardisé} : $E\left[\left(\frac{X - \mu}{\sigma}\right)^m\right]$.
\end{itemize}
Les moments centrés et standardisés permettent d'étudier les propriétés de la distribution indépendamment de sa position ($\mu$) et de son échelle ($\sigma$).
\end{definitionbox}

\subsection{Asymétrie (Skewness)}

Le premier moment nous donne la tendance centrale. Le deuxième moment (la variance) nous donne la dispersion. Le troisième moment, lui, va nous renseigner sur la \textit{symétrie} de la distribution.

\begin{definitionbox}[Asymétrie (Skewness)]
L'\textbf{asymétrie} (ou \textit{skewness}) d'une variable aléatoire $X$ de moyenne $\mu$ et d'écart-type $\sigma$ est définie comme le \textbf{troisième moment standardisé} :
$$ \text{Skew}(X) = E\left[ \left( \frac{X - \mu}{\sigma} \right)^3 \right]. $$
\end{definitionbox}

\begin{intuitionbox}[Comprendre la Formule du Skewness]
Pour une variable aléatoire $X$ de moyenne $\mu$ et d'écart-type $\sigma$, le \textbf{skewness} est défini comme :
\[
\text{Skew}(X) = \frac{E[(X - \mu)^3]}{\sigma^3}
\]

\medskip

\textbf{Logique du numérateur : le moment centré d'ordre 3}
\begin{itemize}
    \item Le terme $(X - \mu)^3$ est le \textbf{cube de l'écart à la moyenne}
    \item Contrairement à $(X - \mu)^2$ (toujours positif), le cube \textbf{conserve le signe} de l'écart
    \item Il pondère différemment les observations à gauche et à droite de la moyenne
\end{itemize}

\medskip

% --- MODIFIÉ : Tableau supprimé et fusionné dans la liste ---
\textbf{Interprétation intuitive}
\begin{itemize}
    \item \textbf{Skewness = 0 (Symétrique)} : La distribution est symétrique. Les écarts positifs et négatifs s'annulent. Typiquement : Moyenne = Médiane = Mode.
    \item \textbf{Skewness > 0 (Queue à droite)} : La distribution présente une queue longue à droite. Les grandes valeurs positives sont amplifiées par le cube. Les valeurs extrêmes tirent la moyenne vers la droite.
    \item \textbf{Skewness < 0 (Queue à gauche)} : La distribution présente une queue longue à gauche. Les écarts négatifs dominent. Les valeurs extrêmes tirent la moyenne vers la gauche.
\end{itemize}
% --- FIN MODIFICATION ---

\medskip

\textbf{Pourquoi $\sigma^3$ au dénominateur ?}
\begin{itemize}
    \item Le moment d'ordre 3 est homogène à des unités au cube
    \item On divise par $\sigma^3$ pour obtenir un coefficient \textbf{sans dimension}
    \item Permet la comparaison entre distributions de différentes échelles
\end{itemize}
\end{intuitionbox}

\begin{remarquebox}[Pourquoi Standardiser ?]
En standardisant d'abord ($\frac{X-\mu}{\sigma}$), la définition de $\text{Skew}(X)$ ne dépend ni de la position ($\mu$) ni de l'échelle ($\sigma$) de la distribution, ce qui est raisonnable puisque ces informations sont déjà fournies par la moyenne et l'écart-type. De plus, cette standardisation garantit que l'asymétrie est invariante par changement d'unité de mesure (par exemple, passer des pouces aux mètres n'affecte pas la valeur de l'asymétrie).
\end{remarquebox}

\subsection{Propriétés de symétrie}

Le skewness est une mesure numérique de l'asymétrie. Mais nous pouvons aussi définir la symétrie de manière formelle.

\begin{definitionbox}[Symétrie d'une Variable Aléatoire]
On dit qu'une variable aléatoire $X$ a une distribution \textbf{symétrique} autour de $\mu$ si la variable $X - \mu$ a la même distribution que $\mu - X$. On dit aussi que $X$ est symétrique ou que sa distribution est symétrique. Ces trois formulations ont le même sens.
\end{definitionbox}

\begin{theorembox}[Symétrie en Termes de Fonction de Densité]
Soit $X$ une variable aléatoire continue de fonction de densité de probabilité (PDF) $f$. Alors, $X$ est symétrique autour de $\mu$ si et seulement si :
$$ f(x) = f(2\mu - x) \quad \text{pour tout } x. $$
\end{theorembox}

\begin{proofbox}[Preuve du Théorème de Symétrie]
Soit $F$ la fonction de répartition (CDF) de $X$. Si la symétrie tient, alors :
$$ F(x) = P(X \le x) = P(X - \mu \le x - \mu) = P(\mu - X \le x - \mu) = P(X \ge 2\mu - x) = 1 - F(2\mu - x). $$

En prenant la dérivée des deux côtés par rapport à $x$, on obtient :
$$ f(x) = \frac{d}{dx}F(x) = \frac{d}{dx}[1 - F(2\mu - x)] = f(2\mu - x). $$

Cela démontre que la condition $f(x) = f(2\mu - x)$ est nécessaire et suffisante pour la symétrie.
\end{proofbox}

\subsection{Aplatissement (Kurtosis)}

Après l'asymétrie (ordre 3), le moment d'ordre 4 nous informe sur "l'épaisseur" des queues de la distribution, c'est-à-dire la probabilité d'obtenir des valeurs très éloignées de la moyenne.

\begin{definitionbox}[Kurtosis (Aplatissement)]
Pour une variable aléatoire $X$ de moyenne $\mu$ et d'écart-type $\sigma$, le \textbf{kurtosis} est défini comme le \textbf{quatrième moment standardisé} :
$$ \text{Kurtosis}(X) = E\left[ \left( \frac{X - \mu}{\sigma} \right)^4 \right]. $$

Dans la pratique, on utilise plus souvent le \textbf{kurtosis excessif} (ou excès de kurtosis), défini comme :
$$ \text{Excess Kurtosis}(X) = E\left[ \left( \frac{X - \mu}{\sigma} \right)^4 \right] - 3. $$
La soustraction de 3 fait en sorte que le kurtosis d'une loi normale soit égal à 0.
\end{definitionbox}

\begin{intuitionbox}[Comprendre la Kurtosis]
Pour une variable aléatoire $X$, le \textbf{kurtosis} est défini comme :
\[
\text{Kurt}(X) = \frac{E[(X - \mu)^4]}{\sigma^4}
\]
et l'\textbf{excess kurtosis} (kurtosis excédentaire) comme : $\text{Excess Kurtosis} = \text{Kurt}(X) - 3$.

\medskip

\textbf{Pourquoi le moment d'ordre 4 ?}
\begin{itemize}
    \item Comme la variance, on utilise une puissance paire (pas d'effet de signe)
    \item La puissance 4 \textbf{amplifie énormément les écarts extrêmes}
    \item Mesure le \textbf{poids des queues} et la \textbf{concentration autour de la moyenne}
\end{itemize}

\medskip

% --- MODIFIÉ : Tableau supprimé et fusionné dans la liste ---
\textbf{Interprétation intuitive (basée sur l'Excess Kurtosis)}
\begin{itemize}
    \item \textbf{Leptokurtique (Excess Kurtosis > 0)} : Kurtosis total > 3. Distribution pointue avec des queues épaisses. Les événements extrêmes sont plus probables que pour une loi normale.
    \item \textbf{Mésocurtique (Excess Kurtosis = 0)} : Kurtosis total = 3. C'est la référence (loi normale).
    \item \textbf{Platykurtique (Excess Kurtosis < 0)} : Kurtosis total < 3. Distribution aplatie avec des queues légères et un centre large. Les événements extrêmes sont moins probables.
\end{itemize}
% --- FIN MODIFICATION ---

\medskip

\textbf{Application en finance}
\begin{itemize}
    \item Les rendements financiers ont souvent un excès de kurtosis positif
    \item Indique une probabilité plus élevée d'événements extrêmes que la loi normale
    \item Justifie le "vol smile" dans les options
\end{itemize}

\medskip

\textbf{Pourquoi $\sigma^4$ au dénominateur ?}
\begin{itemize}
    \item Le moment d'ordre 4 est homogène à des unités$^4$
    \item On divise par $\sigma^4$ pour un coefficient \textbf{sans dimension}
\end{itemize}
\end{intuitionbox}

\subsection{Exemples de distributions}

Pour bien fixer les idées, comparons le skewness et le kurtosis de plusieurs distributions classiques. Notez que dans les graphiques suivants, le "Kurtosis" affiché est l'\textit{excess kurtosis} (centré à 0).

\begin{examplebox}[La Distribution Normale (Mésokurtique)]

\begin{center}
\begin{tikzpicture}
  \begin{axis}[
    width=0.8\textwidth,
    height=0.5\textwidth,
    xlabel={$x$},
    title={Densité Normale ($\mu=0$, $\sigma=1$)},
    grid=both,
    grid style={line width=.1pt, draw=gray!30},
    major grid style={line width=.2pt,draw=gray!50},
    domain=-4:4,
    samples=100,
    enlargelimits=false,
    axis lines=middle,
    xmin=-4, xmax=4,
    ymin=0, ymax=0.55
  ]
 
  % Courbe de densité
  \addplot [thick, color=blue, fill=blue!20, fill opacity=0.5] 
    {1/(sqrt(2*pi))*exp(-x^2/2)} \closedcycle;
 
  % Ligne de la moyenne
  \addplot [red, dashed, thick] coordinates {(0,0) (0,0.4)};
  \node[red, fill=white, font=\scriptsize, rounded corners, inner sep=2pt, opacity=0.8] at (axis cs:0.5,0.35) {Moyenne};
 
  % Boîte de texte avec moments seulement
  \node [draw=black, fill=white, rounded corners, font=\scriptsize, align=left, anchor=north east, opacity=0.8] 
    at (axis description cs:0.95,0.95) { % MODIFIÉ
    \textbf{Moments :}\\
    • Moyenne = 0.00\\
    • Variance = 1.00\\
    • Skewness = 0.00\\
    • Kurtosis = 0.00
    };
    
  \end{axis}
\end{tikzpicture}
\end{center}

La distribution normale est l'archétype de la courbe en cloche. Imaginez une cible : la majorité des flèches touchent le centre, et plus on s'éloigne du centre, moins il y a de chances d'être touché. C'est une distribution parfaitement symétrique, ce qui se traduit par un \textbf{skewness nul (0.00)}. Son pic est ni trop pointu, ni trop plat : c'est notre point de référence, on dit qu'elle est \textbf{mésokurtique}, d'où son kurtosis de \textbf{0.00}. C'est la base de nombreuses analyses statistiques car elle modélise naturellement beaucoup de phénomènes.

\end{examplebox}

\begin{examplebox}[La Distribution Exponentielle (Asymétrique à Droite)]

\begin{center}
\begin{tikzpicture}
  \begin{axis}[
    width=0.8\textwidth,
    height=0.5\textwidth,
    xlabel={$x$},
    title={Densité Exponentielle ($\lambda=1$)},
    grid=both,
    grid style={line width=.1pt, draw=gray!30},
    major grid style={line width=.2pt,draw=gray!50},
    domain=0:6,
    samples=100,
    enlargelimits=false,
    axis lines=middle,
    xmin=0, xmax=6,
    ymin=0, ymax=1.1
  ]
 
  % Courbe de densité exponentielle
  \addplot [thick, color=blue, fill=blue!20, fill opacity=0.5] 
    {exp(-x)} \closedcycle;
 
  % Ligne de la moyenne
  \addplot [red, dashed, thick] coordinates {(1,0) (1,0.37)};
  \node[red, fill=white, font=\scriptsize, rounded corners, inner sep=2pt, opacity=0.8] at (axis cs:1.5,0.3) {Moyenne};
 
  % Boîte de texte avec moments seulement
  \node [draw=black, fill=white, rounded corners, font=\scriptsize, align=left, anchor=north east, opacity=0.8] 
    at (axis description cs:0.95,0.95) { % MODIFIÉ
    \textbf{Moments :}\\
    • Moyenne = 1.00\\
    • Variance = 1.00\\
    • Skewness = 2.00\\
    • Kurtosis = 6.00
    };
    
  \end{axis}
\end{tikzpicture}
\end{center}

Imaginez le temps d'attente avant un événement rare, comme un appel téléphonique. La plupart du temps, l'appel arrive vite, mais il peut parfois y avoir de longues attentes. C'est exactement ce que modélise la distribution exponentielle : un pic à gauche et une longue queue à droite. Cela se traduit par un \textbf{skewness positif élevé (2.00)}, indiquant une asymétrie marquée. Elle est aussi \textbf{leptokurtique} (\textbf{kurtosis = 6.00}) : son pic est pointu, et la longue queue droite signifie qu'il y a une probabilité non négligeable de valeurs extrêmes.

\end{examplebox}

\begin{examplebox}[La Distribution Uniforme (Platykurtique)]

\begin{center}
\begin{tikzpicture}
  \begin{axis}[
    width=0.8\textwidth,
    height=0.5\textwidth,
    xlabel={$x$},
    title={Densité Uniforme ($a=0$, $b=2$)},
    grid=both,
    grid style={line width=.1pt, draw=gray!30},
    major grid style={line width=.2pt,draw=gray!50},
    domain=-0.5:2.5,
    samples=100,
    enlargelimits=false,
    axis lines=middle,
    ymin=0,
    ymax=.75,
    xmin=-1, xmax=4
  ]
 
  % Courbe de densité uniforme
  \addplot [thick, color=blue, fill=blue!20, fill opacity=0.5, const plot] 
    coordinates {(-0.5,0) (0,0) (0,0.5) (2,0.5) (2,0) (2.5,0)};
 
  % Ligne de la moyenne
  \addplot [red, dashed, thick] coordinates {(1,0) (1,0.5)};
  \node[red, fill=white, font=\scriptsize, rounded corners, inner sep=2pt, opacity=0.8] at (axis cs:1.3,0.4) {Moyenne};
 
  % Boîte de texte avec moments seulement
  \node [draw=black, fill=white, rounded corners, font=\scriptsize, align=left, anchor=north east, opacity=0.8] 
    at (axis description cs:0.95,0.95) { % MODIFIÉ
    \textbf{Moments :}\\
    • Moyenne = 1.00\\
    • Variance = 0.33\\
    • Skewness = 0.00\\
    • Kurtosis = -1.20
    };
    
  \end{axis}
\end{tikzpicture}
\end{center}

La distribution uniforme, c'est le "tirage au sort parfait" : chaque valeur sur un intervalle a la même chance d'être tirée. Visuellement, c'est un rectangle, donc aucune valeur n'est privilégiée. Elle est symétrique (\textbf{skewness = 0.00}), mais contrairement à la normale, elle est "plate", sans pic central. Cela se traduit par un \textbf{kurtosis négatif (-1.20)}, ce qui signifie qu'elle est \textbf{platykurtique}. Elle est donc très différente des distributions avec un pic central comme la normale.

\end{examplebox}

\begin{examplebox}[La Distribution Log-Normale (Fortement Leptokurtique)]

\begin{center}
\begin{tikzpicture}
  \begin{axis}[
    width=0.8\textwidth,
    height=0.5\textwidth,
    xlabel={$x$},
    title={Densité Log-Normale ($\sigma=0.7$)},
    grid=both,
    grid style={line width=.1pt, draw=gray!30},
    major grid style={line width=.2pt,draw=gray!50},
    domain=0:6,
    samples=100,
    enlargelimits=false,
    axis lines=middle,
    xmin=0, xmax=6,
    ymin=0, ymax=1
  ]
 
  % Courbe de densité log-normale
  \addplot [thick, color=blue, fill=blue!20, fill opacity=0.5] 
    {1/(x*0.7*sqrt(2*pi))*exp(-(ln(x))^2/(2*0.7^2))};
 
  % Ligne de la moyenne
  \addplot [red, dashed, thick] coordinates {(1.28,0) (1.28,0.4)};
  \node[red, fill=white, font=\scriptsize, rounded corners, inner sep=2pt, opacity=0.8] at (axis cs:1.8,0.5) {Moyenne};
 
  % Boîte de texte avec moments seulement
  \node [draw=black, fill=white, rounded corners, font=\scriptsize, align=left, anchor=north east, opacity=0.8] 
    at (axis description cs:0.95,0.95) { % MODIFIÉ
    \textbf{Moments :}\\
    • Moyenne = 1.28\\
    • Variance = 1.03\\
    • Skewness = 2.89\\
    • Kurtosis = 20.78
    };
    
  \end{axis}
\end{tikzpicture}
\end{center}

La log-normale est une distribution très asymétrique. Imaginez la richesse d'une population : la majorité est modeste, mais il existe une petite proportion de très riches, ce qui "étire" la droite de la courbe. Cela donne un \textbf{skewness très élevé (2.89)}. Elle est extrêmement \textbf{leptokurtique} (\textbf{kurtosis = 20.78}) : un pic très aigu et une queue droite très lourde. Cela signifie qu'il y a un risque élevé de valeurs extrêmement grandes, ce qui la rend très utile pour modéliser des phénomènes avec de rares événements extrêmes.

\end{examplebox}

Nous avons défini les moments d'une \textit{distribution} (moments de population), tels que $\mu = E[X]$ ou $\sigma^2 = E[(X-\mu)^2]$. Ce sont des valeurs théoriques, la "vérité" sous-jacente.

En pratique, nous ne connaissons presque jamais cette "vérité". Nous ne disposons que de données. Notre but est d'utiliser ces données pour \textit{estimer} les moments de la population.

\subsection{Moments d'échantillon (Sample Moments)}

\begin{definitionbox}[Moments d'Échantillon]
Soit $X_1, X_2, \dots, X_n$ un échantillon de $n$ observations.
\begin{itemize}
    \item La \textbf{moyenne d'échantillon} (notre "meilleure estimation" de $\mu$) est :
    $$ \bar{X} = \frac{1}{n} \sum_{i=1}^n X_i $$
    \item La \textbf{variance d'échantillon (non biaisée)} (notre "meilleure estimation" de $\sigma^2$) est :
    $$ s^2 = \frac{1}{n-1} \sum_{i=1}^n (X_i - \bar{X})^2 $$
\end{itemize}
De même, on peut calculer un \textit{skewness d'échantillon} et un \textit{kurtosis d'échantillon} en utilisant $\bar{X}$ et $s$, qui seront nos estimations du vrai skewness et du vrai kurtosis de la population.
\end{definitionbox}

\begin{examplebox}[Application : Contrôle Qualité ]
Imaginez une usine qui produit des sacs de sucre de 1kg.
\begin{itemize}
    \item \textbf{Population :} L'infinité de tous les sacs de sucre que la machine produira.
    \item \textbf{Moment de population (inconnu) :} Le poids moyen \textit{réel} $\mu$ que la machine verse, et la variance \textit{réelle} $\sigma^2$ (sa constance).
    \item \textbf{Problème :} Nous ne pouvons pas peser tous les sacs !
    \item \textbf{Solution :} Nous prélevons un \textbf{échantillon} de $n=10$ sacs.
    
    Nous les pesons : $\{ 1002g, 998g, 1001g, 995g, 1003g, 1000g, 997g, 1005g, 999g, 1000g \}$.
    
    \item \textbf{Calcul des moments d'échantillon :}
    \begin{itemize}
        \item $\bar{X} = (1002 + 998 + \dots + 1000) / 10 = 1000g$.
        \item $s^2 = \frac{1}{10-1} \left( (1002-1000)^2 + (998-1000)^2 + \dots \right) = 7.33 g^2$.
    \end{itemize}
    \item \textbf{Conclusion :} Notre meilleure estimation est que la machine est bien réglée sur $\mu = 1000g$. L'écart-type de notre échantillon est $s = \sqrt{7.33} \approx 2.7g$. Nous pouvons utiliser cela pour affirmer, par exemple, que 95\% des sacs se situent probablement entre $1000 \pm 2s$ (si la distribution est normale).
\end{itemize}
\end{examplebox}

\begin{remarquebox}[L'Intuition du "$n-1$"]
Pourquoi diviser par $n-1$ pour la variance ? C'est la \textbf{correction de Bessel}.

Imaginez un échantillon de 1 seule personne ($n=1$). Sa taille est 170cm.
\begin{itemize}
    \item Quelle est la moyenne de l'échantillon ? $\bar{X} = 170$ cm.
    \item Quelle est la variance de l'échantillon ? $\sum (X_i - \bar{X})^2 = (170 - 170)^2 = 0$.
    \item Si on divisait par $n=1$, on estimerait que la variance de la population est 0. C'est absurde ! Cela voudrait dire que tout le monde mesure 170cm.
\end{itemize}
En divisant par $n-1$ (donc $1-1=0$), la formule devient $0/0$ (indéfinie), ce qui nous dit à juste titre : "Je ne peux pas estimer la dispersion avec une seule personne."

\textbf{Intuition plus générale :} Nous "perdons un degré de liberté". Pour calculer la variance, nous avons besoin de connaître la moyenne. Mais nous ne connaissons pas la vraie moyenne $\mu$. Nous devons donc utiliser $\bar{X}$, une \textit{estimation}. Le fait d'utiliser une estimation calculée \textit{à partir de ce même échantillon} introduit un léger biais (nos données sont, par définition, centrées sur $\bar{X}$). Diviser par $n-1$ au lieu de $n$ "gonfle" légèrement le résultat pour compenser ce biais.
\end{remarquebox}

\subsection{Fonctions génératrices des moments (MGF)}

\begin{definitionbox}[Fonction Génératrice des Moments (MGF)]
La \textbf{fonction génératrice des moments} (MGF) d'une variable aléatoire $X$, notée $M_X(t)$, est définie comme :
$$ M_X(t) = E[e^{tX}] $$
\end{definitionbox}

\begin{intuitionbox}[L'ADN, le Code-Barres, ou le Fichier .zip]
Ce concept est abstrait, alors utilisons des analogies :

\textbf{Analogie 1 : L'ADN ou l'Empreinte Digitale}
\begin{itemize}
    \item La MGF est l'**empreinte digitale unique** d'une distribution.
    \item Elle "compresse" \textit{toutes} les informations sur votre distribution (moyenne, variance, skewness, kurtosis, etc.) en une seule, unique fonction.
    \item Si deux distributions ont la même MGF, elles sont identiques. C'est la \textbf{propriété d'unicité}.
\end{itemize}

\textbf{Analogie 2 : Le Code-Barres}
\begin{itemize}
    \item Pensez à une distribution (ex: Loi Normale) comme à un produit au supermarché.
    \item La MGF, $M_X(t)$, est son **code-barres unique**.
    \item Le processus de "génération de moments" (que nous verrons ci-dessous) est le \textbf{scanner}.
    \item En scannant le code-barres ($M_X(t)$), vous pouvez obtenir n'importe quelle information :
        \item Scan 1 ($M_X'(0)$) $\to$ vous donne le prix ($E[X]$).
        \item Scan 2 ($M_X''(0)$) $\to$ vous donne le poids ($E[X^2]$).
        \item Scan 3 ($M_X'''(0)$) $\to$ vous donne le pays d'origine ($E[X^3]$).
\end{itemize}

\textbf{Pourquoi $e^{tX}$ ?}
La "magie" vient du développement en série de Taylor de $e^x$:
$$ e^{tX} = 1 + (tX) + \frac{(tX)^2}{2!} + \frac{(tX)^3}{3!} + \dots $$
Quand on prend l'espérance, $E[\cdot]$, les puissances de $X$ (c'est-à-dire $X, X^2, X^3\dots$) apparaissent. Ce sont les moments ! La MGF "stocke" tous ces moments en les organisant comme coefficients d'un polynôme infini en $t$.
\end{intuitionbox}

\subsection{Génération des moments via les MGF}

\begin{theorembox}[Moments par Dérivation]
Si la MGF $M_X(t)$ existe, alors le $m$-ième moment non centré $E[X^m]$ est la $m$-ième dérivée de $M_X(t)$, évaluée en $t=0$ :
$$ E[X^m] = \frac{d^m}{dt^m} M_X(t) \bigg|_{t=0} = M_X^{(m)}(0) $$
\end{theorembox}

\begin{examplebox}[Application : La Loi de Poisson]
Une loi de Poisson modélise le nombre d'événements (ex: appels à un centre d'appels) par heure. Soit $X \sim \text{Poisson}(\lambda)$, où $\lambda$ est le nombre moyen d'appels.

La MGF (l'ADN) d'une loi de Poisson est (on l'admet) :
$$ M_X(t) = e^{\lambda(e^t - 1)} $$

Utilisons notre "scanner" (les dérivées) pour trouver les moments.

\textbf{1. Trouver la Moyenne $E[X]$ :}
On dérive une fois (règle de la chaîne) :
$$ M_X'(t) = \frac{d}{dt} \left( e^{\lambda(e^t - 1)} \right) = \underbrace{e^{\lambda(e^t - 1)}}_{\text{répète}} \cdot \underbrace{(\lambda e^t)}_{\text{dérivée interne}} $$
Maintenant, on évalue en $t=0$ :
$$ E[X] = M_X'(0) = e^{\lambda(e^0 - 1)} \cdot (\lambda e^0) = e^{\lambda(1 - 1)} \cdot (\lambda \cdot 1) = e^0 \cdot \lambda = 1 \cdot \lambda = \lambda $$
\textbf{Résultat :} La moyenne est $\lambda$, ce qui est la définition même du paramètre de la loi de Poisson. Parfait.

\textbf{2. Trouver $E[X^2]$ (pour la variance) :}
On dérive une seconde fois (règle du produit sur $M_X'(t) = (\lambda e^t) \cdot (e^{\lambda(e^t - 1)})$) :
$$ M_X''(t) = \underbrace{(\lambda e^t)}_{\text{dérivée de u}} \cdot \underbrace{(e^{\lambda(e^t - 1)})}_{\text{v}} + \underbrace{(\lambda e^t)}_{\text{u}} \cdot \underbrace{(e^{\lambda(e^t - 1)} \cdot \lambda e^t)}_{\text{dérivée de v}} $$
Maintenant, on évalue en $t=0$ (tous les $e^0$ deviennent 1) :
$$ E[X^2] = M_X''(0) = (\lambda \cdot 1) \cdot (e^{\lambda(1-1)}) + (\lambda \cdot 1) \cdot (e^{\lambda(1-1)} \cdot \lambda \cdot 1) $$
$$ E[X^2] = (\lambda) \cdot (e^0) + (\lambda) \cdot (e^0 \cdot \lambda) = \lambda \cdot 1 + \lambda \cdot (1 \cdot \lambda) = \lambda + \lambda^2 $$

\textbf{3. Trouver la Variance $\text{Var}(X)$ :}
$\text{Var}(X) = E[X^2] - (E[X])^2 = (\lambda + \lambda^2) - (\lambda)^2 = \lambda$
\textbf{Résultat :} Nous avons prouvé par les MGF que pour une loi de Poisson, $\text{Moyenne} = \text{Variance} = \lambda$. C'est une propriété fondamentale de cette loi.
\end{examplebox}

\subsection{Sommes de variables aléatoires indépendantes via les MGF}

C'est la super-puissance des MGF.

\begin{theorembox}[MGF d'une Somme]
Soient $X$ et $Y$ deux variables aléatoires \textbf{indépendantes}. Soit $S = X + Y$. Alors la MGF de $S$ est le produit des MGF individuelles :
$$ M_S(t) = M_{X+Y}(t) = M_X(t) \cdot M_Y(t) $$
\end{theorembox}

\begin{intuitionbox}[La Magie de l'Exponentielle]
Pourquoi est-ce vrai ? $M_{X+Y}(t) = E[e^{t(X+Y)}] = E[e^{tX} \cdot e^{tY}]$.
Parce que $X$ et $Y$ sont indépendantes, $E[f(X)g(Y)] = E[f(X)]E[g(Y)]$.
Donc, $E[e^{tX} \cdot e^{tY}] = E[e^{tX}] \cdot E[e^{tY}] = M_X(t) \cdot M_Y(t)$.

Les MGF transforment une opération analytiquement horrible (la "convolution" de densités) en une simple multiplication algébrique.
\end{intuitionbox}

\begin{examplebox}[Application : Portefeuille d'Actifs ou Tailles Humaines]
C'est l'un des théorèmes les plus importants des statistiques.
\textbf{Problème :} Soit $X$ la taille d'un homme, $X \sim N(\mu_X, \sigma_X^2)$. Soit $Y$ la taille d'une femme, $Y \sim N(\mu_Y, \sigma_Y^2)$. Si on les choisit au hasard, quelle est la loi de la somme de leurs tailles $S = X+Y$ ?

\begin{enumerate}
    \item \textbf{ADN de $X$} : La MGF d'une loi Normale $N(\mu, \sigma^2)$ est $M(t) = \exp(\mu t + \frac{1}{2}\sigma^2 t^2)$.
    \item \textbf{ADN de $X$ et $Y$} :
    $M_X(t) = \exp(\mu_X t + \frac{1}{2}\sigma_X^2 t^2)$
    $M_Y(t) = \exp(\mu_Y t + \frac{1}{2}\sigma_Y^2 t^2)$
    
    \item \textbf{ADN de $S = X+Y$} (on multiplie) :
    $M_S(t) = M_X(t) \cdot M_Y(t) = \exp(\mu_X t + \frac{1}{2}\sigma_X^2 t^2) \cdot \exp(\mu_Y t + \frac{1}{2}\sigma_Y^2 t^2)$
    
    \item \textbf{Simplification} (en additionnant les exposants) :
    $M_S(t) = \exp\left( (\mu_X t + \mu_Y t) + (\frac{1}{2}\sigma_X^2 t^2 + \frac{1}{2}\sigma_Y^2 t^2) \right)$
    $M_S(t) = \exp\left( (\mu_X + \mu_Y)t + \frac{1}{2}(\sigma_X^2 + \sigma_Y^2)t^2 \right)$
    
    \item \textbf{Conclusion (par Unicité)} :
    Regardez cet ADN ! C'est l'ADN d'une loi Normale !
    Le nouveau $\mu$ est $(\mu_X + \mu_Y)$.
    La nouvelle $\sigma^2$ est $(\sigma_X^2 + \sigma_Y^2)$.
\end{enumerate}

\textbf{Résultat :} Nous avons prouvé que \textbf{la somme de deux Normales indépendantes est une nouvelle Normale}.
Si $X \sim N(175cm, 7^2)$ et $Y \sim N(165cm, 6^2)$, alors $S \sim N(340cm, 7^2 + 6^2 = 85)$.
Notez que les écarts-types \textit{ne s'additionnent pas} ($\sqrt{85} \approx 9.2 \ne 7+6$). Ce sont les variances qui s'additionnent.
\end{examplebox}
\newpage

\section{Les Lois des Grands Nombres (LLN)}

Dans la section précédente, nous avons fait une distinction cruciale entre les \textbf{moments de population} (les "vraies" valeurs théoriques, inconnues, comme $\mu$ et $\sigma^2$) et les \textbf{moments d'échantillon} (nos estimations calculées à partir des données, comme $\bar{X}$ et $s^2$).

Par exemple, nous avons défini la moyenne d'échantillon $\bar{X} = \frac{1}{n} \sum X_i$ comme notre "meilleure estimation" de la moyenne de population $\mu$. Mais qu'est-ce qui nous garantit que cette estimation est "bonne" ? Qu'est-ce qui nous assure que si nous collections plus de données (en augmentant $n$), notre $\bar{X}$ se rapprocherait de $\mu$ ?

La réponse à cette question fondamentale est fournie par les \textbf{Lois des Grands Nombres (LLN)}. Elles forment le pont théorique entre les probabilités (la théorie) et les statistiques (la pratique).

\begin{intuitionbox}[L'Idée Fondamentale : L'Exemple du Dé]
Supposons que nous voulons connaître la valeur moyenne d'un lancer de dé équilibré.
\begin{itemize}
    \item \textbf{Moment de Population :} Nous savons par la théorie que $\mu = E[X] = \frac{1+2+3+4+5+6}{6} = 3.5$.
    
    \item \textbf{Moments d'Échantillon :} Nous n'avons pas cette information, alors nous lançons le dé.
    \begin{itemize}
        \item $n=2$ lancers : On obtient (2, 6). $\bar{X}_2 = (2+6)/2 = 4.0$. (Assez loin de 3.5)
        \item $n=10$ lancers : On obtient (1, 6, 3, 3, 5, 2, 4, 1, 6, 4). $\bar{X}_{10} = 3.5$. (Pile dessus !)
        \item $n=100$ lancers : On obtiendra $\bar{X}_{100} \approx 3.48$ (par exemple).
        \item $n=1,000,000$ lancers : On obtiendra $\bar{X}_{1,000,000} \approx 3.5001$ (par exemple).
    \end{itemize}
\end{itemize}
La Loi des Grands Nombres formalise cette intuition : à mesure que $n \to \infty$, la moyenne de notre échantillon $\bar{X}_n$ \textbf{converge} vers la vraie moyenne $\mu$.

La distinction entre les lois "Faible" et "Forte" réside dans la \textit{manière} dont nous définissons cette convergence.
\end{intuitionbox}


\subsection{L'Inégalité de Chebyshev}

Avant de prouver la Loi Faible, nous avons besoin d'un outil fondamental qui relie la variance d'une variable à la probabilité qu'elle s'éloigne de sa moyenne. C'est l'Inégalité de Chebyshev.

Sa puissance réside dans son universalité : elle s'applique à \textit{n'importe quelle} distribution, à condition qu'elle ait une moyenne et une variance finies.

\begin{theorembox}[Inégalité de Chebyshev]
Soit $Y$ une variable aléatoire avec une espérance finie $\mu = E[Y]$ et une variance finie $\sigma^2 = \text{Var}(Y)$.

Alors, pour tout nombre réel $k > 0$ :
$$ P(|Y - \mu| \ge k) \le \frac{\text{Var}(Y)}{k^2} = \frac{\sigma^2}{k^2} $$
\end{theorembox}

% --- DÉBUT DE L'AJOUT DE LA PREUVE ---
\begin{proofbox}[Preuve de l'Inégalité de Chebyshev]
Nous présentons la preuve pour une variable aléatoire continue $Y$ de densité $f(y)$. La preuve pour le cas discret est similaire en remplaçant les intégrales par des sommes.

\begin{enumerate}
    \item Par définition, la variance $\sigma^2$ est $E[(Y - \mu)^2]$ :
    $$ \sigma^2 = E[(Y - \mu)^2] = \int_{-\infty}^{\infty} (y - \mu)^2 f(y) dy $$
    
    \item Nous pouvons scinder cette intégrale en deux parties : la région où $Y$ est proche de $\mu$ ($|y - \mu| < k$) et la région où $Y$ est loin de $\mu$ ($|y - \mu| \ge k$) :
    $$ \sigma^2 = \int_{|y - \mu| < k} (y - \mu)^2 f(y) dy + \int_{|y - \mu| \ge k} (y - \mu)^2 f(y) dy $$
    
    \item L'intégrande $(y - \mu)^2 f(y)$ est toujours non-négative (un carré fois une densité). Par conséquent, la première intégrale est $\ge 0$. En la supprimant, nous ne pouvons que diminuer la valeur totale :
    $$ \sigma^2 \ge \int_{|y - \mu| \ge k} (y - \mu)^2 f(y) dy $$
    
    \item Maintenant, concentrons-nous sur la région d'intégration : $|y - \mu| \ge k$. Dans cette région, par définition, nous avons $(y - \mu)^2 \ge k^2$.
    
    \item Nous pouvons remplacer $(y - \mu)^2$ par $k^2$ dans l'intégrale. Puisque nous remplaçons un terme par quelque chose de plus petit ou égal, la valeur de l'intégrale diminue (ou reste égale) :
    $$ \sigma^2 \ge \int_{|y - \mu| \ge k} k^2 f(y) dy $$
    
    \item $k^2$ est une constante, nous pouvons la sortir de l'intégrale :
    $$ \sigma^2 \ge k^2 \int_{|y - \mu| \ge k} f(y) dy $$
    
    \item Par définition, l'intégrale de la densité $f(y)$ sur la région $|y - \mu| \ge k$ n'est autre que la probabilité $P(|Y - \mu| \ge k)$.
    $$ \sigma^2 \ge k^2 \cdot P(|Y - \mu| \ge k) $$
    
    \item En réarrangeant les termes (puisque $k > 0$, $k^2 > 0$), nous obtenons l'inégalité désirée :
    $$ P(|Y - \mu| \ge k) \le \frac{\sigma^2}{k^2} $$
\end{enumerate}
Cette preuve est un cas particulier de l'Inégalité de Markov (appliquée à la variable aléatoire non-négative $X = (Y-\mu)^2$ et à la constante $a = k^2$).
\end{proofbox}
% --- FIN DE L'AJOUT DE LA PREUVE ---

\begin{intuitionbox}[Comprendre l'Inégalité de Chebyshev]
Cette formule peut être lue comme suit :

\textbf{"La probabilité de s'écarter de la moyenne ($\mu$) d'au moins $k$ est bornée par la variance divisée par $k^2$."}

\begin{itemize}
    \item \textbf{Le rôle de la variance ($\sigma^2$) :} Si la variance est grande, la borne supérieure est élevée. L'inégalité nous dit "il est possible que la variable s'éloigne", ce qui est logique pour une grande dispersion. Si la variance est faible, la borne est basse, ce qui force la probabilité d'être loin à être faible.
    \item \textbf{Le rôle de l'écart ($k$) :} Le terme $k^2$ au dénominateur est crucial. Il signifie que la probabilité de s'écarter de la moyenne diminue \textit{quadratiquement} avec la distance $k$. Être très loin est (relativement) très improbable.
\end{itemize}
\end{intuitionbox}

\begin{examplebox}[Une Borne Universelle]
Exprimons l'inégalité en termes d'écarts-types (en posant $k = c \cdot \sigma$) :
$$ P(|Y - \mu| \ge c\sigma) \le \frac{\sigma^2}{(c\sigma)^2} = \frac{1}{c^2} $$

\begin{itemize}
    \item \textbf{Pour $c=2$ :} $P(|Y - \mu| \ge 2\sigma) \le \frac{1}{4} = 25\%$.
    Peu importe la distribution (symétrique, asymétrique, bizarre...), la probabilité d'être à 2 écarts-types ou plus de la moyenne est \textbf{au maximum} de 25\%. (Pour une loi normale, cette probabilité est bien plus faible, $\approx 4.55\%$).
    
    \item \textbf{Pour $c=3$ :} $P(|Y - \mu| \ge 3\sigma) \le \frac{1}{9} \approx 11.1\%$.
    La probabilité d'être à 3 écarts-types ou plus est au maximum de 11.1\%. (Pour une loi normale, c'est $\approx 0.27\%$).
\end{itemize}
Chebyshev fournit une borne "garantie", bien que souvent non optimale. Elle est l'outil parfait pour la preuve qui suit.
\end{examplebox}


\subsection{La Loi Faible des Grands Nombres (LFGN / WLLN)}

La loi faible stipule que la probabilité que notre moyenne d'échantillon s'écarte de la vraie moyenne de plus qu'une petite quantité $\epsilon$ tend vers zéro. C'est une \textbf{convergence en probabilité}.

\begin{definitionbox}[Convergence en Probabilité]
On dit qu'une suite de variables aléatoires $Y_n$ converge en probabilité vers une constante $c$, noté $Y_n \xrightarrow{P} c$, si pour tout $\epsilon > 0$ (aussi petit soit-il) :
$$\lim_{n \to \infty} P(|Y_n - c| > \epsilon) = 0$$
\end{definitionbox}

\begin{intuitionbox}[Comprendre la Convergence en Probabilité]
La définition $P(|\bar{X}_n - \mu| > \epsilon) \to 0$ signifie :
\begin{itemize}
    \item $\epsilon$ est votre \textbf{marge d'erreur} acceptable (ex: 0.01).
    \item $|\bar{X}_n - \mu|$ est l'erreur réelle de votre estimation.
    \item $P(\dots)$ est la probabilité que votre erreur \textbf{dépasse} votre marge.
    \item $\lim_{n \to \infty} (\dots) = 0$ signifie : "Si vous prenez un échantillon $n$ suffisamment grand, la probabilité de faire une erreur plus grande que $\epsilon$ devient négligeable."
\end{itemize}
C'est une affirmation sur ce qui se passe pour un $n$ fixe et très grand.
\end{intuitionbox}

\begin{theorembox}[Loi Faible des Grands Nombres (Khinchine)]
Soit $X_1, X_2, \dots, X_n$ une suite de variables aléatoires \textbf{i.i.d.} (indépendantes et identiquement distribuées) avec une espérance finie $E[X_i] = \mu$.
Soit $\bar{X}_n = \frac{1}{n} \sum_{i=1}^n X_i$ la moyenne d'échantillon.

Alors, $\bar{X}_n$ converge en probabilité vers $\mu$ :
$$\bar{X}_n \xrightarrow{P} \mu$$
\end{theorembox}

\begin{proofbox}[Preuve (simplifiée) via l'Inégalité de Chebyshev]
La loi faible de Khinchine ne nécessite qu'une moyenne finie. Cependant, si nous ajoutons la condition que la \textbf{variance $\sigma^2$ est aussi finie}, la preuve devient très simple.

Elle repose directement sur l'Inégalité de Chebyshev, que nous venons de voir. Nous l'appliquons à la variable aléatoire $Y = \bar{X}_n$.

\begin{enumerate}
    \item Identifions les termes pour l'inégalité $P(|Y - E[Y]| \ge k) \le \frac{\text{Var}(Y)}{k^2}$:
    \begin{itemize}
        \item Notre variable est $Y = \bar{X}_n$.
        \item Son espérance est $E[Y] = E[\bar{X}_n] = \mu$.
        \item Sa variance est $\text{Var}(Y) = \text{Var}(\bar{X}_n) = \frac{\sigma^2}{n}$.
        \item Notre écart $k$ est la marge d'erreur $\epsilon$.
    \end{itemize}

    \item (Rappel du calcul de la variance de $\bar{X}_n$) :
    Puisque les $X_i$ sont i.i.d., $\text{Var}(\bar{X}_n) = \text{Var}\left(\frac{1}{n}\sum X_i\right) = \frac{1}{n^2}\sum \text{Var}(X_i) = \frac{1}{n^2}(n\sigma^2) = \frac{\sigma^2}{n}$.

    \item Appliquons l'inégalité de Chebyshev avec ces termes :
    $$ P(|\bar{X}_n - \mu| \ge \epsilon) \le \frac{\text{Var}(\bar{X}_n)}{\epsilon^2} = \frac{\sigma^2 / n}{\epsilon^2} = \frac{\sigma^2}{n \epsilon^2} $$
    
    \item Prenons maintenant la limite quand $n \to \infty$ :
    $$ \lim_{n \to \infty} P(|\bar{X}_n - \mu| \ge \epsilon) \le \lim_{n \to \infty} \frac{\sigma^2}{n \epsilon^2} $$
    
    \item Puisque $\sigma^2$ et $\epsilon^2$ sont des constantes finies, le terme de droite $\frac{\text{constante}}{n}$ tend vers 0.
    
    \item Comme une probabilité ne peut pas être négative, nous avons :
    $$ \lim_{n \to \infty} P(|\bar{X}_n - \mu| > \epsilon) = 0 $$
\end{enumerate}
C'est exactement la définition de la convergence en probabilité.
\end{proofbox}

\subsection{La Loi Forte des Grands Nombres (LFGN / SLLN)}

La loi forte est une affirmation beaucoup plus puissante. Elle ne dit pas seulement qu'un "gros" écart est improbable pour un $n$ "grand" ; elle dit que la probabilité que la suite $\bar{X}_n$ \textit{ne converge pas} vers $\mu$ est nulle. C'est une \textbf{convergence presque sûre}.

\begin{definitionbox}[Convergence Presque Sûre]
On dit qu'une suite de variables aléatoires $Y_n$ converge presque sûrement vers une constante $c$, noté $Y_n \xrightarrow{p.s.} c$, si :
$$P\left( \lim_{n \to \infty} Y_n = c \right) = 1$$
\end{definitionbox}

\begin{theorembox}[Loi Forte des Grands Nombres (Kolmogorov)]
Soit $X_1, X_2, \dots, X_n$ une suite de variables aléatoires \textbf{i.i.d.} avec une espérance finie $E[X_i] = \mu$.
Alors, $\bar{X}_n$ converge presque sûrement vers $\mu$ :
$$\bar{X}_n \xrightarrow{p.s.} \mu$$
\end{theorembox}

\begin{remarquebox}[Forte implique Faible]
La convergence "presque sûre" (SLLN) est une condition plus stricte que la convergence "en probabilité" (WLLN). Si une suite converge presque sûrement, elle converge aussi en probabilité. L'inverse n'est pas toujours vrai.
\end{remarquebox}

\subsection{Différence : Faible vs. Forte}

\begin{intuitionbox}[Faible vs. Forte : L'Analogie du Casino]
Soit $\bar{X}_n$ votre gain moyen par partie après avoir joué $n$ fois à la roulette. La vraie moyenne (l'avantage de la maison) est $\mu = -0.053$ (pour une roulette américaine).

\begin{itemize}
    \item \textbf{Loi Faible (WLLN) :} "Si vous prévoyez de jouer $n = 1 \text{ million}$ de parties ce soir. La probabilité qu'à la fin de votre millionième partie, votre moyenne $\bar{X}_{1,000,000}$ soit loin de $-0.053$ (par exemple, que vous soyez gagnant, $\bar{X}_n > 0$) est infinitésimale."
    \item C'est une affirmation sur la distribution de $\bar{X}_n$ \textbf{à un point fixe $n$ (très grand)}. Elle n'exclut pas la possibilité théorique (mais improbable) que si vous continuiez à jouer, votre moyenne $\bar{X}_n$ puisse à nouveau diverger follement avant de reconverger plus tard.

    \item \textbf{Loi Forte (SLLN) :} "Si vous jouez à la roulette \textit{pour l'éternité}, en regardant la séquence de vos moyennes $\bar{X}_1, \bar{X}_2, \bar{X}_3, \dots, \bar{X}_n, \dots$."
    \item "La probabilité que cette \textbf{séquence entière} ne converge pas exactement vers $\mu = -0.053$ est de 0."
    \item C'est une affirmation sur la \textbf{trajectoire complète}. Elle dit que, avec une probabilité de 1, la trajectoire de $\bar{X}_n$ va s'approcher de $\mu$ et \textbf{ne plus s'en écarter} de manière significative.
\end{itemize}

En résumé :
\begin{itemize}
    \item \textbf{Faible :} Pour $n$ assez grand, un écart est \textbf{improbable}.
    \item \textbf{Forte :} La \textbf{trajectoire} converge vers $\mu$ (avec une probabilité de 1).
\end{itemize}
\end{intuitionbox}

\subsection{Application : La Méthode de Monte-Carlo}

La Loi Forte des Grands Nombres est le moteur de l'une des techniques de calcul les plus puissantes : la simulation de Monte-Carlo. Elle nous permet d'estimer des quantités complexes (comme des intégrales) en utilisant le hasard.

\begin{examplebox}[Estimer la valeur de $\pi$]

\textbf{Problème :} Comment calculer $\pi$ sans formule géométrique ?

\textbf{Méthode (Statistique) :}
\begin{enumerate}
    \item Imaginez un carré de côté 1 (de $(0,0)$ à $(1,1)$). Son aire est $A_{\text{carré}} = 1$.
    \item Imaginez un quart de cercle de rayon $r=1$ inscrit dans ce carré. Son aire est $A_{\text{cercle}} = \frac{1}{4}\pi r^2 = \frac{\pi}{4}$.
    \item Le \textit{ratio} des aires est $\frac{A_{\text{cercle}}}{A_{\text{carré}}} = \frac{\pi / 4}{1} = \frac{\pi}{4}$.
\end{enumerate}

\textbf{Simulation :}
\begin{enumerate}
    \item Nous allons "lancer des fléchettes" au hasard sur ce carré $n$ fois.
    \item Pour ce faire, nous générons $n$ paires de nombres aléatoires $(X_i, Y_i)$, où $X_i \sim U(0, 1)$ et $Y_i \sim U(0, 1)$.
    \item Pour chaque point $i$, nous vérifions s'il a atterri \textbf{dans le cercle}. La condition est $X_i^2 + Y_i^2 \le 1$.
    \item Nous définissons une nouvelle variable aléatoire $Z_i$ (de Bernoulli) :
$$ Z_i = \begin{cases} 1 & \text{si } X_i^2 + Y_i^2 \le 1 \quad \text{(le point est dans le cercle)} \\ 0 & \text{sinon} \end{cases} $$
\end{enumerate}

\textbf{Application de la LLN :}
\begin{itemize}
    \item Quelle est la "vraie moyenne" $\mu$ de cette variable $Z_i$ ?
    \item $\mu = E[Z_i] = 1 \cdot P(Z_i=1) + 0 \cdot P(Z_i=0) = P(Z_i=1)$.
    \item $P(Z_i=1)$ est la probabilité qu'un point aléatoire tombe dans le cercle. Puisque les points sont uniformes, cette probabilité est simplement le ratio des aires !
    \item Donc, la vraie moyenne (inconnue) est $\mu = \frac{A_{\text{cercle}}}{A_{\text{carré}}} = \frac{\pi}{4}$.
    
    \item Comment estimer $\mu$ ? Nous utilisons la moyenne d'échantillon $\bar{Z}_n$ :
    $$\bar{Z}_n = \frac{1}{n} \sum_{i=1}^n Z_i = \frac{\text{Nombre de points dans le cercle}}{n}$$
    
    \item Par la \textbf{Loi Forte des Grands Nombres}, nous avons la garantie que :
    $$\bar{Z}_n \xrightarrow{p.s.} \mu = \frac{\pi}{4}$$
\end{itemize}

\textbf{Conclusion :}
Pour estimer $\pi$, il suffit de calculer $\bar{Z}_n$ (une simple proportion) et de la multiplier par 4.
$$\pi \approx 4 \cdot \bar{Z}_n$$
Plus notre nombre de simulations $n$ est grand, plus la SLLN nous garantit que notre estimation sera proche de la vraie valeur de $\pi$.
\end{examplebox}
\newpage

\section{Le Théorème Central Limite (TCL)}

\subsection{Introduction : L'omniprésence de la loi normale}

Dans la section précédente, la Loi des Grands Nombres (LLN) nous a donné une garantie fondamentale : la moyenne d'échantillon $\bar{X}_n$ converge vers la vraie moyenne $\mu$ lorsque $n$ devient grand.
$$ \bar{X}_n \xrightarrow{p.s.} \mu $$
La LLN nous dit \textbf{où} la moyenne d'échantillon converge (vers la constante $\mu$), mais elle ne nous dit rien sur la \textit{forme} de la distribution de $\bar{X}_n$ autour de $\mu$ pour un $n$ grand, mais fini.

Le \textbf{Théorème Central Limite (TCL)} comble cette lacune. Il décrit la \textit{manière} dont $\bar{X}_n$ converge, en nous donnant la forme de sa distribution. C'est sans doute le théorème le plus important des statistiques.

\begin{intuitionbox}[L'Idée Fondamentale]
Intuitivement, ce résultat affirme qu'une \textbf{somme} d'un grand nombre de variables aléatoires indépendantes et identiquement distribuées (i.i.d.) tend, le plus souvent, à suivre une \textbf{loi normale} (aussi appelée loi de Laplace-Gauss ou "courbe en cloche").

Ce théorème et ses généralisations offrent une explication à l'omniprésence de la loi normale dans la nature. De nombreux phénomènes (la taille d'un individu, l'erreur de mesure d'un instrument, le bruit de fond d'un signal) sont le résultat de l'addition d'un très grand nombre de petites perturbations aléatoires. Le TCL nous dit que le résultat de cette somme sera, inévitablement, distribué selon une loi normale.
\end{intuitionbox}

\subsection{L'illustration : la somme des "Pile ou Face"}

Prenons l'exemple le plus simple pour illustrer ce phénomène : le jeu de "pile ou face".



\begin{examplebox}[Distribution de la Somme de $n$ Lancers]
Soit $X_i$ le résultat du $i$-ème lancer, avec $X_i = 1$ pour "Face" (probabilité 0,5) et $X_i = 0$ pour "Pile" (probabilité 0,5). La distribution d'origine (pour $n=1$) n'est pas du tout une courbe en cloche : c'est une distribution discrète avec deux bâtons de même hauteur.

Considérons la \textbf{somme} $S_n = X_1 + X_2 + \dots + X_n$, qui représente le nombre total de "Face" obtenus en $n$ lancers.

\begin{itemize}
    \item \textbf{Pour $n=1$ :} La distribution de $S_1$ est :
    \begin{itemize}
        \item Valeurs de la somme : \{0, 1\}
        \item Fréquences : \{0.5, 0.5\}
    \end{itemize}
    
    \item \textbf{Pour $n=2$ :} Les sommes possibles sont \{0, 1, 2\}. La distribution de $S_2$ est :
    \begin{itemize}
        \item Valeurs de la somme : \{0, 1, 2\}
        \item Fréquences : \{0.25, 0.5, 0.25\} (elle forme un triangle).
    \end{itemize}
    
    \item \textbf{Pour $n=3$ :} Les sommes possibles sont \{0, 1, 2, 3\}. La distribution de $S_3$ est :
    \begin{itemize}
        \item Valeurs de la somme : \{0, 1, 2, 3\}
        \item Fréquences : \{0.125, 0.375, 0.375, 0.125\}
    \end{itemize}
\end{itemize}

\begin{center}
\begin{tikzpicture}
  \begin{axis}[
    width=0.8\textwidth,
    height=0.5\textwidth,
    xlabel={Valeurs de la somme},
    title={Fonction de fréquence pour des tirages à pile ou face},
    grid=both,
    grid style={line width=.1pt, draw=gray!30},
    major grid style={line width=.2pt,draw=gray!50},
    domain=0:4,
    samples=100,
    enlargelimits=false,
    axis lines=middle,
    xmin=0, xmax=4,
    ymin=0, ymax=0.6
  ]
  % n=1
  \addplot [thick, color=blue, fill=blue!20, fill opacity=0.5] coordinates {(0,0.5) (1,0.5)} \closedcycle;
  % n=2
  \addplot [thick, color=green, fill=green!20, fill opacity=0.5] coordinates {(0,0.25) (1,0.5) (2,0.25)} \closedcycle;
  % n=3
  \addplot [thick, color=red, fill=red!20, fill opacity=0.5] coordinates {(0,0.125) (1,0.375) (2,0.375) (3,0.125)} \closedcycle;
  % n=4 (suggéré)
  \addplot [thick, color=black, fill=black!20, fill opacity=0.5] coordinates {(0,0.0625) (1,0.25) (2,0.375) (3,0.25) (4,0.0625)} \closedcycle;
  
  \legend{$n=1$,$n=2$,$n=3$,$n=4$}
  \end{axis}
\end{tikzpicture}
\end{center}

Graphiquement, on constate que plus le nombre de tirages $n$ augmente (par exemple, jusqu'à $n=12$), plus la courbe de fréquence (qui reste discrète) se rapproche d'une courbe en cloche symétrique, caractéristique de la loi normale.
\end{examplebox}

\subsection{Distribution de la population vs. Distribution d'échantillonnage}

Le point le plus remarquable du TCL est qu'il fonctionne \textit{quelle que soit} la distribution de départ.

\begin{intuitionbox}[Population vs. Échantillonnage]
Imaginez deux univers de distributions :

\begin{itemize}
    \item \textbf{1. La Distribution de la Population ($X_i$) :} C'est la loi de nos variables $X_i$ individuelles. Elle peut avoir \textbf{n'importe quelle forme} (par exemple, une distribution bimodale, asymétrique, ou uniforme). Cette distribution a une "vraie" moyenne $\mu$ et un "vrai" écart-type $\sigma$.
    
    \item \textbf{2. La Distribution d'Échantillonnage ($\bar{X}_n$) :} C'est la distribution de la \textit{moyenne} $\bar{X}_n = (X_1 + \dots + X_n)/n$, calculée sur des échantillons de taille $n$. C'est la distribution de "toutes les moyennes d'échantillon possibles".
\end{itemize}

Le TCL énonce la relation magique entre les deux :

\textbf{Quelle que soit la forme de la distribution de la population, plus la taille de l'échantillon $n$ croît, plus la distribution d'échantillonnage de la moyenne $\bar{X}_n$ est proche d'une loi normale (gaussienne).}

De plus, les paramètres de cette loi normale sont :
\begin{itemize}
    \item \textbf{Moyenne :} La distribution de $\bar{X}_n$ est centrée sur la même moyenne $\mu$ que la population.
    \item \textbf{Écart-type :} La distribution de $\bar{X}_n$ est beaucoup plus resserrée. Son écart-type (appelé "erreur standard") est $\sigma_{\bar{X}} = \frac{\sigma}{\sqrt{n}}$.
\end{itemize}
Cette dispersion $\sigma/\sqrt{n}$ qui tend vers 0 est la manifestation de la Loi des Grands Nombres. Le TCL précise que la \textit{forme} de cette convergence est gaussienne.
\end{intuitionbox}

\subsection{Énoncé formel du Théorème Central Limite}

Pour énoncer le théorème formellement, nous devons d'abord définir les propriétés de la somme $S_n$ et de la moyenne $\bar{X}_n$.

Soit $X_1, \dots, X_n$ des variables aléatoires i.i.d. avec $E[X_i] = \mu$ et $\text{Var}(X_i) = \sigma^2$.

\begin{itemize}
    \item \textbf{La Somme $S_n = \sum X_i$} :
    \begin{itemize}
        \item Espérance : $E[S_n] = E[\sum X_i] = \sum E[X_i] = n\mu$
        \item Variance : $\text{Var}(S_n) = \text{Var}(\sum X_i) = \sum \text{Var}(X_i) = n\sigma^2$
        \item Écart-type : $\sigma_{S_n} = \sqrt{n\sigma^2} = \sigma\sqrt{n}$
    \end{itemize}
    
    \item \textbf{La Moyenne $\bar{X}_n = S_n / n$} :
    \begin{itemize}
        \item Espérance : $E[\bar{X}_n] = E[S_n / n] = \frac{1}{n} E[S_n] = \frac{1}{n} (n\mu) = \mu$
        \item Variance : $\text{Var}(\bar{X}_n) = \text{Var}(S_n / n) = \frac{1}{n^2} \text{Var}(S_n) = \frac{1}{n^2} (n\sigma^2) = \frac{\sigma^2}{n}$
        \item Écart-type : $\sigma_{\bar{X}_n} = \sqrt{\sigma^2 / n} = \frac{\sigma}{\sqrt{n}}$
    \end{itemize}
\end{itemize}

Nous voyons que la distribution de $S_n$ s'étale (variance $\to \infty$) tandis que celle de $\bar{X}_n$ se contracte (variance $\to 0$). Pour étudier la \textit{forme} de la convergence, nous créons une variable "stable" en la centrant (soustrayant la moyenne) et en la réduisant (divisant par l'écart-type). C'est la variable $Z_n$.

\begin{theorembox}[Théorème Central Limite (Lindeberg-Lévy)]
Soit $X_1, X_2, \dots, X_n$ une suite de variables aléatoires \textbf{i.i.d.} (indépendantes et identiquement distribuées) suivant la même loi $D$.
Supposons que l'\textbf{espérance $\mu$} et l'\textbf{écart-type $\sigma$} de cette loi $D$ existent, sont finis, et $\sigma \neq 0$.

Considérons la variable aléatoire standardisée $Z_n$ :
$$ Z_n = \frac{S_n - E[S_n]}{\sigma_{S_n}} = \frac{S_n - n\mu}{\sigma\sqrt{n}} $$
Cette variable est équivalente à la moyenne standardisée :
$$ Z_n = \frac{\bar{X}_n - E[\bar{X}_n]}{\sigma_{\bar{X}_n}} = \frac{\bar{X}_n - \mu}{\sigma / \sqrt{n}} $$
(Pour tout $n$, $Z_n$ est une variable centrée-réduite : $E[Z_n] = 0$ et $\text{Var}(Z_n) = 1$).

Alors, la suite de variables aléatoires $Z_1, Z_2, \dots, Z_n, \dots$ \textbf{converge en loi} vers une variable aléatoire $Z$ qui suit la \textbf{loi normale centrée réduite $N(0, 1)$}, lorsque $n$ tend vers l'infini.

Cela signifie que si $\Phi$ est la fonction de répartition de la loi $N(0, 1)$, alors pour tout réel $z$ :
$$ \lim_{n \to \infty} P(Z_n \le z) = \lim_{n \to \infty} P\left( \frac{\bar{X}_n - \mu}{\sigma/\sqrt{n}} \le z \right) = \Phi(z) $$
\end{theorembox}

\subsection{Applications Pratiques du TCL}

Le TCL n'est pas seulement une curiosité mathématique ; c'est le fondement de l'inférence statistique. Voici comment l'appliquer concrètement pour résoudre des problèmes.

\begin{examplebox}[La taille des individus]
\textbf{Contexte :} La taille des individus dans une population suit une courbe en cloche. Pourquoi ? Car elle est la \textbf{somme} de milliers de petites influences (gènes, nutrition, etc.). Le TCL s'applique.

\textbf{Données :} Supposons que dans une population, la taille $X$ des individus ait une espérance $\mu = 175$ cm et un écart-type $\sigma = 8$ cm. (Note : la loi de $X$ n'est pas forcément normale, même si en pratique elle l'est).

\textbf{Problème :} On prélève un échantillon aléatoire de $n=64$ individus. Quelle est la probabilité que la \textbf{moyenne de cet échantillon} ($\bar{X}_{64}$) soit supérieure à 177 cm ?

\textbf{Solution :}
\begin{enumerate}
    \item \textbf{Identifier les paramètres :}
    \begin{itemize}
        \item Moyenne de la population : $\mu = 175$ cm
        \item Écart-type de la population : $\sigma = 8$ cm
        \item Taille de l'échantillon : $n = 64$
    \end{itemize}
    
    \item \textbf{Appliquer le TCL :}
    Puisque $n=64$ est grand (généralement $n \ge 30$ est suffisant), le TCL s'applique. La distribution d'échantillonnage de la moyenne $\bar{X}_n$ suit approximativement une loi normale.
    $$ \bar{X}_n \approx N\left(\mu, \frac{\sigma^2}{n}\right) $$
    
    \item \textbf{Calculer les paramètres de la loi normale de $\bar{X}_n$ :}
    \begin{itemize}
        \item Espérance de $\bar{X}_n$ : $E[\bar{X}_n] = \mu = 175$ cm.
        \item Écart-type de $\bar{X}_n$ (appelé "Erreur Standard") :
        $$ \sigma_{\bar{X}_n} = \frac{\sigma}{\sqrt{n}} = \frac{8}{\sqrt{64}} = \frac{8}{8} = 1 \text{ cm} $$
    \end{itemize}
    Donc, $\bar{X}_{64} \approx N(175, 1^2)$.
    
    \item \textbf{Standardiser (Calculer le Z-score) :}
    Nous cherchons $P(\bar{X}_{64} > 177)$. Nous transformons cette valeur en un score $Z$ pour utiliser la loi normale centrée réduite $N(0, 1)$.
    $$ Z = \frac{\bar{X}_n - \mu}{\sigma_{\bar{X}_n}} = \frac{177 - 175}{1} = 2 $$
    
    \item \textbf{Trouver la probabilité :}
    Chercher $P(\bar{X}_{64} > 177)$ revient à chercher $P(Z > 2)$.
    En utilisant la table de la loi normale (ou une calculatrice) :
    $$ P(Z > 2) = 1 - P(Z \le 2) = 1 - \Phi(2) $$
    Sachant que $\Phi(2) \approx 0.9772$,
    $$ P(Z > 2) = 1 - 0.9772 = 0.0228 $$
\end{enumerate}
\textbf{Conclusion :} Il y a environ 2.28\% de chances qu'un échantillon de 64 personnes ait une taille moyenne supérieure à 177 cm.
\end{examplebox}

\begin{examplebox}[Remplissage de bouteilles]
\textbf{Contexte :} Une machine remplit des bouteilles de soda. Le volume versé $X_i$ fluctue légèrement. La loi de $X_i$ est inconnue.

\textbf{Données :} La machine est réglée pour verser en moyenne $\mu = 500$ ml. L'écart-type du processus est connu et vaut $\sigma = 6$ ml. Pour un contrôle, on prélève un échantillon de $n=36$ bouteilles.

\textbf{Problème :} On considère que la machine est déréglée si la moyenne de l'échantillon $\bar{X}_{36}$ est inférieure à 498 ml. Quelle est la probabilité d'une "fausse alarme" (c'est-à-dire, la machine fonctionne bien à $\mu=500$, mais l'échantillon a une moyenne $\bar{X}_{36} < 498$) ?

\textbf{Solution :}
\begin{enumerate}
    \item \textbf{Identifier les paramètres :}
    $\mu = 500$ ml, $\sigma = 6$ ml, $n = 36$.
    
    \item \textbf{Appliquer le TCL :}
    $n=36 \ge 30$, donc le TCL s'applique.
    $$ \bar{X}_{36} \approx N\left(\mu, \frac{\sigma^2}{n}\right) $$
    
    \item \textbf{Calculer les paramètres de $\bar{X}_{36}$ :}
    \begin{itemize}
        \item Espérance : $E[\bar{X}_{36}] = \mu = 500$ ml.
        \item Erreur Standard : $\sigma_{\bar{X}} = \frac{\sigma}{\sqrt{n}} = \frac{6}{\sqrt{36}} = \frac{6}{6} = 1$ ml.
    \end{itemize}
    Donc, $\bar{X}_{36} \approx N(500, 1^2)$.
    
    \item \textbf{Standardiser (Calculer le Z-score) :}
    Nous cherchons la probabilité $P(\bar{X}_{36} < 498)$.
    $$ Z = \frac{\bar{X}_n - \mu}{\sigma_{\bar{X}_n}} = \frac{498 - 500}{1} = -2 $$
    
    \item \textbf{Trouver la probabilité :}
    Chercher $P(\bar{X}_{36} < 498)$ revient à chercher $P(Z < -2)$.
    $$ P(Z < -2) = \Phi(-2) $$
    Par symétrie de la loi normale, $\Phi(-z) = 1 - \Phi(z)$.
    $$ P(Z < -2) = 1 - \Phi(2) = 1 - 0.9772 = 0.0228 $$
\end{enumerate}
\textbf{Conclusion :} Il y a 2.28\% de chances d'avoir une fausse alarme, c'est-à-dire de croire à tort que la machine est déréglée alors qu'elle fonctionne normalement.
\end{examplebox}

\begin{examplebox}[Rendement d'un portefeuille (sur la Somme)]
\textbf{Contexte :} Le rendement quotidien $X_i$ d'un actif est très volatile. On s'intéresse au rendement annuel \textbf{total}, qui est la \textbf{somme} des rendements quotidiens.

\textbf{Données :} Supposons que le rendement quotidien $X_i$ ait une espérance $\mu = 0.04\%$ et un écart-type $\sigma = 1\%$. (La loi de $X_i$ est inconnue, mais $\mu$ et $\sigma$ existent). Il y a $n=252$ jours de trading dans l'année.

\textbf{Problème :} Quelle est la probabilité que le rendement annuel total $S_{252} = X_1 + \dots + X_{252}$ soit négatif (inférieur à 0) ?

\textbf{Solution :}
\begin{enumerate}
    \item \textbf{Identifier les paramètres (pour une seule v.a. $X_i$) :}
    $\mu = 0.0004$, $\sigma = 0.01$, $n = 252$.
    
    \item \textbf{Appliquer le TCL (pour la somme $S_n$) :}
    $n=252$ est grand. Le TCL s'applique à la somme $S_n$.
    $$ S_n \approx N\left(n\mu, n\sigma^2\right) $$
    
    \item \textbf{Calculer les paramètres de la loi normale de $S_{252}$ :}
    \begin{itemize}
        \item Espérance de $S_{252}$ : $E[S_n] = n\mu = 252 \times 0.0004 = 0.1008$ (soit 10.08\%).
        \item Variance de $S_{252}$ : $\text{Var}(S_n) = n\sigma^2 = 252 \times (0.01)^2 = 252 \times 0.0001 = 0.0252$.
        \item Écart-type de $S_{252}$ : $\sigma_{S_n} = \sqrt{n\sigma^2} = \sqrt{0.0252} \approx 0.1587$ (soit 15.87\%).
    \end{itemize}
    Donc, $S_{252} \approx N(0.1008, 0.1587^2)$.
    
    \item \textbf{Standardiser (Calculer le Z-score) :}
    Nous cherchons $P(S_{252} < 0)$.
    $$ Z = \frac{S_n - E[S_n]}{\sigma_{S_n}} = \frac{0 - 0.1008}{0.1587} \approx -0.635 $$
    
    \item \textbf{Trouver la probabilité :}
    Chercher $P(S_{252} < 0)$ revient à chercher $P(Z < -0.635)$.
    $$ P(Z < -0.635) = \Phi(-0.635) = 1 - \Phi(0.635) $$
    En interpolant dans la table, $\Phi(0.635) \approx 0.7373$.
    $$ P(Z < -0.635) \approx 1 - 0.7373 = 0.2627 $$
\end{enumerate}
\textbf{Conclusion :} Malgré une espérance de rendement quotidien positive, il y a environ 26.3\% de chances que le rendement annuel total soit négatif.
\end{examplebox}

\begin{examplebox}[Estimation d'une proportion (Marge d'erreur)]
\textbf{Contexte :} On veut estimer la proportion $p$ de votants qui approuvent un candidat. On modélise chaque personne $i$ par une variable de Bernoulli $X_i$ (1 si "oui", 0 si "non").
L'espérance de la population est $\mu = E[X_i] = p$.
La variance de la population est $\sigma^2 = \text{Var}(X_i) = p(1-p)$.
Le résultat du sondage est la moyenne d'échantillon $\bar{X}_n = \hat{p}$ (la proportion observée).

\textbf{Données :} On sonde $n=1000$ personnes. Le résultat est que 540 personnes disent "oui". Donc $\hat{p} = 540/1000 = 0.54$.

\textbf{Problème :} Calculer l'intervalle de confiance à 95\% pour la vraie proportion $p$ (la fameuse "marge d'erreur").

\textbf{Solution :}
\begin{enumerate}
    \item \textbf{Appliquer le TCL :}
    $n=1000$ est grand. Le TCL nous dit que la proportion d'échantillon $\hat{p} = \bar{X}_n$ suit une loi normale :
    $$ \hat{p} \approx N\left(p, \frac{p(1-p)}{n}\right) $$
    
    \item \textbf{Formule de l'Intervalle de Confiance :}
    Un intervalle de confiance à 95\% est centré sur notre estimation $\hat{p}$ et s'étend de $\pm 1.96$ erreurs standard (car $P(-1.96 \le Z \le 1.96) = 0.95$).
    $$ I.C._{95\%} = \left[ \hat{p} - 1.96 \cdot \sigma_{\hat{p}} \ ; \ \hat{p} + 1.96 \cdot \sigma_{\hat{p}} \right] $$
    où $\sigma_{\hat{p}} = \sqrt{p(1-p)/n}$.
    
    \item \textbf{Estimer l'Erreur Standard :}
    Problème : nous ne connaissons pas $p$ (c'est ce que nous cherchons !). Nous ne pouvons donc pas calculer $\sigma_{\hat{p}}$.
    \textbf{Solution :} Nous l'estimons en utilisant notre meilleur estimateur pour $p$, qui est $\hat{p} = 0.54$.
    $$ \text{Erreur Standard Estimée (SE)} = \sqrt{\frac{\hat{p}(1-\hat{p})}{n}} $$
    $$ SE = \sqrt{\frac{0.54 \times (1 - 0.54)}{1000}} = \sqrt{\frac{0.54 \times 0.46}{1000}} = \sqrt{\frac{0.2484}{1000}} \approx \sqrt{0.0002484} \approx 0.01576 $$
    
    \item \textbf{Calculer la Marge d'Erreur :}
    La marge d'erreur (ME) est la demi-largeur de l'intervalle.
    $$ ME = 1.96 \times SE = 1.96 \times 0.01576 \approx 0.0309 $$
    
    \item \textbf{Construire l'Intervalle :}
    $$ I.C._{95\%} = [ 0.54 - 0.0309 \ ; \ 0.54 + 0.0309 ] = [ 0.5091 \ ; \ 0.5709 ] $$
\end{enumerate}
\textbf{Conclusion :} Avec 54\% d'intentions de vote sur un échantillon de 1000 personnes, nous sommes confiants à 95\% que la vraie proportion $p$ dans la population se situe entre 50.9\% et 57.1\%. La marge d'erreur du sondage est de $\pm 3.1\%$.
\end{examplebox}
\input{sections/section11.tex}

% ---

\newpage

\section{Appendice A: Séries de Taylor et Maclaurin}

\begin{definitionbox}[Séries de Taylor et Maclaurin]
Si une fonction $f$ est indéfiniment dérivable au voisinage d'un point $a$, sa \textbf{série de Taylor} centrée en $a$ est définie par :
$$ f(x) = \sum_{k=0}^{\infty} \frac{f^{(k)}(a)}{k!} (x-a)^k $$
où $f^{(k)}(a)$ est la $k$-ième dérivée de $f$ évaluée en $a$.
\newline
\newline
Dans le cas particulier où $\mathbf{a=0}$, la série est appelée une \textbf{série de Maclaurin}. C'est la forme la plus courante, car elle approxime les fonctions autour de l'origine.
\end{definitionbox}

\subsection{Construction pas à pas d'une série de Taylor}

\begin{intuitionbox}[La logique de la correspondance des dérivées]
L'objectif fondamental d'une série de Taylor est de construire un polynôme, $P(x)$, qui soit une "copie conforme" d'une fonction $f(x)$ autour d'un point $a$. Pour ce faire, on force le polynôme à avoir exactement les mêmes propriétés locales que la fonction : même valeur, même pente, même courbure, etc. Cela se traduit mathématiquement par une exigence : \textbf{la n-ième dérivée du polynôme en $a$ doit être égale à la n-ième dérivée de la fonction en $a$}, et ce pour tous les ordres $n$.

Prenons l'exemple de $f(x) = e^x$ et construisons sa série de Maclaurin (centrée en $a=0$), où $f^{(k)}(0)=1$ pour tout $k$.

\begin{enumerate}
    \item \textbf{Ordre 0 : Faire correspondre la valeur}
    \newline
    \textbf{Objectif :} Le polynôme $P_0(x)$ doit avoir la même valeur que $f(x)$ en $x=0$. On veut $P_0(0) = f(0)$.
    \newline
    \textbf{Solution :} On choisit le polynôme le plus simple, une constante : $P_0(x) = f(0)$. Pour $e^x$, $f(0)=1$, donc $\mathbf{P_0(x) = 1}$.
    \newline
    \textbf{Vérification :} $P_0(0) = 1$. L'objectif est atteint.

    \item \textbf{Ordre 1 : Faire correspondre la première dérivée}
    \newline
    \textbf{Objectif :} On veut un nouveau polynôme $P_1(x)$ qui préserve la correspondance précédente ($P_1(0) = f(0)$) ET qui a la même pente, c'est-à-dire $P_1'(0) = f'(0)$.
    \newline
    \textbf{Solution :} On ajoute un terme en $x$ à notre polynôme précédent : $P_1(x) = P_0(x) + c_1 x = 1 + c_1 x$.
    \newline
    \textbf{Vérification :}
    \begin{itemize}
        \item $P_1(0) = 1 + c_1(0) = 1$. La valeur correspond toujours, car le nouveau terme s'annule en 0.
        \item On dérive : $P_1'(x) = c_1$. Pour que les pentes correspondent en 0, il faut $P_1'(0) = c_1 = f'(0)$. Comme $f'(0)=1$, on doit choisir $\mathbf{c_1=1}$.
    \end{itemize}
    Notre polynôme est maintenant $\mathbf{P_1(x) = 1+x}$.

    \item \textbf{Ordre 2 : Faire correspondre la deuxième dérivée}
    \newline
    \textbf{Objectif :} On veut $P_2(x)$ tel que $P_2(0)=f(0)$, $P_2'(0)=f'(0)$ ET $P_2''(0)=f''(0)$.
    \newline
    \textbf{Solution :} On ajoute un terme en $x^2$ : $P_2(x) = P_1(x) + c_2 x^2 = 1 + x + c_2 x^2$.
    \newline
    \textbf{Vérification :}
    \begin{itemize}
        \item Les dérivées d'ordre 0 et 1 en $x=0$ ne sont pas affectées, car la dérivée de $c_2x^2$ (soit $2c_2x$) et le terme lui-même s'annulent en 0. Les objectifs précédents sont préservés.
        \item On dérive deux fois : $P_2'(x) = 1 + 2c_2x$ et $P_2''(x) = 2c_2$.
        \item Pour que les courbures correspondent, il faut $P_2''(0) = 2c_2 = f''(0)$. Comme $f''(0)=1$, on doit choisir $\mathbf{c_2 = 1/2}$.
    \end{itemize}
    Notre polynôme est $\mathbf{P_2(x) = 1+x+\frac{1}{2}x^2}$.

    \item \textbf{Le schéma général : L'importance de la factorielle}
    \newline
    Pour faire correspondre la $k$-ième dérivée, on ajoute un terme $c_k x^k$.
    \newline
    Quand on dérive $c_k x^k$ exactement $k$ fois, on obtient $c_k \times k!$.
    \newline
    Toutes les dérivées d'ordre inférieur s'annulent en $x=0$. On doit donc avoir :
    $$ P_k^{(k)}(0) = c_k \cdot k! = f^{(k)}(0) $$
    Cela nous donne la règle pour trouver chaque coefficient :
    $$ c_k = \frac{f^{(k)}(0)}{k!} $$
    C'est précisément le coefficient qui apparaît dans la formule de Taylor, et il est choisi pour cette unique raison : forcer la $k$-ième dérivée du polynôme à correspondre parfaitement à celle de la fonction au point de développement.
\end{enumerate}

\tcblower
\centering
\begin{tikzpicture}
    \begin{axis}[
        xlabel={$x$},
        ylabel={$y$},
        xmin=-2, xmax=2,
        ymin=-0.5, ymax=4,
        axis lines=middle,
        legend style={at={(0.05,0.95)}, anchor=north west, font=\small},
        grid=major,
        samples=150,
        domain=-2:2,
        height=9cm,
        width=\linewidth-1cm,
        tick label style={font=\tiny}
    ]
    
    \addplot[black, dashed, ultra thick] {exp(x)};
    \addlegendentry{$e^x$}

    \addplot[red, thick] {1};
    \addlegendentry{$P_0(x)=1$}

    \addplot[blue, thick] {1+x};
    \addlegendentry{$P_1(x)=1+x$}

    \addplot[green!70!black, thick] {1+x+x^2/2};
    \addlegendentry{$P_2(x)=1+x+\frac{x^2}{2!}$}

    \addplot[orange, thick] {1+x+x^2/2+x^3/6};
    \addlegendentry{$P_3(x)=1+x+\frac{x^2}{2!}+\frac{x^3}{3!}$}

    \end{axis}
\end{tikzpicture}
\par\small\textit{Visualisation de la construction progressive de la série de Maclaurin pour $e^x$.}
\end{intuitionbox}

\subsection{Intuition de la série de Taylor en un point quelconque $a$}

\begin{intuitionbox}[Construire une approximation loin de l'origine]
La série de Maclaurin est puissante, mais elle nous contraint à approximer une fonction uniquement autour de $x=0$. Que faire si l'on s'intéresse au comportement d'une fonction ailleurs, par exemple $f(x)=\ln(x)$ autour de $x=1$ (puisque $\ln(0)$ n'est pas défini) ? C'est là qu'intervient la série de Taylor générale.

L'objectif reste le même : construire un polynôme $P(x)$ qui est une "copie conforme" de $f(x)$ au point $a$. Pour cela, on force les dérivées du polynôme à correspondre à celles de la fonction en ce point $a$. La seule différence est que notre "variable" de base n'est plus $x$, mais l'écart par rapport au centre, c'est-à-dire $(x-a)$.

Prenons l'exemple de $f(x) = \ln(x)$ et construisons sa série centrée en $\mathbf{a=1}$.

\begin{enumerate}
    \item \textbf{Ordre 0 : Faire correspondre la valeur}
    \newline
    \textbf{Objectif :} $P_0(a) = f(a)$.
    \newline
    \textbf{Solution :} On calcule $f(1) = \ln(1) = 0$. Le polynôme est la constante $\mathbf{P_0(x) = 0}$.

    \item \textbf{Ordre 1 : Faire correspondre la pente}
    \newline
    \textbf{Objectif :} $P_1(a) = f(a)$ et $P_1'(a) = f'(a)$.
    \newline
    \textbf{Solution :} On ajoute un terme proportionnel à l'écart $(x-a)$ : $P_1(x) = f(a) + c_1 (x-a)$.
    \newline
    \textbf{Vérification :}
    \begin{itemize}
        \item $P_1(1) = 0 + c_1(1-1) = 0$. La valeur correspond.
        \item On dérive : $P_1'(x) = c_1$. On veut $P_1'(1) = c_1 = f'(1)$.
        \item La dérivée de $f(x)=\ln(x)$ est $f'(x) = 1/x$, donc $f'(1)=1$. On doit choisir $\mathbf{c_1=1}$.
    \end{itemize}
    Notre polynôme est $\mathbf{P_1(x) = (x-1)}$. C'est la tangente à $\ln(x)$ en $x=1$.

    \item \textbf{Ordre 2 : Faire correspondre la courbure}
    \newline
    \textbf{Objectif :} Les dérivées jusqu'à l'ordre 2 doivent correspondre en $a=1$.
    \newline
    \textbf{Solution :} On ajoute un terme en $(x-a)^2$ : $P_2(x) = (x-1) + c_2 (x-1)^2$.
    \newline
    \textbf{Vérification :}
    \begin{itemize}
        \item Les correspondances d'ordre 0 et 1 sont préservées.
        \item On dérive deux fois : $P_2'(x) = 1 + 2c_2(x-1)$ et $P_2''(x) = 2c_2$.
        \item On veut $P_2''(1) = 2c_2 = f''(1)$.
        \item La dérivée seconde de $f(x)$ est $f''(x) = -1/x^2$, donc $f''(1)=-1$. On choisit $\mathbf{c_2 = -1/2}$.
    \end{itemize}
    Notre polynôme est $\mathbf{P_2(x) = (x-1) - \frac{1}{2}(x-1)^2}$.

    \item \textbf{Le schéma général}
    \newline
    Le coefficient $c_k$ du terme $(x-a)^k$ est choisi pour faire correspondre la $k$-ième dérivée. La dérivation de $c_k(x-a)^k$ $k$ fois donne $c_k \cdot k!$. On impose donc $c_k \cdot k! = f^{(k)}(a)$, ce qui mène directement à la formule générale $c_k = \frac{f^{(k)}(a)}{k!}$.
\end{enumerate}

\tcblower
\centering
\begin{tikzpicture}
    \begin{axis}[
        xlabel={$x$},
        ylabel={$y$},
        xmin=-0.5, xmax=2.5,
        ymin=-2, ymax=1,
        axis lines=middle,
        legend style={at={(0.05,0.05)}, anchor=south west, font=\small},
        grid=major,
        samples=150,
        domain=0.01:2.5,
        height=9cm,
        width=\linewidth-1cm,
        tick label style={font=\tiny}
    ]
    
    \addplot[black, dashed, ultra thick] {ln(x)};
    \addlegendentry{$\ln(x)$}

    \addplot[red, thick, domain=-0.5:2.5] {0};
    \addlegendentry{$P_0(x)=0$}

    \addplot[blue, thick, domain=-0.5:2.5] {x-1};
    \addlegendentry{$P_1(x)=(x-1)$}

    \addplot[green!70!black, thick, domain=-0.5:2.5] {(x-1) - 0.5*(x-1)^2};
    \addlegendentry{$P_2(x)=(x-1)-\frac{(x-1)^2}{2}$}

    \end{axis}
\end{tikzpicture}
\par\small\textit{Approximation de $\ln(x)$ autour de $a=1$. Le polynôme "colle" à la fonction près de $x=1$.}
\end{intuitionbox}


\subsection{La Fonction Exponentielle ($e^x$)}

\begin{theorembox}[Série de Maclaurin pour $e^x$]
Pour tout nombre réel $x$, la fonction exponentielle peut s'écrire :
$$ e^x = \sum_{k=0}^{\infty} \frac{x^k}{k!} = 1 + x + \frac{x^2}{2!} + \frac{x^3}{3!} + \frac{x^4}{4!} + \cdots $$
\end{theorembox}

\begin{intuitionbox}[Visualiser la Croissance Exponentielle]
La fonction exponentielle est unique car elle est sa propre dérivée. Cela signifie que toutes ses informations locales (valeur, pente, courbure) en $a=0$ sont égales à \textbf{1}. La série pour $e^x$ est donc le polynôme le plus « pur », où chaque terme $x^k$ est simplement normalisé par $k!$. Le graphique ci-dessous montre comment les polynômes de Taylor convergent rapidement vers la véritable courbe exponentielle, illustrant sa croissance puissante.

\tcblower

\centering
\begin{tikzpicture}
    \begin{axis}[
        xlabel={$x$},
        ylabel={$y$},
        xmin=-3, xmax=3,
        ymin=-1, ymax=9,
        axis lines=middle,
        legend style={at={(0.05,0.95)}, anchor=north west, font=\small},
        grid=major,
        samples=150,
        domain=-3:3,
        height=9cm,
        width=\linewidth-1cm,
        tick label style={font=\tiny}
    ]
    
    \addplot[black, dashed, ultra thick] {exp(x)};
    \addlegendentry{$e^x$}

    \addplot[red, thick] {1+x};
    \addlegendentry{$T_1(x)$}

    \addplot[blue, thick] {1+x+x^2/2};
    \addlegendentry{$T_2(x)$}

    \addplot[green!70!black, thick] {1+x+x^2/2+x^3/6};
    \addlegendentry{$T_3(x)$}

    \addplot[orange, thick] {1+x+x^2/2+x^3/6+x^4/24};
    \addlegendentry{$T_4(x)$}

    \end{axis}
\end{tikzpicture}
\par\small\textit{Approximation de $e^x$ par ses polynômes de Maclaurin.}
\end{intuitionbox}

\begin{proofbox}
Soit $f(x) = e^x$. Pour tout entier $k \ge 0$, la $k$-ième dérivée est $f^{(k)}(x) = e^x$. En évaluant en $a=0$, on obtient $f^{(k)}(0) = e^0 = 1$ pour tout $k$. En appliquant la formule de Maclaurin :
$$ e^x = \sum_{k=0}^{\infty} \frac{f^{(k)}(0)}{k!} x^k = \sum_{k=0}^{\infty} \frac{1}{k!} x^k = 1 + x + \frac{x^2}{2} + \frac{x^3}{6} + \cdots $$
\end{proofbox}


\subsection{La Fonction Sinus ($\sin(x)$)}

\begin{theorembox}[Série de Maclaurin pour $\sin(x)$]
Pour tout nombre réel $x$ :
$$ \sin(x) = \sum_{k=0}^{\infty} (-1)^k \frac{x^{2k+1}}{(2k+1)!} = x - \frac{x^3}{3!} + \frac{x^5}{5!} - \frac{x^7}{7!} + \cdots $$
\end{theorembox}

\begin{intuitionbox}[Visualiser l'Oscillation du Sinus]
La série du sinus reflète ses propriétés fondamentales. En tant que fonction \textbf{impaire} ($ \sin(-x) = -\sin(x) $), son développement ne contient que des puissances \textbf{impaires} de $x$. Les signes alternés capturent sa nature oscillatoire. Le graphique ci-dessous montre comment l'ajout de termes permet au polynôme d'« épouser » la courbe du sinus sur un plus grand nombre de périodes.

\tcblower

\centering
\begin{tikzpicture}
    \begin{axis}[
        xlabel={$x$},
        ylabel={$y$},
        xmin=-2*pi, xmax=2*pi,
        ymin=-2.0, ymax=2.0,
        axis lines=middle,
        legend style={at={(0.5,1.15)}, anchor=south, font=\small, column sep=5pt},
        legend columns=3,
        grid=major,
        samples=200,
        domain=-2*pi:2*pi,
        height=9cm,
        width=\linewidth-1cm,
        tick label style={font=\tiny}
    ]
    \addplot[black, dashed, ultra thick] {sin(deg(x))};
    \addlegendentry{$\sin(x)$}
    \addplot[red, thick] {x};
    \addlegendentry{$T_1(x)$}
    \addplot[blue, thick] {x - (x^3)/6};
    \addlegendentry{$T_3(x)$}
    \addplot[green!70!black, thick] {x - (x^3)/6 + (x^5)/120};
    \addlegendentry{$T_5(x)$}
    \addplot[orange, thick] {x - (x^3)/6 + (x^5)/120 - (x^7)/5040};
    \addlegendentry{$T_7(x)$}
    \end{axis}
\end{tikzpicture}
\par\small\textit{Approximation de $\sin(x)$ par ses polynômes de Maclaurin.}
\end{intuitionbox}

\begin{proofbox}
Soit $f(x) = \sin(x)$. Les dérivées en $a=0$ suivent un cycle $(0, 1, 0, -1, \dots)$. Seuls les termes d'ordre impair ($2k+1$) sont non nuls, avec des valeurs de $(-1)^k$, ce qui donne la formule.
\end{proofbox}


\subsection{La Fonction Cosinus ($\cos(x)$)}

\begin{theorembox}[Série de Maclaurin pour $\cos(x)$]
Pour tout nombre réel $x$ :
$$ \cos(x) = \sum_{k=0}^{\infty} (-1)^k \frac{x^{2k}}{(2k)!} = 1 - \frac{x^2}{2!} + \frac{x^4}{4!} - \frac{x^6}{6!} + \cdots $$
\end{theorembox}

\begin{intuitionbox}[Visualiser la Symétrie du Cosinus]
En tant que fonction \textbf{paire} ($ \cos(-x) = \cos(x) $), la série du cosinus ne contient, de manière appropriée, que des puissances \textbf{paires} de $x$. Elle commence à 1 (son maximum) puis oscille, un comportement capturé par les signes alternés.

\tcblower

\centering
\begin{tikzpicture}
    \begin{axis}[
        xlabel={$x$},
        ylabel={$y$},
        xmin=-2*pi, xmax=2*pi,
        ymin=-2.0, ymax=2.0,
        axis lines=middle,
        legend style={at={(0.5,1.15)}, anchor=south, font=\small, column sep=5pt},
        legend columns=3,
        grid=major,
        samples=200,
        domain=-2*pi:2*pi,
        height=9cm,
        width=\linewidth-1cm,
        tick label style={font=\tiny}
    ]
    \addplot[black, dashed, ultra thick] {cos(deg(x))};
    \addlegendentry{$\cos(x)$}
    \addplot[red, thick] {1};
    \addlegendentry{$T_0(x)$}
    \addplot[blue, thick] {1 - x^2/2};
    \addlegendentry{$T_2(x)$}
    \addplot[green!70!black, thick] {1 - x^2/2 + x^4/24};
    \addlegendentry{$T_4(x)$}
    \addplot[orange, thick] {1 - x^2/2 + x^4/24 - x^6/720};
    \addlegendentry{$T_6(x)$}
    \end{axis}
\end{tikzpicture}
\par\small\textit{Approximation de $\cos(x)$ par ses polynômes de Maclaurin.}
\end{intuitionbox}

\begin{proofbox}
Soit $g(x) = \cos(x)$. Les dérivées en $a=0$ suivent un cycle $(1, 0, -1, 0, \dots)$. Seuls les termes d'ordre pair ($2k$) sont non nuls, avec des valeurs de $(-1)^k$, ce qui donne la formule.
\end{proofbox}


\subsection{Le Logarithme Népérien ($\ln(1+x)$)}

\begin{theorembox}[Série de Maclaurin pour $\ln(1+x)$]
Pour $|x| < 1$ :
$$ \ln(1+x) = \sum_{k=1}^{\infty} (-1)^{k-1} \frac{x^k}{k} = x - \frac{x^2}{2} + \frac{x^3}{3} - \frac{x^4}{4} + \cdots $$
\end{theorembox}

\begin{intuitionbox}[Visualiser l'Approximation Logarithmique]
Cette série est essentielle pour approximer les logarithmes près de 1. Contrairement aux fonctions précédentes, elle ne converge que pour $|x|<1$. Le graphique montre que l'approximation est excellente près de $x=0$ mais diverge rapidement lorsque $x$ s'approche de la frontière de convergence à $x=1$.

\tcblower

\centering
\begin{tikzpicture}
    \begin{axis}[
        xlabel={$x$},
        ylabel={$y$},
        xmin=-1.2, xmax=1.2,
        ymin=-4, ymax=2,
        axis lines=middle,
        legend style={at={(0.05,0.95)}, anchor=north west, font=\small},
        grid=major,
        samples=150,
        domain=-0.99:1, % Domain restricted for ln
        height=9cm,
        width=\linewidth-1cm,
        tick label style={font=\tiny}
    ]
    \addplot[black, dashed, ultra thick] {ln(1+x)};
    \addlegendentry{$\ln(1+x)$}
    
    \addplot[red, thick, domain=-1.2:1.2] {x};
    \addlegendentry{$T_1(x)$}

    \addplot[blue, thick, domain=-1.2:1.2] {x - x^2/2};
    \addlegendentry{$T_2(x)$}

    \addplot[green!70!black, thick, domain=-1.2:1.2] {x - x^2/2 + x^3/3};
    \addlegendentry{$T_3(x)$}
    
    \addplot[orange, thick, domain=-1.2:1.2] {x - x^2/2 + x^3/3 - x^4/4};
    \addlegendentry{$T_4(x)$}

    \end{axis}
\end{tikzpicture}
\par\small\textit{Approximation de $\ln(1+x)$ par ses polynômes de Maclaurin.}
\end{intuitionbox}

\begin{proofbox}
Soit $f(x) = \ln(1+x)$. Pour $k \ge 1$, la $k$-ième dérivée en $a=0$ est $f^{(k)}(0) = (-1)^{k-1} (k-1)!$. En substituant cela dans la formule de Maclaurin, le $(k-1)!$ au numérateur annule partiellement le $k!$ au dénominateur, laissant un $k$ en bas.
\end{proofbox}

\subsection{La Série Géométrique ($\frac{1}{1-x}$)}

\begin{theorembox}[Série de Maclaurin pour $\frac{1}{1-x}$]
Pour $|x| < 1$ :
$$ \frac{1}{1-x} = \sum_{k=0}^{\infty} x^k = 1 + x + x^2 + x^3 + \cdots $$
\end{theorembox}

\begin{intuitionbox}[Le Fondement de Nombreuses Séries]
Cette série, connue sous le nom de série géométrique, est l'un des développements en série de puissances les plus fondamentaux. Elle converge uniquement lorsque la valeur absolue de $x$ est inférieure à 1. Chaque coefficient est simplement 1, ce qui en fait la série de Maclaurin la plus simple. De nombreuses autres séries, comme celle de $\ln(1+x)$ ou de $\arctan(x)$, peuvent être dérivées de celle-ci par intégration ou substitution.
\end{intuitionbox}

\begin{proofbox}
Soit $f(x) = (1-x)^{-1}$. Les dérivées successives sont $f'(x) = 1(1-x)^{-2}$, $f''(x) = 2(1-x)^{-3}$, $f'''(x) = 6(1-x)^{-4}$, et ainsi de suite. La formule générale pour la $k$-ième dérivée est $f^{(k)}(x) = k!(1-x)^{-(k+1)}$. En évaluant en $a=0$, on obtient $f^{(k)}(0) = k!$. En substituant dans la formule de Maclaurin :
$$ \frac{1}{1-x} = \sum_{k=0}^{\infty} \frac{f^{(k)}(0)}{k!} x^k = \sum_{k=0}^{\infty} \frac{k!}{k!} x^k = \sum_{k=0}^{\infty} x^k $$
\end{proofbox}
\newpage

\section{Appendice B: Intégrales Doubles}

\begin{definitionbox}[Intégrales Doubles]
Une \textbf{intégrale double} permet de calculer le volume sous une surface dans l'espace tridimensionnel, ou l'accumulation d'une quantité sur une région plane. Pour une fonction $f(x,y)$ définie sur un domaine $D \subset \mathbb{R}^2$, elle s'écrit :
$$ \iint_D f(x,y) \, dA = \iint_D f(x,y) \, dx\,dy $$
où $dA$ représente un élément d'aire infinitésimal.
\end{definitionbox}

\subsection{Construction pas à pas d'une intégrale double}

\begin{intuitionbox}[De l'intégrale simple à l'intégrale double]
L'objectif fondamental d'une intégrale double est d'étendre le concept d'accumulation (aire sous une courbe) à deux dimensions. Au lieu de sommer des rectangles de largeur $dx$ sous une courbe, on somme des \textbf{parallélépipèdes} de base $dx\,dy$ sous une surface. Le principe clé : \textbf{on intègre par étapes successives}, d'abord selon une variable en gardant l'autre fixe, puis selon la seconde variable.

Prenons l'exemple du calcul du volume sous la surface $f(x,y) = xy$ sur le rectangle $D = [0,1] \times [0,2]$.

\begin{enumerate}
    \item \textbf{Visualisation : Comprendre le domaine}
    \newline
    \textbf{Objectif :} Identifier la région d'intégration $D$ dans le plan $xy$.
    \newline
    \textbf{Notre cas :} $D$ est un rectangle avec $0 \le x \le 1$ et $0 \le y \le 2$. C'est le domaine le plus simple : les bornes sont constantes.
    \newline
    \textbf{Notation :} On écrira $\displaystyle\int_0^1 \int_0^2 xy \, dy\,dx$.

    \item \textbf{Première intégration : Fixer $x$, intégrer selon $y$}
    \newline
    \textbf{Objectif :} Pour chaque valeur fixée de $x$, calculer l'aire de la "tranche verticale" obtenue en variant $y$ de 0 à 2.
    \newline
    \textbf{Calcul :} On traite $x$ comme une constante et on intègre $xy$ par rapport à $y$ :
    $$ \int_0^2 xy \, dy = x \int_0^2 y \, dy = x \left[\frac{y^2}{2}\right]_0^2 = x \cdot \frac{4}{2} = 2x $$
    \textbf{Interprétation :} Le résultat $2x$ est une fonction de $x$ seulement. Elle représente l'aire de chaque tranche verticale à la position $x$.

    \item \textbf{Seconde intégration : Sommer toutes les tranches}
    \newline
    \textbf{Objectif :} Additionner les contributions de toutes les tranches verticales en intégrant selon $x$ de 0 à 1.
    \newline
    \textbf{Calcul :}
    $$ \int_0^1 2x \, dx = 2 \int_0^1 x \, dx = 2 \left[\frac{x^2}{2}\right]_0^1 = 2 \cdot \frac{1}{2} = 1 $$
    \textbf{Résultat final :} Le volume sous la surface $f(x,y)=xy$ sur $D$ est $\mathbf{1}$.

    \item \textbf{L'ordre d'intégration peut être inversé}
    \newline
    On aurait pu intégrer d'abord selon $x$, puis selon $y$ :
    $$ \int_0^2 \int_0^1 xy \, dx\,dy = \int_0^2 \left[\frac{x^2y}{2}\right]_0^1 dy = \int_0^2 \frac{y}{2} \, dy = \left[\frac{y^2}{4}\right]_0^2 = 1 $$
    \textbf{Principe de Fubini :} Pour les domaines rectangulaires avec fonctions continues, l'ordre d'intégration n'affecte pas le résultat, mais un ordre peut être plus simple que l'autre selon la fonction.

    \item \textbf{Le schéma général : Domaines non rectangulaires}
    \newline
    Lorsque $D$ n'est pas un rectangle, les bornes de l'intégrale intérieure dépendent de la variable extérieure.
    \newline
    \textbf{Exemple :} Si $D$ est délimité par $0 \le x \le 1$ et $0 \le y \le x^2$, on écrit :
    $$ \iint_D f(x,y) \, dA = \int_0^1 \int_0^{x^2} f(x,y) \, dy\,dx $$
    Les bornes $0$ et $x^2$ pour $y$ changent avec $x$, définissant ainsi la forme courbe de $D$.
\end{enumerate}
\end{intuitionbox}

\subsection{Intuition géométrique de l'intégrale double}

\begin{intuitionbox}[Interpréter l'intégrale double comme un volume]
Une intégrale simple $\int_a^b f(x) \, dx$ calcule l'aire sous une courbe. Une intégrale double $\iint_D f(x,y) \, dA$ calcule le \textbf{volume sous une surface} $z=f(x,y)$ au-dessus d'un domaine plan $D$.

\textbf{Cas particulier : Aire d'une région plane}
\newline
Si $f(x,y) = 1$ partout sur $D$, alors :
$$ \iint_D 1 \, dA = \text{Aire}(D) $$
L'intégrale double de la fonction constante 1 donne simplement l'aire du domaine.

\tcblower
\centering
\begin{tikzpicture}[scale=1.2]
    \begin{axis}[
        view={120}{30},
        xlabel={$x$},
        ylabel={$y$},
        zlabel={$z$},
        domain=-1:1,
        y domain=-1:1,
        samples=20,
        height=8cm,
        width=0.9\linewidth,
        colormap/cool,
    ]
    
    \addplot3[
        surf,
        shader=flat,
        opacity=0.7,
    ] {x*y};
    
    \end{axis}
\end{tikzpicture}
\par\small\textit{Surface $z=xy$ au-dessus d'un domaine carré. Le volume sous cette surface est donné par l'intégrale double.}
\end{intuitionbox}


\subsection{Le Théorème de Fubini}

\begin{theorembox}[Théorème de Fubini]
Si $f(x,y)$ est continue sur le rectangle $R = [a,b] \times [c,d]$, alors :
$$ \iint_R f(x,y) \, dA = \int_a^b \int_c^d f(x,y) \, dy\,dx = \int_c^d \int_a^b f(x,y) \, dx\,dy $$
L'intégrale double peut être calculée comme deux intégrales simples itérées, et l'ordre d'intégration peut être inversé.
\end{theorembox}

\begin{intuitionbox}[Choisir le bon ordre d'intégration]
Bien que le théorème de Fubini garantisse que les deux ordres donnent le même résultat, l'un peut être beaucoup plus simple à calculer que l'autre.

\textbf{Exemple : $\displaystyle\iint_D e^{y^2} \, dA$ où $D = \{(x,y) : 0 \le x \le 1, \, x \le y \le 1\}$}

\textbf{Ordre 1 : Intégrer d'abord selon $x$, puis selon $y$}
\newline
Il faut reformuler les bornes : pour $0 \le y \le 1$, on a $0 \le x \le y$.
$$ \int_0^1 \int_0^y e^{y^2} \, dx\,dy = \int_0^1 e^{y^2} [x]_0^y \, dy = \int_0^1 y e^{y^2} \, dy $$
Par substitution $u=y^2$, $du=2y\,dy$ :
$$ = \frac{1}{2} \int_0^1 e^u \, du = \frac{1}{2}[e^u]_0^1 = \frac{e-1}{2} $$

\textbf{Ordre 2 : Intégrer d'abord selon $y$, puis selon $x$}
$$ \int_0^1 \int_x^1 e^{y^2} \, dy\,dx $$
L'intégrale $\int e^{y^2} \, dy$ n'a pas de forme close élémentaire ! Cet ordre est impraticable.

\textbf{Leçon :} Toujours examiner la fonction et le domaine avant de choisir l'ordre d'intégration.

\end{intuitionbox}

\subsection{Changement de l'ordre d'intégration : Exemple simple}

\begin{examplebox}[Exemple concret : Triangle]
Calculons l'intégrale : $$\int_0^1 \int_0^x y \, dy \, dx$$

\textbf{Région d'intégration initiale :}
\begin{itemize}
    \item $x$ varie de 0 à 1
    \item Pour chaque $x$ fixé, $y$ varie de 0 à $x$
\end{itemize}

C'est un \textbf{triangle} : $D = \{(x,y) : 0 \le x \le 1 \text{ et } 0 \le y \le x\}$

\tcblower

\textbf{Méthode 1 : Ordre initial (on intègre d'abord en $y$)}

\begin{align*}
\int_0^1 \left(\int_0^x y \, dy\right) dx &= \int_0^1 \left[\frac{y^2}{2}\right]_0^x dx \\
&= \int_0^1 \frac{x^2}{2} \, dx \\
&= \left[\frac{x^3}{6}\right]_0^1 \\
&= \frac{1}{6}
\end{align*}

\tcblower

\textbf{Méthode 2 : Changement de l'ordre (on intègre d'abord en $x$)}

\textit{L'astuce :} Redécrire la même région en fixant $y$ d'abord !

Dans le triangle, si je fixe $y$ :
\begin{itemize}
    \item $y$ varie de 0 à 1 (minimum et maximum de $y$ dans le triangle)
    \item Pour chaque $y$ fixé, $x$ varie de $y$ à 1 (car $x \ge y$ dans notre région)
\end{itemize}

\textbf{Nouvelles bornes :}
$$ \int_0^1 \int_y^1 y \, dx \, dy $$

\textbf{Calcul :}
\begin{align*}
\int_0^1 \left(\int_y^1 y \, dx\right) dy &= \int_0^1 y[x]_y^1 \, dy \\
&= \int_0^1 y(1-y) \, dy \\
&= \int_0^1 (y - y^2) \, dy \\
&= \left[\frac{y^2}{2} - \frac{y^3}{3}\right]_0^1 \\
&= \frac{1}{2} - \frac{1}{3} \\
&= \frac{1}{6} \quad \checkmark
\end{align*}
\end{examplebox}
\newpage

\section{Appendice C: La Règle de la Chaîne}

\begin{definitionbox}[Règle de la Chaîne]
La \textbf{règle de la chaîne} permet de calculer la dérivée d'une fonction composée. Si $y = f(u)$ et $u = g(x)$, alors la fonction composée $y = f(g(x))$ a pour dérivée :
$$ \frac{dy}{dx} = \frac{dy}{du} \cdot \frac{du}{dx} = f'(g(x)) \cdot g'(x) $$
Cette règle exprime que la variation de $y$ par rapport à $x$ est le produit des variations intermédiaires.
\end{definitionbox}

\subsection{Construction pas à pas de la règle de la chaîne}

\begin{intuitionbox}[De la dérivée simple à la composition]
L'objectif fondamental de la règle de la chaîne est de gérer les fonctions imbriquées. Au lieu de dériver directement une expression complexe, on la \textbf{décompose en couches} et on dérive chaque couche successivement. Le principe clé : \textbf{on dérive de l'extérieur vers l'intérieur}, en multipliant les dérivées à chaque étape.

Prenons l'exemple du calcul de la dérivée de $y = \sin(x^2)$.

\begin{enumerate}
    \item \textbf{Identification : Reconnaître la composition}
    \newline
    \textbf{Objectif :} Identifier les fonctions emboîtées pour appliquer la règle.
    \newline
    \textbf{Notre cas :} $y = \sin(x^2)$ est une composition de deux fonctions :
    \begin{itemize}
        \item Fonction extérieure : $f(u) = \sin(u)$
        \item Fonction intérieure : $u = g(x) = x^2$
    \end{itemize}
    \textbf{Notation :} On peut écrire $y = f(g(x))$ ou $y = (f \circ g)(x)$.

    \item \textbf{Première dérivation : La couche extérieure}
    \newline
    \textbf{Objectif :} Dériver la fonction extérieure par rapport à son argument, sans toucher à l'intérieur.
    \newline
    \textbf{Calcul :} On dérive $\sin(u)$ par rapport à $u$ :
    $$ \frac{dy}{du} = \cos(u) = \cos(x^2) $$
    \textbf{Interprétation :} On garde $x^2$ intact à l'intérieur du cosinus. Cette dérivée partielle nous dit comment $y$ change quand $u$ change.

    \item \textbf{Seconde dérivation : La couche intérieure}
    \newline
    \textbf{Objectif :} Dériver la fonction intérieure par rapport à la variable originale $x$.
    \newline
    \textbf{Calcul :}
    $$ \frac{du}{dx} = \frac{d}{dx}(x^2) = 2x $$
    \textbf{Interprétation :} Cette dérivée nous dit comment $u$ change quand $x$ change.

    \item \textbf{Multiplication finale : Combiner les dérivées}
    \newline
    \textbf{Objectif :} Appliquer la règle de la chaîne en multipliant les deux dérivées.
    \newline
    \textbf{Calcul :}
    $$ \frac{dy}{dx} = \frac{dy}{du} \cdot \frac{du}{dx} = \cos(x^2) \cdot 2x = 2x\cos(x^2) $$
    \textbf{Résultat final :} La dérivée de $\sin(x^2)$ est $\mathbf{2x\cos(x^2)}$.

    \item \textbf{Notation alternative : Leibniz}
    \newline
    La notation de Leibniz rend la règle intuitive : les "$du$" se "simplifient" symboliquement :
    $ \frac{dy}{dx} = \frac{dy}{du} \cdot \frac{du}{dx} $
    Bien que ce ne soit pas une vraie simplification algébrique, cette visualisation aide à retenir la structure de la règle.

    \item \textbf{Compositions multiples : Chaînes plus longues}
    \newline
    Pour trois fonctions ou plus, on continue à multiplier les dérivées.
    \newline
    \textbf{Exemple :} Si $y = f(g(h(x)))$, alors :
    $$ \frac{dy}{dx} = f'(g(h(x))) \cdot g'(h(x)) \cdot h'(x) $$
    \textbf{Cas concret :} Pour $y = e^{\sin(x^2)}$ :
    \begin{align*}
    \frac{dy}{dx} &= e^{\sin(x^2)} \cdot \frac{d}{dx}[\sin(x^2)] \\
    &= e^{\sin(x^2)} \cdot \cos(x^2) \cdot 2x \\
    &= 2x \cos(x^2) e^{\sin(x^2)}
    \end{align*}
\end{enumerate}
\end{intuitionbox}

\subsection{Intuition géométrique de la règle de la chaîne}

\begin{intuitionbox}[Interpréter la règle comme un taux de changement composé]
Une dérivée simple $f'(x)$ mesure le taux de changement instantané de $f$ par rapport à $x$. La règle de la chaîne mesure le \textbf{taux de changement à travers une variable intermédiaire}.

\textbf{Analogie physique : Vitesse et position}
\newline
Supposons qu'une particule se déplace sur un axe et que sa position $s$ dépend du temps $t$ selon $s = t^3$. Sa température $T$ dépend de sa position selon $T = \sqrt{s}$. Quelle est la variation de température par rapport au temps ?

La règle de la chaîne nous dit :
$$ \frac{dT}{dt} = \frac{dT}{ds} \cdot \frac{ds}{dt} = \frac{1}{2\sqrt{s}} \cdot 3t^2 = \frac{3t^2}{2\sqrt{t^3}} = \frac{3t^2}{2t^{3/2}} = \frac{3\sqrt{t}}{2} $$

La variation de $T$ par rapport à $t$ est le produit :
\begin{itemize}
    \item de la sensibilité de $T$ aux changements de position ($\frac{dT}{ds}$)
    \item et de la vitesse de changement de position ($\frac{ds}{dt}$)
\end{itemize}

\textbf{Cas particulier : Amplification ou atténuation}
\newline
Si $|g'(x)| > 1$, la fonction intérieure "amplifie" les variations de $x$.
\newline
Si $|g'(x)| < 1$, elle les "atténue".
\newline
La règle de la chaîne capture cet effet multiplicatif.

\tcblower
\centering
\begin{tikzpicture}[scale=1.2]
    \begin{axis}[
        axis lines=middle,
        domain=0:2,
        samples=100,
        height=6cm,
        width=0.9\linewidth,
        ymin=0, ymax=1.7,
        legend pos=north west,
    ]
    
    \addplot[blue, thick] {sin(deg(x^2))};
    \addlegendentry{$y=\sin(x^2)$}
    
    \addplot[red, dashed, thick] {2*x*cos(deg(x^2))};
    \addlegendentry{$y'=2x\cos(x^2)$}
    
    \end{axis}
\end{tikzpicture}
\par\small\textit{La fonction $\sin(x^2)$ et sa dérivée obtenue par la règle de la chaîne. La pente s'annule quand $\cos(x^2)=0$ ou $x=0$.}
\end{intuitionbox}


\subsection{Applications classiques de la règle de la chaîne}

\begin{theorembox}[Formules dérivées courantes avec composition]
Voici les applications les plus fréquentes de la règle de la chaîne :
\begin{align*}
\frac{d}{dx}[f(x)^n] &= n \cdot f(x)^{n-1} \cdot f'(x) \quad \text{(règle de puissance généralisée)} \\
\frac{d}{dx}[e^{f(x)}] &= e^{f(x)} \cdot f'(x) \\
\frac{d}{dx}[\ln(f(x))] &= \frac{f'(x)}{f(x)} \\
\frac{d}{dx}[\sin(f(x))] &= \cos(f(x)) \cdot f'(x) \\
\frac{d}{dx}[\cos(f(x))] &= -\sin(f(x)) \cdot f'(x)
\end{align*}
\end{theorembox}

\begin{intuitionbox}[Stratégie pour des compositions complexes]
Face à une fonction compliquée, une approche systématique évite les erreurs.

\textbf{Exemple : $\displaystyle y = \ln\left(\frac{1+\sqrt{x}}{1-\sqrt{x}}\right)$}

\textbf{Stratégie 1 : Décomposition explicite}
\newline
Posons des variables intermédiaires :
\begin{align*}
u &= \sqrt{x} = x^{1/2}, \quad \frac{du}{dx} = \frac{1}{2\sqrt{x}} \\
v &= \frac{1+u}{1-u}, \quad \frac{dv}{du} = \frac{(1-u) - (1+u)(-1)}{(1-u)^2} = \frac{2}{(1-u)^2} \\
y &= \ln(v), \quad \frac{dy}{dv} = \frac{1}{v}
\end{align*}

Application de la règle de la chaîne :
$$ \frac{dy}{dx} = \frac{dy}{dv} \cdot \frac{dv}{du} \cdot \frac{du}{dx} = \frac{1}{v} \cdot \frac{2}{(1-u)^2} \cdot \frac{1}{2\sqrt{x}} $$

Substitution :
$$ = \frac{1}{\frac{1+\sqrt{x}}{1-\sqrt{x}}} \cdot \frac{2}{(1-\sqrt{x})^2} \cdot \frac{1}{2\sqrt{x}} = \frac{1-\sqrt{x}}{1+\sqrt{x}} \cdot \frac{2}{(1-\sqrt{x})^2} \cdot \frac{1}{2\sqrt{x}} $$

Simplification :
$$ = \frac{1}{(1+\sqrt{x})(1-\sqrt{x})} \cdot \frac{1}{\sqrt{x}} = \frac{1}{(1-x)\sqrt{x}} $$

\textbf{Stratégie 2 : Simplification avant dérivation}
\newline
On peut parfois simplifier avec les propriétés des logarithmes :
$$ y = \ln(1+\sqrt{x}) - \ln(1-\sqrt{x}) $$

Puis dériver directement :
$$ \frac{dy}{dx} = \frac{1}{1+\sqrt{x}} \cdot \frac{1}{2\sqrt{x}} - \frac{1}{1-\sqrt{x}} \cdot \frac{-1}{2\sqrt{x}} = \frac{1}{2\sqrt{x}} \left(\frac{1}{1+\sqrt{x}} + \frac{1}{1-\sqrt{x}}\right) $$

Réduction au même dénominateur :
$$ = \frac{1}{2\sqrt{x}} \cdot \frac{2}{1-x} = \frac{1}{(1-x)\sqrt{x}} $$

\textbf{Leçon :} Toujours chercher à simplifier l'expression avant de dériver. Les propriétés algébriques peuvent considérablement réduire la complexité du calcul.

\end{intuitionbox}

% ---

\section{Tests}

\begin{examplebox}[Simulation d'un lancer de dé]
On peut simuler $n$ lancers d'un dé équilibré à 6 faces en utilisant la bibliothèque \texttt{random} de Python.

\begin{codecell}
import random
import numpy
\end{codecell}

\begin{outputcell}
>> "vamonos"
\end{outputcell}

\end{examplebox}

\begin{definitionbox}[Variable Aléatoire]
Une variable aléatoire est une fonction qui associe un nombre réel à chaque résultat possible d'une expérience aléatoire.
\end{definitionbox}

\begin{codecell}
import math
import random

def poisson_knuth(lmbda: float) -> int:
  """
  Simule une variable aleatoire suivant une loi de Poisson()
  en utilisant l algorithme de Knuth.
  """
  L = math.exp(-lmbda)
  k = 0
  p = 1.0

  while p > L:
  k += 1
  p *= random.random()

  return k - 1
\end{codecell}



\end{document}