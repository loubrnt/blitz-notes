\documentclass{article}

% --- GESTION DES MARGES DE PAGE ---
\usepackage[a4paper, top=2.5cm, bottom=2.5cm, left=3cm, right=3cm]{geometry}

% --- PRÉAMBULE STANDARD ---
\usepackage[utf8]{inputenc}
\usepackage[T1]{fontenc}
\usepackage{lmodern}
\usepackage[french]{babel}
\usepackage{xcolor}
\usepackage{tcolorbox}
\usepackage{listings}
\usepackage{amsmath}
\usepackage{graphicx} % Requis pour inclure des images
\usepackage{amssymb} % Pour les symboles mathématiques comme \subseteq
\usepackage{sectsty} % Pour le style des sections
\usepackage{etoolbox} % Pour les conditions
\usepackage[dvipsnames]{xcolor}
\usetikzlibrary{
    matrix, 
    patterns.meta,
    calc,
    positioning,
    decorations.pathreplacing,
    trees
}

% Styles de hachures (inchangés)
\tikzset{
  red_hatch/.style={
    pattern={Lines[angle=45, line width=0.8pt, distance=4pt]}, 
    pattern color=red
  },
  blue_hatch/.style={
    pattern={Lines[angle=-45, line width=0.8pt, distance=4pt]}, 
    pattern color=blue
  },
  purple_hatch/.style={
    pattern={Lines[angle=45, line width=0.8pt, distance=4pt]}, 
    pattern color=red,
    postaction={
      pattern={Lines[angle=-45, line width=0.8pt, distance=4pt]}, 
      pattern color=blue
    }
  }
}

% --- BIBLIOTHÈQUES TCOLORBOX ---
\tcbuselibrary{listings, skins, breakable}

% --- GESTION DES LIENS HYPERTEXTE ---
\usepackage[colorlinks=true, linkcolor=black, urlcolor=blue]{hyperref}

% --- SOULIGNER LES TITRES ET SOUS-TITRES ---
\sectionfont{\underline}
\subsectionfont{\underline}

% --- PAGE DE GARDE AMÉLIORÉE ---
\makeatletter
\renewcommand{\maketitle}{%
\begin{titlepage}
\centering
\vspace*{\stretch{1.5}}
{\Huge \bfseries Mes Notes de Lecture\par}
\vspace{0.4cm}
\rule{0.8\linewidth}{0.4pt}
\vspace{1cm}
{\LARGE \bfseries Introduction à la Probabilité\par}
\vspace*{\stretch{2.5}}
{\Large \scshape Lou Brunet\par}
\vspace{0.5cm}
{\large \today\par}
\vspace*{\stretch{1}}
\end{titlepage}
}
\makeatother

% --- DÉFINITION DES COULEURS STYLE VS CODE ---
\definecolor{vscodeBlue}{HTML}{569CD6}
\definecolor{vscodeOrange}{HTML}{CE9178}
\definecolor{vscodeGreen}{HTML}{6A9955}
\definecolor{vscodePurple}{HTML}{C586C0}
\definecolor{vscodeGray}{HTML}{9B9B9B}
\definecolor{codeBackground}{HTML}{1E1E1E}

% --- CONFIGURATION DU STYLE LISTINGS ---
\lstdefinestyle{vscode}{
    language=Python,
    backgroundcolor=\color{codeBackground},
    basicstyle=\ttfamily\small\color{white},
    keywordstyle=\color{vscodeBlue},
    stringstyle=\color{vscodeOrange},
    commentstyle=\color{vscodeGreen},
    numberstyle=\tiny\color{vscodeGray},
    otherkeywords={self, True, False, None},
    keywordstyle=[2]\color{vscodePurple},
    showstringspaces=false,
    breaklines=true,
    frame=none,
    tabsize=4
}

% --- STYLE DE L'OUTPUT ---
\lstdefinestyle{outputstyle}{
    basicstyle=\ttfamily\small\color{white},
    numbers=left,
    numberstyle=\color{white}\def\thelstnumber{>>},
    numbersep=5pt,
    breaklines=true,
    frame=none
}

% --- DÉFINITION DES CELLULES DE CODE ET OUTPUT ---
\newtcblisting{codecell}{
  arc=2mm, boxrule=0.8pt, colframe=vscodeBlue, colback=codeBackground,
  coltitle=white, fonttitle=\bfseries, title=Code Python,
  listing only, listing options={style=vscode, basicstyle=\ttfamily\footnotesize\color{white}},
  left=6mm, right=6mm, top=3mm, bottom=3mm, boxsep=4mm
}
\newtcblisting{outputcell}{
  arc=2mm, boxrule=0.8pt, colframe=black!95, colback=black!85,
  fonttitle=\bfseries\color{white}, title=Output, coltitle=black!95,
  listing only, listing options={style=outputstyle, basicstyle=\ttfamily\footnotesize\color{white}},
  left=6mm, right=6mm, top=3mm, bottom=3mm, boxsep=4mm
}

% --- DÉFINITION DES COULEURS POUR DÉF/THÉO/PREUVE/INTUITION/EXEMPLE ---
\definecolor{defColor}{HTML}{1b1f3a}       % Bleu nuit pour les définitions
\definecolor{theoColor}{HTML}{53354a}      % Violet aubergine pour les théorèmes
\definecolor{proofColor}{HTML}{a64942}     % Rouge brique pour les preuves
\definecolor{intuitionColor}{HTML}{16A085} % Turquoise pour l'intuition
\definecolor{exampleColor}{HTML}{4a6982}   % Bleu ardoise pour les exemples
\definecolor{exoColor}{HTML}{1E8449}       % Vert
\definecolor{corrColor}{HTML}{7F8C8D}      % Gris

% --- DÉFINITION D'UN STYLE DE BASE COMMUN ---
\tcbset{
    baseboxstyle/.style={
        arc=2mm, boxrule=0.8pt,
        fontupper=\color{black}, left=6mm, right=6mm, top=3mm, bottom=3mm,
        boxsep=4mm, breakable
    }
}

% --- DÉFINITION DES CELLULES THÉORIQUES AVEC FOND DE COULEUR ---
\newtcolorbox{definitionbox}[1][]{
  baseboxstyle, colframe=defColor, colback=defColor!10, coltitle=defColor,
  fonttitle=\bfseries\color{white},
  title=Définition\ifstrempty{#1}{}{ : #1}
}
\newtcolorbox{theorembox}[1][]{
  baseboxstyle, colframe=theoColor, colback=theoColor!10, coltitle=theoColor,
  fonttitle=\bfseries\color{white},
  title=Théorème\ifstrempty{#1}{}{ : #1}
}
\newtcolorbox{proofbox}[1][]{
  baseboxstyle, colframe=proofColor, colback=proofColor!10, coltitle=proofColor,
  fonttitle=\bfseries\color{white},
  title=Preuve\ifstrempty{#1}{}{ : #1}
}
\newtcolorbox{intuitionbox}[1][]{
  baseboxstyle, colframe=intuitionColor, colback=intuitionColor!10, coltitle=intuitionColor,
  fonttitle=\bfseries\color{white},
  title=Intuition\ifstrempty{#1}{}{ : #1}
}
\newtcolorbox{examplebox}[1][]{
  baseboxstyle, colframe=exampleColor, colback=exampleColor!10, coltitle=exampleColor,
  fonttitle=\bfseries\color{white},
  title=Exemple\ifstrempty{#1}{}{ : #1}
}

% --- DÉFINITION DES CELLULES EXERCICE (GRIS) ET CORRECTION (VERT) ---
\newtcolorbox{exercicebox}[1][]{
    baseboxstyle, colframe=corrColor, colback=corrColor!10, coltitle=corrColor,
    fonttitle=\bfseries\color{white},
    title=Exercice\ifstrempty{#1}{}{ : #1}
}

\newtcolorbox{correctionbox}[1][]{
    baseboxstyle, colframe=exoColor, colback=exoColor!10, coltitle=exoColor,
    fonttitle=\bfseries\color{white},
    title=Correction\ifstrempty{#1}{}{ : #1}
}

% --- DÉFINITION D'UNE CELLULE INTERLUDE ---
\definecolor{interludeColor}{HTML}{D35400} 

\newtcolorbox{interludebox}[1][]{
  baseboxstyle, colframe=interludeColor, colback=interludeColor!10, coltitle=interludeColor,
  fonttitle=\bfseries\color{white},
  title=Interlude\ifstrempty{#1}{}{ : #1}
}

% =============================================
% --- CORPS DU DOCUMENT ---
% =============================================
% =============================================
% --- CORPS DU DOCUMENT ---
% =============================================
\begin{document}

\maketitle

\newpage
\tableofcontents 
\newpage

\newpage
\section{Probabilités et Dénombrement}

\subsection{Concepts fondamentaux}

Avant de pouvoir calculer des probabilités, il est essentiel d'établir un vocabulaire commun pour décrire les expériences aléatoires.

\begin{intuitionbox}[Nécessité d'un Cadre Formel]
Avant de calculer des probabilités, il est crucial de définir les règles du jeu :

\textbf{Qu'est-ce qui peut arriver ?}

On définit l'ensemble de tous les résultats possibles de l'expérience.

\textbf{À quoi s'intéresse-t-on ?} 

On identifie les sous-ensembles de résultats spécifiques qui nous intéressent.

Ces deux idées nous conduisent aux notions d'Univers et d'Événement, qui sont les piliers de toute théorie des probabilités.
\end{intuitionbox}

Cette intuition se traduit formellement par deux définitions clés :

\begin{definitionbox}[Concepts Fondamentaux]
\textbf{Univers (ou Espace Échantillon), $S$ :} 

L'ensemble de tous les résultats possibles d'une expérience aléatoire.

\textbf{Événement, $A$ :} 

Un sous-ensemble de l'univers ($A \subseteq S$). C'est un ensemble de résultats auxquels on s'intéresse.
\end{definitionbox}

Un exemple simple permet de solidifier ces concepts :

\begin{examplebox}[Univers et Événement]
Pour l'expérience du "lancer d'un dé à six faces" :

L'\textbf{univers} est $S = \{1, 2, 3, 4, 5, 6\}$.

"Obtenir un nombre impair" est un événement, représenté par le sous-ensemble $A = \{1, 3, 5\}$.
\end{examplebox}

\subsection{Définition Naïve de la Probabilité}

Pour de nombreuses expériences simples, comme lancer un dé non pipé, chaque résultat possible est "équiprobable". Cette hypothèse est la base de la première définition formelle de la probabilité.

\begin{definitionbox}[Probabilité Naïve]
Pour une expérience où chaque issue dans un espace échantillon fini $S$ est équiprobable, la probabilité d'un événement $A$ est le rapport du nombre d'issues favorables à $A$ sur le nombre total d'issues :
$$ P(A) = \frac{\text{Nombre d'issues favorables}}{\text{Nombre total d'issues}} = \frac{|A|}{|S|} $$
\end{definitionbox}

Appliquons cette formule à quelques cas classiques :

\begin{examplebox}[Applications de la définition naïve]
\begin{enumerate}
    \item \textbf{Lancer une pièce équilibrée :}
    L'espace échantillon est $S = \{\text{Pile, Face}\}$, donc $|S| = 2$.
    Si l'événement $A$ est "obtenir Pile", alors $A = \{\text{Pile}\}$ et $|A| = 1$.
    La probabilité est $P(A) = \frac{1}{2}$.

    \item \textbf{Lancer un dé à six faces non pipé :}
    L'espace échantillon est $S = \{1, 2, 3, 4, 5, 6\}$, donc $|S| = 6$.
    Si l'événement $B$ est "obtenir un nombre pair", alors $B = \{2, 4, 6\}$ et $|B| = 3$.
    La probabilité est $P(B) = \frac{3}{6} = \frac{1}{2}$.

    \item \textbf{Tirer une carte d'un jeu de 52 cartes :}
    L'espace échantillon $S$ contient 52 cartes, donc $|S| = 52$.
    Si l'événement $C$ est "tirer un Roi", il y a 4 Rois dans le jeu, donc $|C| = 4$.
    La probabilité est $P(C) = \frac{4}{52} = \frac{1}{13}$.
\end{enumerate}
\end{examplebox}

\subsection{Permutations (Arrangements)}

Le dénombrement, qui est l'art de compter les tailles $|A|$ et $|S|$, est fondamental pour appliquer la définition naïve. Le premier outil que nous verrons est la permutation, qui compte les arrangements \textbf{ordonnés}.

\begin{definitionbox}[Permutation de $k$ objets parmi $n$]
Le nombre de façons d'arranger $k$ objets choisis parmi $n$ objets distincts (où l'ordre compte et il n'y a pas de répétition) est noté $P(n, k)$ ou $A_n^k$ et est défini par :
$$ P(n, k) = \frac{n!}{(n-k)!} $$
où $n!$ est la factorielle de $n$, et par convention $0! = 1$.
\end{definitionbox}

Cette formule peut sembler abstraite, mais elle provient d'un raisonnement logique simple par "cases" :

\begin{intuitionbox}[Permutations de $k$ parmi $n$]
Pour placer $k$ objets dans un ordre spécifique en les choisissant parmi $n$ objets disponibles, on a $n$ choix pour la première position, $(n-1)$ choix pour la deuxième, ..., et $(n-k+1)$ choix pour la $k$-ième position. Cela donne $n \times (n-1) \times \cdots \times (n-k+1)$ arrangements. Ce produit contient $k$ termes. Il est égal à $\frac{n!}{(n-k)!}$, car cela revient à diviser la suite complète $n!$ par les facteurs non utilisés $(n-k) \times (n-k-1) \times \cdots \times 1$.
\end{intuitionbox}

Voyons une application classique de ce principe :

\begin{examplebox}[Permutations de $k$ parmi $n$]
\textbf{Podium d'une course :} Une course réunit 8 coureurs. Combien y a-t-il de podiums (1er, 2e, 3e) possibles ?

On cherche le nombre de façons d'ordonner 3 coureurs parmi 8 : $P(8, 3)$. 
$$ P(8, 3) = \frac{8!}{(8-3)!} = \frac{8!}{5!} = 8 \times 7 \times 6 = 336 $$
Il y a 336 podiums possibles.
\end{examplebox}

\subsection{Le Coefficient Binomial}

Que se passe-t-il si l'ordre ne compte pas ? Au lieu de compter des podiums, nous voulons compter des comités. C'est le rôle du coefficient binomial.

\begin{theorembox}[Formule du Coefficient Binomial]
Le nombre de façons de choisir $k$ objets parmi un ensemble de $n$ objets distincts (sans remise et sans ordre) est donné par le coefficient binomial :
$$ \binom{n}{k} = \frac{n!}{k!(n-k)!} $$
\end{theorembox}

% NOUVEAU :
La preuve de cette formule repose sur un argument combinatoire élégant : nous allons compter la même chose (les permutations) de deux façons différentes.
% FIN NOUVEAU

\newpage

\begin{proofbox}
Considérons le nombre de permutations de $k$ objets parmi $n$, noté $P(n,k)$.
\begin{enumerate}
    \item \textbf{Méthode 1 :} Par définition (vue ci-dessus), nous savons que $P(n,k) = \frac{n!}{(n-k)!}$.
    
    \item \textbf{Méthode 2 :} Nous pouvons construire une telle permutation en deux étapes successives :
    \begin{itemize}
        \item D'abord, \textbf{choisir un sous-ensemble} de $k$ objets parmi $n$ (l'ordre ne compte pas). C'est le nombre que nous cherchons, notons-le $\binom{n}{k}$.
        \item Ensuite, \textbf{ordonner} ces $k$ objets choisis. Il y a $k!$ façons de les arranger.
    \end{itemize}
    Le nombre total de permutations est donc le produit de ces étapes : $P(n,k) = \binom{n}{k} \times k!$.
\end{enumerate}
En égalisant les deux méthodes, on obtient :
\[ \binom{n}{k} \cdot k! = \frac{n!}{(n-k)!} \]
En divisant par $k!$, on trouve bien la formule :
\[ \binom{n}{k} = \frac{n!}{k!(n-k)!} \]
\end{proofbox}

% NOUVEAU :
L'intuition visuelle derrière cette preuve est de voir comment chaque "choix" (une colonne du tableau) génère $k!$ "ordres" (les lignes de cette colonne).
% FIN NOUVEAU

\begin{intuitionbox}
Pour rendre cela concret, voici le cas $\binom{5}{3}$.  
Il y a 10 sous-ensembles de 3 éléments parmi $\{a,b,c,d,e\}$. Chacun donne lieu à $3! = 6$ permutations.  
Le tableau ci-dessous montre \textbf{toutes les 60 permutations}, regroupées par sous-ensemble :

\vspace{3mm}

\begin{center}
\small
\renewcommand{\arraystretch}{0.9}
\setlength{\tabcolsep}{2pt}
\begin{tabular}{|c|c|c|c|c|c|c|c|c|c|}
\hline
\textbf{$\{a,b,c\}$} & \textbf{$\{a,b,d\}$} & \textbf{$\{a,b,e\}$} & \textbf{$\{a,c,d\}$} & \textbf{$\{a,c,e\}$} & \textbf{$\{a,d,e\}$} & \textbf{$\{b,c,d\}$} & \textbf{$\{b,c,e\}$} & \textbf{$\{b,d,e\}$} & \textbf{$\{c,d,e\}$} \\
\hline
$abc$ & $abd$ & $abe$ & $acd$ & $ace$ & $ade$ & $bcd$ & $bce$ & $bde$ & $cde$ \\
\hline
$acb$ & $adb$ & $aeb$ & $adc$ & $aec$ & $aed$ & $bdc$ & $bec$ & $bed$ & $ced$ \\
\hline
$bac$ & $bad$ & $bae$ & $cad$ & $cae$ & $dae$ & $cbd$ & $ceb$ & $dbe$ & $dce$ \\
\hline
$bca$ & $bda$ & $bea$ & $cda$ & $cea$ & $dea$ & $cdb$ & $ceb$ & $deb$ & $dec$ \\
\hline
$cab$ & $dab$ & $eab$ & $dac$ & $eac$ & $ead$ & $dbc$ & $ebc$ & $edb$ & $ecd$ \\
\hline
$cba$ & $dba$ & $eba$ & $dca$ & $eca$ & $eda$ & $dcb$ & $ebc$ & $edb$ & $edc$ \\
\hline
\end{tabular}
\end{center}

\vspace{3mm}

\smallskip

Chaque colonne correspond à \textbf{un seul et même choix non ordonné} (par exemple $\{a,b,c\}$), mais à 6 listes différentes selon l’ordre.  
Ainsi, pour obtenir le nombre de \textit{choix non ordonnés}, on divise le nombre total de listes ($60$) par le nombre d’ordres par groupe ($6$) :
\[
\binom{5}{3} = \frac{60}{6} = 10.
\]
\end{intuitionbox}

L'application la plus directe est le tirage d'un groupe où l'ordre n'importe pas :

\begin{examplebox}[Utilisation du Coefficient Binomial]
    \textbf{Comité d'étudiants :} De combien de manières peut-on former un comité de 3 étudiants à partir d'une classe de 10 ? L'ordre ne compte pas.
    $$ \binom{10}{3} = \frac{10!}{3!(10-3)!} = \frac{10 \times 9 \times 8}{3 \times 2 \times 1} = 120 \text{ comités possibles.} $$
\end{examplebox}

\newpage

\subsection{Identité de Vandermonde}

Les coefficients binomiaux obéissent à de nombreuses identités. L'identité de Vandermonde est l'une des plus utiles, car elle montre comment décomposer un problème de comptage complexe en sous-problèmes.

\begin{theorembox}[Identité de Vandermonde]
Cette identité offre une relation remarquable entre les coefficients binomiaux. Pour des entiers non négatifs $m, n$ et $k$, on a :
$$ \binom{m+n}{k} = \sum_{j=0}^{k} \binom{m}{j} \binom{n}{k-j} $$
\end{theorembox}

% NOUVEAU :
La preuve la plus intuitive est une "preuve par l'histoire" (proof by story), qui consiste à trouver un scénario de dénombrement que les deux côtés de l'équation résolvent.
% FIN NOUVEAU

\begin{proofbox}[Preuve combinatoire]
Imaginons un groupe composé de $m$ hommes et $n$ femmes. Nous souhaitons former un comité de $k$ personnes. Nous allons compter le nombre de comités possibles de deux façons.

\textbf{Côté gauche : $\binom{m+n}{k}$}
Le groupe total contient $m+n$ personnes. Le nombre de façons de choisir un comité de $k$ personnes parmi ce total est, par définition, $\binom{m+n}{k}$.

\textbf{Côté droit : $\sum_{j=0}^{k} \binom{m}{j} \binom{n}{k-j}$}
Nous pouvons compter le même nombre en conditionnant sur le nombre d'hommes (noté $j$) dans le comité. Un comité de $k$ personnes doit contenir $j$ hommes ET $k-j$ femmes, où $j$ peut aller de $0$ à $k$.
\begin{itemize}
    \item Pour $j=0$ : Choisir 0 homme ($\binom{m}{0}$) ET $k$ femmes ($\binom{n}{k}$).
    \item Pour $j=1$ : Choisir 1 homme ($\binom{m}{1}$) ET $k-1$ femmes ($\binom{n}{k-1}$).
    \item ...
    \item Pour $j=k$ : Choisir $k$ hommes ($\binom{m}{k}$) ET 0 femme ($\binom{n}{0}$).
\end{itemize}
Puisque ces cas (0 homme, 1 homme, etc.) sont mutuellement exclusifs, le nombre total de comités est la somme de toutes ces possibilités :
\[ \sum_{j=0}^{k} \binom{m}{j} \binom{n}{k-j} \]
Puisque les deux côtés comptent exactement la même chose (le nombre total de comités), ils doivent être égaux.
\end{proofbox}

Vérifions cette identité avec un exemple numérique concret, en reprenant l'analogie du comité :

\begin{examplebox}[Application de l'Identité de Vandermonde]
On veut former un comité de 3 personnes ($k=3$) à partir d'un groupe de 5 hommes ($m=5$) et 4 femmes ($n=4$).

\textbf{Méthode directe (côté gauche) :}
On choisit 3 personnes parmi les $5+4=9$ au total.
$$ \binom{9}{3} = \frac{9 \times 8 \times 7}{3 \times 2 \times 1} = 84 $$

\textbf{Méthode par cas (côté droit) :}
La somme est $\binom{5}{0}\binom{4}{3} + \binom{5}{1}\binom{4}{2} + \binom{5}{2}\binom{4}{1} + \binom{5}{3}\binom{4}{0} = 84$. Les deux méthodes donnent bien le même résultat.
\end{examplebox}

\newpage

\subsection{Bose-Einstein (Étoiles et Bâtons)}

Jusqu'à présent, nous avons supposé un "tirage sans remise". La statistique de Bose-Einstein, ou plus visuellement la méthode des "étoiles et bâtons", s'attaque au problème du \textbf{tirage avec remise} où l'ordre ne compte pas.

\begin{theorembox}[Combinaisons avec répétition]
Le nombre de façons de distribuer $k$ objets indiscernables dans $n$ boîtes discernables (ou de choisir $k$ objets parmi $n$ avec remise, où l'ordre ne compte pas) est donné par la formule :
$$ \binom{n+k-1}{k} = \binom{n+k-1}{n-1} $$
\end{theorembox}

% NOUVEAU :
La preuve de cette formule est l'un des résultats les plus élégants du dénombrement. L'astuce consiste à transformer le problème de distribution en un problème d'arrangement de symboles.
% FIN NOUVEAU

\begin{proofbox}[Par les "Étoiles et Bâtons"]
Nous cherchons à distribuer $k$ objets indiscernables ($\star$) dans $n$ boîtes discernables.
Nous pouvons représenter n'importe quelle distribution comme une séquence de symboles. Nous avons besoin de $k$ étoiles (les objets) et de $n-1$ bâtons ($|$) pour servir de séparateurs entre les $n$ boîtes.

Par exemple, pour distribuer $k=7$ étoiles dans $n=4$ boîtes, la séquence :
$$ \star\star\star \mid \star \mid \mid \star\star\star $$
correspond à : 3 étoiles dans la boîte 1, 1 étoile dans la boîte 2, 0 étoile dans la boîte 3 (l'espace entre deux bâtons), et 3 étoiles dans la boîte 4.

Chaque arrangement unique de ces symboles correspond à une distribution unique. Le problème revient donc à trouver le nombre de façons d'arranger ces $k$ étoiles et ces $n-1$ bâtons.

Nous avons un total de $n+k-1$ positions à remplir. Le nombre de façons de le faire est simplement le nombre de manières de choisir les $k$ positions pour les étoiles (les autres positions étant automatiquement remplies par des bâtons).
C'est exactement :
$$ \binom{n+k-1}{k} $$
(Ce qui est aussi égal à $\binom{n+k-1}{n-1}$, le nombre de façons de choisir les positions des $n-1$ bâtons).
\end{proofbox}

C'est la méthode parfaite pour tout problème de distribution d'objets identiques :

\begin{examplebox}[Distribution de biens identiques]
De combien de manières peut-on distribuer 10 croissants identiques à 4 enfants ?

Ici, $k=10$ (les croissants, objets indiscernables) et $n=4$ (les enfants, boîtes discernables).
Le nombre de distributions possibles est :
$$ \binom{4+10-1}{10} = \binom{13}{10} = \binom{13}{3} = \frac{13 \times 12 \times 11}{3 \times 2 \times 1} = 13 \times 2 \times 11 = 286 $$
Il y a 286 façons de distribuer les croissants.
\end{examplebox}

\subsection{Principe d'Inclusion-Exclusion}

Comment compter le nombre d'éléments dans l'union de plusieurs ensembles ? Si on additionne simplement leurs tailles, on compte les intersections plusieurs fois. Le principe d'inclusion-exclusion corrige systématiquement ce sur-comptage.

\begin{theorembox}[Principe d'Inclusion-Exclusion pour 3 ensembles]
Pour trois ensembles finis $A$, $B$ et $C$, le nombre d'éléments dans leur union est donné par :
$$ |A \cup B \cup C| = |A| + |B| + |C| - |A \cap B| - |A \cap C| - |B \cap C| + |A \cap B \cap C| $$
\end{theorembox}

% NOUVEAU :
La preuve pour 3 ensembles se fait en appliquant la formule pour 2 ensembles de manière répétée.
% FIN NOUVEAU

\begin{proofbox}
Nous utilisons la formule pour deux ensembles, $|X \cup Y| = |X| + |Y| - |X \cap Y|$, de manière imbriquée.
Posons $X = A \cup B$ et $Y = C$.
\begin{align*}
|A \cup B \cup C| &= |(A \cup B) \cup C| \\
&= |A \cup B| + |C| - |(A \cup B) \cap C|
\end{align*}
Nous devons maintenant développer les deux termes compliqués :
\begin{enumerate}
    \item $|A \cup B| = |A| + |B| - |A \cap B|$
    \item Par distributivité de l'intersection sur l'union, $(A \cup B) \cap C = (A \cap C) \cup (B \cap C)$.
\end{enumerate}
Appliquons la formule pour 2 ensembles à ce deuxième terme :
\[ |(A \cap C) \cup (B \cap C)| = |A \cap C| + |B \cap C| - |(A \cap C) \cap (B \cap C)| \]
Ce qui se simplifie en $|A \cap C| + |B \cap C| - |A \cap B \cap C|$.

Finalement, en substituant tout dans l'équation de départ :
\begin{align*}
|A \cup B \cup C| &= \underbrace{(|A| + |B| - |A \cap B|)}_{|A \cup B|} + |C| \\
                 &\quad - \underbrace{(|A \cap C| + |B \cap C| - |A \cap B \cap C|)}_{|(A \cup B) \cap C|}
\end{align*}
En réarrangeant les termes, on obtient la formule voulue :
\[ |A| + |B| + |C| - |A \cap B| - |A \cap C| - |B \cap C| + |A \cap B \cap C| \]
\end{proofbox}

La formule devient évidente lorsque l'on utilise un diagramme de Venn pour visualiser le sur-comptage et sa correction.

\begin{intuitionbox}[Visualisation avec 3 ensembles]
Le principe d'inclusion-exclusion permet de compter le nombre d'éléments dans une union d'ensembles sans double-comptage. Pour comprendre intuitivement pourquoi on ajoute et soustrait alternativement, considérons trois ensembles $A$, $B$ et $C$ :

\begin{center}
\begin{tikzpicture}[set/.style = {draw,
    circle,
    minimum size = 6cm,
    fill=Rhodamine,
    opacity = 0.4,
    text opacity = 1}]
 
\node (A) [set] {$A$};
\node (B) at (60:4cm) [set] {$B$};
\node (C) at (0:4cm) [set] {$C$};
 
\node at (barycentric cs:A=1,B=1) [left] {$X$};
\node at (barycentric cs:A=1,C=1) [below] {$Y$};
\node at (barycentric cs:B=1,C=1) [right] {$Z$};
\node at (barycentric cs:A=1,B=1,C=1) [] {$T$};
 
\end{tikzpicture}
\end{center}

\textbf{Le problème :} Si on additionne simplement $|A| + |B| + |C|$, on compte certaines zones plusieurs fois :
\begin{itemize}
    \item Les intersections deux à deux ($X$, $Y$, $Z$) sont comptées \textbf{deux fois}
    \item L'intersection triple ($T$) est comptée \textbf{trois fois}
\end{itemize}

\textbf{La solution :} On corrige en soustrayant les intersections deux à deux, mais alors l'intersection triple est comptée :
\begin{itemize}
    \item $+3$ fois dans la somme initiale
    \item $-3$ fois dans la soustraction des intersections deux à deux (car elle appartient à chacune)
    \item Donc $0$ fois au total ! Il faut la rajouter.
\end{itemize}

D'où la formule : $|A \cup B \cup C| = |A| + |B| + |C| - |A \cap B| - |A \cap C| - |B \cap C| + |A \cap B \cap C|$
\end{intuitionbox}

Ce que nous avons fait visuellement pour 3 ensembles peut être généralisé par récurrence à $n$ ensembles. La formule générale suit le même principe d'alternance des signes :

\begin{theorembox}[Principe d'Inclusion-Exclusion généralisé]
Pour $n$ ensembles finis $A_1, A_2, \dots, A_n$, on a :
\begin{align*}
|A_1 \cup A_2 \cup \cdots \cup A_n| = & \sum_{i=1}^n |A_i| \\
& - \sum_{1 \leq i < j \leq n} |A_i \cap A_j| \\
& + \sum_{1 \leq i < j < k \leq n} |A_i \cap A_j \cap A_k| \\
& - \cdots \\
& + (-1)^{n+1} |A_1 \cap A_2 \cap \cdots \cap A_n|
\end{align*}
Ce qui s'écrit plus compactement :
$$ \left| \bigcup_{i=1}^n A_i \right| = \sum_{k=1}^n (-1)^{k+1} \sum_{1 \leq i_1 < i_2 < \cdots < i_k \leq n} |A_{i_1} \cap A_{i_2} \cap \cdots \cap A_{i_k}| $$
\end{theorembox}

% NOUVEAU :
La preuve formelle que cette formule gigantesque fonctionne est fascinante. Il suffit de montrer que n'importe quel élément $x$ de l'union, peu importe à combien d'ensembles il appartient, est compté \textbf{exactement une fois} au final.
% FIN NOUVEAU

\begin{proofbox}[Preuve par comptage d'un élément]
Considérons un élément $x$ qui appartient à exactement $k$ ensembles parmi les $n$ ensembles $A_1, \ldots, A_n$ (où $k \ge 1$). Nous devons montrer que $x$ est compté exactement 1 fois par la formule.

Analysons combien de fois $x$ est compté dans chaque somme de la formule :
\begin{itemize}
    \item \textbf{Première somme ($\sum |A_i|$)} : $x$ est dans $k$ ensembles, donc il est ajouté $k$ fois. Le nombre de fois est $\binom{k}{1}$.
    
    \item \textbf{Deuxième somme ($-\sum |A_i \cap A_j|$)} : $x$ est compté (et soustrait) pour chaque \textit{paire} d'ensembles auxquels il appartient. Comme il appartient à $k$ ensembles, il y a $\binom{k}{2}$ telles paires.
    
    \item \textbf{Troisième somme ($+\sum |A_i \cap A_j \cap A_k|$)} : $x$ est ajouté pour chaque \textit{triplet} d'ensembles auxquels il appartient. Il y en a $\binom{k}{3}$.
    
    \item \textbf{Et ainsi de suite...}
\end{itemize}
Au total, l'élément $x$ est compté :
$$ \text{Total} = \binom{k}{1} - \binom{k}{2} + \binom{k}{3} - \cdots + (-1)^{k-1}\binom{k}{k} \text{ fois.} $$
Pour évaluer cette somme, rappelons l'identité fondamentale du binôme de Newton :
$$ (1 + x)^k = \sum_{j=0}^{k} \binom{k}{j} x^j = \binom{k}{0} + \binom{k}{1}x + \binom{k}{2}x^2 + \cdots $$
Si nous posons $x = -1$, nous obtenons :
$$ (1-1)^k = 0 = \binom{k}{0} - \binom{k}{1} + \binom{k}{2} - \binom{k}{3} + \cdots + (-1)^k\binom{k}{k} $$
Sachant que $\binom{k}{0} = 1$, on a :
$$ 0 = 1 - \left( \binom{k}{1} - \binom{k}{2} + \binom{k}{3} - \cdots + (-1)^{k-1}\binom{k}{k} \right) $$
En réarrangeant, on trouve :
$$ 1 = \binom{k}{1} - \binom{k}{2} + \binom{k}{3} - \cdots + (-1)^{k-1}\binom{k}{k} $$
Cela prouve que n'importe quel élément de l'union est compté exactement une fois.
\end{proofbox}

Ce principe est très utile en probabilité, car il permet de calculer $P(A \cup B \cup \dots)$ en se basant sur les probabilités des intersections, qui sont souvent plus faciles à trouver.

\begin{examplebox}[Application probabiliste]
On lance trois dés équilibrés. Quelle est la probabilité d'obtenir au moins un 6 ?

\textbf{Solution avec inclusion-exclusion :}

Soit $A$ = "le premier dé montre 6", $B$ = "le deuxième dé montre 6", $C$ = "le troisième dé montre 6".

On veut $P(A \cup B \cup C)$.

\begin{align*}
P(A \cup B \cup C) &= P(A) + P(B) + P(C) \\
&\quad - P(A \cap B) - P(A \cap C) - P(B \cap C) \\
&\quad + P(A \cap B \cap C) \\
&= \frac{1}{6} + \frac{1}{6} + \frac{1}{6} - \frac{1}{36} - \frac{1}{36} - \frac{1}{36} + \frac{1}{216} \\
&= \frac{3}{6} - \frac{3}{36} + \frac{1}{216} = \frac{1}{2} - \frac{1}{12} + \frac{1}{216} \\
&= \frac{108 - 18 + 1}{216} = \frac{91}{216} \approx 0.421
\end{align*}

\textbf{Vérification par la méthode complémentaire :}

La probabilité de n'obtenir aucun 6 est $\left(\frac{5}{6}\right)^3 = \frac{125}{216}$, donc la probabilité d'au moins un 6 est $1 - \frac{125}{216} = \frac{91}{216}$.
\end{examplebox}

\subsection{Exercices}

Cette série d'exercices vise à renforcer votre compréhension des concepts fondamentaux du dénombrement et de la probabilité naïve. La difficulté augmente progressivement.

% --- Concepts de Base et Probabilité Naïve ---

\begin{exercicebox}[Exercice 1 : Univers et Événements]
On lance deux dés à 6 faces, un rouge et un bleu.
\begin{enumerate}
    \item Décrivez l'univers $S$ de cette expérience. Quelle est sa taille $|S|$ ?
    \item Soit $A$ l'événement "la somme des dés est égale à 7". Listez les issues appartenant à $A$. Calculez $P(A)$.
    \item Soit $B$ l'événement "le dé rouge montre un 3". Listez les issues appartenant à $B$. Calculez $P(B)$.
    \item Décrivez l'événement $A \cap B$ et calculez sa probabilité.
\end{enumerate}
\end{exercicebox}

\begin{exercicebox}[Exercice 2 : Tirage de Cartes (Prob. Naïve)]
On tire une carte au hasard d'un jeu standard de 52 cartes.
\begin{enumerate}
    \item Quelle est la probabilité de tirer un Roi ?
    \item Quelle est la probabilité de tirer une carte rouge (Cœur ou Carreau) ?
    \item Quelle est la probabilité de tirer une figure (Valet, Dame, Roi) ?
    \item Quelle est la probabilité de tirer un As rouge ?
\end{enumerate}
\end{exercicebox}

\begin{exercicebox}[Exercice 3 : Urne Simple (Prob. Naïve)]
Une urne contient 5 boules rouges, 3 boules bleues et 2 boules vertes. On tire une boule au hasard.
\begin{enumerate}
    \item Quelle est la probabilité qu'elle soit bleue ?
    \item Quelle est la probabilité qu'elle ne soit pas verte ?
\end{enumerate}
\end{exercicebox}

% --- Permutations ---

\begin{exercicebox}[Exercice 4 : Anagrammes (Permutation Simple)]
Combien d'anagrammes distinctes peut-on former avec les lettres du mot "MATHS" ?
\end{exercicebox}

\begin{exercicebox}[Exercice 5 : Course (Arrangement)]
Dix athlètes participent à une course. Combien y a-t-il de classements possibles pour les 3 premières places (médaille d'or, d'argent, de bronze) ?
\end{exercicebox}

\begin{exercicebox}[Exercice 6 : Anagrammes (Permutation avec Répétition)]
Combien d'anagrammes distinctes peut-on former avec les lettres du mot "PROBABILITE" ?
\end{exercicebox}

% --- Combinaisons ---

\begin{exercicebox}[Exercice 7 : Choix d'un Comité (Combinaison)]
Une classe compte 15 étudiants. De combien de manières peut-on choisir un comité de 4 étudiants ?
\end{exercicebox}

\begin{exercicebox}[Exercice 8 : Mains de Poker (Combinaison)]
Dans un jeu de 52 cartes, combien de "mains" de 5 cartes différentes peut-on former ?
\end{exercicebox}

\begin{exercicebox}[Exercice 9 : Comité Mixte (Combinaison)]
À partir d'un groupe de 6 hommes et 4 femmes, combien de comités de 3 personnes peut-on former contenant exactement 2 hommes et 1 femme ?
\end{exercicebox}

\begin{exercicebox}[Exercice 10 : Probabilité avec Combinaisons]
On tire simultanément 3 cartes d'un jeu de 52 cartes. Quelle est la probabilité d'obtenir exactement 2 Rois ?
\end{exercicebox}

% --- Combinaisons avec Répétition (Étoiles et Bâtons) ---

\begin{exercicebox}[Exercice 11 : Distribution de Bonbons (Étoiles et Bâtons)]
De combien de manières peut-on distribuer 8 bonbons identiques à 3 enfants ? (Certains enfants peuvent ne rien recevoir).
\end{exercicebox}

\begin{exercicebox}[Exercice 12 : Solutions d'Équation (Étoiles et Bâtons)]
Combien y a-t-il de solutions entières non négatives ($x_i \ge 0$) à l'équation $x_1 + x_2 + x_3 + x_4 = 10$ ?
\end{exercicebox}

\begin{exercicebox}[Exercice 13 : Distribution avec Minimum (Étoiles et Bâtons avec Contrainte)]
De combien de manières peut-on distribuer 12 pommes identiques à 4 enfants, si chaque enfant doit recevoir au moins une pomme ?
\end{exercicebox}

% --- Principe d'Inclusion-Exclusion ---

\begin{exercicebox}[Exercice 14 : Divisibilité (Inclusion-Exclusion 2 Ensembles)]
Parmi les entiers de 1 à 100, combien sont divisibles par 2 OU par 3 ?
\end{exercicebox}

\begin{exercicebox}[Exercice 15 : Langues (Inclusion-Exclusion 2 Ensembles)]
Dans un groupe de 50 étudiants, 30 étudient l'anglais, 25 étudient l'espagnol et 10 étudient les deux langues. Combien d'étudiants étudient au moins une de ces deux langues ? Combien n'en étudient aucune ?
\end{exercicebox}

\begin{exercicebox}[Exercice 16 : Divisibilité (Inclusion-Exclusion 3 Ensembles)]
Parmi les entiers de 1 à 100, combien sont divisibles par 2, 3 OU 5 ?
\end{exercicebox}

% --- Problèmes Combinés et Plus Difficiles ---

\begin{exercicebox}[Exercice 17 : Chemins sur un Grillage (Combinaison)]
Sur un grillage, combien y a-t-il de chemins pour aller du point (0,0) au point (4,3) en se déplaçant uniquement vers la droite (D) ou vers le haut (H) ?
\end{exercicebox}

\begin{exercicebox}[Exercice 18 : Probabilité Hypergéométrique]
Une urne contient 7 boules blanches et 5 boules noires. On tire successivement et sans remise 4 boules. Quelle est la probabilité d'obtenir 2 blanches et 2 noires ?
\end{exercicebox}

\begin{exercicebox}[Exercice 19 : Arrangement Circulaire]
De combien de manières 6 personnes peuvent-elles s'asseoir autour d'une table ronde ? (Deux arrangements sont considérés identiques si chaque personne a les mêmes voisins).
\end{exercicebox}

\begin{exercicebox}[Exercice 20 : Problème des Dérangements (Inclusion-Exclusion)]
Quatre lettres sont adressées à quatre personnes différentes, avec les enveloppes correspondantes. On met chaque lettre au hasard dans une enveloppe. Quelle est la probabilité qu'\textit{aucune} lettre ne soit dans la bonne enveloppe ?
\end{exercicebox}



\subsection{Corrections des Exercices}

% --- Corrections : Concepts de Base et Probabilité Naïve ---

\begin{correctionbox}[Correction Exercice 1 : Univers et Événements]
1) L'univers $S$ est l'ensemble de toutes les paires $(r, b)$ où $r$ est le résultat du dé rouge et $b$ celui du dé bleu. $S = \{ (1,1), (1,2), \dots, (1,6), (2,1), \dots, (6,6) \}$. La taille de l'univers est $|S| = 6 \times 6 = 36$.

2) L'événement $A$ (somme égale à 7) est $A = \{ (1,6), (2,5), (3,4), (4,3), (5,2), (6,1) \}$. Il y a $|A|=6$ issues favorables. La probabilité est $P(A) = |A|/|S| = 6/36 = 1/6$.

3) L'événement $B$ (dé rouge montre 3) est $B = \{ (3,1), (3,2), (3,3), (3,4), (3,5), (3,6) \}$. Il y a $|B|=6$ issues favorables. La probabilité est $P(B) = |B|/|S| = 6/36 = 1/6$.

4) L'événement $A \cap B$ est l'ensemble des issues où la somme est 7 ET le dé rouge est 3. La seule issue possible est $(3,4)$. Donc $A \cap B = \{ (3,4) \}$. La probabilité est $P(A \cap B) = |A \cap B|/|S| = 1/36$.
\end{correctionbox}

\begin{correctionbox}[Correction Exercice 2 : Tirage de Cartes (Prob. Naïve)]
Le nombre total d'issues est $|S| = 52$.

a) Il y a 4 Rois. $P(\text{Roi}) = 4/52 = 1/13$.

b) Il y a 26 cartes rouges (13 Cœurs + 13 Carreaux). $P(\text{Rouge}) = 26/52 = 1/2$.

c) Il y a 12 figures (4 Valets + 4 Dames + 4 Rois). $P(\text{Figure}) = 12/52 = 3/13$.

d) Il y a 2 As rouges (As de Cœur, As de Carreau). $P(\text{As Rouge}) = 2/52 = 1/26$.
\end{correctionbox}

\begin{correctionbox}[Correction Exercice 3 : Urne Simple (Prob. Naïve)]
Le nombre total de boules est $5+3+2 = 10$.

a) Il y a 3 boules bleues. $P(\text{Bleue}) = 3/10$.

b) L'événement "ne pas être verte" est le complémentaire de "être verte". Il y a 2 boules vertes, donc $P(\text{Verte}) = 2/10$. La probabilité cherchée est $P(\text{Non Verte}) = 1 - P(\text{Verte}) = 1 - 2/10 = 8/10 = 4/5$. (Alternativement, il y a $5+3=8$ boules non vertes, donc $P=8/10$).
\end{correctionbox}

% --- Corrections : Permutations ---

\begin{correctionbox}[Correction Exercice 4 : Anagrammes (Permutation Simple)]
Le mot "MATHS" a 5 lettres distinctes. Le nombre d'anagrammes est le nombre de permutations de ces 5 lettres, soit $5! = 5 \times 4 \times 3 \times 2 \times 1 = 120$.
\end{correctionbox}

\begin{correctionbox}[Correction Exercice 5 : Course (Arrangement)]
On cherche le nombre de façons d'ordonner 3 athlètes parmi 10. C'est un arrangement (permutation de $k$ parmi $n$) :
$P(10, 3) = \frac{10!}{(10-3)!} = \frac{10!}{7!} = 10 \times 9 \times 8 = 720$.
Il y a 720 podiums possibles.
\end{correctionbox}

\begin{correctionbox}[Correction Exercice 6 : Anagrammes (Permutation avec Répétition)]
Le mot "PROBABILITE" a 11 lettres. Les répétitions sont : B (2 fois), I (2 fois). Les autres lettres (P, R, O, A, L, T, E) apparaissent une fois.
Le nombre d'anagrammes distinctes est :
$$ \frac{11!}{2! \times 2!} = \frac{39,916,800}{2 \times 2} = \frac{39,916,800}{4} = 9,979,200 $$
\end{correctionbox}

% --- Corrections : Combinaisons ---

\begin{correctionbox}[Correction Exercice 7 : Choix d'un Comité (Combinaison)]
L'ordre ne compte pas, c'est donc une combinaison de 4 étudiants parmi 15 :
$$ \binom{15}{4} = \frac{15!}{4!(15-4)!} = \frac{15!}{4!11!} = \frac{15 \times 14 \times 13 \times 12}{4 \times 3 \times 2 \times 1} = 15 \times 7 \times 13 \times 1 = 1365 $$
Il y a 1365 comités possibles.
\end{correctionbox}

\begin{correctionbox}[Correction Exercice 8 : Mains de Poker (Combinaison)]
On choisit 5 cartes parmi 52, sans ordre. C'est une combinaison :
$$ \binom{52}{5} = \frac{52!}{5!(52-5)!} = \frac{52!}{5!47!} = \frac{52 \times 51 \times 50 \times 49 \times 48}{5 \times 4 \times 3 \times 2 \times 1} = 2,598,960 $$
Il y a 2,598,960 mains de poker possibles.
\end{correctionbox}

\begin{correctionbox}[Correction Exercice 9 : Comité Mixte (Combinaison, Principe Multiplicatif)]
Il faut choisir 2 hommes parmi 6 ET 1 femme parmi 4. On multiplie les possibilités pour chaque choix :
Nombre de façons = (choix des hommes) $\times$ (choix des femmes)
$$ = \binom{6}{2} \times \binom{4}{1} = \frac{6 \times 5}{2 \times 1} \times \frac{4}{1} = 15 \times 4 = 60 $$
Il y a 60 comités possibles.
\end{correctionbox}

\begin{correctionbox}[Correction Exercice 10 : Probabilité avec Combinaisons]
L'univers $S$ est l'ensemble de toutes les mains de 3 cartes. $|S| = \binom{52}{3}$.
L'événement $A$ est "obtenir exactement 2 Rois". Pour cela, il faut choisir 2 Rois parmi les 4 Rois ET 1 carte qui n'est pas un Roi parmi les 48 autres cartes.
$|A| = \binom{4}{2} \times \binom{48}{1}$.
La probabilité est $P(A) = \frac{|A|}{|S|} = \frac{\binom{4}{2} \binom{48}{1}}{\binom{52}{3}}$.
$$ P(A) = \frac{\frac{4 \times 3}{2 \times 1} \times 48}{\frac{52 \times 51 \times 50}{3 \times 2 \times 1}} = \frac{6 \times 48}{22100} = \frac{288}{22100} \approx 0.013 $$
\end{correctionbox}

% --- Corrections : Combinaisons avec Répétition (Étoiles et Bâtons) ---

\begin{correctionbox}[Correction Exercice 11 : Distribution de Bonbons (Étoiles et Bâtons)]
C'est un problème de distribution de $k=8$ objets identiques (bonbons) dans $n=3$ boîtes distinctes (enfants). On utilise la formule $\binom{n+k-1}{k}$.
Nombre de manières = $\binom{3+8-1}{8} = \binom{10}{8} = \binom{10}{2} = \frac{10 \times 9}{2 \times 1} = 45$.
\end{correctionbox}

\begin{correctionbox}[Correction Exercice 12 : Solutions d'Équation (Étoiles et Bâtons)]
Cela revient à distribuer $k=10$ unités identiques dans $n=4$ variables distinctes.
Nombre de solutions = $\binom{n+k-1}{k} = \binom{4+10-1}{10} = \binom{13}{10} = \binom{13}{3} = \frac{13 \times 12 \times 11}{3 \times 2 \times 1} = 286$.
\end{correctionbox}

\begin{correctionbox}[Correction Exercice 13 : Distribution avec Minimum (Étoiles et Bâtons avec Contrainte)]
On doit distribuer $k=12$ pommes à $n=4$ enfants, avec $x_i \ge 1$.
On commence par donner une pomme à chaque enfant. Il reste $12 - 4 = 8$ pommes à distribuer sans contrainte (les $x'_i$ peuvent être nuls).
Le problème devient : distribuer $k'=8$ pommes à $n=4$ enfants.
Nombre de manières = $\binom{n+k'-1}{k'} = \binom{4+8-1}{8} = \binom{11}{8} = \binom{11}{3} = \frac{11 \times 10 \times 9}{3 \times 2 \times 1} = 165$.
\end{correctionbox}

% --- Corrections : Principe d'Inclusion-Exclusion ---

\begin{correctionbox}[Correction Exercice 14 : Divisibilité (Inclusion-Exclusion 2 Ensembles)]
Soit $A$ l'ensemble des entiers $\le 100$ divisibles par 2, et $B$ l'ensemble des entiers $\le 100$ divisibles par 3. On cherche $|A \cup B|$.
$|A| = \lfloor 100/2 \rfloor = 50$.
$|B| = \lfloor 100/3 \rfloor = 33$.
$|A \cap B|$ = ensemble des entiers divisibles par $2 \times 3 = 6$. $|A \cap B| = \lfloor 100/6 \rfloor = 16$.
Par inclusion-exclusion : $|A \cup B| = |A| + |B| - |A \cap B| = 50 + 33 - 16 = 67$.
\end{correctionbox}

\begin{correctionbox}[Correction Exercice 15 : Langues (Inclusion-Exclusion 2 Ensembles)]
Soit $E$ l'ensemble des étudiants étudiant l'anglais, $S$ l'ensemble de ceux étudiant l'espagnol.
$|E| = 30$, $|S| = 25$, $|E \cap S| = 10$.
Nombre d'étudiants étudiant au moins une langue : $|E \cup S| = |E| + |S| - |E \cap S| = 30 + 25 - 10 = 45$.
Nombre total d'étudiants = 50.
Nombre d'étudiants n'étudiant aucune de ces langues = Total - $|E \cup S| = 50 - 45 = 5$.
\end{correctionbox}

\begin{correctionbox}[Correction Exercice 16 : Divisibilité (Inclusion-Exclusion 3 Ensembles)]
Soit $A_2, A_3, A_5$ les ensembles des entiers $\le 100$ divisibles respectivement par 2, 3, 5. On cherche $|A_2 \cup A_3 \cup A_5|$.
$|A_2|=50$, $|A_3|=33$, $|A_5|=20$.
$|A_2 \cap A_3| = |A_6| = \lfloor 100/6 \rfloor = 16$.
$|A_2 \cap A_5| = |A_{10}| = \lfloor 100/10 \rfloor = 10$.
$|A_3 \cap A_5| = |A_{15}| = \lfloor 100/15 \rfloor = 6$.
$|A_2 \cap A_3 \cap A_5| = |A_{30}| = \lfloor 100/30 \rfloor = 3$.
Par inclusion-exclusion :
$|A_2 \cup A_3 \cup A_5| = (|A_2|+|A_3|+|A_5|) - (|A_2 \cap A_3|+|A_2 \cap A_5|+|A_3 \cap A_5|) + |A_2 \cap A_3 \cap A_5|$
$= (50+33+20) - (16+10+6) + 3 = 103 - 32 + 3 = 74$.
\end{correctionbox}

% --- Corrections : Problèmes Combinés et Plus Difficiles ---

\begin{correctionbox}[Correction Exercice 17 : Chemins sur un Grillage (Combinaison)]
Pour aller de (0,0) à (4,3), il faut faire un total de $4+3=7$ déplacements. Parmi ces 7 déplacements, il faut choisir les 4 moments où l'on va à droite (les 3 autres seront obligatoirement vers le haut), ou choisir les 3 moments où l'on va vers le haut.
Le nombre de chemins est $\binom{7}{4} = \binom{7}{3} = \frac{7 \times 6 \times 5}{3 \times 2 \times 1} = 35$.
\end{correctionbox}

\begin{correctionbox}[Correction Exercice 18 : Probabilité Hypergéométrique]
C'est un tirage sans remise. On peut utiliser la loi hypergéométrique ou le dénombrement.
Population totale = $7+5=12$ boules. On en tire $m=4$.
On veut $k=2$ blanches (parmi $w=7$) et $m-k=2$ noires (parmi $b=5$).
Probabilité = $\frac{\binom{w}{k} \binom{b}{m-k}}{\binom{w+b}{m}} = \frac{\binom{7}{2} \binom{5}{2}}{\binom{12}{4}}$.
$$ P = \frac{(\frac{7 \times 6}{2}) \times (\frac{5 \times 4}{2})}{(\frac{12 \times 11 \times 10 \times 9}{4 \times 3 \times 2 \times 1})} = \frac{21 \times 10}{495} = \frac{210}{495} = \frac{14}{33} \approx 0.424 $$
\end{correctionbox}

\begin{correctionbox}[Correction Exercice 19 : Arrangement Circulaire]
Pour $n$ objets distincts, le nombre d'arrangements circulaires est $(n-1)!$.
Ici, $n=6$. Le nombre de manières est $(6-1)! = 5! = 120$.
L'idée est de fixer une personne, puis d'arranger les 5 autres par rapport à elle.
\end{correctionbox}

\begin{correctionbox}[Correction Exercice 20 : Problème des Dérangements (Inclusion-Exclusion)]
On cherche le nombre de dérangements de 4 éléments, noté $D_4$ ou $!4$. La probabilité sera $D_4 / 4!$.
La formule générale des dérangements (obtenue par inclusion-exclusion) est $D_n = n! \sum_{i=0}^n \frac{(-1)^i}{i!}$.
Pour $n=4$:
$D_4 = 4! (1/0! - 1/1! + 1/2! - 1/3! + 1/4!)$
$D_4 = 24 (1 - 1 + 1/2 - 1/6 + 1/24)$
$D_4 = 24 (1/2 - 1/6 + 1/24) = 24 (12/24 - 4/24 + 1/24) = 24 (9/24) = 9$.
Il y a 9 dérangements possibles sur un total de $4! = 24$ permutations.
La probabilité est $P(\text{aucun match}) = D_4 / 4! = 9/24 = 3/8 = 0.375$.
\end{correctionbox}
\section{Probabilités et Dénombrement}

\subsection{Concepts fondamentaux}

\begin{intuitionbox}[Nécessité d'un Cadre Formel]
Avant de calculer des probabilités, il est crucial de définir les règles du jeu :
\newline
\textbf{Qu'est-ce qui peut arriver ?}
\newline
On définit l'ensemble de tous les résultats possibles de l'expérience.
\newline
\textbf{À quoi s'intéresse-t-on ?} 
\newline
On identifie les sous-ensembles de résultats spécifiques qui nous intéressent.
\newline
Ces deux idées nous conduisent aux notions d'Univers et d'Événement, qui sont les piliers de toute théorie des probabilités.
\end{intuitionbox}

\begin{definitionbox}[Concepts Fondamentaux]
\textbf{Univers (ou Espace Échantillon), $S$ :} 
\newline
L'ensemble de tous les résultats possibles d'une expérience aléatoire.
\newline
\textbf{Événement, $A$ :} 
\newline
Un sous-ensemble de l'univers ($A \subseteq S$). C'est un ensemble de résultats auxquels on s'intéresse.
\end{definitionbox}

\begin{examplebox}[Univers et Événement]
Pour l'expérience du "lancer d'un dé à six faces" :
\newline
L'\textbf{univers} est $S = \{1, 2, 3, 4, 5, 6\}$.
"Obtenir un nombre impair" est un événement, représenté par le sous-ensemble $A = \{1, 3, 5\}$.
\end{examplebox}

\subsection{Définition Naïve de la Probabilité}

\begin{definitionbox}[Probabilité Naïve]
Pour une expérience où chaque issue dans un espace échantillon fini $S$ est équiprobable, la probabilité d'un événement $A$ est le rapport du nombre d'issues favorables à $A$ sur le nombre total d'issues :
$$ P(A) = \frac{\text{Nombre d'issues favorables}}{\text{Nombre total d'issues}} = \frac{|A|}{|S|} $$
\end{definitionbox}

\begin{examplebox}[Applications de la définition naïve]
\begin{enumerate}
    \item \textbf{Lancer une pièce équilibrée :}
    L'espace échantillon est $S = \{\text{Pile, Face}\}$, donc $|S| = 2$.
    Si l'événement $A$ est "obtenir Pile", alors $A = \{\text{Pile}\}$ et $|A| = 1$.
    La probabilité est $P(A) = \frac{1}{2}$.

    \item \textbf{Lancer un dé à six faces non pipé :}
    L'espace échantillon est $S = \{1, 2, 3, 4, 5, 6\}$, donc $|S| = 6$.
    Si l'événement $B$ est "obtenir un nombre pair", alors $B = \{2, 4, 6\}$ et $|B| = 3$.
    La probabilité est $P(B) = \frac{3}{6} = \frac{1}{2}$.

    \item \textbf{Tirer une carte d'un jeu de 52 cartes :}
    L'espace échantillon $S$ contient 52 cartes, donc $|S| = 52$.
    Si l'événement $C$ est "tirer un Roi", il y a 4 Rois dans le jeu, donc $|C| = 4$.
    La probabilité est $P(C) = \frac{4}{52} = \frac{1}{13}$.
\end{enumerate}
\end{examplebox}

\subsection{Permutations (Arrangements)}

\begin{definitionbox}[Permutation de $k$ objets parmi $n$]
Le nombre de façons d'arranger $k$ objets choisis parmi $n$ objets distincts (où l'ordre compte et il n'y a pas de répétition) est noté $P(n, k)$ ou $A_n^k$ et est défini par :
$$ P(n, k) = \frac{n!}{(n-k)!} $$
où $n!$ est la factorielle de $n$, et par convention $0! = 1$.
\end{definitionbox}

\begin{intuitionbox}[Permutations de $k$ parmi $n$]
Pour placer $k$ objets dans un ordre spécifique en les choisissant parmi $n$ objets disponibles, on a $n$ choix pour la première position, $(n-1)$ choix pour la deuxième, ..., et $(n-k+1)$ choix pour la $k$-ième position. Cela donne $n \times (n-1) \times \cdots \times (n-k+1)$ arrangements. Ce produit contient $k$ termes. Il est égal à $\frac{n!}{(n-k)!}$, car cela revient à diviser la suite complète $n!$ par les facteurs non utilisés $(n-k) \times (n-k-1) \times \cdots \times 1$.
\end{intuitionbox}

\begin{examplebox}[Permutations de $k$ parmi $n$]
\textbf{Podium d'une course :} Une course réunit 8 coureurs. Combien y a-t-il de podiums (1er, 2e, 3e) possibles ? \\
On cherche le nombre de façons d'ordonner 3 coureurs parmi 8 : $P(8, 3)$. 
$$ P(8, 3) = \frac{8!}{(8-3)!} = \frac{8!}{5!} = 8 \times 7 \times 6 = 336 $$
Il y a 336 podiums possibles.
\end{examplebox}

\subsection{Le Coefficient Binomial}

\begin{theorembox}[Formule du Coefficient Binomial]
Le nombre de façons de choisir $k$ objets parmi un ensemble de $n$ objets distincts (sans remise et sans ordre) est donné par le coefficient binomial :
$$ \binom{n}{k} = \frac{n!}{k!(n-k)!} $$
\end{theorembox}

\begin{intuitionbox}

L’idée est de relier $\binom{n}{k}$ à quelque chose de plus facile à compter : les \textbf{permutations} de $k$ objets parmi $n$, c’est-à-dire les listes ordonnées.  
On sait qu’il y en a :
\[
P(n,k) = \frac{n!}{(n-k)!}.
\]

D’un autre côté, on peut construire chaque permutation en deux étapes :
\begin{enumerate}
    \item Choisir un \textbf{sous-ensemble} de $k$ objets (sans ordre), il y a $\binom{n}{k}$ façons de le faire.
    \item Ordonner ces $k$ objets, il y a $k!$ façons de le faire.
\end{enumerate}
Donc, le nombre total de permutations est aussi $\binom{n}{k} \cdot k!$.

\medskip

\noindent En égalisant les deux expressions :
\[
\binom{n}{k} \cdot k! = \frac{n!}{(n-k)!}
\quad\Longrightarrow\quad
\binom{n}{k} = \frac{n!}{k!(n-k)!}.
\]

\medskip

\noindent Pour rendre cela concret, voici le cas $\binom{5}{3}$.  
Il y a 10 sous-ensembles de 3 éléments parmi $\{a,b,c,d,e\}$. Chacun donne lieu à $3! = 6$ permutations.  
Le tableau ci-dessous montre \textbf{toutes les 60 permutations}, regroupées par sous-ensemble :

\begin{center}
\small
\renewcommand{\arraystretch}{0.9}
\setlength{\tabcolsep}{2pt}
\begin{tabular}{|c|c|c|c|c|c|c|c|c|c|}
\hline
\textbf{$\{a,b,c\}$} & \textbf{$\{a,b,d\}$} & \textbf{$\{a,b,e\}$} & \textbf{$\{a,c,d\}$} & \textbf{$\{a,c,e\}$} & \textbf{$\{a,d,e\}$} & \textbf{$\{b,c,d\}$} & \textbf{$\{b,c,e\}$} & \textbf{$\{b,d,e\}$} & \textbf{$\{c,d,e\}$} \\
\hline
$abc$ & $abd$ & $abe$ & $acd$ & $ace$ & $ade$ & $bcd$ & $bce$ & $bde$ & $cde$ \\
\hline
$acb$ & $adb$ & $aeb$ & $adc$ & $aec$ & $aed$ & $bdc$ & $bec$ & $bed$ & $ced$ \\
\hline
$bac$ & $bad$ & $bae$ & $cad$ & $cae$ & $dae$ & $cbd$ & $ceb$ & $dbe$ & $dce$ \\
\hline
$bca$ & $bda$ & $bea$ & $cda$ & $cea$ & $dea$ & $cdb$ & $ceb$ & $deb$ & $dec$ \\
\hline
$cab$ & $dab$ & $eab$ & $dac$ & $eac$ & $ead$ & $dbc$ & $ebc$ & $edb$ & $ecd$ \\
\hline
$cba$ & $dba$ & $eba$ & $dca$ & $eca$ & $eda$ & $dcb$ & $ebc$ & $edb$ & $edc$ \\
\hline
\end{tabular}
\end{center}

\smallskip

Chaque colonne correspond à \textbf{un seul et même choix non ordonné} (par exemple $\{a,b,c\}$), mais à 6 listes différentes selon l’ordre.  
Ainsi, pour obtenir le nombre de \textit{choix non ordonnés}, on divise le nombre total de listes ($60$) par le nombre d’ordres par groupe ($6$) :
\[
\binom{5}{3} = \frac{60}{6} = 10.
\]

\medskip

\noindent C’est exactement ce que fait la formule :
\[
\binom{n}{k} = \frac{\text{nombre de permutations de } k \text{ parmi } n}{k!} = \frac{n!}{k!(n-k)!}.
\]

\end{intuitionbox}

\begin{examplebox}[Utilisation du Coefficient Binomial]
    \textbf{Comité d'étudiants :} De combien de manières peut-on former un comité de 3 étudiants à partir d'une classe de 10 ? L'ordre ne compte pas.
    $$ \binom{10}{3} = \frac{10!}{3!(10-3)!} = \frac{10 \times 9 \times 8}{3 \times 2 \times 1} = 120 \text{ comités possibles.} $$
\end{examplebox}

\subsection{Identité de Vandermonde}

\begin{theorembox}[Identité de Vandermonde]
Cette identité offre une relation remarquable entre les coefficients binomiaux. Pour des entiers non négatifs $m, n$ et $k$, on a :
$$ \binom{m+n}{k} = \sum_{j=0}^{k} \binom{m}{j} \binom{n}{k-j} $$
\end{theorembox}

\begin{intuitionbox}
C'est le "principe du diviser pour régner". Imaginez que vous devez choisir un comité de $k$ personnes à partir d'un groupe contenant $m$ hommes et $n$ femmes.
Le côté gauche, $\binom{m+n}{k}$, compte directement le nombre total de comités possibles.
Le côté droit arrive au même résultat en additionnant toutes les compositions possibles du comité : choisir 0 homme et $k$ femmes, PLUS 1 homme et $k-1$ femmes, PLUS 2 hommes et $k-2$ femmes, etc., jusqu'à choisir $k$ hommes et 0 femme. La somme de toutes ces possibilités doit être égale au total.
\end{intuitionbox}

\begin{examplebox}[Application de l'Identité de Vandermonde]
On veut former un comité de 3 personnes ($k=3$) à partir d'un groupe de 5 hommes ($m=5$) et 4 femmes ($n=4$).
\vspace{0.3cm}
\noindent\textbf{Méthode directe (côté gauche) :} \\
On choisit 3 personnes parmi les $5+4=9$ au total.
$$ \binom{9}{3} = \frac{9 \times 8 \times 7}{3 \times 2 \times 1} = 84 $$
\vspace{0.3cm}
\noindent\textbf{Méthode par cas (côté droit) :} \\
La somme est $\binom{5}{0}\binom{4}{3} + \binom{5}{1}\binom{4}{2} + \binom{5}{2}\binom{4}{1} + \binom{5}{3}\binom{4}{0} = 84$. Les deux méthodes donnent bien le même résultat.
\end{examplebox}

\subsection{Bose-Einstein (Étoiles et Bâtons)}

\begin{theorembox}[Combinaisons avec répétition]
Le nombre de façons de distribuer $k$ objets indiscernables dans $n$ boîtes discernables (ou de choisir $k$ objets parmi $n$ avec remise, où l'ordre ne compte pas) est donné par la formule :
$$ \binom{n+k-1}{k} = \binom{n+k-1}{n-1} $$
\end{theorembox}

\begin{intuitionbox}[Étoiles et Bâtons]
Imaginez que les $k$ objets sont des étoiles ($\star$) et que nous avons besoin de $n-1$ bâtons ($|$) pour les séparer en $n$ groupes. Par exemple, pour distribuer $k=7$ étoiles dans $n=4$ boîtes, une configuration possible serait :
$$ \star\star\star \mid \star \mid \mid \star\star\star $$
Cela correspond à 3 objets dans la première boîte, 1 dans la deuxième, 0 dans la troisième et 3 dans la quatrième.
Le problème revient à trouver le nombre de façons d'arranger ces $k$ étoiles et $n-1$ bâtons. Nous avons un total de $n+k-1$ positions, et nous devons choisir les $k$ positions pour les étoiles (ou les $n-1$ positions pour les bâtons). Le nombre de manières de le faire est précisément $\binom{n+k-1}{k}$.
\end{intuitionbox}

\begin{examplebox}[Distribution de biens identiques]
De combien de manières peut-on distribuer 10 croissants identiques à 4 enfants ?
\newline
Ici, $k=10$ (les croissants, objets indiscernables) et $n=4$ (les enfants, boîtes discernables).
Le nombre de distributions possibles est :
$$ \binom{4+10-1}{10} = \binom{13}{10} = \binom{13}{3} = \frac{13 \times 12 \times 11}{3 \times 2 \times 1} = 13 \times 2 \times 11 = 286 $$
Il y a 286 façons de distribuer les croissants.
\end{examplebox}

\subsection{Principe d'Inclusion-Exclusion}

\begin{theorembox}[Principe d'Inclusion-Exclusion pour 3 ensembles]
Pour trois ensembles finis $A$, $B$ et $C$, le nombre d'éléments dans leur union est donné par :
$$ |A \cup B \cup C| = |A| + |B| + |C| - |A \cap B| - |A \cap C| - |B \cap C| + |A \cap B \cap C| $$
\end{theorembox}

\begin{intuitionbox}[Visualisation avec 3 ensembles]
Le principe d'inclusion-exclusion permet de compter le nombre d'éléments dans une union d'ensembles sans double-comptage. Pour comprendre intuitivement pourquoi on ajoute et soustrait alternativement, considérons trois ensembles $A$, $B$ et $C$ :

\begin{center}
\begin{tikzpicture}[set/.style = {draw,
    circle,
    minimum size = 6cm,
    fill=Rhodamine,
    opacity = 0.4,
    text opacity = 1}]
 
\node (A) [set] {$A$};
\node (B) at (60:4cm) [set] {$B$};
\node (C) at (0:4cm) [set] {$C$};
 
\node at (barycentric cs:A=1,B=1) [left] {$X$};
\node at (barycentric cs:A=1,C=1) [below] {$Y$};
\node at (barycentric cs:B=1,C=1) [right] {$Z$};
\node at (barycentric cs:A=1,B=1,C=1) [] {$T$};
 
\end{tikzpicture}
\end{center}

\textbf{Le problème :} Si on additionne simplement $|A| + |B| + |C|$, on compte certaines zones plusieurs fois :
\begin{itemize}
    \item Les intersections deux à deux ($X$, $Y$, $Z$) sont comptées \textbf{deux fois}
    \item L'intersection triple ($T$) est comptée \textbf{trois fois}
\end{itemize}

\textbf{La solution :} On corrige en soustrayant les intersections deux à deux, mais alors l'intersection triple est comptée :
\begin{itemize}
    \item $+3$ fois dans la somme initiale
    \item $-3$ fois dans la soustraction des intersections deux à deux (car elle appartient à chacune)
    \item Donc $0$ fois au total ! Il faut la rajouter.
\end{itemize}

D'où la formule : $|A \cup B \cup C| = |A| + |B| + |C| - |A \cap B| - |A \cap C| - |B \cap C| + |A \cap B \cap C|$
\end{intuitionbox}

\begin{theorembox}[Principe d'Inclusion-Exclusion généralisé]
Pour $n$ ensembles finis $A_1, A_2, \dots, A_n$, on a :
\begin{align*}
|A_1 \cup A_2 \cup \cdots \cup A_n| = & \sum_{i=1}^n |A_i| \\
& - \sum_{1 \leq i < j \leq n} |A_i \cap A_j| \\
& + \sum_{1 \leq i < j < k \leq n} |A_i \cap A_j \cap A_k| \\
& - \cdots \\
& + (-1)^{n+1} |A_1 \cap A_2 \cap \cdots \cap A_n|
\end{align*}
Ce qui s'écrit plus compactement :
$$ \left| \bigcup_{i=1}^n A_i \right| = \sum_{k=1}^n (-1)^{k+1} \sum_{1 \leq i_1 < i_2 < \cdots < i_k \leq n} |A_{i_1} \cap A_{i_2} \cap \cdots \cap A_{i_k}| $$
\end{theorembox}

\begin{intuitionbox}[Généralisation]
La logique reste la même que pour trois ensembles, mais l'argument clé est de prouver que chaque élément est compté \textbf{exactement une fois}, peu importe le nombre d'ensembles auxquels il appartient.

Supposons qu'un élément $x$ est membre d'exactement $k$ ensembles parmi les $n$ ensembles $A_1, \ldots, A_n$. Analysons combien de fois $x$ est compté dans la formule :
\begin{itemize}
    \item \textbf{Première somme ($\sum |A_i|$)} : $x$ est dans $k$ ensembles, donc il est ajouté $k$ fois. Le nombre de fois est $\binom{k}{1}$.
    
    \item \textbf{Deuxième somme ($-\sum |A_i \cap A_j|$)} : On soustrait $x$ pour chaque paire d'ensembles auxquels il appartient. Il y a $\binom{k}{2}$ telles paires.
    
    \item \textbf{Troisième somme ($+\sum |A_i \cap A_j \cap A_k|$)} : On ajoute de nouveau $x$ pour chaque triplet d'ensembles auxquels il appartient. Il y en a $\binom{k}{3}$.
    
    \item \textbf{Et ainsi de suite...}
\end{itemize}

Au total, l'élément $x$ est compté :
$$ \binom{k}{1} - \binom{k}{2} + \binom{k}{3} - \cdots + (-1)^{k-1}\binom{k}{k} \text{ fois.} $$

Pour voir que cette somme vaut exactement 1, rappelons une identité fondamentale issue du binôme de Newton :
$$ (1-1)^k = \sum_{j=0}^{k} (-1)^j \binom{k}{j} = \binom{k}{0} - \binom{k}{1} + \binom{k}{2} - \cdots + (-1)^k \binom{k}{k} = 0 $$

En réarrangeant cette équation, sachant que $\binom{k}{0}=1$ :
$$ \binom{k}{0} = \binom{k}{1} - \binom{k}{2} + \binom{k}{3} - \cdots - (-1)^{k}\binom{k}{k} $$
$$ 1 = \binom{k}{1} - \binom{k}{2} + \binom{k}{3} - \cdots + (-1)^{k-1}\binom{k}{k} $$

Cela prouve que n'importe quel élément, qu'il soit dans un seul ensemble ($k=1$) ou dans plusieurs ($k>1$), contribue précisément pour 1 au décompte final. Le principe d'inclusion-exclusion est donc une méthode infaillible pour corriger les comptages multiples de manière systématique.
\end{intuitionbox}


\begin{examplebox}[Application probabiliste]
On lance trois dés équilibrés. Quelle est la probabilité d'obtenir au moins un 6 ?

\vspace{0.3cm}
\noindent\textbf{Solution avec inclusion-exclusion :}

Soit $A$ = "le premier dé montre 6", $B$ = "le deuxième dé montre 6", $C$ = "le troisième dé montre 6".

On veut $P(A \cup B \cup C)$.

\begin{align*}
P(A \cup B \cup C) &= P(A) + P(B) + P(C) \\
&\quad - P(A \cap B) - P(A \cap C) - P(B \cap C) \\
&\quad + P(A \cap B \cap C) \\
&= \frac{1}{6} + \frac{1}{6} + \frac{1}{6} - \frac{1}{36} - \frac{1}{36} - \frac{1}{36} + \frac{1}{216} \\
&= \frac{3}{6} - \frac{3}{36} + \frac{1}{216} = \frac{1}{2} - \frac{1}{12} + \frac{1}{216} \\
&= \frac{108 - 18 + 1}{216} = \frac{91}{216} \approx 0.421
\end{align*}

\vspace{0.3cm}
\noindent\textbf{Vérification par la méthode complémentaire :}
La probabilité de n'obtenir aucun 6 est $\left(\frac{5}{6}\right)^3 = \frac{125}{216}$, donc la probabilité d'au moins un 6 est $1 - \frac{125}{216} = \frac{91}{216}$.
\end{examplebox}
\newpage

\section{Probabilité conditionnelle}

\begin{intuitionbox}[Question Fondamentale]
La probabilité conditionnelle est le concept qui répond à la question fondamentale : comment devons-nous mettre à jour nos croyances à la lumière des nouvelles informations que nous observons ?
\end{intuitionbox}

\subsection{Définition de la Probabilité Conditionnelle}

\begin{definitionbox}[Probabilité Conditionnelle]
Si $A$ et $B$ sont deux événements avec $P(B) > 0$, alors la probabilité conditionnelle de $A$ sachant $B$, notée $P(A|B)$, est définie comme :
$$P(A|B) = \frac{P(A \cap B)}{P(B)}$$
\end{definitionbox}

\begin{intuitionbox}
Imaginez que l'ensemble de tous les résultats possibles est un grand terrain. Savoir que l'événement $B$ s'est produit, c'est comme si on vous disait que le résultat se trouve dans une zone spécifique de ce terrain. La probabilité conditionnelle $P(A|B)$ ne s'intéresse plus au terrain entier, mais seulement à la proportion de la zone $B$ qui est également occupée par $A$. On "zoome" sur le monde où $B$ est vrai, et on recalcule les probabilités dans ce nouveau monde plus petit.
\end{intuitionbox}

\subsection{Règle du Produit (Intersection de deux événements)}

\begin{theorembox}[Probabilité de l'intersection de deux événements]
Pour tous événements $A$ et $B$ avec des probabilités positives, nous avons :
$$P(A \cap B) = P(A)P(B|A) = P(B)P(A|B)$$
Cela découle directement de la définition de la probabilité conditionnelle.
\end{theorembox}

\begin{intuitionbox}
Pour que deux événements se produisent, le premier doit se produire, PUIS le second doit se produire, sachant que le premier a eu lieu. Cette formule exprime mathématiquement cette idée séquentielle.
\end{intuitionbox}

\begin{examplebox}
Quelle est la probabilité de tirer deux As d'un jeu de 52 cartes sans remise ?
Soit $A$ l'événement "le premier tirage est un As", avec $P(A) = \frac{4}{52}$. Soit $B$ l'événement "le deuxième tirage est un As". Nous cherchons $P(A \cap B)$, que l'on calcule avec la formule $P(A \cap B) = P(A) \times P(B|A)$. La probabilité $P(B|A)$ correspond à tirer un As sachant que la première carte était un As. Il reste alors 51 cartes, dont 3 As. Donc, $P(B|A) = \frac{3}{51}$. Finalement, la probabilité de l'intersection est $P(A \cap B) = \frac{4}{52} \times \frac{3}{51} = \frac{12}{2652} \approx 0.0045$.
\end{examplebox}

\subsection{Règle de la Chaîne (Intersection de n événements)}

\begin{theorembox}[Probabilité de l'intersection de n événements]
Pour tous événements $A_1, \dots, A_n$ avec $P(A_1 \cap A_2 \cap \dots \cap A_{n-1}) > 0$, nous avons :
$$P(A_1 \cap \dots \cap A_n) = P(A_1)P(A_2|A_1)P(A_3|A_1 \cap A_2) \cdots P(A_n|A_1 \cap \dots \cap A_{n-1})$$
\end{theorembox}

\begin{intuitionbox}
Ceci est une généralisation de l'idée précédente, souvent appelée "règle de la chaîne" (chain rule). Pour qu'une séquence d'événements se produise, chaque événement doit se réaliser tour à tour, en tenant compte de tous les événements précédents qui se sont déjà produits.
\end{intuitionbox}

\begin{examplebox}
On tire 3 cartes sans remise. Quelle est la probabilité d'obtenir la séquence Roi, Dame, Valet ?
La probabilité de tirer un Roi en premier ($A_1$) est $P(A_1) = \frac{4}{52}$.
Ensuite, la probabilité de tirer une Dame ($A_2$) sachant qu'un Roi a été tiré est $P(A_2|A_1) = \frac{4}{51}$.
Enfin, la probabilité de tirer un Valet ($A_3$) sachant qu'un Roi et une Dame ont été tirés est $P(A_3|A_1 \cap A_2) = \frac{4}{50}$.
La probabilité totale de la séquence est donc le produit de ces probabilités : $P(A_1 \cap A_2 \cap A_3) = \frac{4}{52} \times \frac{4}{51} \times \frac{4}{50} \approx 0.00048$.
\end{examplebox}

\subsection{Règle de Bayes}

\begin{theorembox}[Règle de Bayes]
$$P(A|B) = \frac{P(B|A)P(A)}{P(B)}$$
\end{theorembox}

\begin{intuitionbox}
La règle de Bayes est la formule pour "inverser" une probabilité conditionnelle. Souvent, il est facile de connaître la probabilité d'un effet étant donné une cause ($P(\text{symptôme}|\text{maladie})$), mais ce qui nous intéresse vraiment, c'est la probabilité de la cause étant donné l'effet observé ($P(\text{maladie}|\text{symptôme})$). La règle de Bayes nous permet de faire ce retournement en utilisant notre connaissance initiale de la probabilité de la cause ($P(\text{maladie})$). C'est le fondement mathématique de la mise à jour de nos croyances.
\end{intuitionbox}

\begin{examplebox}[Dépistage médical]
Une maladie touche 1\% de la population ($P(M) = 0.01$). Un test de dépistage est fiable à 95\% : il est positif pour 95\% des malades ($P(T|M)=0.95$) et négatif pour 95\% des non-malades, ce qui implique un taux de faux positifs de $P(T|\neg M) = 0.05$.
Une personne est testée positive. Quelle est la probabilité qu'elle soit réellement malade, $P(M|T)$ ?
On cherche $P(M|T) = \frac{P(T|M)P(M)}{P(T)}$.
D'abord, on calcule $P(T)$ avec la formule des probabilités totales :
$P(T) = P(T|M)P(M) + P(T|\neg M)P(\neg M) = (0.95 \times 0.01) + (0.05 \times 0.99) = 0.0095 + 0.0495 = 0.059$.
Ensuite, on applique la règle de Bayes : $P(M|T) = \frac{0.95 \times 0.01}{0.059} \approx 0.161$.
Malgré un test positif, il n'y a que 16.1\% de chance que la personne soit malade.
\end{examplebox}

\subsection{Formule des Probabilités Totales}

\begin{theorembox}[Formule des probabilités totales]
Soit $A_1, \dots, A_n$ une partition de l'espace échantillon $S$ (c'est-à-dire que les $A_i$ sont des événements disjoints et leur union est $S$), avec $P(A_i) > 0$ pour tout $i$. Alors pour tout événement $B$ :
$$P(B) = \sum_{i=1}^{n} P(B|A_i)P(A_i)$$
\end{theorembox}

\begin{intuitionbox}
C'est une stratégie de "diviser pour régner". Pour calculer la probabilité totale d'un événement $B$, on peut décomposer le monde en plusieurs scénarios mutuellement exclusifs (la partition $A_i$). On calcule ensuite la probabilité de $B$ dans chacun de ces scénarios ($P(B|A_i)$), on pondère chaque résultat par la probabilité du scénario en question ($P(A_i)$), et on additionne le tout.

\begin{center}
\begin{tikzpicture}
% 1. Dessiner le grand rectangle et les lignes verticales de partition
\draw (0,0) rectangle (12,7);

% 3. Dessiner une grande ellipse pour la forme B
\filldraw[
    fill=gray!30, % Remplissage gris clair
    thick % Trait épais pour le contour
] (6, 3.5) ellipse (5.5cm and 2.5cm); % Centre (6,3.5), rayon x=5.5cm, rayon y=2.5cm

\foreach \x in {2,4,6,8,10} {
    \draw (\x,0) -- (\x,7);
}

% 2. Placer les étiquettes A_1, A_2, ... en bas
\foreach \i [evaluate=\i as \xpos using \i*2-1] in {1,...,6} {
    \node at (\xpos, -0.5) {$A_{\i}$};
}

% 4. Placer l'étiquette pour l'ensemble B
\node at (11, 6) {$B$}; % Ajusté pour être au-dessus de l'ellipse

% 5. Placer les étiquettes pour les intersections B ∩ A_i, toutes au même niveau
\node at (1.2, 3.5) {$B \cap A_1$};
\node at (3, 3.5) {$B \cap A_2$};
\node at (5, 3.5) {$B \cap A_3$};
\node at (7, 3.5) {$B \cap A_4$};
\node at (9, 3.5) {$B \cap A_5$};
\node at (10.8, 3.5) {$B \cap A_6$};
\end{tikzpicture}
\end{center}
\end{intuitionbox}

\begin{examplebox}
Une usine possède trois machines, M1, M2, et M3, qui produisent respectivement 50\%, 30\% et 20\% des articles. Leurs taux de production défectueuse sont de 4\%, 2\% et 5\%. Quelle est la probabilité qu'un article choisi au hasard soit défectueux ?
Soit $D$ l'événement "l'article est défectueux". Les machines forment une partition avec $P(M1)=0.5$, $P(M2)=0.3$, et $P(M3)=0.2$. Les probabilités conditionnelles de défaut sont $P(D|M1)=0.04$, $P(D|M2)=0.02$, et $P(D|M3)=0.05$.
En appliquant la formule, on obtient :
$P(D) = P(D|M1)P(M1) + P(D|M2)P(M2) + P(D|M3)P(M3) = (0.04 \times 0.5) + (0.02 \times 0.3) + (0.05 \times 0.2) = 0.02 + 0.006 + 0.01 = 0.036$.
La probabilité qu'un article soit défectueux est de 3.6\%.
\end{examplebox}

\begin{proofbox}[Démonstration de la formule des probabilités totales]
Puisque les $A_i$ forment une partition de $S$, on peut décomposer $B$ comme :
$$B = (B \cap A_1) \cup (B \cap A_2) \cup \cdots \cup (B \cap A_n)$$
Comme les $A_i$ sont disjoints, les événements $(B \cap A_i)$ le sont aussi. On peut donc sommer leurs probabilités :
$$P(B) = P(B \cap A_1) + P(B \cap A_2) + \cdots + P(B \cap A_n)$$
En appliquant le théorème de l'intersection des probabilités à chaque terme, on obtient :
$$P(B) = P(B|A_1)P(A_1) + P(B|A_2)P(A_2) + \cdots + P(B|A_n) = \sum_{i=1}^{n} P(B|A_i)P(A_i)$$
\end{proofbox}

\subsection{Règle de Bayes avec Conditionnement Additionnel}

\begin{theorembox}[Règle de Bayes avec conditionnement additionnel]
À condition que $P(A \cap E) > 0$ et $P(B \cap E) > 0$, nous avons :
$$P(A|B, E) = \frac{P(B|A, E)P(A|E)}{P(B|E)}$$
\end{theorembox}

\begin{intuitionbox}
Cette formule est simplement la règle de Bayes standard, mais appliquée à l'intérieur d'un univers que l'on a déjà "rétréci".

Imaginez que vous recevez une information \textbf{E} qui élimine une grande partie des possibilités. C'est votre nouveau point de départ, votre monde est plus petit. Toutes les probabilités que vous calculez désormais sont relatives à ce monde restreint.

Dans ce nouveau monde, vous recevez une autre information, l'évidence \textbf{B}. La règle de Bayes conditionnelle vous permet alors de mettre à jour votre croyance sur un événement \textbf{A}, en utilisant exactement la même logique que la règle de Bayes classique, mais en vous assurant que chaque calcul reste confiné à l'intérieur des frontières de l'univers défini par \textbf{E}.
\end{intuitionbox}

\subsection{Formule des Probabilités Totales avec Conditionnement Additionnel}

\begin{theorembox}[Formule des probabilités totales avec conditionnement additionnel]
Soit $A_1, \dots, A_n$ une partition de $S$. À condition que $P(A_i \cap E) > 0$ pour tout $i$, nous avons :
$$P(B|E) = \sum_{i=1}^{n} P(B|A_i, E)P(A_i|E)$$
\end{theorembox}

\begin{intuitionbox}
\begin{center}
\begin{tikzpicture}
  % Matrice principale, nommée "m"
  \matrix (m) [
    matrix of nodes,
    row sep = -\pgflinewidth,
    column sep = -\pgflinewidth,
    nodes={
      rectangle, draw=black, anchor=center,
      text height=4ex, text depth=0.5ex, minimum width=4em, fill=intuitionColor!10
    }
  ]
  {
    | |              & | |              & |[red_hatch]|    & | |              & | |              & | |            \\
    |[red_hatch]|    & |[purple_hatch]| & |[purple_hatch]| & | |              & |[red_hatch]|    & |[red_hatch]|  \\
    |[red_hatch]|    & |[blue_hatch]|   & |[red_hatch]|    & |[red_hatch]|    & |[red_hatch]|    & | |            \\
  };

  % --- DÉLIMITATION DES COLONNES AVEC ACCOLADES ---
  \draw [decorate, decoration={brace, amplitude=5pt, raise=4mm}]
    (m-1-1.north west) -- (m-1-2.north east) 
    node [midway, yshift=8mm, font=\bfseries] {A1};
    
  \draw [decorate, decoration={brace, amplitude=5pt, raise=4mm}]
    (m-1-3.north west) -- (m-1-4.north east) 
    node [midway, yshift=8mm, font=\bfseries] {A2};
    
  \draw [decorate, decoration={brace, amplitude=5pt, raise=4mm}]
    (m-1-5.north west) -- (m-1-6.north east) 
    node [midway, yshift=8mm, font=\bfseries] {A3};
\end{tikzpicture}
\end{center}
Imaginez que le graphique ci-dessus représente la carte d'un trésor. La carte est partitionnée en trois grandes régions : \textbf{A1}, \textbf{A2}, et \textbf{A3}. Sur cette carte, on a identifié deux types de terrains : une \textbf{zone marécageuse} (événement E, hachures rouges) qui s'étend sur \textbf{10 parcelles}, et une \textbf{zone près d'un vieux chêne} (événement B, hachures bleues) qui couvre \textbf{3 parcelles}.

On vous donne un premier indice : "Le trésor est dans la zone marécageuse (E)". Votre univers de recherche se réduit instantanément à ces 10 parcelles rouges. Puis, on vous donne un second indice : "Le trésor est aussi près d'un chêne (B)". Votre recherche se concentre alors sur les parcelles qui sont à la fois marécageuses et proches d'un chêne (les cases violettes, $B \cap E$).

La question est : "Sachant que le trésor est dans une parcelle violette, quelle est la probabilité qu'il se trouve dans la région A2 ?". On cherche donc $P(A_2 | B, E)$. La règle de Bayes nous permet de le calculer.

\textbf{Calcul des termes nécessaires :} D'abord, nous devons évaluer les probabilités à l'intérieur du "monde marécageux" (sachant E).

La \textbf{vraisemblance} est $P(B|A_2, E)$. En se limitant aux 4 parcelles marécageuses de la région A2, une seule est aussi près d'un chêne. Donc, $P(B|A_2, E) = 1/4$.

La \textbf{probabilité a priori} est $P(A_2|E)$. Sur les 10 parcelles marécageuses, 4 sont dans la région A2. Donc, $P(A_2|E) = 4/10$.

L'\textbf{évidence}, $P(B|E)$, est la probabilité de trouver un chêne dans l'ensemble de la zone marécageuse. On peut la calculer avec la formule des probabilités totales :
$$P(B|E) = P(B|A_1, E)P(A_1|E) + P(B|A_2, E)P(A_2|E) + P(B|A_3, E)P(A_3|E)$$
$$P(B|E) = (\frac{1}{3} \times \frac{3}{10}) + (\frac{1}{4} \times \frac{4}{10}) + (0 \times \frac{3}{10}) = \frac{1}{10} + \frac{1}{10} = \frac{2}{10}$$

\textbf{Application de la règle de Bayes :} Maintenant, nous assemblons le tout.
$$P(A_2|B, E) = \frac{P(B|A_2, E)P(A_2|E)}{P(B|E)} = \frac{(1/4) \times (4/10)}{2/10} = \frac{1/10}{2/10} = \frac{1}{2}$$
L'intuition confirme le calcul : sachant que le trésor est sur une parcelle violette, et qu'il n'y en a que deux (une en A1, une en A2), il y a bien une chance sur deux qu'il se trouve dans la région A2.
\end{intuitionbox}

\subsection{Indépendance de Deux Événements}

\begin{definitionbox}[Indépendance de deux événements]
Les événements $A$ et $B$ sont indépendants si :
$$P(A \cap B) = P(A)P(B)$$
Si $P(A) > 0$ et $P(B) > 0$, cela est équivalent à :
$$P(A|B) = P(A)$$
\end{definitionbox}

\begin{intuitionbox}
L'indépendance est l'absence d'information. Si deux événements sont indépendants, apprendre que l'un s'est produit ne change absolument rien à la probabilité de l'autre. Savoir qu'il pleut à Tokyo ($B$) ne modifie pas la probabilité que vous obteniez pile en lançant une pièce ($A$).
\end{intuitionbox}

\subsection{Indépendance Conditionnelle}

\begin{definitionbox}[Indépendance Conditionnelle]
Les événements $A$ et $B$ sont dits conditionnellement indépendants étant donné $E$ si :
$$P(A \cap B | E) = P(A|E)P(B|E)$$
\end{definitionbox}

\begin{intuitionbox}
L'indépendance peut apparaître ou disparaître quand on observe un autre événement. Par exemple, vos notes en maths ($A$) et en physique ($B$) ne sont probablement pas indépendantes. Mais si l'on sait que vous avez beaucoup travaillé ($E$), alors vos notes en maths et en physique pourraient devenir indépendantes. L'information "vous avez beaucoup travaillé" explique la corrélation ; une fois qu'on la connaît, connaître votre note en maths n'apporte plus d'information sur votre note en physique.
\end{intuitionbox}

\subsection{Le Problème de Monty Hall}

\begin{remarquebox}[Le problème de Monty Hall]
Imaginez que vous êtes à un jeu télévisé. Face à vous se trouvent trois portes fermées. Derrière l'une d'elles se trouve une voiture, et derrière les deux autres, des chèvres.
\begin{enumerate}
    \item Vous choisissez une porte (disons, la porte n°1).
    \item L'animateur, qui sait où se trouve la voiture, ouvre une autre porte (par exemple, la n°3) derrière laquelle se trouve une chèvre.
    \item Il vous demande alors : "Voulez-vous conserver votre choix initial (porte n°1) ou changer pour l'autre porte restante (la n°2) ?"
\end{enumerate}
\textbf{Question :} Avez-vous intérêt à changer de porte ? Votre probabilité de gagner la voiture est-elle plus grande si vous changez, si vous ne changez pas, ou est-elle la même dans les deux cas ?
\end{remarquebox}

\begin{correctionbox}[Solution du problème de Monty Hall]
La réponse est sans équivoque : il faut \textbf{toujours changer de porte}. Cette stratégie fait passer la probabilité de gagner de $1/3$ à $2/3$. L'intuition et la preuve ci-dessous détaillent ce résultat surprenant.
\end{correctionbox}

\begin{intuitionbox}[Le secret : l'information de l'animateur]
L'erreur commune est de supposer qu'il reste deux portes avec une chance égale de $1/2$. Cela ignore une information capitale : le choix de l'animateur n'est \textbf{pas aléatoire}. Il sait où se trouve la voiture et ouvrira toujours une porte perdante.

Le raisonnement correct se déroule en deux temps. D'abord, votre choix initial a $\mathbf{1/3}$ de chance d'être correct. Cela implique qu'il y a $\mathbf{2/3}$ de chance que la voiture soit derrière l'une des \textit{deux autres portes}. Ensuite, lorsque l'animateur ouvre l'une de ces deux portes, il ne fait que vous montrer où la voiture n'est \textit{pas} dans cet ensemble. La probabilité de $2/3$ se \textbf{concentre} alors entièrement sur la seule porte qu'il a laissée fermée. Changer de porte revient à miser sur cette probabilité de $2/3$.
\end{intuitionbox}

\begin{proofbox}[Preuve par l'arbre de décision]
L'analyse de la meilleure stratégie peut être visualisée à l'aide de l'arbre de décision ci-dessous. Il décompose le problème en deux scénarios initiaux : avoir choisi la bonne porte (probabilité $1/3$) ou une mauvaise porte (probabilité $2/3$).

\vspace{0.5cm}
\begin{center}
\begin{tikzpicture}[
  grow=right,
  level distance=4.5cm,
  level 1/.style={sibling distance=3cm},
  level 2/.style={sibling distance=2.5cm},
  edge from parent/.style={draw, -latex},
  % --- Définition des styles pour les cadres ---
  porte_style/.style={rectangle, rounded corners, draw=black, fill=gray!20, thick, inner sep=4pt, text width=2.5cm, align=center},
  gain_style/.style={rectangle, rounded corners, draw=green!60!black, fill=green!20, thick, inner sep=4pt},
  perte_style/.style={rectangle, rounded corners, draw=red!60!black, fill=red!20, thick, inner sep=4pt}
]

\node {S}
    % --- Branche du haut ---
    child {
        node[porte_style] {Bonne porte}
        child {
            node[gain_style] {Gain}
            edge from parent
            node[above, sloped] {$1/2$}
        }
        child {
            node[perte_style] {Perte}
            edge from parent
            node[below, sloped] {$1/2$}
        }
        edge from parent
        node[above, sloped] {1/3}
    }
    % --- Branche du bas ---
    child {
        node[porte_style] {Mauvaise porte}
        child {
            node[gain_style] {Gain}
            edge from parent
            node[above, sloped] {1}
        }
        edge from parent
        node[below, sloped] {2/3}
    };
\end{tikzpicture}
\end{center}
\vspace{0.5cm}

\noindent\textbf{Analyse de l'arbre :}

\vspace{0.3cm}
\noindent\textbf{Branche du bas (cas le plus probable) :}
\newline
Avec une probabilité de $\mathbf{2/3}$, votre choix initial se porte sur une "Mauvaise porte". L'animateur est alors obligé de révéler l'autre porte perdante. La seule porte restante est donc la bonne. L'arbre montre que cela mène à un "Gain" avec une probabilité de $\mathbf{1}$. Ce chemin correspond au résultat de la stratégie \textbf{"Changer"}.

\vspace{0.3cm}
\noindent\textbf{Branche du haut (cas le moins probable) :}
\newline
Avec une probabilité de $\mathbf{1/3}$, vous avez choisi la "Bonne porte" du premier coup. L'arbre se divise alors en deux issues équiprobables ($1/2$ chacune). L'issue "Gain" correspond à la stratégie \textbf{"Garder"} votre choix initial, tandis que l'issue "Perte" correspond à la stratégie \textbf{"Changer"} pour la porte perdante restante.

\vspace{0.3cm}
\noindent\textbf{Conclusion :}
\newline
Pour évaluer la meilleure stratégie, il suffit de sommer les probabilités de gain. La \textbf{probabilité de gain en changeant} est de $\mathbf{2/3}$, car vous gagnez uniquement si votre choix initial était mauvais (branche du bas). La \textbf{probabilité de gain en gardant} est de $\mathbf{1/3}$, car vous gagnez uniquement si votre choix initial était bon (branche "Gain" du haut). La stratégie optimale est donc bien de toujours changer de porte.
\end{proofbox}


\end{document}