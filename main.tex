\documentclass{article}

% --- GESTION DES MARGES DE PAGE ---
\usepackage[a4paper, top=2.5cm, bottom=2.5cm, left=3cm, right=3cm]{geometry}

% --- PRÉAMBULE STANDARD ---
\usepackage[utf8]{inputenc}
\usepackage[T1]{fontenc}
\usepackage{lmodern}
\usepackage[french]{babel}
\usepackage{parskip} % NOUVEAU : Supprime l'indentation et ajoute espace inter-paragraphe
\usepackage{xcolor}
\usepackage{tcolorbox}
\usepackage{listings}
\usepackage{amsmath}
\usepackage{graphicx} % Requis pour inclure des images
\usepackage{amssymb} % Pour les symboles mathématiques comme \subseteq
\usepackage{sectsty} % Pour le style des sections
\usepackage{etoolbox} % Pour les conditions
\usepackage[dvipsnames]{xcolor}
\usepackage{pgfplots} % Le package principal pour les graphiques
\usepackage{enumitem}
\usepackage{diagbox}

\setlist[itemize,1]{label=$\cdot$}

\usetikzlibrary{
    matrix, 
    patterns.meta,
    calc,
    positioning,
    decorations.pathreplacing,
    trees,
    backgrounds
}

% Redéfinir le symbole pour le premier niveau de la liste
\renewcommand{\labelitemi}{$\cdot$}
\renewcommand{\labelitemii}{$\circ$}
\renewcommand{\labelitemiii}{$\ast$}

% Styles de hachures (inchangés)
\tikzset{
  red_hatch/.style={
    pattern={Lines[angle=45, line width=0.8pt, distance=4pt]}, 
    pattern color=red
  },
  blue_hatch/.style={
    pattern={Lines[angle=-45, line width=0.8pt, distance=4pt]}, 
    pattern color=blue
  },
  purple_hatch/.style={
    pattern={Lines[angle=45, line width=0.8pt, distance=4pt]}, 
    pattern color=red,
    postaction={
      pattern={Lines[angle=-45, line width=0.8pt, distance=4pt]}, 
      pattern color=blue
    }
  }
}

% --- BIBLIOTHÈQUES TCOLORBOX ---
\tcbuselibrary{listings, skins, breakable}

% --- GESTION DES LIENS HYPERTEXTE ---
\usepackage[colorlinks=true, linkcolor=black, urlcolor=blue]{hyperref}

% --- SOULIGNER LES TITRES ET SOUS-TITRES ---
\sectionfont{\underline}
\subsectionfont{\underline}

% --- PAGE DE GARDE AMÉLIORÉE ---
\makeatletter
\renewcommand{\maketitle}{%
\begin{titlepage}
\centering
\vspace*{\stretch{1.5}}
{\Huge \bfseries Mes Notes de Lecture\par}
\vspace{0.4cm}
\rule{0.8\linewidth}{0.4pt}
\vspace{1cm}
{\LARGE \bfseries Introduction à la Probabilité\par}
\vspace*{\stretch{2.5}}
{\Large \scshape Lou Brunet\par}
\vspace{0.5cm}
{\large \today\par}
\vspace*{\stretch{1}}
\end{titlepage}
}
\makeatother

% ==================================================================
% --- MODIFIÉ : DÉFINITION DES COULEURS STYLE VS CODE (LIGHT) ---
% ==================================================================
\definecolor{vscodeBlue}{HTML}{569CD6}
\definecolor{vscodeOrange}{HTML}{CE9178}
\definecolor{vscodeGreen}{HTML}{6A9955}
\definecolor{vscodePurple}{HTML}{C586C0}
\definecolor{vscodeGray}{HTML}{9B9B9B}
\definecolor{codeBackground}{HTML}{F8F8F8} % Fond gris très clair
\definecolor{codeText}{HTML}{242424}       % Texte principal (presque noir)
\definecolor{codeGray}{HTML}{A0A0A0}       % Numéros de ligne (gris moyen)

% ==================================================================
% --- MODIFIÉ : CONFIGURATION DU STYLE LISTINGS (LIGHT) ---
% ==================================================================
\lstdefinestyle{vscode}{
    language=Python,
    backgroundcolor=\color{codeBackground},     % Utilise le nouveau fond F8F8F8
    basicstyle=\ttfamily\small\color{codeText}, % Texte principal noir
    keywordstyle=\color{vscodeBlue},
    stringstyle=\color{vscodeOrange},
    commentstyle=\color{vscodeGreen},
    numberstyle=\tiny\color{codeGray},         % Numéros de ligne en gris
    otherkeywords={self, True, False, None},
    keywordstyle=[2]\color{vscodePurple},
    showstringspaces=false,
    breaklines=true,
    frame=none,
    tabsize=4
}

% --- STYLE DE L'OUTPUT ---
\lstdefinestyle{outputstyle}{
    basicstyle=\ttfamily\small\color{codeText}, % Texte principal noir
    breaklines=true,
    frame=none
}

% ==================================================================
% --- MODIFIÉ : DÉFINITION DES CELLULES DE CODE ET OUTPUT (LIGHT) ---
% ==================================================================
% Elles utilisent maintenant le même style "sidebar" que les autres boîtes.

\newtcblisting{codecell}{
  skin=enhanced, % Pour la bordure latérale
  arc=0mm,       % Coins carrés
  boxrule=0pt,   % Pas de cadre
  colback=codeBackground, % Fond clair (F8F8F8)
  borderline west={2pt}{0pt}{vscodeBlue}, % Barre latérale bleue
  fonttitle=\bfseries\color{vscodeBlue}, % Titre en bleu
  listing only,
  listing options={style=vscode, basicstyle=\ttfamily\footnotesize\color{codeText}}, % TEXTE NOIR
  left=3mm, right=3mm, top=2mm, bottom=2mm, % Padding (identique à sidebarstyle)
  boxsep=0mm, % (identique à sidebarstyle)
  breakable  % (identique à sidebarstyle)
}
\newtcblisting{outputcell}{
  skin=enhanced, % Pour la bordure latérale
  arc=0mm,       % Coins carrés
  boxrule=0pt,   % Pas de cadre
  colback=black!5, % Fond gris très clair (Output)
  borderline west={2pt}{0pt}{corrColor}, % Barre latérale grise
  fonttitle=\bfseries\color{corrColor}, % Titre en gris
  listing only,
  listing options={style=outputstyle, basicstyle=\ttfamily\footnotesize\color{codeText}}, % TEXTE NOIR
  left=3mm, right=3mm, top=2mm, bottom=2mm, % Padding (identique à sidebarstyle)
  boxsep=0mm, % (identique à sidebarstyle)
  breakable  % (identique à sidebarstyle)
}

% --- DÉFINITION DES COULEURS POUR DÉF/THÉO/PREUVE/INTUITION/EXEMPLE ---
\definecolor{defColor}{HTML}{1b1f3a}       % Bleu nuit pour les définitions
\definecolor{theoColor}{HTML}{53354a}      % Violet aubergine pour les théorèmes
\definecolor{proofColor}{HTML}{a64942}     % Rouge brique pour les preuves
\definecolor{intuitionColor}{HTML}{16A085} % Turquoise pour l'intuition
\definecolor{exampleColor}{HTML}{4a6982}   % Bleu ardoise pour les exemples
\definecolor{exoColor}{HTML}{1E8449}       % Vert
\definecolor{corrColor}{HTML}{7F8C8D}      % Gris
\definecolor{remarqueColor}{HTML}{D35400} 

% ==================================================================
% --- DÉFINITION DU STYLE DE BASE (MARGES RÉDUITES) ---
% ==================================================================

% --- STYLE "ORGANIQUE" AVEC BARRE LATÉRALE (Pour TOUTES les boîtes) ---
\tcbset{
    sidebarstyle/.style={
        skin=enhanced, % Nécessaire pour les bordures partielles
        arc=0mm,       % Coins carrés
        boxrule=0pt,   % Pas de cadre
        colback=white, % Fond blanc
        colframe=white,
        coltitle=white,
        fontupper=\color{black},
        left=3mm, right=3mm, top=2mm, bottom=2mm, % Padding interne réduit
        boxsep=0mm, % TRÈS IMPORTANT
        breakable
    }
}

% ==================================================================
% --- DÉFINITION DE TOUTES LES CELLULES (TEXTE) ---
% ==================================================================

% --- CELLULES "SIDEBAR" (Pour le contenu théorique) ---
\newtcolorbox{definitionbox}[1][]{
  sidebarstyle,
  borderline west={2pt}{0pt}{defColor}, % Barre latérale gauche
  fonttitle=\bfseries\color{defColor},  % Titre en couleur (sans fond)
  title=Définition\ifstrempty{#1}{}{ : #1}
}
\newtcolorbox{theorembox}[1][]{
  sidebarstyle,
  borderline west={2pt}{0pt}{theoColor},
  fonttitle=\bfseries\color{theoColor},
  title=Théorème\ifstrempty{#1}{}{ : #1}
}
\newtcolorbox{proofbox}[1][]{
  sidebarstyle,
  borderline west={2pt}{0pt}{proofColor},
  fonttitle=\bfseries\color{proofColor},
  title=Preuve\ifstrempty{#1}{}{ : #1}
}
\newtcolorbox{intuitionbox}[1][]{
  sidebarstyle,
  borderline west={2pt}{0pt}{intuitionColor},
  fonttitle=\bfseries\color{intuitionColor},
  title=Intuition\ifstrempty{#1}{}{ : #1}
}
\newtcolorbox{remarquebox}[1][]{
  sidebarstyle,
  borderline west={2pt}{0pt}{remarqueColor},
  fonttitle=\bfseries\color{remarqueColor},
  title=Remarque\ifstrempty{#1}{}{ : #1}
}


% --- CELLULES "SIDEBAR" (Pour les applications) ---
\newtcolorbox{examplebox}[1][]{
  sidebarstyle,
  borderline west={2pt}{0pt}{exampleColor},
  fonttitle=\bfseries\color{exampleColor},
  title=Exemple\ifstrempty{#1}{}{ : #1}
}
\newtcolorbox{exercicebox}[1][]{
    sidebarstyle,
    borderline west={2pt}{0pt}{corrColor},
    fonttitle=\bfseries\color{corrColor},
    title=#1 % <-- On utilise directement l'argument fourni
}
\newtcolorbox{correctionbox}[1][]{
    sidebarstyle,
    borderline west={2pt}{0pt}{exoColor},
    fonttitle=\bfseries\color{exoColor},
    title=#1
}


% =============================================
% --- CORPS DU DOCUMENT ---
% =============================================
\begin{document}

\maketitle

\newpage
\tableofcontents 
\newpage

\newpage
\section{Probabilités et Dénombrement}

\subsection{Concepts fondamentaux}

Avant de pouvoir calculer des probabilités, il est essentiel d'établir un vocabulaire commun pour décrire les expériences aléatoires.

\begin{intuitionbox}[Nécessité d'un Cadre Formel]
Avant de calculer des probabilités, il est crucial de définir les règles du jeu :

\textbf{Qu'est-ce qui peut arriver ?}

On définit l'ensemble de tous les résultats possibles de l'expérience.

\textbf{À quoi s'intéresse-t-on ?} 

On identifie les sous-ensembles de résultats spécifiques qui nous intéressent.

Ces deux idées nous conduisent aux notions d'Univers et d'Événement, qui sont les piliers de toute théorie des probabilités.
\end{intuitionbox}

Cette intuition se traduit formellement par deux définitions clés :

\begin{definitionbox}[Concepts Fondamentaux]
\textbf{Univers (ou Espace Échantillon), $S$ :} 

L'ensemble de tous les résultats possibles d'une expérience aléatoire.

\textbf{Événement, $A$ :} 

Un sous-ensemble de l'univers ($A \subseteq S$). C'est un ensemble de résultats auxquels on s'intéresse.
\end{definitionbox}

Un exemple simple permet de solidifier ces concepts :

\begin{examplebox}[Univers et Événement]
Pour l'expérience du "lancer d'un dé à six faces" :

L'\textbf{univers} est $S = \{1, 2, 3, 4, 5, 6\}$.

"Obtenir un nombre impair" est un événement, représenté par le sous-ensemble $A = \{1, 3, 5\}$.
\end{examplebox}

\subsection{Définition Naïve de la Probabilité}

Pour de nombreuses expériences simples, comme lancer un dé non pipé, chaque résultat possible est "équiprobable". Cette hypothèse est la base de la première définition formelle de la probabilité.

\begin{definitionbox}[Probabilité Naïve]
Pour une expérience où chaque issue dans un espace échantillon fini $S$ est équiprobable, la probabilité d'un événement $A$ est le rapport du nombre d'issues favorables à $A$ sur le nombre total d'issues :
$$ P(A) = \frac{\text{Nombre d'issues favorables}}{\text{Nombre total d'issues}} = \frac{|A|}{|S|} $$
\end{definitionbox}

Appliquons cette formule à quelques cas classiques :

\begin{examplebox}[Applications de la définition naïve]
\begin{enumerate}
    \item \textbf{Lancer une pièce équilibrée :}
    L'espace échantillon est $S = \{\text{Pile, Face}\}$, donc $|S| = 2$.
    Si l'événement $A$ est "obtenir Pile", alors $A = \{\text{Pile}\}$ et $|A| = 1$.
    La probabilité est $P(A) = \frac{1}{2}$.

    \item \textbf{Lancer un dé à six faces non pipé :}
    L'espace échantillon est $S = \{1, 2, 3, 4, 5, 6\}$, donc $|S| = 6$.
    Si l'événement $B$ est "obtenir un nombre pair", alors $B = \{2, 4, 6\}$ et $|B| = 3$.
    La probabilité est $P(B) = \frac{3}{6} = \frac{1}{2}$.

    \item \textbf{Tirer une carte d'un jeu de 52 cartes :}
    L'espace échantillon $S$ contient 52 cartes, donc $|S| = 52$.
    Si l'événement $C$ est "tirer un Roi", il y a 4 Rois dans le jeu, donc $|C| = 4$.
    La probabilité est $P(C) = \frac{4}{52} = \frac{1}{13}$.
\end{enumerate}
\end{examplebox}

\subsection{Permutations (Arrangements)}

Le dénombrement, qui est l'art de compter les tailles $|A|$ et $|S|$, est fondamental pour appliquer la définition naïve. Le premier outil que nous verrons est la permutation, qui compte les arrangements \textbf{ordonnés}.

\begin{definitionbox}[Permutation de $k$ objets parmi $n$]
Le nombre de façons d'arranger $k$ objets choisis parmi $n$ objets distincts (où l'ordre compte et il n'y a pas de répétition) est noté $P(n, k)$ ou $A_n^k$ et est défini par :
$$ P(n, k) = \frac{n!}{(n-k)!} $$
où $n!$ est la factorielle de $n$, et par convention $0! = 1$.
\end{definitionbox}

Cette formule peut sembler abstraite, mais elle provient d'un raisonnement logique simple par "cases" :

\begin{intuitionbox}[Permutations de $k$ parmi $n$]
Pour placer $k$ objets dans un ordre spécifique en les choisissant parmi $n$ objets disponibles, on a $n$ choix pour la première position, $(n-1)$ choix pour la deuxième, ..., et $(n-k+1)$ choix pour la $k$-ième position. Cela donne $n \times (n-1) \times \cdots \times (n-k+1)$ arrangements. Ce produit contient $k$ termes. Il est égal à $\frac{n!}{(n-k)!}$, car cela revient à diviser la suite complète $n!$ par les facteurs non utilisés $(n-k) \times (n-k-1) \times \cdots \times 1$.
\end{intuitionbox}

Voyons une application classique de ce principe :

\begin{examplebox}[Permutations de $k$ parmi $n$]
\textbf{Podium d'une course :} Une course réunit 8 coureurs. Combien y a-t-il de podiums (1er, 2e, 3e) possibles ?

On cherche le nombre de façons d'ordonner 3 coureurs parmi 8 : $P(8, 3)$. 
$$ P(8, 3) = \frac{8!}{(8-3)!} = \frac{8!}{5!} = 8 \times 7 \times 6 = 336 $$
Il y a 336 podiums possibles.
\end{examplebox}

\subsection{Le Coefficient Binomial}

Que se passe-t-il si l'ordre ne compte pas ? Au lieu de compter des podiums, nous voulons compter des comités. C'est le rôle du coefficient binomial.

\begin{theorembox}[Formule du Coefficient Binomial]
Le nombre de façons de choisir $k$ objets parmi un ensemble de $n$ objets distincts (sans remise et sans ordre) est donné par le coefficient binomial :
$$ \binom{n}{k} = \frac{n!}{k!(n-k)!} $$
\end{theorembox}

% NOUVEAU :
La preuve de cette formule repose sur un argument combinatoire élégant : nous allons compter la même chose (les permutations) de deux façons différentes.
% FIN NOUVEAU

\newpage

\begin{proofbox}
Considérons le nombre de permutations de $k$ objets parmi $n$, noté $P(n,k)$.
\begin{enumerate}
    \item \textbf{Méthode 1 :} Par définition (vue ci-dessus), nous savons que $P(n,k) = \frac{n!}{(n-k)!}$.
    
    \item \textbf{Méthode 2 :} Nous pouvons construire une telle permutation en deux étapes successives :
    \begin{itemize}
        \item D'abord, \textbf{choisir un sous-ensemble} de $k$ objets parmi $n$ (l'ordre ne compte pas). C'est le nombre que nous cherchons, notons-le $\binom{n}{k}$.
        \item Ensuite, \textbf{ordonner} ces $k$ objets choisis. Il y a $k!$ façons de les arranger.
    \end{itemize}
    Le nombre total de permutations est donc le produit de ces étapes : $P(n,k) = \binom{n}{k} \times k!$.
\end{enumerate}
En égalisant les deux méthodes, on obtient :
\[ \binom{n}{k} \cdot k! = \frac{n!}{(n-k)!} \]
En divisant par $k!$, on trouve bien la formule :
\[ \binom{n}{k} = \frac{n!}{k!(n-k)!} \]
\end{proofbox}

% NOUVEAU :
L'intuition visuelle derrière cette preuve est de voir comment chaque "choix" (une colonne du tableau) génère $k!$ "ordres" (les lignes de cette colonne).
% FIN NOUVEAU

\begin{intuitionbox}
Pour rendre cela concret, voici le cas $\binom{5}{3}$.  
Il y a 10 sous-ensembles de 3 éléments parmi $\{a,b,c,d,e\}$. Chacun donne lieu à $3! = 6$ permutations.  
Le tableau ci-dessous montre \textbf{toutes les 60 permutations}, regroupées par sous-ensemble :

\vspace{3mm}

\begin{center}
\small
\renewcommand{\arraystretch}{0.9}
\setlength{\tabcolsep}{2pt}
\begin{tabular}{|c|c|c|c|c|c|c|c|c|c|}
\hline
\textbf{$\{a,b,c\}$} & \textbf{$\{a,b,d\}$} & \textbf{$\{a,b,e\}$} & \textbf{$\{a,c,d\}$} & \textbf{$\{a,c,e\}$} & \textbf{$\{a,d,e\}$} & \textbf{$\{b,c,d\}$} & \textbf{$\{b,c,e\}$} & \textbf{$\{b,d,e\}$} & \textbf{$\{c,d,e\}$} \\
\hline
$abc$ & $abd$ & $abe$ & $acd$ & $ace$ & $ade$ & $bcd$ & $bce$ & $bde$ & $cde$ \\
\hline
$acb$ & $adb$ & $aeb$ & $adc$ & $aec$ & $aed$ & $bdc$ & $bec$ & $bed$ & $ced$ \\
\hline
$bac$ & $bad$ & $bae$ & $cad$ & $cae$ & $dae$ & $cbd$ & $ceb$ & $dbe$ & $dce$ \\
\hline
$bca$ & $bda$ & $bea$ & $cda$ & $cea$ & $dea$ & $cdb$ & $ceb$ & $deb$ & $dec$ \\
\hline
$cab$ & $dab$ & $eab$ & $dac$ & $eac$ & $ead$ & $dbc$ & $ebc$ & $edb$ & $ecd$ \\
\hline
$cba$ & $dba$ & $eba$ & $dca$ & $eca$ & $eda$ & $dcb$ & $ebc$ & $edb$ & $edc$ \\
\hline
\end{tabular}
\end{center}

\vspace{3mm}

\smallskip

Chaque colonne correspond à \textbf{un seul et même choix non ordonné} (par exemple $\{a,b,c\}$), mais à 6 listes différentes selon l’ordre.  
Ainsi, pour obtenir le nombre de \textit{choix non ordonnés}, on divise le nombre total de listes ($60$) par le nombre d’ordres par groupe ($6$) :
\[
\binom{5}{3} = \frac{60}{6} = 10.
\]
\end{intuitionbox}

L'application la plus directe est le tirage d'un groupe où l'ordre n'importe pas :

\begin{examplebox}[Utilisation du Coefficient Binomial]
    \textbf{Comité d'étudiants :} De combien de manières peut-on former un comité de 3 étudiants à partir d'une classe de 10 ? L'ordre ne compte pas.
    $$ \binom{10}{3} = \frac{10!}{3!(10-3)!} = \frac{10 \times 9 \times 8}{3 \times 2 \times 1} = 120 \text{ comités possibles.} $$
\end{examplebox}

\newpage

\subsection{Identité de Vandermonde}

Les coefficients binomiaux obéissent à de nombreuses identités. L'identité de Vandermonde est l'une des plus utiles, car elle montre comment décomposer un problème de comptage complexe en sous-problèmes.

\begin{theorembox}[Identité de Vandermonde]
Cette identité offre une relation remarquable entre les coefficients binomiaux. Pour des entiers non négatifs $m, n$ et $k$, on a :
$$ \binom{m+n}{k} = \sum_{j=0}^{k} \binom{m}{j} \binom{n}{k-j} $$
\end{theorembox}

% NOUVEAU :
La preuve la plus intuitive est une "preuve par l'histoire" (proof by story), qui consiste à trouver un scénario de dénombrement que les deux côtés de l'équation résolvent.
% FIN NOUVEAU

\begin{proofbox}[Preuve combinatoire]
Imaginons un groupe composé de $m$ hommes et $n$ femmes. Nous souhaitons former un comité de $k$ personnes. Nous allons compter le nombre de comités possibles de deux façons.

\textbf{Côté gauche : $\binom{m+n}{k}$}
Le groupe total contient $m+n$ personnes. Le nombre de façons de choisir un comité de $k$ personnes parmi ce total est, par définition, $\binom{m+n}{k}$.

\textbf{Côté droit : $\sum_{j=0}^{k} \binom{m}{j} \binom{n}{k-j}$}
Nous pouvons compter le même nombre en conditionnant sur le nombre d'hommes (noté $j$) dans le comité. Un comité de $k$ personnes doit contenir $j$ hommes ET $k-j$ femmes, où $j$ peut aller de $0$ à $k$.
\begin{itemize}
    \item Pour $j=0$ : Choisir 0 homme ($\binom{m}{0}$) ET $k$ femmes ($\binom{n}{k}$).
    \item Pour $j=1$ : Choisir 1 homme ($\binom{m}{1}$) ET $k-1$ femmes ($\binom{n}{k-1}$).
    \item ...
    \item Pour $j=k$ : Choisir $k$ hommes ($\binom{m}{k}$) ET 0 femme ($\binom{n}{0}$).
\end{itemize}
Puisque ces cas (0 homme, 1 homme, etc.) sont mutuellement exclusifs, le nombre total de comités est la somme de toutes ces possibilités :
\[ \sum_{j=0}^{k} \binom{m}{j} \binom{n}{k-j} \]
Puisque les deux côtés comptent exactement la même chose (le nombre total de comités), ils doivent être égaux.
\end{proofbox}

Vérifions cette identité avec un exemple numérique concret, en reprenant l'analogie du comité :

\begin{examplebox}[Application de l'Identité de Vandermonde]
On veut former un comité de 3 personnes ($k=3$) à partir d'un groupe de 5 hommes ($m=5$) et 4 femmes ($n=4$).

\textbf{Méthode directe (côté gauche) :}
On choisit 3 personnes parmi les $5+4=9$ au total.
$$ \binom{9}{3} = \frac{9 \times 8 \times 7}{3 \times 2 \times 1} = 84 $$

\textbf{Méthode par cas (côté droit) :}
La somme est $\binom{5}{0}\binom{4}{3} + \binom{5}{1}\binom{4}{2} + \binom{5}{2}\binom{4}{1} + \binom{5}{3}\binom{4}{0} = 84$. Les deux méthodes donnent bien le même résultat.
\end{examplebox}

\newpage

\subsection{Bose-Einstein (Étoiles et Bâtons)}

Jusqu'à présent, nous avons supposé un "tirage sans remise". La statistique de Bose-Einstein, ou plus visuellement la méthode des "étoiles et bâtons", s'attaque au problème du \textbf{tirage avec remise} où l'ordre ne compte pas.

\begin{theorembox}[Combinaisons avec répétition]
Le nombre de façons de distribuer $k$ objets indiscernables dans $n$ boîtes discernables (ou de choisir $k$ objets parmi $n$ avec remise, où l'ordre ne compte pas) est donné par la formule :
$$ \binom{n+k-1}{k} = \binom{n+k-1}{n-1} $$
\end{theorembox}

% NOUVEAU :
La preuve de cette formule est l'un des résultats les plus élégants du dénombrement. L'astuce consiste à transformer le problème de distribution en un problème d'arrangement de symboles.
% FIN NOUVEAU

\begin{proofbox}[Par les "Étoiles et Bâtons"]
Nous cherchons à distribuer $k$ objets indiscernables ($\star$) dans $n$ boîtes discernables.
Nous pouvons représenter n'importe quelle distribution comme une séquence de symboles. Nous avons besoin de $k$ étoiles (les objets) et de $n-1$ bâtons ($|$) pour servir de séparateurs entre les $n$ boîtes.

Par exemple, pour distribuer $k=7$ étoiles dans $n=4$ boîtes, la séquence :
$$ \star\star\star \mid \star \mid \mid \star\star\star $$
correspond à : 3 étoiles dans la boîte 1, 1 étoile dans la boîte 2, 0 étoile dans la boîte 3 (l'espace entre deux bâtons), et 3 étoiles dans la boîte 4.

Chaque arrangement unique de ces symboles correspond à une distribution unique. Le problème revient donc à trouver le nombre de façons d'arranger ces $k$ étoiles et ces $n-1$ bâtons.

Nous avons un total de $n+k-1$ positions à remplir. Le nombre de façons de le faire est simplement le nombre de manières de choisir les $k$ positions pour les étoiles (les autres positions étant automatiquement remplies par des bâtons).
C'est exactement :
$$ \binom{n+k-1}{k} $$
(Ce qui est aussi égal à $\binom{n+k-1}{n-1}$, le nombre de façons de choisir les positions des $n-1$ bâtons).
\end{proofbox}

C'est la méthode parfaite pour tout problème de distribution d'objets identiques :

\begin{examplebox}[Distribution de biens identiques]
De combien de manières peut-on distribuer 10 croissants identiques à 4 enfants ?

Ici, $k=10$ (les croissants, objets indiscernables) et $n=4$ (les enfants, boîtes discernables).
Le nombre de distributions possibles est :
$$ \binom{4+10-1}{10} = \binom{13}{10} = \binom{13}{3} = \frac{13 \times 12 \times 11}{3 \times 2 \times 1} = 13 \times 2 \times 11 = 286 $$
Il y a 286 façons de distribuer les croissants.
\end{examplebox}

\subsection{Principe d'Inclusion-Exclusion}

Comment compter le nombre d'éléments dans l'union de plusieurs ensembles ? Si on additionne simplement leurs tailles, on compte les intersections plusieurs fois. Le principe d'inclusion-exclusion corrige systématiquement ce sur-comptage.

\begin{theorembox}[Principe d'Inclusion-Exclusion pour 3 ensembles]
Pour trois ensembles finis $A$, $B$ et $C$, le nombre d'éléments dans leur union est donné par :
$$ |A \cup B \cup C| = |A| + |B| + |C| - |A \cap B| - |A \cap C| - |B \cap C| + |A \cap B \cap C| $$
\end{theorembox}

% NOUVEAU :
La preuve pour 3 ensembles se fait en appliquant la formule pour 2 ensembles de manière répétée.
% FIN NOUVEAU

\begin{proofbox}
Nous utilisons la formule pour deux ensembles, $|X \cup Y| = |X| + |Y| - |X \cap Y|$, de manière imbriquée.
Posons $X = A \cup B$ et $Y = C$.
\begin{align*}
|A \cup B \cup C| &= |(A \cup B) \cup C| \\
&= |A \cup B| + |C| - |(A \cup B) \cap C|
\end{align*}
Nous devons maintenant développer les deux termes compliqués :
\begin{enumerate}
    \item $|A \cup B| = |A| + |B| - |A \cap B|$
    \item Par distributivité de l'intersection sur l'union, $(A \cup B) \cap C = (A \cap C) \cup (B \cap C)$.
\end{enumerate}
Appliquons la formule pour 2 ensembles à ce deuxième terme :
\[ |(A \cap C) \cup (B \cap C)| = |A \cap C| + |B \cap C| - |(A \cap C) \cap (B \cap C)| \]
Ce qui se simplifie en $|A \cap C| + |B \cap C| - |A \cap B \cap C|$.

Finalement, en substituant tout dans l'équation de départ :
\begin{align*}
|A \cup B \cup C| &= \underbrace{(|A| + |B| - |A \cap B|)}_{|A \cup B|} + |C| \\
                 &\quad - \underbrace{(|A \cap C| + |B \cap C| - |A \cap B \cap C|)}_{|(A \cup B) \cap C|}
\end{align*}
En réarrangeant les termes, on obtient la formule voulue :
\[ |A| + |B| + |C| - |A \cap B| - |A \cap C| - |B \cap C| + |A \cap B \cap C| \]
\end{proofbox}

La formule devient évidente lorsque l'on utilise un diagramme de Venn pour visualiser le sur-comptage et sa correction.

\begin{intuitionbox}[Visualisation avec 3 ensembles]
Le principe d'inclusion-exclusion permet de compter le nombre d'éléments dans une union d'ensembles sans double-comptage. Pour comprendre intuitivement pourquoi on ajoute et soustrait alternativement, considérons trois ensembles $A$, $B$ et $C$ :

\begin{center}
\begin{tikzpicture}[set/.style = {draw,
    circle,
    minimum size = 6cm,
    fill=Rhodamine,
    opacity = 0.4,
    text opacity = 1}]
 
\node (A) [set] {$A$};
\node (B) at (60:4cm) [set] {$B$};
\node (C) at (0:4cm) [set] {$C$};
 
\node at (barycentric cs:A=1,B=1) [left] {$X$};
\node at (barycentric cs:A=1,C=1) [below] {$Y$};
\node at (barycentric cs:B=1,C=1) [right] {$Z$};
\node at (barycentric cs:A=1,B=1,C=1) [] {$T$};
 
\end{tikzpicture}
\end{center}

\textbf{Le problème :} Si on additionne simplement $|A| + |B| + |C|$, on compte certaines zones plusieurs fois :
\begin{itemize}
    \item Les intersections deux à deux ($X$, $Y$, $Z$) sont comptées \textbf{deux fois}
    \item L'intersection triple ($T$) est comptée \textbf{trois fois}
\end{itemize}

\textbf{La solution :} On corrige en soustrayant les intersections deux à deux, mais alors l'intersection triple est comptée :
\begin{itemize}
    \item $+3$ fois dans la somme initiale
    \item $-3$ fois dans la soustraction des intersections deux à deux (car elle appartient à chacune)
    \item Donc $0$ fois au total ! Il faut la rajouter.
\end{itemize}

D'où la formule : $|A \cup B \cup C| = |A| + |B| + |C| - |A \cap B| - |A \cap C| - |B \cap C| + |A \cap B \cap C|$
\end{intuitionbox}

Ce que nous avons fait visuellement pour 3 ensembles peut être généralisé par récurrence à $n$ ensembles. La formule générale suit le même principe d'alternance des signes :

\begin{theorembox}[Principe d'Inclusion-Exclusion généralisé]
Pour $n$ ensembles finis $A_1, A_2, \dots, A_n$, on a :
\begin{align*}
|A_1 \cup A_2 \cup \cdots \cup A_n| = & \sum_{i=1}^n |A_i| \\
& - \sum_{1 \leq i < j \leq n} |A_i \cap A_j| \\
& + \sum_{1 \leq i < j < k \leq n} |A_i \cap A_j \cap A_k| \\
& - \cdots \\
& + (-1)^{n+1} |A_1 \cap A_2 \cap \cdots \cap A_n|
\end{align*}
Ce qui s'écrit plus compactement :
$$ \left| \bigcup_{i=1}^n A_i \right| = \sum_{k=1}^n (-1)^{k+1} \sum_{1 \leq i_1 < i_2 < \cdots < i_k \leq n} |A_{i_1} \cap A_{i_2} \cap \cdots \cap A_{i_k}| $$
\end{theorembox}

% NOUVEAU :
La preuve formelle que cette formule gigantesque fonctionne est fascinante. Il suffit de montrer que n'importe quel élément $x$ de l'union, peu importe à combien d'ensembles il appartient, est compté \textbf{exactement une fois} au final.
% FIN NOUVEAU

\begin{proofbox}[Preuve par comptage d'un élément]
Considérons un élément $x$ qui appartient à exactement $k$ ensembles parmi les $n$ ensembles $A_1, \ldots, A_n$ (où $k \ge 1$). Nous devons montrer que $x$ est compté exactement 1 fois par la formule.

Analysons combien de fois $x$ est compté dans chaque somme de la formule :
\begin{itemize}
    \item \textbf{Première somme ($\sum |A_i|$)} : $x$ est dans $k$ ensembles, donc il est ajouté $k$ fois. Le nombre de fois est $\binom{k}{1}$.
    
    \item \textbf{Deuxième somme ($-\sum |A_i \cap A_j|$)} : $x$ est compté (et soustrait) pour chaque \textit{paire} d'ensembles auxquels il appartient. Comme il appartient à $k$ ensembles, il y a $\binom{k}{2}$ telles paires.
    
    \item \textbf{Troisième somme ($+\sum |A_i \cap A_j \cap A_k|$)} : $x$ est ajouté pour chaque \textit{triplet} d'ensembles auxquels il appartient. Il y en a $\binom{k}{3}$.
    
    \item \textbf{Et ainsi de suite...}
\end{itemize}
Au total, l'élément $x$ est compté :
$$ \text{Total} = \binom{k}{1} - \binom{k}{2} + \binom{k}{3} - \cdots + (-1)^{k-1}\binom{k}{k} \text{ fois.} $$
Pour évaluer cette somme, rappelons l'identité fondamentale du binôme de Newton :
$$ (1 + x)^k = \sum_{j=0}^{k} \binom{k}{j} x^j = \binom{k}{0} + \binom{k}{1}x + \binom{k}{2}x^2 + \cdots $$
Si nous posons $x = -1$, nous obtenons :
$$ (1-1)^k = 0 = \binom{k}{0} - \binom{k}{1} + \binom{k}{2} - \binom{k}{3} + \cdots + (-1)^k\binom{k}{k} $$
Sachant que $\binom{k}{0} = 1$, on a :
$$ 0 = 1 - \left( \binom{k}{1} - \binom{k}{2} + \binom{k}{3} - \cdots + (-1)^{k-1}\binom{k}{k} \right) $$
En réarrangeant, on trouve :
$$ 1 = \binom{k}{1} - \binom{k}{2} + \binom{k}{3} - \cdots + (-1)^{k-1}\binom{k}{k} $$
Cela prouve que n'importe quel élément de l'union est compté exactement une fois.
\end{proofbox}

Ce principe est très utile en probabilité, car il permet de calculer $P(A \cup B \cup \dots)$ en se basant sur les probabilités des intersections, qui sont souvent plus faciles à trouver.

\begin{examplebox}[Application probabiliste]
On lance trois dés équilibrés. Quelle est la probabilité d'obtenir au moins un 6 ?

\textbf{Solution avec inclusion-exclusion :}

Soit $A$ = "le premier dé montre 6", $B$ = "le deuxième dé montre 6", $C$ = "le troisième dé montre 6".

On veut $P(A \cup B \cup C)$.

\begin{align*}
P(A \cup B \cup C) &= P(A) + P(B) + P(C) \\
&\quad - P(A \cap B) - P(A \cap C) - P(B \cap C) \\
&\quad + P(A \cap B \cap C) \\
&= \frac{1}{6} + \frac{1}{6} + \frac{1}{6} - \frac{1}{36} - \frac{1}{36} - \frac{1}{36} + \frac{1}{216} \\
&= \frac{3}{6} - \frac{3}{36} + \frac{1}{216} = \frac{1}{2} - \frac{1}{12} + \frac{1}{216} \\
&= \frac{108 - 18 + 1}{216} = \frac{91}{216} \approx 0.421
\end{align*}

\textbf{Vérification par la méthode complémentaire :}

La probabilité de n'obtenir aucun 6 est $\left(\frac{5}{6}\right)^3 = \frac{125}{216}$, donc la probabilité d'au moins un 6 est $1 - \frac{125}{216} = \frac{91}{216}$.
\end{examplebox}

\subsection{Exercices}

Cette série d'exercices vise à renforcer votre compréhension des concepts fondamentaux du dénombrement et de la probabilité naïve. La difficulté augmente progressivement.

% --- Concepts de Base et Probabilité Naïve ---

\begin{exercicebox}[Exercice 1 : Univers et Événements]
On lance deux dés à 6 faces, un rouge et un bleu.
\begin{enumerate}
    \item Décrivez l'univers $S$ de cette expérience. Quelle est sa taille $|S|$ ?
    \item Soit $A$ l'événement "la somme des dés est égale à 7". Listez les issues appartenant à $A$. Calculez $P(A)$.
    \item Soit $B$ l'événement "le dé rouge montre un 3". Listez les issues appartenant à $B$. Calculez $P(B)$.
    \item Décrivez l'événement $A \cap B$ et calculez sa probabilité.
\end{enumerate}
\end{exercicebox}

\begin{exercicebox}[Exercice 2 : Tirage de Cartes (Prob. Naïve)]
On tire une carte au hasard d'un jeu standard de 52 cartes.
\begin{enumerate}
    \item Quelle est la probabilité de tirer un Roi ?
    \item Quelle est la probabilité de tirer une carte rouge (Cœur ou Carreau) ?
    \item Quelle est la probabilité de tirer une figure (Valet, Dame, Roi) ?
    \item Quelle est la probabilité de tirer un As rouge ?
\end{enumerate}
\end{exercicebox}

\begin{exercicebox}[Exercice 3 : Urne Simple (Prob. Naïve)]
Une urne contient 5 boules rouges, 3 boules bleues et 2 boules vertes. On tire une boule au hasard.
\begin{enumerate}
    \item Quelle est la probabilité qu'elle soit bleue ?
    \item Quelle est la probabilité qu'elle ne soit pas verte ?
\end{enumerate}
\end{exercicebox}

% --- Permutations ---

\begin{exercicebox}[Exercice 4 : Anagrammes (Permutation Simple)]
Combien d'anagrammes distinctes peut-on former avec les lettres du mot "MATHS" ?
\end{exercicebox}

\begin{exercicebox}[Exercice 5 : Course (Arrangement)]
Dix athlètes participent à une course. Combien y a-t-il de classements possibles pour les 3 premières places (médaille d'or, d'argent, de bronze) ?
\end{exercicebox}

\begin{exercicebox}[Exercice 6 : Anagrammes (Permutation avec Répétition)]
Combien d'anagrammes distinctes peut-on former avec les lettres du mot "PROBABILITE" ?
\end{exercicebox}

% --- Combinaisons ---

\begin{exercicebox}[Exercice 7 : Choix d'un Comité (Combinaison)]
Une classe compte 15 étudiants. De combien de manières peut-on choisir un comité de 4 étudiants ?
\end{exercicebox}

\begin{exercicebox}[Exercice 8 : Mains de Poker (Combinaison)]
Dans un jeu de 52 cartes, combien de "mains" de 5 cartes différentes peut-on former ?
\end{exercicebox}

\begin{exercicebox}[Exercice 9 : Comité Mixte (Combinaison)]
À partir d'un groupe de 6 hommes et 4 femmes, combien de comités de 3 personnes peut-on former contenant exactement 2 hommes et 1 femme ?
\end{exercicebox}

\begin{exercicebox}[Exercice 10 : Probabilité avec Combinaisons]
On tire simultanément 3 cartes d'un jeu de 52 cartes. Quelle est la probabilité d'obtenir exactement 2 Rois ?
\end{exercicebox}

% --- Combinaisons avec Répétition (Étoiles et Bâtons) ---

\begin{exercicebox}[Exercice 11 : Distribution de Bonbons (Étoiles et Bâtons)]
De combien de manières peut-on distribuer 8 bonbons identiques à 3 enfants ? (Certains enfants peuvent ne rien recevoir).
\end{exercicebox}

\begin{exercicebox}[Exercice 12 : Solutions d'Équation (Étoiles et Bâtons)]
Combien y a-t-il de solutions entières non négatives ($x_i \ge 0$) à l'équation $x_1 + x_2 + x_3 + x_4 = 10$ ?
\end{exercicebox}

\begin{exercicebox}[Exercice 13 : Distribution avec Minimum (Étoiles et Bâtons avec Contrainte)]
De combien de manières peut-on distribuer 12 pommes identiques à 4 enfants, si chaque enfant doit recevoir au moins une pomme ?
\end{exercicebox}

% --- Principe d'Inclusion-Exclusion ---

\begin{exercicebox}[Exercice 14 : Divisibilité (Inclusion-Exclusion 2 Ensembles)]
Parmi les entiers de 1 à 100, combien sont divisibles par 2 OU par 3 ?
\end{exercicebox}

\begin{exercicebox}[Exercice 15 : Langues (Inclusion-Exclusion 2 Ensembles)]
Dans un groupe de 50 étudiants, 30 étudient l'anglais, 25 étudient l'espagnol et 10 étudient les deux langues. Combien d'étudiants étudient au moins une de ces deux langues ? Combien n'en étudient aucune ?
\end{exercicebox}

\begin{exercicebox}[Exercice 16 : Divisibilité (Inclusion-Exclusion 3 Ensembles)]
Parmi les entiers de 1 à 100, combien sont divisibles par 2, 3 OU 5 ?
\end{exercicebox}

% --- Problèmes Combinés et Plus Difficiles ---

\begin{exercicebox}[Exercice 17 : Chemins sur un Grillage (Combinaison)]
Sur un grillage, combien y a-t-il de chemins pour aller du point (0,0) au point (4,3) en se déplaçant uniquement vers la droite (D) ou vers le haut (H) ?
\end{exercicebox}

\begin{exercicebox}[Exercice 18 : Probabilité Hypergéométrique]
Une urne contient 7 boules blanches et 5 boules noires. On tire successivement et sans remise 4 boules. Quelle est la probabilité d'obtenir 2 blanches et 2 noires ?
\end{exercicebox}

\begin{exercicebox}[Exercice 19 : Arrangement Circulaire]
De combien de manières 6 personnes peuvent-elles s'asseoir autour d'une table ronde ? (Deux arrangements sont considérés identiques si chaque personne a les mêmes voisins).
\end{exercicebox}

\begin{exercicebox}[Exercice 20 : Problème des Dérangements (Inclusion-Exclusion)]
Quatre lettres sont adressées à quatre personnes différentes, avec les enveloppes correspondantes. On met chaque lettre au hasard dans une enveloppe. Quelle est la probabilité qu'\textit{aucune} lettre ne soit dans la bonne enveloppe ?
\end{exercicebox}



\subsection{Corrections des Exercices}

% --- Corrections : Concepts de Base et Probabilité Naïve ---

\begin{correctionbox}[Correction Exercice 1 : Univers et Événements]
1) L'univers $S$ est l'ensemble de toutes les paires $(r, b)$ où $r$ est le résultat du dé rouge et $b$ celui du dé bleu. $S = \{ (1,1), (1,2), \dots, (1,6), (2,1), \dots, (6,6) \}$. La taille de l'univers est $|S| = 6 \times 6 = 36$.

2) L'événement $A$ (somme égale à 7) est $A = \{ (1,6), (2,5), (3,4), (4,3), (5,2), (6,1) \}$. Il y a $|A|=6$ issues favorables. La probabilité est $P(A) = |A|/|S| = 6/36 = 1/6$.

3) L'événement $B$ (dé rouge montre 3) est $B = \{ (3,1), (3,2), (3,3), (3,4), (3,5), (3,6) \}$. Il y a $|B|=6$ issues favorables. La probabilité est $P(B) = |B|/|S| = 6/36 = 1/6$.

4) L'événement $A \cap B$ est l'ensemble des issues où la somme est 7 ET le dé rouge est 3. La seule issue possible est $(3,4)$. Donc $A \cap B = \{ (3,4) \}$. La probabilité est $P(A \cap B) = |A \cap B|/|S| = 1/36$.
\end{correctionbox}

\begin{correctionbox}[Correction Exercice 2 : Tirage de Cartes (Prob. Naïve)]
Le nombre total d'issues est $|S| = 52$.

a) Il y a 4 Rois. $P(\text{Roi}) = 4/52 = 1/13$.

b) Il y a 26 cartes rouges (13 Cœurs + 13 Carreaux). $P(\text{Rouge}) = 26/52 = 1/2$.

c) Il y a 12 figures (4 Valets + 4 Dames + 4 Rois). $P(\text{Figure}) = 12/52 = 3/13$.

d) Il y a 2 As rouges (As de Cœur, As de Carreau). $P(\text{As Rouge}) = 2/52 = 1/26$.
\end{correctionbox}

\begin{correctionbox}[Correction Exercice 3 : Urne Simple (Prob. Naïve)]
Le nombre total de boules est $5+3+2 = 10$.

a) Il y a 3 boules bleues. $P(\text{Bleue}) = 3/10$.

b) L'événement "ne pas être verte" est le complémentaire de "être verte". Il y a 2 boules vertes, donc $P(\text{Verte}) = 2/10$. La probabilité cherchée est $P(\text{Non Verte}) = 1 - P(\text{Verte}) = 1 - 2/10 = 8/10 = 4/5$. (Alternativement, il y a $5+3=8$ boules non vertes, donc $P=8/10$).
\end{correctionbox}

% --- Corrections : Permutations ---

\begin{correctionbox}[Correction Exercice 4 : Anagrammes (Permutation Simple)]
Le mot "MATHS" a 5 lettres distinctes. Le nombre d'anagrammes est le nombre de permutations de ces 5 lettres, soit $5! = 5 \times 4 \times 3 \times 2 \times 1 = 120$.
\end{correctionbox}

\begin{correctionbox}[Correction Exercice 5 : Course (Arrangement)]
On cherche le nombre de façons d'ordonner 3 athlètes parmi 10. C'est un arrangement (permutation de $k$ parmi $n$) :
$P(10, 3) = \frac{10!}{(10-3)!} = \frac{10!}{7!} = 10 \times 9 \times 8 = 720$.
Il y a 720 podiums possibles.
\end{correctionbox}

\begin{correctionbox}[Correction Exercice 6 : Anagrammes (Permutation avec Répétition)]
Le mot "PROBABILITE" a 11 lettres. Les répétitions sont : B (2 fois), I (2 fois). Les autres lettres (P, R, O, A, L, T, E) apparaissent une fois.
Le nombre d'anagrammes distinctes est :
$$ \frac{11!}{2! \times 2!} = \frac{39,916,800}{2 \times 2} = \frac{39,916,800}{4} = 9,979,200 $$
\end{correctionbox}

% --- Corrections : Combinaisons ---

\begin{correctionbox}[Correction Exercice 7 : Choix d'un Comité (Combinaison)]
L'ordre ne compte pas, c'est donc une combinaison de 4 étudiants parmi 15 :
$$ \binom{15}{4} = \frac{15!}{4!(15-4)!} = \frac{15!}{4!11!} = \frac{15 \times 14 \times 13 \times 12}{4 \times 3 \times 2 \times 1} = 15 \times 7 \times 13 \times 1 = 1365 $$
Il y a 1365 comités possibles.
\end{correctionbox}

\begin{correctionbox}[Correction Exercice 8 : Mains de Poker (Combinaison)]
On choisit 5 cartes parmi 52, sans ordre. C'est une combinaison :
$$ \binom{52}{5} = \frac{52!}{5!(52-5)!} = \frac{52!}{5!47!} = \frac{52 \times 51 \times 50 \times 49 \times 48}{5 \times 4 \times 3 \times 2 \times 1} = 2,598,960 $$
Il y a 2,598,960 mains de poker possibles.
\end{correctionbox}

\begin{correctionbox}[Correction Exercice 9 : Comité Mixte (Combinaison, Principe Multiplicatif)]
Il faut choisir 2 hommes parmi 6 ET 1 femme parmi 4. On multiplie les possibilités pour chaque choix :
Nombre de façons = (choix des hommes) $\times$ (choix des femmes)
$$ = \binom{6}{2} \times \binom{4}{1} = \frac{6 \times 5}{2 \times 1} \times \frac{4}{1} = 15 \times 4 = 60 $$
Il y a 60 comités possibles.
\end{correctionbox}

\begin{correctionbox}[Correction Exercice 10 : Probabilité avec Combinaisons]
L'univers $S$ est l'ensemble de toutes les mains de 3 cartes. $|S| = \binom{52}{3}$.
L'événement $A$ est "obtenir exactement 2 Rois". Pour cela, il faut choisir 2 Rois parmi les 4 Rois ET 1 carte qui n'est pas un Roi parmi les 48 autres cartes.
$|A| = \binom{4}{2} \times \binom{48}{1}$.
La probabilité est $P(A) = \frac{|A|}{|S|} = \frac{\binom{4}{2} \binom{48}{1}}{\binom{52}{3}}$.
$$ P(A) = \frac{\frac{4 \times 3}{2 \times 1} \times 48}{\frac{52 \times 51 \times 50}{3 \times 2 \times 1}} = \frac{6 \times 48}{22100} = \frac{288}{22100} \approx 0.013 $$
\end{correctionbox}

% --- Corrections : Combinaisons avec Répétition (Étoiles et Bâtons) ---

\begin{correctionbox}[Correction Exercice 11 : Distribution de Bonbons (Étoiles et Bâtons)]
C'est un problème de distribution de $k=8$ objets identiques (bonbons) dans $n=3$ boîtes distinctes (enfants). On utilise la formule $\binom{n+k-1}{k}$.
Nombre de manières = $\binom{3+8-1}{8} = \binom{10}{8} = \binom{10}{2} = \frac{10 \times 9}{2 \times 1} = 45$.
\end{correctionbox}

\begin{correctionbox}[Correction Exercice 12 : Solutions d'Équation (Étoiles et Bâtons)]
Cela revient à distribuer $k=10$ unités identiques dans $n=4$ variables distinctes.
Nombre de solutions = $\binom{n+k-1}{k} = \binom{4+10-1}{10} = \binom{13}{10} = \binom{13}{3} = \frac{13 \times 12 \times 11}{3 \times 2 \times 1} = 286$.
\end{correctionbox}

\begin{correctionbox}[Correction Exercice 13 : Distribution avec Minimum (Étoiles et Bâtons avec Contrainte)]
On doit distribuer $k=12$ pommes à $n=4$ enfants, avec $x_i \ge 1$.
On commence par donner une pomme à chaque enfant. Il reste $12 - 4 = 8$ pommes à distribuer sans contrainte (les $x'_i$ peuvent être nuls).
Le problème devient : distribuer $k'=8$ pommes à $n=4$ enfants.
Nombre de manières = $\binom{n+k'-1}{k'} = \binom{4+8-1}{8} = \binom{11}{8} = \binom{11}{3} = \frac{11 \times 10 \times 9}{3 \times 2 \times 1} = 165$.
\end{correctionbox}

% --- Corrections : Principe d'Inclusion-Exclusion ---

\begin{correctionbox}[Correction Exercice 14 : Divisibilité (Inclusion-Exclusion 2 Ensembles)]
Soit $A$ l'ensemble des entiers $\le 100$ divisibles par 2, et $B$ l'ensemble des entiers $\le 100$ divisibles par 3. On cherche $|A \cup B|$.
$|A| = \lfloor 100/2 \rfloor = 50$.
$|B| = \lfloor 100/3 \rfloor = 33$.
$|A \cap B|$ = ensemble des entiers divisibles par $2 \times 3 = 6$. $|A \cap B| = \lfloor 100/6 \rfloor = 16$.
Par inclusion-exclusion : $|A \cup B| = |A| + |B| - |A \cap B| = 50 + 33 - 16 = 67$.
\end{correctionbox}

\begin{correctionbox}[Correction Exercice 15 : Langues (Inclusion-Exclusion 2 Ensembles)]
Soit $E$ l'ensemble des étudiants étudiant l'anglais, $S$ l'ensemble de ceux étudiant l'espagnol.
$|E| = 30$, $|S| = 25$, $|E \cap S| = 10$.
Nombre d'étudiants étudiant au moins une langue : $|E \cup S| = |E| + |S| - |E \cap S| = 30 + 25 - 10 = 45$.
Nombre total d'étudiants = 50.
Nombre d'étudiants n'étudiant aucune de ces langues = Total - $|E \cup S| = 50 - 45 = 5$.
\end{correctionbox}

\begin{correctionbox}[Correction Exercice 16 : Divisibilité (Inclusion-Exclusion 3 Ensembles)]
Soit $A_2, A_3, A_5$ les ensembles des entiers $\le 100$ divisibles respectivement par 2, 3, 5. On cherche $|A_2 \cup A_3 \cup A_5|$.
$|A_2|=50$, $|A_3|=33$, $|A_5|=20$.
$|A_2 \cap A_3| = |A_6| = \lfloor 100/6 \rfloor = 16$.
$|A_2 \cap A_5| = |A_{10}| = \lfloor 100/10 \rfloor = 10$.
$|A_3 \cap A_5| = |A_{15}| = \lfloor 100/15 \rfloor = 6$.
$|A_2 \cap A_3 \cap A_5| = |A_{30}| = \lfloor 100/30 \rfloor = 3$.
Par inclusion-exclusion :
$|A_2 \cup A_3 \cup A_5| = (|A_2|+|A_3|+|A_5|) - (|A_2 \cap A_3|+|A_2 \cap A_5|+|A_3 \cap A_5|) + |A_2 \cap A_3 \cap A_5|$
$= (50+33+20) - (16+10+6) + 3 = 103 - 32 + 3 = 74$.
\end{correctionbox}

% --- Corrections : Problèmes Combinés et Plus Difficiles ---

\begin{correctionbox}[Correction Exercice 17 : Chemins sur un Grillage (Combinaison)]
Pour aller de (0,0) à (4,3), il faut faire un total de $4+3=7$ déplacements. Parmi ces 7 déplacements, il faut choisir les 4 moments où l'on va à droite (les 3 autres seront obligatoirement vers le haut), ou choisir les 3 moments où l'on va vers le haut.
Le nombre de chemins est $\binom{7}{4} = \binom{7}{3} = \frac{7 \times 6 \times 5}{3 \times 2 \times 1} = 35$.
\end{correctionbox}

\begin{correctionbox}[Correction Exercice 18 : Probabilité Hypergéométrique]
C'est un tirage sans remise. On peut utiliser la loi hypergéométrique ou le dénombrement.
Population totale = $7+5=12$ boules. On en tire $m=4$.
On veut $k=2$ blanches (parmi $w=7$) et $m-k=2$ noires (parmi $b=5$).
Probabilité = $\frac{\binom{w}{k} \binom{b}{m-k}}{\binom{w+b}{m}} = \frac{\binom{7}{2} \binom{5}{2}}{\binom{12}{4}}$.
$$ P = \frac{(\frac{7 \times 6}{2}) \times (\frac{5 \times 4}{2})}{(\frac{12 \times 11 \times 10 \times 9}{4 \times 3 \times 2 \times 1})} = \frac{21 \times 10}{495} = \frac{210}{495} = \frac{14}{33} \approx 0.424 $$
\end{correctionbox}

\begin{correctionbox}[Correction Exercice 19 : Arrangement Circulaire]
Pour $n$ objets distincts, le nombre d'arrangements circulaires est $(n-1)!$.
Ici, $n=6$. Le nombre de manières est $(6-1)! = 5! = 120$.
L'idée est de fixer une personne, puis d'arranger les 5 autres par rapport à elle.
\end{correctionbox}

\begin{correctionbox}[Correction Exercice 20 : Problème des Dérangements (Inclusion-Exclusion)]
On cherche le nombre de dérangements de 4 éléments, noté $D_4$ ou $!4$. La probabilité sera $D_4 / 4!$.
La formule générale des dérangements (obtenue par inclusion-exclusion) est $D_n = n! \sum_{i=0}^n \frac{(-1)^i}{i!}$.
Pour $n=4$:
$D_4 = 4! (1/0! - 1/1! + 1/2! - 1/3! + 1/4!)$
$D_4 = 24 (1 - 1 + 1/2 - 1/6 + 1/24)$
$D_4 = 24 (1/2 - 1/6 + 1/24) = 24 (12/24 - 4/24 + 1/24) = 24 (9/24) = 9$.
Il y a 9 dérangements possibles sur un total de $4! = 24$ permutations.
La probabilité est $P(\text{aucun match}) = D_4 / 4! = 9/24 = 3/8 = 0.375$.
\end{correctionbox}
\section{Probabilités et Dénombrement}

\subsection{Concepts fondamentaux}

\begin{intuitionbox}[Nécessité d'un Cadre Formel]
Avant de calculer des probabilités, il est crucial de définir les règles du jeu :
\newline
\textbf{Qu'est-ce qui peut arriver ?}
\newline
On définit l'ensemble de tous les résultats possibles de l'expérience.
\newline
\textbf{À quoi s'intéresse-t-on ?} 
\newline
On identifie les sous-ensembles de résultats spécifiques qui nous intéressent.
\newline
Ces deux idées nous conduisent aux notions d'Univers et d'Événement, qui sont les piliers de toute théorie des probabilités.
\end{intuitionbox}

\begin{definitionbox}[Concepts Fondamentaux]
\textbf{Univers (ou Espace Échantillon), $S$ :} 
\newline
L'ensemble de tous les résultats possibles d'une expérience aléatoire.
\newline
\textbf{Événement, $A$ :} 
\newline
Un sous-ensemble de l'univers ($A \subseteq S$). C'est un ensemble de résultats auxquels on s'intéresse.
\end{definitionbox}

\begin{examplebox}[Univers et Événement]
Pour l'expérience du "lancer d'un dé à six faces" :
\newline
L'\textbf{univers} est $S = \{1, 2, 3, 4, 5, 6\}$.
"Obtenir un nombre impair" est un événement, représenté par le sous-ensemble $A = \{1, 3, 5\}$.
\end{examplebox}

\subsection{Définition Naïve de la Probabilité}

\begin{definitionbox}[Probabilité Naïve]
Pour une expérience où chaque issue dans un espace échantillon fini $S$ est équiprobable, la probabilité d'un événement $A$ est le rapport du nombre d'issues favorables à $A$ sur le nombre total d'issues :
$$ P(A) = \frac{\text{Nombre d'issues favorables}}{\text{Nombre total d'issues}} = \frac{|A|}{|S|} $$
\end{definitionbox}

\begin{examplebox}[Applications de la définition naïve]
\begin{enumerate}
    \item \textbf{Lancer une pièce équilibrée :}
    L'espace échantillon est $S = \{\text{Pile, Face}\}$, donc $|S| = 2$.
    Si l'événement $A$ est "obtenir Pile", alors $A = \{\text{Pile}\}$ et $|A| = 1$.
    La probabilité est $P(A) = \frac{1}{2}$.

    \item \textbf{Lancer un dé à six faces non pipé :}
    L'espace échantillon est $S = \{1, 2, 3, 4, 5, 6\}$, donc $|S| = 6$.
    Si l'événement $B$ est "obtenir un nombre pair", alors $B = \{2, 4, 6\}$ et $|B| = 3$.
    La probabilité est $P(B) = \frac{3}{6} = \frac{1}{2}$.

    \item \textbf{Tirer une carte d'un jeu de 52 cartes :}
    L'espace échantillon $S$ contient 52 cartes, donc $|S| = 52$.
    Si l'événement $C$ est "tirer un Roi", il y a 4 Rois dans le jeu, donc $|C| = 4$.
    La probabilité est $P(C) = \frac{4}{52} = \frac{1}{13}$.
\end{enumerate}
\end{examplebox}

\subsection{Permutations (Arrangements)}

\begin{definitionbox}[Permutation de $k$ objets parmi $n$]
Le nombre de façons d'arranger $k$ objets choisis parmi $n$ objets distincts (où l'ordre compte et il n'y a pas de répétition) est noté $P(n, k)$ ou $A_n^k$ et est défini par :
$$ P(n, k) = \frac{n!}{(n-k)!} $$
où $n!$ est la factorielle de $n$, et par convention $0! = 1$.
\end{definitionbox}

\begin{intuitionbox}[Permutations de $k$ parmi $n$]
Pour placer $k$ objets dans un ordre spécifique en les choisissant parmi $n$ objets disponibles, on a $n$ choix pour la première position, $(n-1)$ choix pour la deuxième, ..., et $(n-k+1)$ choix pour la $k$-ième position. Cela donne $n \times (n-1) \times \cdots \times (n-k+1)$ arrangements. Ce produit contient $k$ termes. Il est égal à $\frac{n!}{(n-k)!}$, car cela revient à diviser la suite complète $n!$ par les facteurs non utilisés $(n-k) \times (n-k-1) \times \cdots \times 1$.
\end{intuitionbox}

\begin{examplebox}[Permutations de $k$ parmi $n$]
\textbf{Podium d'une course :} Une course réunit 8 coureurs. Combien y a-t-il de podiums (1er, 2e, 3e) possibles ? \\
On cherche le nombre de façons d'ordonner 3 coureurs parmi 8 : $P(8, 3)$. 
$$ P(8, 3) = \frac{8!}{(8-3)!} = \frac{8!}{5!} = 8 \times 7 \times 6 = 336 $$
Il y a 336 podiums possibles.
\end{examplebox}

\subsection{Le Coefficient Binomial}

\begin{theorembox}[Formule du Coefficient Binomial]
Le nombre de façons de choisir $k$ objets parmi un ensemble de $n$ objets distincts (sans remise et sans ordre) est donné par le coefficient binomial :
$$ \binom{n}{k} = \frac{n!}{k!(n-k)!} $$
\end{theorembox}

\begin{intuitionbox}

L’idée est de relier $\binom{n}{k}$ à quelque chose de plus facile à compter : les \textbf{permutations} de $k$ objets parmi $n$, c’est-à-dire les listes ordonnées.  
On sait qu’il y en a :
\[
P(n,k) = \frac{n!}{(n-k)!}.
\]

D’un autre côté, on peut construire chaque permutation en deux étapes :
\begin{enumerate}
    \item Choisir un \textbf{sous-ensemble} de $k$ objets (sans ordre), il y a $\binom{n}{k}$ façons de le faire.
    \item Ordonner ces $k$ objets, il y a $k!$ façons de le faire.
\end{enumerate}
Donc, le nombre total de permutations est aussi $\binom{n}{k} \cdot k!$.

\medskip

\noindent En égalisant les deux expressions :
\[
\binom{n}{k} \cdot k! = \frac{n!}{(n-k)!}
\quad\Longrightarrow\quad
\binom{n}{k} = \frac{n!}{k!(n-k)!}.
\]

\medskip

\noindent Pour rendre cela concret, voici le cas $\binom{5}{3}$.  
Il y a 10 sous-ensembles de 3 éléments parmi $\{a,b,c,d,e\}$. Chacun donne lieu à $3! = 6$ permutations.  
Le tableau ci-dessous montre \textbf{toutes les 60 permutations}, regroupées par sous-ensemble :

\begin{center}
\small
\renewcommand{\arraystretch}{0.9}
\setlength{\tabcolsep}{2pt}
\begin{tabular}{|c|c|c|c|c|c|c|c|c|c|}
\hline
\textbf{$\{a,b,c\}$} & \textbf{$\{a,b,d\}$} & \textbf{$\{a,b,e\}$} & \textbf{$\{a,c,d\}$} & \textbf{$\{a,c,e\}$} & \textbf{$\{a,d,e\}$} & \textbf{$\{b,c,d\}$} & \textbf{$\{b,c,e\}$} & \textbf{$\{b,d,e\}$} & \textbf{$\{c,d,e\}$} \\
\hline
$abc$ & $abd$ & $abe$ & $acd$ & $ace$ & $ade$ & $bcd$ & $bce$ & $bde$ & $cde$ \\
\hline
$acb$ & $adb$ & $aeb$ & $adc$ & $aec$ & $aed$ & $bdc$ & $bec$ & $bed$ & $ced$ \\
\hline
$bac$ & $bad$ & $bae$ & $cad$ & $cae$ & $dae$ & $cbd$ & $ceb$ & $dbe$ & $dce$ \\
\hline
$bca$ & $bda$ & $bea$ & $cda$ & $cea$ & $dea$ & $cdb$ & $ceb$ & $deb$ & $dec$ \\
\hline
$cab$ & $dab$ & $eab$ & $dac$ & $eac$ & $ead$ & $dbc$ & $ebc$ & $edb$ & $ecd$ \\
\hline
$cba$ & $dba$ & $eba$ & $dca$ & $eca$ & $eda$ & $dcb$ & $ebc$ & $edb$ & $edc$ \\
\hline
\end{tabular}
\end{center}

\smallskip

Chaque colonne correspond à \textbf{un seul et même choix non ordonné} (par exemple $\{a,b,c\}$), mais à 6 listes différentes selon l’ordre.  
Ainsi, pour obtenir le nombre de \textit{choix non ordonnés}, on divise le nombre total de listes ($60$) par le nombre d’ordres par groupe ($6$) :
\[
\binom{5}{3} = \frac{60}{6} = 10.
\]

\medskip

\noindent C’est exactement ce que fait la formule :
\[
\binom{n}{k} = \frac{\text{nombre de permutations de } k \text{ parmi } n}{k!} = \frac{n!}{k!(n-k)!}.
\]

\end{intuitionbox}

\begin{examplebox}[Utilisation du Coefficient Binomial]
    \textbf{Comité d'étudiants :} De combien de manières peut-on former un comité de 3 étudiants à partir d'une classe de 10 ? L'ordre ne compte pas.
    $$ \binom{10}{3} = \frac{10!}{3!(10-3)!} = \frac{10 \times 9 \times 8}{3 \times 2 \times 1} = 120 \text{ comités possibles.} $$
\end{examplebox}

\subsection{Identité de Vandermonde}

\begin{theorembox}[Identité de Vandermonde]
Cette identité offre une relation remarquable entre les coefficients binomiaux. Pour des entiers non négatifs $m, n$ et $k$, on a :
$$ \binom{m+n}{k} = \sum_{j=0}^{k} \binom{m}{j} \binom{n}{k-j} $$
\end{theorembox}

\begin{intuitionbox}
C'est le "principe du diviser pour régner". Imaginez que vous devez choisir un comité de $k$ personnes à partir d'un groupe contenant $m$ hommes et $n$ femmes.
Le côté gauche, $\binom{m+n}{k}$, compte directement le nombre total de comités possibles.
Le côté droit arrive au même résultat en additionnant toutes les compositions possibles du comité : choisir 0 homme et $k$ femmes, PLUS 1 homme et $k-1$ femmes, PLUS 2 hommes et $k-2$ femmes, etc., jusqu'à choisir $k$ hommes et 0 femme. La somme de toutes ces possibilités doit être égale au total.
\end{intuitionbox}

\begin{examplebox}[Application de l'Identité de Vandermonde]
On veut former un comité de 3 personnes ($k=3$) à partir d'un groupe de 5 hommes ($m=5$) et 4 femmes ($n=4$).
\vspace{0.3cm}
\noindent\textbf{Méthode directe (côté gauche) :} \\
On choisit 3 personnes parmi les $5+4=9$ au total.
$$ \binom{9}{3} = \frac{9 \times 8 \times 7}{3 \times 2 \times 1} = 84 $$
\vspace{0.3cm}
\noindent\textbf{Méthode par cas (côté droit) :} \\
La somme est $\binom{5}{0}\binom{4}{3} + \binom{5}{1}\binom{4}{2} + \binom{5}{2}\binom{4}{1} + \binom{5}{3}\binom{4}{0} = 84$. Les deux méthodes donnent bien le même résultat.
\end{examplebox}

\subsection{Bose-Einstein (Étoiles et Bâtons)}

\begin{theorembox}[Combinaisons avec répétition]
Le nombre de façons de distribuer $k$ objets indiscernables dans $n$ boîtes discernables (ou de choisir $k$ objets parmi $n$ avec remise, où l'ordre ne compte pas) est donné par la formule :
$$ \binom{n+k-1}{k} = \binom{n+k-1}{n-1} $$
\end{theorembox}

\begin{intuitionbox}[Étoiles et Bâtons]
Imaginez que les $k$ objets sont des étoiles ($\star$) et que nous avons besoin de $n-1$ bâtons ($|$) pour les séparer en $n$ groupes. Par exemple, pour distribuer $k=7$ étoiles dans $n=4$ boîtes, une configuration possible serait :
$$ \star\star\star \mid \star \mid \mid \star\star\star $$
Cela correspond à 3 objets dans la première boîte, 1 dans la deuxième, 0 dans la troisième et 3 dans la quatrième.
Le problème revient à trouver le nombre de façons d'arranger ces $k$ étoiles et $n-1$ bâtons. Nous avons un total de $n+k-1$ positions, et nous devons choisir les $k$ positions pour les étoiles (ou les $n-1$ positions pour les bâtons). Le nombre de manières de le faire est précisément $\binom{n+k-1}{k}$.
\end{intuitionbox}

\begin{examplebox}[Distribution de biens identiques]
De combien de manières peut-on distribuer 10 croissants identiques à 4 enfants ?
\newline
Ici, $k=10$ (les croissants, objets indiscernables) et $n=4$ (les enfants, boîtes discernables).
Le nombre de distributions possibles est :
$$ \binom{4+10-1}{10} = \binom{13}{10} = \binom{13}{3} = \frac{13 \times 12 \times 11}{3 \times 2 \times 1} = 13 \times 2 \times 11 = 286 $$
Il y a 286 façons de distribuer les croissants.
\end{examplebox}

\subsection{Principe d'Inclusion-Exclusion}

\begin{theorembox}[Principe d'Inclusion-Exclusion pour 3 ensembles]
Pour trois ensembles finis $A$, $B$ et $C$, le nombre d'éléments dans leur union est donné par :
$$ |A \cup B \cup C| = |A| + |B| + |C| - |A \cap B| - |A \cap C| - |B \cap C| + |A \cap B \cap C| $$
\end{theorembox}

\begin{intuitionbox}[Visualisation avec 3 ensembles]
Le principe d'inclusion-exclusion permet de compter le nombre d'éléments dans une union d'ensembles sans double-comptage. Pour comprendre intuitivement pourquoi on ajoute et soustrait alternativement, considérons trois ensembles $A$, $B$ et $C$ :

\begin{center}
\begin{tikzpicture}[set/.style = {draw,
    circle,
    minimum size = 6cm,
    fill=Rhodamine,
    opacity = 0.4,
    text opacity = 1}]
 
\node (A) [set] {$A$};
\node (B) at (60:4cm) [set] {$B$};
\node (C) at (0:4cm) [set] {$C$};
 
\node at (barycentric cs:A=1,B=1) [left] {$X$};
\node at (barycentric cs:A=1,C=1) [below] {$Y$};
\node at (barycentric cs:B=1,C=1) [right] {$Z$};
\node at (barycentric cs:A=1,B=1,C=1) [] {$T$};
 
\end{tikzpicture}
\end{center}

\textbf{Le problème :} Si on additionne simplement $|A| + |B| + |C|$, on compte certaines zones plusieurs fois :
\begin{itemize}
    \item Les intersections deux à deux ($X$, $Y$, $Z$) sont comptées \textbf{deux fois}
    \item L'intersection triple ($T$) est comptée \textbf{trois fois}
\end{itemize}

\textbf{La solution :} On corrige en soustrayant les intersections deux à deux, mais alors l'intersection triple est comptée :
\begin{itemize}
    \item $+3$ fois dans la somme initiale
    \item $-3$ fois dans la soustraction des intersections deux à deux (car elle appartient à chacune)
    \item Donc $0$ fois au total ! Il faut la rajouter.
\end{itemize}

D'où la formule : $|A \cup B \cup C| = |A| + |B| + |C| - |A \cap B| - |A \cap C| - |B \cap C| + |A \cap B \cap C|$
\end{intuitionbox}

\begin{theorembox}[Principe d'Inclusion-Exclusion généralisé]
Pour $n$ ensembles finis $A_1, A_2, \dots, A_n$, on a :
\begin{align*}
|A_1 \cup A_2 \cup \cdots \cup A_n| = & \sum_{i=1}^n |A_i| \\
& - \sum_{1 \leq i < j \leq n} |A_i \cap A_j| \\
& + \sum_{1 \leq i < j < k \leq n} |A_i \cap A_j \cap A_k| \\
& - \cdots \\
& + (-1)^{n+1} |A_1 \cap A_2 \cap \cdots \cap A_n|
\end{align*}
Ce qui s'écrit plus compactement :
$$ \left| \bigcup_{i=1}^n A_i \right| = \sum_{k=1}^n (-1)^{k+1} \sum_{1 \leq i_1 < i_2 < \cdots < i_k \leq n} |A_{i_1} \cap A_{i_2} \cap \cdots \cap A_{i_k}| $$
\end{theorembox}

\begin{intuitionbox}[Généralisation]
La logique reste la même que pour trois ensembles, mais l'argument clé est de prouver que chaque élément est compté \textbf{exactement une fois}, peu importe le nombre d'ensembles auxquels il appartient.

Supposons qu'un élément $x$ est membre d'exactement $k$ ensembles parmi les $n$ ensembles $A_1, \ldots, A_n$. Analysons combien de fois $x$ est compté dans la formule :
\begin{itemize}
    \item \textbf{Première somme ($\sum |A_i|$)} : $x$ est dans $k$ ensembles, donc il est ajouté $k$ fois. Le nombre de fois est $\binom{k}{1}$.
    
    \item \textbf{Deuxième somme ($-\sum |A_i \cap A_j|$)} : On soustrait $x$ pour chaque paire d'ensembles auxquels il appartient. Il y a $\binom{k}{2}$ telles paires.
    
    \item \textbf{Troisième somme ($+\sum |A_i \cap A_j \cap A_k|$)} : On ajoute de nouveau $x$ pour chaque triplet d'ensembles auxquels il appartient. Il y en a $\binom{k}{3}$.
    
    \item \textbf{Et ainsi de suite...}
\end{itemize}

Au total, l'élément $x$ est compté :
$$ \binom{k}{1} - \binom{k}{2} + \binom{k}{3} - \cdots + (-1)^{k-1}\binom{k}{k} \text{ fois.} $$

Pour voir que cette somme vaut exactement 1, rappelons une identité fondamentale issue du binôme de Newton :
$$ (1-1)^k = \sum_{j=0}^{k} (-1)^j \binom{k}{j} = \binom{k}{0} - \binom{k}{1} + \binom{k}{2} - \cdots + (-1)^k \binom{k}{k} = 0 $$

En réarrangeant cette équation, sachant que $\binom{k}{0}=1$ :
$$ \binom{k}{0} = \binom{k}{1} - \binom{k}{2} + \binom{k}{3} - \cdots - (-1)^{k}\binom{k}{k} $$
$$ 1 = \binom{k}{1} - \binom{k}{2} + \binom{k}{3} - \cdots + (-1)^{k-1}\binom{k}{k} $$

Cela prouve que n'importe quel élément, qu'il soit dans un seul ensemble ($k=1$) ou dans plusieurs ($k>1$), contribue précisément pour 1 au décompte final. Le principe d'inclusion-exclusion est donc une méthode infaillible pour corriger les comptages multiples de manière systématique.
\end{intuitionbox}


\begin{examplebox}[Application probabiliste]
On lance trois dés équilibrés. Quelle est la probabilité d'obtenir au moins un 6 ?

\vspace{0.3cm}
\noindent\textbf{Solution avec inclusion-exclusion :}

Soit $A$ = "le premier dé montre 6", $B$ = "le deuxième dé montre 6", $C$ = "le troisième dé montre 6".

On veut $P(A \cup B \cup C)$.

\begin{align*}
P(A \cup B \cup C) &= P(A) + P(B) + P(C) \\
&\quad - P(A \cap B) - P(A \cap C) - P(B \cap C) \\
&\quad + P(A \cap B \cap C) \\
&= \frac{1}{6} + \frac{1}{6} + \frac{1}{6} - \frac{1}{36} - \frac{1}{36} - \frac{1}{36} + \frac{1}{216} \\
&= \frac{3}{6} - \frac{3}{36} + \frac{1}{216} = \frac{1}{2} - \frac{1}{12} + \frac{1}{216} \\
&= \frac{108 - 18 + 1}{216} = \frac{91}{216} \approx 0.421
\end{align*}

\vspace{0.3cm}
\noindent\textbf{Vérification par la méthode complémentaire :}
La probabilité de n'obtenir aucun 6 est $\left(\frac{5}{6}\right)^3 = \frac{125}{216}$, donc la probabilité d'au moins un 6 est $1 - \frac{125}{216} = \frac{91}{216}$.
\end{examplebox}
\newpage

\section{Probabilité conditionnelle}

\begin{intuitionbox}[Question Fondamentale]
La probabilité conditionnelle est le concept qui répond à la question fondamentale : comment devons-nous mettre à jour nos croyances à la lumière des nouvelles informations que nous observons ?
\end{intuitionbox}

\subsection{Définition de la Probabilité Conditionnelle}

\begin{definitionbox}[Probabilité Conditionnelle]
Si $A$ et $B$ sont deux événements avec $P(B) > 0$, alors la probabilité conditionnelle de $A$ sachant $B$, notée $P(A|B)$, est définie comme :
$$P(A|B) = \frac{P(A \cap B)}{P(B)}$$
\end{definitionbox}

\begin{intuitionbox}
Imaginez que l'ensemble de tous les résultats possibles est un grand terrain. Savoir que l'événement $B$ s'est produit, c'est comme si on vous disait que le résultat se trouve dans une zone spécifique de ce terrain. La probabilité conditionnelle $P(A|B)$ ne s'intéresse plus au terrain entier, mais seulement à la proportion de la zone $B$ qui est également occupée par $A$. On "zoome" sur le monde où $B$ est vrai, et on recalcule les probabilités dans ce nouveau monde plus petit.
\end{intuitionbox}

\subsection{Règle du Produit (Intersection de deux événements)}

\begin{theorembox}[Probabilité de l'intersection de deux événements]
Pour tous événements $A$ et $B$ avec des probabilités positives, nous avons :
$$P(A \cap B) = P(A)P(B|A) = P(B)P(A|B)$$
Cela découle directement de la définition de la probabilité conditionnelle.
\end{theorembox}

\begin{intuitionbox}
Pour que deux événements se produisent, le premier doit se produire, PUIS le second doit se produire, sachant que le premier a eu lieu. Cette formule exprime mathématiquement cette idée séquentielle.
\end{intuitionbox}

\begin{examplebox}
Quelle est la probabilité de tirer deux As d'un jeu de 52 cartes sans remise ?
Soit $A$ l'événement "le premier tirage est un As", avec $P(A) = \frac{4}{52}$. Soit $B$ l'événement "le deuxième tirage est un As". Nous cherchons $P(A \cap B)$, que l'on calcule avec la formule $P(A \cap B) = P(A) \times P(B|A)$. La probabilité $P(B|A)$ correspond à tirer un As sachant que la première carte était un As. Il reste alors 51 cartes, dont 3 As. Donc, $P(B|A) = \frac{3}{51}$. Finalement, la probabilité de l'intersection est $P(A \cap B) = \frac{4}{52} \times \frac{3}{51} = \frac{12}{2652} \approx 0.0045$.
\end{examplebox}

\subsection{Règle de la Chaîne (Intersection de n événements)}

\begin{theorembox}[Probabilité de l'intersection de n événements]
Pour tous événements $A_1, \dots, A_n$ avec $P(A_1 \cap A_2 \cap \dots \cap A_{n-1}) > 0$, nous avons :
$$P(A_1 \cap \dots \cap A_n) = P(A_1)P(A_2|A_1)P(A_3|A_1 \cap A_2) \cdots P(A_n|A_1 \cap \dots \cap A_{n-1})$$
\end{theorembox}

\begin{intuitionbox}
Ceci est une généralisation de l'idée précédente, souvent appelée "règle de la chaîne" (chain rule). Pour qu'une séquence d'événements se produise, chaque événement doit se réaliser tour à tour, en tenant compte de tous les événements précédents qui se sont déjà produits.
\end{intuitionbox}

\begin{examplebox}
On tire 3 cartes sans remise. Quelle est la probabilité d'obtenir la séquence Roi, Dame, Valet ?
La probabilité de tirer un Roi en premier ($A_1$) est $P(A_1) = \frac{4}{52}$.
Ensuite, la probabilité de tirer une Dame ($A_2$) sachant qu'un Roi a été tiré est $P(A_2|A_1) = \frac{4}{51}$.
Enfin, la probabilité de tirer un Valet ($A_3$) sachant qu'un Roi et une Dame ont été tirés est $P(A_3|A_1 \cap A_2) = \frac{4}{50}$.
La probabilité totale de la séquence est donc le produit de ces probabilités : $P(A_1 \cap A_2 \cap A_3) = \frac{4}{52} \times \frac{4}{51} \times \frac{4}{50} \approx 0.00048$.
\end{examplebox}

\subsection{Règle de Bayes}

\begin{theorembox}[Règle de Bayes]
$$P(A|B) = \frac{P(B|A)P(A)}{P(B)}$$
\end{theorembox}

\begin{intuitionbox}
La règle de Bayes est la formule pour "inverser" une probabilité conditionnelle. Souvent, il est facile de connaître la probabilité d'un effet étant donné une cause ($P(\text{symptôme}|\text{maladie})$), mais ce qui nous intéresse vraiment, c'est la probabilité de la cause étant donné l'effet observé ($P(\text{maladie}|\text{symptôme})$). La règle de Bayes nous permet de faire ce retournement en utilisant notre connaissance initiale de la probabilité de la cause ($P(\text{maladie})$). C'est le fondement mathématique de la mise à jour de nos croyances.
\end{intuitionbox}

\begin{examplebox}[Dépistage médical]
Une maladie touche 1\% de la population ($P(M) = 0.01$). Un test de dépistage est fiable à 95\% : il est positif pour 95\% des malades ($P(T|M)=0.95$) et négatif pour 95\% des non-malades, ce qui implique un taux de faux positifs de $P(T|\neg M) = 0.05$.
Une personne est testée positive. Quelle est la probabilité qu'elle soit réellement malade, $P(M|T)$ ?
On cherche $P(M|T) = \frac{P(T|M)P(M)}{P(T)}$.
D'abord, on calcule $P(T)$ avec la formule des probabilités totales :
$P(T) = P(T|M)P(M) + P(T|\neg M)P(\neg M) = (0.95 \times 0.01) + (0.05 \times 0.99) = 0.0095 + 0.0495 = 0.059$.
Ensuite, on applique la règle de Bayes : $P(M|T) = \frac{0.95 \times 0.01}{0.059} \approx 0.161$.
Malgré un test positif, il n'y a que 16.1\% de chance que la personne soit malade.
\end{examplebox}

\subsection{Formule des Probabilités Totales}

\begin{theorembox}[Formule des probabilités totales]
Soit $A_1, \dots, A_n$ une partition de l'espace échantillon $S$ (c'est-à-dire que les $A_i$ sont des événements disjoints et leur union est $S$), avec $P(A_i) > 0$ pour tout $i$. Alors pour tout événement $B$ :
$$P(B) = \sum_{i=1}^{n} P(B|A_i)P(A_i)$$
\end{theorembox}

\begin{intuitionbox}
C'est une stratégie de "diviser pour régner". Pour calculer la probabilité totale d'un événement $B$, on peut décomposer le monde en plusieurs scénarios mutuellement exclusifs (la partition $A_i$). On calcule ensuite la probabilité de $B$ dans chacun de ces scénarios ($P(B|A_i)$), on pondère chaque résultat par la probabilité du scénario en question ($P(A_i)$), et on additionne le tout.

\begin{center}
\begin{tikzpicture}
% 1. Dessiner le grand rectangle et les lignes verticales de partition
\draw (0,0) rectangle (12,7);

% 3. Dessiner une grande ellipse pour la forme B
\filldraw[
    fill=gray!30, % Remplissage gris clair
    thick % Trait épais pour le contour
] (6, 3.5) ellipse (5.5cm and 2.5cm); % Centre (6,3.5), rayon x=5.5cm, rayon y=2.5cm

\foreach \x in {2,4,6,8,10} {
    \draw (\x,0) -- (\x,7);
}

% 2. Placer les étiquettes A_1, A_2, ... en bas
\foreach \i [evaluate=\i as \xpos using \i*2-1] in {1,...,6} {
    \node at (\xpos, -0.5) {$A_{\i}$};
}

% 4. Placer l'étiquette pour l'ensemble B
\node at (11, 6) {$B$}; % Ajusté pour être au-dessus de l'ellipse

% 5. Placer les étiquettes pour les intersections B ∩ A_i, toutes au même niveau
\node at (1.2, 3.5) {$B \cap A_1$};
\node at (3, 3.5) {$B \cap A_2$};
\node at (5, 3.5) {$B \cap A_3$};
\node at (7, 3.5) {$B \cap A_4$};
\node at (9, 3.5) {$B \cap A_5$};
\node at (10.8, 3.5) {$B \cap A_6$};
\end{tikzpicture}
\end{center}
\end{intuitionbox}

\begin{examplebox}
Une usine possède trois machines, M1, M2, et M3, qui produisent respectivement 50\%, 30\% et 20\% des articles. Leurs taux de production défectueuse sont de 4\%, 2\% et 5\%. Quelle est la probabilité qu'un article choisi au hasard soit défectueux ?
Soit $D$ l'événement "l'article est défectueux". Les machines forment une partition avec $P(M1)=0.5$, $P(M2)=0.3$, et $P(M3)=0.2$. Les probabilités conditionnelles de défaut sont $P(D|M1)=0.04$, $P(D|M2)=0.02$, et $P(D|M3)=0.05$.
En appliquant la formule, on obtient :
$P(D) = P(D|M1)P(M1) + P(D|M2)P(M2) + P(D|M3)P(M3) = (0.04 \times 0.5) + (0.02 \times 0.3) + (0.05 \times 0.2) = 0.02 + 0.006 + 0.01 = 0.036$.
La probabilité qu'un article soit défectueux est de 3.6\%.
\end{examplebox}

\begin{proofbox}[Démonstration de la formule des probabilités totales]
Puisque les $A_i$ forment une partition de $S$, on peut décomposer $B$ comme :
$$B = (B \cap A_1) \cup (B \cap A_2) \cup \cdots \cup (B \cap A_n)$$
Comme les $A_i$ sont disjoints, les événements $(B \cap A_i)$ le sont aussi. On peut donc sommer leurs probabilités :
$$P(B) = P(B \cap A_1) + P(B \cap A_2) + \cdots + P(B \cap A_n)$$
En appliquant le théorème de l'intersection des probabilités à chaque terme, on obtient :
$$P(B) = P(B|A_1)P(A_1) + P(B|A_2)P(A_2) + \cdots + P(B|A_n) = \sum_{i=1}^{n} P(B|A_i)P(A_i)$$
\end{proofbox}

\subsection{Règle de Bayes avec Conditionnement Additionnel}

\begin{theorembox}[Règle de Bayes avec conditionnement additionnel]
À condition que $P(A \cap E) > 0$ et $P(B \cap E) > 0$, nous avons :
$$P(A|B, E) = \frac{P(B|A, E)P(A|E)}{P(B|E)}$$
\end{theorembox}

\begin{intuitionbox}
Cette formule est simplement la règle de Bayes standard, mais appliquée à l'intérieur d'un univers que l'on a déjà "rétréci".

Imaginez que vous recevez une information \textbf{E} qui élimine une grande partie des possibilités. C'est votre nouveau point de départ, votre monde est plus petit. Toutes les probabilités que vous calculez désormais sont relatives à ce monde restreint.

Dans ce nouveau monde, vous recevez une autre information, l'évidence \textbf{B}. La règle de Bayes conditionnelle vous permet alors de mettre à jour votre croyance sur un événement \textbf{A}, en utilisant exactement la même logique que la règle de Bayes classique, mais en vous assurant que chaque calcul reste confiné à l'intérieur des frontières de l'univers défini par \textbf{E}.
\end{intuitionbox}

\subsection{Formule des Probabilités Totales avec Conditionnement Additionnel}

\begin{theorembox}[Formule des probabilités totales avec conditionnement additionnel]
Soit $A_1, \dots, A_n$ une partition de $S$. À condition que $P(A_i \cap E) > 0$ pour tout $i$, nous avons :
$$P(B|E) = \sum_{i=1}^{n} P(B|A_i, E)P(A_i|E)$$
\end{theorembox}

\begin{intuitionbox}
\begin{center}
\begin{tikzpicture}
  % Matrice principale, nommée "m"
  \matrix (m) [
    matrix of nodes,
    row sep = -\pgflinewidth,
    column sep = -\pgflinewidth,
    nodes={
      rectangle, draw=black, anchor=center,
      text height=4ex, text depth=0.5ex, minimum width=4em, fill=intuitionColor!10
    }
  ]
  {
    | |              & | |              & |[red_hatch]|    & | |              & | |              & | |            \\
    |[red_hatch]|    & |[purple_hatch]| & |[purple_hatch]| & | |              & |[red_hatch]|    & |[red_hatch]|  \\
    |[red_hatch]|    & |[blue_hatch]|   & |[red_hatch]|    & |[red_hatch]|    & |[red_hatch]|    & | |            \\
  };

  % --- DÉLIMITATION DES COLONNES AVEC ACCOLADES ---
  \draw [decorate, decoration={brace, amplitude=5pt, raise=4mm}]
    (m-1-1.north west) -- (m-1-2.north east) 
    node [midway, yshift=8mm, font=\bfseries] {A1};
    
  \draw [decorate, decoration={brace, amplitude=5pt, raise=4mm}]
    (m-1-3.north west) -- (m-1-4.north east) 
    node [midway, yshift=8mm, font=\bfseries] {A2};
    
  \draw [decorate, decoration={brace, amplitude=5pt, raise=4mm}]
    (m-1-5.north west) -- (m-1-6.north east) 
    node [midway, yshift=8mm, font=\bfseries] {A3};
\end{tikzpicture}
\end{center}
Imaginez que le graphique ci-dessus représente la carte d'un trésor. La carte est partitionnée en trois grandes régions : \textbf{A1}, \textbf{A2}, et \textbf{A3}. Sur cette carte, on a identifié deux types de terrains : une \textbf{zone marécageuse} (événement E, hachures rouges) qui s'étend sur \textbf{10 parcelles}, et une \textbf{zone près d'un vieux chêne} (événement B, hachures bleues) qui couvre \textbf{3 parcelles}.

On vous donne un premier indice : "Le trésor est dans la zone marécageuse (E)". Votre univers de recherche se réduit instantanément à ces 10 parcelles rouges. Puis, on vous donne un second indice : "Le trésor est aussi près d'un chêne (B)". Votre recherche se concentre alors sur les parcelles qui sont à la fois marécageuses et proches d'un chêne (les cases violettes, $B \cap E$).

La question est : "Sachant que le trésor est dans une parcelle violette, quelle est la probabilité qu'il se trouve dans la région A2 ?". On cherche donc $P(A_2 | B, E)$. La règle de Bayes nous permet de le calculer.

\textbf{Calcul des termes nécessaires :} D'abord, nous devons évaluer les probabilités à l'intérieur du "monde marécageux" (sachant E).

La \textbf{vraisemblance} est $P(B|A_2, E)$. En se limitant aux 4 parcelles marécageuses de la région A2, une seule est aussi près d'un chêne. Donc, $P(B|A_2, E) = 1/4$.

La \textbf{probabilité a priori} est $P(A_2|E)$. Sur les 10 parcelles marécageuses, 4 sont dans la région A2. Donc, $P(A_2|E) = 4/10$.

L'\textbf{évidence}, $P(B|E)$, est la probabilité de trouver un chêne dans l'ensemble de la zone marécageuse. On peut la calculer avec la formule des probabilités totales :
$$P(B|E) = P(B|A_1, E)P(A_1|E) + P(B|A_2, E)P(A_2|E) + P(B|A_3, E)P(A_3|E)$$
$$P(B|E) = (\frac{1}{3} \times \frac{3}{10}) + (\frac{1}{4} \times \frac{4}{10}) + (0 \times \frac{3}{10}) = \frac{1}{10} + \frac{1}{10} = \frac{2}{10}$$

\textbf{Application de la règle de Bayes :} Maintenant, nous assemblons le tout.
$$P(A_2|B, E) = \frac{P(B|A_2, E)P(A_2|E)}{P(B|E)} = \frac{(1/4) \times (4/10)}{2/10} = \frac{1/10}{2/10} = \frac{1}{2}$$
L'intuition confirme le calcul : sachant que le trésor est sur une parcelle violette, et qu'il n'y en a que deux (une en A1, une en A2), il y a bien une chance sur deux qu'il se trouve dans la région A2.
\end{intuitionbox}

\subsection{Indépendance de Deux Événements}

\begin{definitionbox}[Indépendance de deux événements]
Les événements $A$ et $B$ sont indépendants si :
$$P(A \cap B) = P(A)P(B)$$
Si $P(A) > 0$ et $P(B) > 0$, cela est équivalent à :
$$P(A|B) = P(A)$$
\end{definitionbox}

\begin{intuitionbox}
L'indépendance est l'absence d'information. Si deux événements sont indépendants, apprendre que l'un s'est produit ne change absolument rien à la probabilité de l'autre. Savoir qu'il pleut à Tokyo ($B$) ne modifie pas la probabilité que vous obteniez pile en lançant une pièce ($A$).
\end{intuitionbox}

\subsection{Indépendance Conditionnelle}

\begin{definitionbox}[Indépendance Conditionnelle]
Les événements $A$ et $B$ sont dits conditionnellement indépendants étant donné $E$ si :
$$P(A \cap B | E) = P(A|E)P(B|E)$$
\end{definitionbox}

\begin{intuitionbox}
L'indépendance peut apparaître ou disparaître quand on observe un autre événement. Par exemple, vos notes en maths ($A$) et en physique ($B$) ne sont probablement pas indépendantes. Mais si l'on sait que vous avez beaucoup travaillé ($E$), alors vos notes en maths et en physique pourraient devenir indépendantes. L'information "vous avez beaucoup travaillé" explique la corrélation ; une fois qu'on la connaît, connaître votre note en maths n'apporte plus d'information sur votre note en physique.
\end{intuitionbox}

\subsection{Le Problème de Monty Hall}

\begin{remarquebox}[Le problème de Monty Hall]
Imaginez que vous êtes à un jeu télévisé. Face à vous se trouvent trois portes fermées. Derrière l'une d'elles se trouve une voiture, et derrière les deux autres, des chèvres.
\begin{enumerate}
    \item Vous choisissez une porte (disons, la porte n°1).
    \item L'animateur, qui sait où se trouve la voiture, ouvre une autre porte (par exemple, la n°3) derrière laquelle se trouve une chèvre.
    \item Il vous demande alors : "Voulez-vous conserver votre choix initial (porte n°1) ou changer pour l'autre porte restante (la n°2) ?"
\end{enumerate}
\textbf{Question :} Avez-vous intérêt à changer de porte ? Votre probabilité de gagner la voiture est-elle plus grande si vous changez, si vous ne changez pas, ou est-elle la même dans les deux cas ?
\end{remarquebox}

\begin{correctionbox}[Solution du problème de Monty Hall]
La réponse est sans équivoque : il faut \textbf{toujours changer de porte}. Cette stratégie fait passer la probabilité de gagner de $1/3$ à $2/3$. L'intuition et la preuve ci-dessous détaillent ce résultat surprenant.
\end{correctionbox}

\begin{intuitionbox}[Le secret : l'information de l'animateur]
L'erreur commune est de supposer qu'il reste deux portes avec une chance égale de $1/2$. Cela ignore une information capitale : le choix de l'animateur n'est \textbf{pas aléatoire}. Il sait où se trouve la voiture et ouvrira toujours une porte perdante.

Le raisonnement correct se déroule en deux temps. D'abord, votre choix initial a $\mathbf{1/3}$ de chance d'être correct. Cela implique qu'il y a $\mathbf{2/3}$ de chance que la voiture soit derrière l'une des \textit{deux autres portes}. Ensuite, lorsque l'animateur ouvre l'une de ces deux portes, il ne fait que vous montrer où la voiture n'est \textit{pas} dans cet ensemble. La probabilité de $2/3$ se \textbf{concentre} alors entièrement sur la seule porte qu'il a laissée fermée. Changer de porte revient à miser sur cette probabilité de $2/3$.
\end{intuitionbox}

\begin{proofbox}[Preuve par l'arbre de décision]
L'analyse de la meilleure stratégie peut être visualisée à l'aide de l'arbre de décision ci-dessous. Il décompose le problème en deux scénarios initiaux : avoir choisi la bonne porte (probabilité $1/3$) ou une mauvaise porte (probabilité $2/3$).

\vspace{0.5cm}
\begin{center}
\begin{tikzpicture}[
  grow=right,
  level distance=4.5cm,
  level 1/.style={sibling distance=3cm},
  level 2/.style={sibling distance=2.5cm},
  edge from parent/.style={draw, -latex},
  % --- Définition des styles pour les cadres ---
  porte_style/.style={rectangle, rounded corners, draw=black, fill=gray!20, thick, inner sep=4pt, text width=2.5cm, align=center},
  gain_style/.style={rectangle, rounded corners, draw=green!60!black, fill=green!20, thick, inner sep=4pt},
  perte_style/.style={rectangle, rounded corners, draw=red!60!black, fill=red!20, thick, inner sep=4pt}
]

\node {S}
    % --- Branche du haut ---
    child {
        node[porte_style] {Bonne porte}
        child {
            node[gain_style] {Gain}
            edge from parent
            node[above, sloped] {$1/2$}
        }
        child {
            node[perte_style] {Perte}
            edge from parent
            node[below, sloped] {$1/2$}
        }
        edge from parent
        node[above, sloped] {1/3}
    }
    % --- Branche du bas ---
    child {
        node[porte_style] {Mauvaise porte}
        child {
            node[gain_style] {Gain}
            edge from parent
            node[above, sloped] {1}
        }
        edge from parent
        node[below, sloped] {2/3}
    };
\end{tikzpicture}
\end{center}
\vspace{0.5cm}

\noindent\textbf{Analyse de l'arbre :}

\vspace{0.3cm}
\noindent\textbf{Branche du bas (cas le plus probable) :}
\newline
Avec une probabilité de $\mathbf{2/3}$, votre choix initial se porte sur une "Mauvaise porte". L'animateur est alors obligé de révéler l'autre porte perdante. La seule porte restante est donc la bonne. L'arbre montre que cela mène à un "Gain" avec une probabilité de $\mathbf{1}$. Ce chemin correspond au résultat de la stratégie \textbf{"Changer"}.

\vspace{0.3cm}
\noindent\textbf{Branche du haut (cas le moins probable) :}
\newline
Avec une probabilité de $\mathbf{1/3}$, vous avez choisi la "Bonne porte" du premier coup. L'arbre se divise alors en deux issues équiprobables ($1/2$ chacune). L'issue "Gain" correspond à la stratégie \textbf{"Garder"} votre choix initial, tandis que l'issue "Perte" correspond à la stratégie \textbf{"Changer"} pour la porte perdante restante.

\vspace{0.3cm}
\noindent\textbf{Conclusion :}
\newline
Pour évaluer la meilleure stratégie, il suffit de sommer les probabilités de gain. La \textbf{probabilité de gain en changeant} est de $\mathbf{2/3}$, car vous gagnez uniquement si votre choix initial était mauvais (branche du bas). La \textbf{probabilité de gain en gardant} est de $\mathbf{1/3}$, car vous gagnez uniquement si votre choix initial était bon (branche "Gain" du haut). La stratégie optimale est donc bien de toujours changer de porte.
\end{proofbox}
\newpage
\section{Espérance et Variance }

\subsection{Espérance d'une variable aléatoire discrète}

Maintenant que nous avons défini les variables aléatoires discrètes et leur distribution (PMF), l'étape suivante est de résumer ces distributions. La mesure la plus importante est leur "centre", ou leur valeur moyenne.

\begin{definitionbox}[Espérance]
L'espérance (ou valeur attendue) d'une variable aléatoire discrète $X$, qui prend les valeurs distinctes $x_1, x_2, \dots$, est définie par :
$$ E(X) = \sum_j x_j P(X=x_j) $$
\end{definitionbox}

Cette formule est une moyenne pondérée de toutes les valeurs possibles.

\begin{intuitionbox}
L'espérance représente la valeur moyenne que l'on obtiendrait si l'on répétait l'expérience un très grand nombre de fois. C'est le \textbf{centre de gravité} de la distribution de probabilité. Si les probabilités étaient des masses placées sur une tige aux positions $x_j$, l'espérance serait le point d'équilibre.
\end{intuitionbox}

L'exemple le plus simple est le lancer d'un dé.

\begin{examplebox}[Lancer d'un dé]
Soit $X$ le résultat d'un lancer de dé équilibré. Chaque face a une probabilité de $1/6$. L'espérance est :
$$ E(X) = 1\left(\frac{1}{6}\right) + 2\left(\frac{1}{6}\right) + 3\left(\frac{1}{6}\right) + 4\left(\frac{1}{6}\right) + 5\left(\frac{1}{6}\right) + 6\left(\frac{1}{6}\right) = \frac{21}{6} = 3.5 $$
Même si 3.5 n'est pas un résultat possible, c'est la valeur moyenne sur un grand nombre de lancers.
\end{examplebox}

\subsection{Linéarité de l'espérance}

Le calcul de l'espérance deviendrait très fastidieux si nous devions toujours utiliser la définition. Heureusement, l'espérance possède une propriété fondamentale qui simplifie énormément les calculs.

\begin{theorembox}[Linéarité de l'espérance]
Pour toutes variables aléatoires $X$ et $Y$, et pour toute constante $c$, on a :
\begin{align*}
E(X+Y) &= E(X) + E(Y) \\
E(cX) &= cE(X)
\end{align*}
Cette propriété est extrêmement puissante car elle ne requiert pas que $X$ et $Y$ soient indépendantes.
\end{theorembox}

La preuve de $E(cX) = cE(X)$ est directe à partir de la définition. La preuve pour la somme $E(X+Y)$ est plus complexe mais essentielle.

\begin{proofbox}
La première propriété est directe :
$$ E(cX) = \sum_x (cx) P(X=x) = c \sum_x x P(X=x) = cE(X) $$
Pour la seconde, nous devons utiliser la définition de l'espérance pour une fonction de deux variables (une extension de LOTUS). Soit $S = X+Y$. L'espérance $E(S)$ se calcule en sommant sur toutes les paires possibles $(x, y)$:
\begin{align*}
E(X+Y) &= \sum_x \sum_y (x+y) P(X=x, Y=y) \\
&= \sum_x \sum_y x P(X=x, Y=y) + \sum_x \sum_y y P(X=x, Y=y) \\
&= \sum_x x \left( \sum_y P(X=x, Y=y) \right) + \sum_y y \left( \sum_x P(X=x, Y=y) \right)
\end{align*}
Par la loi des probabilités totales (ou "marginalisation"), la somme interne $\sum_y P(X=x, Y=y)$ est simplement $P(X=x)$. De même, $\sum_x P(X=x, Y=y) = P(Y=y)$.
$$ E(X+Y) = \sum_x x P(X=x) + \sum_y y P(Y=y) = E(X) + E(Y) $$
Notez que l'indépendance n'a jamais été requise pour cette preuve.
\end{proofbox}

Cette propriété est incroyablement utile.

\begin{intuitionbox}
Cette propriété formalise une idée très simple : "la moyenne d'une somme est la somme des moyennes". Si vous jouez à deux jeux de hasard, votre gain moyen total est simplement la somme de ce que vous gagnez en moyenne à chaque jeu, que les jeux soient liés ou non.
\end{intuitionbox}

Cette propriété rend le calcul de l'espérance d'une somme trivial, comme le montre l'exemple des deux dés.

\begin{examplebox}[Somme de deux dés]
Soit $X_1$ le résultat du premier dé et $X_2$ celui du second. On sait que $E(X_1) = 3.5$ et $E(X_2) = 3.5$.
Soit $S = X_1 + X_2$ la somme des deux dés. Grâce à la linéarité, on peut calculer l'espérance de la somme sans avoir à lister les 36 résultats possibles :
$$ E(S) = E(X_1 + X_2) = E(X_1) + E(X_2) = 3.5 + 3.5 = 7 $$
\end{examplebox}

\subsection{Espérance de la loi binomiale}

Nous pouvons maintenant utiliser cette puissante propriété de linéarité pour trouver l'espérance de nos distributions de référence, en évitant des sommes complexes.

\begin{theorembox}[Espérance de la loi binomiale]
Si $X \sim \text{Bin}(n, p)$, alors son espérance est $E(X) = np$.
\end{theorembox}

Ce résultat est profondément intuitif.

\begin{intuitionbox}
Ce résultat est très naturel. Si vous lancez une pièce 100 fois ($n=100$) avec une probabilité de 50\% d'obtenir Pile ($p=0.5$), vous vous attendez en moyenne à obtenir $100 \times 0.5 = 50$ Piles. La formule $np$ généralise cette idée.
\end{intuitionbox}

La preuve formelle est un exemple parfait de l'élégance de la linéarité, utilisant les variables indicatrices.

\begin{proofbox}
Le calcul direct de l'espérance avec la PMF binomiale est possible, mais long. En utilisant la linéarité de l'espérance, on obtient une preuve beaucoup plus courte et élégante.

On peut voir une variable binomiale $X$ comme la somme de $n$ variables de Bernoulli indépendantes, $X = I_1 + I_2 + \dots + I_n$, où chaque $I_j$ représente le succès (1) ou l'échec (0) du $j$-ième essai.

Chaque $I_j$ a pour espérance $E(I_j) = 1 \cdot p + 0 \cdot (1-p) = p$.

Par linéarité de l'espérance, on a :
$$ E(X) = E(I_1) + E(I_2) + \dots + E(I_n) = \underbrace{p + p + \dots + p}_{n \text{ fois}} = np $$
\end{proofbox}

\subsection{Espérance de la loi géométrique}

Calculons maintenant l'espérance pour la loi qui modélise le temps d'attente.

\begin{theorembox}[Espérance de la loi géométrique]
L'espérance d'une variable aléatoire $X \sim \text{Geom}(p)$ (comptant le nombre d'échecs) est :
$$ E(X) = \frac{1-p}{p} = \frac{q}{p} $$
\end{theorembox}

L'intuition est aussi très forte ici :

\begin{intuitionbox}
Si un événement a 1 chance sur 10 de se produire ($p=0.1$), il est logique de penser qu'il faudra en moyenne 9 échecs ($q/p = 0.9/0.1=9$) avant qu'il ne se produise. L'espérance du nombre total d'essais (échecs + 1 succès) serait alors $1/p$.
\end{intuitionbox}

Contrairement à la loi binomiale, la preuve la plus directe ne repose pas sur la linéarité mais sur une manipulation de séries.

\begin{proofbox}[Démonstration de l'espérance géométrique via les séries entières]
Soit $X \sim \text{Geom}(p)$, où $X$ compte le nombre d'échecs avant le premier succès. La PMF est $P(X=k) = q^k p$ pour $k=0, 1, 2, \dots$, avec $q=1-p$.

Par définition, l'espérance est :
$$ E(X) = \sum_{k=0}^{\infty} k \cdot P(X=k) = \sum_{k=0}^{\infty} k q^k p $$
Le terme pour $k=0$ est nul, on peut donc commencer la somme à $k=1$ :
$$ E(X) = p \sum_{k=1}^{\infty} k q^k $$
L'astuce consiste à reconnaître que la somme ressemble à la dérivée d'une série géométrique. Rappelons la formule de la série géométrique pour $|q|<1$ :
$$ \sum_{k=0}^{\infty} q^k = \frac{1}{1-q} $$
En dérivant les deux côtés par rapport à $q$, on obtient :
$$ \frac{d}{dq} \left( \sum_{k=0}^{\infty} q^k \right) = \frac{d}{dq} \left( \frac{1}{1-q} \right) $$
$$ \sum_{k=1}^{\infty} k q^{k-1} = \frac{1}{(1-q)^2} $$
Pour faire apparaître ce terme dans notre formule d'espérance, on factorise $q$ dans la somme :
$$ E(X) = p \cdot q \sum_{k=1}^{\infty} k q^{k-1} $$
On peut maintenant remplacer la somme par son expression analytique :
$$ E(X) = p \cdot q \cdot \frac{1}{(1-q)^2} $$
Puisque $p = 1-q$, on a :
$$ E(X) = p \cdot q \cdot \frac{1}{p^2} = \frac{q}{p} $$
Ce qui démontre que l'espérance du nombre d'échecs avant le premier succès est $\frac{q}{p}$.
\end{proofbox}

\subsection{Loi du statisticien inconscient (LOTUS)}

Souvent, nous ne sommes pas intéressés par l'espérance de $X$ elle-même, mais par l'espérance d'une fonction de $X$, par exemple $E(X^2)$ ou $E(e^X)$.

\begin{theorembox}[Théorème de Transfert (LOTUS)]
Si $X$ est une variable aléatoire discrète et $g(x)$ est une fonction de $\mathbb{R}$ dans $\mathbb{R}$, alors l'espérance de la variable aléatoire $g(X)$ est donnée par :
$$ E[g(X)] = \sum_x g(x) P(X=x) $$
La somme porte sur toutes les valeurs possibles de $X$. Ce théorème est utile car il évite d'avoir à trouver la PMF de $g(X)$.
\end{theorembox}

La preuve dans le cas discret consiste simplement à regrouper les termes.

\begin{proofbox}
Soit $Y = g(X)$. Par définition, l'espérance de $Y$ est $E(Y) = \sum_y y P(Y=y)$.
L'ensemble des valeurs $y$ que $Y$ peut prendre est $\{g(x) \mid x \in \text{support de } X\}$.
Pour une valeur $y$ donnée, l'événement $\{Y=y\}$ est l'union de tous les événements $\{X=x\}$ tels que $g(x)=y$.
$$ P(Y=y) = P(g(X)=y) = \sum_{x: g(x)=y} P(X=x) $$
En substituant cela dans la définition de $E(Y)$ :
$$ E(Y) = \sum_y y \left( \sum_{x: g(x)=y} P(X=x) \right) $$
On peut réécrire $y$ comme $g(x)$ à l'intérieur de la seconde somme :
$$ E(g(X)) = \sum_y \sum_{x: g(x)=y} g(x) P(X=x) $$
Cette double somme parcourt toutes les valeurs de $y$, et pour chaque $y$, elle parcourt tous les $x$ correspondants. Cela revient à simplement sommer sur tous les $x$ possibles dès le départ :
$$ E[g(X)] = \sum_x g(x) P(X=x) $$
\end{proofbox}

Ce théorème justifie son nom : c'est ce que l'on ferait "inconsciemment".

\begin{intuitionbox}
Pour trouver la valeur moyenne d'une fonction d'une variable aléatoire (par exemple, le carré du résultat d'un dé), vous n'avez pas besoin de déterminer d'abord la distribution de ce carré. Vous pouvez simplement prendre chaque valeur possible du résultat original, lui appliquer la fonction, et pondérer ce nouveau résultat par la probabilité du résultat original.
\end{intuitionbox}

Utilisons ce théorème pour calculer $E(X^2)$ pour notre dé.

\begin{examplebox}[Calcul de $E(X^2)$ pour un dé]
Soit $X$ le résultat d'un lancer de dé. Calculons l'espérance de $Y=X^2$. La fonction est $g(x)=x^2$.
\begin{align*}
E(X^2) &= \sum_{k=1}^6 k^2 P(X=k) \\
&= 1^2\left(\frac{1}{6}\right) + 2^2\left(\frac{1}{6}\right) + 3^2\left(\frac{1}{6}\right) + 4^2\left(\frac{1}{6}\right) + 5^2\left(\frac{1}{6}\right) + 6^2\left(\frac{1}{6}\right) \\
&= \frac{1+4+9+16+25+36}{6} = \frac{91}{6} \approx 15.17
\end{align*}
\end{examplebox}

\subsection{Variance}

L'espérance nous donne le centre d'une distribution, mais elle ne dit rien sur sa "largeur" ou sa "dispersion". C'est le rôle de la variance.

\begin{definitionbox}[Variance et écart-type]
La \textbf{variance} d'une variable aléatoire $X$ mesure la dispersion de sa distribution autour de son espérance. Elle est définie par :
$$ \text{Var}(X) = E\left[ (X - E(X))^2 \right] $$
La racine carrée de la variance est appelée l' \textbf{écart-type} :
$$ \text{SD}(X) = \sqrt{\text{Var}(X)} $$
\end{definitionbox}

L'idée est de mesurer l'écart quadratique moyen à l'espérance.

\begin{intuitionbox}
La variance est la "distance carrée moyenne à la moyenne". On prend l'écart de chaque valeur par rapport à la moyenne, on le met au carré (pour que les écarts positifs et négatifs ne s'annulent pas), puis on en calcule la moyenne. L'écart-type est souvent plus interprétable car il ramène cette mesure de dispersion dans les mêmes unités que la variable aléatoire elle-même.
\end{intuitionbox}

La définition $E[(X-E(X))^2]$ est excellente pour l'interprétation, mais pénible pour le calcul. Une formule alternative est presque toujours utilisée.

\begin{theorembox}[Formule de calcul de la variance]
Pour toute variable aléatoire $X$, une formule plus pratique pour le calcul de la variance est :
$$ \text{Var}(X) = E(X^2) - [E(X)]^2 $$
\end{theorembox}

La preuve est une simple expansion algébrique utilisant la linéarité de l'espérance.

\begin{proofbox}
Soit $\mu = E(X)$. On part de la définition de la variance :
\begin{align*}
\text{Var}(X) &= E[ (X - \mu)^2 ] \\
&= E[ X^2 - 2X\mu + \mu^2 ] \quad \text{(On développe le carré)} \\
&= E(X^2) - E(2\mu X) + E(\mu^2) \quad \text{(Par linéarité de l'espérance)} \\
&= E(X^2) - 2\mu E(X) + \mu^2 \quad \text{(Car $2\mu$ et $\mu^2$ sont des constantes)} \\
&= E(X^2) - 2\mu(\mu) + \mu^2 \quad \text{(Car $E(X) = \mu$)} \\
&= E(X^2) - 2\mu^2 + \mu^2 \\
&= E(X^2) - \mu^2 = E(X^2) - [E(X)]^2
\end{align*}
\end{proofbox}

Nous pouvons maintenant calculer la variance de notre lancer de dé.

\begin{examplebox}[Variance d'un lancer de dé]
Nous avons déjà calculé pour un dé que $E(X) = 3.5$ et $E(X^2) = 91/6$. On peut maintenant trouver la variance facilement :
$$ \text{Var}(X) = E(X^2) - [E(X)]^2 = \frac{91}{6} - (3.5)^2 = \frac{91}{6} - 12.25 = 15.166... - 12.25 \approx 2.917 $$
L'écart-type est $\text{SD}(X) = \sqrt{2.917} \approx 1.708$.
\end{examplebox}

\subsection{Exercices}

% --- Espérance de base et LOTUS ---

\begin{exercicebox}[Exercice 1 : Calcul d'Espérance (PMF Simple)]
Une variable aléatoire $X$ a la distribution de probabilité suivante :
$P(X=-1) = 0.3$, $P(X=0) = 0.5$, $P(X=2) = 0.2$.
Calculez l'espérance $E(X)$.
\end{exercicebox}

\begin{exercicebox}[Exercice 2 : LOTUS (Calcul de $E(X^2)$)]
En utilisant la même variable aléatoire $X$ que dans l'exercice 1, calculez $E(X^2)$.
\end{exercicebox}

\begin{exercicebox}[Exercice 3 : Variance (Calcul de base)]
En utilisant les résultats des exercices 1 et 2, calculez la variance $\text{Var}(X)$.
\end{exercicebox}

\begin{exercicebox}[Exercice 4 : Espérance (Jeu Simple)]
Un jeu consiste à payer 2 pour lancer un dé à 6 faces. Si le dé tombe sur 6, vous gagnez 10. Sinon, vous ne gagnez rien. Soit $G$ votre gain net (gain - mise).
\begin{enumerate}
    \item Quelle est la PMF de $G$ ?
    \item Calculez $E(G)$. Le jeu est-il favorable au joueur ?
\end{enumerate}
\end{exercicebox}

\begin{exercicebox}[Exercice 5 : Variance (Jeu Simple)]
En utilisant la variable aléatoire $G$ de l'exercice 4 :
\begin{enumerate}
    \item Calculez $E(G^2)$.
    \item Calculez $\text{Var}(G)$.
\end{enumerate}
\end{exercicebox}

\begin{exercicebox}[Exercice 6 : Espérance de Bernoulli]
Soit $X$ une variable aléatoire $X \sim \text{Bern}(p)$ (variable indicatrice). En utilisant la définition de l'espérance, montrez que $E(X) = p$.
\end{exercicebox}

\begin{exercicebox}[Exercice 7 : Variance de Bernoulli]
En utilisant le résultat de l'exercice 6 et le théorème de LOTUS, montrez que $\text{Var}(X) = p(1-p)$ pour $X \sim \text{Bern}(p)$. (Indice : $X^2 = X$ pour une variable de Bernoulli).
\end{exercicebox}

% --- Linéarité de l'Espérance ---

\begin{exercicebox}[Exercice 8 : Linéarité (Simple)]
Soient $X$ et $Y$ deux variables aléatoires. On sait que $E(X) = 10$ et $E(Y) = -5$.
Calculez $E(3X - 2Y + 4)$.
\end{exercicebox}

\begin{exercicebox}[Exercice 9 : Linéarité (Trois Dés)]
On lance trois dés équilibrés à 6 faces. Soit $S$ la somme des trois résultats.
En utilisant la linéarité de l'espérance, calculez $E(S)$.
\end{exercicebox}

\begin{exercicebox}[Exercice 10 : Linéarité (Somme de Bernoulli)]
Soit $X \sim \text{Bin}(n, p)$. On rappelle que $X$ peut s'écrire comme la somme de $n$ variables de Bernoulli indépendantes $X = I_1 + \dots + I_n$, où $E(I_j) = p$.
Utilisez la linéarité de l'espérance pour prouver que $E(X) = np$.
\end{exercicebox}

% --- Espérances des Lois Classiques ---

\begin{exercicebox}[Exercice 11 : Espérance Binomiale (Application)]
Un QCM (questionnaire à choix multiples) comporte 40 questions. Chaque question a 4 options de réponse, dont une seule est correcte. Un étudiant répond à tout au hasard.
Quel est le nombre attendu (l'espérance) de bonnes réponses ?
\end{exercicebox}

\begin{exercicebox}[Exercice 12 : Espérance Géométrique (Application)]
On lance une paire de dés équilibrés. Un "succès" est d'obtenir un double-six.
\begin{enumerate}
    \item Quelle est la probabilité $p$ d'un succès ?
    \item Soit $X$ le nombre d'échecs avant le premier double-six. Quelle est l'espérance $E(X)$ ?
\end{enumerate}
\end{exercicebox}

\begin{exercicebox}[Exercice 13 : Espérance Géométrique (Attente Totale)]
En reprenant la situation de l'exercice 12 ($p=1/36$), soit $Y$ le \textit{nombre total de lancers} nécessaires pour obtenir le premier double-six ($Y = X + 1$).
Calculez $E(Y)$.
\end{exercicebox}

\begin{exercicebox}[Exercice 14 : Espérance (Loi Hypergéométrique)]
On tire 5 cartes d'un jeu de 52 cartes sans remise. Soit $X$ le nombre d'As tirés. On peut écrire $X = I_1 + I_2 + I_3 + I_4 + I_5$, où $I_j=1$ si la $j$-ème carte tirée est un As, et 0 sinon.
\begin{enumerate}
    \item Quelle est la probabilité $P(I_1 = 1)$ (que la 1ère carte soit un As) ?
    \item Quelle est la probabilité $P(I_2 = 1)$ (que la 2ème carte soit un As) ? (Indice : Pensez par symétrie ou utilisez la LTP).
    \item Calculez $E(X)$ en utilisant la linéarité.
\end{enumerate}
\end{exercicebox}

% --- Variance et E[X^2] ---

\begin{exercicebox}[Exercice 15 : Espérance et Variance (Dé à 4 faces)]
Soit $X$ le résultat d'un lancer de dé équilibré à 4 faces ($X \in \{1, 2, 3, 4\}$).
\begin{enumerate}
    \item Calculez $E(X)$.
    \item Calculez $E(X^2)$.
    \item Calculez $\text{Var}(X)$.
\end{enumerate}
\end{exercicebox}

\begin{exercicebox}[Exercice 16 : Formule de la Variance (Inverse)]
Une variable aléatoire $Y$ a une espérance $E(Y) = 5$ et une variance $\text{Var}(Y) = 4$.
Quelle est la valeur de $E(Y^2)$ ?
\end{exercicebox}

\begin{exercicebox}[Exercice 17 : Formule de la Variance (Inverse 2)]
Une variable aléatoire $W$ a $E(W^2) = 50$ et $\text{Var}(W) = 1$.
Quelles sont les deux valeurs possibles pour $E(W)$ ?
\end{exercicebox}

\begin{exercicebox}[Exercice 18 : Variance Nulle]
Une variable aléatoire $X$ a une variance $\text{Var}(X) = 0$. Que pouvez-vous conclure sur la distribution de $X$ ?
(Indice : $\text{Var}(X) = E[(X-\mu)^2]$).
\end{exercicebox}

\begin{exercicebox}[Exercice 19 : LOTUS et Linéarité]
Soit $X$ une variable aléatoire avec $E(X)=3$ et $E(X^2)=10$.
Calculez $E[(X+1)^2]$.
(Indice : Développez $(X+1)^2$ avant de prendre l'espérance).
\end{exercicebox}

\begin{exercicebox}[Exercice 20 : Synthèse (Jeu de Roulette)]
À la roulette, vous misez 1 sur "Rouge". Il y a 18 cases rouges, 18 noires, et 1 verte (le 0). Total = 37 cases.
Si "Rouge" sort, vous récupérez votre mise de 1 et gagnez 1 de plus (gain net $G=+1$).
Si "Noir" or "Vert" sort, vous perdez votre mise (gain net $G=-1$).
\begin{enumerate}
    \item Calculez $E(G)$.
    \item Calculez $E(G^2)$.
    \item Calculez $\text{Var}(G)$.
\end{enumerate}
\end{exercicebox}

\subsection{Corrections des Exercices}

% --- Corrections : Concepts de Base (PMF, CDF) ---

\begin{correctionbox}[Correction Exercice 1 : Identification de Variables Aléatoires]
1.  \textbf{Discrète}. $X$ ne peut prendre que des valeurs entières $\{0, 1, \dots, 10\}$.
2.  \textbf{Continue}. Le temps peut prendre n'importe quelle valeur dans un intervalle (par ex. $T \in [2.5, 5]$ heures).
3.  \textbf{Discrète}. $X$ ne peut prendre que des valeurs entières $\{0, 1, 2, \dots\}$.
4.  \textbf{Continue}. La température peut prendre n'importe quelle valeur dans un intervalle (par ex. $T \in [15.0, 25.0]^\circ\text{C}$).
5.  \textbf{Discrète}. $X$ ne peut prendre que des valeurs entières $\{0, 1, 2, \dots\}$ (si on compte les échecs) ou $\{1, 2, 3, \dots\}$ (si on compte les lancers).
\end{correctionbox}

\begin{correctionbox}[Correction Exercice 2 : Construction d'une PMF]
On lance un dé à 4 faces (1, 2, 3, 4). $X$ est le résultat.
1.  Valeurs possibles : $S_X = \{1, 2, 3, 4\}$.
2.  PMF : Le dé est équilibré, donc chaque face a la même probabilité $1/4$.
    $P(X=1) = 1/4$
    $P(X=2) = 1/4$
    $P(X=3) = 1/4$
    $P(X=4) = 1/4$
    Et $P(X=k) = 0$ pour tout autre $k$.
3.  Vérification : $\sum P(X=k) = 1/4 + 1/4 + 1/4 + 1/4 = 4/4 = 1$.
\end{correctionbox}

\begin{correctionbox}[Correction Exercice 3 : PMF d'une Somme]
$Y = D_1 + D_2$, où $D_1, D_2 \in \{1, 2, 3, 4\}$. Il y a $4 \times 4 = 16$ issues équiprobables (prob. 1/16 chacune).
1.  Valeurs possibles : Min = $1+1=2$. Max = $4+4=8$. $S_Y = \{2, 3, 4, 5, 6, 7, 8\}$.
2.  PMF (en comptant les issues favorables sur 16) :
    - $P(Y=2) = P(1,1) \implies 1/16$
    - $P(Y=3) = P(1,2) + P(2,1) \implies 2/16$
    - $P(Y=4) = P(1,3) + P(2,2) + P(3,1) \implies 3/16$
    - $P(Y=5) = P(1,4) + P(2,3) + P(3,2) + P(4,1) \implies 4/16$
    - $P(Y=6) = P(2,4) + P(3,3) + P(4,2) \implies 3/16$
    - $P(Y=7) = P(3,4) + P(4,3) \implies 2/16$
    - $P(Y=8) = P(4,4) \implies 1/16$
    (Vérification : $1+2+3+4+3+2+1 = 16$. La somme est $16/16 = 1$).
\end{correctionbox}

\begin{correctionbox}[Correction Exercice 4 : Construction d'une CDF]
On utilise la PMF de l'exercice 3. $F_Y(y) = P(Y \le y)$.
1.  CDF aux points de masse :
    - $F_Y(2) = P(Y \le 2) = P(Y=2) = 1/16$
    - $F_Y(3) = P(Y \le 3) = P(Y=2)+P(Y=3) = 1/16 + 2/16 = 3/16$
    - $F_Y(4) = P(Y \le 4) = 3/16 + P(Y=4) = 3/16 + 3/16 = 6/16$
    - $F_Y(5) = P(Y \le 5) = 6/16 + P(Y=5) = 6/16 + 4/16 = 10/16$
    - $F_Y(6) = P(Y \le 6) = 10/16 + P(Y=6) = 10/16 + 3/16 = 13/16$
    - $F_Y(7) = P(Y \le 7) = 13/16 + P(Y=7) = 13/16 + 2/16 = 15/16$
    - $F_Y(8) = P(Y \le 8) = 15/16 + P(Y=8) = 15/16 + 1/16 = 16/16 = 1$
2.  $F_Y(1.5) = P(Y \le 1.5) = 0$ (car la valeur minimale est 2).
3.  $F_Y(5.2) = P(Y \le 5.2) = P(Y \le 5) = F_Y(5) = 10/16$.
4.  $F_Y(10) = P(Y \le 10) = 1$ (car la valeur maximale est 8).
\end{correctionbox}

% --- Corrections : Loi de Bernoulli et Loi Binomiale ---

\begin{correctionbox}[Correction Exercice 5 : Loi de Bernoulli]
1.  $X$ suit une \textbf{loi de Bernoulli}. Le paramètre est $p=0.05$. On note $X \sim \text{Bern}(0.05)$.
2.  La PMF est :
    $P(X=1) = p = 0.05$ (succès = défectueux)
    $P(X=0) = 1-p = 0.95$ (échec = non défectueux)
\end{correctionbox}

\begin{correctionbox}[Correction Exercice 6 : Loi Binomiale (Calcul Direct)]
1.  $X$ est le nombre de succès (Pile) en $n=5$ essais indépendants avec probabilité $p=0.7$.
    $X$ suit une \textbf{loi Binomiale}. $X \sim \text{Bin}(n=5, p=0.7)$.
2.  $P(X=k) = \binom{n}{k} p^k (1-p)^{n-k}$.
    $P(X=3) = \binom{5}{3} (0.7)^3 (1-0.7)^{5-3} = 10 \times (0.343) \times (0.3)^2 = 10 \times 0.343 \times 0.09 = 0.3087$.
3.  $P(X=5) = \binom{5}{5} (0.7)^5 (0.3)^0 = 1 \times (0.7)^5 \times 1 = 0.16807$.
\end{correctionbox}

\begin{correctionbox}[Correction Exercice 7 : Loi Binomiale (Calcul Cumulé)]
On a $X \sim \text{Bin}(5, 0.7)$.
1.  $P(X=0) = \binom{5}{0} (0.7)^0 (0.3)^5 = 1 \times 1 \times (0.3)^5 = 0.00243$.
2.  L'événement "au moins 1 Pile" ($X \ge 1$) est le complémentaire de "0 Pile" ($X=0$).
    $P(X \ge 1) = 1 - P(X=0) = 1 - 0.00243 = 0.99757$.
\end{correctionbox}

\begin{correctionbox}[Correction Exercice 8 : Problème Binomial (Contrôle Qualité)]
Le tirage est \textit{avec remise}, donc les essais sont indépendants. C'est une loi binomiale.
$n = 20$ (nombre d'essais).
$p = 0.10$ (probabilité de succès = défectueux).
On cherche $P(X=2)$.
$P(X=2) = \binom{20}{2} (0.1)^2 (1-0.1)^{20-2}$
$P(X=2) = \frac{20 \times 19}{2} (0.1)^2 (0.9)^{18} = 190 \times 0.01 \times (0.9)^{18}$
$P(X=2) = 1.9 \times (0.9)^{18} \approx 1.9 \times 0.15009 \approx 0.2852$.
\end{correctionbox}

% --- Corrections : Loi Hypergéométrique ---

\begin{correctionbox}[Correction Exercice 9 : Loi Hypergéométrique (Urne)]
Le tirage est \textit{sans remise} d'une population finie.
1.  $X$ suit une \textbf{loi Hypergéométrique}.
    Paramètres : $w=7$ (blanches, succès), $b=5$ (noires, échecs), $m=4$ (nombre de tirages).
    $X \sim \text{HG}(w=7, b=5, m=4)$.
2.  On cherche $P(X=2)$.
    $P(X=k) = \frac{\binom{w}{k} \binom{b}{m-k}}{\binom{w+b}{m}}$
    $P(X=2) = \frac{\binom{7}{2} \binom{5}{4-2}}{\binom{12}{4}} = \frac{\binom{7}{2} \binom{5}{2}}{\binom{12}{4}}$
    $P(X=2) = \frac{(\frac{7 \times 6}{2}) \times (\frac{5 \times 4}{2})}{(\frac{12 \times 11 \times 10 \times 9}{4 \times 3 \times 2 \times 1})} = \frac{21 \times 10}{495} = \frac{210}{495} = \frac{14}{33} \approx 0.4242$.
\end{correctionbox}

\begin{correctionbox}[Correction Exercice 10 : Problème Hypergéométrique (Comité)]
Tirage sans remise. C'est une loi Hypergéométrique.
$w=10$ (hommes), $b=8$ (femmes), $m=6$ (taille du comité). Total $N=18$.
On cherche $P(X=3)$ (exactement 3 hommes, ce qui implique $m-k = 6-3=3$ femmes).
$P(X=3) = \frac{\binom{10}{3} \binom{8}{3}}{\binom{18}{6}}$
$P(X=3) = \frac{(\frac{10 \times 9 \times 8}{3 \times 2 \times 1}) \times (\frac{8 \times 7 \times 6}{3 \times 2 \times 1})}{(\frac{18 \times 17 \times 16 \times 15 \times 14 \times 13}{6 \times 5 \times 4 \times 3 \times 2 \times 1})} = \frac{120 \times 56}{18564} = \frac{6720}{18564} \approx 0.362$.
\end{correctionbox}

\begin{correctionbox}[Correction Exercice 11 : Binomiale vs Hypergéométrique]
Population totale $N=10000$. 10\% défectueux, donc $w=1000$ (défectueux), $b=9000$ (non défectueux).
Tirage de $m=20$ \textit{sans remise}.
1.  Loi exacte : \textbf{Loi Hypergéométrique}.
    $X \sim \text{HG}(w=1000, b=9000, m=20)$.
2.  Probabilité exacte $P(X=2)$ :
    $P(X=2) = \frac{\binom{1000}{2} \binom{9000}{18}}{\binom{10000}{20}}$
    $P(X=2) = \frac{(\frac{1000 \times 999}{2}) \times (\frac{9000 \times \dots \times 8983}{18!})}{(\frac{10000 \times \dots \times 9981}{20!})} \approx 0.2854$.
    (Le calcul est très complexe, mais on peut montrer qu'il est très proche de la binomiale).
3.  Le résultat de l'exercice 8 (Binomiale) était $\approx 0.2852$.
    L'approximation binomiale est excellente. La raison est que la taille de l'échantillon ($m=20$) est très petite par rapport à la taille de la population ($N=10000$). Le fait de ne pas remettre les 20 articles change à peine les probabilités pour les tirages suivants.
\end{correctionbox}

% --- Corrections : Loi Géométrique ---

\begin{correctionbox}[Correction Exercice 12 : Loi Géométrique (Calcul Direct)]
1.  $X$ est le nombre d'échecs avant le premier succès. $X$ suit une \textbf{loi Géométrique}.
    Le succès est "obtenir 6", donc $p = 1/6$. $X \sim \text{Geom}(p=1/6)$.
2.  "Premier 6 au 3ème lancer" signifie 2 échecs (lancers 1 et 2) puis 1 succès (lancer 3).
    C'est $P(X=2)$. $q = 1-p = 5/6$.
    $P(X=2) = q^2 p^1 = (5/6)^2 (1/6) = 25/216 \approx 0.1157$.
3.  "Premier 6 au 1er lancer" signifie 0 échec. C'est $P(X=0)$.
    $P(X=0) = q^0 p^1 = 1 \times (1/6) = 1/6$.
\end{correctionbox}

\begin{correctionbox}[Correction Exercice 13 : Loi Géométrique (Calcul Cumulé)]
$p=0.2$ (succès), $q=0.8$ (échec). $X$ compte les échecs. $X \sim \text{Geom}(0.2)$.
1.  "Exactement 4 tirs au total" signifie 3 échecs suivis d'un succès. On cherche $P(X=3)$.
    $P(X=3) = q^3 p^1 = (0.8)^3 (0.2) = 0.512 \times 0.2 = 0.1024$.
2.  "Plus de 2 tirs au total" signifie qu'il faut au moins 3 tirs. C'est l'événement "les 2 premiers tirs sont des échecs".
    La probabilité est $P(\text{Echec 1} \cap \text{Echec 2}) = q \times q = q^2$.
    $P(X \ge 2) = (0.8)^2 = 0.64$.
\end{correctionbox}

\begin{correctionbox}[Correction Exercice 14 : Variante de la Loi Géométrique]
$Y$ est le nombre total d'essais ($k=1, 2, 3, \dots$). $p$ est la prob. de succès.
1.  Pour que $Y=k$, il faut $k-1$ échecs, suivis d'un succès.
    $P(Y=k) = (1-p)^{k-1} p = q^{k-1} p$, pour $k=1, 2, \dots$
2.  Avec $p=1/6$, on cherche $P(Y=3)$.
    $P(Y=3) = (5/6)^{3-1} (1/6) = (5/6)^2 (1/6) = 25/216$.
    C'est le même résultat que $P(X=2)$ de l'exercice 12. Les deux définitions décrivent la même situation (3 lancers au total).
\end{correctionbox}

% --- Corrections : Loi de Poisson ---

\begin{correctionbox}[Correction Exercice 15 : Loi de Poisson (Calcul Direct)]
$X \sim \text{Poisson}(\lambda=5)$. PMF : $P(X=k) = \frac{e^{-\lambda} \lambda^k}{k!}$.
1.  $P(X=0) = \frac{e^{-5} 5^0}{0!} = \frac{e^{-5} \times 1}{1} = e^{-5} \approx 0.0067$.
2.  $P(X=5) = \frac{e^{-5} 5^5}{5!} = \frac{e^{-5} \times 3125}{120} = e^{-5} \times \frac{625}{24} \approx 26.04 \times e^{-5} \approx 0.1755$.
\end{correctionbox}

\begin{correctionbox}[Correction Exercice 16 : Loi de Poisson (Calcul Cumulé)]
$X \sim \text{Poisson}(\lambda=2)$. On cherche $P(X \le 2)$.
$P(X \le 2) = P(X=0) + P(X=1) + P(X=2)$
$P(X=0) = \frac{e^{-2} 2^0}{0!} = e^{-2}$
$P(X=1) = \frac{e^{-2} 2^1}{1!} = 2e^{-2}$
$P(X=2) = \frac{e^{-2} 2^2}{2!} = \frac{4e^{-2}}{2} = 2e^{-2}$
$P(X \le 2) = e^{-2} + 2e^{-2} + 2e^{-2} = 5e^{-2} \approx 5 \times 0.1353 = 0.6767$.
\end{correctionbox}

\begin{correctionbox}[Correction Exercice 17 : Loi de Poisson (Changement de $\lambda$)]
1.  Pour une page, $X \sim \text{Poisson}(\lambda=0.5)$.
    $P(X=0) = \frac{e^{-0.5} (0.5)^0}{0!} = e^{-0.5} \approx 0.6065$.
2.  Si le taux est 0.5 faute/page, le taux pour 10 pages est $\lambda_Y = 0.5 \times 10 = 5$.
    $Y \sim \text{Poisson}(\lambda_Y=5)$.
3.  On cherche $P(Y=0)$.
    $P(Y=0) = \frac{e^{-5} 5^0}{0!} = e^{-5} \approx 0.0067$.
\end{correctionbox}

\begin{correctionbox}[Correction Exercice 18 : Approximation Binomiale par Poisson]
1.  C'est un tirage de $n=10000$ clients, où chaque client est un essai de Bernoulli avec $p=0.0003$. La loi exacte est $X \sim \text{Bin}(10000, 0.0003)$.
2.  Le paramètre $\lambda$ pour l'approximation Poisson est $\lambda = np = 10000 \times 0.0003 = 3$.
3.  On utilise $Y \sim \text{Poisson}(\lambda=3)$ pour approximer $X$.
    $P(X=2) \approx P(Y=2) = \frac{e^{-3} 3^2}{2!} = \frac{9e^{-3}}{2} = 4.5 e^{-3} \approx 4.5 \times 0.04979 \approx 0.224$.
\end{correctionbox}

% --- Corrections : Synthèse et Variables Indicatrices ---

\begin{correctionbox}[Correction Exercice 19 : Choisir la Bonne Loi]
1.  Tirage sans remise d'une population finie : \textbf{Loi Hypergéométrique}.
2.  Comptage d'événements sur un intervalle de temps fixe : \textbf{Loi de Poisson}.
3.  Comptage d'essais jusqu'au premier succès : \textbf{Loi Géométrique}.
4.  Comptage de succès sur un nombre fixe d'essais indépendants : \textbf{Loi Binomiale}.
5.  Comptage d'événements rares sur un intervalle (temps/espace) : \textbf{Loi de Poisson}.
\end{correctionbox}

\begin{correctionbox}[Correction Exercice 20 : Variable Indicatrice]
$A$ = "obtenir 6". $P(A) = 1/6$.
$I_A = 1$ si $A$ se produit, $I_A = 0$ sinon.
1.  C'est une expérience avec deux issues (succès/échec). $I_A$ suit une \textbf{Loi de Bernoulli}.
    Le paramètre est $p = P(A) = 1/6$. $I_A \sim \text{Bern}(1/6)$.
2.  La PMF de $I_A$ est :
    $P(I_A = 1) = p = 1/6$
    $P(I_A = 0) = 1-p = 5/6$
\end{correctionbox}

\subsection{Exercices Pratiques (Python)}

Ces exercices vous aideront à calculer et à vérifier empiriquement les concepts d'espérance et de variance en utilisant des simulations.

Pour ces exercices, vous aurez besoin de la bibliothèque \texttt{numpy}.

\begin{codecell}
pip install numpy
\end{codecell}

\begin{exercicebox}[Exercice 1 : $E(X)$ $E(X^2)$ et Variance (Dé)]
Nous allons simuler $N$ lancers d'un dé à 6 faces pour vérifier empiriquement la définition de l'espérance, le théorème LOTUS, et la formule de calcul de la variance.

\textbf{Votre tâche :}
\begin{enumerate}
    \item Simulez 100 000 lancers d'un dé équilibré (valeurs de 1 à 6) et stockez les résultats dans un tableau NumPy.
    \item Calculez l'espérance empirique $E(X)$ en prenant la moyenne du tableau.
    \item En utilisant LOTUS, calculez l'espérance empirique $E(X^2)$ (en créant un nouveau tableau des carrés, puis en prenant sa moyenne).
    \item Calculez la variance empirique en utilisant la formule : $\text{Var}(X) = E(X^2) - [E(X)]^2$.
    \item Comparez votre résultat à la variance calculée directement avec \texttt{numpy.var()}.
\end{enumerate}

\begin{codecell}
import numpy as np

N_simulations = 100000

# 1. Simuler N lancers d'un de a 6 faces
# lancers = ...

# 2. Calculer E(X) (moyenne empirique)
# E_X = ...
# print(f"E(X) empirique: {E_X:.4f} (Theorique: 3.5)")

# 3. Calculer E(X^2) (LOTUS)
# lancers_carres = ...
# E_X2 = ...
# print(f"E(X^2) empirique: {E_X2:.4f} (Theorique: 91/6 = 15.1667)")

# 4. Calculer Var(X) avec la formule
# var_calc = ...
# print(f"Variance (calculee): {var_calc:.4f}")

# 5. Calculer Var(X) avec la fonction numpy
# var_np = ...
# print(f"Variance (numpy.var): {var_np:.4f}")
# print(f"Difference: {np.abs(var_calc - var_np):.6f}")
\end{codecell}
\end{exercicebox}

\begin{exercicebox}[Exercice 2 : Linearite de l'Esperance]
Vérifions empiriquement que $E(X+Y) = E(X) + E(Y)$. Nous allons simuler deux variables aléatoires différentes : $X$ (un dé à 4 faces) et $Y$ (un dé à 6 faces).

\textbf{Votre tâche :}
\begin{enumerate}
    \item Simulez $N=100000$ lancers d'un dé à 4 faces ($X$).
    \item Simulez $N=100000$ lancers d'un dé à 6 faces ($Y$).
    \item Créez la variable aléatoire $Z = X + Y$.
    \item Calculez les moyennes empiriques $E(X)$, $E(Y)$, et $E(Z)$.
    \item Vérifiez que $E(Z)$ est très proche de $E(X) + E(Y)$.
\end{enumerate}

\begin{codecell}
import numpy as np

N_simulations = 100000

# 1. Simuler X (de a 4 faces) et Y (de a 6 faces)
# X = ...
# Y = ...

# 2. Creer Z = X + Y
# Z = ...

# 3. Calculer les moyennes empiriques
# E_X = ...
# E_Y = ...
# E_Z = ...

# 4. Verifier la linearite
# print(f"E(X) = {E_X:.4f}")
# print(f"E(Y) = {E_Y:.4f}")
# print(f"E(X) + E(Y) = {E_X + E_Y:.4f}")
# print(f"E(Z) = E(X+Y) = {E_Z:.4f}")
\end{codecell}
\end{exercicebox}

\begin{exercicebox}[Exercice 3 : Esperance Binomiale (Simulation)]
La théorie nous dit que pour $X \sim \text{Bin}(n, p)$, $E(X) = np$. Nous allons vérifier cela par simulation.

\textbf{Votre tâche :}
\begin{enumerate}
    \item Définissez les paramètres $n=20$ et $p=0.4$.
    \item Simulez 100 000 réalisations d'une variable aléatoire $X \sim \text{Bin}(n, p)$ en utilisant \texttt{numpy.random.binomial()}.
    \item Calculez la moyenne empirique de vos simulations.
    \item Comparez la moyenne empirique à l'espérance théorique $np$.
\end{enumerate}

\begin{codecell}
import numpy as np

n, p = 20, 0.4
N_simulations = 100000

# 1. Simuler N fois une loi Bin(n, p)
# resultats_bin = ...

# 2. Calculer la moyenne empirique
# moyenne_empirique = ...

# 3. Calculer la moyenne theorique
# moyenne_theorique = ...

# 4. Afficher
# print(f"Moyenne empirique: {moyenne_empirique:.4f}")
# print(f"Esperance theorique (np): {moyenne_theorique:.4f}")
\end{codecell}
\end{exercicebox}

\begin{exercicebox}[Exercice 4 : Esperance Geometrique (Simulation)]
Pour $X \sim \text{Geom}(p)$ (comptant les échecs), $E(X) = q/p$. Vérifions cela.

\textbf{Votre tâche :}
\begin{enumerate}
    \item Définissez $p=0.2$ (et $q=1-p$).
    \item Simulez 100 000 réalisations d'une variable $Y \sim \text{Geom}(p)$ en utilisant \texttt{numpy.random.geometric()}.
    \item \textbf{Attention :} \texttt{numpy.random.geometric} compte le nombre d'essais ($k=1, 2, \dots$). Pour obtenir $X$ (le nombre d'échecs, $k=0, 1, \dots$), vous devez soustraire 1 de chaque résultat.
    \item Calculez la moyenne empirique de $X$ (le nombre d'échecs).
    \item Comparez cette moyenne à l'espérance théorique $q/p$.
\end{enumerate}

\begin{codecell}
import numpy as np

p = 0.2
q = 1 - p
N_simulations = 100000

# 1. Simuler N fois une loi Geom(p) (nb d'essais)
# resultats_geom_essais = ...

# 3. Convertir en nombre d'echecs
# resultats_geom_echecs = ...

# 4. Calculer la moyenne empirique des echecs
# moyenne_empirique = ...

# 5. Calculer la moyenne theorique des echecs
# moyenne_theorique = ...

# 6. Afficher
# print(f"Moyenne empirique (echecs): {moyenne_empirique:.4f}")
# print(f"Esperance theorique (q/p): {moyenne_theorique:.4f}")
\end{codecell}
\end{exercicebox}

\begin{exercicebox}[Exercice 5 : Esperance et Variance de Bernoulli]
La variable aléatoire de Bernoulli $X \sim \text{Bern}(p)$ est la brique de base. Théoriquement, $E(X) = p$ et $\text{Var}(X) = p(1-p)$.

\textbf{Votre tâche :}
\begin{enumerate}
    \item Définissez $p=0.8$.
    \item Simulez 100 000 essais de Bernoulli (résultats 0 ou 1) avec probabilité $p$. (Indice : \texttt{numpy.random.choice} ou \texttt{numpy.random.binomial} avec $n=1$).
    \item Calculez l'espérance empirique (la moyenne) et la variance empirique (\texttt{numpy.var}).
    \item Comparez-les aux valeurs théoriques $p$ et $p(1-p)$.
\end{enumerate}

\begin{codecell}
import numpy as np

p = 0.8
N_simulations = 100000

# 1. Simuler N essais de Bernoulli
# essais = ...

# 2. Calculer l'esperance et la variance empiriques
# E_empirique = ...
# Var_empirique = ...

# 3. Calculer les valeurs theoriques
# E_theorique = ...
# Var_theorique = ...

# 4. Afficher
# print(f"Esperance: Empirique={E_empirique:.4f}, Theorique={E_theorique:.4f}")
# print(f"Variance:  Empirique={Var_empirique:.4f}, Theorique={Var_theorique:.4f}")
\end{codecell}
\end{exercicebox}
\newpage
\section{Espérance et autres notions associées aux variables aléatoires discrètes }

\subsection{Espérance d'une variable aléatoire discrète}

\begin{definitionbox}[Espérance]
L'espérance (ou valeur attendue) d'une variable aléatoire discrète $X$, qui prend les valeurs distinctes $x_1, x_2, \dots$, est définie par :
$$ E(X) = \sum_j x_j P(X=x_j) $$
\end{definitionbox}

\begin{intuitionbox}
L'espérance représente la valeur moyenne que l'on obtiendrait si l'on répétait l'expérience un très grand nombre de fois. C'est le \textbf{centre de gravité} de la distribution de probabilité. Si les probabilités étaient des masses placées sur une tige aux positions $x_j$, l'espérance serait le point d'équilibre.
\end{intuitionbox}

\begin{examplebox}[Lancer d'un dé]
Soit $X$ le résultat d'un lancer de dé équilibré. Chaque face a une probabilité de $1/6$. L'espérance est :
$$ E(X) = 1\left(\frac{1}{6}\right) + 2\left(\frac{1}{6}\right) + 3\left(\frac{1}{6}\right) + 4\left(\frac{1}{6}\right) + 5\left(\frac{1}{6}\right) + 6\left(\frac{1}{6}\right) = \frac{21}{6} = 3.5 $$
Même si 3.5 n'est pas un résultat possible, c'est la valeur moyenne sur un grand nombre de lancers.
\end{examplebox}

\subsection{Linéarité de l'espérance}

\begin{theorembox}[Linéarité de l'espérance]
Pour toutes variables aléatoires $X$ et $Y$, et pour toute constante $c$, on a :
\begin{align*}
E(X+Y) &= E(X) + E(Y) \\
E(cX) &= cE(X)
\end{align*}
Cette propriété est extrêmement puissante car elle ne requiert pas que $X$ et $Y$ soient indépendantes.
\end{theorembox}

\begin{intuitionbox}
Cette propriété formalise une idée très simple : "la moyenne d'une somme est la somme des moyennes". Si vous jouez à deux jeux de hasard, votre gain moyen total est simplement la somme de ce que vous gagnez en moyenne à chaque jeu, que les jeux soient liés ou non.
\end{intuitionbox}

\begin{examplebox}[Somme de deux dés]
Soit $X_1$ le résultat du premier dé et $X_2$ celui du second. On sait que $E(X_1) = 3.5$ et $E(X_2) = 3.5$.
Soit $S = X_1 + X_2$ la somme des deux dés. Grâce à la linéarité, on peut calculer l'espérance de la somme sans avoir à lister les 36 résultats possibles :
$$ E(S) = E(X_1 + X_2) = E(X_1) + E(X_2) = 3.5 + 3.5 = 7 $$
\end{examplebox}

\subsection{Espérance de la loi binomiale}

\begin{theorembox}[Espérance de la loi binomiale]
Si $X \sim \text{Bin}(n, p)$, alors son espérance est $E(X) = np$.
\end{theorembox}

\begin{intuitionbox}
Ce résultat est très naturel. Si vous lancez une pièce 100 fois ($n=100$) avec une probabilité de 50\% d'obtenir Pile ($p=0.5$), vous vous attendez en moyenne à obtenir $100 \times 0.5 = 50$ Piles. La formule $np$ généralise cette idée.
\end{intuitionbox}

\begin{proofbox}
Le calcul direct de l'espérance avec la PMF binomiale est possible, mais long. En utilisant la linéarité de l'espérance, on obtient une preuve beaucoup plus courte et élégante.
\newline
On peut voir une variable binomiale $X$ comme la somme de $n$ variables de Bernoulli indépendantes, $X = I_1 + I_2 + \dots + I_n$, où chaque $I_j$ représente le succès (1) ou l'échec (0) du $j$-ième essai.
\newline
Chaque $I_j$ a pour espérance $E(I_j) = 1 \cdot p + 0 \cdot (1-p) = p$.
\newline
Par linéarité de l'espérance, on a :
$$ E(X) = E(I_1) + E(I_2) + \dots + E(I_n) = \underbrace{p + p + \dots + p}_{n \text{ fois}} = np $$
\end{proofbox}

\subsection{Espérance de la loi géométrique}
\begin{theorembox}[Espérance de la loi géométrique]
L'espérance d'une variable aléatoire $X \sim \text{Geom}(p)$ (comptant le nombre d'échecs) est :
$$ E(X) = \frac{1-p}{p} = \frac{q}{p} $$
\end{theorembox}

\begin{intuitionbox}
Si un événement a 1 chance sur 10 de se produire ($p=0.1$), il est logique de penser qu'il faudra en moyenne 9 échecs ($q/p = 0.9/0.1=9$) avant qu'il ne se produise. L'espérance du nombre total d'essais (échecs + 1 succès) serait alors $1/p$.
\end{intuitionbox}

\begin{proofbox}[Démonstration de l'espérance géométrique via les séries entières]
Soit $X \sim \text{Geom}(p)$, où $X$ compte le nombre d'échecs avant le premier succès. La PMF est $P(X=k) = q^k p$ pour $k=0, 1, 2, \dots$, avec $q=1-p$.
\newline
Par définition, l'espérance est :
$$ E(X) = \sum_{k=0}^{\infty} k \cdot P(X=k) = \sum_{k=0}^{\infty} k q^k p $$
Le terme pour $k=0$ est nul, on peut donc commencer la somme à $k=1$ :
$$ E(X) = p \sum_{k=1}^{\infty} k q^k $$
L'astuce consiste à reconnaître que la somme ressemble à la dérivée d'une série géométrique. Rappelons la formule de la série géométrique pour $|q|<1$ :
$$ \sum_{k=0}^{\infty} q^k = \frac{1}{1-q} $$
En dérivant les deux côtés par rapport à $q$, on obtient :
$$ \frac{d}{dq} \left( \sum_{k=0}^{\infty} q^k \right) = \frac{d}{dq} \left( \frac{1}{1-q} \right) $$
$$ \sum_{k=1}^{\infty} k q^{k-1} = \frac{1}{(1-q)^2} $$
Pour faire apparaître ce terme dans notre formule d'espérance, on factorise $q$ dans la somme :
$$ E(X) = p \cdot q \sum_{k=1}^{\infty} k q^{k-1} $$
On peut maintenant remplacer la somme par son expression analytique :
$$ E(X) = p \cdot q \cdot \frac{1}{(1-q)^2} $$
Puisque $p = 1-q$, on a :
$$ E(X) = p \cdot q \cdot \frac{1}{p^2} = \frac{q}{p} $$
Ce qui démontre que l'espérance du nombre d'échecs avant le premier succès est $\frac{q}{p}$.
\end{proofbox}

\subsection{Loi du statisticien inconscient (LOTUS)}

\begin{theorembox}[Théorème de Transfert (LOTUS)]
Si $X$ est une variable aléatoire discrète et $g(x)$ est une fonction de $\mathbb{R}$ dans $\mathbb{R}$, alors l'espérance de la variable aléatoire $g(X)$ est donnée par :
$$ E[g(X)] = \sum_x g(x) P(X=x) $$
La somme porte sur toutes les valeurs possibles de $X$. Ce théorème est utile car il évite d'avoir à trouver la PMF de $g(X)$.
\end{theorembox}

\begin{intuitionbox}
Pour trouver la valeur moyenne d'une fonction d'une variable aléatoire (par exemple, le carré du résultat d'un dé), vous n'avez pas besoin de déterminer d'abord la distribution de ce carré. Vous pouvez simplement prendre chaque valeur possible du résultat original, lui appliquer la fonction, et pondérer ce nouveau résultat par la probabilité du résultat original.
\end{intuitionbox}

\begin{examplebox}[Calcul de $E(X^2)$ pour un dé]
Soit $X$ le résultat d'un lancer de dé. Calculons l'espérance de $Y=X^2$. La fonction est $g(x)=x^2$.
\begin{align*}
E(X^2) &= \sum_{k=1}^6 k^2 P(X=k) \\
&= 1^2\left(\frac{1}{6}\right) + 2^2\left(\frac{1}{6}\right) + 3^2\left(\frac{1}{6}\right) + 4^2\left(\frac{1}{6}\right) + 5^2\left(\frac{1}{6}\right) + 6^2\left(\frac{1}{6}\right) \\
&= \frac{1+4+9+16+25+36}{6} = \frac{91}{6} \approx 15.17
\end{align*}
\end{examplebox}

\subsection{Variance}

\begin{definitionbox}[Variance et écart-type]
La \textbf{variance} d'une variable aléatoire $X$ mesure la dispersion de sa distribution autour de son espérance. Elle est définie par :
$$ \text{Var}(X) = E\left[ (X - E(X))^2 \right] $$
La racine carrée de la variance est appelée l' \textbf{écart-type} :
$$ \text{SD}(X) = \sqrt{\text{Var}(X)} $$
\end{definitionbox}

\begin{intuitionbox}
La variance est la "distance carrée moyenne à la moyenne". On prend l'écart de chaque valeur par rapport à la moyenne, on le met au carré (pour que les écarts positifs et négatifs ne s'annulent pas), puis on en calcule la moyenne. L'écart-type est souvent plus interprétable car il ramène cette mesure de dispersion dans les mêmes unités que la variable aléatoire elle-même.
\end{intuitionbox}

\begin{theorembox}[Formule de calcul de la variance]
Pour toute variable aléatoire $X$, une formule plus pratique pour le calcul de la variance est :
$$ \text{Var}(X) = E(X^2) - [E(X)]^2 $$
\end{theorembox}

\begin{examplebox}[Variance d'un lancer de dé]
Nous avons déjà calculé pour un dé que $E(X) = 3.5$ et $E(X^2) = 91/6$. On peut maintenant trouver la variance facilement :
$$ \text{Var}(X) = E(X^2) - [E(X)]^2 = \frac{91}{6} - (3.5)^2 = \frac{91}{6} - 12.25 = 15.166... - 12.25 \approx 2.917 $$
L'écart-type est $\text{SD}(X) = \sqrt{2.917} \approx 1.708$.
\end{examplebox}

\subsection{Exercices}

\begin{exercicebox}[Espérance d'un jeu]
Un jeu consiste à lancer un dé à six faces. Si vous obtenez un 6, vous gagnez 10€. Si vous obtenez un 4 ou un 5, vous gagnez 1€. Sinon, vous ne gagnez rien. Quelle est l'espérance de gain pour une partie ?
\end{exercicebox}

\begin{correctionbox}
Soit $X$ la variable aléatoire représentant le gain. Les valeurs possibles de $X$ sont 10, 1, et 0.
Les probabilités associées sont :
$P(X=10) = P(\text{obtenir un 6}) = 1/6$.
$P(X=1) = P(\text{obtenir un 4 ou 5}) = 2/6 = 1/3$.
$P(X=0) = P(\text{obtenir 1, 2 ou 3}) = 3/6 = 1/2$.

L'espérance de gain est :
$$ E(X) = 10 \cdot P(X=10) + 1 \cdot P(X=1) + 0 \cdot P(X=0) $$
$$ E(X) = 10 \cdot \frac{1}{6} + 1 \cdot \frac{2}{6} + 0 \cdot \frac{3}{6} = \frac{10+2}{6} = \frac{12}{6} = 2 $$
L'espérance de gain est de 2€ par partie.
\end{correctionbox}

\begin{exercicebox}[Linéarité de l'espérance]
On lance deux dés non truqués. Soit $X$ la somme des résultats. Calculez $E(X)$ en utilisant la linéarité de l'espérance.
\end{exercicebox}

\begin{correctionbox}
Soit $D_1$ le résultat du premier dé et $D_2$ le résultat du second dé. On a $X = D_1 + D_2$.
L'espérance du résultat d'un seul dé est $E(D_1) = E(D_2) = 3.5$.
Par linéarité de l'espérance :
$$ E(X) = E(D_1 + D_2) = E(D_1) + E(D_2) = 3.5 + 3.5 = 7 $$
L'espérance de la somme est 7.
\end{correctionbox}

\begin{exercicebox}[Espérance binomiale]
Un étudiant répond au hasard à un QCM de 20 questions, chaque question ayant 4 options de réponse (une seule correcte). Quelle est l'espérance du nombre de bonnes réponses ?
\end{exercicebox}

\begin{correctionbox}
Soit $X$ le nombre de bonnes réponses. Chaque question est une épreuve de Bernoulli avec une probabilité de succès $p=1/4$. Le nombre total d'épreuves est $n=20$.
$X$ suit donc une loi binomiale $X \sim \text{Bin}(n=20, p=0.25)$.
L'espérance d'une loi binomiale est $E(X) = np$.
$$ E(X) = 20 \times 0.25 = 5 $$
L'étudiant peut s'attendre à avoir 5 bonnes réponses en moyenne.
\end{correctionbox}

\begin{exercicebox}[Loi Géométrique]
On lance une pièce de monnaie jusqu'à obtenir "Pile" pour la première fois. La probabilité d'obtenir "Pile" est $p=0.5$.
\begin{enumerate}
    \item Quelle est la probabilité que le premier "Pile" apparaisse au 4ème lancer (c'est-à-dire après 3 "Face") ?
    \item Quel est le nombre moyen d'échecs ("Face") attendu avant le premier succès ?
\end{enumerate}
\end{exercicebox}

\begin{correctionbox}
Soit $X$ le nombre d'échecs avant le premier succès. $X \sim \text{Geom}(p=0.5)$.
1. On cherche $P(X=3)$. La PMF est $P(X=k) = (1-p)^k p$.
$$ P(X=3) = (0.5)^3 \times 0.5 = 0.125 \times 0.5 = 0.0625 $$
La probabilité est de 6.25\%.

2. On cherche l'espérance $E(X)$.
$$ E(X) = \frac{1-p}{p} = \frac{0.5}{0.5} = 1 $$
On s'attend en moyenne à 1 échec ("Face") avant le premier "Pile".
\end{correctionbox}

\begin{exercicebox}[Variance d'un dé]
Calculez la variance du résultat $X$ d'un lancer de dé équilibré.
\end{exercicebox}

\begin{correctionbox}
On utilise la formule $\text{Var}(X) = E(X^2) - [E(X)]^2$.
On sait déjà que $E(X)=3.5$.
Calculons $E(X^2)$ avec LOTUS :
$$ E(X^2) = \sum_{k=1}^6 k^2 P(X=k) = \frac{1}{6}(1^2+2^2+3^2+4^2+5^2+6^2) $$
$$ E(X^2) = \frac{1}{6}(1+4+9+16+25+36) = \frac{91}{6} \approx 15.167 $$
Maintenant, la variance :
$$ \text{Var}(X) = \frac{91}{6} - (3.5)^2 = \frac{91}{6} - 12.25 = \frac{91 - 73.5}{6} = \frac{17.5}{6} \approx 2.917 $$
\end{correctionbox}

\begin{exercicebox}[Loi de Poisson]
Un livre de 500 pages contient 250 fautes de frappe distribuées au hasard.
\begin{enumerate}
    \item Quel est le nombre moyen de fautes par page ?
    \item Quelle est la probabilité qu'une page choisie au hasard contienne exactement 2 fautes ?
\end{enumerate}
\end{exercicebox}

\begin{correctionbox}
1. Le taux moyen d'erreurs est $\lambda = \frac{250 \text{ fautes}}{500 \text{ pages}} = 0.5$ fautes par page.
Le nombre de fautes par page, $X$, suit une loi de Poisson $X \sim \text{Poisson}(\lambda=0.5)$.

2. On cherche $P(X=2)$. La PMF de Poisson est $P(X=k) = \frac{e^{-\lambda}\lambda^k}{k!}$.
$$ P(X=2) = \frac{e^{-0.5}(0.5)^2}{2!} = \frac{0.6065 \times 0.25}{2} \approx 0.0758 $$
La probabilité est d'environ 7.58\%.
\end{correctionbox}

\begin{exercicebox}[Variance et constante]
Soit $X$ une variable aléatoire avec $E(X)=10$ et $\text{Var}(X)=2$. Calculez l'espérance et la variance de $Y = 3X + 5$.
\end{exercicebox}

\begin{correctionbox}
On utilise les propriétés de l'espérance et de la variance.
Pour l'espérance :
$E(Y) = E(3X+5) = E(3X) + E(5) = 3E(X) + 5$.
$$ E(Y) = 3(10) + 5 = 35 $$
Pour la variance :
$\text{Var}(Y) = \text{Var}(3X+5) = \text{Var}(3X) = 3^2 \text{Var}(X)$.
$$ \text{Var}(Y) = 9 \times 2 = 18 $$
\end{correctionbox}

\begin{exercicebox}[Variable indicatrice]
On lance une pièce deux fois. Soit $A$ l'événement "obtenir au moins un Pile". Soit $I_A$ la variable indicatrice de cet événement. Donnez la PMF, l'espérance et la variance de $I_A$.
\end{exercicebox}

\begin{correctionbox}
L'univers est $\{PP, PF, FP, FF\}$. L'événement $A$ est $\{PP, PF, FP\}$.
La probabilité de A est $P(A) = 3/4$.
La variable $I_A$ vaut 1 si $A$ se produit, 0 sinon. C'est une loi de Bernoulli.
$I_A \sim \text{Bern}(p=3/4)$.

PMF : $P(I_A=1) = 3/4$ et $P(I_A=0) = 1/4$.
Espérance : $E(I_A) = p = 3/4$.
Variance : $\text{Var}(I_A) = p(1-p) = \frac{3}{4} \left(1-\frac{3}{4}\right) = \frac{3}{4} \cdot \frac{1}{4} = \frac{3}{16}$.
\end{correctionbox}

\begin{exercicebox}[Poisson comme approximation]
Une compagnie aérienne observe que 0.2\% des passagers qui réservent un vol ne se présentent pas. Si un avion a 200 sièges et que la compagnie vend 200 billets, quelle est la probabilité qu'exactement 3 passagers ne se présentent pas ? Utilisez l'approximation de Poisson.
\end{exercicebox}

\begin{correctionbox}
C'est une situation binomiale avec $n=200$ (grand) et $p=0.002$ (petit). On peut l'approximer par une loi de Poisson.
Le paramètre $\lambda$ est $\lambda = np = 200 \times 0.002 = 0.4$.
Soit $X$ le nombre de passagers absents, $X \sim \text{Poisson}(0.4)$. On cherche $P(X=3)$.
$$ P(X=3) = \frac{e^{-0.4}(0.4)^3}{3!} = \frac{0.6703 \times 0.064}{6} \approx 0.00715 $$
La probabilité est d'environ 0.715\%.
\end{correctionbox}

\begin{exercicebox}[LOTUS]
Une variable aléatoire discrète $X$ a la PMF suivante : $P(X=-1)=0.2$, $P(X=0)=0.5$, $P(X=1)=0.3$. Calculez $E[ (X+1)^2 ]$.
\end{exercicebox}

\begin{correctionbox}
On utilise le théorème de transfert (LOTUS) avec la fonction $g(x) = (x+1)^2$.
$$ E[g(X)] = \sum_x g(x) P(X=x) $$
$$ E[(X+1)^2] = (-1+1)^2 P(X=-1) + (0+1)^2 P(X=0) + (1+1)^2 P(X=1) $$
$$ E[(X+1)^2] = (0)^2 \cdot (0.2) + (1)^2 \cdot (0.5) + (2)^2 \cdot (0.3) $$
$$ E[(X+1)^2] = 0 + 1 \cdot 0.5 + 4 \cdot 0.3 = 0.5 + 1.2 = 1.7 $$
\end{correctionbox}

\begin{exercicebox}[Espérance d'une fonction]
Soit $X$ une variable aléatoire représentant le résultat d'un lancer d'un dé équilibré à 4 faces (valeurs 1, 2, 3, 4). Calculez l'espérance de $Y = 1/X$.
\end{exercicebox}

\begin{correctionbox}
Chaque face a une probabilité de $1/4$. On utilise le théorème de transfert (LOTUS) avec $g(x) = 1/x$.
$$ E(Y) = E(1/X) = \sum_{k=1}^4 \frac{1}{k} P(X=k) $$
$$ E(1/X) = \frac{1}{1}\left(\frac{1}{4}\right) + \frac{1}{2}\left(\frac{1}{4}\right) + \frac{1}{3}\left(\frac{1}{4}\right) + \frac{1}{4}\left(\frac{1}{4}\right) $$
$$ E(1/X) = \frac{1}{4} \left(1 + \frac{1}{2} + \frac{1}{3} + \frac{1}{4}\right) = \frac{1}{4} \left(\frac{12+6+4+3}{12}\right) = \frac{1}{4} \cdot \frac{25}{12} = \frac{25}{48} \approx 0.521 $$
\end{correctionbox}

\begin{exercicebox}[Loi Géométrique : "Sans mémoire"]
Un archer touche sa cible avec une probabilité $p=1/3$. Sachant qu'il a déjà manqué ses 5 premiers tirs, quelle est la probabilité qu'il touche la cible au 8ème tir (c'est-à-dire après 2 échecs supplémentaires) ?
\end{exercicebox}

\begin{correctionbox}
La loi géométrique est "sans mémoire". Le fait qu'il ait déjà manqué 5 tirs ne change pas les probabilités pour les tirs futurs. Le problème revient à se demander la probabilité qu'il lui faille 3 tirs supplémentaires pour réussir, ce qui équivaut à 2 échecs suivis d'un succès.
Soit $X$ le nombre d'échecs avant le premier succès. On cherche $P(X=2)$.
$$ P(X=2) = (1-p)^2 p = \left(\frac{2}{3}\right)^2 \left(\frac{1}{3}\right) = \frac{4}{9} \cdot \frac{1}{3} = \frac{4}{27} $$
\end{correctionbox}

\begin{exercicebox}[Loi de Poisson : Événements rares]
Dans une grande forêt, on observe en moyenne 0.8 ours par kilomètre carré. On étudie une zone de 5 km².
\begin{enumerate}
    \item Quel est le nombre moyen d'ours attendu dans cette zone ?
    \item Quelle est la probabilité de n'observer aucun ours dans cette zone ?
\end{enumerate}
\end{exercicebox}

\begin{correctionbox}
1. Le taux d'ours est de 0.8 par km². Pour une zone de 5 km², le paramètre $\lambda$ de la loi de Poisson est :
$$ \lambda = 0.8 \text{ ours/km²} \times 5 \text{ km²} = 4 $$
On s'attend donc à observer 4 ours en moyenne dans cette zone.

2. Soit $X$ le nombre d'ours observés, $X \sim \text{Poisson}(\lambda=4)$. On cherche $P(X=0)$.
$$ P(X=0) = \frac{e^{-4} 4^0}{0!} = e^{-4} \approx 0.0183 $$
La probabilité de n'observer aucun ours est d'environ 1.83\%.
\end{correctionbox}

\begin{exercicebox}[Variance d'une loi de Bernoulli]
Soit $X \sim \text{Bern}(p)$. Montrez en utilisant la formule $\text{Var}(X) = E(X^2) - [E(X)]^2$ que $\text{Var}(X) = p(1-p)$.
\end{exercicebox}

\begin{correctionbox}
Pour une variable de Bernoulli, $X$ ne peut prendre que les valeurs 0 et 1.
L'espérance est $E(X) = 1 \cdot p + 0 \cdot (1-p) = p$.
Pour calculer $E(X^2)$, on utilise LOTUS. Comme $X$ ne prend que les valeurs 0 et 1, $X^2$ est identique à $X$.
En effet, $0^2=0$ et $1^2=1$. Donc $X^2=X$.
Par conséquent, $E(X^2) = E(X) = p$.
On applique la formule de la variance :
$$ \text{Var}(X) = E(X^2) - [E(X)]^2 = p - p^2 = p(1-p) $$
\end{correctionbox}

\begin{exercicebox}[Loi Hypergéométrique]
Dans un groupe de 10 amis (6 hommes et 4 femmes), on tire au sort 3 personnes pour organiser une fête. Quelle est la probabilité que le groupe tiré au sort soit composé exclusivement de femmes ?
\end{exercicebox}

\begin{correctionbox}
Il s'agit d'un tirage sans remise. Soit $X$ le nombre de femmes tirées.
$X \sim \text{HG}(w=4 \text{ femmes}, b=6 \text{ hommes}, m=3 \text{ tirages})$.
On cherche $P(X=3)$.
$$ P(X=3) = \frac{\binom{\text{femmes}}{3} \binom{\text{hommes}}{0}}{\binom{\text{total}}{3}} = \frac{\binom{4}{3} \binom{6}{0}}{\binom{10}{3}} $$
$$ \binom{4}{3} = 4 \quad ; \quad \binom{6}{0} = 1 \quad ; \quad \binom{10}{3} = \frac{10 \cdot 9 \cdot 8}{3 \cdot 2 \cdot 1} = 120 $$
$$ P(X=3) = \frac{4 \times 1}{120} = \frac{4}{120} = \frac{1}{30} \approx 0.0333 $$
La probabilité est d'environ 3.33\%.
\end{correctionbox}

\begin{exercicebox}[Loi Binomiale : "Au moins"]
Un test de dépistage rapide a une probabilité de 0.1 de donner un faux positif. Si 10 personnes saines passent ce test, quelle est la probabilité qu'au moins deux d'entre elles reçoivent un faux positif ?
\end{exercicebox}

\begin{correctionbox}
Soit $X$ le nombre de faux positifs. $X \sim \text{Bin}(n=10, p=0.1)$.
On cherche $P(X \ge 2)$. Il est plus simple de calculer le complémentaire : $1 - P(X < 2)$.
$P(X < 2) = P(X=0) + P(X=1)$.
$$ P(X=0) = \binom{10}{0}(0.1)^0(0.9)^{10} = (0.9)^{10} \approx 0.3487 $$
$$ P(X=1) = \binom{10}{1}(0.1)^1(0.9)^9 = 10 \cdot 0.1 \cdot (0.9)^9 \approx 0.3874 $$
$P(X < 2) \approx 0.3487 + 0.3874 = 0.7361$.
$$ P(X \ge 2) = 1 - 0.7361 = 0.2639 $$
La probabilité d'avoir au moins deux faux positifs est d'environ 26.39\%.
\end{correctionbox}

\begin{exercicebox}[Écart-type]
Une machine distribue des boissons. La quantité versée $X$ (en cL) a pour espérance $E(X)=20$ et on a $E(X^2)=404$. Quel est l'écart-type de la quantité versée ?
\end{exercicebox}

\begin{correctionbox}
L'écart-type est la racine carrée de la variance. Calculons d'abord la variance.
$$ \text{Var}(X) = E(X^2) - [E(X)]^2 = 404 - (20)^2 = 404 - 400 = 4 $$
La variance est de 4 cL².
L'écart-type est :
$$ \text{SD}(X) = \sqrt{\text{Var}(X)} = \sqrt{4} = 2 $$
L'écart-type est de 2 cL.
\end{correctionbox}

\begin{exercicebox}[Espérance et prise de décision]
Vous avez le choix entre deux loteries.
Loterie A : Vous gagnez 100€ avec une probabilité de 0.1, sinon rien.
Loterie B : Vous gagnez 20€ avec une probabilité de 0.4, sinon rien.
Quelle loterie est la plus avantageuse en termes d'espérance de gain ?
\end{exercicebox}

\begin{correctionbox}
Calculons l'espérance de gain pour chaque loterie.
Soit $G_A$ le gain de la loterie A.
$$ E(G_A) = 100 \cdot P(G_A=100) + 0 \cdot P(G_A=0) = 100 \cdot 0.1 = 10€ $$
Soit $G_B$ le gain de la loterie B.
$$ E(G_B) = 20 \cdot P(G_B=20) + 0 \cdot P(G_B=0) = 20 \cdot 0.4 = 8€ $$
L'espérance de gain de la loterie A (10€) est supérieure à celle de la loterie B (8€). La loterie A est donc plus avantageuse en moyenne.
\end{correctionbox}

\begin{exercicebox}[Poisson : Somme de variables]
Deux sources radioactives émettent des particules indépendamment. La source 1 émet des particules selon une loi de Poisson de paramètre $\lambda_1=2$ par minute. La source 2 suit une loi de Poisson de paramètre $\lambda_2=3$ par minute. Quelle est la probabilité qu'un total de 4 particules soit émis en une minute ?
\end{exercicebox}

\begin{correctionbox}
Une propriété importante de la loi de Poisson est que la somme de deux variables de Poisson indépendantes est aussi une variable de Poisson dont le paramètre est la somme des paramètres.
Soit $X_1 \sim \text{Poisson}(2)$ et $X_2 \sim \text{Poisson}(3)$.
Le nombre total de particules $Y = X_1 + X_2$ suit une loi de Poisson de paramètre $\lambda = \lambda_1 + \lambda_2 = 2+3=5$.
Donc, $Y \sim \text{Poisson}(5)$.
On cherche $P(Y=4)$.
$$ P(Y=4) = \frac{e^{-5} 5^4}{4!} = \frac{e^{-5} \cdot 625}{24} \approx 0.1755 $$
La probabilité est d'environ 17.55\%.
\end{correctionbox}

\begin{exercicebox}[Contexte et choix du modèle]
Une petite ville compte 5000 habitants. En moyenne, 1 personne sur 1000 est allergique à une substance X.
\begin{enumerate}
    \item Quel modèle (Binomial ou Poisson) utiliseriez-vous pour estimer la probabilité qu'il y ait exactement 5 personnes allergiques dans cette ville ? Justifiez.
    \item Calculez cette probabilité.
\end{enumerate}
\end{exercicebox}

\begin{correctionbox}
1. Le modèle exact est une loi binomiale avec $n=5000$ et $p=1/1000=0.001$. Cependant, comme $n$ est très grand et $p$ est très petit, la loi de Poisson est une excellente approximation et bien plus simple à calculer. On utilisera donc une loi de Poisson.

2. Le paramètre de la loi de Poisson est $\lambda = np = 5000 \times 0.001 = 5$.
Soit $X$ le nombre de personnes allergiques, $X \sim \text{Poisson}(5)$.
On cherche $P(X=5)$.
$$ P(X=5) = \frac{e^{-5} 5^5}{5!} = \frac{e^{-5} \cdot 3125}{120} \approx 0.1755 $$
La probabilité est d'environ 17.55\%.
\end{correctionbox}
\newpage
\section{La Loi Normale (ou Gaussienne)}

\subsection{Introduction et Fonction de Densité (PDF)}

Après les lois discrètes et les lois continues de base (Uniforme, Exponentielle), nous abordons la distribution la plus célèbre et la plus utilisée en probabilités et statistiques.

\begin{definitionbox}[Loi Normale]
Une variable aléatoire continue $X$ suit une \textbf{loi normale} (ou loi de Gauss) de paramètres $\mu$ (l'espérance) et $\sigma^2$ (la variance), notée $X \sim \mathcal{N}(\mu, \sigma^2)$, si sa fonction de densité de probabilité (PDF) est donnée par :
$$ f(x; \mu, \sigma) = \frac{1}{\sigma \sqrt{2\pi}} e^{ -\frac{1}{2} \left( \frac{x-\mu}{\sigma} \right)^2 } $$
pour tout $x \in (-\infty, \infty)$, où $\sigma > 0$.
\end{definitionbox}

Cette formule, bien qu'imposante, décrit une forme très familière : la courbe en cloche.

\begin{intuitionbox}[La Courbe en Cloche]
La loi normale est sans doute la distribution la plus importante en probabilités et statistiques. Pourquoi ? Parce qu'elle modélise remarquablement bien de nombreux phénomènes naturels et processus aléatoires où les valeurs tendent à se regrouper autour d'une moyenne, avec des écarts symétriques devenant de plus en plus rares à mesure qu'on s'éloigne de cette moyenne. Pensez à la taille des individus dans une population, aux erreurs de mesure répétées, ou même aux notes d'un grand groupe d'étudiants à un examen bien conçu. 

Sa densité a une forme caractéristique de \textbf{cloche symétrique} :
\begin{itemize}
    \item \textbf{Le Centre ($\mu$)} : Le paramètre $\mu$ représente l'\textbf{espérance} (la moyenne) de la distribution. C'est le centre de symétrie de la courbe, là où la cloche atteint son \textbf{sommet}. C'est la valeur la plus probable (le mode) et aussi la valeur qui coupe la distribution en deux moitiés égales (la médiane). Changer $\mu$ \textit{translate} la cloche horizontalement sans changer sa forme.
    \item \textbf{La Dispersion ($\sigma$)} : Le paramètre $\sigma$ est l'\textbf{écart-type} ($\sigma^2$ est la variance). Il mesure la \textbf{dispersion} des valeurs autour de la moyenne $\mu$. Géométriquement, $\sigma$ contrôle la \textbf{largeur} de la cloche.
        \begin{itemize}
            \item Un \textit{petit} $\sigma$ signifie que les données sont très concentrées autour de la moyenne, donnant une cloche \textbf{étroite et pointue}.
            \item Un \textit{grand} $\sigma$ signifie que les données sont plus étalées, donnant une cloche \textbf{large et aplatie}.
        \end{itemize}
    Les points d'inflexion de la courbe (là où la courbure change de sens) se situent exactement à $\mu \pm \sigma$.
\end{itemize}

\tcblower
\centering
\begin{tikzpicture}
    \begin{axis}[
        title={La Courbe en Cloche (PDF de la Loi Normale)},
        xlabel={$x$},
        ylabel={$f(x)$},
        axis lines=middle,
        no markers,
        samples=100,
        domain=-4:4,
        height=8cm,
        width=\linewidth-1cm,
        tick label style={font=\tiny},
        legend style={at={(0.5,-0.15)}, anchor=north, font=\small},
        legend columns=2
    ]
    % N(0, 1)
    \addplot [blue, ultra thick] {1/(sqrt(2*pi))*exp(-x^2/2)};
    \addlegendentry{$\mu=0, \sigma=1$};
    % N(0, 0.25) => sigma=0.5
    \addplot [red, ultra thick] {1/(0.5*sqrt(2*pi))*exp(-x^2/(2*0.5^2))};
    \addlegendentry{$\mu=0, \sigma=0.5$ (étroite)};
    % N(1, 2.25) => sigma=1.5
    \addplot [green!70!black, ultra thick] {1/(1.5*sqrt(2*pi))*exp(-(x-1)^2/(2*1.5^2))};
    \addlegendentry{$\mu=1, \sigma=1.5$ (large, décalée)};

    \draw [dashed] (axis cs:0,0) -- (axis cs:0, {1/(sqrt(2*pi))}) node[above, font=\tiny] {pic à $\mu=0$}; % Ligne pour mu=0
    \draw [dashed] (axis cs:1,0) -- (axis cs:1, {1/(1.5*sqrt(2*pi))}) node[above right, font=\tiny] {pic à $\mu=1$}; % Ligne pour mu=1
    \end{axis}
\end{tikzpicture}
\par\small\textit{Influence de $\mu$ (position) et $\sigma$ (largeur) sur la forme de la cloche.}
\end{intuitionbox}

Mais d'où vient cette formule spécifique ? Il existe une dérivation fascinante à partir d'hypothèses fondamentales sur les erreurs aléatoires (argument d'Herschel-Maxwell).

\begin{proofbox}[Dérivation de la Densité Normale à partir des Principes Fondamentaux]

\textbf{Contexte Visuel :} Imaginons un nuage de points dispersés autour d'une cible à l'origine $(0,0)$, comme des impacts de fléchettes. Le graphique ci-dessous illustre cette dispersion. On s'intéresse à la probabilité de tomber dans une petite zone, comme $dA$, autour d'un point $(x, y)$.

\begin{center}
\begin{tikzpicture}
\begin{axis}[
    axis lines=middle, % Axes qui passent par l'origine (0,0)
    xlabel=$x$,       % Étiquette axe X
    ylabel=$y$,       % Étiquette axe Y
    axis line style={magenta}, % Couleur des axes
    xlabel style={anchor=west, magenta}, % Style de l'étiquette X
    ylabel style={anchor=south, magenta}, % Style de l'étiquette Y
    xmin=-3.5, xmax=3.5, % Limites du graphique
    ymin=-3.5, ymax=3.5,
    tick label style={font=\tiny} % Police plus petite pour les graduations
]

% Ajout des points du nuage. Ce sont des coordonnées approximatives.
\addplot [only marks, mark=*, cyan, mark size=1.5pt]
coordinates {
    (0.2, 0.1) (-0.5, 0.2) (0.1, -0.3) (0.5, -0.8) (-0.3, -1.2) (0,0.1)
    (1.5, 1.5) (0.8, 0.8) (2.0, -0.5) (2.8, -1.4) (2.5, -0.2)
    (-1.8, -1.3) (-2.5, 0.5) (-1.5, 0.3) (-2.2, -0.8) (-0.8, 0.5)
    (0.5, 1.2) (0.7, 2.8) (0.2, 3.2) (-0.5, 1.5) (-1.0, -2.0)
    (1.8, -1.0) (1.0, -1.5) (0.3, -2.5) (-1.8, -2.8) (1.2, 0.5)
    (-1.2, -1.5) (-0.8, 1.1)
};

% --- Annotations ---

% Points et boîte pour 'dA'
\addplot [only marks, mark=*, red, mark size=1.5pt] coordinates {(1.3, 2.0)};
\draw [red, thick] (axis cs:1.1, 1.8) rectangle (axis cs:1.5, 2.2);
\node [red, above] at (axis cs:1.3, 2.2) {$dA$};

% Points et boîte pour 'dB'
\addplot [only marks, mark=*, cyan, mark size=1.5pt] 
coordinates {(-1.2, 2.0) (-1.3, 2.3) (-1.0, 2.2) (-1.1, 1.9)};
\draw [blue, thick] (axis cs:-1.8, 1.2) rectangle (axis cs:-0.8, 2.8);
\node [blue, above] at (axis cs:-1.3, 2.8) {$dB$};

\end{axis}
\end{tikzpicture}
\end{center}

\textbf{Objectif :} Expliquer comment arriver à la formule mathématique de la courbe en cloche (densité de probabilité normale) en partant de principes fondamentaux sur les erreurs aléatoires.

\textbf{1. Le Point de Départ : Densité et Aire $dA$}
Dans une distribution continue, la probabilité de tomber \textit{exactement} sur un point $(x, y)$ est nulle. On ne peut donc pas parler de "probabilité d'un point". On parle de la probabilité de tomber \textit{dans une petite zone}, comme un rectangle $dA = dx \cdot dy$ autour du point $(x, y)$.
Cette probabilité, notée $P(\text{dans } dA)$, est \textit{proportionnelle} à l'aire de la zone $dA$. La \textit{constante de proportionnalité} est la \textbf{fonction de densité de probabilité} $p(x, y)$ évaluée en ce point. En d'autres termes, la densité $p(x, y)$ \textit{représente} localement la concentration de probabilité. Ainsi, la probabilité de tomber dans la zone $dA$ est approximativement :
$$ P(\text{dans } dA) \approx p(x, y) \cdot dA $$

\textbf{2. Les Hypothèses Fondamentales}
On pose deux hypothèses sur la nature de ces erreurs (représentées par la densité $p(x, y)$) :
\begin{enumerate}
    \item \textbf{Indépendance des axes :} L'erreur horizontale ($x$) est indépendante de l'erreur verticale ($y$). Cela implique que la densité jointe $p(x, y)$ peut s'écrire comme le produit de la densité marginale sur $x$, notée $f(x)$, et de la densité marginale sur $y$, notée $f(y)$. Donc, $p(x, y) = f(x) \cdot f(y)$.
    \item \textbf{Symétrie de rotation (Isotropie) :} La densité ne dépend que de la distance $r = \sqrt{x^2 + y^2}$ au centre, pas de l'angle. Il existe donc une fonction $\phi(r)$ telle que la densité en $(x,y)$ est $p(x, y) = \phi(\sqrt{x^2 + y^2})$.
\end{enumerate}

\textbf{3. L'Équation Fonctionnelle}
En égalant les deux expressions pour la même densité $p(x, y)$ (à une constante près), on obtient :
$$ f(x) \cdot f(y) = \phi(\sqrt{x^2 + y^2}) $$
Pour $y=0$, on a $f(x) \cdot f(0) = \phi(x)$. Posons $f(0) = \lambda$. Alors $\phi(x) = \lambda f(x)$.
L'équation devient :
$$ f(x) \cdot f(y) = \lambda f(\sqrt{x^2 + y^2}) $$

\textbf{4. Résolution de l'Équation Fonctionnelle}
Posons $g(x) = f(x)/\lambda$, avec $g(0)=1$. L'équation se simplifie en :
$$ g(x) g(y) = g(\sqrt{x^2 + y^2}) $$
Posons $g(x) = h(x^2)$. L'équation devient $h(x^2)h(y^2) = h(x^2+y^2)$. Avec $a=x^2$ et $b=y^2$, on a :
$$ h(a) h(b) = h(a+b) $$
La solution continue de cette équation de Cauchy est $h(a) = e^{Aa}$ pour une constante $A$.
Retour aux fonctions : $g(x) = h(x^2) = e^{Ax^2}$. $f(x) = \lambda g(x) = \lambda e^{Ax^2}$.
Comme la densité doit diminuer loin du centre, $A$ doit être négative. Posons $A = -k$ avec $k>0$.
$$ f(x) = \lambda e^{-k x^2} $$

\textbf{5. Normalisation et Identification des Paramètres}
\begin{enumerate}
    \item \textbf{Condition $\int f(x) dx = 1$} : L'intégrale Gaussienne $\int_{-\infty}^{\infty} e^{-k x^2} \, \mathrm{d}x = \sqrt{\frac{\pi}{k}}$.
    Donc, $\int_{-\infty}^{\infty} f(x) dx = \lambda \sqrt{\frac{\pi}{k}} = 1 \implies \lambda = \sqrt{\frac{k}{\pi}}$.
    \item \textbf{Lien avec la Variance ($\sigma^2$)} : Pour une distribution centrée, $\sigma^2 = E[X^2] = \int x^2 f(x) dx$.
    $$ \sigma^2 = \int_{-\infty}^{\infty} x^2 \left( \sqrt{\frac{k}{\pi}} e^{-k x^2} \right) \, \mathrm{d}x = \sqrt{\frac{k}{\pi}} \left( \frac{1}{2k} \sqrt{\frac{\pi}{k}} \right) = \frac{1}{2k} $$
    Donc, $k = \frac{1}{2\sigma^2}$.
    \item \textbf{Substitution Finale :} Remplaçons $k$ dans $\lambda$ et $f(x)$.
    $$ \lambda = \sqrt{\frac{1/(2\sigma^2)}{\pi}} = \frac{1}{\sigma\sqrt{2\pi}} $$
    $$ f(x) = \frac{1}{\sigma\sqrt{2\pi}} e^{-\frac{1}{2\sigma^2} x^2} = \frac{1}{\sigma\sqrt{2\pi}} e^{ -\frac{x^2}{2\sigma^2} } $$
    \item \textbf{Généralisation (Moyenne $\mu$)} : Pour centrer la distribution sur $\mu$, on remplace $x$ par $(x-\mu)$ dans l'exposant :
    $$ f(x; \mu, \sigma) = \frac{1}{\sigma \sqrt{2\pi}} e^{ -\frac{(x-\mu)^2}{2\sigma^2} } $$
\end{enumerate}
C'est la fonction de densité de la loi normale $\mathcal{N}(\mu, \sigma^2)$.
\end{proofbox}

\subsection{La Loi Normale Centrée Réduite $\mathcal{N}(0, 1)$}

Avant d'explorer les propriétés de la loi normale générale, concentrons-nous sur son cas le plus simple et le plus fondamental.

\begin{definitionbox}[Loi Normale Standard (ou Centrée Réduite)]
Un cas particulier extraordinairement utile est la loi normale avec une moyenne $\mu=0$ et une variance $\sigma^2=1$ (donc $\sigma=1$). On l'appelle la \textbf{loi normale standard} ou \textbf{centrée réduite}, et on la note souvent $Z$. Sa PDF est traditionnellement notée $\phi(z)$ :
$$ \phi(z) = \frac{1}{\sqrt{2\pi}} e^{-z^2/2} $$
Sa fonction de répartition (CDF), qui donne $P(Z \le z)$, est notée $\Phi(z)$ :
$$ \Phi(z) = P(Z \le z) = \int_{-\infty}^z \frac{1}{\sqrt{2\pi}} e^{-t^2/2} \, \mathrm{d}t $$
\end{definitionbox}

Pourquoi cette version standard est-elle si importante ? Elle sert de référence universelle.

\begin{intuitionbox}[La Référence Universelle et le Changement d'Unités]
Pourquoi cette loi $\mathcal{N}(0, 1)$ est-elle si centrale ? Imaginez que vous ayez des mesures en degrés Celsius ($\mathcal{N}(\mu_C, \sigma_C^2)$) et d'autres en degrés Fahrenheit ($\mathcal{N}(\mu_F, \sigma_F^2)$). Comment les comparer ? La loi normale standard fournit un \textbf{système d'unités universel}.

Toute variable normale $X \sim \mathcal{N}(\mu, \sigma^2)$ peut être transformée ("standardisée") en une variable $Z \sim \mathcal{N}(0, 1)$ par un simple changement d'échelle et de position : $Z = (X-\mu)/\sigma$. 

Cela signifie qu'au lieu de devoir calculer des aires (probabilités) pour une infinité de courbes en cloche différentes (une pour chaque paire $\mu, \sigma$), on peut tout ramener à \textbf{une seule courbe de référence}, $\mathcal{N}(0, 1)$. Les aires sous cette courbe standard ($\Phi(z)$) ont été calculées une fois pour toutes et sont disponibles dans des tables ou des logiciels. On n'a plus qu'à convertir notre problème dans cette "langue" standard, trouver la probabilité, et interpréter le résultat.
\end{intuitionbox}

La notation est très standardisée pour cette loi.

\begin{remarquebox}[Notation $\phi$ et $\Phi$]
Les symboles $\phi$ (phi minuscule) pour la PDF et $\Phi$ (phi majuscule) pour la CDF de la loi normale standard sont quasi universels. Il est important de ne pas les confondre. $\phi(z)$ est la \textit{hauteur} de la courbe en $z$, tandis que $\Phi(z)$ est l'\textit{aire} sous la courbe à gauche de $z$.
\end{remarquebox}

Un détail technique important concerne le calcul de $\Phi(z)$.

\begin{remarquebox}[Absence de Primitive Simple]
L'intégrale $\int e^{-t^2/2} \, \mathrm{d}t$, nécessaire pour calculer $\Phi(z)$, n'a \textbf{pas d'expression analytique} en termes de fonctions élémentaires (polynômes, exponentielles, log, sin, cos...). C'est une fonction spéciale, connue sous le nom de \textbf{fonction d'erreur} (liée à $\Phi$ par une transformation simple). C'est la raison pour laquelle on dépend de tables ou de calculs numériques pour obtenir les valeurs de $\Phi(z)$. Heureusement, ces outils sont omniprésents aujourd'hui.
\end{remarquebox}

\subsection{Standardisation : Le Score Z}

Formalisons cette transformation clé qui relie toute loi normale à la loi standard.

\begin{theorembox}[Standardisation d'une Variable Normale]
Si $X \sim \mathcal{N}(\mu, \sigma^2)$, alors la variable $Z$ définie par :
$$ Z = \frac{X - \mu}{\sigma} $$
suit la loi normale standard, $Z \sim \mathcal{N}(0, 1)$.
\end{theorembox}

La preuve formelle utilise un changement de variable dans la fonction de répartition.

\begin{proofbox}
Soit $F_X(x)$ la CDF de $X$ et $F_Z(z)$ la CDF de $Z$. Nous voulons montrer que $F_Z(z) = \Phi(z)$.
\begin{align*}
F_Z(z) &= P(Z \le z) \\
&= P\left( \frac{X-\mu}{\sigma} \le z \right) \\
&= P(X - \mu \le z\sigma) \\
&= P(X \le \mu + z\sigma) \\
&= F_X(\mu + z\sigma)
\end{align*}
Par définition de la CDF de $X$ :
$$ F_X(x) = \int_{-\infty}^x \frac{1}{\sigma \sqrt{2\pi}} e^{ -\frac{1}{2} \left( \frac{t-\mu}{\sigma} \right)^2 } \, dt $$
Donc,
$$ F_Z(z) = \int_{-\infty}^{\mu + z\sigma} \frac{1}{\sigma \sqrt{2\pi}} e^{ -\frac{1}{2} \left( \frac{t-\mu}{\sigma} \right)^2 } \, dt $$
Effectuons le changement de variable $u = (t-\mu)/\sigma$. Alors $t = \mu + u\sigma$ et $dt = \sigma du$.
Les bornes d'intégration changent :
\begin{itemize}
    \item Quand $t \to -\infty$, $u \to -\infty$.
    \item Quand $t = \mu + z\sigma$, $u = ((\mu + z\sigma)-\mu)/\sigma = z$.
\end{itemize}
L'intégrale devient :
$$ F_Z(z) = \int_{-\infty}^{z} \frac{1}{\sigma \sqrt{2\pi}} e^{ -\frac{1}{2} u^2 } (\sigma du) $$
$$ F_Z(z) = \int_{-\infty}^{z} \frac{1}{\sqrt{2\pi}} e^{ -u^2/2 } \, du $$
C'est exactement la définition de $\Phi(z)$, la CDF de la loi normale standard. Ainsi, $Z \sim \mathcal{N}(0, 1)$.
\end{proofbox}

Cette transformation a une interprétation très concrète.

\begin{intuitionbox}[Mesurer en "Unités d'Écart-Type"]
Transformer $X$ en $Z$ s'appelle \textbf{standardiser} la variable. Le résultat, $z = \frac{x-\mu}{\sigma}$, est appelé le \textbf{Score Z} (ou cote Z). Ce score Z est une mesure \textit{sans unité} qui indique \textbf{à combien d'écarts-types} une valeur observée $x$ se situe par rapport à la moyenne $\mu$ de sa distribution.
\begin{itemize}
    \item $z = 0$ : $x$ est exactement à la moyenne ($\mathbf{x = \mu}$).
    \item $z = +1$ : $x$ est un écart-type \textit{au-dessus} de la moyenne ($\mathbf{x = \mu + \sigma}$).
    \item $z = -2$ : $x$ est deux écarts-types \textit{en dessous} de la moyenne ($\mathbf{x = \mu - 2\sigma}$).
\end{itemize}
Cette transformation est extrêmement utile pour :
\begin{enumerate}
    \item \textbf{Comparer des valeurs} issues de distributions normales différentes. Un score Z de +1.5 a toujours la même signification relative, que l'on parle de QI, de taille, ou de température.
    \item \textbf{Calculer des probabilités} en utilisant la table unique de la loi $\mathcal{N}(0, 1)$.
\end{enumerate}
\end{intuitionbox}

Un exemple classique est la comparaison de notes.

\begin{examplebox}[Comparaison de Performances]
Un étudiant A obtient 80 points à un examen où la moyenne est $\mu_A=70$ et l'écart-type $\sigma_A=5$. Un étudiant B obtient 85 points à un autre examen où $\mu_B=75$ et $\sigma_B=10$. Qui a le mieux réussi relativement à son groupe ?

Calculons les Z-scores :
$$ Z_A = \frac{80 - 70}{5} = \frac{10}{5} = +2.0 $$
$$ Z_B = \frac{85 - 75}{10} = \frac{10}{10} = +1.0 $$
L'étudiant A a un score Z plus élevé (+2.0 contre +1.0), ce qui signifie qu'il se situe plus d'écarts-types au-dessus de la moyenne de son groupe que l'étudiant B. L'étudiant A a donc relativement mieux réussi.
\end{examplebox}

\subsection{Propriétés Importantes de la Loi Normale}

La loi normale possède des propriétés de stabilité remarquables sous certaines transformations.

\begin{theorembox}[Stabilité par Transformation Linéaire]
Si $X \sim \mathcal{N}(\mu, \sigma^2)$ et $Y = aX + b$ (avec $a \neq 0$), alors $Y$ suit aussi une loi normale :
$$ Y \sim \mathcal{N}(a\mu + b, \, (a\sigma)^2) $$
L'espérance est transformée linéairement ($E[aX+b] = aE[X]+b$), et la variance est multipliée par $a^2$ ($\text{Var}(aX+b) = a^2\text{Var}(X)$).
\end{theorembox}

\begin{proofbox}
Nous utilisons le fait que si $X \sim \mathcal{N}(\mu, \sigma^2)$, alors $Z = (X-\mu)/\sigma \sim \mathcal{N}(0,1)$.
Exprimons $X$ en fonction de $Z$ : $X = \mu + \sigma Z$.
Substituons cela dans l'expression de $Y$:
$$ Y = a(\mu + \sigma Z) + b = (a\mu + b) + (a\sigma)Z $$
Posons $\mu_Y = a\mu + b$ et $\sigma_Y = |a|\sigma$. Alors $Y = \mu_Y + \sigma_Y Z$ (si $a>0$) ou $Y = \mu_Y - \sigma_Y Z$ (si $a<0$).
Dans les deux cas, $Y$ est une transformation linéaire d'une variable normale standard $Z$.
La CDF de $Y$ peut être exprimée en termes de la CDF $\Phi$ de $Z$.
Si $a>0$ :
$$ P(Y \le y) = P(\mu_Y + a\sigma Z \le y) = P(a\sigma Z \le y - \mu_Y) = P\left( Z \le \frac{y - \mu_Y}{a\sigma} \right) = \Phi\left(\frac{y - \mu_Y}{a\sigma}\right) $$
C'est la CDF d'une loi $\mathcal{N}(\mu_Y, (a\sigma)^2)$.
Le cas $a<0$ est similaire et mène au même résultat pour la distribution (la variance dépend de $a^2$).
Ainsi, $Y \sim \mathcal{N}(a\mu + b, (a\sigma)^2)$.
\end{proofbox}

Cette propriété est très utile pour les changements d'unités.

\begin{examplebox}[Changement d'Unités]
Si la température en Celsius $T_C$ suit $\mathcal{N}(20, 5^2)$, quelle est la loi de la température en Fahrenheit $T_F = \frac{9}{5}T_C + 32$ ?

$a = 9/5$, $b=32$.

Nouvelle moyenne : $E[T_F] = \frac{9}{5}(20) + 32 = 36 + 32 = 68$.

Nouvel écart-type : $\sigma_{T_F} = |a|\sigma_{T_C} = \frac{9}{5}(5) = 9$. Nouvelle variance : $\sigma_{T_F}^2 = 9^2 = 81$.

Donc, $T_F \sim \mathcal{N}(68, 9^2)$.
\end{examplebox}

Une autre propriété cruciale concerne la somme de variables normales indépendantes.

\begin{theorembox}[Stabilité par Addition (Indépendance)]
Si $X \sim \mathcal{N}(\mu_X, \sigma_X^2)$ et $Y \sim \mathcal{N}(\mu_Y, \sigma_Y^2)$ sont des variables aléatoires \textbf{indépendantes}, alors leur somme $S = X + Y$ suit aussi une loi normale :
$$ S \sim \mathcal{N}(\mu_X + \mu_Y, \, \sigma_X^2 + \sigma_Y^2) $$
Les moyennes s'ajoutent, et (grâce à l'indépendance) les variances s'ajoutent.
\end{theorembox}

La preuve formelle de ce théorème est plus avancée et utilise généralement les fonctions caractéristiques ou les fonctions génératrices des moments.

\begin{proofbox}[Idée de la preuve (via Fonctions Caractéristiques)]
La fonction caractéristique $\varphi_X(t)$ d'une variable aléatoire $X$ est définie comme $\varphi_X(t) = E[e^{itX}]$.
Pour une loi normale $X \sim \mathcal{N}(\mu, \sigma^2)$, sa fonction caractéristique est $\varphi_X(t) = e^{i\mu t - \frac{1}{2}\sigma^2 t^2}$.
Si $X$ et $Y$ sont indépendantes, la fonction caractéristique de leur somme $S=X+Y$ est le produit de leurs fonctions caractéristiques : $\varphi_S(t) = \varphi_X(t) \varphi_Y(t)$.
\begin{align*}
\varphi_S(t) &= \left( e^{i\mu_X t - \frac{1}{2}\sigma_X^2 t^2} \right) \left( e^{i\mu_Y t - \frac{1}{2}\sigma_Y^2 t^2} \right) \\
&= e^{i(\mu_X + \mu_Y)t - \frac{1}{2}(\sigma_X^2 + \sigma_Y^2)t^2}
\end{align*}
On reconnaît ici la fonction caractéristique d'une loi normale avec pour moyenne $\mu_X + \mu_Y$ et pour variance $\sigma_X^2 + \sigma_Y^2$. Comme la fonction caractéristique détermine de manière unique la distribution, on conclut que $S \sim \mathcal{N}(\mu_X + \mu_Y, \sigma_X^2 + \sigma_Y^2)$.
\end{proofbox}

Il est essentiel de se souvenir de la condition d'indépendance pour l'addition des variances.

\begin{remarquebox}[Attention à l'Indépendance]
La propriété d'addition des variances ($\sigma_S^2 = \sigma_X^2 + \sigma_Y^2$) est cruciale et ne tient \textbf{que si $X$ et $Y$ sont indépendantes}. Si elles ne le sont pas, la variance de la somme inclut un terme de covariance : $\text{Var}(X+Y) = \text{Var}(X) + \text{Var}(Y) + 2\text{Cov}(X, Y)$. Cependant, la somme de variables normales (même dépendantes) reste normale (si elles sont conjointement normales).
\end{remarquebox}

Appliquons ce théorème à un exemple concret.

\begin{examplebox}[Poids Total]
Le poids d'une pomme suit $\mathcal{N}(150g, 10^2)$. Le poids d'une orange suit $\mathcal{N}(200g, 15^2)$. On suppose les poids indépendants. Quel est la loi du poids total d'une pomme et d'une orange ?

Soit $P$ le poids de la pomme, $O$ celui de l'orange. $T = P+O$.

$E[T] = E[P] + E[O] = 150 + 200 = 350g$.

$\text{Var}(T) = \text{Var}(P) + \text{Var}(O) = 10^2 + 15^2 = 100 + 225 = 325$.

Donc, $T \sim \mathcal{N}(350, 325)$. L'écart-type du poids total est $\sqrt{325} \approx 18.03g$.
\end{examplebox}

\subsection{La Règle Empirique (68-95-99.7)}

Une conséquence directe des aires sous la courbe normale standard est une règle approximative très utile.

\begin{theorembox}[Règle Empirique]
Pour toute variable $X \sim \mathcal{N}(\mu, \sigma^2)$ :
\begin{itemize}
    \item $P(\mu - \sigma \le X \le \mu + \sigma) \approx 0.6827$ (Environ \textbf{68\%} des valeurs dans $\mu \pm \sigma$).
    \item $P(\mu - 2\sigma \le X \le \mu + 2\sigma) \approx 0.9545$ (Environ \textbf{95\%} des valeurs dans $\mu \pm 2\sigma$).
    \item $P(\mu - 3\sigma \le X \le \mu + 3\sigma) \approx 0.9973$ (Environ \textbf{99.7\%} des valeurs dans $\mu \pm 3\sigma$).
\end{itemize}
\end{theorembox}

\begin{proofbox}[Dérivation à partir de $\Phi(z)$]
Ces valeurs sont obtenues en calculant les aires sous la PDF de la loi normale standard $\mathcal{N}(0, 1)$ entre les Z-scores correspondants.
\begin{itemize}
    \item $P(-1 \le Z \le 1) = \Phi(1) - \Phi(-1) = \Phi(1) - (1 - \Phi(1)) = 2\Phi(1) - 1$.
    Avec $\Phi(1) \approx 0.8413$, on obtient $2(0.8413) - 1 \approx 0.6826$.
    \item $P(-2 \le Z \le 2) = \Phi(2) - \Phi(-2) = 2\Phi(2) - 1$.
    Avec $\Phi(2) \approx 0.9772$, on obtient $2(0.9772) - 1 \approx 0.9544$.
    \item $P(-3 \le Z \le 3) = \Phi(3) - \Phi(-3) = 2\Phi(3) - 1$.
    Avec $\Phi(3) \approx 0.99865$, on obtient $2(0.99865) - 1 \approx 0.9973$.
\end{itemize}
Ces valeurs sont souvent arrondies à 68%, 95%, et 99.7% pour faciliter la mémorisation.
\end{proofbox}

Cette règle fournit des repères très pratiques.

\begin{intuitionbox}[Repères Essentiels sur la Cloche]
Cette règle, dérivée directement des aires sous la courbe $\mathcal{N}(0, 1)$ entre $z=\pm 1$, $z=\pm 2$ et $z=\pm 3$, fournit des repères extrêmement utiles pour interpréter l'écart-type $\sigma$. Elle nous dit où se trouve la grande majorité des données.


Une observation qui tombe en dehors de l'intervalle $\mu \pm 3\sigma$ est très inhabituelle (elle n'a que 0.3% de chances de se produire). C'est souvent considéré comme une \textit{valeur aberrante} (outlier) potentielle.
\end{intuitionbox}

\subsection{Calcul de Probabilités Normales}

En pratique, pour calculer une probabilité $P(a \le X \le b)$ pour une loi $\mathcal{N}(\mu, \sigma^2)$, on utilise systématiquement la standardisation.

\begin{examplebox}[Utilisation du Z-score]
Supposons que le QI d'une population suit $\mathcal{N}(100, 15^2)$. Quelle est la probabilité $P(X > 130)$ ?

1.  \textbf{Standardiser :} $z = \frac{130 - 100}{15} = 2$. On cherche $P(Z > 2)$.
2.  \textbf{Utiliser la CDF Standard :} $P(Z > 2) = 1 - P(Z \le 2) = 1 - \Phi(2)$.
3.  \textbf{Chercher dans la table / Calculer :} $\Phi(2) \approx 0.9772$.
4.  \textbf{Résultat :} $P(X > 130) = 1 - 0.9772 = 0.0228$. Environ 2.3% de la population a un QI supérieur à 130.
\end{examplebox}

Pour les intervalles, on utilise la propriété $P(a \le Z \le b) = \Phi(b) - \Phi(a)$.

\begin{examplebox}[Probabilité entre deux valeurs]
Quelle est la probabilité $P(85 \le X \le 115)$ ? ($\mu=100, \sigma=15$)

1.  \textbf{Standardiser :} $z_1 = \frac{85 - 100}{15} = -1$, $z_2 = \frac{115 - 100}{15} = 1$. On cherche $P(-1 \le Z \le 1)$.
2.  \textbf{Utiliser la CDF Standard :} $P(-1 \le Z \le 1) = \Phi(1) - \Phi(-1)$.
3.  \textbf{Utiliser la symétrie :} $\Phi(-z) = 1 - \Phi(z)$. Donc $\Phi(-1) = 1 - \Phi(1)$.
    $P(-1 \le Z \le 1) = \Phi(1) - (1 - \Phi(1)) = 2\Phi(1) - 1$.
4.  \textbf{Chercher dans la table / Calculer :} $\Phi(1) \approx 0.8413$.
5.  \textbf{Résultat :} $P(85 \le X \le 115) \approx 2(0.8413) - 1 = 1.6826 - 1 = 0.6826$. (On retrouve la règle des 68% !)
\end{examplebox}

On peut aussi inverser le processus : trouver la valeur $x$ correspondant à une probabilité donnée.

\begin{examplebox}[Trouver une valeur pour une probabilité donnée (Problème Inverse)]
Quel est le QI minimum requis pour être dans le top 10\% de la population ? ($\mu=100, \sigma=15$).

1.  \textbf{Trouver le Z-score correspondant :} On cherche $x$ tel que $P(X > x) = 0.10$. Cela équivaut à $P(Z > z) = 0.10$, où $z = (x-100)/15$.
    Si $P(Z > z) = 0.10$, alors $P(Z \le z) = \Phi(z) = 1 - 0.10 = 0.90$.
2.  \textbf{Chercher dans la table inverse / Calculer :} On cherche la valeur $z$ pour laquelle l'aire à gauche est 0.90 (le 90ème percentile). On trouve $z \approx 1.28$.
3.  \textbf{Convertir en X :} On utilise la relation $z = (x-\mu)/\sigma$ pour trouver $x$:
    $1.28 = \frac{x - 100}{15}$
    $x = 100 + 1.28 \times 15 = 100 + 19.2 = 119.2$.
    Il faut un QI d'environ 119.2 pour être dans le top 10\%.
\end{examplebox}
\newpage
\section{La Loi Normale (ou Gaussienne)}

\subsection{Introduction et Fonction de Densité (PDF)}

Après les lois discrètes et les lois continues de base (Uniforme, Exponentielle), nous abordons la distribution la plus célèbre et la plus utilisée en probabilités et statistiques.

\begin{definitionbox}[Loi Normale]
Une variable aléatoire continue $X$ suit une \textbf{loi normale} (ou loi de Gauss) de paramètres $\mu$ (l'espérance) et $\sigma^2$ (la variance), notée $X \sim \mathcal{N}(\mu, \sigma^2)$, si sa fonction de densité de probabilité (PDF) est donnée par :
$$ f(x; \mu, \sigma) = \frac{1}{\sigma \sqrt{2\pi}} e^{ -\frac{1}{2} \left( \frac{x-\mu}{\sigma} \right)^2 } $$
pour tout $x \in (-\infty, \infty)$, où $\sigma > 0$.
\end{definitionbox}

Cette formule, bien qu'imposante, décrit une forme très familière : la courbe en cloche.

\begin{intuitionbox}[La Courbe en Cloche]
La loi normale est sans doute la distribution la plus importante en probabilités et statistiques. Pourquoi ? Parce qu'elle modélise remarquablement bien de nombreux phénomènes naturels et processus aléatoires où les valeurs tendent à se regrouper autour d'une moyenne, avec des écarts symétriques devenant de plus en plus rares à mesure qu'on s'éloigne de cette moyenne. Pensez à la taille des individus dans une population, aux erreurs de mesure répétées, ou même aux notes d'un grand groupe d'étudiants à un examen bien conçu. 

Sa densité a une forme caractéristique de \textbf{cloche symétrique} :
\begin{itemize}
    \item \textbf{Le Centre ($\mu$)} : Le paramètre $\mu$ représente l'\textbf{espérance} (la moyenne) de la distribution. C'est le centre de symétrie de la courbe, là où la cloche atteint son \textbf{sommet}. C'est la valeur la plus probable (le mode) et aussi la valeur qui coupe la distribution en deux moitiés égales (la médiane). Changer $\mu$ \textit{translate} la cloche horizontalement sans changer sa forme.
    \item \textbf{La Dispersion ($\sigma$)} : Le paramètre $\sigma$ est l'\textbf{écart-type} ($\sigma^2$ est la variance). Il mesure la \textbf{dispersion} des valeurs autour de la moyenne $\mu$. Géométriquement, $\sigma$ contrôle la \textbf{largeur} de la cloche.
        \begin{itemize}
            \item Un \textit{petit} $\sigma$ signifie que les données sont très concentrées autour de la moyenne, donnant une cloche \textbf{étroite et pointue}.
            \item Un \textit{grand} $\sigma$ signifie que les données sont plus étalées, donnant une cloche \textbf{large et aplatie}.
        \end{itemize}
    Les points d'inflexion de la courbe (là où la courbure change de sens) se situent exactement à $\mu \pm \sigma$.
\end{itemize}

\tcblower
\centering
\begin{tikzpicture}
    \begin{axis}[
        title={La Courbe en Cloche (PDF de la Loi Normale)},
        xlabel={$x$},
        ylabel={$f(x)$},
        axis lines=middle,
        no markers,
        samples=100,
        domain=-4:4,
        height=8cm,
        width=\linewidth-1cm,
        tick label style={font=\tiny},
        legend style={at={(0.5,-0.15)}, anchor=north, font=\small},
        legend columns=2
    ]
    % N(0, 1)
    \addplot [blue, ultra thick] {1/(sqrt(2*pi))*exp(-x^2/2)};
    \addlegendentry{$\mu=0, \sigma=1$};
    % N(0, 0.25) => sigma=0.5
    \addplot [red, ultra thick] {1/(0.5*sqrt(2*pi))*exp(-x^2/(2*0.5^2))};
    \addlegendentry{$\mu=0, \sigma=0.5$ (étroite)};
    % N(1, 2.25) => sigma=1.5
    \addplot [green!70!black, ultra thick] {1/(1.5*sqrt(2*pi))*exp(-(x-1)^2/(2*1.5^2))};
    \addlegendentry{$\mu=1, \sigma=1.5$ (large, décalée)};

    \draw [dashed] (axis cs:0,0) -- (axis cs:0, {1/(sqrt(2*pi))}) node[above, font=\tiny] {pic à $\mu=0$}; % Ligne pour mu=0
    \draw [dashed] (axis cs:1,0) -- (axis cs:1, {1/(1.5*sqrt(2*pi))}) node[above right, font=\tiny] {pic à $\mu=1$}; % Ligne pour mu=1
    \end{axis}
\end{tikzpicture}
\par\small\textit{Influence de $\mu$ (position) et $\sigma$ (largeur) sur la forme de la cloche.}
\end{intuitionbox}

Mais d'où vient cette formule spécifique ? Il existe une dérivation fascinante à partir d'hypothèses fondamentales sur les erreurs aléatoires (argument d'Herschel-Maxwell).

\begin{proofbox}[Dérivation de la Densité Normale à partir des Principes Fondamentaux]

\textbf{Contexte Visuel :} Imaginons un nuage de points dispersés autour d'une cible à l'origine $(0,0)$, comme des impacts de fléchettes. Le graphique ci-dessous illustre cette dispersion. On s'intéresse à la probabilité de tomber dans une petite zone, comme $dA$, autour d'un point $(x, y)$.

\begin{center}
\begin{tikzpicture}
\begin{axis}[
    axis lines=middle, % Axes qui passent par l'origine (0,0)
    xlabel=$x$,       % Étiquette axe X
    ylabel=$y$,       % Étiquette axe Y
    axis line style={magenta}, % Couleur des axes
    xlabel style={anchor=west, magenta}, % Style de l'étiquette X
    ylabel style={anchor=south, magenta}, % Style de l'étiquette Y
    xmin=-3.5, xmax=3.5, % Limites du graphique
    ymin=-3.5, ymax=3.5,
    tick label style={font=\tiny} % Police plus petite pour les graduations
]

% Ajout des points du nuage. Ce sont des coordonnées approximatives.
\addplot [only marks, mark=*, cyan, mark size=1.5pt]
coordinates {
    (0.2, 0.1) (-0.5, 0.2) (0.1, -0.3) (0.5, -0.8) (-0.3, -1.2) (0,0.1)
    (1.5, 1.5) (0.8, 0.8) (2.0, -0.5) (2.8, -1.4) (2.5, -0.2)
    (-1.8, -1.3) (-2.5, 0.5) (-1.5, 0.3) (-2.2, -0.8) (-0.8, 0.5)
    (0.5, 1.2) (0.7, 2.8) (0.2, 3.2) (-0.5, 1.5) (-1.0, -2.0)
    (1.8, -1.0) (1.0, -1.5) (0.3, -2.5) (-1.8, -2.8) (1.2, 0.5)
    (-1.2, -1.5) (-0.8, 1.1)
};

% --- Annotations ---

% Points et boîte pour 'dA'
\addplot [only marks, mark=*, red, mark size=1.5pt] coordinates {(1.3, 2.0)};
\draw [red, thick] (axis cs:1.1, 1.8) rectangle (axis cs:1.5, 2.2);
\node [red, above] at (axis cs:1.3, 2.2) {$dA$};

% Points et boîte pour 'dB'
\addplot [only marks, mark=*, cyan, mark size=1.5pt] 
coordinates {(-1.2, 2.0) (-1.3, 2.3) (-1.0, 2.2) (-1.1, 1.9)};
\draw [blue, thick] (axis cs:-1.8, 1.2) rectangle (axis cs:-0.8, 2.8);
\node [blue, above] at (axis cs:-1.3, 2.8) {$dB$};

\end{axis}
\end{tikzpicture}
\end{center}

\textbf{Objectif :} Expliquer comment arriver à la formule mathématique de la courbe en cloche (densité de probabilité normale) en partant de principes fondamentaux sur les erreurs aléatoires.

\textbf{1. Le Point de Départ : Densité et Aire $dA$}
Dans une distribution continue, la probabilité de tomber \textit{exactement} sur un point $(x, y)$ est nulle. On ne peut donc pas parler de "probabilité d'un point". On parle de la probabilité de tomber \textit{dans une petite zone}, comme un rectangle $dA = dx \cdot dy$ autour du point $(x, y)$.
Cette probabilité, notée $P(\text{dans } dA)$, est \textit{proportionnelle} à l'aire de la zone $dA$. La \textit{constante de proportionnalité} est la \textbf{fonction de densité de probabilité} $p(x, y)$ évaluée en ce point. En d'autres termes, la densité $p(x, y)$ \textit{représente} localement la concentration de probabilité. Ainsi, la probabilité de tomber dans la zone $dA$ est approximativement :
$$ P(\text{dans } dA) \approx p(x, y) \cdot dA $$

\textbf{2. Les Hypothèses Fondamentales}
On pose deux hypothèses sur la nature de ces erreurs (représentées par la densité $p(x, y)$) :
\begin{enumerate}
    \item \textbf{Indépendance des axes :} L'erreur horizontale ($x$) est indépendante de l'erreur verticale ($y$). Cela implique que la densité jointe $p(x, y)$ peut s'écrire comme le produit de la densité marginale sur $x$, notée $f(x)$, et de la densité marginale sur $y$, notée $f(y)$. Donc, $p(x, y) = f(x) \cdot f(y)$.
    \item \textbf{Symétrie de rotation (Isotropie) :} La densité ne dépend que de la distance $r = \sqrt{x^2 + y^2}$ au centre, pas de l'angle. Il existe donc une fonction $\phi(r)$ telle que la densité en $(x,y)$ est $p(x, y) = \phi(\sqrt{x^2 + y^2})$.
\end{enumerate}

\textbf{3. L'Équation Fonctionnelle}
En égalant les deux expressions pour la même densité $p(x, y)$ (à une constante près), on obtient :
$$ f(x) \cdot f(y) = \phi(\sqrt{x^2 + y^2}) $$
Pour $y=0$, on a $f(x) \cdot f(0) = \phi(x)$. Posons $f(0) = \lambda$. Alors $\phi(x) = \lambda f(x)$.
L'équation devient :
$$ f(x) \cdot f(y) = \lambda f(\sqrt{x^2 + y^2}) $$

\textbf{4. Résolution de l'Équation Fonctionnelle}
Posons $g(x) = f(x)/\lambda$, avec $g(0)=1$. L'équation se simplifie en :
$$ g(x) g(y) = g(\sqrt{x^2 + y^2}) $$
Posons $g(x) = h(x^2)$. L'équation devient $h(x^2)h(y^2) = h(x^2+y^2)$. Avec $a=x^2$ et $b=y^2$, on a :
$$ h(a) h(b) = h(a+b) $$
La solution continue de cette équation de Cauchy est $h(a) = e^{Aa}$ pour une constante $A$.
Retour aux fonctions : $g(x) = h(x^2) = e^{Ax^2}$. $f(x) = \lambda g(x) = \lambda e^{Ax^2}$.
Comme la densité doit diminuer loin du centre, $A$ doit être négative. Posons $A = -k$ avec $k>0$.
$$ f(x) = \lambda e^{-k x^2} $$

\textbf{5. Normalisation et Identification des Paramètres}
\begin{enumerate}
    \item \textbf{Condition $\int f(x) dx = 1$} : L'intégrale Gaussienne $\int_{-\infty}^{\infty} e^{-k x^2} \, \mathrm{d}x = \sqrt{\frac{\pi}{k}}$.
    Donc, $\int_{-\infty}^{\infty} f(x) dx = \lambda \sqrt{\frac{\pi}{k}} = 1 \implies \lambda = \sqrt{\frac{k}{\pi}}$.
    \item \textbf{Lien avec la Variance ($\sigma^2$)} : Pour une distribution centrée, $\sigma^2 = E[X^2] = \int x^2 f(x) dx$.
    $$ \sigma^2 = \int_{-\infty}^{\infty} x^2 \left( \sqrt{\frac{k}{\pi}} e^{-k x^2} \right) \, \mathrm{d}x = \sqrt{\frac{k}{\pi}} \left( \frac{1}{2k} \sqrt{\frac{\pi}{k}} \right) = \frac{1}{2k} $$
    Donc, $k = \frac{1}{2\sigma^2}$.
    \item \textbf{Substitution Finale :} Remplaçons $k$ dans $\lambda$ et $f(x)$.
    $$ \lambda = \sqrt{\frac{1/(2\sigma^2)}{\pi}} = \frac{1}{\sigma\sqrt{2\pi}} $$
    $$ f(x) = \frac{1}{\sigma\sqrt{2\pi}} e^{-\frac{1}{2\sigma^2} x^2} = \frac{1}{\sigma\sqrt{2\pi}} e^{ -\frac{x^2}{2\sigma^2} } $$
    \item \textbf{Généralisation (Moyenne $\mu$)} : Pour centrer la distribution sur $\mu$, on remplace $x$ par $(x-\mu)$ dans l'exposant :
    $$ f(x; \mu, \sigma) = \frac{1}{\sigma \sqrt{2\pi}} e^{ -\frac{(x-\mu)^2}{2\sigma^2} } $$
\end{enumerate}
C'est la fonction de densité de la loi normale $\mathcal{N}(\mu, \sigma^2)$.
\end{proofbox}

\subsection{La Loi Normale Centrée Réduite $\mathcal{N}(0, 1)$}

Avant d'explorer les propriétés de la loi normale générale, concentrons-nous sur son cas le plus simple et le plus fondamental.

\begin{definitionbox}[Loi Normale Standard (ou Centrée Réduite)]
Un cas particulier extraordinairement utile est la loi normale avec une moyenne $\mu=0$ et une variance $\sigma^2=1$ (donc $\sigma=1$). On l'appelle la \textbf{loi normale standard} ou \textbf{centrée réduite}, et on la note souvent $Z$. Sa PDF est traditionnellement notée $\phi(z)$ :
$$ \phi(z) = \frac{1}{\sqrt{2\pi}} e^{-z^2/2} $$
Sa fonction de répartition (CDF), qui donne $P(Z \le z)$, est notée $\Phi(z)$ :
$$ \Phi(z) = P(Z \le z) = \int_{-\infty}^z \frac{1}{\sqrt{2\pi}} e^{-t^2/2} \, \mathrm{d}t $$
\end{definitionbox}

Pourquoi cette version standard est-elle si importante ? Elle sert de référence universelle.

\begin{intuitionbox}[La Référence Universelle et le Changement d'Unités]
Pourquoi cette loi $\mathcal{N}(0, 1)$ est-elle si centrale ? Imaginez que vous ayez des mesures en degrés Celsius ($\mathcal{N}(\mu_C, \sigma_C^2)$) et d'autres en degrés Fahrenheit ($\mathcal{N}(\mu_F, \sigma_F^2)$). Comment les comparer ? La loi normale standard fournit un \textbf{système d'unités universel}.

Toute variable normale $X \sim \mathcal{N}(\mu, \sigma^2)$ peut être transformée ("standardisée") en une variable $Z \sim \mathcal{N}(0, 1)$ par un simple changement d'échelle et de position : $Z = (X-\mu)/\sigma$. 

Cela signifie qu'au lieu de devoir calculer des aires (probabilités) pour une infinité de courbes en cloche différentes (une pour chaque paire $\mu, \sigma$), on peut tout ramener à \textbf{une seule courbe de référence}, $\mathcal{N}(0, 1)$. Les aires sous cette courbe standard ($\Phi(z)$) ont été calculées une fois pour toutes et sont disponibles dans des tables ou des logiciels. On n'a plus qu'à convertir notre problème dans cette "langue" standard, trouver la probabilité, et interpréter le résultat.
\end{intuitionbox}

La notation est très standardisée pour cette loi.

\begin{remarquebox}[Notation $\phi$ et $\Phi$]
Les symboles $\phi$ (phi minuscule) pour la PDF et $\Phi$ (phi majuscule) pour la CDF de la loi normale standard sont quasi universels. Il est important de ne pas les confondre. $\phi(z)$ est la \textit{hauteur} de la courbe en $z$, tandis que $\Phi(z)$ est l'\textit{aire} sous la courbe à gauche de $z$.
\end{remarquebox}

Un détail technique important concerne le calcul de $\Phi(z)$.

\begin{remarquebox}[Absence de Primitive Simple]
L'intégrale $\int e^{-t^2/2} \, \mathrm{d}t$, nécessaire pour calculer $\Phi(z)$, n'a \textbf{pas d'expression analytique} en termes de fonctions élémentaires (polynômes, exponentielles, log, sin, cos...). C'est une fonction spéciale, connue sous le nom de \textbf{fonction d'erreur} (liée à $\Phi$ par une transformation simple). C'est la raison pour laquelle on dépend de tables ou de calculs numériques pour obtenir les valeurs de $\Phi(z)$. Heureusement, ces outils sont omniprésents aujourd'hui.
\end{remarquebox}

\subsection{Standardisation : Le Score Z}

Formalisons cette transformation clé qui relie toute loi normale à la loi standard.

\begin{theorembox}[Standardisation d'une Variable Normale]
Si $X \sim \mathcal{N}(\mu, \sigma^2)$, alors la variable $Z$ définie par :
$$ Z = \frac{X - \mu}{\sigma} $$
suit la loi normale standard, $Z \sim \mathcal{N}(0, 1)$.
\end{theorembox}

La preuve formelle utilise un changement de variable dans la fonction de répartition.

\begin{proofbox}
Soit $F_X(x)$ la CDF de $X$ et $F_Z(z)$ la CDF de $Z$. Nous voulons montrer que $F_Z(z) = \Phi(z)$.
\begin{align*}
F_Z(z) &= P(Z \le z) \\
&= P\left( \frac{X-\mu}{\sigma} \le z \right) \\
&= P(X - \mu \le z\sigma) \\
&= P(X \le \mu + z\sigma) \\
&= F_X(\mu + z\sigma)
\end{align*}
Par définition de la CDF de $X$ :
$$ F_X(x) = \int_{-\infty}^x \frac{1}{\sigma \sqrt{2\pi}} e^{ -\frac{1}{2} \left( \frac{t-\mu}{\sigma} \right)^2 } \, dt $$
Donc,
$$ F_Z(z) = \int_{-\infty}^{\mu + z\sigma} \frac{1}{\sigma \sqrt{2\pi}} e^{ -\frac{1}{2} \left( \frac{t-\mu}{\sigma} \right)^2 } \, dt $$
Effectuons le changement de variable $u = (t-\mu)/\sigma$. Alors $t = \mu + u\sigma$ et $dt = \sigma du$.
Les bornes d'intégration changent :
\begin{itemize}
    \item Quand $t \to -\infty$, $u \to -\infty$.
    \item Quand $t = \mu + z\sigma$, $u = ((\mu + z\sigma)-\mu)/\sigma = z$.
\end{itemize}
L'intégrale devient :
$$ F_Z(z) = \int_{-\infty}^{z} \frac{1}{\sigma \sqrt{2\pi}} e^{ -\frac{1}{2} u^2 } (\sigma du) $$
$$ F_Z(z) = \int_{-\infty}^{z} \frac{1}{\sqrt{2\pi}} e^{ -u^2/2 } \, du $$
C'est exactement la définition de $\Phi(z)$, la CDF de la loi normale standard. Ainsi, $Z \sim \mathcal{N}(0, 1)$.
\end{proofbox}

Cette transformation a une interprétation très concrète.

\begin{intuitionbox}[Mesurer en "Unités d'Écart-Type"]
Transformer $X$ en $Z$ s'appelle \textbf{standardiser} la variable. Le résultat, $z = \frac{x-\mu}{\sigma}$, est appelé le \textbf{Score Z} (ou cote Z). Ce score Z est une mesure \textit{sans unité} qui indique \textbf{à combien d'écarts-types} une valeur observée $x$ se situe par rapport à la moyenne $\mu$ de sa distribution.
\begin{itemize}
    \item $z = 0$ : $x$ est exactement à la moyenne ($\mathbf{x = \mu}$).
    \item $z = +1$ : $x$ est un écart-type \textit{au-dessus} de la moyenne ($\mathbf{x = \mu + \sigma}$).
    \item $z = -2$ : $x$ est deux écarts-types \textit{en dessous} de la moyenne ($\mathbf{x = \mu - 2\sigma}$).
\end{itemize}
Cette transformation est extrêmement utile pour :
\begin{enumerate}
    \item \textbf{Comparer des valeurs} issues de distributions normales différentes. Un score Z de +1.5 a toujours la même signification relative, que l'on parle de QI, de taille, ou de température.
    \item \textbf{Calculer des probabilités} en utilisant la table unique de la loi $\mathcal{N}(0, 1)$.
\end{enumerate}
\end{intuitionbox}

Un exemple classique est la comparaison de notes.

\begin{examplebox}[Comparaison de Performances]
Un étudiant A obtient 80 points à un examen où la moyenne est $\mu_A=70$ et l'écart-type $\sigma_A=5$. Un étudiant B obtient 85 points à un autre examen où $\mu_B=75$ et $\sigma_B=10$. Qui a le mieux réussi relativement à son groupe ?

Calculons les Z-scores :
$$ Z_A = \frac{80 - 70}{5} = \frac{10}{5} = +2.0 $$
$$ Z_B = \frac{85 - 75}{10} = \frac{10}{10} = +1.0 $$
L'étudiant A a un score Z plus élevé (+2.0 contre +1.0), ce qui signifie qu'il se situe plus d'écarts-types au-dessus de la moyenne de son groupe que l'étudiant B. L'étudiant A a donc relativement mieux réussi.
\end{examplebox}

\subsection{Propriétés Importantes de la Loi Normale}

La loi normale possède des propriétés de stabilité remarquables sous certaines transformations.

\begin{theorembox}[Stabilité par Transformation Linéaire]
Si $X \sim \mathcal{N}(\mu, \sigma^2)$ et $Y = aX + b$ (avec $a \neq 0$), alors $Y$ suit aussi une loi normale :
$$ Y \sim \mathcal{N}(a\mu + b, \, (a\sigma)^2) $$
L'espérance est transformée linéairement ($E[aX+b] = aE[X]+b$), et la variance est multipliée par $a^2$ ($\text{Var}(aX+b) = a^2\text{Var}(X)$).
\end{theorembox}

\begin{proofbox}
Nous utilisons le fait que si $X \sim \mathcal{N}(\mu, \sigma^2)$, alors $Z = (X-\mu)/\sigma \sim \mathcal{N}(0,1)$.
Exprimons $X$ en fonction de $Z$ : $X = \mu + \sigma Z$.
Substituons cela dans l'expression de $Y$:
$$ Y = a(\mu + \sigma Z) + b = (a\mu + b) + (a\sigma)Z $$
Posons $\mu_Y = a\mu + b$ et $\sigma_Y = |a|\sigma$. Alors $Y = \mu_Y + \sigma_Y Z$ (si $a>0$) ou $Y = \mu_Y - \sigma_Y Z$ (si $a<0$).
Dans les deux cas, $Y$ est une transformation linéaire d'une variable normale standard $Z$.
La CDF de $Y$ peut être exprimée en termes de la CDF $\Phi$ de $Z$.
Si $a>0$ :
$$ P(Y \le y) = P(\mu_Y + a\sigma Z \le y) = P(a\sigma Z \le y - \mu_Y) = P\left( Z \le \frac{y - \mu_Y}{a\sigma} \right) = \Phi\left(\frac{y - \mu_Y}{a\sigma}\right) $$
C'est la CDF d'une loi $\mathcal{N}(\mu_Y, (a\sigma)^2)$.
Le cas $a<0$ est similaire et mène au même résultat pour la distribution (la variance dépend de $a^2$).
Ainsi, $Y \sim \mathcal{N}(a\mu + b, (a\sigma)^2)$.
\end{proofbox}

Cette propriété est très utile pour les changements d'unités.

\begin{examplebox}[Changement d'Unités]
Si la température en Celsius $T_C$ suit $\mathcal{N}(20, 5^2)$, quelle est la loi de la température en Fahrenheit $T_F = \frac{9}{5}T_C + 32$ ?

$a = 9/5$, $b=32$.

Nouvelle moyenne : $E[T_F] = \frac{9}{5}(20) + 32 = 36 + 32 = 68$.

Nouvel écart-type : $\sigma_{T_F} = |a|\sigma_{T_C} = \frac{9}{5}(5) = 9$. Nouvelle variance : $\sigma_{T_F}^2 = 9^2 = 81$.

Donc, $T_F \sim \mathcal{N}(68, 9^2)$.
\end{examplebox}

Une autre propriété cruciale concerne la somme de variables normales indépendantes.

\begin{theorembox}[Stabilité par Addition (Indépendance)]
Si $X \sim \mathcal{N}(\mu_X, \sigma_X^2)$ et $Y \sim \mathcal{N}(\mu_Y, \sigma_Y^2)$ sont des variables aléatoires \textbf{indépendantes}, alors leur somme $S = X + Y$ suit aussi une loi normale :
$$ S \sim \mathcal{N}(\mu_X + \mu_Y, \, \sigma_X^2 + \sigma_Y^2) $$
Les moyennes s'ajoutent, et (grâce à l'indépendance) les variances s'ajoutent.
\end{theorembox}

La preuve formelle de ce théorème est plus avancée et utilise généralement les fonctions caractéristiques ou les fonctions génératrices des moments.

\begin{proofbox}[Idée de la preuve (via Fonctions Caractéristiques)]
La fonction caractéristique $\varphi_X(t)$ d'une variable aléatoire $X$ est définie comme $\varphi_X(t) = E[e^{itX}]$.
Pour une loi normale $X \sim \mathcal{N}(\mu, \sigma^2)$, sa fonction caractéristique est $\varphi_X(t) = e^{i\mu t - \frac{1}{2}\sigma^2 t^2}$.
Si $X$ et $Y$ sont indépendantes, la fonction caractéristique de leur somme $S=X+Y$ est le produit de leurs fonctions caractéristiques : $\varphi_S(t) = \varphi_X(t) \varphi_Y(t)$.
\begin{align*}
\varphi_S(t) &= \left( e^{i\mu_X t - \frac{1}{2}\sigma_X^2 t^2} \right) \left( e^{i\mu_Y t - \frac{1}{2}\sigma_Y^2 t^2} \right) \\
&= e^{i(\mu_X + \mu_Y)t - \frac{1}{2}(\sigma_X^2 + \sigma_Y^2)t^2}
\end{align*}
On reconnaît ici la fonction caractéristique d'une loi normale avec pour moyenne $\mu_X + \mu_Y$ et pour variance $\sigma_X^2 + \sigma_Y^2$. Comme la fonction caractéristique détermine de manière unique la distribution, on conclut que $S \sim \mathcal{N}(\mu_X + \mu_Y, \sigma_X^2 + \sigma_Y^2)$.
\end{proofbox}

Il est essentiel de se souvenir de la condition d'indépendance pour l'addition des variances.

\begin{remarquebox}[Attention à l'Indépendance]
La propriété d'addition des variances ($\sigma_S^2 = \sigma_X^2 + \sigma_Y^2$) est cruciale et ne tient \textbf{que si $X$ et $Y$ sont indépendantes}. Si elles ne le sont pas, la variance de la somme inclut un terme de covariance : $\text{Var}(X+Y) = \text{Var}(X) + \text{Var}(Y) + 2\text{Cov}(X, Y)$. Cependant, la somme de variables normales (même dépendantes) reste normale (si elles sont conjointement normales).
\end{remarquebox}

Appliquons ce théorème à un exemple concret.

\begin{examplebox}[Poids Total]
Le poids d'une pomme suit $\mathcal{N}(150g, 10^2)$. Le poids d'une orange suit $\mathcal{N}(200g, 15^2)$. On suppose les poids indépendants. Quel est la loi du poids total d'une pomme et d'une orange ?

Soit $P$ le poids de la pomme, $O$ celui de l'orange. $T = P+O$.

$E[T] = E[P] + E[O] = 150 + 200 = 350g$.

$\text{Var}(T) = \text{Var}(P) + \text{Var}(O) = 10^2 + 15^2 = 100 + 225 = 325$.

Donc, $T \sim \mathcal{N}(350, 325)$. L'écart-type du poids total est $\sqrt{325} \approx 18.03g$.
\end{examplebox}

\subsection{La Règle Empirique (68-95-99.7)}

Une conséquence directe des aires sous la courbe normale standard est une règle approximative très utile.

\begin{theorembox}[Règle Empirique]
Pour toute variable $X \sim \mathcal{N}(\mu, \sigma^2)$ :
\begin{itemize}
    \item $P(\mu - \sigma \le X \le \mu + \sigma) \approx 0.6827$ (Environ \textbf{68\%} des valeurs dans $\mu \pm \sigma$).
    \item $P(\mu - 2\sigma \le X \le \mu + 2\sigma) \approx 0.9545$ (Environ \textbf{95\%} des valeurs dans $\mu \pm 2\sigma$).
    \item $P(\mu - 3\sigma \le X \le \mu + 3\sigma) \approx 0.9973$ (Environ \textbf{99.7\%} des valeurs dans $\mu \pm 3\sigma$).
\end{itemize}
\end{theorembox}

\begin{proofbox}[Dérivation à partir de $\Phi(z)$]
Ces valeurs sont obtenues en calculant les aires sous la PDF de la loi normale standard $\mathcal{N}(0, 1)$ entre les Z-scores correspondants.
\begin{itemize}
    \item $P(-1 \le Z \le 1) = \Phi(1) - \Phi(-1) = \Phi(1) - (1 - \Phi(1)) = 2\Phi(1) - 1$.
    Avec $\Phi(1) \approx 0.8413$, on obtient $2(0.8413) - 1 \approx 0.6826$.
    \item $P(-2 \le Z \le 2) = \Phi(2) - \Phi(-2) = 2\Phi(2) - 1$.
    Avec $\Phi(2) \approx 0.9772$, on obtient $2(0.9772) - 1 \approx 0.9544$.
    \item $P(-3 \le Z \le 3) = \Phi(3) - \Phi(-3) = 2\Phi(3) - 1$.
    Avec $\Phi(3) \approx 0.99865$, on obtient $2(0.99865) - 1 \approx 0.9973$.
\end{itemize}
Ces valeurs sont souvent arrondies à 68%, 95%, et 99.7% pour faciliter la mémorisation.
\end{proofbox}

Cette règle fournit des repères très pratiques.

\begin{intuitionbox}[Repères Essentiels sur la Cloche]
Cette règle, dérivée directement des aires sous la courbe $\mathcal{N}(0, 1)$ entre $z=\pm 1$, $z=\pm 2$ et $z=\pm 3$, fournit des repères extrêmement utiles pour interpréter l'écart-type $\sigma$. Elle nous dit où se trouve la grande majorité des données.


Une observation qui tombe en dehors de l'intervalle $\mu \pm 3\sigma$ est très inhabituelle (elle n'a que 0.3% de chances de se produire). C'est souvent considéré comme une \textit{valeur aberrante} (outlier) potentielle.
\end{intuitionbox}

\subsection{Calcul de Probabilités Normales}

En pratique, pour calculer une probabilité $P(a \le X \le b)$ pour une loi $\mathcal{N}(\mu, \sigma^2)$, on utilise systématiquement la standardisation.

\begin{examplebox}[Utilisation du Z-score]
Supposons que le QI d'une population suit $\mathcal{N}(100, 15^2)$. Quelle est la probabilité $P(X > 130)$ ?

1.  \textbf{Standardiser :} $z = \frac{130 - 100}{15} = 2$. On cherche $P(Z > 2)$.
2.  \textbf{Utiliser la CDF Standard :} $P(Z > 2) = 1 - P(Z \le 2) = 1 - \Phi(2)$.
3.  \textbf{Chercher dans la table / Calculer :} $\Phi(2) \approx 0.9772$.
4.  \textbf{Résultat :} $P(X > 130) = 1 - 0.9772 = 0.0228$. Environ 2.3% de la population a un QI supérieur à 130.
\end{examplebox}

Pour les intervalles, on utilise la propriété $P(a \le Z \le b) = \Phi(b) - \Phi(a)$.

\begin{examplebox}[Probabilité entre deux valeurs]
Quelle est la probabilité $P(85 \le X \le 115)$ ? ($\mu=100, \sigma=15$)

1.  \textbf{Standardiser :} $z_1 = \frac{85 - 100}{15} = -1$, $z_2 = \frac{115 - 100}{15} = 1$. On cherche $P(-1 \le Z \le 1)$.
2.  \textbf{Utiliser la CDF Standard :} $P(-1 \le Z \le 1) = \Phi(1) - \Phi(-1)$.
3.  \textbf{Utiliser la symétrie :} $\Phi(-z) = 1 - \Phi(z)$. Donc $\Phi(-1) = 1 - \Phi(1)$.
    $P(-1 \le Z \le 1) = \Phi(1) - (1 - \Phi(1)) = 2\Phi(1) - 1$.
4.  \textbf{Chercher dans la table / Calculer :} $\Phi(1) \approx 0.8413$.
5.  \textbf{Résultat :} $P(85 \le X \le 115) \approx 2(0.8413) - 1 = 1.6826 - 1 = 0.6826$. (On retrouve la règle des 68% !)
\end{examplebox}

On peut aussi inverser le processus : trouver la valeur $x$ correspondant à une probabilité donnée.

\begin{examplebox}[Trouver une valeur pour une probabilité donnée (Problème Inverse)]
Quel est le QI minimum requis pour être dans le top 10\% de la population ? ($\mu=100, \sigma=15$).

1.  \textbf{Trouver le Z-score correspondant :} On cherche $x$ tel que $P(X > x) = 0.10$. Cela équivaut à $P(Z > z) = 0.10$, où $z = (x-100)/15$.
    Si $P(Z > z) = 0.10$, alors $P(Z \le z) = \Phi(z) = 1 - 0.10 = 0.90$.
2.  \textbf{Chercher dans la table inverse / Calculer :} On cherche la valeur $z$ pour laquelle l'aire à gauche est 0.90 (le 90ème percentile). On trouve $z \approx 1.28$.
3.  \textbf{Convertir en X :} On utilise la relation $z = (x-\mu)/\sigma$ pour trouver $x$:
    $1.28 = \frac{x - 100}{15}$
    $x = 100 + 1.28 \times 15 = 100 + 19.2 = 119.2$.
    Il faut un QI d'environ 119.2 pour être dans le top 10\%.
\end{examplebox}

\subsection{Exercices}

% --- PDF, CDF et Loi Normale Standard ---

\begin{exercicebox}[Exercice 1 : Concepts de Base $\Phi(z)$]
Soit $Z \sim \mathcal{N}(0, 1)$ la loi normale standard. Sa CDF est $\Phi(z)$.
Exprimez les probabilités suivantes en termes de $\Phi(z)$ :
\begin{enumerate}
    \item $P(Z \le 1.5)$
    \item $P(Z > 1)$
    \item $P(Z \le -1.5)$ (Indice : utilisez la symétrie $\Phi(-z) = 1 - \Phi(z)$)
    \item $P(-1.5 \le Z \le 1.5)$
\end{enumerate}
\end{exercicebox}

\begin{exercicebox}[Exercice 2 : Utilisation d'une Table $\Phi(z)$]
En utilisant une table ou une calculatrice pour la loi $\mathcal{N}(0, 1)$, on sait que $\Phi(1) \approx 0.8413$, $\Phi(1.96) \approx 0.975$ et $\Phi(2) \approx 0.9772$.
Calculez :
\begin{enumerate}
    \item $P(Z > 1)$
    \item $P(Z \le -2)$
    \item $P(-1.96 \le Z \le 1.96)$
\end{enumerate}
\end{exercicebox}

\begin{exercicebox}[Exercice 3 : Propriétés de la PDF $\phi(z)$]
Soit $\phi(z)$ la PDF de la loi $\mathcal{N}(0, 1)$.
\begin{enumerate}
    \item Quelle est la valeur de $\phi(0)$ ? (Le pic de la courbe).
    \item Que vaut $\phi(z)$ par rapport à $\phi(-z)$ ?
    \item Que vaut $\int_{-\infty}^{\infty} \phi(z) \, dz$ ?
\end{enumerate}
\end{exercicebox}

% --- Standardisation (Z-score) et Calcul de Probabilités ---

\begin{exercicebox}[Exercice 4 : Calcul de Z-scores]
Une variable aléatoire $X$ suit une loi normale $\mathcal{N}(\mu=50, \sigma^2=100)$. Notez que $\sigma=10$.
Calculez le Z-score pour les valeurs suivantes de $X$ :
\begin{enumerate}
    \item $x = 60$
    \item $x = 50$
    \item $x = 35$
\end{enumerate}
\end{exercicebox}

\begin{exercicebox}[Exercice 5 : Calcul de Probabilité (Général)]
La taille des hommes adultes dans un pays suit une loi normale $\mathcal{N}(175 \text{ cm}, 7^2 \text{ cm}^2)$.
Soit $X$ la taille d'un homme choisi au hasard. Calculez :
\begin{enumerate}
    \item $P(X \le 182 \text{ cm})$ (Indice : Standardisez $x=182$ et utilisez $\Phi(1) \approx 0.8413$)
    \item $P(X > 168 \text{ cm})$
\end{enumerate}
\end{exercicebox}

\begin{exercicebox}[Exercice 6 : Calcul de Probabilité (Intervalle)]
Les scores à un test de QI suivent une loi normale $\mathcal{N}(100, 15^2)$.
Quelle est la probabilité qu'une personne choisie au hasard ait un QI compris entre 85 et 115 ?
(Indice : Standardisez les deux bornes).
\end{exercicebox}

\begin{exercicebox}[Exercice 7 : Calcul de Probabilité (Queue Extrême)]
En utilisant la même loi $\mathcal{N}(100, 15^2)$ pour le QI :
Quelle est la probabilité qu'une personne ait un QI supérieur à 130 ?
(Indice : Utilisez $\Phi(2) \approx 0.9772$).
\end{exercicebox}

% --- Problèmes Inverses (Trouver x) ---

\begin{exercicebox}[Exercice 8 : Problème Inverse (Z-score)]
Soit $Z \sim \mathcal{N}(0, 1)$. Trouvez la valeur $z$ telle que :
(Utilisez $\Phi(1.28) \approx 0.90$ et $\Phi(1.645) \approx 0.95$)
\begin{enumerate}
    \item $P(Z \le z) = 0.90$
    \item $P(Z > z) = 0.05$ (Indice : si $P(Z>z)=0.05$, que vaut $P(Z \le z)$ ?)
    \item $P(Z \le z) = 0.10$ (Indice : utilisez la symétrie)
\end{enumerate}
\end{exercicebox}

\begin{exercicebox}[Exercice 9 : Problème Inverse (Général)]
Les scores au test $\mathcal{N}(100, 15^2)$ sont utilisés pour sélectionner des candidats. Seul le top 5\% des scores est accepté.
Quel est le score minimum requis pour être accepté ?
(Indice : Utilisez $z \approx 1.645$ pour le top 5\%).
\end{exercicebox}

\begin{exercicebox}[Exercice 10 : Problème Inverse (Intervalle Central)]
Soit $Z \sim \mathcal{N}(0, 1)$. Trouvez la valeur $z$ telle que $P(-z \le Z \le z) = 0.95$.
(Indice : si 95\% est au centre, combien reste-t-il dans chaque queue ? Utilisez $\Phi(1.96) \approx 0.975$).
\end{exercicebox}

\begin{exercicebox}[Exercice 11 : Problème Inverse (Général)]
La durée de vie d'une batterie suit $\mathcal{N}(500 \text{ heures}, 50^2 \text{ heures}^2)$.
Le fabricant veut offrir une garantie. Il ne veut remplacer que 2.5\% des batteries.
Quelle durée de garantie (en heures) doit-il proposer ?
(Indice : $P(Z \le -1.96) \approx 0.025$).
\end{exercicebox}

% --- Règle Empirique (68-95-99.7) ---

\begin{exercicebox}[Exercice 12 : Règle Empirique (Application)]
Le poids de paquets de café suit $\mathcal{N}(250g, 5^2g^2)$.
En utilisant la règle empirique (68-95-99.7), donnez un intervalle qui contient :
\begin{enumerate}
    \item Environ 68\% des poids.
    \item Environ 95\% des poids.
    \item Environ 99.7\% des poids.
\end{enumerate}
\end{exercicebox}

\begin{exercicebox}[Exercice 13 : Règle Empirique (Probabilité)]
En utilisant la situation de l'exercice 12 ($\mathcal{N}(250, 5^2)$) et la règle empirique :
\begin{enumerate}
    \item Estimez $P(245 \le X \le 255)$.
    \item Estimez $P(X \le 240)$. (Indice : L'intervalle $\mu \pm 2\sigma$ est [240, 260] et contient 95\%. Utilisez la symétrie).
\end{enumerate}
\end{exercicebox}

% --- Propriétés (Transformations Linéaires et Sommes) ---

\begin{exercicebox}[Exercice 14 : Transformation Linéaire (Celsius -> Fahrenheit)]
La température $T_C$ à midi en été dans une ville suit $\mathcal{N}(25, 3^2)$ (en degrés Celsius).
On convertit la température en Fahrenheit : $T_F = 1.8 \times T_C + 32$.
Quelle est la loi de $T_F$ ? (Donnez sa moyenne et sa variance).
\end{exercicebox}

\begin{exercicebox}[Exercice 15 : Transformation Linéaire (Z-score)]
Soit $X \sim \mathcal{N}(\mu, \sigma^2)$. Soit $Y = aX+b$.
Trouvez $a$ et $b$ (en fonction de $\mu$ et $\sigma$) tels que $Y \sim \mathcal{N}(0, 1)$.
\end{exercicebox}

\begin{exercicebox}[Exercice 16 : Somme de Normales Indépendantes]
Soit $X \sim \mathcal{N}(10, 3^2)$ et $Y \sim \mathcal{N}(20, 4^2)$. $X$ et $Y$ sont indépendantes.
Soit $S = X + Y$.
\begin{enumerate}
    \item Quelle est la loi de $S$ ? (Donnez sa moyenne et sa variance).
    \item Quel est l'écart-type de $S$ ?
\end{enumerate}
\end{exercicebox}

\begin{exercicebox}[Exercice 17 : Différence de Normales Indépendantes]
En utilisant $X$ et $Y$ de l'exercice 16, soit $D = Y - X$.
\begin{enumerate}
    \item Quelle est la loi de $D$ ? (Donnez sa moyenne et sa variance).
    \item Quel est l'écart-type de $D$ ? (Comparez-le à celui de $S$).
\end{enumerate}
\end{exercicebox}

\begin{exercicebox}[Exercice 18 : Application (Somme)]
Le poids d'une boîte vide $B$ suit $\mathcal{N}(100g, 5^2)$. Le poids du contenu $C$ suit $\mathcal{N}(800g, 10^2)$. $B$ et $C$ sont indépendants.
Soit $T = B+C$ le poids total.
\begin{enumerate}
    \item Quelle est la loi de $T$ ?
    \item Calculez $P(T > 925g)$. (Utilisez $\Phi(2) \approx 0.9772$).
\end{enumerate}
\end{exercicebox}

\begin{exercicebox}[Exercice 19 : Moyenne d'un Échantillon (Avancé)]
Soient $X_1, X_2, X_3, X_4$ quatre observations indépendantes de la loi $\mathcal{N}(10, 4^2)$.
Soit $\bar{X} = \frac{X_1 + X_2 + X_3 + X_4}{4}$ la moyenne de l'échantillon.
\begin{enumerate}
    \item Soit $S = X_1+X_2+X_3+X_4$. Quelle est la loi de $S$ ?
    \item En utilisant la transformation linéaire $\bar{X} = \frac{1}{4}S$, quelle est la loi de $\bar{X}$ ?
\end{enumerate}
\end{exercicebox}

\begin{exercicebox}[Exercice 20 : Comparaison (Différence)]
Alice et Bob passent un examen. Les notes d'Alice $A$ suivent $\mathcal{N}(80, 5^2)$. Les notes de Bob $B$ suivent $\mathcal{N}(78, 3^2)$. On suppose leurs notes indépendantes.
Quelle est la probabilité que Bob ait une meilleure note qu'Alice ?
(Indice : Calculez $P(B > A)$, ce qui est équivalent à $P(B - A > 0)$).
\end{exercicebox}

\subsection{Corrections des Exercices}

% --- Corrections : PDF, CDF et Loi Normale Standard ---

\begin{correctionbox}[Correction Exercice 1 : Concepts de Base $\Phi(z)$]
1.  $P(Z \le 1.5) = \Phi(1.5)$.
2.  $P(Z > 1) = 1 - P(Z \le 1) = 1 - \Phi(1)$.
3.  $P(Z \le -1.5) = 1 - P(Z \le 1.5) = 1 - \Phi(1.5)$.
4.  $P(-1.5 \le Z \le 1.5) = P(Z \le 1.5) - P(Z \le -1.5) = \Phi(1.5) - (1 - \Phi(1.5)) = 2\Phi(1.5) - 1$.
\end{correctionbox}

\begin{correctionbox}[Correction Exercice 2 : Utilisation d'une Table $\Phi(z)$]
Données : $\Phi(1) \approx 0.8413$, $\Phi(1.96) \approx 0.975$, $\Phi(2) \approx 0.9772$.
1.  $P(Z > 1) = 1 - \Phi(1) \approx 1 - 0.8413 = 0.1587$.
2.  $P(Z \le -2) = 1 - \Phi(2) \approx 1 - 0.9772 = 0.0228$.
3.  $P(-1.96 \le Z \le 1.96) = \Phi(1.96) - \Phi(-1.96) = \Phi(1.96) - (1 - \Phi(1.96))$
    $= 2\Phi(1.96) - 1 \approx 2(0.975) - 1 = 1.95 - 1 = 0.95$.
    (C'est l'intervalle de confiance à 95\%).
\end{correctionbox}

\begin{correctionbox}[Correction Exercice 3 : Propriétés de la PDF $\phi(z)$]
$\phi(z) = \frac{1}{\sqrt{2\pi}} e^{-z^2/2}$.
1.  $\phi(0) = \frac{1}{\sqrt{2\pi}} e^{0} = \frac{1}{\sqrt{2\pi}} \approx 0.3989$.
2.  Puisque $z^2 = (-z)^2$, on a $\phi(z) = \phi(-z)$. La fonction est paire (symétrique par rapport à l'axe y).
3.  Par définition d'une PDF, l'aire totale sous la courbe doit être 1. $\int_{-\infty}^{\infty} \phi(z) \, dz = 1$.
\end{correctionbox}

% --- Corrections : Standardisation (Z-score) et Calcul de Probabilités ---

\begin{correctionbox}[Correction Exercice 4 : Calcul de Z-scores]
$X \sim \mathcal{N}(\mu=50, \sigma^2=100) \implies \sigma=10$.
$Z = \frac{X - \mu}{\sigma}$.
1.  $x = 60 \implies z = (60 - 50) / 10 = 10 / 10 = 1$.
2.  $x = 50 \implies z = (50 - 50) / 10 = 0 / 10 = 0$.
3.  $x = 35 \implies z = (35 - 50) / 10 = -15 / 10 = -1.5$.
\end{correctionbox}

\begin{correctionbox}[Correction Exercice 5 : Calcul de Probabilité (Général)]
$X \sim \mathcal{N}(175, 7^2)$. $\mu=175, \sigma=7$.
1.  $P(X \le 182) = P\left(Z \le \frac{182 - 175}{7}\right) = P(Z \le \frac{7}{7}) = P(Z \le 1) = \Phi(1) \approx 0.8413$.
2.  $P(X > 168) = P\left(Z > \frac{168 - 175}{7}\right) = P(Z > \frac{-7}{7}) = P(Z > -1)$.
    Par symétrie, $P(Z > -1) = P(Z < 1) = \Phi(1) \approx 0.8413$.
\end{correctionbox}

\begin{correctionbox}[Correction Exercice 6 : Calcul de Probabilité (Intervalle)]
$X \sim \mathcal{N}(100, 15^2)$. $\mu=100, \sigma=15$.
On cherche $P(85 \le X \le 115)$.
$z_1 = (85 - 100) / 15 = -15 / 15 = -1$.
$z_2 = (115 - 100) / 15 = 15 / 15 = 1$.
$P(-1 \le Z \le 1) = \Phi(1) - \Phi(-1) = \Phi(1) - (1 - \Phi(1)) = 2\Phi(1) - 1$.
En utilisant $\Phi(1) \approx 0.8413$, $P \approx 2(0.8413) - 1 = 1.6826 - 1 = 0.6826$.
(On retrouve la règle des 68\%).
\end{correctionbox}

\begin{correctionbox}[Correction Exercice 7 : Calcul de Probabilité (Queue Extrême)]
$X \sim \mathcal{N}(100, 15^2)$.
On cherche $P(X > 130)$.
$z = (130 - 100) / 15 = 30 / 15 = 2$.
$P(X > 130) = P(Z > 2) = 1 - P(Z \le 2) = 1 - \Phi(2)$.
En utilisant $\Phi(2) \approx 0.9772$, $P \approx 1 - 0.9772 = 0.0228$.
\end{correctionbox}

% --- Corrections : Problèmes Inverses (Trouver x) ---

\begin{correctionbox}[Correction Exercice 8 : Problème Inverse (Z-score)]
1.  $P(Z \le z) = 0.90 \implies z = \Phi^{-1}(0.90) \approx 1.28$.
2.  $P(Z > z) = 0.05 \implies P(Z \le z) = 1 - 0.05 = 0.95$.
    $z = \Phi^{-1}(0.95) \approx 1.645$.
3.  $P(Z \le z) = 0.10$. C'est dans la queue gauche. Par symétrie, $z = - \Phi^{-1}(1 - 0.10) = - \Phi^{-1}(0.90)$.
    $z \approx -1.28$.
\end{correctionbox}

\begin{correctionbox}[Correction Exercice 9 : Problème Inverse (Général)]
$X \sim \mathcal{N}(100, 15^2)$. On cherche $x$ tel que $P(X > x) = 0.05$.
1.  Trouver le Z-score : $P(Z > z) = 0.05 \implies P(Z \le z) = 0.95 \implies z \approx 1.645$.
2.  Convertir en $x$ : $z = (x-\mu)/\sigma \implies x = \mu + z\sigma$.
    $x = 100 + (1.645)(15) = 100 + 24.675 = 124.675$.
    Le score minimum est d'environ 125.
\end{correctionbox}

\begin{correctionbox}[Correction Exercice 10 : Problème Inverse (Intervalle Central)]
$P(-z \le Z \le z) = 0.95$.
Si 95\% est au centre, il reste $1 - 0.95 = 0.05$ (ou 5\%) dans les deux queues.
Par symétrie, chaque queue a $0.05 / 2 = 0.025$.
La probabilité à gauche de $z$ est $P(Z \le z) = 0.95 + 0.025 = 0.975$.
On cherche $z = \Phi^{-1}(0.975)$.
En utilisant l'indice, $z \approx 1.96$.
\end{correctionbox}

\begin{correctionbox}[Correction Exercice 11 : Problème Inverse (Général)]
$X \sim \mathcal{N}(500, 50^2)$. On cherche $x$ tel que $P(X \le x) = 0.025$.
1.  Trouver le Z-score : $P(Z \le z) = 0.025$. C'est la queue gauche.
    En utilisant l'indice $P(Z \le -1.96) \approx 0.025$, on a $z \approx -1.96$.
2.  Convertir en $x$ : $x = \mu + z\sigma$.
    $x = 500 + (-1.96)(50) = 500 - 98 = 402$.
    Le fabricant doit proposer une garantie de 402 heures.
\end{correctionbox}

% --- Corrections : Règle Empirique (68-95-99.7) ---

\begin{correctionbox}[Correction Exercice 12 : Règle Empirique (Application)]
$X \sim \mathcal{N}(\mu=250, \sigma=5)$.
1.  68\% $\implies \mu \pm 1\sigma = 250 \pm 5 \implies [245, 255]$.
2.  95\% $\implies \mu \pm 2\sigma = 250 \pm 2(5) = 250 \pm 10 \implies [240, 260]$.
3.  99.7\% $\implies \mu \pm 3\sigma = 250 \pm 3(5) = 250 \pm 15 \implies [235, 265]$.
\end{correctionbox}

\begin{correctionbox}[Correction Exercice 13 : Règle Empirique (Probabilité)]
1.  $P(245 \le X \le 255)$ est l'intervalle $\mu \pm 1\sigma$.
    La probabilité est d'environ 68\% ou 0.68.
2.  L'intervalle $\mu \pm 2\sigma$ est $[240, 260]$ et contient 95\% des données.
    Il reste $100\% - 95\% = 5\%$ dans les deux queues (i.e., $P(X < 240) + P(X > 260) = 0.05$).
    Par symétrie, la queue gauche $P(X < 240)$ est $0.05 / 2 = 0.025$.
    La probabilité est d'environ 2.5\% ou 0.025.
\end{correctionbox}

% --- Corrections : Propriétés (Transformations Linéaires et Sommes) ---

\begin{correctionbox}[Correction Exercice 14 : Transformation Linéaire]
$T_C \sim \mathcal{N}(25, 3^2)$. $T_F = a T_C + b$ avec $a=1.8$ et $b=32$.
Loi de $T_F$ : $T_F \sim \mathcal{N}(a\mu + b, (a\sigma)^2)$.
Moyenne : $E[T_F] = 1.8(25) + 32 = 45 + 32 = 77$.
Variance : $\text{Var}(T_F) = (1.8)^2 \text{Var}(T_C) = (1.8)^2 (3^2) = (1.8 \times 3)^2 = (5.4)^2 = 29.16$.
Donc, $T_F \sim \mathcal{N}(77, 29.16)$.
\end{correctionbox}

\begin{correctionbox}[Correction Exercice 15 : Transformation Linéaire (Z-score)]
On veut $Y = aX+b \sim \mathcal{N}(0, 1)$.
$E[Y] = aE[X] + b = a\mu + b$. On veut $a\mu + b = 0$.
$\text{Var}(Y) = a^2 \text{Var}(X) = a^2 \sigma^2$. On veut $a^2 \sigma^2 = 1$.
De $\text{Var}(Y)=1 \implies a^2 = 1/\sigma^2 \implies a = 1/\sigma$ (en supposant $a>0$).
De $E[Y]=0 \implies (1/\sigma)\mu + b = 0 \implies b = -\mu/\sigma$.
Les constantes sont $a = 1/\sigma$ et $b = -\mu/\sigma$. (C'est la définition de la standardisation).
\end{correctionbox}

\begin{correctionbox}[Correction Exercice 16 : Somme de Normales Indépendantes]
$X \sim \mathcal{N}(10, 9)$ et $Y \sim \mathcal{N}(20, 16)$. $S = X+Y$.
1.  La somme de normales indépendantes est une normale.
    $E[S] = E[X] + E[Y] = 10 + 20 = 30$.
    $\text{Var}(S) = \text{Var}(X) + \text{Var}(Y) = 9 + 16 = 25$.
    Donc, $S \sim \mathcal{N}(30, 25)$.
2.  $\text{Var}(S) = 25 \implies \sigma_S = \sqrt{25} = 5$.
\end{correctionbox}

\begin{correctionbox}[Correction Exercice 17 : Différence de Normales Indépendantes]
$D = Y - X$.
1.  La différence est aussi une normale.
    $E[D] = E[Y] - E[X] = 20 - 10 = 10$.
    $\text{Var}(D) = \text{Var}(Y + (-1)X) = \text{Var}(Y) + (-1)^2 \text{Var}(X) = \text{Var}(Y) + \text{Var}(X)$.
    $\text{Var}(D) = 16 + 9 = 25$.
    Donc, $D \sim \mathcal{N}(10, 25)$.
2.  $\sigma_D = \sqrt{25} = 5$. (Identique à $\sigma_S$. La variance s'additionne toujours).
\end{correctionbox}

\begin{correctionbox}[Correction Exercice 18 : Application (Somme)]
$B \sim \mathcal{N}(100, 25)$, $C \sim \mathcal{N}(800, 100)$. $T = B+C$.
1.  $E[T] = E[B] + E[C] = 100 + 800 = 900$.
    $\text{Var}(T) = \text{Var}(B) + \text{Var}(C) = 25 + 100 = 125$.
    $T \sim \mathcal{N}(900, 125)$.
2.  $P(T > 925)$. $\sigma_T = \sqrt{125} = \sqrt{25 \times 5} = 5\sqrt{5} \approx 11.18$.
    $z = (925 - 900) / \sqrt{125} = 25 / (5\sqrt{5}) = 5/\sqrt{5} = \sqrt{5} \approx 2.236$.
    $P(T > 925) = P(Z > 2.236) = 1 - \Phi(2.236) \approx 1 - 0.9873 = 0.0127$.
    (Note : L'indice $\Phi(2) \approx 0.9772$ semble être une approximation pour un $z$ de 2, qui n'est pas le bon $z$ ici).
\end{correctionbox}

\begin{correctionbox}[Correction Exercice 19 : Moyenne d'un Échantillon (Avancé)]
$X_i \sim \mathcal{N}(10, 16)$ (indép.). $\bar{X} = \frac{1}{4} S$ où $S = X_1+X_2+X_3+X_4$.
1.  $S$ est une somme de normales indépendantes.
    $E[S] = E[X_1] + \dots + E[X_4] = 4 \times 10 = 40$.
    $\text{Var}(S) = \text{Var}(X_1) + \dots + \text{Var}(X_4) = 4 \times 16 = 64$.
    $S \sim \mathcal{N}(40, 64)$.
2.  $\bar{X}$ est une transformation linéaire de $S$.
    $E[\bar{X}] = E[\frac{1}{4}S] = \frac{1}{4}E[S] = \frac{1}{4}(40) = 10$.
    $\text{Var}(\bar{X}) = \text{Var}(\frac{1}{4}S) = (\frac{1}{4})^2 \text{Var}(S) = \frac{1}{16}(64) = 4$.
    $\bar{X} \sim \mathcal{N}(10, 4)$.
\end{correctionbox}

\begin{correctionbox}[Correction Exercice 20 : Comparaison (Différence)]
$A \sim \mathcal{N}(80, 25)$, $B \sim \mathcal{N}(78, 9)$. Indép.
On cherche $P(B > A)$, ce qui est $P(B - A > 0)$.
Soit $D = B - A$. $D$ suit une loi normale.
$E[D] = E[B] - E[A] = 78 - 80 = -2$.
$\text{Var}(D) = \text{Var}(B) + \text{Var}(A) = 9 + 25 = 34$.
Donc $D \sim \mathcal{N}(-2, 34)$. $\sigma_D = \sqrt{34} \approx 5.83$.
On cherche $P(D > 0)$.
$z = (0 - (-2)) / \sqrt{34} = 2 / \sqrt{34} \approx 0.342$.
$P(D > 0) = P(Z > 0.342) = 1 - \Phi(0.342) \approx 1 - 0.6338 = 0.3662$.
Il y a environ 36.6\% de chance que Bob ait une meilleure note.
\end{correctionbox}
\newpage

\section{Moments d'une distribution}

\subsection{Définitions fondamentales des moments}

Après avoir défini l'espérance ($\mu$) et la variance ($\sigma^2$), qui sont les moments d'ordre 1 et 2, nous pouvons généraliser cette idée pour capturer des informations plus subtiles sur la forme d'une distribution.

\begin{definitionbox}[Types de Moments]
Soit $X$ une variable aléatoire ayant une espérance $\mu$ et une variance $\sigma^2$. Pour tout entier positif $m$, on définit les moments suivants :
\begin{itemize}
    \item \textbf{$m$-ième moment (non centré)} : $E[X^m]$.
    \item \textbf{$m$-ième moment centré} : $E[(X - \mu)^m]$.
    \item \textbf{$m$-ième moment standardisé} : $E\left[\left(\frac{X - \mu}{\sigma}\right)^m\right]$.
\end{itemize}
Les moments centrés et standardisés permettent d'étudier les propriétés de la distribution indépendamment de sa position ($\mu$) et de son échelle ($\sigma$).
\end{definitionbox}

\subsection{Asymétrie (Skewness)}

Le premier moment nous donne la tendance centrale. Le deuxième moment (la variance) nous donne la dispersion. Le troisième moment, lui, va nous renseigner sur la \textit{symétrie} de la distribution.

\begin{definitionbox}[Asymétrie (Skewness)]
L'\textbf{asymétrie} (ou \textit{skewness}) d'une variable aléatoire $X$ de moyenne $\mu$ et d'écart-type $\sigma$ est définie comme le \textbf{troisième moment standardisé} :
$$ \text{Skew}(X) = E\left[ \left( \frac{X - \mu}{\sigma} \right)^3 \right]. $$
\end{definitionbox}

\begin{intuitionbox}[Comprendre la Formule du Skewness]
Pour une variable aléatoire $X$ de moyenne $\mu$ et d'écart-type $\sigma$, le \textbf{skewness} est défini comme :
\[
\text{Skew}(X) = \frac{E[(X - \mu)^3]}{\sigma^3}
\]

\medskip

\textbf{Logique du numérateur : le moment centré d'ordre 3}
\begin{itemize}
    \item Le terme $(X - \mu)^3$ est le \textbf{cube de l'écart à la moyenne}
    \item Contrairement à $(X - \mu)^2$ (toujours positif), le cube \textbf{conserve le signe} de l'écart
    \item Il pondère différemment les observations à gauche et à droite de la moyenne
\end{itemize}

\medskip

% --- MODIFIÉ : Tableau supprimé et fusionné dans la liste ---
\textbf{Interprétation intuitive}
\begin{itemize}
    \item \textbf{Skewness = 0 (Symétrique)} : La distribution est symétrique. Les écarts positifs et négatifs s'annulent. Typiquement : Moyenne = Médiane = Mode.
    \item \textbf{Skewness > 0 (Queue à droite)} : La distribution présente une queue longue à droite. Les grandes valeurs positives sont amplifiées par le cube. Les valeurs extrêmes tirent la moyenne vers la droite.
    \item \textbf{Skewness < 0 (Queue à gauche)} : La distribution présente une queue longue à gauche. Les écarts négatifs dominent. Les valeurs extrêmes tirent la moyenne vers la gauche.
\end{itemize}
% --- FIN MODIFICATION ---

\medskip

\textbf{Pourquoi $\sigma^3$ au dénominateur ?}
\begin{itemize}
    \item Le moment d'ordre 3 est homogène à des unités au cube
    \item On divise par $\sigma^3$ pour obtenir un coefficient \textbf{sans dimension}
    \item Permet la comparaison entre distributions de différentes échelles
\end{itemize}
\end{intuitionbox}

\begin{remarquebox}[Pourquoi Standardiser ?]
En standardisant d'abord ($\frac{X-\mu}{\sigma}$), la définition de $\text{Skew}(X)$ ne dépend ni de la position ($\mu$) ni de l'échelle ($\sigma$) de la distribution, ce qui est raisonnable puisque ces informations sont déjà fournies par la moyenne et l'écart-type. De plus, cette standardisation garantit que l'asymétrie est invariante par changement d'unité de mesure (par exemple, passer des pouces aux mètres n'affecte pas la valeur de l'asymétrie).
\end{remarquebox}

\subsection{Propriétés de symétrie}

Le skewness est une mesure numérique de l'asymétrie. Mais nous pouvons aussi définir la symétrie de manière formelle.

\begin{definitionbox}[Symétrie d'une Variable Aléatoire]
On dit qu'une variable aléatoire $X$ a une distribution \textbf{symétrique} autour de $\mu$ si la variable $X - \mu$ a la même distribution que $\mu - X$. On dit aussi que $X$ est symétrique ou que sa distribution est symétrique. Ces trois formulations ont le même sens.
\end{definitionbox}

\begin{theorembox}[Symétrie en Termes de Fonction de Densité]
Soit $X$ une variable aléatoire continue de fonction de densité de probabilité (PDF) $f$. Alors, $X$ est symétrique autour de $\mu$ si et seulement si :
$$ f(x) = f(2\mu - x) \quad \text{pour tout } x. $$
\end{theorembox}

\begin{proofbox}[Preuve du Théorème de Symétrie]
Soit $F$ la fonction de répartition (CDF) de $X$. Si la symétrie tient, alors :
$$ F(x) = P(X \le x) = P(X - \mu \le x - \mu) = P(\mu - X \le x - \mu) = P(X \ge 2\mu - x) = 1 - F(2\mu - x). $$

En prenant la dérivée des deux côtés par rapport à $x$, on obtient :
$$ f(x) = \frac{d}{dx}F(x) = \frac{d}{dx}[1 - F(2\mu - x)] = f(2\mu - x). $$

Cela démontre que la condition $f(x) = f(2\mu - x)$ est nécessaire et suffisante pour la symétrie.
\end{proofbox}

\subsection{Aplatissement (Kurtosis)}

Après l'asymétrie (ordre 3), le moment d'ordre 4 nous informe sur "l'épaisseur" des queues de la distribution, c'est-à-dire la probabilité d'obtenir des valeurs très éloignées de la moyenne.

\begin{definitionbox}[Kurtosis (Aplatissement)]
Pour une variable aléatoire $X$ de moyenne $\mu$ et d'écart-type $\sigma$, le \textbf{kurtosis} est défini comme le \textbf{quatrième moment standardisé} :
$$ \text{Kurtosis}(X) = E\left[ \left( \frac{X - \mu}{\sigma} \right)^4 \right]. $$

Dans la pratique, on utilise plus souvent le \textbf{kurtosis excessif} (ou excès de kurtosis), défini comme :
$$ \text{Excess Kurtosis}(X) = E\left[ \left( \frac{X - \mu}{\sigma} \right)^4 \right] - 3. $$
La soustraction de 3 fait en sorte que le kurtosis d'une loi normale soit égal à 0.
\end{definitionbox}

\begin{intuitionbox}[Comprendre la Kurtosis]
Pour une variable aléatoire $X$, le \textbf{kurtosis} est défini comme :
\[
\text{Kurt}(X) = \frac{E[(X - \mu)^4]}{\sigma^4}
\]
et l'\textbf{excess kurtosis} (kurtosis excédentaire) comme : $\text{Excess Kurtosis} = \text{Kurt}(X) - 3$.

\medskip

\textbf{Pourquoi le moment d'ordre 4 ?}
\begin{itemize}
    \item Comme la variance, on utilise une puissance paire (pas d'effet de signe)
    \item La puissance 4 \textbf{amplifie énormément les écarts extrêmes}
    \item Mesure le \textbf{poids des queues} et la \textbf{concentration autour de la moyenne}
\end{itemize}

\medskip

% --- MODIFIÉ : Tableau supprimé et fusionné dans la liste ---
\textbf{Interprétation intuitive (basée sur l'Excess Kurtosis)}
\begin{itemize}
    \item \textbf{Leptokurtique (Excess Kurtosis > 0)} : Kurtosis total > 3. Distribution pointue avec des queues épaisses. Les événements extrêmes sont plus probables que pour une loi normale.
    \item \textbf{Mésocurtique (Excess Kurtosis = 0)} : Kurtosis total = 3. C'est la référence (loi normale).
    \item \textbf{Platykurtique (Excess Kurtosis < 0)} : Kurtosis total < 3. Distribution aplatie avec des queues légères et un centre large. Les événements extrêmes sont moins probables.
\end{itemize}
% --- FIN MODIFICATION ---

\medskip

\textbf{Application en finance}
\begin{itemize}
    \item Les rendements financiers ont souvent un excès de kurtosis positif
    \item Indique une probabilité plus élevée d'événements extrêmes que la loi normale
    \item Justifie le "vol smile" dans les options
\end{itemize}

\medskip

\textbf{Pourquoi $\sigma^4$ au dénominateur ?}
\begin{itemize}
    \item Le moment d'ordre 4 est homogène à des unités$^4$
    \item On divise par $\sigma^4$ pour un coefficient \textbf{sans dimension}
\end{itemize}
\end{intuitionbox}

\subsection{Exemples de distributions}

Pour bien fixer les idées, comparons le skewness et le kurtosis de plusieurs distributions classiques. Notez que dans les graphiques suivants, le "Kurtosis" affiché est l'\textit{excess kurtosis} (centré à 0).

\begin{examplebox}[La Distribution Normale (Mésokurtique)]

\begin{center}
\begin{tikzpicture}
  \begin{axis}[
    width=0.8\textwidth,
    height=0.5\textwidth,
    xlabel={$x$},
    title={Densité Normale ($\mu=0$, $\sigma=1$)},
    grid=both,
    grid style={line width=.1pt, draw=gray!30},
    major grid style={line width=.2pt,draw=gray!50},
    domain=-4:4,
    samples=100,
    enlargelimits=false,
    axis lines=middle,
    xmin=-4, xmax=4,
    ymin=0, ymax=0.55
  ]
 
  % Courbe de densité
  \addplot [thick, color=blue, fill=blue!20, fill opacity=0.5] 
    {1/(sqrt(2*pi))*exp(-x^2/2)} \closedcycle;
 
  % Ligne de la moyenne
  \addplot [red, dashed, thick] coordinates {(0,0) (0,0.4)};
  \node[red, fill=white, font=\scriptsize, rounded corners, inner sep=2pt, opacity=0.8] at (axis cs:0.5,0.35) {Moyenne};
 
  % Boîte de texte avec moments seulement
  \node [draw=black, fill=white, rounded corners, font=\scriptsize, align=left, anchor=north east, opacity=0.8] 
    at (axis description cs:0.95,0.95) { % MODIFIÉ
    \textbf{Moments :}\\
    • Moyenne = 0.00\\
    • Variance = 1.00\\
    • Skewness = 0.00\\
    • Kurtosis = 0.00
    };
    
  \end{axis}
\end{tikzpicture}
\end{center}

La distribution normale est l'archétype de la courbe en cloche. Imaginez une cible : la majorité des flèches touchent le centre, et plus on s'éloigne du centre, moins il y a de chances d'être touché. C'est une distribution parfaitement symétrique, ce qui se traduit par un \textbf{skewness nul (0.00)}. Son pic est ni trop pointu, ni trop plat : c'est notre point de référence, on dit qu'elle est \textbf{mésokurtique}, d'où son kurtosis de \textbf{0.00}. C'est la base de nombreuses analyses statistiques car elle modélise naturellement beaucoup de phénomènes.

\end{examplebox}

\begin{examplebox}[La Distribution Exponentielle (Asymétrique à Droite)]

\begin{center}
\begin{tikzpicture}
  \begin{axis}[
    width=0.8\textwidth,
    height=0.5\textwidth,
    xlabel={$x$},
    title={Densité Exponentielle ($\lambda=1$)},
    grid=both,
    grid style={line width=.1pt, draw=gray!30},
    major grid style={line width=.2pt,draw=gray!50},
    domain=0:6,
    samples=100,
    enlargelimits=false,
    axis lines=middle,
    xmin=0, xmax=6,
    ymin=0, ymax=1.1
  ]
 
  % Courbe de densité exponentielle
  \addplot [thick, color=blue, fill=blue!20, fill opacity=0.5] 
    {exp(-x)} \closedcycle;
 
  % Ligne de la moyenne
  \addplot [red, dashed, thick] coordinates {(1,0) (1,0.37)};
  \node[red, fill=white, font=\scriptsize, rounded corners, inner sep=2pt, opacity=0.8] at (axis cs:1.5,0.3) {Moyenne};
 
  % Boîte de texte avec moments seulement
  \node [draw=black, fill=white, rounded corners, font=\scriptsize, align=left, anchor=north east, opacity=0.8] 
    at (axis description cs:0.95,0.95) { % MODIFIÉ
    \textbf{Moments :}\\
    • Moyenne = 1.00\\
    • Variance = 1.00\\
    • Skewness = 2.00\\
    • Kurtosis = 6.00
    };
    
  \end{axis}
\end{tikzpicture}
\end{center}

Imaginez le temps d'attente avant un événement rare, comme un appel téléphonique. La plupart du temps, l'appel arrive vite, mais il peut parfois y avoir de longues attentes. C'est exactement ce que modélise la distribution exponentielle : un pic à gauche et une longue queue à droite. Cela se traduit par un \textbf{skewness positif élevé (2.00)}, indiquant une asymétrie marquée. Elle est aussi \textbf{leptokurtique} (\textbf{kurtosis = 6.00}) : son pic est pointu, et la longue queue droite signifie qu'il y a une probabilité non négligeable de valeurs extrêmes.

\end{examplebox}

\begin{examplebox}[La Distribution Uniforme (Platykurtique)]

\begin{center}
\begin{tikzpicture}
  \begin{axis}[
    width=0.8\textwidth,
    height=0.5\textwidth,
    xlabel={$x$},
    title={Densité Uniforme ($a=0$, $b=2$)},
    grid=both,
    grid style={line width=.1pt, draw=gray!30},
    major grid style={line width=.2pt,draw=gray!50},
    domain=-0.5:2.5,
    samples=100,
    enlargelimits=false,
    axis lines=middle,
    ymin=0,
    ymax=.75,
    xmin=-1, xmax=4
  ]
 
  % Courbe de densité uniforme
  \addplot [thick, color=blue, fill=blue!20, fill opacity=0.5, const plot] 
    coordinates {(-0.5,0) (0,0) (0,0.5) (2,0.5) (2,0) (2.5,0)};
 
  % Ligne de la moyenne
  \addplot [red, dashed, thick] coordinates {(1,0) (1,0.5)};
  \node[red, fill=white, font=\scriptsize, rounded corners, inner sep=2pt, opacity=0.8] at (axis cs:1.3,0.4) {Moyenne};
 
  % Boîte de texte avec moments seulement
  \node [draw=black, fill=white, rounded corners, font=\scriptsize, align=left, anchor=north east, opacity=0.8] 
    at (axis description cs:0.95,0.95) { % MODIFIÉ
    \textbf{Moments :}\\
    • Moyenne = 1.00\\
    • Variance = 0.33\\
    • Skewness = 0.00\\
    • Kurtosis = -1.20
    };
    
  \end{axis}
\end{tikzpicture}
\end{center}

La distribution uniforme, c'est le "tirage au sort parfait" : chaque valeur sur un intervalle a la même chance d'être tirée. Visuellement, c'est un rectangle, donc aucune valeur n'est privilégiée. Elle est symétrique (\textbf{skewness = 0.00}), mais contrairement à la normale, elle est "plate", sans pic central. Cela se traduit par un \textbf{kurtosis négatif (-1.20)}, ce qui signifie qu'elle est \textbf{platykurtique}. Elle est donc très différente des distributions avec un pic central comme la normale.

\end{examplebox}

\begin{examplebox}[La Distribution Log-Normale (Fortement Leptokurtique)]

\begin{center}
\begin{tikzpicture}
  \begin{axis}[
    width=0.8\textwidth,
    height=0.5\textwidth,
    xlabel={$x$},
    title={Densité Log-Normale ($\sigma=0.7$)},
    grid=both,
    grid style={line width=.1pt, draw=gray!30},
    major grid style={line width=.2pt,draw=gray!50},
    domain=0:6,
    samples=100,
    enlargelimits=false,
    axis lines=middle,
    xmin=0, xmax=6,
    ymin=0, ymax=1
  ]
 
  % Courbe de densité log-normale
  \addplot [thick, color=blue, fill=blue!20, fill opacity=0.5] 
    {1/(x*0.7*sqrt(2*pi))*exp(-(ln(x))^2/(2*0.7^2))};
 
  % Ligne de la moyenne
  \addplot [red, dashed, thick] coordinates {(1.28,0) (1.28,0.4)};
  \node[red, fill=white, font=\scriptsize, rounded corners, inner sep=2pt, opacity=0.8] at (axis cs:1.8,0.5) {Moyenne};
 
  % Boîte de texte avec moments seulement
  \node [draw=black, fill=white, rounded corners, font=\scriptsize, align=left, anchor=north east, opacity=0.8] 
    at (axis description cs:0.95,0.95) { % MODIFIÉ
    \textbf{Moments :}\\
    • Moyenne = 1.28\\
    • Variance = 1.03\\
    • Skewness = 2.89\\
    • Kurtosis = 20.78
    };
    
  \end{axis}
\end{tikzpicture}
\end{center}

La log-normale est une distribution très asymétrique. Imaginez la richesse d'une population : la majorité est modeste, mais il existe une petite proportion de très riches, ce qui "étire" la droite de la courbe. Cela donne un \textbf{skewness très élevé (2.89)}. Elle est extrêmement \textbf{leptokurtique} (\textbf{kurtosis = 20.78}) : un pic très aigu et une queue droite très lourde. Cela signifie qu'il y a un risque élevé de valeurs extrêmement grandes, ce qui la rend très utile pour modéliser des phénomènes avec de rares événements extrêmes.

\end{examplebox}


\subsection{Exercices}

% --- Définitions Fondamentales des Moments ---

\begin{exercicebox}[Exercice 1 : Associer les Moments]
Associez chaque description au moment correspondant :
\begin{enumerate}
    \item $E[X]$
    \item $\sqrt{E[(X-\mu)^2]}$
    \item $E[(X-\mu)^2]$
    \item $E[\left(\frac{X-\mu}{\sigma}\right)^3]$
    \item $E[\left(\frac{X-\mu}{\sigma}\right)^4] - 3$
\end{enumerate}

\textbf{Termes :} Variance, Asymétrie (Skewness), Espérance (Moyenne), Écart-type, Excès de Kurtosis.
\end{exercicebox}

\begin{exercicebox}[Exercice 2 : Calcul des Moments Centrés]
Soit $X$ une v.a. avec $\mu = 2$. La PMF de $X$ est :
$P(X=0)=0.2$, $P(X=2)=0.6$, $P(X=4)=0.2$.
\begin{enumerate}
    \item Vérifiez que $E[X] = 2$.
    \item Calculez le 2ème moment centré, $E[(X-\mu)^2]$ (la Variance).
    \item Calculez le 3ème moment centré, $E[(X-\mu)^3]$.
\end{enumerate}
\end{exercicebox}

\begin{exercicebox}[Exercice 3 : Calcul des Moments Standardisés]
En utilisant la v.a. $X$ de l'exercice 2 (avec $\mu=2$ et $\text{Var}(X) = 1.6$) :
\begin{enumerate}
    \item Calculez le Skewness, $\text{Skew}(X)$.
    \item Que pouvez-vous conclure sur la symétrie de cette distribution ?
\end{enumerate}
(Rappel : $\sigma = \sqrt{1.6} \approx 1.265$).
\end{exercicebox}

% --- Asymétrie (Skewness) ---

\begin{exercicebox}[Exercice 4 : Interprétation Visuelle du Skewness]
Pour chacune des distributions décrites, indiquez si le skewness est \textbf{positif (> 0)}, \textbf{négatif (< 0)} ou \textbf{nul (= 0)}.
\begin{enumerate}
    \item La distribution $\text{Exp}(\lambda)$ (queue longue à droite).
    \item Une distribution où Moyenne = 10, Médiane = 12, Mode = 13 (queue longue à gauche).
    \item La distribution $\mathcal{N}(\mu, \sigma^2)$.
    \item La distribution des revenus dans un pays (beaucoup de salaires bas, quelques salaires très élevés).
\end{enumerate}
\end{exercicebox}

\begin{exercicebox}[Exercice 5 : Skewness d'une Distribution Simple]
Soit $X$ une v.a. : $P(X=0)=0.1$, $P(X=1)=0.8$, $P(X=10)=0.1$.
\begin{enumerate}
    \item Calculez $\mu = E[X]$.
    \item Calculez $\sigma^2 = \text{Var}(X)$.
    \item Calculez $E[(X-\mu)^3]$.
    \item Le skewness est-il positif, négatif ou nul ? (Le calcul complet n'est pas nécessaire si vous justifiez).
\end{enumerate}
\end{exercicebox}

\begin{exercicebox}[Exercice 6 : Skewness d'une Variable de Bernoulli]
Soit $X \sim \text{Bern}(p)$. On rappelle que $\mu = p$ et $\sigma^2 = p(1-p)$.
\begin{enumerate}
    \item Calculez $E[X^3]$. (Indice : $X^3 = X$).
    \item Calculez le 3ème moment centré $E[(X-p)^3]$ en développant l'expression.
    \item Calculez le skewness $\text{Skew}(X) = \frac{E[(X-p)^3]}{\sigma^3}$.
    \item Pour quelle valeur de $p$ cette distribution est-elle symétrique ?
\end{enumerate}
\end{exercicebox}

% --- Symétrie ---

\begin{exercicebox}[Exercice 7 : Définition de la Symétrie]
Soit $X \sim \text{Unif}(-5, 5)$.
\begin{enumerate}
    \item Autour de quel point $\mu$ cette distribution est-elle symétrique ?
    \item Montrez que $f(x) = f(2\mu - x)$ en utilisant la PDF de la loi uniforme.
\end{enumerate}
\end{exercicebox}

\begin{exercicebox}[Exercice 8 : Symétrie et Moments Centrés]
Soit $X$ une v.a. symétrique autour de sa moyenne $\mu$.
Montrez que tous ses moments centrés d'ordre impair sont nuls, c'est-à-dire $E[(X-\mu)^k] = 0$ pour $k=1, 3, 5, \dots$
(Indice : Soit $Y = X-\mu$. $Y$ est symétrique autour de 0. Que vaut $E[Y^k]$ ?).
\end{exercicebox}

\begin{exercicebox}[Exercice 9 : Symétrie de la Loi Normale]
Soit $X \sim \mathcal{N}(\mu, \sigma^2)$.
\begin{enumerate}
    \item La distribution est-elle symétrique ?
    \item Que vaut $\text{Skew}(X)$ ? (Sans calcul, en utilisant le résultat de l'exercice 8).
\end{enumerate}
\end{exercicebox}

\begin{exercicebox}[Exercice 10 : Symétrie et Skewness]
Si $\text{Skew}(X) = 0$, peut-on affirmer que la distribution de $X$ est symétrique ?
(Indice : Pensez à une distribution qui n'est pas symétrique mais où les asymétries se compensent, ex: $P(X=-4)=0.1, P(X=-1)=0.4, P(X=2)=0.5$).
\end{exercicebox}

% --- Aplatissement (Kurtosis) ---

\begin{exercicebox}[Exercice 11 : Interprétation du Kurtosis]
Associez chaque type de distribution à sa description du kurtosis (excédentaire) :
\begin{enumerate}
    \item \textbf{Leptokurtique}
    \item \textbf{Mésokurtique}
    \item \textbf{Platykurtique}
\end{enumerate}

\textbf{Descriptions :}
A. Excès de Kurtosis = 0 (Référence de la loi normale).
B. Excès de Kurtosis < 0 (Distribution "plate" avec des queues fines).
C. Excès de Kurtosis > 0 (Distribution "pointue" avec des queues épaisses).
\end{exercicebox}

\begin{exercicebox}[Exercice 12 : Kurtosis et queues]
Une distribution A a un excès de kurtosis de 5. Une distribution B a un excès de kurtosis de 1.
Laquelle des deux distributions est la plus susceptible de produire des valeurs "extrêmes" (très loin de la moyenne) ?
\end{exercicebox}

\begin{exercicebox}[Exercice 13 : Kurtosis de la Loi Normale]
Soit $Z \sim \mathcal{N}(0, 1)$.
\begin{enumerate}
    \item Que vaut le 4ème moment standardisé $E[Z^4]$ ? (Indice : Pour une $\mathcal{N}(0,1)$, $E[Z^4]=3$).
    \item Que vaut l'excès de kurtosis de $Z$ ?
\end{enumerate}
\end{exercicebox}

\begin{exercicebox}[Exercice 14 : Kurtosis de la Loi Uniforme]
D'après les exemples du cours, la loi $\text{Unif}(a, b)$ a un excès de kurtosis de -1.2.
Est-elle leptokurtique, mésokurtique ou platykurtique ?
\end{exercicebox}

% --- Synthèse et Propriétés ---

\begin{exercicebox}[Exercice 15 : Invariance des Moments Standardisés]
Soit $X$ une v.a. avec $\mu=10$, $\sigma=2$, $\text{Skew}(X) = 0.5$ et $\text{Excess Kurtosis}(X) = 1$.
Soit $Y = 3X + 5$.
\begin{enumerate}
    \item Calculez $E[Y]$ et $\text{Var}(Y)$.
    \item Que vaut $\text{Skew}(Y)$ ?
    \item Que vaut $\text{Excess Kurtosis}(Y)$ ?
\end{enumerate}
(Indice : Les moments standardisés sont invariants par transformation linéaire $aX+b$ avec $a>0$).
\end{exercicebox}

\begin{exercicebox}[Exercice 16 : Moments d'une Distribution Inconnue]
Une v.a. $Z$ a été standardisée ($E[Z]=0, \text{Var}(Z)=1$).
On sait que $E[Z^3] = 0.8$ et $E[Z^4] = 4.5$.
\begin{enumerate}
    \item Calculez le skewness de $Z$.
    \item Calculez l'excès de kurtosis de $Z$.
\end{enumerate}
\end{exercicebox}

\begin{exercicebox}[Exercice 17 : Identifier la Distribution]
Une distribution $X$ est analysée. On trouve :
$\text{Skew}(X) = 0$ et $\text{Excess Kurtosis}(X) = 0$.
Quelle distribution célèbre partage ces deux propriétés ?
\end{exercicebox}

\begin{exercicebox}[Exercice 18 : Identifier la Distribution (2)]
Une distribution $Y$ est analysée. On trouve :
$\text{Skew}(Y) = 2.0$ et $\text{Excess Kurtosis}(Y) = 6.0$.
Quelle distribution vue en cours (et dans les exemples) correspond à ces valeurs ?
\end{exercicebox}

\begin{exercicebox}[Exercice 19 : Moments non centrés vs centrés]
Soit $X$ une v.a. avec $E[X] = \mu$.
Exprimez le 2ème moment centré $E[(X-\mu)^2]$ en fonction des moments non centrés $E[X^2]$ et $E[X]$. (C'est la formule de calcul de la variance).
\end{exercicebox}

\begin{exercicebox}[Exercice 20 : Moments centrés vs non centrés]
Soit $X$ une v.a. avec $E[X] = \mu$.
Exprimez le 3ème moment centré $E[(X-\mu)^3]$ en fonction des moments non centrés ($E[X^3]$, $E[X^2]$, $E[X]$).
(Indice : Développez $(X-\mu)^3 = X^3 - 3X^2\mu + 3X\mu^2 - \mu^3$ et prenez l'espérance).
\end{exercicebox}

\subsection{Corrections des Exercices}

% --- Corrections : Définitions Fondamentales des Moments ---

\begin{correctionbox}[Correction Exercice 1 : Associer les Moments]
1.  $E[X]$ $\rightarrow$ \textbf{Espérance (Moyenne)}
2.  $\sqrt{E[(X-\mu)^2]}$ $\rightarrow$ \textbf{Écart-type}
3.  $E[(X-\mu)^2]$ $\rightarrow$ \textbf{Variance}
4.  $E[\left(\frac{X-\mu}{\sigma}\right)^3]$ $\rightarrow$ \textbf{Asymétrie (Skewness)}
5.  $E[\left(\frac{X-\mu}{\sigma}\right)^4] - 3$ $\rightarrow$ \textbf{Excès de Kurtosis}
\end{correctionbox}

\begin{correctionbox}[Correction Exercice 2 : Calcul des Moments Centrés]
$P(X=0)=0.2$, $P(X=2)=0.6$, $P(X=4)=0.2$.
1.  $E[X] = (0)(0.2) + (2)(0.6) + (4)(0.2) = 0 + 1.2 + 0.8 = 2.0$. $\mu=2$.
2.  Variance = $E[(X-2)^2]$
    $= (0-2)^2(0.2) + (2-2)^2(0.6) + (4-2)^2(0.2)$
    $= (-2)^2(0.2) + (0)^2(0.6) + (2)^2(0.2)$
    $= 4(0.2) + 0 + 4(0.2) = 0.8 + 0.8 = 1.6$.
3.  3ème moment centré = $E[(X-2)^3]$
    $= (0-2)^3(0.2) + (2-2)^3(0.6) + (4-2)^3(0.2)$
    $= (-8)(0.2) + (0)(0.6) + (8)(0.2)$
    $= -1.6 + 0 + 1.6 = 0$.
\end{correctionbox}

\begin{correctionbox}[Correction Exercice 3 : Calcul des Moments Standardisés]
De l'exercice 2, $\mu=2$, $\text{Var}(X) = 1.6$ et $E[(X-\mu)^3] = 0$.
1.  $\text{Skew}(X) = \frac{E[(X-\mu)^3]}{\sigma^3} = \frac{0}{(1.6)^{3/2}} = 0$.
2.  Le skewness est nul. Cela indique que la distribution est symétrique, ce que l'on peut vérifier (les probabilités $P(\mu-2k)$ et $P(\mu+2k)$ sont égales).
\end{correctionbox}

% --- Corrections : Asymétrie (Skewness) ---

\begin{correctionbox}[Correction Exercice 4 : Interprétation Visuelle du Skewness]
1.  \textbf{Positif (> 0)}. La loi exponentielle a une queue longue à droite.
2.  \textbf{Négatif (< 0)}. L'ordre $\text{Moyenne} < \text{Médiane} < \text{Mode}$ est caractéristique d'une queue à gauche.
3.  \textbf{Nul (= 0)}. La loi normale est parfaitement symétrique.
4.  \textbf{Positif (> 0)}. La grande majorité des gens a un revenu bas/moyen, et une petite minorité a un revenu très élevé, créant une queue longue à droite.
\end{correctionbox}

\begin{correctionbox}[Correction Exercice 5 : Skewness d'une Distribution Simple]
$P(X=0)=0.1$, $P(X=1)=0.8$, $P(X=10)=0.1$.
1.  $\mu = E[X] = (0)(0.1) + (1)(0.8) + (10)(0.1) = 0 + 0.8 + 1.0 = 1.8$.
2.  $E[X^2] = (0^2)(0.1) + (1^2)(0.8) + (10^2)(0.1) = 0 + 0.8 + 10 = 10.8$.
    $\sigma^2 = E[X^2] - \mu^2 = 10.8 - (1.8)^2 = 10.8 - 3.24 = 7.56$.
3.  $E[(X-\mu)^3] = (0-1.8)^3(0.1) + (1-1.8)^3(0.8) + (10-1.8)^3(0.1)$
    $= (-5.832)(0.1) + (-0.512)(0.8) + (551.368)(0.1)$
    $= -0.5832 - 0.4096 + 55.1368 = 54.144$.
4.  Puisque $E[(X-\mu)^3] = 54.144 > 0$, le skewness est \textbf{positif}. Cela est dû à la valeur extrême $X=10$ qui tire la distribution vers la droite.
\end{correctionbox}

\begin{correctionbox}[Correction Exercice 6 : Skewness d'une Variable de Bernoulli]
$X \sim \text{Bern}(p)$, $\mu=p$, $\sigma^2=p(1-p)$.
1.  $X$ ne vaut que 0 ou 1, donc $X^3 = X$. $E[X^3] = E[X] = p$.
2.  $E[(X-p)^3] = E[X^3 - 3X^2p + 3Xp^2 - p^3]$
    $= E[X^3] - 3pE[X^2] + 3p^2E[X] - p^3$
    (Car $X^2=X$ et $X^3=X$)
    $= E[X] - 3pE[X] + 3p^2E[X] - p^3$
    $= p - 3p(p) + 3p^2(p) - p^3 = p - 3p^2 + 3p^3 - p^3 = p - 3p^2 + 2p^3$
    $= p(1 - 3p + 2p^2) = p(1-p)(1-2p)$.
3.  $\text{Skew}(X) = \frac{E[(X-p)^3]}{\sigma^3} = \frac{p(1-p)(1-2p)}{[p(1-p)]^{3/2}} = \frac{1-2p}{\sqrt{p(1-p)}}$.
4.  La distribution est symétrique si $\text{Skew}(X) = 0$. Cela se produit si $1-2p = 0$, donc $p=0.5$.
\end{correctionbox}

% --- Corrections : Symétrie ---

\begin{correctionbox}[Correction Exercice 7 : Définition de la Symétrie]
$X \sim \text{Unif}(-5, 5)$.
1.  La moyenne est $\mu = (-5+5)/2 = 0$. La distribution est symétrique autour de $\mu=0$.
2.  On doit montrer $f(x) = f(2\mu - x) = f(-x)$.
    La PDF est $f(x) = 1/10$ si $x \in [-5, 5]$ et $0$ sinon.
    - Si $x \in [-5, 5]$, alors $-x \in [-5, 5]$. $f(x) = 1/10$ et $f(-x) = 1/10$. Ils sont égaux.
    - Si $x > 5$, $f(x) = 0$. Alors $-x < -5$, $f(-x) = 0$. Ils sont égaux.
    La condition est vérifiée pour tout $x$.
\end{correctionbox}

\begin{correctionbox}[Correction Exercice 8 : Symétrie et Moments Centrés]
Soit $Y = X-\mu$. Si $X$ est symétrique autour de $\mu$, $Y$ est symétrique autour de 0. Sa PDF $f_Y(y)$ est paire : $f_Y(y) = f_Y(-y)$.
On veut calculer $E[Y^k] = \int_{-\infty}^{\infty} y^k f_Y(y) dy$ pour $k$ impair.
La fonction $y^k$ est impaire (car $k$ est impair).
La fonction $f_Y(y)$ est paire.
Le produit d'une fonction impaire et d'une fonction paire est une fonction impaire.
L'intégrale d'une fonction impaire sur un intervalle symétrique $(-\infty, \infty)$ est 0.
Donc, $E[(X-\mu)^k] = E[Y^k] = 0$ pour $k=1, 3, 5, \dots$.
\end{correctionbox}

\begin{correctionbox}[Correction Exercice 9 : Symétrie de la Loi Normale]
1.  Oui, la PDF de la loi normale est parfaitement symétrique autour de sa moyenne $\mu$.
2.  Le Skewness est basé sur le 3ème moment centré ($k=3$, qui est impair). D'après l'exercice 8, tous les moments centrés impairs d'une distribution symétrique sont nuls.
    Donc $\text{Skew}(X) = 0$.
\end{correctionbox}

\begin{correctionbox}[Correction Exercice 10 : Symétrie et Skewness]
Non. $\text{Skew}(X)=0$ est une condition nécessaire mais non suffisante pour la symétrie. Une distribution peut avoir un skewness nul tout en n'étant pas symétrique, si les asymétries de gauche et de droite "s'annulent" dans le calcul du 3ème moment.
\end{correctionbox}

% --- Corrections : Aplatissement (Kurtosis) ---

\begin{correctionbox}[Correction Exercice 11 : Interprétation du Kurtosis]
1.  \textbf{Leptokurtique} $\rightarrow$ \textbf{C.} Excès de Kurtosis > 0 (Pointue, queues épaisses).
2.  \textbf{Mésokurtique} $\rightarrow$ \textbf{A.} Excès de Kurtosis = 0 (Référence normale).
3.  \textbf{Platykurtique} $\rightarrow$ \textbf{B.} Excès de Kurtosis < 0 (Plate, queues fines).
\end{correctionbox}

\begin{correctionbox}[Correction Exercice 12 : Kurtosis et queues]
Un excès de kurtosis plus élevé signifie des queues plus épaisses.
La \textbf{Distribution A} (kurtosis=5) est beaucoup plus susceptible de produire des valeurs extrêmes que la Distribution B (kurtosis=1).
\end{correctionbox}

\begin{correctionbox}[Correction Exercice 13 : Kurtosis de la Loi Normale]
$Z \sim \mathcal{N}(0, 1)$.
1.  $E[Z^4]$ est le 4ème moment standardisé. Par définition, c'est le kurtosis (non excessif). Pour la loi normale, $\text{Kurtosis}(Z) = 3$. Donc $E[Z^4] = 3$.
2.  Excès de Kurtosis = $\text{Kurtosis}(Z) - 3 = 3 - 3 = 0$.
\end{correctionbox}

\begin{correctionbox}[Correction Exercice 14 : Kurtosis de la Loi Uniforme]
L'excès de kurtosis est -1.2, ce qui est négatif.
La distribution uniforme est \textbf{platykurtique} (plus "plate" que la normale).
\end{correctionbox}

% --- Corrections : Synthèse et Propriétés ---

\begin{correctionbox}[Correction Exercice 15 : Invariance des Moments Standardisés]
$Y = 3X + 5$.
1.  $E[Y] = 3E[X] + 5 = 3(10) + 5 = 35$.
    $\text{Var}(Y) = 3^2 \text{Var}(X) = 9 (2^2) = 36$.
2.  Le skewness est invariant par transformation linéaire affine (si $a>0$). $\text{Skew}(Y) = \text{Skew}(X) = 0.5$.
3.  L'excès de kurtosis est invariant par transformation linéaire affine. $\text{Excess Kurtosis}(Y) = \text{Excess Kurtosis}(X) = 1$.
\end{correctionbox}

\begin{correctionbox}[Correction Exercice 16 : Moments d'une Distribution Inconnue]
$Z$ est déjà standardisée ($\mu=0, \sigma=1$).
1.  $\text{Skew}(Z) = E\left[ \left( \frac{Z - 0}{1} \right)^3 \right] = E[Z^3] = 0.8$.
2.  $\text{Excess Kurtosis}(Z) = E\left[ \left( \frac{Z - 0}{1} \right)^4 \right] - 3 = E[Z^4] - 3 = 4.5 - 3 = 1.5$.
\end{correctionbox}

\begin{correctionbox}[Correction Exercice 17 : Identifier la Distribution]
La \textbf{Loi Normale} $\mathcal{N}(\mu, \sigma^2)$ est la distribution de référence qui est symétrique ($\text{Skew}=0$) et mésokurtique ($\text{Excess Kurtosis}=0$).
\end{correctionbox}

\begin{correctionbox}[Correction Exercice 18 : Identifier la Distribution (2)]
La \textbf{Loi Exponentielle} $\text{Exp}(\lambda)$ (pour n'importe quel $\lambda$) a un skewness de 2.0 et un excès de kurtosis de 6.0.
\end{correctionbox}

\begin{correctionbox}[Correction Exercice 19 : Moments non centrés vs centrés]
C'est la dérivation de la formule de calcul de la variance.
$E[(X-\mu)^2] = E[X^2 - 2X\mu + \mu^2]$
$= E[X^2] - E[2X\mu] + E[\mu^2]$ (par linéarité)
$= E[X^2] - 2\mu E[X] + \mu^2$ (car $\mu$ est une constante)
$= E[X^2] - 2\mu(\mu) + \mu^2 = E[X^2] - 2\mu^2 + \mu^2$
$= E[X^2] - \mu^2 = E[X^2] - (E[X])^2$.
\end{correctionbox}

\begin{correctionbox}[Correction Exercice 20 : Moments centrés vs non centrés]
On développe et on utilise la linéarité de l'espérance, en se rappelant que $\mu=E[X]$ est une constante.
$E[(X-\mu)^3] = E[X^3 - 3X^2\mu + 3X\mu^2 - \mu^3]$
$= E[X^3] - E[3X^2\mu] + E[3X\mu^2] - E[\mu^3]$
$= E[X^3] - 3\mu E[X^2] + 3\mu^2 E[X] - \mu^3$
En remplaçant $E[X]$ par $\mu$ :
$= E[X^3] - 3\mu E[X^2] + 3\mu^2(\mu) - \mu^3$
$= E[X^3] - 3\mu E[X^2] + 3\mu^3 - \mu^3$
$= E[X^3] - 3E[X]E[X^2] + 2(E[X])^3$.
\end{correctionbox}

\subsection{Exercices Pratiques (Python)}

Dans ce chapitre, nous avons vu que le 3ème moment (skewness) et le 4ème moment (kurtosis) décrivent la forme d'une distribution. En finance, ces mesures sont cruciales pour évaluer le risque.

Un rendement avec un \textbf{skewness négatif} signifie que les pertes extrêmes sont plus probables que les gains extrêmes. Un \textbf{excès de kurtosis positif} (leptokurtique) signifie que les "queues" de la distribution sont épaisses, indiquant qu'il y a une probabilité plus élevée d'événements extrêmes (gains ou pertes majeurs) que ne le prédit une loi normale.

Nous allons calculer ces moments pour des actifs financiers réels.

\begin{codecell}
# Cellule d'installation et d'importation
pip install numpy pandas yfinance scipy
\end{codecell}

\begin{codecell}
import numpy as np
import pandas as pd
import yfinance as yf
from scipy import stats
\end{codecell}

\begin{exercicebox}[Exercice 1 : Calcul Manuel des Moments]
Calculons les 4 moments des rendements quotidiens de l'action Tesla (TSLA), connue pour sa volatilité.

\textbf{Votre tâche :}
\begin{enumerate}
    \item Télécharger 5 ans de données pour TSLA.
    \item Calculer les rendements logarithmiques quotidiens $X$.
    \item Calculer la moyenne $\mu = E[X]$ et l'écart-type $\sigma$.
    \item Standardiser les rendements : $Z = (X - \mu) / \sigma$.
    \item Calculer le Skewness (3ème moment standardisé) : $E[Z^3]$.
    \item Calculer l'Excès de Kurtosis (4ème moment standardisé - 3) : $E[Z^4] - 3$.
\end{enumerate}

\begin{codecell}
# 1. Telecharger les donnees
ticker = 'TSLA'
data = yf.download(ticker, period='5y')

# 2. Calculer les rendements log
log_returns = np.log(data['Close'] / data['Close'].shift(1)).dropna()

# 3. Calculer mu et sigma
# mu = ...
# sigma = ...

# 4. Standardiser les rendements
# z_scores = ...

# 5. Calculer le Skewness
# Indice : (z_scores**3).mean()
# skewness_manuel = ...

# 6. Calculer l'Exces de Kurtosis
# Indice : (z_scores**4).mean() - 3
# kurtosis_manuel = ...

# print(f"--- Calculs Manuels pour {ticker} ---")
# print(f"Skewness: {skewness_manuel:.4f}")
# print(f"Exces de Kurtosis: {kurtosis_manuel:.4f}")
\end{codecell}
\end{exercicebox}

\begin{exercicebox}[Exercice 2 : Vérification avec SciPy]
Vérifions nos calculs manuels de l'exercice 1 en utilisant les fonctions optimisées de la bibliothèque scipy.stats.

\textbf{Votre tâche :}
\begin{enumerate}
    \item Utiliser stats.skew(log\_returns) pour calculer le skewness.
    \item Utiliser stats.kurtosis(log\_returns) pour calculer l'excès de kurtosis. (Note : cette fonction calcule l'excès de kurtosis par défaut, en soustrayant 3).
    \item Afficher et comparer les résultats avec ceux de l'exercice 1.
\end{enumerate}

\begin{codecell}
from scipy import stats

# log_returns de l'exercice 1

# 1. Calculer le skewness avec scipy
# skew_scipy = ...

# 2. Calculer l'exces de kurtosis avec scipy
# kurtosis_scipy = ...

# print(f"--- Verification avec SciPy pour {ticker} ---")
# print(f"Skewness: {skew_scipy:.4f}")
# print(f"Exces de Kurtosis: {kurtosis_scipy:.4f}")
\end{codecell}
\end{exercicebox}

\begin{exercicebox}[Exercice 3 : Interprétation des Moments]
Cet exercice est purement théorique et ne nécessite pas de code. Répondez en vous basant sur les résultats (probablement) obtenus aux exercices 1 et 2 pour TSLA.

\begin{enumerate}
    \item Le skewness calculé est-il positif, négatif ou proche de zéro ? Que cela implique-t-il sur la probabilité des gains quotidiens extrêmes par rapport aux pertes quotidiennes extrêmes ?
    \item L'excès de kurtosis est-il positif, négatif ou proche de zéro ? La distribution est-elle leptokurtique, mésokurtique ou platykurtique ?
    \item Que signifie cette valeur de kurtosis pour un investisseur en termes de risque, comparé à une distribution normale ?
\end{enumerate}
\end{exercicebox}

\begin{exercicebox}[Exercice 4 : Comparaison de Distributions]
Comparons les moments de TSLA (actif volatil) à ceux d'un indice large comme le S\&P 500 (GSPC) (actif plus stable).

\textbf{Votre tâche :}
\begin{enumerate}
    \item Télécharger 5 ans de données pour GSPC.
    \item Calculer ses rendements logarithmiques (log\_returns\_sp500).
    \item Calculer le skewness et l'excès de kurtosis pour GSPC en utilisant scipy.stats
    \item Comparer les kurtosis de TSLA et GSPC. Lequel a les "queues les plus épaisses" (fatter tails) ?
\end{enumerate}

\begin{codecell}
# 1. Telecharger les donnees pour S&P 500
ticker_sp500 = 'GSPC'
# data_sp500 = ...

# 2. Calculer les rendements log
# log_returns_sp500 = ...
log_returns_sp500 = log_returns_sp500.dropna()

# 3. Calculer skew et kurtosis pour S&P 500
# skew_sp500 = ...
# kurtosis_sp500 = ...

# print(f"--- Comparaison des Moments (5 ans) ---")
# print(f"TSLA Skew: {skew_scipy:.4f} | Kurtosis: {kurtosis_scipy:.4f}")
# print(f"SP500 Skew: {skew_sp500:.4f} | Kurtosis: {kurtosis_sp500:.4f}")
\end{codecell}
\end{exercicebox}
\newpage

\section{Moments d'une distribution}

\subsection{Définitions fondamentales des moments}

Après avoir défini l'espérance ($\mu$) et la variance ($\sigma^2$), qui sont les moments d'ordre 1 et 2, nous pouvons généraliser cette idée pour capturer des informations plus subtiles sur la forme d'une distribution.

\begin{definitionbox}[Types de Moments]
Soit $X$ une variable aléatoire ayant une espérance $\mu$ et une variance $\sigma^2$. Pour tout entier positif $m$, on définit les moments suivants :
\begin{itemize}
    \item \textbf{$m$-ième moment (non centré)} : $E[X^m]$.
    \item \textbf{$m$-ième moment centré} : $E[(X - \mu)^m]$.
    \item \textbf{$m$-ième moment standardisé} : $E\left[\left(\frac{X - \mu}{\sigma}\right)^m\right]$.
\end{itemize}
Les moments centrés et standardisés permettent d'étudier les propriétés de la distribution indépendamment de sa position ($\mu$) et de son échelle ($\sigma$).
\end{definitionbox}

\subsection{Asymétrie (Skewness)}

Le premier moment nous donne la tendance centrale. Le deuxième moment (la variance) nous donne la dispersion. Le troisième moment, lui, va nous renseigner sur la \textit{symétrie} de la distribution.

\begin{definitionbox}[Asymétrie (Skewness)]
L'\textbf{asymétrie} (ou \textit{skewness}) d'une variable aléatoire $X$ de moyenne $\mu$ et d'écart-type $\sigma$ est définie comme le \textbf{troisième moment standardisé} :
$$ \text{Skew}(X) = E\left[ \left( \frac{X - \mu}{\sigma} \right)^3 \right]. $$
\end{definitionbox}

\begin{intuitionbox}[Comprendre la Formule du Skewness]
Pour une variable aléatoire $X$ de moyenne $\mu$ et d'écart-type $\sigma$, le \textbf{skewness} est défini comme :
\[
\text{Skew}(X) = \frac{E[(X - \mu)^3]}{\sigma^3}
\]

\medskip

\textbf{Logique du numérateur : le moment centré d'ordre 3}
\begin{itemize}
    \item Le terme $(X - \mu)^3$ est le \textbf{cube de l'écart à la moyenne}
    \item Contrairement à $(X - \mu)^2$ (toujours positif), le cube \textbf{conserve le signe} de l'écart
    \item Il pondère différemment les observations à gauche et à droite de la moyenne
\end{itemize}

\medskip

% --- MODIFIÉ : Tableau supprimé et fusionné dans la liste ---
\textbf{Interprétation intuitive}
\begin{itemize}
    \item \textbf{Skewness = 0 (Symétrique)} : La distribution est symétrique. Les écarts positifs et négatifs s'annulent. Typiquement : Moyenne = Médiane = Mode.
    \item \textbf{Skewness > 0 (Queue à droite)} : La distribution présente une queue longue à droite. Les grandes valeurs positives sont amplifiées par le cube. Les valeurs extrêmes tirent la moyenne vers la droite.
    \item \textbf{Skewness < 0 (Queue à gauche)} : La distribution présente une queue longue à gauche. Les écarts négatifs dominent. Les valeurs extrêmes tirent la moyenne vers la gauche.
\end{itemize}
% --- FIN MODIFICATION ---

\medskip

\textbf{Pourquoi $\sigma^3$ au dénominateur ?}
\begin{itemize}
    \item Le moment d'ordre 3 est homogène à des unités au cube
    \item On divise par $\sigma^3$ pour obtenir un coefficient \textbf{sans dimension}
    \item Permet la comparaison entre distributions de différentes échelles
\end{itemize}
\end{intuitionbox}

\begin{remarquebox}[Pourquoi Standardiser ?]
En standardisant d'abord ($\frac{X-\mu}{\sigma}$), la définition de $\text{Skew}(X)$ ne dépend ni de la position ($\mu$) ni de l'échelle ($\sigma$) de la distribution, ce qui est raisonnable puisque ces informations sont déjà fournies par la moyenne et l'écart-type. De plus, cette standardisation garantit que l'asymétrie est invariante par changement d'unité de mesure (par exemple, passer des pouces aux mètres n'affecte pas la valeur de l'asymétrie).
\end{remarquebox}

\subsection{Propriétés de symétrie}

Le skewness est une mesure numérique de l'asymétrie. Mais nous pouvons aussi définir la symétrie de manière formelle.

\begin{definitionbox}[Symétrie d'une Variable Aléatoire]
On dit qu'une variable aléatoire $X$ a une distribution \textbf{symétrique} autour de $\mu$ si la variable $X - \mu$ a la même distribution que $\mu - X$. On dit aussi que $X$ est symétrique ou que sa distribution est symétrique. Ces trois formulations ont le même sens.
\end{definitionbox}

\begin{theorembox}[Symétrie en Termes de Fonction de Densité]
Soit $X$ une variable aléatoire continue de fonction de densité de probabilité (PDF) $f$. Alors, $X$ est symétrique autour de $\mu$ si et seulement si :
$$ f(x) = f(2\mu - x) \quad \text{pour tout } x. $$
\end{theorembox}

\begin{proofbox}[Preuve du Théorème de Symétrie]
Soit $F$ la fonction de répartition (CDF) de $X$. Si la symétrie tient, alors :
$$ F(x) = P(X \le x) = P(X - \mu \le x - \mu) = P(\mu - X \le x - \mu) = P(X \ge 2\mu - x) = 1 - F(2\mu - x). $$

En prenant la dérivée des deux côtés par rapport à $x$, on obtient :
$$ f(x) = \frac{d}{dx}F(x) = \frac{d}{dx}[1 - F(2\mu - x)] = f(2\mu - x). $$

Cela démontre que la condition $f(x) = f(2\mu - x)$ est nécessaire et suffisante pour la symétrie.
\end{proofbox}

\subsection{Aplatissement (Kurtosis)}

Après l'asymétrie (ordre 3), le moment d'ordre 4 nous informe sur "l'épaisseur" des queues de la distribution, c'est-à-dire la probabilité d'obtenir des valeurs très éloignées de la moyenne.

\begin{definitionbox}[Kurtosis (Aplatissement)]
Pour une variable aléatoire $X$ de moyenne $\mu$ et d'écart-type $\sigma$, le \textbf{kurtosis} est défini comme le \textbf{quatrième moment standardisé} :
$$ \text{Kurtosis}(X) = E\left[ \left( \frac{X - \mu}{\sigma} \right)^4 \right]. $$

Dans la pratique, on utilise plus souvent le \textbf{kurtosis excessif} (ou excès de kurtosis), défini comme :
$$ \text{Excess Kurtosis}(X) = E\left[ \left( \frac{X - \mu}{\sigma} \right)^4 \right] - 3. $$
La soustraction de 3 fait en sorte que le kurtosis d'une loi normale soit égal à 0.
\end{definitionbox}

\begin{intuitionbox}[Comprendre la Kurtosis]
Pour une variable aléatoire $X$, le \textbf{kurtosis} est défini comme :
\[
\text{Kurt}(X) = \frac{E[(X - \mu)^4]}{\sigma^4}
\]
et l'\textbf{excess kurtosis} (kurtosis excédentaire) comme : $\text{Excess Kurtosis} = \text{Kurt}(X) - 3$.

\medskip

\textbf{Pourquoi le moment d'ordre 4 ?}
\begin{itemize}
    \item Comme la variance, on utilise une puissance paire (pas d'effet de signe)
    \item La puissance 4 \textbf{amplifie énormément les écarts extrêmes}
    \item Mesure le \textbf{poids des queues} et la \textbf{concentration autour de la moyenne}
\end{itemize}

\medskip

% --- MODIFIÉ : Tableau supprimé et fusionné dans la liste ---
\textbf{Interprétation intuitive (basée sur l'Excess Kurtosis)}
\begin{itemize}
    \item \textbf{Leptokurtique (Excess Kurtosis > 0)} : Kurtosis total > 3. Distribution pointue avec des queues épaisses. Les événements extrêmes sont plus probables que pour une loi normale.
    \item \textbf{Mésocurtique (Excess Kurtosis = 0)} : Kurtosis total = 3. C'est la référence (loi normale).
    \item \textbf{Platykurtique (Excess Kurtosis < 0)} : Kurtosis total < 3. Distribution aplatie avec des queues légères et un centre large. Les événements extrêmes sont moins probables.
\end{itemize}
% --- FIN MODIFICATION ---

\medskip

\textbf{Application en finance}
\begin{itemize}
    \item Les rendements financiers ont souvent un excès de kurtosis positif
    \item Indique une probabilité plus élevée d'événements extrêmes que la loi normale
    \item Justifie le "vol smile" dans les options
\end{itemize}

\medskip

\textbf{Pourquoi $\sigma^4$ au dénominateur ?}
\begin{itemize}
    \item Le moment d'ordre 4 est homogène à des unités$^4$
    \item On divise par $\sigma^4$ pour un coefficient \textbf{sans dimension}
\end{itemize}
\end{intuitionbox}

\subsection{Exemples de distributions}

Pour bien fixer les idées, comparons le skewness et le kurtosis de plusieurs distributions classiques. Notez que dans les graphiques suivants, le "Kurtosis" affiché est l'\textit{excess kurtosis} (centré à 0).

\begin{examplebox}[La Distribution Normale (Mésokurtique)]

\begin{center}
\begin{tikzpicture}
  \begin{axis}[
    width=0.8\textwidth,
    height=0.5\textwidth,
    xlabel={$x$},
    title={Densité Normale ($\mu=0$, $\sigma=1$)},
    grid=both,
    grid style={line width=.1pt, draw=gray!30},
    major grid style={line width=.2pt,draw=gray!50},
    domain=-4:4,
    samples=100,
    enlargelimits=false,
    axis lines=middle,
    xmin=-4, xmax=4,
    ymin=0, ymax=0.55
  ]
 
  % Courbe de densité
  \addplot [thick, color=blue, fill=blue!20, fill opacity=0.5] 
    {1/(sqrt(2*pi))*exp(-x^2/2)} \closedcycle;
 
  % Ligne de la moyenne
  \addplot [red, dashed, thick] coordinates {(0,0) (0,0.4)};
  \node[red, fill=white, font=\scriptsize, rounded corners, inner sep=2pt, opacity=0.8] at (axis cs:0.5,0.35) {Moyenne};
 
  % Boîte de texte avec moments seulement
  \node [draw=black, fill=white, rounded corners, font=\scriptsize, align=left, anchor=north east, opacity=0.8] 
    at (axis description cs:0.95,0.95) { % MODIFIÉ
    \textbf{Moments :}\\
    • Moyenne = 0.00\\
    • Variance = 1.00\\
    • Skewness = 0.00\\
    • Kurtosis = 0.00
    };
    
  \end{axis}
\end{tikzpicture}
\end{center}

La distribution normale est l'archétype de la courbe en cloche. Imaginez une cible : la majorité des flèches touchent le centre, et plus on s'éloigne du centre, moins il y a de chances d'être touché. C'est une distribution parfaitement symétrique, ce qui se traduit par un \textbf{skewness nul (0.00)}. Son pic est ni trop pointu, ni trop plat : c'est notre point de référence, on dit qu'elle est \textbf{mésokurtique}, d'où son kurtosis de \textbf{0.00}. C'est la base de nombreuses analyses statistiques car elle modélise naturellement beaucoup de phénomènes.

\end{examplebox}

\begin{examplebox}[La Distribution Exponentielle (Asymétrique à Droite)]

\begin{center}
\begin{tikzpicture}
  \begin{axis}[
    width=0.8\textwidth,
    height=0.5\textwidth,
    xlabel={$x$},
    title={Densité Exponentielle ($\lambda=1$)},
    grid=both,
    grid style={line width=.1pt, draw=gray!30},
    major grid style={line width=.2pt,draw=gray!50},
    domain=0:6,
    samples=100,
    enlargelimits=false,
    axis lines=middle,
    xmin=0, xmax=6,
    ymin=0, ymax=1.1
  ]
 
  % Courbe de densité exponentielle
  \addplot [thick, color=blue, fill=blue!20, fill opacity=0.5] 
    {exp(-x)} \closedcycle;
 
  % Ligne de la moyenne
  \addplot [red, dashed, thick] coordinates {(1,0) (1,0.37)};
  \node[red, fill=white, font=\scriptsize, rounded corners, inner sep=2pt, opacity=0.8] at (axis cs:1.5,0.3) {Moyenne};
 
  % Boîte de texte avec moments seulement
  \node [draw=black, fill=white, rounded corners, font=\scriptsize, align=left, anchor=north east, opacity=0.8] 
    at (axis description cs:0.95,0.95) { % MODIFIÉ
    \textbf{Moments :}\\
    • Moyenne = 1.00\\
    • Variance = 1.00\\
    • Skewness = 2.00\\
    • Kurtosis = 6.00
    };
    
  \end{axis}
\end{tikzpicture}
\end{center}

Imaginez le temps d'attente avant un événement rare, comme un appel téléphonique. La plupart du temps, l'appel arrive vite, mais il peut parfois y avoir de longues attentes. C'est exactement ce que modélise la distribution exponentielle : un pic à gauche et une longue queue à droite. Cela se traduit par un \textbf{skewness positif élevé (2.00)}, indiquant une asymétrie marquée. Elle est aussi \textbf{leptokurtique} (\textbf{kurtosis = 6.00}) : son pic est pointu, et la longue queue droite signifie qu'il y a une probabilité non négligeable de valeurs extrêmes.

\end{examplebox}

\begin{examplebox}[La Distribution Uniforme (Platykurtique)]

\begin{center}
\begin{tikzpicture}
  \begin{axis}[
    width=0.8\textwidth,
    height=0.5\textwidth,
    xlabel={$x$},
    title={Densité Uniforme ($a=0$, $b=2$)},
    grid=both,
    grid style={line width=.1pt, draw=gray!30},
    major grid style={line width=.2pt,draw=gray!50},
    domain=-0.5:2.5,
    samples=100,
    enlargelimits=false,
    axis lines=middle,
    ymin=0,
    ymax=.75,
    xmin=-1, xmax=4
  ]
 
  % Courbe de densité uniforme
  \addplot [thick, color=blue, fill=blue!20, fill opacity=0.5, const plot] 
    coordinates {(-0.5,0) (0,0) (0,0.5) (2,0.5) (2,0) (2.5,0)};
 
  % Ligne de la moyenne
  \addplot [red, dashed, thick] coordinates {(1,0) (1,0.5)};
  \node[red, fill=white, font=\scriptsize, rounded corners, inner sep=2pt, opacity=0.8] at (axis cs:1.3,0.4) {Moyenne};
 
  % Boîte de texte avec moments seulement
  \node [draw=black, fill=white, rounded corners, font=\scriptsize, align=left, anchor=north east, opacity=0.8] 
    at (axis description cs:0.95,0.95) { % MODIFIÉ
    \textbf{Moments :}\\
    • Moyenne = 1.00\\
    • Variance = 0.33\\
    • Skewness = 0.00\\
    • Kurtosis = -1.20
    };
    
  \end{axis}
\end{tikzpicture}
\end{center}

La distribution uniforme, c'est le "tirage au sort parfait" : chaque valeur sur un intervalle a la même chance d'être tirée. Visuellement, c'est un rectangle, donc aucune valeur n'est privilégiée. Elle est symétrique (\textbf{skewness = 0.00}), mais contrairement à la normale, elle est "plate", sans pic central. Cela se traduit par un \textbf{kurtosis négatif (-1.20)}, ce qui signifie qu'elle est \textbf{platykurtique}. Elle est donc très différente des distributions avec un pic central comme la normale.

\end{examplebox}

\begin{examplebox}[La Distribution Log-Normale (Fortement Leptokurtique)]

\begin{center}
\begin{tikzpicture}
  \begin{axis}[
    width=0.8\textwidth,
    height=0.5\textwidth,
    xlabel={$x$},
    title={Densité Log-Normale ($\sigma=0.7$)},
    grid=both,
    grid style={line width=.1pt, draw=gray!30},
    major grid style={line width=.2pt,draw=gray!50},
    domain=0:6,
    samples=100,
    enlargelimits=false,
    axis lines=middle,
    xmin=0, xmax=6,
    ymin=0, ymax=1
  ]
 
  % Courbe de densité log-normale
  \addplot [thick, color=blue, fill=blue!20, fill opacity=0.5] 
    {1/(x*0.7*sqrt(2*pi))*exp(-(ln(x))^2/(2*0.7^2))};
 
  % Ligne de la moyenne
  \addplot [red, dashed, thick] coordinates {(1.28,0) (1.28,0.4)};
  \node[red, fill=white, font=\scriptsize, rounded corners, inner sep=2pt, opacity=0.8] at (axis cs:1.8,0.5) {Moyenne};
 
  % Boîte de texte avec moments seulement
  \node [draw=black, fill=white, rounded corners, font=\scriptsize, align=left, anchor=north east, opacity=0.8] 
    at (axis description cs:0.95,0.95) { % MODIFIÉ
    \textbf{Moments :}\\
    • Moyenne = 1.28\\
    • Variance = 1.03\\
    • Skewness = 2.89\\
    • Kurtosis = 20.78
    };
    
  \end{axis}
\end{tikzpicture}
\end{center}

La log-normale est une distribution très asymétrique. Imaginez la richesse d'une population : la majorité est modeste, mais il existe une petite proportion de très riches, ce qui "étire" la droite de la courbe. Cela donne un \textbf{skewness très élevé (2.89)}. Elle est extrêmement \textbf{leptokurtique} (\textbf{kurtosis = 20.78}) : un pic très aigu et une queue droite très lourde. Cela signifie qu'il y a un risque élevé de valeurs extrêmement grandes, ce qui la rend très utile pour modéliser des phénomènes avec de rares événements extrêmes.

\end{examplebox}

Nous avons défini les moments d'une \textit{distribution} (moments de population), tels que $\mu = E[X]$ ou $\sigma^2 = E[(X-\mu)^2]$. Ce sont des valeurs théoriques, la "vérité" sous-jacente.

En pratique, nous ne connaissons presque jamais cette "vérité". Nous ne disposons que de données. Notre but est d'utiliser ces données pour \textit{estimer} les moments de la population.

\subsection{Moments d'échantillon (Sample Moments)}

\begin{definitionbox}[Moments d'Échantillon]
Soit $X_1, X_2, \dots, X_n$ un échantillon de $n$ observations.
\begin{itemize}
    \item La \textbf{moyenne d'échantillon} (notre "meilleure estimation" de $\mu$) est :
    $$ \bar{X} = \frac{1}{n} \sum_{i=1}^n X_i $$
    \item La \textbf{variance d'échantillon (non biaisée)} (notre "meilleure estimation" de $\sigma^2$) est :
    $$ s^2 = \frac{1}{n-1} \sum_{i=1}^n (X_i - \bar{X})^2 $$
\end{itemize}
De même, on peut calculer un \textit{skewness d'échantillon} et un \textit{kurtosis d'échantillon} en utilisant $\bar{X}$ et $s$, qui seront nos estimations du vrai skewness et du vrai kurtosis de la population.
\end{definitionbox}

\begin{examplebox}[Application : Contrôle Qualité ]
Imaginez une usine qui produit des sacs de sucre de 1kg.
\begin{itemize}
    \item \textbf{Population :} L'infinité de tous les sacs de sucre que la machine produira.
    \item \textbf{Moment de population (inconnu) :} Le poids moyen \textit{réel} $\mu$ que la machine verse, et la variance \textit{réelle} $\sigma^2$ (sa constance).
    \item \textbf{Problème :} Nous ne pouvons pas peser tous les sacs !
    \item \textbf{Solution :} Nous prélevons un \textbf{échantillon} de $n=10$ sacs.
    
    Nous les pesons : $\{ 1002g, 998g, 1001g, 995g, 1003g, 1000g, 997g, 1005g, 999g, 1000g \}$.
    
    \item \textbf{Calcul des moments d'échantillon :}
    \begin{itemize}
        \item $\bar{X} = (1002 + 998 + \dots + 1000) / 10 = 1000g$.
        \item $s^2 = \frac{1}{10-1} \left( (1002-1000)^2 + (998-1000)^2 + \dots \right) = 7.33 g^2$.
    \end{itemize}
    \item \textbf{Conclusion :} Notre meilleure estimation est que la machine est bien réglée sur $\mu = 1000g$. L'écart-type de notre échantillon est $s = \sqrt{7.33} \approx 2.7g$. Nous pouvons utiliser cela pour affirmer, par exemple, que 95\% des sacs se situent probablement entre $1000 \pm 2s$ (si la distribution est normale).
\end{itemize}
\end{examplebox}

\begin{remarquebox}[L'Intuition du "$n-1$"]
Pourquoi diviser par $n-1$ pour la variance ? C'est la \textbf{correction de Bessel}.

Imaginez un échantillon de 1 seule personne ($n=1$). Sa taille est 170cm.
\begin{itemize}
    \item Quelle est la moyenne de l'échantillon ? $\bar{X} = 170$ cm.
    \item Quelle est la variance de l'échantillon ? $\sum (X_i - \bar{X})^2 = (170 - 170)^2 = 0$.
    \item Si on divisait par $n=1$, on estimerait que la variance de la population est 0. C'est absurde ! Cela voudrait dire que tout le monde mesure 170cm.
\end{itemize}
En divisant par $n-1$ (donc $1-1=0$), la formule devient $0/0$ (indéfinie), ce qui nous dit à juste titre : "Je ne peux pas estimer la dispersion avec une seule personne."

\textbf{Intuition plus générale :} Nous "perdons un degré de liberté". Pour calculer la variance, nous avons besoin de connaître la moyenne. Mais nous ne connaissons pas la vraie moyenne $\mu$. Nous devons donc utiliser $\bar{X}$, une \textit{estimation}. Le fait d'utiliser une estimation calculée \textit{à partir de ce même échantillon} introduit un léger biais (nos données sont, par définition, centrées sur $\bar{X}$). Diviser par $n-1$ au lieu de $n$ "gonfle" légèrement le résultat pour compenser ce biais.
\end{remarquebox}

\subsection{Fonctions génératrices des moments (MGF)}

\begin{definitionbox}[Fonction Génératrice des Moments (MGF)]
La \textbf{fonction génératrice des moments} (MGF) d'une variable aléatoire $X$, notée $M_X(t)$, est définie comme :
$$ M_X(t) = E[e^{tX}] $$
\end{definitionbox}

\begin{intuitionbox}[L'ADN, le Code-Barres, ou le Fichier .zip]
Ce concept est abstrait, alors utilisons des analogies :

\textbf{Analogie 1 : L'ADN ou l'Empreinte Digitale}
\begin{itemize}
    \item La MGF est l'**empreinte digitale unique** d'une distribution.
    \item Elle "compresse" \textit{toutes} les informations sur votre distribution (moyenne, variance, skewness, kurtosis, etc.) en une seule, unique fonction.
    \item Si deux distributions ont la même MGF, elles sont identiques. C'est la \textbf{propriété d'unicité}.
\end{itemize}

\textbf{Analogie 2 : Le Code-Barres}
\begin{itemize}
    \item Pensez à une distribution (ex: Loi Normale) comme à un produit au supermarché.
    \item La MGF, $M_X(t)$, est son **code-barres unique**.
    \item Le processus de "génération de moments" (que nous verrons ci-dessous) est le \textbf{scanner}.
    \item En scannant le code-barres ($M_X(t)$), vous pouvez obtenir n'importe quelle information :
        \item Scan 1 ($M_X'(0)$) $\to$ vous donne le prix ($E[X]$).
        \item Scan 2 ($M_X''(0)$) $\to$ vous donne le poids ($E[X^2]$).
        \item Scan 3 ($M_X'''(0)$) $\to$ vous donne le pays d'origine ($E[X^3]$).
\end{itemize}

\textbf{Pourquoi $e^{tX}$ ?}
La "magie" vient du développement en série de Taylor de $e^x$:
$$ e^{tX} = 1 + (tX) + \frac{(tX)^2}{2!} + \frac{(tX)^3}{3!} + \dots $$
Quand on prend l'espérance, $E[\cdot]$, les puissances de $X$ (c'est-à-dire $X, X^2, X^3\dots$) apparaissent. Ce sont les moments ! La MGF "stocke" tous ces moments en les organisant comme coefficients d'un polynôme infini en $t$.
\end{intuitionbox}

\subsection{Génération des moments via les MGF}

\begin{theorembox}[Moments par Dérivation]
Si la MGF $M_X(t)$ existe, alors le $m$-ième moment non centré $E[X^m]$ est la $m$-ième dérivée de $M_X(t)$, évaluée en $t=0$ :
$$ E[X^m] = \frac{d^m}{dt^m} M_X(t) \bigg|_{t=0} = M_X^{(m)}(0) $$
\end{theorembox}

\begin{examplebox}[Application : La Loi de Poisson]
Une loi de Poisson modélise le nombre d'événements (ex: appels à un centre d'appels) par heure. Soit $X \sim \text{Poisson}(\lambda)$, où $\lambda$ est le nombre moyen d'appels.

La MGF (l'ADN) d'une loi de Poisson est (on l'admet) :
$$ M_X(t) = e^{\lambda(e^t - 1)} $$

Utilisons notre "scanner" (les dérivées) pour trouver les moments.

\textbf{1. Trouver la Moyenne $E[X]$ :}
On dérive une fois (règle de la chaîne) :
$$ M_X'(t) = \frac{d}{dt} \left( e^{\lambda(e^t - 1)} \right) = \underbrace{e^{\lambda(e^t - 1)}}_{\text{répète}} \cdot \underbrace{(\lambda e^t)}_{\text{dérivée interne}} $$
Maintenant, on évalue en $t=0$ :
$$ E[X] = M_X'(0) = e^{\lambda(e^0 - 1)} \cdot (\lambda e^0) = e^{\lambda(1 - 1)} \cdot (\lambda \cdot 1) = e^0 \cdot \lambda = 1 \cdot \lambda = \lambda $$
\textbf{Résultat :} La moyenne est $\lambda$, ce qui est la définition même du paramètre de la loi de Poisson. Parfait.

\textbf{2. Trouver $E[X^2]$ (pour la variance) :}
On dérive une seconde fois (règle du produit sur $M_X'(t) = (\lambda e^t) \cdot (e^{\lambda(e^t - 1)})$) :
$$ M_X''(t) = \underbrace{(\lambda e^t)}_{\text{dérivée de u}} \cdot \underbrace{(e^{\lambda(e^t - 1)})}_{\text{v}} + \underbrace{(\lambda e^t)}_{\text{u}} \cdot \underbrace{(e^{\lambda(e^t - 1)} \cdot \lambda e^t)}_{\text{dérivée de v}} $$
Maintenant, on évalue en $t=0$ (tous les $e^0$ deviennent 1) :
$$ E[X^2] = M_X''(0) = (\lambda \cdot 1) \cdot (e^{\lambda(1-1)}) + (\lambda \cdot 1) \cdot (e^{\lambda(1-1)} \cdot \lambda \cdot 1) $$
$$ E[X^2] = (\lambda) \cdot (e^0) + (\lambda) \cdot (e^0 \cdot \lambda) = \lambda \cdot 1 + \lambda \cdot (1 \cdot \lambda) = \lambda + \lambda^2 $$

\textbf{3. Trouver la Variance $\text{Var}(X)$ :}
$\text{Var}(X) = E[X^2] - (E[X])^2 = (\lambda + \lambda^2) - (\lambda)^2 = \lambda$
\textbf{Résultat :} Nous avons prouvé par les MGF que pour une loi de Poisson, $\text{Moyenne} = \text{Variance} = \lambda$. C'est une propriété fondamentale de cette loi.
\end{examplebox}

\subsection{Sommes de variables aléatoires indépendantes via les MGF}

C'est la super-puissance des MGF.

\begin{theorembox}[MGF d'une Somme]
Soient $X$ et $Y$ deux variables aléatoires \textbf{indépendantes}. Soit $S = X + Y$. Alors la MGF de $S$ est le produit des MGF individuelles :
$$ M_S(t) = M_{X+Y}(t) = M_X(t) \cdot M_Y(t) $$
\end{theorembox}

\begin{intuitionbox}[La Magie de l'Exponentielle]
Pourquoi est-ce vrai ? $M_{X+Y}(t) = E[e^{t(X+Y)}] = E[e^{tX} \cdot e^{tY}]$.
Parce que $X$ et $Y$ sont indépendantes, $E[f(X)g(Y)] = E[f(X)]E[g(Y)]$.
Donc, $E[e^{tX} \cdot e^{tY}] = E[e^{tX}] \cdot E[e^{tY}] = M_X(t) \cdot M_Y(t)$.

Les MGF transforment une opération analytiquement horrible (la "convolution" de densités) en une simple multiplication algébrique.
\end{intuitionbox}

\begin{examplebox}[Application : Portefeuille d'Actifs ou Tailles Humaines]
C'est l'un des théorèmes les plus importants des statistiques.
\textbf{Problème :} Soit $X$ la taille d'un homme, $X \sim N(\mu_X, \sigma_X^2)$. Soit $Y$ la taille d'une femme, $Y \sim N(\mu_Y, \sigma_Y^2)$. Si on les choisit au hasard, quelle est la loi de la somme de leurs tailles $S = X+Y$ ?

\begin{enumerate}
    \item \textbf{ADN de $X$} : La MGF d'une loi Normale $N(\mu, \sigma^2)$ est $M(t) = \exp(\mu t + \frac{1}{2}\sigma^2 t^2)$.
    \item \textbf{ADN de $X$ et $Y$} :
    $M_X(t) = \exp(\mu_X t + \frac{1}{2}\sigma_X^2 t^2)$
    $M_Y(t) = \exp(\mu_Y t + \frac{1}{2}\sigma_Y^2 t^2)$
    
    \item \textbf{ADN de $S = X+Y$} (on multiplie) :
    $M_S(t) = M_X(t) \cdot M_Y(t) = \exp(\mu_X t + \frac{1}{2}\sigma_X^2 t^2) \cdot \exp(\mu_Y t + \frac{1}{2}\sigma_Y^2 t^2)$
    
    \item \textbf{Simplification} (en additionnant les exposants) :
    $M_S(t) = \exp\left( (\mu_X t + \mu_Y t) + (\frac{1}{2}\sigma_X^2 t^2 + \frac{1}{2}\sigma_Y^2 t^2) \right)$
    $M_S(t) = \exp\left( (\mu_X + \mu_Y)t + \frac{1}{2}(\sigma_X^2 + \sigma_Y^2)t^2 \right)$
    
    \item \textbf{Conclusion (par Unicité)} :
    Regardez cet ADN ! C'est l'ADN d'une loi Normale !
    Le nouveau $\mu$ est $(\mu_X + \mu_Y)$.
    La nouvelle $\sigma^2$ est $(\sigma_X^2 + \sigma_Y^2)$.
\end{enumerate}

\textbf{Résultat :} Nous avons prouvé que \textbf{la somme de deux Normales indépendantes est une nouvelle Normale}.
Si $X \sim N(175cm, 7^2)$ et $Y \sim N(165cm, 6^2)$, alors $S \sim N(340cm, 7^2 + 6^2 = 85)$.
Notez que les écarts-types \textit{ne s'additionnent pas} ($\sqrt{85} \approx 9.2 \ne 7+6$). Ce sont les variances qui s'additionnent.
\end{examplebox}
\newpage

\section{Le Théorème Central Limite (TCL)}

\subsection{Introduction : L'omniprésence de la loi normale}

Dans la section précédente, la Loi des Grands Nombres (LLN) nous a donné une garantie fondamentale : la moyenne d'échantillon $\bar{X}_n$ converge vers la vraie moyenne $\mu$ lorsque $n$ devient grand.
$$ \bar{X}_n \xrightarrow{p.s.} \mu $$
La LLN nous dit \textbf{où} la moyenne d'échantillon converge (vers la constante $\mu$), mais elle ne nous dit rien sur la \textit{forme} de la distribution de $\bar{X}_n$ autour de $\mu$ pour un $n$ grand, mais fini.

Le \textbf{Théorème Central Limite (TCL)} comble cette lacune. Il décrit la \textit{manière} dont $\bar{X}_n$ converge, en nous donnant la forme de sa distribution. C'est sans doute le théorème le plus important des statistiques.

\begin{intuitionbox}[L'Idée Fondamentale]
Intuitivement, ce résultat affirme qu'une \textbf{somme} d'un grand nombre de variables aléatoires indépendantes et identiquement distribuées (i.i.d.) tend, le plus souvent, à suivre une \textbf{loi normale} (aussi appelée loi de Laplace-Gauss ou "courbe en cloche").

Ce théorème et ses généralisations offrent une explication à l'omniprésence de la loi normale dans la nature. De nombreux phénomènes (la taille d'un individu, l'erreur de mesure d'un instrument, le bruit de fond d'un signal) sont le résultat de l'addition d'un très grand nombre de petites perturbations aléatoires. Le TCL nous dit que le résultat de cette somme sera, inévitablement, distribué selon une loi normale.
\end{intuitionbox}

\subsection{L'illustration : la somme des "Pile ou Face"}

Prenons l'exemple le plus simple pour illustrer ce phénomène : le jeu de "pile ou face".



\begin{examplebox}[Distribution de la Somme de $n$ Lancers]
Soit $X_i$ le résultat du $i$-ème lancer, avec $X_i = 1$ pour "Face" (probabilité 0,5) et $X_i = 0$ pour "Pile" (probabilité 0,5). La distribution d'origine (pour $n=1$) n'est pas du tout une courbe en cloche : c'est une distribution discrète avec deux bâtons de même hauteur.

Considérons la \textbf{somme} $S_n = X_1 + X_2 + \dots + X_n$, qui représente le nombre total de "Face" obtenus en $n$ lancers.

\begin{itemize}
    \item \textbf{Pour $n=1$ :} La distribution de $S_1$ est :
    \begin{itemize}
        \item Valeurs de la somme : \{0, 1\}
        \item Fréquences : \{0.5, 0.5\}
    \end{itemize}
    
    \item \textbf{Pour $n=2$ :} Les sommes possibles sont \{0, 1, 2\}. La distribution de $S_2$ est :
    \begin{itemize}
        \item Valeurs de la somme : \{0, 1, 2\}
        \item Fréquences : \{0.25, 0.5, 0.25\} (elle forme un triangle).
    \end{itemize}
    
    \item \textbf{Pour $n=3$ :} Les sommes possibles sont \{0, 1, 2, 3\}. La distribution de $S_3$ est :
    \begin{itemize}
        \item Valeurs de la somme : \{0, 1, 2, 3\}
        \item Fréquences : \{0.125, 0.375, 0.375, 0.125\}
    \end{itemize}
\end{itemize}

\begin{center}
\begin{tikzpicture}
  \begin{axis}[
    width=0.8\textwidth,
    height=0.5\textwidth,
    xlabel={Valeurs de la somme},
    title={Fonction de fréquence pour des tirages à pile ou face},
    grid=both,
    grid style={line width=.1pt, draw=gray!30},
    major grid style={line width=.2pt,draw=gray!50},
    domain=0:4,
    samples=100,
    enlargelimits=false,
    axis lines=middle,
    xmin=0, xmax=4,
    ymin=0, ymax=0.6
  ]
  % n=1
  \addplot [thick, color=blue, fill=blue!20, fill opacity=0.5] coordinates {(0,0.5) (1,0.5)} \closedcycle;
  % n=2
  \addplot [thick, color=green, fill=green!20, fill opacity=0.5] coordinates {(0,0.25) (1,0.5) (2,0.25)} \closedcycle;
  % n=3
  \addplot [thick, color=red, fill=red!20, fill opacity=0.5] coordinates {(0,0.125) (1,0.375) (2,0.375) (3,0.125)} \closedcycle;
  % n=4 (suggéré)
  \addplot [thick, color=black, fill=black!20, fill opacity=0.5] coordinates {(0,0.0625) (1,0.25) (2,0.375) (3,0.25) (4,0.0625)} \closedcycle;
  
  \legend{$n=1$,$n=2$,$n=3$,$n=4$}
  \end{axis}
\end{tikzpicture}
\end{center}

Graphiquement, on constate que plus le nombre de tirages $n$ augmente (par exemple, jusqu'à $n=12$), plus la courbe de fréquence (qui reste discrète) se rapproche d'une courbe en cloche symétrique, caractéristique de la loi normale.
\end{examplebox}

\subsection{Distribution de la population vs. Distribution d'échantillonnage}

Le point le plus remarquable du TCL est qu'il fonctionne \textit{quelle que soit} la distribution de départ.

\begin{intuitionbox}[Population vs. Échantillonnage]
Imaginez deux univers de distributions :

\begin{itemize}
    \item \textbf{1. La Distribution de la Population ($X_i$) :} C'est la loi de nos variables $X_i$ individuelles. Elle peut avoir \textbf{n'importe quelle forme} (par exemple, une distribution bimodale, asymétrique, ou uniforme). Cette distribution a une "vraie" moyenne $\mu$ et un "vrai" écart-type $\sigma$.
    
    \item \textbf{2. La Distribution d'Échantillonnage ($\bar{X}_n$) :} C'est la distribution de la \textit{moyenne} $\bar{X}_n = (X_1 + \dots + X_n)/n$, calculée sur des échantillons de taille $n$. C'est la distribution de "toutes les moyennes d'échantillon possibles".
\end{itemize}

Le TCL énonce la relation magique entre les deux :

\textbf{Quelle que soit la forme de la distribution de la population, plus la taille de l'échantillon $n$ croît, plus la distribution d'échantillonnage de la moyenne $\bar{X}_n$ est proche d'une loi normale (gaussienne).}

De plus, les paramètres de cette loi normale sont :
\begin{itemize}
    \item \textbf{Moyenne :} La distribution de $\bar{X}_n$ est centrée sur la même moyenne $\mu$ que la population.
    \item \textbf{Écart-type :} La distribution de $\bar{X}_n$ est beaucoup plus resserrée. Son écart-type (appelé "erreur standard") est $\sigma_{\bar{X}} = \frac{\sigma}{\sqrt{n}}$.
\end{itemize}
Cette dispersion $\sigma/\sqrt{n}$ qui tend vers 0 est la manifestation de la Loi des Grands Nombres. Le TCL précise que la \textit{forme} de cette convergence est gaussienne.
\end{intuitionbox}

\subsection{Énoncé formel du Théorème Central Limite}

Pour énoncer le théorème formellement, nous devons d'abord définir les propriétés de la somme $S_n$ et de la moyenne $\bar{X}_n$.

Soit $X_1, \dots, X_n$ des variables aléatoires i.i.d. avec $E[X_i] = \mu$ et $\text{Var}(X_i) = \sigma^2$.

\begin{itemize}
    \item \textbf{La Somme $S_n = \sum X_i$} :
    \begin{itemize}
        \item Espérance : $E[S_n] = E[\sum X_i] = \sum E[X_i] = n\mu$
        \item Variance : $\text{Var}(S_n) = \text{Var}(\sum X_i) = \sum \text{Var}(X_i) = n\sigma^2$
        \item Écart-type : $\sigma_{S_n} = \sqrt{n\sigma^2} = \sigma\sqrt{n}$
    \end{itemize}
    
    \item \textbf{La Moyenne $\bar{X}_n = S_n / n$} :
    \begin{itemize}
        \item Espérance : $E[\bar{X}_n] = E[S_n / n] = \frac{1}{n} E[S_n] = \frac{1}{n} (n\mu) = \mu$
        \item Variance : $\text{Var}(\bar{X}_n) = \text{Var}(S_n / n) = \frac{1}{n^2} \text{Var}(S_n) = \frac{1}{n^2} (n\sigma^2) = \frac{\sigma^2}{n}$
        \item Écart-type : $\sigma_{\bar{X}_n} = \sqrt{\sigma^2 / n} = \frac{\sigma}{\sqrt{n}}$
    \end{itemize}
\end{itemize}

Nous voyons que la distribution de $S_n$ s'étale (variance $\to \infty$) tandis que celle de $\bar{X}_n$ se contracte (variance $\to 0$). Pour étudier la \textit{forme} de la convergence, nous créons une variable "stable" en la centrant (soustrayant la moyenne) et en la réduisant (divisant par l'écart-type). C'est la variable $Z_n$.

\begin{theorembox}[Théorème Central Limite (Lindeberg-Lévy)]
Soit $X_1, X_2, \dots, X_n$ une suite de variables aléatoires \textbf{i.i.d.} (indépendantes et identiquement distribuées) suivant la même loi $D$.
Supposons que l'\textbf{espérance $\mu$} et l'\textbf{écart-type $\sigma$} de cette loi $D$ existent, sont finis, et $\sigma \neq 0$.

Considérons la variable aléatoire standardisée $Z_n$ :
$$ Z_n = \frac{S_n - E[S_n]}{\sigma_{S_n}} = \frac{S_n - n\mu}{\sigma\sqrt{n}} $$
Cette variable est équivalente à la moyenne standardisée :
$$ Z_n = \frac{\bar{X}_n - E[\bar{X}_n]}{\sigma_{\bar{X}_n}} = \frac{\bar{X}_n - \mu}{\sigma / \sqrt{n}} $$
(Pour tout $n$, $Z_n$ est une variable centrée-réduite : $E[Z_n] = 0$ et $\text{Var}(Z_n) = 1$).

Alors, la suite de variables aléatoires $Z_1, Z_2, \dots, Z_n, \dots$ \textbf{converge en loi} vers une variable aléatoire $Z$ qui suit la \textbf{loi normale centrée réduite $N(0, 1)$}, lorsque $n$ tend vers l'infini.

Cela signifie que si $\Phi$ est la fonction de répartition de la loi $N(0, 1)$, alors pour tout réel $z$ :
$$ \lim_{n \to \infty} P(Z_n \le z) = \lim_{n \to \infty} P\left( \frac{\bar{X}_n - \mu}{\sigma/\sqrt{n}} \le z \right) = \Phi(z) $$
\end{theorembox}

\subsection{Applications Pratiques du TCL}

Le TCL n'est pas seulement une curiosité mathématique ; c'est le fondement de l'inférence statistique. Voici comment l'appliquer concrètement pour résoudre des problèmes.

\begin{examplebox}[La taille des individus]
\textbf{Contexte :} La taille des individus dans une population suit une courbe en cloche. Pourquoi ? Car elle est la \textbf{somme} de milliers de petites influences (gènes, nutrition, etc.). Le TCL s'applique.

\textbf{Données :} Supposons que dans une population, la taille $X$ des individus ait une espérance $\mu = 175$ cm et un écart-type $\sigma = 8$ cm. (Note : la loi de $X$ n'est pas forcément normale, même si en pratique elle l'est).

\textbf{Problème :} On prélève un échantillon aléatoire de $n=64$ individus. Quelle est la probabilité que la \textbf{moyenne de cet échantillon} ($\bar{X}_{64}$) soit supérieure à 177 cm ?

\textbf{Solution :}
\begin{enumerate}
    \item \textbf{Identifier les paramètres :}
    \begin{itemize}
        \item Moyenne de la population : $\mu = 175$ cm
        \item Écart-type de la population : $\sigma = 8$ cm
        \item Taille de l'échantillon : $n = 64$
    \end{itemize}
    
    \item \textbf{Appliquer le TCL :}
    Puisque $n=64$ est grand (généralement $n \ge 30$ est suffisant), le TCL s'applique. La distribution d'échantillonnage de la moyenne $\bar{X}_n$ suit approximativement une loi normale.
    $$ \bar{X}_n \approx N\left(\mu, \frac{\sigma^2}{n}\right) $$
    
    \item \textbf{Calculer les paramètres de la loi normale de $\bar{X}_n$ :}
    \begin{itemize}
        \item Espérance de $\bar{X}_n$ : $E[\bar{X}_n] = \mu = 175$ cm.
        \item Écart-type de $\bar{X}_n$ (appelé "Erreur Standard") :
        $$ \sigma_{\bar{X}_n} = \frac{\sigma}{\sqrt{n}} = \frac{8}{\sqrt{64}} = \frac{8}{8} = 1 \text{ cm} $$
    \end{itemize}
    Donc, $\bar{X}_{64} \approx N(175, 1^2)$.
    
    \item \textbf{Standardiser (Calculer le Z-score) :}
    Nous cherchons $P(\bar{X}_{64} > 177)$. Nous transformons cette valeur en un score $Z$ pour utiliser la loi normale centrée réduite $N(0, 1)$.
    $$ Z = \frac{\bar{X}_n - \mu}{\sigma_{\bar{X}_n}} = \frac{177 - 175}{1} = 2 $$
    
    \item \textbf{Trouver la probabilité :}
    Chercher $P(\bar{X}_{64} > 177)$ revient à chercher $P(Z > 2)$.
    En utilisant la table de la loi normale (ou une calculatrice) :
    $$ P(Z > 2) = 1 - P(Z \le 2) = 1 - \Phi(2) $$
    Sachant que $\Phi(2) \approx 0.9772$,
    $$ P(Z > 2) = 1 - 0.9772 = 0.0228 $$
\end{enumerate}
\textbf{Conclusion :} Il y a environ 2.28\% de chances qu'un échantillon de 64 personnes ait une taille moyenne supérieure à 177 cm.
\end{examplebox}

\begin{examplebox}[Remplissage de bouteilles]
\textbf{Contexte :} Une machine remplit des bouteilles de soda. Le volume versé $X_i$ fluctue légèrement. La loi de $X_i$ est inconnue.

\textbf{Données :} La machine est réglée pour verser en moyenne $\mu = 500$ ml. L'écart-type du processus est connu et vaut $\sigma = 6$ ml. Pour un contrôle, on prélève un échantillon de $n=36$ bouteilles.

\textbf{Problème :} On considère que la machine est déréglée si la moyenne de l'échantillon $\bar{X}_{36}$ est inférieure à 498 ml. Quelle est la probabilité d'une "fausse alarme" (c'est-à-dire, la machine fonctionne bien à $\mu=500$, mais l'échantillon a une moyenne $\bar{X}_{36} < 498$) ?

\textbf{Solution :}
\begin{enumerate}
    \item \textbf{Identifier les paramètres :}
    $\mu = 500$ ml, $\sigma = 6$ ml, $n = 36$.
    
    \item \textbf{Appliquer le TCL :}
    $n=36 \ge 30$, donc le TCL s'applique.
    $$ \bar{X}_{36} \approx N\left(\mu, \frac{\sigma^2}{n}\right) $$
    
    \item \textbf{Calculer les paramètres de $\bar{X}_{36}$ :}
    \begin{itemize}
        \item Espérance : $E[\bar{X}_{36}] = \mu = 500$ ml.
        \item Erreur Standard : $\sigma_{\bar{X}} = \frac{\sigma}{\sqrt{n}} = \frac{6}{\sqrt{36}} = \frac{6}{6} = 1$ ml.
    \end{itemize}
    Donc, $\bar{X}_{36} \approx N(500, 1^2)$.
    
    \item \textbf{Standardiser (Calculer le Z-score) :}
    Nous cherchons la probabilité $P(\bar{X}_{36} < 498)$.
    $$ Z = \frac{\bar{X}_n - \mu}{\sigma_{\bar{X}_n}} = \frac{498 - 500}{1} = -2 $$
    
    \item \textbf{Trouver la probabilité :}
    Chercher $P(\bar{X}_{36} < 498)$ revient à chercher $P(Z < -2)$.
    $$ P(Z < -2) = \Phi(-2) $$
    Par symétrie de la loi normale, $\Phi(-z) = 1 - \Phi(z)$.
    $$ P(Z < -2) = 1 - \Phi(2) = 1 - 0.9772 = 0.0228 $$
\end{enumerate}
\textbf{Conclusion :} Il y a 2.28\% de chances d'avoir une fausse alarme, c'est-à-dire de croire à tort que la machine est déréglée alors qu'elle fonctionne normalement.
\end{examplebox}

\begin{examplebox}[Rendement d'un portefeuille (sur la Somme)]
\textbf{Contexte :} Le rendement quotidien $X_i$ d'un actif est très volatile. On s'intéresse au rendement annuel \textbf{total}, qui est la \textbf{somme} des rendements quotidiens.

\textbf{Données :} Supposons que le rendement quotidien $X_i$ ait une espérance $\mu = 0.04\%$ et un écart-type $\sigma = 1\%$. (La loi de $X_i$ est inconnue, mais $\mu$ et $\sigma$ existent). Il y a $n=252$ jours de trading dans l'année.

\textbf{Problème :} Quelle est la probabilité que le rendement annuel total $S_{252} = X_1 + \dots + X_{252}$ soit négatif (inférieur à 0) ?

\textbf{Solution :}
\begin{enumerate}
    \item \textbf{Identifier les paramètres (pour une seule v.a. $X_i$) :}
    $\mu = 0.0004$, $\sigma = 0.01$, $n = 252$.
    
    \item \textbf{Appliquer le TCL (pour la somme $S_n$) :}
    $n=252$ est grand. Le TCL s'applique à la somme $S_n$.
    $$ S_n \approx N\left(n\mu, n\sigma^2\right) $$
    
    \item \textbf{Calculer les paramètres de la loi normale de $S_{252}$ :}
    \begin{itemize}
        \item Espérance de $S_{252}$ : $E[S_n] = n\mu = 252 \times 0.0004 = 0.1008$ (soit 10.08\%).
        \item Variance de $S_{252}$ : $\text{Var}(S_n) = n\sigma^2 = 252 \times (0.01)^2 = 252 \times 0.0001 = 0.0252$.
        \item Écart-type de $S_{252}$ : $\sigma_{S_n} = \sqrt{n\sigma^2} = \sqrt{0.0252} \approx 0.1587$ (soit 15.87\%).
    \end{itemize}
    Donc, $S_{252} \approx N(0.1008, 0.1587^2)$.
    
    \item \textbf{Standardiser (Calculer le Z-score) :}
    Nous cherchons $P(S_{252} < 0)$.
    $$ Z = \frac{S_n - E[S_n]}{\sigma_{S_n}} = \frac{0 - 0.1008}{0.1587} \approx -0.635 $$
    
    \item \textbf{Trouver la probabilité :}
    Chercher $P(S_{252} < 0)$ revient à chercher $P(Z < -0.635)$.
    $$ P(Z < -0.635) = \Phi(-0.635) = 1 - \Phi(0.635) $$
    En interpolant dans la table, $\Phi(0.635) \approx 0.7373$.
    $$ P(Z < -0.635) \approx 1 - 0.7373 = 0.2627 $$
\end{enumerate}
\textbf{Conclusion :} Malgré une espérance de rendement quotidien positive, il y a environ 26.3\% de chances que le rendement annuel total soit négatif.
\end{examplebox}

\begin{examplebox}[Estimation d'une proportion (Marge d'erreur)]
\textbf{Contexte :} On veut estimer la proportion $p$ de votants qui approuvent un candidat. On modélise chaque personne $i$ par une variable de Bernoulli $X_i$ (1 si "oui", 0 si "non").
L'espérance de la population est $\mu = E[X_i] = p$.
La variance de la population est $\sigma^2 = \text{Var}(X_i) = p(1-p)$.
Le résultat du sondage est la moyenne d'échantillon $\bar{X}_n = \hat{p}$ (la proportion observée).

\textbf{Données :} On sonde $n=1000$ personnes. Le résultat est que 540 personnes disent "oui". Donc $\hat{p} = 540/1000 = 0.54$.

\textbf{Problème :} Calculer l'intervalle de confiance à 95\% pour la vraie proportion $p$ (la fameuse "marge d'erreur").

\textbf{Solution :}
\begin{enumerate}
    \item \textbf{Appliquer le TCL :}
    $n=1000$ est grand. Le TCL nous dit que la proportion d'échantillon $\hat{p} = \bar{X}_n$ suit une loi normale :
    $$ \hat{p} \approx N\left(p, \frac{p(1-p)}{n}\right) $$
    
    \item \textbf{Formule de l'Intervalle de Confiance :}
    Un intervalle de confiance à 95\% est centré sur notre estimation $\hat{p}$ et s'étend de $\pm 1.96$ erreurs standard (car $P(-1.96 \le Z \le 1.96) = 0.95$).
    $$ I.C._{95\%} = \left[ \hat{p} - 1.96 \cdot \sigma_{\hat{p}} \ ; \ \hat{p} + 1.96 \cdot \sigma_{\hat{p}} \right] $$
    où $\sigma_{\hat{p}} = \sqrt{p(1-p)/n}$.
    
    \item \textbf{Estimer l'Erreur Standard :}
    Problème : nous ne connaissons pas $p$ (c'est ce que nous cherchons !). Nous ne pouvons donc pas calculer $\sigma_{\hat{p}}$.
    \textbf{Solution :} Nous l'estimons en utilisant notre meilleur estimateur pour $p$, qui est $\hat{p} = 0.54$.
    $$ \text{Erreur Standard Estimée (SE)} = \sqrt{\frac{\hat{p}(1-\hat{p})}{n}} $$
    $$ SE = \sqrt{\frac{0.54 \times (1 - 0.54)}{1000}} = \sqrt{\frac{0.54 \times 0.46}{1000}} = \sqrt{\frac{0.2484}{1000}} \approx \sqrt{0.0002484} \approx 0.01576 $$
    
    \item \textbf{Calculer la Marge d'Erreur :}
    La marge d'erreur (ME) est la demi-largeur de l'intervalle.
    $$ ME = 1.96 \times SE = 1.96 \times 0.01576 \approx 0.0309 $$
    
    \item \textbf{Construire l'Intervalle :}
    $$ I.C._{95\%} = [ 0.54 - 0.0309 \ ; \ 0.54 + 0.0309 ] = [ 0.5091 \ ; \ 0.5709 ] $$
\end{enumerate}
\textbf{Conclusion :} Avec 54\% d'intentions de vote sur un échantillon de 1000 personnes, nous sommes confiants à 95\% que la vraie proportion $p$ dans la population se situe entre 50.9\% et 57.1\%. La marge d'erreur du sondage est de $\pm 3.1\%$.
\end{examplebox}
\newpage

\section{Le Théorème Central Limite (TCL)}

\subsection{Introduction : L'omniprésence de la loi normale}

Dans la section précédente, la Loi des Grands Nombres (LLN) nous a donné une garantie fondamentale : la moyenne d'échantillon $\bar{X}_n$ converge vers la vraie moyenne $\mu$ lorsque $n$ devient grand.
$$ \bar{X}_n \xrightarrow{p.s.} \mu $$
La LLN nous dit \textbf{où} la moyenne d'échantillon converge (vers la constante $\mu$), mais elle ne nous dit rien sur la \textit{forme} de la distribution de $\bar{X}_n$ autour de $\mu$ pour un $n$ grand, mais fini.

Le \textbf{Théorème Central Limite (TCL)} comble cette lacune. Il décrit la \textit{manière} dont $\bar{X}_n$ converge, en nous donnant la forme de sa distribution. C'est sans doute le théorème le plus important des statistiques.

\begin{intuitionbox}[L'Idée Fondamentale]
Intuitivement, ce résultat affirme qu'une \textbf{somme} d'un grand nombre de variables aléatoires indépendantes et identiquement distribuées (i.i.d.) tend, le plus souvent, à suivre une \textbf{loi normale} (aussi appelée loi de Laplace-Gauss ou "courbe en cloche").

Ce théorème et ses généralisations offrent une explication à l'omniprésence de la loi normale dans la nature. De nombreux phénomènes (la taille d'un individu, l'erreur de mesure d'un instrument, le bruit de fond d'un signal) sont le résultat de l'addition d'un très grand nombre de petites perturbations aléatoires. Le TCL nous dit que le résultat de cette somme sera, inévitablement, distribué selon une loi normale.
\end{intuitionbox}

\subsection{L'illustration : la somme des "Pile ou Face"}

Prenons l'exemple le plus simple pour illustrer ce phénomène : le jeu de "pile ou face".



\begin{examplebox}[Distribution de la Somme de $n$ Lancers]
Soit $X_i$ le résultat du $i$-ème lancer, avec $X_i = 1$ pour "Face" (probabilité 0,5) et $X_i = 0$ pour "Pile" (probabilité 0,5). La distribution d'origine (pour $n=1$) n'est pas du tout une courbe en cloche : c'est une distribution discrète avec deux bâtons de même hauteur.

Considérons la \textbf{somme} $S_n = X_1 + X_2 + \dots + X_n$, qui représente le nombre total de "Face" obtenus en $n$ lancers.

\begin{itemize}
    \item \textbf{Pour $n=1$ :} La distribution de $S_1$ est :
    \begin{itemize}
        \item Valeurs de la somme : \{0, 1\}
        \item Fréquences : \{0.5, 0.5\}
    \end{itemize}
    
    \item \textbf{Pour $n=2$ :} Les sommes possibles sont \{0, 1, 2\}. La distribution de $S_2$ est :
    \begin{itemize}
        \item Valeurs de la somme : \{0, 1, 2\}
        \item Fréquences : \{0.25, 0.5, 0.25\} (elle forme un triangle).
    \end{itemize}
    
    \item \textbf{Pour $n=3$ :} Les sommes possibles sont \{0, 1, 2, 3\}. La distribution de $S_3$ est :
    \begin{itemize}
        \item Valeurs de la somme : \{0, 1, 2, 3\}
        \item Fréquences : \{0.125, 0.375, 0.375, 0.125\}
    \end{itemize}
\end{itemize}

\begin{center}
\begin{tikzpicture}
  \begin{axis}[
    width=0.8\textwidth,
    height=0.5\textwidth,
    xlabel={Valeurs de la somme},
    title={Fonction de fréquence pour des tirages à pile ou face},
    grid=both,
    grid style={line width=.1pt, draw=gray!30},
    major grid style={line width=.2pt,draw=gray!50},
    domain=0:4,
    samples=100,
    enlargelimits=false,
    axis lines=middle,
    xmin=0, xmax=4,
    ymin=0, ymax=0.6
  ]
  % n=1
  \addplot [thick, color=blue, fill=blue!20, fill opacity=0.5] coordinates {(0,0.5) (1,0.5)} \closedcycle;
  % n=2
  \addplot [thick, color=green, fill=green!20, fill opacity=0.5] coordinates {(0,0.25) (1,0.5) (2,0.25)} \closedcycle;
  % n=3
  \addplot [thick, color=red, fill=red!20, fill opacity=0.5] coordinates {(0,0.125) (1,0.375) (2,0.375) (3,0.125)} \closedcycle;
  % n=4 (suggéré)
  \addplot [thick, color=black, fill=black!20, fill opacity=0.5] coordinates {(0,0.0625) (1,0.25) (2,0.375) (3,0.25) (4,0.0625)} \closedcycle;
  
  \legend{$n=1$,$n=2$,$n=3$,$n=4$}
  \end{axis}
\end{tikzpicture}
\end{center}

Graphiquement, on constate que plus le nombre de tirages $n$ augmente (par exemple, jusqu'à $n=12$), plus la courbe de fréquence (qui reste discrète) se rapproche d'une courbe en cloche symétrique, caractéristique de la loi normale.
\end{examplebox}

\subsection{Distribution de la population vs. Distribution d'échantillonnage}

Le point le plus remarquable du TCL est qu'il fonctionne \textit{quelle que soit} la distribution de départ.

\begin{intuitionbox}[Population vs. Échantillonnage]
Imaginez deux univers de distributions :

\begin{itemize}
    \item \textbf{1. La Distribution de la Population ($X_i$) :} C'est la loi de nos variables $X_i$ individuelles. Elle peut avoir \textbf{n'importe quelle forme} (par exemple, une distribution bimodale, asymétrique, ou uniforme). Cette distribution a une "vraie" moyenne $\mu$ et un "vrai" écart-type $\sigma$.
    
    \item \textbf{2. La Distribution d'Échantillonnage ($\bar{X}_n$) :} C'est la distribution de la \textit{moyenne} $\bar{X}_n = (X_1 + \dots + X_n)/n$, calculée sur des échantillons de taille $n$. C'est la distribution de "toutes les moyennes d'échantillon possibles".
\end{itemize}

Le TCL énonce la relation magique entre les deux :

\textbf{Quelle que soit la forme de la distribution de la population, plus la taille de l'échantillon $n$ croît, plus la distribution d'échantillonnage de la moyenne $\bar{X}_n$ est proche d'une loi normale (gaussienne).}

De plus, les paramètres de cette loi normale sont :
\begin{itemize}
    \item \textbf{Moyenne :} La distribution de $\bar{X}_n$ est centrée sur la même moyenne $\mu$ que la population.
    \item \textbf{Écart-type :} La distribution de $\bar{X}_n$ est beaucoup plus resserrée. Son écart-type (appelé "erreur standard") est $\sigma_{\bar{X}} = \frac{\sigma}{\sqrt{n}}$.
\end{itemize}
Cette dispersion $\sigma/\sqrt{n}$ qui tend vers 0 est la manifestation de la Loi des Grands Nombres. Le TCL précise que la \textit{forme} de cette convergence est gaussienne.
\end{intuitionbox}

\subsection{Énoncé formel du Théorème Central Limite}

Pour énoncer le théorème formellement, nous devons d'abord définir les propriétés de la somme $S_n$ et de la moyenne $\bar{X}_n$.

Soit $X_1, \dots, X_n$ des variables aléatoires i.i.d. avec $E[X_i] = \mu$ et $\text{Var}(X_i) = \sigma^2$.

\begin{itemize}
    \item \textbf{La Somme $S_n = \sum X_i$} :
    \begin{itemize}
        \item Espérance : $E[S_n] = E[\sum X_i] = \sum E[X_i] = n\mu$
        \item Variance : $\text{Var}(S_n) = \text{Var}(\sum X_i) = \sum \text{Var}(X_i) = n\sigma^2$
        \item Écart-type : $\sigma_{S_n} = \sqrt{n\sigma^2} = \sigma\sqrt{n}$
    \end{itemize}
    
    \item \textbf{La Moyenne $\bar{X}_n = S_n / n$} :
    \begin{itemize}
        \item Espérance : $E[\bar{X}_n] = E[S_n / n] = \frac{1}{n} E[S_n] = \frac{1}{n} (n\mu) = \mu$
        \item Variance : $\text{Var}(\bar{X}_n) = \text{Var}(S_n / n) = \frac{1}{n^2} \text{Var}(S_n) = \frac{1}{n^2} (n\sigma^2) = \frac{\sigma^2}{n}$
        \item Écart-type : $\sigma_{\bar{X}_n} = \sqrt{\sigma^2 / n} = \frac{\sigma}{\sqrt{n}}$
    \end{itemize}
\end{itemize}

Nous voyons que la distribution de $S_n$ s'étale (variance $\to \infty$) tandis que celle de $\bar{X}_n$ se contracte (variance $\to 0$). Pour étudier la \textit{forme} de la convergence, nous créons une variable "stable" en la centrant (soustrayant la moyenne) et en la réduisant (divisant par l'écart-type). C'est la variable $Z_n$.

\begin{theorembox}[Théorème Central Limite (Lindeberg-Lévy)]
Soit $X_1, X_2, \dots, X_n$ une suite de variables aléatoires \textbf{i.i.d.} (indépendantes et identiquement distribuées) suivant la même loi $D$.
Supposons que l'\textbf{espérance $\mu$} et l'\textbf{écart-type $\sigma$} de cette loi $D$ existent, sont finis, et $\sigma \neq 0$.

Considérons la variable aléatoire standardisée $Z_n$ :
$$ Z_n = \frac{S_n - E[S_n]}{\sigma_{S_n}} = \frac{S_n - n\mu}{\sigma\sqrt{n}} $$
Cette variable est équivalente à la moyenne standardisée :
$$ Z_n = \frac{\bar{X}_n - E[\bar{X}_n]}{\sigma_{\bar{X}_n}} = \frac{\bar{X}_n - \mu}{\sigma / \sqrt{n}} $$
(Pour tout $n$, $Z_n$ est une variable centrée-réduite : $E[Z_n] = 0$ et $\text{Var}(Z_n) = 1$).

Alors, la suite de variables aléatoires $Z_1, Z_2, \dots, Z_n, \dots$ \textbf{converge en loi} vers une variable aléatoire $Z$ qui suit la \textbf{loi normale centrée réduite $N(0, 1)$}, lorsque $n$ tend vers l'infini.

Cela signifie que si $\Phi$ est la fonction de répartition de la loi $N(0, 1)$, alors pour tout réel $z$ :
$$ \lim_{n \to \infty} P(Z_n \le z) = \lim_{n \to \infty} P\left( \frac{\bar{X}_n - \mu}{\sigma/\sqrt{n}} \le z \right) = \Phi(z) $$
\end{theorembox}

\subsection{Applications Pratiques du TCL}

Le TCL n'est pas seulement une curiosité mathématique ; c'est le fondement de l'inférence statistique. Voici comment l'appliquer concrètement pour résoudre des problèmes.

\begin{examplebox}[La taille des individus]
\textbf{Contexte :} La taille des individus dans une population suit une courbe en cloche. Pourquoi ? Car elle est la \textbf{somme} de milliers de petites influences (gènes, nutrition, etc.). Le TCL s'applique.

\textbf{Données :} Supposons que dans une population, la taille $X$ des individus ait une espérance $\mu = 175$ cm et un écart-type $\sigma = 8$ cm. (Note : la loi de $X$ n'est pas forcément normale, même si en pratique elle l'est).

\textbf{Problème :} On prélève un échantillon aléatoire de $n=64$ individus. Quelle est la probabilité que la \textbf{moyenne de cet échantillon} ($\bar{X}_{64}$) soit supérieure à 177 cm ?

\textbf{Solution :}
\begin{enumerate}
    \item \textbf{Identifier les paramètres :}
    \begin{itemize}
        \item Moyenne de la population : $\mu = 175$ cm
        \item Écart-type de la population : $\sigma = 8$ cm
        \item Taille de l'échantillon : $n = 64$
    \end{itemize}
    
    \item \textbf{Appliquer le TCL :}
    Puisque $n=64$ est grand (généralement $n \ge 30$ est suffisant), le TCL s'applique. La distribution d'échantillonnage de la moyenne $\bar{X}_n$ suit approximativement une loi normale.
    $$ \bar{X}_n \approx N\left(\mu, \frac{\sigma^2}{n}\right) $$
    
    \item \textbf{Calculer les paramètres de la loi normale de $\bar{X}_n$ :}
    \begin{itemize}
        \item Espérance de $\bar{X}_n$ : $E[\bar{X}_n] = \mu = 175$ cm.
        \item Écart-type de $\bar{X}_n$ (appelé "Erreur Standard") :
        $$ \sigma_{\bar{X}_n} = \frac{\sigma}{\sqrt{n}} = \frac{8}{\sqrt{64}} = \frac{8}{8} = 1 \text{ cm} $$
    \end{itemize}
    Donc, $\bar{X}_{64} \approx N(175, 1^2)$.
    
    \item \textbf{Standardiser (Calculer le Z-score) :}
    Nous cherchons $P(\bar{X}_{64} > 177)$. Nous transformons cette valeur en un score $Z$ pour utiliser la loi normale centrée réduite $N(0, 1)$.
    $$ Z = \frac{\bar{X}_n - \mu}{\sigma_{\bar{X}_n}} = \frac{177 - 175}{1} = 2 $$
    
    \item \textbf{Trouver la probabilité :}
    Chercher $P(\bar{X}_{64} > 177)$ revient à chercher $P(Z > 2)$.
    En utilisant la table de la loi normale (ou une calculatrice) :
    $$ P(Z > 2) = 1 - P(Z \le 2) = 1 - \Phi(2) $$
    Sachant que $\Phi(2) \approx 0.9772$,
    $$ P(Z > 2) = 1 - 0.9772 = 0.0228 $$
\end{enumerate}
\textbf{Conclusion :} Il y a environ 2.28\% de chances qu'un échantillon de 64 personnes ait une taille moyenne supérieure à 177 cm.
\end{examplebox}

\begin{examplebox}[Remplissage de bouteilles]
\textbf{Contexte :} Une machine remplit des bouteilles de soda. Le volume versé $X_i$ fluctue légèrement. La loi de $X_i$ est inconnue.

\textbf{Données :} La machine est réglée pour verser en moyenne $\mu = 500$ ml. L'écart-type du processus est connu et vaut $\sigma = 6$ ml. Pour un contrôle, on prélève un échantillon de $n=36$ bouteilles.

\textbf{Problème :} On considère que la machine est déréglée si la moyenne de l'échantillon $\bar{X}_{36}$ est inférieure à 498 ml. Quelle est la probabilité d'une "fausse alarme" (c'est-à-dire, la machine fonctionne bien à $\mu=500$, mais l'échantillon a une moyenne $\bar{X}_{36} < 498$) ?

\textbf{Solution :}
\begin{enumerate}
    \item \textbf{Identifier les paramètres :}
    $\mu = 500$ ml, $\sigma = 6$ ml, $n = 36$.
    
    \item \textbf{Appliquer le TCL :}
    $n=36 \ge 30$, donc le TCL s'applique.
    $$ \bar{X}_{36} \approx N\left(\mu, \frac{\sigma^2}{n}\right) $$
    
    \item \textbf{Calculer les paramètres de $\bar{X}_{36}$ :}
    \begin{itemize}
        \item Espérance : $E[\bar{X}_{36}] = \mu = 500$ ml.
        \item Erreur Standard : $\sigma_{\bar{X}} = \frac{\sigma}{\sqrt{n}} = \frac{6}{\sqrt{36}} = \frac{6}{6} = 1$ ml.
    \end{itemize}
    Donc, $\bar{X}_{36} \approx N(500, 1^2)$.
    
    \item \textbf{Standardiser (Calculer le Z-score) :}
    Nous cherchons la probabilité $P(\bar{X}_{36} < 498)$.
    $$ Z = \frac{\bar{X}_n - \mu}{\sigma_{\bar{X}_n}} = \frac{498 - 500}{1} = -2 $$
    
    \item \textbf{Trouver la probabilité :}
    Chercher $P(\bar{X}_{36} < 498)$ revient à chercher $P(Z < -2)$.
    $$ P(Z < -2) = \Phi(-2) $$
    Par symétrie de la loi normale, $\Phi(-z) = 1 - \Phi(z)$.
    $$ P(Z < -2) = 1 - \Phi(2) = 1 - 0.9772 = 0.0228 $$
\end{enumerate}
\textbf{Conclusion :} Il y a 2.28\% de chances d'avoir une fausse alarme, c'est-à-dire de croire à tort que la machine est déréglée alors qu'elle fonctionne normalement.
\end{examplebox}

\begin{examplebox}[Rendement d'un portefeuille (sur la Somme)]
\textbf{Contexte :} Le rendement quotidien $X_i$ d'un actif est très volatile. On s'intéresse au rendement annuel \textbf{total}, qui est la \textbf{somme} des rendements quotidiens.

\textbf{Données :} Supposons que le rendement quotidien $X_i$ ait une espérance $\mu = 0.04\%$ et un écart-type $\sigma = 1\%$. (La loi de $X_i$ est inconnue, mais $\mu$ et $\sigma$ existent). Il y a $n=252$ jours de trading dans l'année.

\textbf{Problème :} Quelle est la probabilité que le rendement annuel total $S_{252} = X_1 + \dots + X_{252}$ soit négatif (inférieur à 0) ?

\textbf{Solution :}
\begin{enumerate}
    \item \textbf{Identifier les paramètres (pour une seule v.a. $X_i$) :}
    $\mu = 0.0004$, $\sigma = 0.01$, $n = 252$.
    
    \item \textbf{Appliquer le TCL (pour la somme $S_n$) :}
    $n=252$ est grand. Le TCL s'applique à la somme $S_n$.
    $$ S_n \approx N\left(n\mu, n\sigma^2\right) $$
    
    \item \textbf{Calculer les paramètres de la loi normale de $S_{252}$ :}
    \begin{itemize}
        \item Espérance de $S_{252}$ : $E[S_n] = n\mu = 252 \times 0.0004 = 0.1008$ (soit 10.08\%).
        \item Variance de $S_{252}$ : $\text{Var}(S_n) = n\sigma^2 = 252 \times (0.01)^2 = 252 \times 0.0001 = 0.0252$.
        \item Écart-type de $S_{252}$ : $\sigma_{S_n} = \sqrt{n\sigma^2} = \sqrt{0.0252} \approx 0.1587$ (soit 15.87\%).
    \end{itemize}
    Donc, $S_{252} \approx N(0.1008, 0.1587^2)$.
    
    \item \textbf{Standardiser (Calculer le Z-score) :}
    Nous cherchons $P(S_{252} < 0)$.
    $$ Z = \frac{S_n - E[S_n]}{\sigma_{S_n}} = \frac{0 - 0.1008}{0.1587} \approx -0.635 $$
    
    \item \textbf{Trouver la probabilité :}
    Chercher $P(S_{252} < 0)$ revient à chercher $P(Z < -0.635)$.
    $$ P(Z < -0.635) = \Phi(-0.635) = 1 - \Phi(0.635) $$
    En interpolant dans la table, $\Phi(0.635) \approx 0.7373$.
    $$ P(Z < -0.635) \approx 1 - 0.7373 = 0.2627 $$
\end{enumerate}
\textbf{Conclusion :} Malgré une espérance de rendement quotidien positive, il y a environ 26.3\% de chances que le rendement annuel total soit négatif.
\end{examplebox}

\begin{examplebox}[Estimation d'une proportion (Marge d'erreur)]
\textbf{Contexte :} On veut estimer la proportion $p$ de votants qui approuvent un candidat. On modélise chaque personne $i$ par une variable de Bernoulli $X_i$ (1 si "oui", 0 si "non").
L'espérance de la population est $\mu = E[X_i] = p$.
La variance de la population est $\sigma^2 = \text{Var}(X_i) = p(1-p)$.
Le résultat du sondage est la moyenne d'échantillon $\bar{X}_n = \hat{p}$ (la proportion observée).

\textbf{Données :} On sonde $n=1000$ personnes. Le résultat est que 540 personnes disent "oui". Donc $\hat{p} = 540/1000 = 0.54$.

\textbf{Problème :} Calculer l'intervalle de confiance à 95\% pour la vraie proportion $p$ (la fameuse "marge d'erreur").

\textbf{Solution :}
\begin{enumerate}
    \item \textbf{Appliquer le TCL :}
    $n=1000$ est grand. Le TCL nous dit que la proportion d'échantillon $\hat{p} = \bar{X}_n$ suit une loi normale :
    $$ \hat{p} \approx N\left(p, \frac{p(1-p)}{n}\right) $$
    
    \item \textbf{Formule de l'Intervalle de Confiance :}
    Un intervalle de confiance à 95\% est centré sur notre estimation $\hat{p}$ et s'étend de $\pm 1.96$ erreurs standard (car $P(-1.96 \le Z \le 1.96) = 0.95$).
    $$ I.C._{95\%} = \left[ \hat{p} - 1.96 \cdot \sigma_{\hat{p}} \ ; \ \hat{p} + 1.96 \cdot \sigma_{\hat{p}} \right] $$
    où $\sigma_{\hat{p}} = \sqrt{p(1-p)/n}$.
    
    \item \textbf{Estimer l'Erreur Standard :}
    Problème : nous ne connaissons pas $p$ (c'est ce que nous cherchons !). Nous ne pouvons donc pas calculer $\sigma_{\hat{p}}$.
    \textbf{Solution :} Nous l'estimons en utilisant notre meilleur estimateur pour $p$, qui est $\hat{p} = 0.54$.
    $$ \text{Erreur Standard Estimée (SE)} = \sqrt{\frac{\hat{p}(1-\hat{p})}{n}} $$
    $$ SE = \sqrt{\frac{0.54 \times (1 - 0.54)}{1000}} = \sqrt{\frac{0.54 \times 0.46}{1000}} = \sqrt{\frac{0.2484}{1000}} \approx \sqrt{0.0002484} \approx 0.01576 $$
    
    \item \textbf{Calculer la Marge d'Erreur :}
    La marge d'erreur (ME) est la demi-largeur de l'intervalle.
    $$ ME = 1.96 \times SE = 1.96 \times 0.01576 \approx 0.0309 $$
    
    \item \textbf{Construire l'Intervalle :}
    $$ I.C._{95\%} = [ 0.54 - 0.0309 \ ; \ 0.54 + 0.0309 ] = [ 0.5091 \ ; \ 0.5709 ] $$
\end{enumerate}
\textbf{Conclusion :} Avec 54\% d'intentions de vote sur un échantillon de 1000 personnes, nous sommes confiants à 95\% que la vraie proportion $p$ dans la population se situe entre 50.9\% et 57.1\%. La marge d'erreur du sondage est de $\pm 3.1\%$.
\end{examplebox}
\newpage

\section{Le Mouvement Brownien}

\subsection{Définition Formelle}

Le mouvement Brownien (ou processus de Wiener) est un concept central en finance et en physique, modélisant des trajectoires aléatoires continues, comme celle d'une particule de pollen dans l'eau ou le prix d'un actif financier.

\begin{theorembox}[Définition : Mouvement Brownien]
Une collection de variables aléatoires $\{X(t), t \ge 0\}$ est un \textbf{mouvement Brownien} avec un paramètre de \textbf{dérive} (drift) $\mu$ et un paramètre de \textbf{variance} $\sigma^2$ si les propriétés suivantes sont vérifiées :

\begin{itemize}
    \item \textbf{(a)} Le processus commence à une valeur constante : $X(0) = c$. (Par convention, on suppose souvent $X(0) = 0$).
    
    \item \textbf{(b) Accroissements indépendants :} Pour toute suite de temps $0 \le s < t < u < v$, les accroissements $X(t) - X(s)$ et $X(v) - X(u)$ sont des variables aléatoires indépendantes. Plus généralement, l'accroissement futur $X(t) - X(s)$ (pour $t>s$) est indépendant du passé du processus (c'est-à-dire, de l'ensemble des $\{X(u) : u \le s\}$).
    
    \item \textbf{(c) Accroissements stationnaires et normaux :} L'accroissement $X(t) - X(s)$ (pour $t>s$) suit une \textbf{loi normale} dont la moyenne et la variance sont proportionnelles à la durée de l'intervalle $(t-s)$ :
    $$ X(t) - X(s) \sim N\left(\mu(t-s), \sigma^2(t-s)\right) $$
\end{itemize}
\end{theorembox}

\subsection{Propriétés : Continuité et Non-Différentiabilité}

Les trajectoires du mouvement Brownien ont deux propriétés fondamentales qui semblent contradictoires, mais qui coexistent.

\begin{intuitionbox}[La Trajectoire Brownienne]
Avec une probabilité de 1, une trajectoire $X(t)$ d'un mouvement Brownien est :
\begin{itemize}
    \item \textbf{1. Continue :} Il n'y a pas de "sauts" instantanés. La fonction $t \mapsto X(t)$ peut être dessinée sans lever le crayon.
    \item \textbf{2. Nulle part Différentiable :} En aucun point $t$, la trajectoire n'est "lisse". Elle est infiniment "rugueuse" ou "agitée", et il est impossible de définir une tangente (une dérivée) en quelque point que ce soit.
\end{itemize}
\end{intuitionbox}

\subsubsection*{Idée de la preuve (selon les notes)}

\textbf{1. Continuité :}
Pour montrer la continuité, nous devons montrer que $\lim_{h \to 0} (X(t+h) - X(t)) = 0$.
Par la définition (c), l'accroissement $X(t+h) - X(t)$ suit la loi $N(\mu h, \sigma^2 h)$.
Lorsque $h \to 0$, la moyenne $\mu h \to 0$ et la variance $\sigma^2 h \to 0$.
La distribution de l'accroissement converge vers une masse de Dirac en 0 (une variable aléatoire constante égale à 0). Cela suggère fortement que la trajectoire est continue.

\textbf{2. Non-Différentiabilité :}
Pour examiner la différentiabilité, nous étudions la limite du taux d'accroissement (la "pente") lorsque $h \to 0$ :
$$ \frac{X(t+h) - X(t)}{h} $$
Cette nouvelle variable aléatoire est aussi normale (car c'est une transformation linéaire d'une v.a. normale). Calculons sa moyenne et sa variance :

\begin{itemize}
    \item \textbf{Moyenne :} 
    $$ E\left[ \frac{X(t+h) - X(t)}{h} \right] = \frac{1}{h} E[X(t+h) - X(t)] = \frac{1}{h} (\mu h) = \mu $$
    
    \item \textbf{Variance :} 
    $$ \text{Var}\left( \frac{X(t+h) - X(t)}{h} \right) = \frac{1}{h^2} \text{Var}(X(t+h) - X(t)) = \frac{1}{h^2} (\sigma^2 h) = \frac{\sigma^2}{h} $$
\end{itemize}

Lorsque $h \to 0$, la moyenne de la pente reste $\mu$, mais sa variance $\frac{\sigma^2}{h}$ \textbf{converge vers l'infini}.
Parce que la variance explose, le taux d'accroissement ne converge pas vers une valeur finie. La limite $\lim_{h \to 0} \frac{X(t+h) - X(t)}{h}$ n'existe pas, et la fonction n'est donc pas différentiable.

\subsection{Construction : Le Mouvement Brownien comme Limite d'un Modèle Simple}

Le mouvement Brownien peut être compris comme la limite d'une simple marche aléatoire à temps discret, lorsque les pas de temps deviennent infiniment petits.

\subsubsection{Le modèle de la marche aléatoire discrète}

Construisons un processus simple. Nous divisons le temps en petits intervalles de durée $\Delta t$. À chaque pas de temps, le processus $X$ fait un "saut" :
\begin{itemize}
    \item Il \textbf{augmente} de $\Delta x = \sigma\sqrt{\Delta t}$ avec une probabilité $p$.
    \item Il \textbf{diminue} de $\Delta x = \sigma\sqrt{\Delta t}$ avec une probabilité $1-p$.
\end{itemize}
Les sauts successifs sont supposés indépendants. Le choix de $\sigma\sqrt{\Delta t}$ (et non $\sigma \Delta t$) est crucial pour obtenir une variance non nulle à la limite.

Pour modéliser une dérive $\mu$, nous ajustons la probabilité $p$ pour qu'elle soit légèrement déséquilibrée :
$$ p = \frac{1}{2} \left( 1 + \frac{\mu}{\sigma} \sqrt{\Delta t} \right) $$

Soit $X_i$ une variable aléatoire décrivant le $i$-ème pas :
$$ X_i = \begin{cases} +1 & \text{avec probabilité } p \text{ (hausse)} \\ -1 & \text{avec probabilité } 1-p \text{ (baisse)} \end{cases} $$
Après un temps total $t$, nous avons effectué $n = t / \Delta t$ pas.
La variation totale du processus, $X(t) - X(0)$, est la somme de tous ces petits sauts :
$$ X(t) - X(0) = \sum_{i=1}^{n} (\text{saut}_i) = \sum_{i=1}^{n} (X_i \cdot \sigma\sqrt{\Delta t}) = \sigma\sqrt{\Delta t} \sum_{i=1}^{n} X_i $$

\subsubsection{Convergence via le Théorème Central Limite}

Nous analysons ce qu'il advient de $X(t) - X(0)$ lorsque $\Delta t \to 0$.
\begin{itemize}
    \item Lorsque $\Delta t \to 0$, le nombre de pas $n = t/\Delta t \to \infty$.
    \item L'expression $X(t) - X(0)$ est (à un facteur près) une \textbf{somme de $n$ variables aléatoires i.i.d.} (les $X_i$).
\end{itemize}
C'est le scénario d'application du \textbf{Théorème Central Limite (TCL)}. Le TCL stipule que la distribution de cette somme, pour $n$ grand, tend vers une loi normale. Pour identifier les paramètres $\mu_t$ et $\sigma_t^2$ de cette loi normale, nous devons calculer l'espérance et la variance de $X(t) - X(0)$ et prendre leur limite.

\begin{examplebox}[Calculs de l'Espérance et de la Variance]
Commençons par l'espérance et la variance d'un seul pas $X_i$.

\textbf{1. Espérance de $X_i$ :}
$$ E[X_i] = (+1) \cdot p + (-1) \cdot (1-p) = 2p - 1 $$
En substituant $p = \frac{1}{2} ( 1 + \frac{\mu}{\sigma} \sqrt{\Delta t} )$ :
$$ E[X_i] = 2 \left[ \frac{1}{2} \left( 1 + \frac{\mu}{\sigma} \sqrt{\Delta t} \right) \right] - 1 = \left( 1 + \frac{\mu}{\sigma} \sqrt{\Delta t} \right) - 1 = \frac{\mu}{\sigma} \sqrt{\Delta t} $$

\textbf{2. Variance de $X_i$ :}
D'abord, calculons $E[X_i^2]$. Puisque $X_i$ vaut $+1$ ou $-1$, $X_i^2$ vaut toujours 1.
$$ E[X_i^2] = (+1)^2 \cdot p + (-1)^2 \cdot (1-p) = p + (1-p) = 1 $$
La variance est donc :
$$ \text{Var}(X_i) = E[X_i^2] - (E[X_i])^2 = 1 - \left( \frac{\mu}{\sigma} \sqrt{\Delta t} \right)^2 = 1 - \frac{\mu^2}{\sigma^2} \Delta t $$
(Note : dans l'image, $\text{Var}(X_i)$ est écrit $1 - (2p-1)^2$, ce qui est la même chose).

\textbf{3. Espérance de $X(t) - X(0)$ :}
Par linéarité de l'espérance ($n = t/\Delta t$) :
$$ E[X(t) - X(0)] = E\left[ \sigma\sqrt{\Delta t} \sum_{i=1}^{n} X_i \right] = \sigma\sqrt{\Delta t} \cdot \sum_{i=1}^{n} E[X_i] $$
$$ = \sigma\sqrt{\Delta t} \cdot n \cdot E[X_i] = \sigma\sqrt{\Delta t} \cdot \left( \frac{t}{\Delta t} \right) \cdot \left( \frac{\mu}{\sigma} \sqrt{\Delta t} \right) $$
$$ = \frac{\sigma \cdot \sqrt{\Delta t} \cdot t \cdot \mu \cdot \sqrt{\Delta t}}{\Delta t \cdot \sigma} = \frac{(\sigma t \mu) (\Delta t)}{(\Delta t \sigma)} = \mu t $$
L'espérance est \textit{exactement} $\mu t$, quel que soit $\Delta t$.

\textbf{4. Variance de $X(t) - X(0)$ :}
Par indépendance des $X_i$ ($\text{Var}(\sum X_i) = \sum \text{Var}(X_i)$) :
$$ \text{Var}(X(t) - X(0)) = \text{Var}\left( \sigma\sqrt{\Delta t} \sum_{i=1}^{n} X_i \right) = (\sigma\sqrt{\Delta t})^2 \cdot \text{Var}\left( \sum_{i=1}^{n} X_i \right) $$
$$ = (\sigma^2 \Delta t) \cdot \sum_{i=1}^{n} \text{Var}(X_i) = (\sigma^2 \Delta t) \cdot n \cdot \text{Var}(X_i) $$
$$ = (\sigma^2 \Delta t) \cdot \left( \frac{t}{\Delta t} \right) \cdot \left( 1 - \frac{\mu^2}{\sigma^2} \Delta t \right) $$
$$ = \sigma^2 t \left( 1 - \frac{\mu^2}{\sigma^2} \Delta t \right) $$
\end{examplebox}

\subsubsection{Le passage à la limite}

Nous avons établi que $X(t) - X(0)$ est une somme de $n \to \infty$ v.a. i.i.d. Le TCL s'applique, et la distribution de $X(t) - X(0)$ converge vers une loi normale.
Les paramètres de cette loi normale sont les limites de l'espérance et de la variance lorsque $\Delta t \to 0$ :

\begin{itemize}
    \item \textbf{Moyenne Limite :} 
    $$ \lim_{\Delta t \to 0} E[X(t) - X(0)] = \lim_{\Delta t \to 0} (\mu t) = \mu t $$
    
    \item \textbf{Variance Limite :} 
    $$ \lim_{\Delta t \to 0} \text{Var}(X(t) - X(0)) = \lim_{\Delta t \to 0} \left[ \sigma^2 t \left( 1 - \frac{\mu^2}{\sigma^2} \Delta t \right) \right] $$
    $$ = \sigma^2 t (1 - 0) = \sigma^2 t $$
\end{itemize}

\begin{intuitionbox}[Conclusion]
Lorsque $\Delta t \to 0$, notre modèle de marche aléatoire $X(t) - X(0)$ converge (en loi) vers une variable aléatoire qui suit une \textbf{loi normale $N(\mu t, \sigma^2 t)$}.

C'est \textbf{exactement} la distribution de l'accroissement $X(t) - X(0)$ requise par la définition formelle du mouvement Brownien (propriété c).

Cela démontre que le mouvement Brownien, un processus continu complexe, peut être construit comme la limite d'une simple marche aléatoire binaire, à condition que les pas soient mis à l'échelle en $\sqrt{\Delta t}$.
\end{intuitionbox}

\newpage
\subsection{Le Mouvement Brownien Géométrique (MBG)}

\subsubsection{Introduction : D'Additif à Multiplicatif}

Le Mouvement Brownien (MB) simple, $X(t)$, que nous avons étudié précédemment, est un processus \textit{additif}. Ses accroissements s'ajoutent les uns aux autres.

Cependant, pour modéliser le prix d'actifs financiers (comme une action), ce modèle présente deux défauts majeurs :
\begin{enumerate}
    \item \textbf{Prix négatifs :} Un MB peut (et va presque sûrement) prendre des valeurs négatives. Le prix d'une action ne peut pas descendre en dessous de zéro.
    \item \textbf{Chocs absolus :} L'ampleur d'un choc aléatoire ($\sigma dW$) est constante. Un choc de $+1€$ a le même impact, que l'action vaille 2€ ou 1000€. En réalité, les investisseurs pensent en \textbf{pourcentages} (un choc de $+1\%$).
\end{enumerate}

Nous avons besoin d'un processus \textit{multiplicatif} ou "géométrique". L'astuce mathématique pour transformer l'addition en multiplication est la fonction \textbf{exponentielle}.

\begin{intuitionbox}[L'Idée Fondamentale]
Si les \textbf{rendements} (en pourcentage continu) sont additifs et suivent un Mouvement Brownien...
...Alors le \textbf{prix} (qui est le résultat de ces rendements) doit être l'exponentielle de ce Mouvement Brownien.

L'hypothèse centrale du MBG est que le \textbf{logarithme du prix} se comporte comme un simple Mouvement Brownien.
$$ \ln(S(t)) = \text{Mouvement Brownien} $$
Cela résout nos deux problèmes :
\begin{enumerate}
    \item Si $\ln(S(t)) = X(t)$, alors $S(t) = e^{X(t)}$. Puisque $e^x$ est toujours positif, le prix $S(t)$ ne peut jamais être négatif.
    \item Si le \textit{log-rendement} $\ln(S(t)) - \ln(S(y))$ suit un MB, cela correspond à une modélisation en pourcentages.
\end{enumerate}
\end{intuitionbox}

\subsubsection{Définition (basée sur les notes)}

Cette intuition nous amène directement à la définition formelle.

\begin{definitionbox}[Définition : Mouvement Brownien Géométrique]
Soit $\{X(t), t \ge 0\}$ un \textbf{Mouvement Brownien} (arithmétique) avec :
\begin{itemize}
    \item Un paramètre de dérive (drift) $\mu$
    \item Un paramètre de variance $\sigma^2$
\end{itemize}
(Rappel : cela signifie que $X(t)$ démarre à 0, a des accroissements indépendants et stationnaires, et $X(t) \sim N(\mu t, \sigma^2 t)$).

Soit $C$ une constante positive. Le processus $\{S(t), t \ge 0\}$ défini par :
$$S(t) = C e^{X(t)}$$
est un \textbf{Mouvement Brownien Géométrique} (MBG).
\end{definitionbox}

\begin{remarquebox}[Point de Départ]
Si nous supposons que le MB $X(t)$ commence à $X(0)=0$, alors :
$$ S(0) = C e^{X(0)} = C e^0 = C $$
La constante $C$ est simplement le prix de départ $S(0)$. Nous écrirons donc toujours :
$$S(t) = S(0) e^{X(t)}$$
\end{remarquebox}

\subsubsection{Propriétés des Accroissements ("Log-Rendements")}

La propriété la plus importante concerne les rendements. Prenons le logarithme de $S(t)$ :
$$ \ln(S(t)) = \ln(S(0) e^{X(t)}) = \ln(S(0)) + \ln(e^{X(t)}) $$
$$ \ln(S(t)) = \ln(S(0)) + X(t) $$
Le logarithme du prix est bien un Mouvement Brownien (translaté par la constante $\ln(S(0))$).

Considérons maintenant le \textbf{rendement continu} (ou "log-rendement") entre deux dates, $y$ et $t$ (avec $t > y$) :
$$ \ln\left(\frac{S(t)}{S(y)}\right) = \ln(S(t)) - \ln(S(y)) $$
En utilisant notre équation :
$$ = \left[ \ln(S(0)) + X(t) \right] - \left[ \ln(S(0)) + X(y) \right] $$
$$ \ln\left(\frac{S(t)}{S(y)}\right) = X(t) - X(y) $$

\begin{theorembox}[Accroissements Géométriques, Indépendants et Stationnaires]
Le \textbf{log-rendement} $\ln(S(t)/S(y))$ d'un MBG est égal à l'accroissement $X(t) - X(y)$ du Mouvement Brownien sous-jacent.

Par conséquent (par définition du MB $X(t)$) :
\begin{enumerate}
    \item Le log-rendement $\ln(S(t)/S(y))$ est \textbf{indépendant} des valeurs passées du processus (avant $y$).
    \item Le log-rendement $\ln(S(t)/S(y))$ suit une \textbf{loi normale} :
    $$ \ln\left(\frac{S(t)}{S(y)}\right) \sim N\left(\mu(t-y), \sigma^2(t-y)\right) $$
\end{enumerate}
Dans cette formulation, $\mu$ est la dérive du log-rendement et $\sigma$ est la \textbf{volatilité}.
\end{theorembox}

\subsubsection{Calcul de l'Espérance $E[S(t)]$}

C'est un point crucial où l'intuition est souvent mise à l'épreuve. Nous voulons connaître le prix \textit{moyen} (espéré) de $S(t)$ dans le futur.

On pourrait penser que si $E[X(t)] = \mu t$, alors $E[S(t)]$ devrait être $S(0)e^{\mu t}$. C'est \textbf{faux}.
L'erreur est de croire que $E[e^{X(t)}] = e^{E[X(t)]}$. Ceci n'est vrai que si $X(t)$ est une constante, pas une variable aléatoire.

\begin{remarquebox}[Outil : Espérance d'une variable Log-Normale]
Si $Y$ est une variable aléatoire qui suit une loi normale, $Y \sim N(m, v)$ (où $m$ est la moyenne et $v$ est la variance), alors l'espérance de son exponentielle est donnée par la formule :
$$ E[e^Y] = e^{m + v/2} $$
\end{remarquebox}

Armés de cet outil, nous pouvons dériver l'espérance de $S(t)$.

\begin{examplebox}[Dérivation Détaillée de l'espérance]
\textbf{1. Objectif :} Calculer $E[S(t)]$.

\textbf{2. Point de départ :} $S(t) = S(0) e^{X(t)}$. (Posons $s = S(0)$).
$$ E[S(t)] = E[s \cdot e^{X(t)}] $$
Puisque $s$ est une constante, elle sort de l'espérance :
$$ E[S(t)] = s \cdot E[e^{X(t)}] $$

\textbf{3. Identifier la variable aléatoire :} La variable aléatoire est $Y = X(t)$.

\textbf{4. Trouver la distribution de cette variable :} Par définition du MB $X(t)$, nous savons que pour un temps $t$ fixé :
$$ X(t) \sim N(\mu t, \sigma^2 t) $$

\textbf{5. Identifier les paramètres de la loi normale :}
\begin{itemize}
    \item La moyenne de $X(t)$ est $m = \mu t$
    \item La variance de $X(t)$ est $v = \sigma^2 t$
\end{itemize}

\textbf{6. Appliquer l'outil (la formule $E[e^Y] = e^{m + v/2}$) :}
$$ E[e^{X(t)}] = e^{\left( m \right) + \left( v \right)/2} $$
$$ E[e^{X(t)}] = e^{\left( \mu t \right) + \left( \sigma^2 t \right)/2} $$

\textbf{7. Finaliser le calcul :}
$$ E[S(t)] = s \cdot E[e^{X(t)}] = s \cdot e^{\mu t + \sigma^2 t / 2} $$
En factorisant le $t$ dans l'exposant :
$$ E[S(t)] = s \cdot e^{(\mu + \sigma^2/2)t} $$
\end{examplebox}

\begin{intuitionbox}[Le "Boost" de la Volatilité sur la Moyenne]
Nous venons de trouver que le taux de croissance du prix \textit{moyen} (l'espérance) n'est pas $\mu$, mais $\alpha = \mu + \sigma^2/2$.

$$ E[S(t)] = S(0) \cdot e^{(\mu + \sigma^2/2)t} $$

Pourquoi ce terme $\sigma^2/2$ supplémentaire ?
\begin{itemize}
    \item $\mu$ est le taux de croissance du \textit{logarithme} du prix. C'est aussi le taux de croissance de la \textbf{médiane} (la trajectoire "du milieu").
    \item $\alpha = \mu + \sigma^2/2$ est le taux de croissance de la \textbf{moyenne} (l'espérance).
\end{itemize}
La fonction $f(x) = e^x$ est \textbf{convexe}. 
Cela signifie qu'une augmentation de $x$ a plus d'impact sur $e^x$ qu'une diminution de $x$ n'en a.

Quand $\sigma > 0$, le processus $X(t)$ fluctue.
\begin{itemize}
    \item Les baisses de $X(t)$ font baisser $S(t)$, mais les pertes sont "capées" à 0.
    \item Les hausses de $X(t)$ font monter $S(t)$, et les gains sont \textit{illimités} et amplifiés par l'exponentielle.
\end{itemize}
La volatilité ($\sigma^2$) crée une asymétrie : elle génère quelques scénarios de gains extrêmes qui sont si grands qu'ils tirent la \textit{moyenne} de tous les scénarios vers le haut.

C'est pourquoi $\text{Moyenne} > \text{Médiane}$ et le taux de croissance de la moyenne ($\mu + \sigma^2/2$) est supérieur au taux de croissance de la médiane ($\mu$).
\end{intuitionbox}

\subsubsection{Exemples Calculatoires Détaillés}

Appliquons ces concepts.

\begin{intuitionbox}[Paramètres de l'exemple]
Supposons qu'une action ait un prix initial $\textbf{S(0) = s = 100 €}$.
Elle suit un MBG défini par les paramètres de son processus $X(t)$ (log-rendement) :
\begin{itemize}
    \item Dérive du log (drift) : $\mu = 0.08$ (soit 8\% par an)
    \item Volatilité (écart-type du log) : $\sigma = 0.20$ (soit 20\% par an)
\end{itemize}
Nous avons donc $\text{Var}(X(t)) = \sigma^2 t = (0.20)^2 t = 0.04 t$.
\end{intuitionbox}

\begin{examplebox}[Problème 1 : Prix Espéré et Médian à 1 an]
\textbf{Question :} Quelle est la valeur \textit{espérée} (moyenne) et la valeur \textit{médiane} du prix de l'action dans un an ($t=1$) ?

\textbf{Solution (Espérance) :}
Nous utilisons la formule $E[S(t)] = s \cdot e^{(\mu + \sigma^2 / 2)t}$.
1.  Calculer le taux de croissance espéré :
    $$\alpha = \mu + \frac{\sigma^2}{2} = 0.08 + \frac{(0.20)^2}{2} = 0.08 + \frac{0.04}{2}$$
    $$\alpha = 0.08 + 0.02 = 0.10 \text{ (soit 10\% par an)}$$
2.  Appliquer ce taux pour $t=1$ :
    $$E[S(1)] = 100 \cdot e^{(0.10) \times 1} = 100 \cdot e^{0.1} \approx 100 \times 1.10517 = 110.52 \text{ €}$$
    
\textbf{Solution (Médiane) :}
La médiane d'une loi log-normale $e^Y$ est $e^{E[Y]}$.
$$\text{Médiane}[S(t)] = s \cdot e^{E[X(t)]} = s \cdot e^{\mu t}$$
1.  Utiliser le taux de croissance de la médiane, $\mu = 0.08$.
2.  Appliquer ce taux pour $t=1$ :
    $$\text{Médiane}[S(1)] = 100 \cdot e^{0.08 \times 1} = 100 \cdot e^{0.08} \approx 100 \times 1.08329 = 108.33 \text{ €}$$

\textbf{Conclusion :} Le prix \textit{moyen} attendu (110.52 €) est supérieur au prix \textit{médian} (108.33 €). 50\% des scénarios seront en dessous de 108.33 €, mais les 50\% au-dessus ont des gains tellement élevés qu'ils tirent la moyenne à 110.52 €.
\end{examplebox}

\begin{examplebox}[Problème 2 : Probabilité d'une Baisse]
\textbf{Question :} Quelle est la probabilité que l'action termine l'année ($t=1$) avec un prix \textit{inférieur} à son prix de départ de 100 € ?

\textbf{Solution :}
1.  Poser le problème : Nous cherchons $P(S(1) < 100)$.
2.  Traduire en "log-espace" (avec $X(t)$) :
    $$P(100 \cdot e^{X(1)} < 100) \implies P(e^{X(1)} < 1) \implies P(X(1) < \ln(1))$$
    $$P(X(1) < 0)$$
3.  Trouver la distribution de $X(1)$ :
    Nous savons que $X(1) \sim N(\mu t, \sigma^2 t)$ avec $t=1$.
    $$X(1) \sim N(0.08 \times 1, \ 0.20^2 \times 1) \implies X(1) \sim N(0.08, 0.04)$$
4.  Standardiser (Calculer le Z-score) :
    Nous cherchons $P(X(1) < 0)$ pour une loi normale de moyenne $m=0.08$ et d'écart-type $\sigma_{std}=\sqrt{0.04}=0.20$.
    $$Z = \frac{\text{Valeur} - \text{Moyenne}}{\text{Écart-type}} = \frac{0 - 0.08}{0.20} = -0.40$$
5.  Trouver la probabilité :
    $$P(Z < -0.40) \approx 0.3446 \text{ (en utilisant une table N(0,1))}$$

\textbf{Conclusion :} Il y a environ 34.5\% de chances que l'action soit en baisse à la fin de l'année, même si sa dérive $\mu$ est positive.
\end{examplebox}

\begin{examplebox}[Problème 3 : Intervalle de Confiance à 95\%]
\textbf{Question :} Trouver l'intervalle de confiance à 95\% pour le prix de l'action dans un an ($t=1$).

\textbf{Solution :}
Nous ne pouvons pas calculer l'intervalle directement sur $S(1)$ (car il n'est pas symétrique). Nous devons le calculer sur $\ln(S(1))$ (ou $X(1)$) puis convertir les bornes.

1.  Intervalle de confiance à 95\% pour $X(1)$ :
    Nous savons que $X(1) \sim N(0.08, 0.04)$. L'écart-type est $0.20$.
    Un IC à 95\% pour une loi normale est $\left[ m - 1.96 \cdot \sigma_{std}, m + 1.96 \cdot \sigma_{std} \right]$.
    \begin{itemize}
        \item Borne inf. $X_1$ : $0.08 - 1.96 \times 0.20 = 0.08 - 0.392 = -0.312$
        \item Borne sup. $X_1$ : $0.08 + 1.96 \times 0.20 = 0.08 + 0.392 = +0.472$
    \end{itemize}
    L'intervalle pour $X(1)$ est $[-0.312, 0.472]$.

2.  Convertir l'intervalle pour $S(1) = 100 \cdot e^{X(1)}$ :
    \begin{itemize}
        \item Borne inf. $S_1$ : $100 \cdot e^{-0.312} \approx 100 \times 0.7320 = 73.20 \text{ €}$
        \item Borne sup. $S_1$ : $100 \cdot e^{+0.472} \approx 100 \times 1.6032 = 160.32 \text{ €}$
    \end{itemize}

\textbf{Conclusion :} Nous sommes confiants à 95\% que le prix de l'action dans un an se situera entre 73.20 € et 160.32 €. Notez que cet intervalle n'est pas symétrique autour de la médiane (108.33 €) ou de la moyenne (110.52 €).
\end{examplebox}

% \newpage

\section{Appendice A: Séries de Taylor et Maclaurin}

\begin{definitionbox}[Séries de Taylor et Maclaurin]
Si une fonction $f$ est indéfiniment dérivable au voisinage d'un point $a$, sa \textbf{série de Taylor} centrée en $a$ est définie par :
$$ f(x) = \sum_{k=0}^{\infty} \frac{f^{(k)}(a)}{k!} (x-a)^k $$
où $f^{(k)}(a)$ est la $k$-ième dérivée de $f$ évaluée en $a$.
\newline
\newline
Dans le cas particulier où $\mathbf{a=0}$, la série est appelée une \textbf{série de Maclaurin}. C'est la forme la plus courante, car elle approxime les fonctions autour de l'origine.
\end{definitionbox}

\subsection{Construction pas à pas d'une série de Taylor}

\begin{intuitionbox}[La logique de la correspondance des dérivées]
L'objectif fondamental d'une série de Taylor est de construire un polynôme, $P(x)$, qui soit une "copie conforme" d'une fonction $f(x)$ autour d'un point $a$. Pour ce faire, on force le polynôme à avoir exactement les mêmes propriétés locales que la fonction : même valeur, même pente, même courbure, etc. Cela se traduit mathématiquement par une exigence : \textbf{la n-ième dérivée du polynôme en $a$ doit être égale à la n-ième dérivée de la fonction en $a$}, et ce pour tous les ordres $n$.

Prenons l'exemple de $f(x) = e^x$ et construisons sa série de Maclaurin (centrée en $a=0$), où $f^{(k)}(0)=1$ pour tout $k$.

\begin{enumerate}
    \item \textbf{Ordre 0 : Faire correspondre la valeur}
    \newline
    \textbf{Objectif :} Le polynôme $P_0(x)$ doit avoir la même valeur que $f(x)$ en $x=0$. On veut $P_0(0) = f(0)$.
    \newline
    \textbf{Solution :} On choisit le polynôme le plus simple, une constante : $P_0(x) = f(0)$. Pour $e^x$, $f(0)=1$, donc $\mathbf{P_0(x) = 1}$.
    \newline
    \textbf{Vérification :} $P_0(0) = 1$. L'objectif est atteint.

    \item \textbf{Ordre 1 : Faire correspondre la première dérivée}
    \newline
    \textbf{Objectif :} On veut un nouveau polynôme $P_1(x)$ qui préserve la correspondance précédente ($P_1(0) = f(0)$) ET qui a la même pente, c'est-à-dire $P_1'(0) = f'(0)$.
    \newline
    \textbf{Solution :} On ajoute un terme en $x$ à notre polynôme précédent : $P_1(x) = P_0(x) + c_1 x = 1 + c_1 x$.
    \newline
    \textbf{Vérification :}
    \begin{itemize}
        \item $P_1(0) = 1 + c_1(0) = 1$. La valeur correspond toujours, car le nouveau terme s'annule en 0.
        \item On dérive : $P_1'(x) = c_1$. Pour que les pentes correspondent en 0, il faut $P_1'(0) = c_1 = f'(0)$. Comme $f'(0)=1$, on doit choisir $\mathbf{c_1=1}$.
    \end{itemize}
    Notre polynôme est maintenant $\mathbf{P_1(x) = 1+x}$.

    \item \textbf{Ordre 2 : Faire correspondre la deuxième dérivée}
    \newline
    \textbf{Objectif :} On veut $P_2(x)$ tel que $P_2(0)=f(0)$, $P_2'(0)=f'(0)$ ET $P_2''(0)=f''(0)$.
    \newline
    \textbf{Solution :} On ajoute un terme en $x^2$ : $P_2(x) = P_1(x) + c_2 x^2 = 1 + x + c_2 x^2$.
    \newline
    \textbf{Vérification :}
    \begin{itemize}
        \item Les dérivées d'ordre 0 et 1 en $x=0$ ne sont pas affectées, car la dérivée de $c_2x^2$ (soit $2c_2x$) et le terme lui-même s'annulent en 0. Les objectifs précédents sont préservés.
        \item On dérive deux fois : $P_2'(x) = 1 + 2c_2x$ et $P_2''(x) = 2c_2$.
        \item Pour que les courbures correspondent, il faut $P_2''(0) = 2c_2 = f''(0)$. Comme $f''(0)=1$, on doit choisir $\mathbf{c_2 = 1/2}$.
    \end{itemize}
    Notre polynôme est $\mathbf{P_2(x) = 1+x+\frac{1}{2}x^2}$.

    \item \textbf{Le schéma général : L'importance de la factorielle}
    \newline
    Pour faire correspondre la $k$-ième dérivée, on ajoute un terme $c_k x^k$.
    \newline
    Quand on dérive $c_k x^k$ exactement $k$ fois, on obtient $c_k \times k!$.
    \newline
    Toutes les dérivées d'ordre inférieur s'annulent en $x=0$. On doit donc avoir :
    $$ P_k^{(k)}(0) = c_k \cdot k! = f^{(k)}(0) $$
    Cela nous donne la règle pour trouver chaque coefficient :
    $$ c_k = \frac{f^{(k)}(0)}{k!} $$
    C'est précisément le coefficient qui apparaît dans la formule de Taylor, et il est choisi pour cette unique raison : forcer la $k$-ième dérivée du polynôme à correspondre parfaitement à celle de la fonction au point de développement.
\end{enumerate}

\tcblower
\centering
\begin{tikzpicture}
    \begin{axis}[
        xlabel={$x$},
        ylabel={$y$},
        xmin=-2, xmax=2,
        ymin=-0.5, ymax=4,
        axis lines=middle,
        legend style={at={(0.05,0.95)}, anchor=north west, font=\small},
        grid=major,
        samples=150,
        domain=-2:2,
        height=9cm,
        width=\linewidth-1cm,
        tick label style={font=\tiny}
    ]
    
    \addplot[black, dashed, ultra thick] {exp(x)};
    \addlegendentry{$e^x$}

    \addplot[red, thick] {1};
    \addlegendentry{$P_0(x)=1$}

    \addplot[blue, thick] {1+x};
    \addlegendentry{$P_1(x)=1+x$}

    \addplot[green!70!black, thick] {1+x+x^2/2};
    \addlegendentry{$P_2(x)=1+x+\frac{x^2}{2!}$}

    \addplot[orange, thick] {1+x+x^2/2+x^3/6};
    \addlegendentry{$P_3(x)=1+x+\frac{x^2}{2!}+\frac{x^3}{3!}$}

    \end{axis}
\end{tikzpicture}
\par\small\textit{Visualisation de la construction progressive de la série de Maclaurin pour $e^x$.}
\end{intuitionbox}

\subsection{Intuition de la série de Taylor en un point quelconque $a$}

\begin{intuitionbox}[Construire une approximation loin de l'origine]
La série de Maclaurin est puissante, mais elle nous contraint à approximer une fonction uniquement autour de $x=0$. Que faire si l'on s'intéresse au comportement d'une fonction ailleurs, par exemple $f(x)=\ln(x)$ autour de $x=1$ (puisque $\ln(0)$ n'est pas défini) ? C'est là qu'intervient la série de Taylor générale.

L'objectif reste le même : construire un polynôme $P(x)$ qui est une "copie conforme" de $f(x)$ au point $a$. Pour cela, on force les dérivées du polynôme à correspondre à celles de la fonction en ce point $a$. La seule différence est que notre "variable" de base n'est plus $x$, mais l'écart par rapport au centre, c'est-à-dire $(x-a)$.

Prenons l'exemple de $f(x) = \ln(x)$ et construisons sa série centrée en $\mathbf{a=1}$.

\begin{enumerate}
    \item \textbf{Ordre 0 : Faire correspondre la valeur}
    \newline
    \textbf{Objectif :} $P_0(a) = f(a)$.
    \newline
    \textbf{Solution :} On calcule $f(1) = \ln(1) = 0$. Le polynôme est la constante $\mathbf{P_0(x) = 0}$.

    \item \textbf{Ordre 1 : Faire correspondre la pente}
    \newline
    \textbf{Objectif :} $P_1(a) = f(a)$ et $P_1'(a) = f'(a)$.
    \newline
    \textbf{Solution :} On ajoute un terme proportionnel à l'écart $(x-a)$ : $P_1(x) = f(a) + c_1 (x-a)$.
    \newline
    \textbf{Vérification :}
    \begin{itemize}
        \item $P_1(1) = 0 + c_1(1-1) = 0$. La valeur correspond.
        \item On dérive : $P_1'(x) = c_1$. On veut $P_1'(1) = c_1 = f'(1)$.
        \item La dérivée de $f(x)=\ln(x)$ est $f'(x) = 1/x$, donc $f'(1)=1$. On doit choisir $\mathbf{c_1=1}$.
    \end{itemize}
    Notre polynôme est $\mathbf{P_1(x) = (x-1)}$. C'est la tangente à $\ln(x)$ en $x=1$.

    \item \textbf{Ordre 2 : Faire correspondre la courbure}
    \newline
    \textbf{Objectif :} Les dérivées jusqu'à l'ordre 2 doivent correspondre en $a=1$.
    \newline
    \textbf{Solution :} On ajoute un terme en $(x-a)^2$ : $P_2(x) = (x-1) + c_2 (x-1)^2$.
    \newline
    \textbf{Vérification :}
    \begin{itemize}
        \item Les correspondances d'ordre 0 et 1 sont préservées.
        \item On dérive deux fois : $P_2'(x) = 1 + 2c_2(x-1)$ et $P_2''(x) = 2c_2$.
        \item On veut $P_2''(1) = 2c_2 = f''(1)$.
        \item La dérivée seconde de $f(x)$ est $f''(x) = -1/x^2$, donc $f''(1)=-1$. On choisit $\mathbf{c_2 = -1/2}$.
    \end{itemize}
    Notre polynôme est $\mathbf{P_2(x) = (x-1) - \frac{1}{2}(x-1)^2}$.

    \item \textbf{Le schéma général}
    \newline
    Le coefficient $c_k$ du terme $(x-a)^k$ est choisi pour faire correspondre la $k$-ième dérivée. La dérivation de $c_k(x-a)^k$ $k$ fois donne $c_k \cdot k!$. On impose donc $c_k \cdot k! = f^{(k)}(a)$, ce qui mène directement à la formule générale $c_k = \frac{f^{(k)}(a)}{k!}$.
\end{enumerate}

\tcblower
\centering
\begin{tikzpicture}
    \begin{axis}[
        xlabel={$x$},
        ylabel={$y$},
        xmin=-0.5, xmax=2.5,
        ymin=-2, ymax=1,
        axis lines=middle,
        legend style={at={(0.05,0.05)}, anchor=south west, font=\small},
        grid=major,
        samples=150,
        domain=0.01:2.5,
        height=9cm,
        width=\linewidth-1cm,
        tick label style={font=\tiny}
    ]
    
    \addplot[black, dashed, ultra thick] {ln(x)};
    \addlegendentry{$\ln(x)$}

    \addplot[red, thick, domain=-0.5:2.5] {0};
    \addlegendentry{$P_0(x)=0$}

    \addplot[blue, thick, domain=-0.5:2.5] {x-1};
    \addlegendentry{$P_1(x)=(x-1)$}

    \addplot[green!70!black, thick, domain=-0.5:2.5] {(x-1) - 0.5*(x-1)^2};
    \addlegendentry{$P_2(x)=(x-1)-\frac{(x-1)^2}{2}$}

    \end{axis}
\end{tikzpicture}
\par\small\textit{Approximation de $\ln(x)$ autour de $a=1$. Le polynôme "colle" à la fonction près de $x=1$.}
\end{intuitionbox}


\subsection{La Fonction Exponentielle ($e^x$)}

\begin{theorembox}[Série de Maclaurin pour $e^x$]
Pour tout nombre réel $x$, la fonction exponentielle peut s'écrire :
$$ e^x = \sum_{k=0}^{\infty} \frac{x^k}{k!} = 1 + x + \frac{x^2}{2!} + \frac{x^3}{3!} + \frac{x^4}{4!} + \cdots $$
\end{theorembox}

\begin{intuitionbox}[Visualiser la Croissance Exponentielle]
La fonction exponentielle est unique car elle est sa propre dérivée. Cela signifie que toutes ses informations locales (valeur, pente, courbure) en $a=0$ sont égales à \textbf{1}. La série pour $e^x$ est donc le polynôme le plus « pur », où chaque terme $x^k$ est simplement normalisé par $k!$. Le graphique ci-dessous montre comment les polynômes de Taylor convergent rapidement vers la véritable courbe exponentielle, illustrant sa croissance puissante.

\tcblower

\centering
\begin{tikzpicture}
    \begin{axis}[
        xlabel={$x$},
        ylabel={$y$},
        xmin=-3, xmax=3,
        ymin=-1, ymax=9,
        axis lines=middle,
        legend style={at={(0.05,0.95)}, anchor=north west, font=\small},
        grid=major,
        samples=150,
        domain=-3:3,
        height=9cm,
        width=\linewidth-1cm,
        tick label style={font=\tiny}
    ]
    
    \addplot[black, dashed, ultra thick] {exp(x)};
    \addlegendentry{$e^x$}

    \addplot[red, thick] {1+x};
    \addlegendentry{$T_1(x)$}

    \addplot[blue, thick] {1+x+x^2/2};
    \addlegendentry{$T_2(x)$}

    \addplot[green!70!black, thick] {1+x+x^2/2+x^3/6};
    \addlegendentry{$T_3(x)$}

    \addplot[orange, thick] {1+x+x^2/2+x^3/6+x^4/24};
    \addlegendentry{$T_4(x)$}

    \end{axis}
\end{tikzpicture}
\par\small\textit{Approximation de $e^x$ par ses polynômes de Maclaurin.}
\end{intuitionbox}

\begin{proofbox}
Soit $f(x) = e^x$. Pour tout entier $k \ge 0$, la $k$-ième dérivée est $f^{(k)}(x) = e^x$. En évaluant en $a=0$, on obtient $f^{(k)}(0) = e^0 = 1$ pour tout $k$. En appliquant la formule de Maclaurin :
$$ e^x = \sum_{k=0}^{\infty} \frac{f^{(k)}(0)}{k!} x^k = \sum_{k=0}^{\infty} \frac{1}{k!} x^k = 1 + x + \frac{x^2}{2} + \frac{x^3}{6} + \cdots $$
\end{proofbox}


\subsection{La Fonction Sinus ($\sin(x)$)}

\begin{theorembox}[Série de Maclaurin pour $\sin(x)$]
Pour tout nombre réel $x$ :
$$ \sin(x) = \sum_{k=0}^{\infty} (-1)^k \frac{x^{2k+1}}{(2k+1)!} = x - \frac{x^3}{3!} + \frac{x^5}{5!} - \frac{x^7}{7!} + \cdots $$
\end{theorembox}

\begin{intuitionbox}[Visualiser l'Oscillation du Sinus]
La série du sinus reflète ses propriétés fondamentales. En tant que fonction \textbf{impaire} ($ \sin(-x) = -\sin(x) $), son développement ne contient que des puissances \textbf{impaires} de $x$. Les signes alternés capturent sa nature oscillatoire. Le graphique ci-dessous montre comment l'ajout de termes permet au polynôme d'« épouser » la courbe du sinus sur un plus grand nombre de périodes.

\tcblower

\centering
\begin{tikzpicture}
    \begin{axis}[
        xlabel={$x$},
        ylabel={$y$},
        xmin=-2*pi, xmax=2*pi,
        ymin=-2.0, ymax=2.0,
        axis lines=middle,
        legend style={at={(0.5,1.15)}, anchor=south, font=\small, column sep=5pt},
        legend columns=3,
        grid=major,
        samples=200,
        domain=-2*pi:2*pi,
        height=9cm,
        width=\linewidth-1cm,
        tick label style={font=\tiny}
    ]
    \addplot[black, dashed, ultra thick] {sin(deg(x))};
    \addlegendentry{$\sin(x)$}
    \addplot[red, thick] {x};
    \addlegendentry{$T_1(x)$}
    \addplot[blue, thick] {x - (x^3)/6};
    \addlegendentry{$T_3(x)$}
    \addplot[green!70!black, thick] {x - (x^3)/6 + (x^5)/120};
    \addlegendentry{$T_5(x)$}
    \addplot[orange, thick] {x - (x^3)/6 + (x^5)/120 - (x^7)/5040};
    \addlegendentry{$T_7(x)$}
    \end{axis}
\end{tikzpicture}
\par\small\textit{Approximation de $\sin(x)$ par ses polynômes de Maclaurin.}
\end{intuitionbox}

\begin{proofbox}
Soit $f(x) = \sin(x)$. Les dérivées en $a=0$ suivent un cycle $(0, 1, 0, -1, \dots)$. Seuls les termes d'ordre impair ($2k+1$) sont non nuls, avec des valeurs de $(-1)^k$, ce qui donne la formule.
\end{proofbox}


\subsection{La Fonction Cosinus ($\cos(x)$)}

\begin{theorembox}[Série de Maclaurin pour $\cos(x)$]
Pour tout nombre réel $x$ :
$$ \cos(x) = \sum_{k=0}^{\infty} (-1)^k \frac{x^{2k}}{(2k)!} = 1 - \frac{x^2}{2!} + \frac{x^4}{4!} - \frac{x^6}{6!} + \cdots $$
\end{theorembox}

\begin{intuitionbox}[Visualiser la Symétrie du Cosinus]
En tant que fonction \textbf{paire} ($ \cos(-x) = \cos(x) $), la série du cosinus ne contient, de manière appropriée, que des puissances \textbf{paires} de $x$. Elle commence à 1 (son maximum) puis oscille, un comportement capturé par les signes alternés.

\tcblower

\centering
\begin{tikzpicture}
    \begin{axis}[
        xlabel={$x$},
        ylabel={$y$},
        xmin=-2*pi, xmax=2*pi,
        ymin=-2.0, ymax=2.0,
        axis lines=middle,
        legend style={at={(0.5,1.15)}, anchor=south, font=\small, column sep=5pt},
        legend columns=3,
        grid=major,
        samples=200,
        domain=-2*pi:2*pi,
        height=9cm,
        width=\linewidth-1cm,
        tick label style={font=\tiny}
    ]
    \addplot[black, dashed, ultra thick] {cos(deg(x))};
    \addlegendentry{$\cos(x)$}
    \addplot[red, thick] {1};
    \addlegendentry{$T_0(x)$}
    \addplot[blue, thick] {1 - x^2/2};
    \addlegendentry{$T_2(x)$}
    \addplot[green!70!black, thick] {1 - x^2/2 + x^4/24};
    \addlegendentry{$T_4(x)$}
    \addplot[orange, thick] {1 - x^2/2 + x^4/24 - x^6/720};
    \addlegendentry{$T_6(x)$}
    \end{axis}
\end{tikzpicture}
\par\small\textit{Approximation de $\cos(x)$ par ses polynômes de Maclaurin.}
\end{intuitionbox}

\begin{proofbox}
Soit $g(x) = \cos(x)$. Les dérivées en $a=0$ suivent un cycle $(1, 0, -1, 0, \dots)$. Seuls les termes d'ordre pair ($2k$) sont non nuls, avec des valeurs de $(-1)^k$, ce qui donne la formule.
\end{proofbox}


\subsection{Le Logarithme Népérien ($\ln(1+x)$)}

\begin{theorembox}[Série de Maclaurin pour $\ln(1+x)$]
Pour $|x| < 1$ :
$$ \ln(1+x) = \sum_{k=1}^{\infty} (-1)^{k-1} \frac{x^k}{k} = x - \frac{x^2}{2} + \frac{x^3}{3} - \frac{x^4}{4} + \cdots $$
\end{theorembox}

\begin{intuitionbox}[Visualiser l'Approximation Logarithmique]
Cette série est essentielle pour approximer les logarithmes près de 1. Contrairement aux fonctions précédentes, elle ne converge que pour $|x|<1$. Le graphique montre que l'approximation est excellente près de $x=0$ mais diverge rapidement lorsque $x$ s'approche de la frontière de convergence à $x=1$.

\tcblower

\centering
\begin{tikzpicture}
    \begin{axis}[
        xlabel={$x$},
        ylabel={$y$},
        xmin=-1.2, xmax=1.2,
        ymin=-4, ymax=2,
        axis lines=middle,
        legend style={at={(0.05,0.95)}, anchor=north west, font=\small},
        grid=major,
        samples=150,
        domain=-0.99:1, % Domain restricted for ln
        height=9cm,
        width=\linewidth-1cm,
        tick label style={font=\tiny}
    ]
    \addplot[black, dashed, ultra thick] {ln(1+x)};
    \addlegendentry{$\ln(1+x)$}
    
    \addplot[red, thick, domain=-1.2:1.2] {x};
    \addlegendentry{$T_1(x)$}

    \addplot[blue, thick, domain=-1.2:1.2] {x - x^2/2};
    \addlegendentry{$T_2(x)$}

    \addplot[green!70!black, thick, domain=-1.2:1.2] {x - x^2/2 + x^3/3};
    \addlegendentry{$T_3(x)$}
    
    \addplot[orange, thick, domain=-1.2:1.2] {x - x^2/2 + x^3/3 - x^4/4};
    \addlegendentry{$T_4(x)$}

    \end{axis}
\end{tikzpicture}
\par\small\textit{Approximation de $\ln(1+x)$ par ses polynômes de Maclaurin.}
\end{intuitionbox}

\begin{proofbox}
Soit $f(x) = \ln(1+x)$. Pour $k \ge 1$, la $k$-ième dérivée en $a=0$ est $f^{(k)}(0) = (-1)^{k-1} (k-1)!$. En substituant cela dans la formule de Maclaurin, le $(k-1)!$ au numérateur annule partiellement le $k!$ au dénominateur, laissant un $k$ en bas.
\end{proofbox}

\subsection{La Série Géométrique ($\frac{1}{1-x}$)}

\begin{theorembox}[Série de Maclaurin pour $\frac{1}{1-x}$]
Pour $|x| < 1$ :
$$ \frac{1}{1-x} = \sum_{k=0}^{\infty} x^k = 1 + x + x^2 + x^3 + \cdots $$
\end{theorembox}

\begin{intuitionbox}[Le Fondement de Nombreuses Séries]
Cette série, connue sous le nom de série géométrique, est l'un des développements en série de puissances les plus fondamentaux. Elle converge uniquement lorsque la valeur absolue de $x$ est inférieure à 1. Chaque coefficient est simplement 1, ce qui en fait la série de Maclaurin la plus simple. De nombreuses autres séries, comme celle de $\ln(1+x)$ ou de $\arctan(x)$, peuvent être dérivées de celle-ci par intégration ou substitution.
\end{intuitionbox}

\begin{proofbox}
Soit $f(x) = (1-x)^{-1}$. Les dérivées successives sont $f'(x) = 1(1-x)^{-2}$, $f''(x) = 2(1-x)^{-3}$, $f'''(x) = 6(1-x)^{-4}$, et ainsi de suite. La formule générale pour la $k$-ième dérivée est $f^{(k)}(x) = k!(1-x)^{-(k+1)}$. En évaluant en $a=0$, on obtient $f^{(k)}(0) = k!$. En substituant dans la formule de Maclaurin :
$$ \frac{1}{1-x} = \sum_{k=0}^{\infty} \frac{f^{(k)}(0)}{k!} x^k = \sum_{k=0}^{\infty} \frac{k!}{k!} x^k = \sum_{k=0}^{\infty} x^k $$
\end{proofbox}

\subsection{Exercices}

\begin{exercicebox}[Termes de base]
Trouvez les quatre premiers termes non nuls de la série de Maclaurin pour $f(x) = \cos(2x)$.
\end{exercicebox}

\begin{correctionbox}
On utilise la série connue de $\cos(u) = 1 - \frac{u^2}{2!} + \frac{u^4}{4!} - \frac{u^6}{6!} + \cdots$.
En substituant $u = 2x$, on obtient :
$$ \cos(2x) = 1 - \frac{(2x)^2}{2!} + \frac{(2x)^4}{4!} - \frac{(2x)^6}{6!} + \cdots $$
$$ \cos(2x) = 1 - \frac{4x^2}{2} + \frac{16x^4}{24} - \frac{64x^6}{5040} + \cdots $$
$$ \cos(2x) = 1 - 2x^2 + \frac{2}{3}x^4 - \frac{4}{315}x^6 + \cdots $$
Les quatre premiers termes sont $1$, $-2x^2$, $\frac{2}{3}x^4$ et $-\frac{4}{315}x^6$.
\end{correctionbox}

\begin{exercicebox}[Dérivation de séries]
Utilisez la série de Maclaurin de $\sin(x)$ pour trouver la série de Maclaurin de $\cos(x)$.
\end{exercicebox}

\begin{correctionbox}
On sait que $\frac{d}{dx}(\sin(x)) = \cos(x)$. On peut dériver la série de $\sin(x)$ terme à terme :
$$ \sin(x) = x - \frac{x^3}{3!} + \frac{x^5}{5!} - \frac{x^7}{7!} + \cdots $$
$$ \frac{d}{dx} \sin(x) = \frac{d}{dx} \left( x - \frac{x^3}{6} + \frac{x^5}{120} - \cdots \right) $$
$$ \cos(x) = 1 - \frac{3x^2}{6} + \frac{5x^4}{120} - \cdots = 1 - \frac{x^2}{2} + \frac{x^4}{24} - \cdots = 1 - \frac{x^2}{2!} + \frac{x^4}{4!} - \cdots $$
On retrouve bien la série de Maclaurin pour $\cos(x)$.
\end{correctionbox}

\begin{exercicebox}[Intégration de séries]
Trouvez la série de Maclaurin pour $\arctan(x)$ en intégrant la série de $\frac{1}{1+x^2}$.
\end{exercicebox}

\begin{correctionbox}
On part de la série géométrique $\frac{1}{1-u} = \sum_{k=0}^{\infty} u^k$. En posant $u = -x^2$, on obtient :
$$ \frac{1}{1+x^2} = \sum_{k=0}^{\infty} (-x^2)^k = \sum_{k=0}^{\infty} (-1)^k x^{2k} = 1 - x^2 + x^4 - x^6 + \cdots $$
Puisque $\int \frac{1}{1+x^2} dx = \arctan(x)$, on intègre la série terme à terme :
$$ \arctan(x) = \int (1 - x^2 + x^4 - \cdots) dx = C + x - \frac{x^3}{3} + \frac{x^5}{5} - \cdots $$
Comme $\arctan(0) = 0$, la constante d'intégration $C$ est nulle.
$$ \arctan(x) = \sum_{k=0}^{\infty} (-1)^k \frac{x^{2k+1}}{2k+1} $$
\end{correctionbox}

\begin{exercicebox}[Approximation de valeur]
Utilisez les trois premiers termes non nuls de la série de Maclaurin de $e^x$ pour approximer la valeur de $\sqrt{e}$.
\end{exercicebox}

\begin{correctionbox}
On veut approximer $\sqrt{e} = e^{0.5}$. La série est $e^x \approx 1 + x + \frac{x^2}{2!}$.
En posant $x=0.5$ :
$$ e^{0.5} \approx 1 + 0.5 + \frac{(0.5)^2}{2} = 1 + 0.5 + \frac{0.25}{2} = 1.5 + 0.125 = 1.625 $$
La valeur réelle est $e^{0.5} \approx 1.6487$. L'approximation est raisonnablement proche.
\end{correctionbox}

\begin{exercicebox}[Série de Taylor non centrée en 0]
Trouvez la série de Taylor pour $f(x) = \ln(x)$ centrée en $a=1$.
\end{exercicebox}

\begin{correctionbox}
On calcule les dérivées de $f(x)=\ln(x)$ et on les évalue en $a=1$.
$f(x) = \ln(x) \implies f(1) = 0$
$f'(x) = 1/x \implies f'(1) = 1$
$f''(x) = -1/x^2 \implies f''(1) = -1$
$f'''(x) = 2/x^3 \implies f'''(1) = 2$
$f^{(k)}(x) = (-1)^{k-1}(k-1)!/x^k \implies f^{(k)}(1) = (-1)^{k-1}(k-1)!$ pour $k \ge 1$.
La série de Taylor est :
$$ \ln(x) = \sum_{k=1}^{\infty} \frac{(-1)^{k-1}(k-1)!}{k!} (x-1)^k = \sum_{k=1}^{\infty} \frac{(-1)^{k-1}}{k} (x-1)^k $$
\end{correctionbox}

\begin{exercicebox}[Identifier une fonction]
Quelle fonction est représentée par la série de Maclaurin $ \sum_{k=0}^{\infty} \frac{(-1)^k x^{2k}}{(2k)!} $ ?
\end{exercicebox}

\begin{correctionbox}
Cette série est $1 - \frac{x^2}{2!} + \frac{x^4}{4!} - \frac{x^6}{6!} + \cdots$. Il s'agit de la série de Maclaurin de la fonction $\cos(x)$.
\end{correctionbox}

\begin{exercicebox}[Série binomiale]
Trouvez les trois premiers termes de la série de Maclaurin pour $f(x) = \sqrt{1+x}$.
\end{exercicebox}

\begin{correctionbox}
On utilise la série binomiale $(1+x)^\alpha$ avec $\alpha=1/2$.
$$ (1+x)^{1/2} = 1 + \alpha x + \frac{\alpha(\alpha-1)}{2!}x^2 + \cdots $$
$$ (1+x)^{1/2} = 1 + \frac{1}{2}x + \frac{\frac{1}{2}(\frac{1}{2}-1)}{2}x^2 + \cdots $$
$$ (1+x)^{1/2} = 1 + \frac{1}{2}x + \frac{\frac{1}{2}(-\frac{1}{2})}{2}x^2 + \cdots = 1 + \frac{1}{2}x - \frac{1}{8}x^2 + \cdots $$
\end{correctionbox}

\begin{exercicebox}[Multiplication de séries]
Trouvez les termes jusqu'à $x^3$ pour la série de Maclaurin de $f(x) = e^x \sin(x)$.
\end{exercicebox}

\begin{correctionbox}
On multiplie les développements de $e^x$ et $\sin(x)$ :
$$ e^x \sin(x) = \left(1 + x + \frac{x^2}{2} + \frac{x^3}{6} + \cdots\right) \left(x - \frac{x^3}{6} + \cdots\right) $$
On collecte les termes par puissance croissante :
\begin{itemize}
    \item Terme en $x$ : $1 \cdot x = x$
    \item Terme en $x^2$ : $x \cdot x = x^2$
    \item Terme en $x^3$ : $1 \cdot (-\frac{x^3}{6}) + \frac{x^2}{2} \cdot x = -\frac{x^3}{6} + \frac{x^3}{2} = \frac{2x^3}{6} = \frac{x^3}{3}$
\end{itemize}
Donc, $e^x \sin(x) = x + x^2 + \frac{x^3}{3} + \cdots$.
\end{correctionbox}

\begin{exercicebox}[Calcul de limite]
Évaluez la limite suivante en utilisant les séries de Maclaurin : $ \lim_{x \to 0} \frac{\sin(x) - x}{x^3} $.
\end{exercicebox}

\begin{correctionbox}
On remplace $\sin(x)$ par son développement :
$$ \lim_{x \to 0} \frac{(x - \frac{x^3}{3!} + \frac{x^5}{5!} - \cdots) - x}{x^3} $$
$$ = \lim_{x \to 0} \frac{-\frac{x^3}{6} + \frac{x^5}{120} - \cdots}{x^3} $$
$$ = \lim_{x \to 0} \left(-\frac{1}{6} + \frac{x^2}{120} - \cdots\right) = -\frac{1}{6} $$
\end{correctionbox}

\begin{exercicebox}[Fonction hyperbolique]
Trouvez la série de Maclaurin pour le cosinus hyperbolique, $\cosh(x) = \frac{e^x + e^{-x}}{2}$.
\end{exercicebox}

\begin{correctionbox}
On utilise les séries de $e^x$ et $e^{-x}$ :
$e^x = 1 + x + \frac{x^2}{2!} + \frac{x^3}{3!} + \cdots$
$e^{-x} = 1 - x + \frac{x^2}{2!} - \frac{x^3}{3!} + \cdots$
En les additionnant, les termes de puissance impaire s'annulent :
$e^x + e^{-x} = 2 + 2\frac{x^2}{2!} + 2\frac{x^4}{4!} + \cdots$
En divisant par 2 :
$$ \cosh(x) = 1 + \frac{x^2}{2!} + \frac{x^4}{4!} + \frac{x^6}{6!} + \cdots = \sum_{k=0}^{\infty} \frac{x^{2k}}{(2k)!} $$
\end{correctionbox}

\begin{exercicebox}[Coefficient de Taylor]
Soit $f(x) = \frac{x^2}{1+x^3}$. Trouvez la valeur de la 8ème dérivée en zéro, $f^{(8)}(0)$.
\end{exercicebox}

\begin{correctionbox}
On sait que le coefficient du terme $x^k$ dans une série de Maclaurin est $\frac{f^{(k)}(0)}{k!}$.
On développe $f(x)$ :
$$ f(x) = x^2 \cdot \frac{1}{1-(-x^3)} = x^2 \sum_{n=0}^{\infty} (-x^3)^n = x^2 \sum_{n=0}^{\infty} (-1)^n x^{3n} = \sum_{n=0}^{\infty} (-1)^n x^{3n+2} $$
On cherche le terme en $x^8$. On doit avoir $3n+2=8$, ce qui donne $3n=6$, soit $n=2$.
Le coefficient de $x^8$ est donc $(-1)^2 = 1$.
On a alors $\frac{f^{(8)}(0)}{8!} = 1$, ce qui implique $f^{(8)}(0) = 8! = 40320$.
\end{correctionbox}

\begin{exercicebox}[Approximation d'intégrale]
Estimez la valeur de $\int_0^1 \sin(x^2) dx$ en utilisant les deux premiers termes non nuls de la série de Maclaurin de la fonction à intégrer.
\end{exercicebox}

\begin{correctionbox}
On part de $\sin(u) = u - \frac{u^3}{6} + \cdots$. On pose $u=x^2$ :
$$ \sin(x^2) = x^2 - \frac{(x^2)^3}{6} + \cdots = x^2 - \frac{x^6}{6} + \cdots $$
On intègre ce polynôme de 0 à 1 :
$$ \int_0^1 \left(x^2 - \frac{x^6}{6}\right) dx = \left[ \frac{x^3}{3} - \frac{x^7}{42} \right]_0^1 $$
$$ = \left(\frac{1}{3} - \frac{1}{42}\right) - 0 = \frac{14 - 1}{42} = \frac{13}{42} \approx 0.3095 $$
\end{correctionbox}

\begin{exercicebox}[Un autre centre]
Trouvez les trois premiers termes de la série de Taylor pour $f(x) = \frac{1}{x}$ centrée en $a=2$.
\end{exercicebox}

\begin{correctionbox}
On calcule les dérivées et on les évalue en $a=2$.
$f(x) = x^{-1} \implies f(2) = 1/2$
$f'(x) = -x^{-2} \implies f'(2) = -1/4$
$f''(x) = 2x^{-3} \implies f''(2) = 2/8 = 1/4$
La série commence par :
$$ f(x) \approx f(2) + f'(2)(x-2) + \frac{f''(2)}{2!}(x-2)^2 $$
$$ f(x) \approx \frac{1}{2} - \frac{1}{4}(x-2) + \frac{1/4}{2}(x-2)^2 = \frac{1}{2} - \frac{1}{4}(x-2) + \frac{1}{8}(x-2)^2 $$
\end{correctionbox}

\begin{exercicebox}[Combinaison de séries]
Trouvez le terme en $x^4$ du développement de Maclaurin de $f(x) = \ln(1-x^2)$.
\end{exercicebox}

\begin{correctionbox}
On utilise la série de $\ln(1+u) = u - \frac{u^2}{2} + \frac{u^3}{3} - \frac{u^4}{4} + \cdots$.
On substitue $u = -x^2$ :
$$ \ln(1-x^2) = (-x^2) - \frac{(-x^2)^2}{2} + \frac{(-x^2)^3}{3} - \frac{(-x^2)^4}{4} + \cdots $$
$$ = -x^2 - \frac{x^4}{2} - \frac{x^6}{3} - \frac{x^8}{4} - \cdots $$
Le terme en $x^4$ est $-\frac{x^4}{2}$.
\end{correctionbox}

\begin{exercicebox}[Application en physique]
En relativité restreinte, l'énergie cinétique d'une particule est $K = mc^2(\gamma - 1)$, où $\gamma = (1-v^2/c^2)^{-1/2}$. Montrez que pour des vitesses faibles ($v \ll c$), cette formule se réduit à la formule classique $K \approx \frac{1}{2}mv^2$.
\end{exercicebox}

\begin{correctionbox}
On utilise le développement binomial $(1+x)^\alpha$ avec $x = -v^2/c^2$ et $\alpha = -1/2$.
$$ \gamma = \left(1 - \frac{v^2}{c^2}\right)^{-1/2} \approx 1 + \alpha x = 1 + \left(-\frac{1}{2}\right)\left(-\frac{v^2}{c^2}\right) = 1 + \frac{1}{2}\frac{v^2}{c^2} $$
On substitue ce résultat dans la formule de l'énergie :
$$ K = mc^2(\gamma - 1) \approx mc^2 \left( \left(1 + \frac{1}{2}\frac{v^2}{c^2}\right) - 1 \right) $$
$$ K \approx mc^2 \left( \frac{1}{2}\frac{v^2}{c^2} \right) = \frac{1}{2}mv^2 $$
On retrouve bien l'énergie cinétique classique comme approximation de premier ordre.
\end{correctionbox}

\section{Exemples de Code}

\begin{examplebox}[Simulation d'un lancer de dé]
On peut simuler $n$ lancers d'un dé équilibré à 6 faces en utilisant la bibliothèque \texttt{random} de Python.

\begin{codecell}
import random
import numpy
\end{codecell}

\begin{outputcell}
>> "vamonos"
\end{outputcell}

\end{examplebox}

\begin{definitionbox}[Variable Aléatoire]
Une variable aléatoire est une fonction qui associe un nombre réel à chaque résultat possible d'une expérience aléatoire.
\end{definitionbox}

\begin{codecell}
import math
import random

def poisson_knuth(lmbda: float) -> int:
  """
  Simule une variable aleatoire suivant une loi de Poisson()
  en utilisant l algorithme de Knuth.
  """
  L = math.exp(-lmbda)
  k = 0
  p = 1.0

  while p > L:
  k += 1
  p *= random.random()

  return k - 1
\end{codecell}



\end{document}