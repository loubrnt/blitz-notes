\newpage
\section{Distributions Multivariées et Concepts Associés}

\subsection{Distributions Jointes et Marginales}

Jusqu'à présent, nous avons étudié les variables aléatoires isolément. Nous allons maintenant examiner comment analyser les relations entre *plusieurs* variables aléatoires.

\begin{definitionbox}[Distribution Jointe (Cas Discret)]
Pour deux variables aléatoires discrètes $X$ et $Y$, la \textbf{distribution jointe} (ou loi jointe) spécifie la probabilité de chaque paire d'issues. La fonction de masse de probabilité jointe (joint PMF) est :
$$P(X=x, Y=y)$$
Si $X$ prend ses valeurs dans un ensemble $S$ et $Y$ dans un ensemble $T$, alors la somme de toutes les probabilités jointes est égale à 1 :
$$\sum_{x \in S} \sum_{y \in T} P(X=x, Y=y) = 1$$
\end{definitionbox}

Cette loi jointe est la "carte" complète de toutes les issues possibles.

\begin{intuitionbox}
La distribution jointe est la "carte" complète de toutes les issues possibles. Elle répond à la question : "Quelle est la probabilité que $X$ prenne cette valeur ET que $Y$ prenne cette autre valeur en même temps ?". Si vous imaginez un tableau à double entrée pour $X$ et $Y$, la loi jointe est l'ensemble de toutes les probabilités à l'intérieur du tableau.
\end{intuitionbox}

Cette "carte" complète contient toutes les informations. Si nous ne nous intéressons qu'à une seule variable, nous pouvons la "réduire" en calculant sa distribution marginale.

\begin{definitionbox}[Distribution Marginale]
À partir de la distribution jointe, on peut obtenir la distribution \textbf{marginale} (ou loi marginale) de chaque variable. Pour obtenir la probabilité que $X$ prenne une valeur $x$, on somme sur toutes les valeurs possibles de $Y$ :
$$P(X=x) = \sum_{y \in T} P(X=x, Y=y)$$
\end{definitionbox}

Visuellement, cela correspond à "écraser" le tableau de probabilités sur un seul de ses axes.

\begin{intuitionbox}
Les distributions marginales sont les "ombres" ou "projections" de la carte jointe sur un seul axe. Si la loi jointe est un tableau, les lois marginales sont les totaux de chaque ligne et de chaque colonne, que l'on écrirait "dans la marge" du tableau. Elles nous disent la probabilité d'une issue pour $X$ sans se soucier de ce qu'il advient de $Y$.
\end{intuitionbox}

L'exemple le plus simple est le lancer de deux dés.

\begin{examplebox}[Lois jointe et marginale]
On lance un dé rouge ($X$) et un dé bleu ($Y$). Il y a 36 issues, chacune avec une probabilité de 1/36.
\textbf{Loi jointe} : $P(X=x, Y=y) = 1/36$ pour tout $x, y \in \{1, \dots, 6\}$.
Par exemple, $P(X=2, Y=5) = 1/36$.

\textbf{Loi marginale} de $X$ : Cherchons $P(X=2)$. C'est la probabilité d'obtenir 2 sur le dé rouge, quel que soit le résultat du bleu.
$$P(X=2) = \sum_{y=1}^6 P(X=2, Y=y)$$
$$P(X=2) = P(X=2,Y=1) + \dots + P(X=2,Y=6)$$
$$P(X=2) = \frac{1}{36} + \frac{1}{36} + \frac{1}{36} + \frac{1}{36} + \frac{1}{36} + \frac{1}{36} = \frac{6}{36} = \frac{1}{6}$$
Ceci est bien la loi d'un seul dé.
\end{examplebox}

\subsection{Espérance d'une fonction de deux variables}

Maintenant que nous avons la loi jointe (la carte des probabilités), nous pouvons l'utiliser pour calculer l'espérance de n'importe quelle fonction qui dépend des deux variables, $g(X,Y)$.

\begin{definitionbox}[Espérance d'une fonction $g(X,Y)$]
L'espérance d'une fonction $g(X,Y)$ de deux variables aléatoires discrètes $X$ et $Y$ est une généralisation du théorème de transfert (LOTUS) :
$$E[g(X,Y)] = \sum_{x \in S} \sum_{y \in T} g(x,y) P(X=x, Y=y)$$
\end{definitionbox}

C'est la moyenne de $g$, pondérée par les probabilités jointes.

\begin{intuitionbox}
C'est la valeur moyenne attendue de la fonction $g$. Pour la calculer, on prend chaque résultat possible de $g(x,y)$, on le pondère par la probabilité que cette combinaison $(x,y)$ se produise (donnée par la loi jointe), et on somme le tout.
\end{intuitionbox}

Le cas le plus important de $g(X,Y)$ est la somme $X+Y$.

\begin{examplebox}{Espérance de $E[X+Y]$}
Avec nos deux dés, calculons l'espérance de la somme $S = X + Y$.  
La fonction est $g(X,Y) = X + Y$.

\[
E[X+Y] = \sum_{x=1}^6 \sum_{y=1}^6 (x+y)\, P(X=x, Y=y)
\]
\[
E[X+Y] = \sum_{x=1}^6 \sum_{y=1}^6 (x+y)\, \frac{1}{36}
\]

Plutôt que de faire ce long calcul, on peut utiliser la linéarité de l'espérance (qui est un cas particulier de ce théorème) :

\[
E[X+Y] = E[X] + E[Y] = 3.5 + 3.5 = 7.
\]
\end{examplebox}

\subsection{Covariance et Corrélation}

La linéarité $E[X+Y] = E[X] + E[Y]$ est un outil puissant. Mais l'espérance ne nous dit rien sur la *relation* entre $X$ et $Y$. Pour cela, nous introduisons la covariance.

\begin{definitionbox}[Covariance]
La \textbf{covariance} entre deux variables aléatoires $X$ et $Y$, avec pour moyennes respectives $\mu_X$ et $\mu_Y$, mesure la façon dont elles varient ensemble.
$$\text{Cov}(X,Y) = E[(X - \mu_X)(Y - \mu_Y)]$$
\end{definitionbox}

Elle mesure la direction de leur relation.

\begin{intuitionbox}
La covariance est positive si les variables ont tendance à "bouger" dans la même direction (quand $X$ est au-dessus de sa moyenne, $Y$ a tendance à l'être aussi). Elle est négative si elles bougent en sens opposé (quand $X$ est au-dessus de sa moyenne, $Y$ a tendance à être en dessous). Si elle est nulle, il n'y a pas de tendance linéaire entre elles.
\end{intuitionbox}

La définition $E[(X - \mu_X)(Y - \mu_Y)]$ est bonne pour l'intuition, mais difficile à calculer. Une formule alternative est presque toujours utilisée.

\begin{theorembox}[Formule de calcul de la covariance]
Une formule computationnelle plus simple pour la covariance est :
$$\text{Cov}(X,Y) = E[XY] - E[X]E[Y]$$
\end{theorembox}

La preuve est une simple expansion algébrique.

\begin{proofbox}
Soit $\mu_X = E[X]$ et $\mu_Y = E[Y]$. On part de la définition :
\begin{align*}
\text{Cov}(X,Y) &= E[(X - \mu_X)(Y - \mu_Y)] \\
&= E[XY - X\mu_Y - Y\mu_X + \mu_X\mu_Y] \quad \text{(On développe)} \\
&= E[XY] - E[X\mu_Y] - E[Y\mu_X] + E[\mu_X\mu_Y] \quad \text{(Par linéarité)} \\
&= E[XY] - \mu_Y E[X] - \mu_X E[Y] + \mu_X\mu_Y \quad \text{(Les moyennes sont des constantes)} \\
&= E[XY] - \mu_Y \mu_X - \mu_X \mu_Y + \mu_X\mu_Y \\
&= E[XY] - \mu_X \mu_Y \\
&= E[XY] - E[X]E[Y]
\end{align*}
\end{proofbox}

Voyons cette formule en action.

\begin{examplebox}[Calcul de covariance]
\textbf{Cas 1 : Dés indépendants}. $X$ et $Y$ sont les résultats de deux dés. $E[X]=3.5, E[Y]=3.5$.
Calculons $E[XY]$. Puisqu'ils sont indépendants, $E[XY] = E[X]E[Y] = 3.5 \times 3.5 = 12.25$.
$\text{Cov}(X,Y) = E[XY] - E[X]E[Y] = 12.25 - 12.25 = 0$.
La covariance est nulle, ce qui est attendu pour des variables indépendantes.

\textbf{Cas 2 : Variables dépendantes}. Soit $X$ un lancer de dé, et $Y = 2X$.
$E[X] = 3.5$. $E[Y] = E[2X] = 2E[X] = 7$.
$E[XY] = E[X \cdot 2X] = E[2X^2] = 2 E[X^2]$.
On sait que $E[X^2] = \frac{1^2+...+6^2}{6} = 91/6$.
$E[XY] = 2(91/6) = 91/3$.
$\text{Cov}(X,Y) = E[XY] - E[X]E[Y] = \frac{91}{3} - (3.5)(7) = \frac{91}{3} - 24.5 = 30.33... - 24.5 \approx 5.833$.
La covariance est positive, ce qui est logique : si $X$ est grand, $Y$ l'est aussi.
\end{examplebox}

La covariance est un bon indicateur de la direction de la relation, mais sa magnitude est difficile à interpréter. Pour cela, nous la normalisons.

\begin{definitionbox}[Corrélation]
La \textbf{corrélation} (ou coefficient de corrélation de Pearson, $r$) est une version normalisée de la covariance, qui se situe toujours entre -1 et 1.
$$\text{Corr}(X,Y) = r = \frac{\text{Cov}(X,Y)}{\sqrt{\text{Var}(X)\text{Var}(Y)}} = \frac{\text{Cov}(X,Y)}{\sigma_X \sigma_Y}$$
\end{definitionbox}

La corrélation résout le problème des unités.

\begin{intuitionbox}
Le problème de la covariance est qu'elle dépend des unités de $X$ et $Y$ (par ex., $\text{kg} \cdot \text{cm}$). Si vous changez les unités (grammes et mètres), la valeur de la covariance change, même si la relation est identique. La corrélation résout ce problème : elle est sans unité. Un coefficient de +1 indique une relation linéaire positive parfaite, -1 une relation linéaire négative parfaite, et 0 une absence de relation linéaire.
\end{intuitionbox}

Cette normalisation se comprend mieux en voyant la corrélation comme une covariance de variables standardisées.

\begin{intuitionbox}[Interprétation de la formule]
On peut voir la corrélation de Pearson comme un processus en 3 étapes :
\begin{enumerate}
    \item \textbf{Centrer les variables :} On calcule l'écart de chaque valeur à sa moyenne ($x_i - \bar{x}$ et $y_i - \bar{y}$). Cela élimine "l'effet de base" (ex: une personne de 180cm vs 170cm ; la moyenne change mais les écarts relatifs restent les mêmes).
    \item \textbf{Normaliser les variables :} On divise chaque écart par l'écart-type de sa variable ($z_{xi} = (x_i - \bar{x})/\sigma_X$ et $z_{yi} = (y_i - \bar{y})/\sigma_Y$). Ces nouvelles variables $Z_X$ et $Z_Y$ sont \textbf{standardisées} : elles ont une moyenne de 0, un écart-type de 1, et sont sans unité.
    \item \textbf{Calculer la covariance des variables standardisées :} La corrélation n'est rien d'autre que la covariance de ces deux nouvelles variables standardisées : $r = \text{Cov}(Z_X, Z_Y)$.
\end{enumerate}
Parce que les deux variables sont maintenant sur la même échelle (écart-type de 1), leur covariance (la corrélation) ne peut pas dépasser 1 en valeur absolue.
\end{intuitionbox}

Reprenons notre exemple de dépendance parfaite :

\begin{examplebox}[Calcul de corrélation]
Reprenons l'exemple $Y=2X$, où $X$ est un lancer de dé.
On a $\text{Cov}(X,Y) = 5.833... = 35/6$.
$\text{Var}(X) = E[X^2] - E[X]^2 = 91/6 - (3.5)^2 = 35/12$.
$\text{Var}(Y) = \text{Var(2X)} = 2^2 \text{Var}(X) = 4(35/12) = 35/3$.
$\sigma_X \sigma_Y = \sqrt{35/12} \cdot \sqrt{35/3} = \sqrt{(35 \cdot 35) / (12 \cdot 3)} = \sqrt{35^2 / 36} = 35/6$.
$$\text{Corr}(X,Y) = \frac{\text{Cov}(X,Y)}{\sigma_X \sigma_Y} = \frac{35/6}{35/6} = 1$$
La corrélation est de 1, ce qui est parfait : $Y$ est une fonction linéaire parfaite de $X$.
\end{examplebox}

\subsection{Linéarité de la Covariance}

Tout comme l'espérance, la covariance possède d'importantes propriétés de linéarité qui simplifient les calculs.

\begin{definitionbox}[Linéarité de la Covariance]
Pour des variables aléatoires $X, Y, Z$ et des constantes $a, b, c$ :
\begin{align*}
\text{Cov}(aX + bY + c, Z) &= a\text{Cov}(X, Z) + b\text{Cov}(Y, Z) \\
\text{Cov}(X, aY + bZ + c) &= a\text{Cov}(X, Y) + b\text{Cov}(X, Z)
\end{align*}
La covariance est linéaire pour chaque argument (elle est \textbf{bilinéaire}). Les constantes additives disparaissent.
\end{definitionbox}

\subsection{Résultats sur la Corrélation}

La propriété la plus importante de la corrélation, qui découle de sa normalisation, est qu'elle est bornée.

\begin{theorembox}[Bornes du Coefficient de Corrélation de Pearson]
Pour toutes variables aléatoires $X$ et $Y$, le coefficient de corrélation $\text{Corr}(X,Y)$ est borné :
$$-1 \le \text{Corr}(X,Y) \le 1$$
De plus, si $\text{Corr}(X,Y) = \pm 1$, alors il existe des constantes $a$ et $b$ telles que $Y = aX + b$, indiquant une relation linéaire parfaite.
\end{theorembox}

La preuve de ces bornes repose sur le fait que la variance est toujours positive.

\begin{proofbox}[Démonstration des Bornes de la Corrélation]
La preuve repose sur le fait que la variance d'une variable aléatoire est toujours positive ou nulle.

\textbf{Étape 1 : Variables Standardisées}
On définit les versions standardisées de $X$ et $Y$ :
$$X^* = \frac{X - \mu_X}{\sigma_X} \quad ; \quad Y^* = \frac{Y - \mu_Y}{\sigma_Y}$$
Par construction, $E[X^*]=E[Y^*]=0$ et $\text{Var}(X^*) = \text{Var}(Y^*) = 1$.

\textbf{Étape 2 : Covariance des variables standardisées}
Calculons la covariance de $X^*$ et $Y^*$, qui est, par définition, la corrélation de $X$ et $Y$.
\begin{align*}
\text{Cov}(X^*, Y^*) &= \text{Cov}\left( \frac{X - \mu_X}{\sigma_X}, \frac{Y - \mu_Y}{\sigma_Y} \right) \\
&= \frac{1}{\sigma_X \sigma_Y} \text{Cov}(X - \mu_X, Y - \mu_Y) \\
&= \frac{1}{\sigma_X \sigma_Y} \text{Cov}(X, Y) \\
&= \text{Corr}(X,Y)
\end{align*}

\textbf{Étape 3 : Variance de la somme et de la différence}
Considérons la variance de la somme et de la différence de ces variables standardisées.
$$\text{Var}(X^* + Y^*) = \text{Var}(X^*) + \text{Var}(Y^*) + 2\text{Cov}(X^*, Y^*)$$
$$\text{Var}(X^* + Y^*) = 1 + 1 + 2\text{Corr}(X,Y) = 2 + 2\text{Corr}(X,Y)$$
De même :
$$\text{Var}(X^* - Y^*) = \text{Var}(X^*) + \text{Var}(Y^*) - 2\text{Cov}(X^*, Y^*)$$
$$\text{Var}(X^* - Y^*) = 1 + 1 - 2\text{Corr}(X,Y) = 2 - 2\text{Corr}(X,Y)$$

\textbf{Étape 4 : La variance est toujours $\ge 0$}
La variance d'une variable aléatoire ne peut pas être négative.
$$\text{Var}(X^* + Y^*) \ge 0 \implies 2 + 2\text{Corr}(X,Y) \ge 0 \implies \text{Corr}(X,Y) \ge -1$$
$$\text{Var}(X^* - Y^*) \ge 0 \implies 2 - 2\text{Corr}(X,Y) \ge 0 \implies \text{Corr}(X,Y) \le 1$$
Ceci nous donne le résultat final :
$$-1 \le \text{Corr}(X,Y) \le 1$$
\end{proofbox}

\subsection{Standardisation et Non-Corrélation}

Le processus de 'standardisation' utilisé dans la preuve de la corrélation et dans l'intuition est un concept fondamental en soi.

\begin{definitionbox}[Variable Centrée Réduite]
Soit $X$ une variable aléatoire avec :
\begin{itemize}
    \item moyenne $\mu_X = E[X]$
    \item écart-type $\sigma_X = \sqrt{\operatorname{Var}(X)} > 0$
\end{itemize}
On définit sa version \textbf{centrée réduite} (standardisée) $Z$ par :
$$Z = \frac{X - \mu_X}{\sigma_X}$$
Alors, $Z$ a les propriétés suivantes :
\begin{enumerate}
    \item \textbf{Centrée (moyenne nulle)} :
    \begin{align*}
    E[Z] &= E\left[\frac{X - \mu_X}{\sigma_X}\right] \\
         &= \frac{1}{\sigma_X} E[X - \mu_X] \quad (\text{par linéarité, } \sigma_X \text{ est une constante}) \\
         &= \frac{1}{\sigma_X} (E[X] - E[\mu_X]) \\
         &= \frac{1}{\sigma_X} (E[X] - \mu_X) \quad (\text{car } \mu_X \text{ est une constante}) \\
         &= \frac{\mu_X - \mu_X}{\sigma_X} = 0
    \end{align*}
    \item \textbf{Réduite (écart-type égal à 1)} :
    \begin{align*}
    \operatorname{Var}(Z) &= \operatorname{Var}\left(\frac{X - \mu_X}{\sigma_X}\right) \\
         &= \left(\frac{1}{\sigma_X}\right)^2 \operatorname{Var}(X - \mu_X) \quad (\text{propriété } \operatorname{Var}(aY) = a^2 \operatorname{Var}(Y)) \\
         &= \frac{1}{\sigma_X^2} \operatorname{Var}(X) \quad (\text{propriété } \operatorname{Var}(Y+b) = \operatorname{Var}(Y)) \\
         &= \frac{1}{\sigma_X^2} \cdot \sigma_X^2 = 1
    \end{align*}
    L'écart-type est donc $\sigma_Z = \sqrt{\operatorname{Var}(Z)} = \sqrt{1} = 1$.
\end{enumerate}
\end{definitionbox}

Cette transformation permet de comparer des variables sur des échelles différentes.

\begin{intuitionbox}[Que signifie centrer-réduire ?]
Standardiser une variable se fait en deux temps, comme le montre la formule $Z = \frac{X - \mu_X}{\sigma_X}$ :
\begin{enumerate}
    \item \textbf{Centrer ($X - \mu_X$)} : C'est la première étape. On soustrait la moyenne $\mu_X$. Cela revient à "déplacer" la distribution pour que son centre de gravité (sa moyenne) soit maintenant à 0. On ne regarde plus les valeurs brutes $X$, mais leurs \textbf{écarts} par rapport à la moyenne. (Propriété 1 : $E[Z]=0$)
    \item \textbf{Réduire ($... / \sigma_X$)} : C'est la deuxième étape. On divise ces écarts par l'écart-type $\sigma_X$. Cela revient à changer d'unité de mesure. L'ancienne unité (kg, cm, points...) est remplacée par une nouvelle unité universelle : "le nombre d'écarts-types". (Propriété 2 : $\text{Var}(Z)=1$)
\end{enumerate}
Au final, une variable $Z$ avec une valeur de 1.5 signifie "cette observation est 1.5 écarts-types au-dessus de la moyenne de sa distribution d'origine", peu importe ce que $X$ mesurait.
\end{intuitionbox}

\begin{intuitionbox}[Analogie simple]
Imaginons 2 élèves :
\begin{itemize}
    \item Alice a des notes entre 80 et 100 (moyenne 90, écart-type 5).
    \item Bob a des notes entre 0 et 20 (moyenne 10, écart-type 4).
\end{itemize}
Comparer leurs notes brutes n'a pas de sens. Mais si on les standardise, on peut se demander : "quand Alice est 1 écart-type au-dessus de sa moyenne (une note de 95), Bob est-il aussi 1 écart-type au-dessus de sa propre moyenne (une note de 14) ?". La standardisation permet cette comparaison.
\end{intuitionbox}

\begin{examplebox}[Centrer-réduire un dé]
Pour un lancer de dé $X$, on a $\mu_X = 3.5$ et $\sigma_X = \sqrt{35/12} \approx 1.708$.
Si on obtient $X=6$ : $Z = (6 - 3.5) / 1.708 \approx 1.46$.
Si on obtient $X=1$ : $Z = (1 - 3.5) / 1.708 \approx -1.46$.
Obtenir 6 est à 1.46 écarts-types au-dessus de la moyenne.
\end{examplebox}

Maintenant, formalisons le concept d'une covariance nulle.

\begin{definitionbox}[Variables Non Corréelées]
On dit que deux variables aléatoires $X$ et $Y$ sont \textbf{non corrélées} si leur covariance est nulle :
$$\text{Cov}(X,Y) = 0$$
Cela est équivalent à dire que $E[XY] = E[X]E[Y]$.
\end{definitionbox}

Il est crucial de ne pas confondre "non corrélées" et "indépendantes".

\begin{intuitionbox}
"Non corrélées" signifie qu'il n'y a \textbf{pas de relation linéaire} entre les variables. C'est plus faible que l'indépendance. Si $X$ et $Y$ sont indépendantes, elles sont forcément non corrélées. Mais l'inverse n'est pas vrai : $X$ et $Y$ peuvent être non corrélées (Cov=0) mais quand même dépendantes (par exemple si $Y=X^2$ pour un $X$ centré).
\end{intuitionbox}

\subsection{Variance d'une Somme de Variables Aléatoires}

Nous pouvons maintenant combiner nos connaissances de la variance et de la covariance pour répondre à une question cruciale : quelle est la variance d'une somme de variables, $X+Y$ ?

\begin{theorembox}[Formules pour la variance d'une somme de deux variables]
Pour deux variables aléatoires $X$ et $Y$ :
\begin{align*}
\text{Var}(X+Y) &= \text{Var}(X) + \text{Var}(Y) + 2\text{Cov}(X,Y) \\
\end{align*}
\end{theorembox}

La preuve découle de la définition de la variance et de la linéarité de l'espérance.

\begin{proofbox}
Soit $\mu_X = E[X]$ et $\mu_Y = E[Y]$.
\begin{align*}
\text{Var}(X+Y) &= E\left[ ((X+Y) - E[X+Y])^2 \right] \\
&= E\left[ ((X+Y) - (\mu_X + \mu_Y))^2 \right] \quad \text{(Par linéarité de E)} \\
&= E\left[ ((X - \mu_X) + (Y - \mu_Y))^2 \right] \quad \text{(On regroupe les termes)} \\
\text{Posons } A = (X - \mu_X) \text{ et } B = (Y - \mu_Y). \\
&= E[ (A + B)^2 ] = E[ A^2 + 2AB + B^2 ] \\
&= E[A^2] + 2E[AB] + E[B^2] \quad \text{(Par linéarité de E)} \\
\text{Or, par définition :} \\
E[A^2] &= E[(X-\mu_X)^2] = \text{Var}(X) \\
E[B^2] &= E[(Y-\mu_Y)^2] = \text{Var}(Y) \\
E[AB] &= E[(X-\mu_X)(Y-\mu_Y)] = \text{Cov}(X,Y) \\
\text{Donc, } \text{Var}(X+Y) &= \text{Var}(X) + \text{Var}(Y) + 2\text{Cov}(X,Y)
\end{align*}
\end{proofbox}

Cette formule est fondamentale en finance et en ingénierie.

\begin{intuitionbox}
La "volatilité" (variance) d'une somme n'est pas juste la somme des volatilités. Il faut ajouter le terme d'interaction (covariance).
Si $\text{Cov}(X,Y) > 0$ (elles bougent ensemble), la somme est \textbf{plus} volatile que la somme des parties.
Si $\text{Cov}(X,Y) < 0$ (elles bougent en sens inverse), elles s'amortissent mutuellement. La somme est \textbf{moins} volatile. C'est le principe de la diversification en finance.
\end{intuitionbox}

Cela mène à un corollaire très important lorsque la covariance est nulle.

\begin{theorembox}[Cas Particulier : Variables Non Corréelées]
Si $X$ et $Y$ sont non corrélées (Cov=0), la formule se simplifie :
$$\text{Var}(X+Y) = \text{Var}(X) + \text{Var}(Y)$$
\end{theorembox}

\begin{proofbox}
Cela découle directement du théorème précédent. Si $X$ et $Y$ sont non corrélées, alors $\text{Cov}(X,Y) = 0$.
Le terme $2\text{Cov}(X,Y)$ dans la formule générale $\text{Var}(X) + \text{Var}(Y) + 2\text{Cov}(X,Y)$ devient nul, laissant :
$$\text{Var}(X+Y) = \text{Var}(X) + \text{Var}(Y)$$
\end{proofbox}

C'est le cas pour nos dés indépendants.

\begin{examplebox}[Variance d'une somme de dés]
Soit $S = X+Y$ la somme de deux dés indépendants.
Puisqu'ils sont indépendants, ils sont non corrélés ($\text{Cov}(X,Y)=0$).
On sait $\text{Var}(X) = 35/12$ et $\text{Var}(Y) = 35/12$.
$$\text{Var}(S) = \text{Var}(X) + \text{Var}(Y) = \frac{35}{12} + \frac{35}{12} = \frac{70}{12} = \frac{35}{6} \approx 5.833$$
C'est bien plus simple que de calculer $E[S^2]$ et $E[S]$.
\end{examplebox}

On peut généraliser cette formule à $N$ variables.

\begin{theorembox}[Variance d'une somme de N variables]
La formule générale pour la somme de $N$ variables aléatoires $X_1, \dots, X_n$ est :
$$\text{Var}\left(\sum_{i=1}^n X_i\right) = \sum_{i=1}^n \text{Var}(X_i) + \sum_{i \neq j} \text{Cov}(X_i, X_j)$$
\end{theorembox}

\begin{proofbox}
On utilise la propriété $\text{Var}(S) = \text{Cov}(S, S)$ et la bilinéarité de la covariance.
Soit $S = \sum_{i=1}^n X_i$.
\begin{align*}
\text{Var}(S) &= \text{Cov}(S, S) = \text{Cov}\left(\sum_{i=1}^n X_i, \sum_{j=1}^n X_j\right) \\
&= \sum_{i=1}^n \sum_{j=1}^n \text{Cov}(X_i, X_j) \quad \text{(Par bilinéarité)}
\end{align*}
On peut séparer cette double somme en deux parties : le cas où $i=j$ et le cas où $i \neq j$.
$$ \text{Var}(S) = \sum_{i=1}^n \text{Cov}(X_i, X_i) + \sum_{i \neq j} \text{Cov}(X_i, X_j) $$
Puisque $\text{Cov}(X_i, X_i) = E[(X_i - \mu_i)(X_i - \mu_i)] = E[(X_i - \mu_i)^2] = \text{Var}(X_i)$, on obtient :
$$ \text{Var}(S) = \sum_{i=1}^n \text{Var}(X_i) + \sum_{i \neq j} \text{Cov}(X_i, X_j) $$
\end{proofbox}

Cette formule est au cœur de la théorie moderne du portefeuille.

\begin{intuitionbox}
La variance totale d'un système (comme un portefeuille d'actions) est la somme de toutes les variances individuelles ("risques propres") plus la somme de \textbf{toutes} les paires de covariances ("risques d'interaction"). Dans un grand portefeuille, le nombre de termes de covariance (environ $n^2$) est bien plus grand que le nombre de termes de variance ($n$), donc le risque total est dominé par la façon dont les actifs interagissent.
\end{intuitionbox}

\subsection{Théorème sur la somme de lois de Poisson}

Terminons avec un théorème très utile qui combine les idées d'indépendance et de somme de variables aléatoires pour une distribution spécifique.

\begin{theorembox}[La Somme de v.a. de Poisson Indépendantes est Poisson]
Soit $X_1, \dots, X_k$ une séquence de variables aléatoires de Poisson indépendantes, avec des paramètres respectifs $\lambda_1, \dots, \lambda_k$.
$$X_i \sim \text{Poisson}(\lambda_i) \quad \text{pour } i=1, \dots, k$$
Alors leur somme $Y = X_1 + \dots + X_k$ suit également une loi de Poisson, dont le paramètre est la somme des paramètres :
$$Y \sim \text{Poisson}(\lambda_1 + \dots + \lambda_k)$$
\end{theorembox}

La preuve pour $k=2$ (qui se généralise par récurrence) utilise l'indépendance et la formule du binôme de Newton.

\begin{proofbox}[Preuve pour la somme de deux v.a.]
Soit $X \sim \text{Poisson}(\lambda_1)$ et $Y \sim \text{Poisson}(\lambda_2)$, indépendantes.
Soit $S = X+Y$. Nous cherchons $P(S=k)$.
Pour que $S=k$, il faut que $X=j$ et $Y=k-j$, pour toutes les valeurs possibles de $j$ (de $0$ à $k$).
$$ P(S=k) = \sum_{j=0}^k P(X=j, Y=k-j) $$
Par indépendance, $P(X=j, Y=k-j) = P(X=j)P(Y=k-j)$.
\begin{align*}
P(S=k) &= \sum_{j=0}^k \left( \frac{e^{-\lambda_1}\lambda_1^j}{j!} \right) \left( \frac{e^{-\lambda_2}\lambda_2^{k-j}}{(k-j)!} \right) \\
&= e^{-(\lambda_1 + \lambda_2)} \sum_{j=0}^k \frac{\lambda_1^j \lambda_2^{k-j}}{j!(k-j)!}
\end{align*}
On multiplie et on divise par $k!$ pour faire apparaître le coefficient binomial :
\begin{align*}
P(S=k) &= \frac{e^{-(\lambda_1 + \lambda_2)}}{k!} \sum_{j=0}^k \frac{k!}{j!(k-j)!} \lambda_1^j \lambda_2^{k-j} \\
&= \frac{e^{-(\lambda_1 + \lambda_2)}}{k!} \sum_{j=0}^k \binom{k}{j} \lambda_1^j \lambda_2^{k-j}
\end{align*}
La somme est l'expansion du binôme de Newton pour $(\lambda_1 + \lambda_2)^k$.
$$ P(S=k) = \frac{e^{-(\lambda_1 + \lambda_2)} (\lambda_1 + \lambda_2)^k}{k!} $$
C'est la PMF d'une loi $\text{Poisson}(\lambda_1 + \lambda_2)$.
\end{proofbox}

Ce résultat est très intuitif :

\begin{intuitionbox}
Si des événements rares se produisent indépendamment à des taux constants, le nombre total d'événements se produisant est aussi un événement rare se produisant au taux total. Si les emails arrivent à $\lambda_1=5$/heure et les appels à $\lambda_2=10$/heure, les "communications totales" arrivent simplement à $\lambda = 5+10 = 15$/heure.
\end{intuitionbox}

\begin{examplebox}[Centre d'appels]
Un centre d'appels reçoit des appels "Ventes" selon $X_1 \sim \text{Poisson}(10 \text{ appels/heure})$ et des appels "Support" selon $X_2 \sim \text{Poisson}(15 \text{ appels/heure})$. Les deux types d'appels sont indépendants.
Le nombre total d'appels $Y = X_1 + X_2$ suit une loi $Y \sim \text{Poisson}(10+15=25 \text{ appels/heure})$.
La probabilité de recevoir exactement 20 appels en une heure est :
$$P(Y=20) = \frac{e^{-25} 25^{20}}{20!}$$
\end{examplebox}

\subsection{Exercices}

% --- Lois Jointes et Marginales (Discret) ---

\begin{exercicebox}[Exercice 1 : Loi Jointe et Marginales]
Soit le tableau suivant représentant la loi de probabilité jointe $P(X=x, Y=y)$ d'un couple de variables aléatoires $(X, Y)$.

\begin{center}
\begin{tabular}{|c|ccc|}
\hline
\diagbox{$X$}{$Y$} & 0 & 1 & 2 \\ \hline
0 & 0.1 & 0.2 & 0.1 \\
1 & 0.3 & 0.1 & 0.2 \\ \hline
\end{tabular}
\end{center}

\begin{enumerate}
    \item Vérifiez qu'il s'agit bien d'une loi de probabilité.
    \item Calculez la loi marginale de $X$, $P(X=x)$.
    \item Calculez la loi marginale de $Y$, $P(Y=y)$.
\end{enumerate}
\end{exercicebox}

\begin{exercicebox}[Exercice 2 : Calcul de Probabilité Jointe]
En utilisant la loi jointe de l'exercice 1 :
\begin{enumerate}
    \item Calculez $P(X=0, Y \le 1)$.
    \item Calculez $P(X=Y)$.
    \item Calculez $P(X > Y)$.
\end{enumerate}
\end{exercicebox}

\begin{exercicebox}[Exercice 3 : Indépendance (Loi Jointe)]
En utilisant la loi jointe de l'exercice 1 :
\begin{enumerate}
    \item Calculez $P(X=0) \times P(Y=0)$.
    \item Comparez ce résultat à $P(X=0, Y=0)$.
    \item Les variables $X$ et $Y$ sont-elles indépendantes ? Justifiez.
\end{enumerate}
\end{exercicebox}

% --- Espérance, Covariance et Corrélation ---

\begin{exercicebox}[Exercice 4 : Espérances Marginales]
En utilisant les lois marginales calculées à l'exercice 1 :
\begin{enumerate}
    \item Calculez l'espérance $E[X]$.
    \item Calculez l'espérance $E[Y]$.
\end{enumerate}
\end{exercicebox}

\begin{exercicebox}[Exercice 5 : Espérance d'une Fonction (LOTUS)]
En utilisant la loi jointe de l'exercice 1, calculez $E[XY]$.
(Indice : $E[XY] = \sum_x \sum_y (xy) P(X=x, Y=y)$).
\end{exercicebox}

\begin{exercicebox}[Exercice 6 : Covariance (Calcul)]
En utilisant les résultats des exercices 4 et 5, calculez la covariance $\text{Cov}(X,Y)$.
\end{exercicebox}

\begin{exercicebox}[Exercice 7 : Variances Marginales]
En utilisant les lois marginales de l'exercice 1 et les espérances de l'exercice 4 :
\begin{enumerate}
    \item Calculez $E[X^2]$ et $\text{Var}(X)$.
    \item Calculez $E[Y^2]$ et $\text{Var}(Y)$.
\end{enumerate}
\end{exercicebox}

\begin{exercicebox}[Exercice 8 : Corrélation (Calcul)]
En utilisant les résultats des exercices 6 et 7, calculez le coefficient de corrélation $\text{Corr}(X,Y)$.
\end{exercicebox}

% --- Propriétés de la Variance et de la Covariance ---

\begin{exercicebox}[Exercice 9 : Variance d'une Somme (Non Indépendant)]
Soient $X$ et $Y$ deux variables aléatoires telles que $\text{Var}(X) = 10$, $\text{Var}(Y) = 5$ et $\text{Cov}(X,Y) = 2$.
Calculez $\text{Var}(X+Y)$.
\end{exercicebox}

\begin{exercicebox}[Exercice 10 : Variance d'une Différence (Indépendant)]
Soient $X$ et $Y$ deux variables aléatoires \textbf{indépendantes} telles que $\text{Var}(X) = 16$ et $\text{Var}(Y) = 9$.
\begin{enumerate}
    \item Que vaut $\text{Cov}(X,Y)$ ?
    \item Calculez $\text{Var}(X-Y)$. (Rappel : $\text{Var}(X-Y) = \text{Var}(X) + \text{Var}(Y) - 2\text{Cov}(X,Y)$).
\end{enumerate}
\end{exercicebox}

\begin{exercicebox}[Exercice 11 : Bilinéarité de la Covariance]
Soient $X, Y, Z$ trois variables aléatoires. Exprimez $\text{Cov}(X+Y, Z)$ en fonction des covariances des variables individuelles.
\end{exercicebox}

\begin{exercicebox}[Exercice 12 : Variance d'une Combinaison Linéaire]
Soient $X$ et $Y$ deux variables aléatoires indépendantes avec $\text{Var}(X) = 4$ et $\text{Var}(Y) = 2$.
Calculez $\text{Var}(3X - 5Y + 1)$.
\end{exercicebox}

\begin{exercicebox}[Exercice 13 : Variance d'une Somme (Dés)]
On lance deux dés équilibrés $D_1$ et $D_2$. Soit $S = D_1 + D_2$.
On rappelle que pour un dé, $\text{Var}(D_i) = 35/12$.
\begin{enumerate}
    \item Les variables $D_1$ et $D_2$ sont-elles indépendantes ?
    \item Calculez $\text{Var}(S)$.
\end{enumerate}
\end{exercicebox}

\begin{exercicebox}[Exercice 14 : Covariance et Variance]
Soit $X$ une variable aléatoire. En utilisant la bilinéarité de la covariance, montrez que $\text{Cov}(X, X) = \text{Var}(X)$.
\end{exercicebox}

\begin{exercicebox}[Exercice 15 : Covariance avec une Constante]
Soit $X$ une variable aléatoire et $c$ une constante.
Montrez que $\text{Cov}(X, c) = 0$. (Indice : $E[c]=c$ et $E[Xc] = cE[X]$).
\end{exercicebox}

% --- Standardisation et Somme de Poissons ---

\begin{exercicebox}[Exercice 16 : Standardisation (Centrer-Réduire)]
Soit $X$ une variable aléatoire avec $E[X] = 10$ et $\text{Var}(X) = 4$.
Soit $Z = \frac{X - E[X]}{\sqrt{\text{Var}(X)}} = \frac{X - 10}{2}$ la variable standardisée.
\begin{enumerate}
    \item Calculez $E[Z]$.
    \item Calculez $\text{Var}(Z)$.
\end{enumerate}
\end{exercicebox}

\begin{exercicebox}[Exercice 17 : Corrélation et Standardisation]
Soit $\text{Corr}(X,Y) = 0.5$. Soient $Z_X$ et $Z_Y$ les versions standardisées de $X$ et $Y$.
Que vaut $\text{Cov}(Z_X, Z_Y)$ ? (Indice : regardez l'intuition de la corrélation).
\end{exercicebox}

\begin{exercicebox}[Exercice 18 : Somme de Lois de Poisson]
Un magasin reçoit des clients au comptoir A selon $X \sim \text{Poisson}(\lambda_1=5 \text{ clients/heure})$ et au comptoir B selon $Y \sim \text{Poisson}(\lambda_2=3 \text{ clients/heure})$. On suppose que $X$ et $Y$ sont indépendantes.
Soit $S = X+Y$ le nombre total de clients arrivant au magasin en une heure.
\begin{enumerate}
    \item Quelle est la loi de $S$ ? Donnez son nom et son paramètre.
    \item Quelle est la probabilité qu'exactement 6 clients au total arrivent en une heure, $P(S=6)$ ?
\end{enumerate}
\end{exercicebox}

\begin{exercicebox}[Exercice 19 : Corrélation Nulle mais Dépendance]
Soit $X$ une variable aléatoire $X \in \{-1, 0, 1\}$, avec $P(X=-1)=1/3$, $P(X=0)=1/3$, $P(X=1)=1/3$.
Soit $Y = X^2$.
\begin{enumerate}
    \item Calculez $E[X]$.
    \item Calculez $E[XY]$. (Indice : $E[XY] = E[X^3]$).
    \item Calculez $\text{Cov}(X,Y)$.
    \item Les variables $X$ et $Y$ sont-elles indépendantes ?
\end{enumerate}
\end{exercicebox}

\begin{exercicebox}[Exercice 20 : Bornes de la Corrélation]
Soit $X$ une variable aléatoire et $Y = -3X + 5$.
Sans faire de calcul, que vaut $\text{Corr}(X,Y)$ ? Justifiez.
\end{exercicebox}

\subsection{Corrections des Exercices}

% --- Corrections : Lois Jointes et Marginales (Discret) ---

\begin{correctionbox}[Correction Exercice 1 : Loi Jointe et Marginales]
1.  On somme toutes les probabilités du tableau :
    $0.1 + 0.2 + 0.1 + 0.3 + 0.1 + 0.2 = 1.0$.
    Puisque la somme est 1 et toutes les probabilités sont non négatives, c'est une loi valide.

2.  Loi marginale de $X$ (somme des lignes) :
    $P(X=0) = P(X=0, Y=0) + P(X=0, Y=1) + P(X=0, Y=2) = 0.1 + 0.2 + 0.1 = 0.4$.
    $P(X=1) = P(X=1, Y=0) + P(X=1, Y=1) + P(X=1, Y=2) = 0.3 + 0.1 + 0.2 = 0.6$.

3.  Loi marginale de $Y$ (somme des colonnes) :
    $P(Y=0) = P(X=0, Y=0) + P(X=1, Y=0) = 0.1 + 0.3 = 0.4$.
    $P(Y=1) = P(X=0, Y=1) + P(X=1, Y=1) = 0.2 + 0.1 = 0.3$.
    $P(Y=2) = P(X=0, Y=2) + P(X=1, Y=2) = 0.1 + 0.2 = 0.3$.
\end{correctionbox}

\begin{correctionbox}[Correction Exercice 2 : Calcul de Probabilité Jointe]
1.  $P(X=0, Y \le 1) = P(X=0, Y=0) + P(X=0, Y=1) = 0.1 + 0.2 = 0.3$.
2.  $P(X=Y) = P(X=0, Y=0) + P(X=1, Y=1) = 0.1 + 0.1 = 0.2$.
3.  $P(X > Y) = P(X=1, Y=0) = 0.3$. (C'est la seule case où $x > y$).
\end{correctionbox}

\begin{correctionbox}[Correction Exercice 3 : Indépendance (Loi Jointe)]
On utilise les lois marginales de l'exercice 1 : $P(X=0)=0.4$ et $P(Y=0)=0.4$.
1.  $P(X=0) \times P(Y=0) = 0.4 \times 0.4 = 0.16$.
2.  Dans le tableau joint, $P(X=0, Y=0) = 0.1$.
3.  Puisque $P(X=0, Y=0) \neq P(X=0) \times P(Y=0)$ (car $0.1 \neq 0.16$), les variables $X$ et $Y$ \textbf{ne sont pas indépendantes}. (Un seul contre-exemple suffit).
\end{correctionbox}

% --- Corrections : Espérance, Covariance et Corrélation ---

\begin{correctionbox}[Correction Exercice 4 : Espérances Marginales]
1.  $E[X] = \sum_x x P(X=x) = (0)(P(X=0)) + (1)(P(X=1))$
    $E[X] = (0)(0.4) + (1)(0.6) = 0.6$.
2.  $E[Y] = \sum_y y P(Y=y) = (0)(P(Y=0)) + (1)(P(Y=1)) + (2)(P(Y=2))$
    $E[Y] = (0)(0.4) + (1)(0.3) + (2)(0.3) = 0 + 0.3 + 0.6 = 0.9$.
\end{correctionbox}

\begin{correctionbox}[Correction Exercice 5 : Espérance d'une Fonction (LOTUS)]
On somme $(xy)P(X=x, Y=y)$ sur les 6 cases. Les termes où $x=0$ ou $y=0$ sont nuls.
$E[XY] = (0 \cdot 0)(0.1) + (0 \cdot 1)(0.2) + (0 \cdot 2)(0.1) + (1 \cdot 0)(0.3) + (1 \cdot 1)(0.1) + (1 \cdot 2)(0.2)$
$E[XY] = 0 + 0 + 0 + 0 + (1)(0.1) + (2)(0.2) = 0.1 + 0.4 = 0.5$.
\end{correctionbox}

\begin{correctionbox}[Correction Exercice 6 : Covariance (Calcul)]
On utilise la formule $\text{Cov}(X,Y) = E[XY] - E[X]E[Y]$.
$$ \text{Cov}(X,Y) = 0.5 - (0.6)(0.9) = 0.5 - 0.54 = -0.04 $$
\end{correctionbox}

\begin{correctionbox}[Correction Exercice 7 : Variances Marginales]
1.  Pour $X$:
    $E[X^2] = (0^2)(0.4) + (1^2)(0.6) = 0.6$.
    $\text{Var}(X) = E[X^2] - (E[X])^2 = 0.6 - (0.6)^2 = 0.6 - 0.36 = 0.24$.
2.  Pour $Y$:
    $E[Y^2] = (0^2)(0.4) + (1^2)(0.3) + (2^2)(0.3) = 0 + 0.3 + (4)(0.3) = 0.3 + 1.2 = 1.5$.
    $\text{Var}(Y) = E[Y^2] - (E[Y])^2 = 1.5 - (0.9)^2 = 1.5 - 0.81 = 0.69$.
\end{correctionbox}

\begin{correctionbox}[Correction Exercice 8 : Corrélation (Calcul)]
On utilise la formule $\text{Corr}(X,Y) = \frac{\text{Cov}(X,Y)}{\sqrt{\text{Var}(X)\text{Var}(Y)}}$.
$$ \text{Corr}(X,Y) = \frac{-0.04}{\sqrt{0.24 \times 0.69}} = \frac{-0.04}{\sqrt{0.1656}} \approx \frac{-0.04}{0.4069} \approx -0.098 $$
La corrélation est très faible et négative.
\end{correctionbox}

% --- Corrections : Propriétés de la Variance et de la Covariance ---

\begin{correctionbox}[Correction Exercice 9 : Variance d'une Somme (Non Indépendant)]
On utilise la formule générale :
$$ \text{Var}(X+Y) = \text{Var}(X) + \text{Var}(Y) + 2\text{Cov}(X,Y) $$
$$ \text{Var}(X+Y) = 10 + 5 + 2(2) = 15 + 4 = 19 $$
\end{correctionbox}

\begin{correctionbox}[Correction Exercice 10 : Variance d'une Différence (Indépendant)]
1.  Puisque $X$ et $Y$ sont indépendantes, leur covariance est nulle : $\text{Cov}(X,Y) = 0$.
2.  On utilise la formule générale :
    $$ \text{Var}(X-Y) = \text{Var}(X + (-1)Y) = \text{Var}(X) + \text{Var}(-1 \cdot Y) + 2\text{Cov}(X, -Y) $$
    $$ = \text{Var}(X) + (-1)^2 \text{Var}(Y) - 2\text{Cov}(X, Y) $$
    $$ \text{Var}(X-Y) = \text{Var}(X) + \text{Var}(Y) - 2(0) $$
    $$ \text{Var}(X-Y) = 16 + 9 = 25 $$
\end{correctionbox}

\begin{correctionbox}[Correction Exercice 11 : Bilinéarité de la Covariance]
La covariance est linéaire sur son premier argument :
$$ \text{Cov}(X+Y, Z) = \text{Cov}(X, Z) + \text{Cov}(Y, Z) $$
\end{correctionbox}

\begin{correctionbox}[Correction Exercice 12 : Variance d'une Combinaison Linéaire]
On utilise $\text{Var}(aX + bY + c) = a^2 \text{Var}(X) + b^2 \text{Var}(Y) + 2ab\text{Cov}(X,Y)$.
Ici $a=3$, $b=-5$, $c=1$. $X$ et $Y$ sont indépendantes, donc $\text{Cov}(X,Y)=0$.
$$ \text{Var}(3X - 5Y + 1) = (3)^2 \text{Var}(X) + (-5)^2 \text{Var}(Y) + 0 $$
$$ = 9 \times (4) + 25 \times (2) = 36 + 50 = 86 $$
(Note : la constante additive $c=1$ ne change pas la variance).
\end{correctionbox}

\begin{correctionbox}[Correction Exercice 13 : Variance d'une Somme (Dés)]
1.  Oui, les lancers de deux dés standards sont des événements physiquement indépendants.
2.  Puisqu'ils sont indépendants, $\text{Cov}(D_1, D_2) = 0$.
    $$ \text{Var}(S) = \text{Var}(D_1 + D_2) = \text{Var}(D_1) + \text{Var}(D_2) $$
    $$ \text{Var}(S) = \frac{35}{12} + \frac{35}{12} = \frac{70}{12} = \frac{35}{6} $$
\end{correctionbox}

\begin{correctionbox}[Correction Exercice 14 : Covariance et Variance]
Par définition, $\text{Cov}(A, B) = E[(A-\mu_A)(B-\mu_B)]$.
Posons $A=X$ et $B=X$. Alors $\mu_A = \mu_X$ et $\mu_B = \mu_X$.
$$ \text{Cov}(X, X) = E[(X-\mu_X)(X-\mu_X)] = E[(X-\mu_X)^2] $$
C'est la définition de $\text{Var}(X)$.
\end{correctionbox}

\begin{correctionbox}[Correction Exercice 15 : Covariance avec une Constante]
On utilise la formule de calcul $\text{Cov}(X,c) = E[Xc] - E[X]E[c]$.
Par linéarité, $E[Xc] = cE[X]$.
L'espérance d'une constante est la constante elle-même : $E[c] = c$.
$$ \text{Cov}(X, c) = cE[X] - E[X]c = 0 $$
\end{correctionbox}

% --- Corrections : Standardisation et Somme de Poissons ---

\begin{correctionbox}[Correction Exercice 16 : Standardisation (Centrer-Réduire)]
$Z = \frac{X - 10}{2} = \frac{1}{2}X - 5$.
1.  Calcul de $E[Z]$ par linéarité :
    $$ E[Z] = E\left[ \frac{1}{2}X - 5 \right] = \frac{1}{2}E[X] - 5 = \frac{1}{2}(10) - 5 = 5 - 5 = 0 $$
2.  Calcul de $\text{Var}(Z)$ par les propriétés de la variance :
    $$ \text{Var}(Z) = \text{Var}\left( \frac{1}{2}X - 5 \right) = \left(\frac{1}{2}\right)^2 \text{Var}(X) = \frac{1}{4} \text{Var}(X) $$
    $$ \text{Var}(Z) = \frac{1}{4}(4) = 1 $$
    Par définition, une variable standardisée a une moyenne de 0 et une variance de 1.
\end{correctionbox}

\begin{correctionbox}[Correction Exercice 17 : Corrélation et Standardisation]
La corrélation $\text{Corr}(X,Y)$ EST, par définition, la covariance des versions standardisées $Z_X$ et $Z_Y$.
$$ \text{Cov}(Z_X, Z_Y) = \text{Corr}(X,Y) = 0.5 $$
\end{correctionbox}

\begin{correctionbox}[Correction Exercice 18 : Somme de Lois de Poisson]
1.  Puisque $X$ et $Y$ sont des v.a. de Poisson \textbf{indépendantes}, leur somme $S=X+Y$ suit aussi une \textbf{loi de Poisson}.
    Le nouveau paramètre est la somme des paramètres : $\lambda_S = \lambda_1 + \lambda_2 = 5 + 3 = 8$.
    Donc, $S \sim \text{Poisson}(\lambda=8)$.
2.  On cherche $P(S=6)$ pour $S \sim \text{Poisson}(8)$.
    $$ P(S=6) = \frac{e^{-8} 8^6}{6!} = \frac{e^{-8} \times 262144}{720} = 364.08 \times e^{-8} \approx 0.122 $$
\end{correctionbox}

\begin{correctionbox}[Correction Exercice 19 : Corrélation Nulle mais Dépendance]
1.  $E[X] = (-1)(1/3) + (0)(1/3) + (1)(1/3) = -1/3 + 0 + 1/3 = 0$.
2.  $E[XY] = E[X(X^2)] = E[X^3]$.
    $E[X^3] = (-1)^3(1/3) + (0)^3(1/3) + (1)^3(1/3) = (-1)(1/3) + 0 + (1)(1/3) = 0$.
3.  $\text{Cov}(X,Y) = E[XY] - E[X]E[Y] = 0 - (0)E[Y] = 0$.
    Les variables sont \textbf{non corrélées}.
4.  Les variables $X$ et $Y$ sont-elles indépendantes ? Non.
    Test : $P(X=1, Y=1) \stackrel{?}{=} P(X=1)P(Y=1)$.
    - $P(X=1, Y=1) = P(X=1, X^2=1) = P(X=1) = 1/3$.
    - $P(X=1) = 1/3$.
    - $P(Y=1) = P(X^2=1) = P(X=1) + P(X=-1) = 1/3 + 1/3 = 2/3$.
    - $P(X=1)P(Y=1) = (1/3)(2/3) = 2/9$.
    Puisque $1/3 \neq 2/9$, elles \textbf{ne sont pas indépendantes}.
    C'est un exemple classique de dépendance non linéaire avec covariance nulle.
\end{correctionbox}

\begin{correctionbox}[Correction Exercice 20 : Bornes de la Corrélation]
$Y$ est une fonction linéaire parfaite de $X$ : $Y = aX + b$ avec $a=-3$ et $b=5$.
La corrélation $\text{Corr}(X,Y)$ mesure la force de la relation \textit{linéaire}. Puisqu'elle est parfaite, la corrélation doit être $\pm 1$.
Le coefficient $a = -3$ est négatif, donc la relation est décroissante.
Par conséquent, $\text{Corr}(X,Y) = -1$.
\end{correctionbox}

\subsection{Exercices Pratiques (Python)}

Nous allons maintenant utiliser Python et les bibliothèques \texttt{pandas}, \texttt{numpy} et \texttt{seaborn} pour explorer les concepts de distributions multivariées sur un jeu de données réel : le dataset "penguins".

Ces exercices vont nous permettre de calculer empiriquement :
\begin{itemize}
    \item Les distributions jointes (discrètes)
    \item Les distributions marginales
    \item La covariance
    \item La corrélation
    \item L'effet de la standardisation
\end{itemize}

\begin{codecell}
# Cellule d'installation et d'importation
pip install numpy pandas seaborn
\end{codecell}

\begin{codecell}
import numpy as np
import pandas as pd
import seaborn as sns

# Charger le jeu de donnees et supprimer les lignes avec des valeurs manquantes
# pour simplifier les calculs.
penguins = sns.load_dataset('penguins').dropna()

print("Apercu des donnees (penguins) :")
print(penguins.head())
\end{codecell}

\begin{exercicebox}[Exercice 1 : Lois Jointe et Marginale (Données Discrètes)]
Nous allons analyser la relation entre deux variables catégorielles : \texttt{species} (l'espèce) et \texttt{island} (l'île où vit le manchot).

\textbf{Votre tâche :}
\begin{enumerate}
    \item Utiliser \texttt{pandas.crosstab} pour créer un tableau de contingence (les effectifs) de \texttt{species} et \texttt{island}.
    \item Normaliser ce tableau par l'effectif total pour obtenir la \textbf{loi jointe} (PMF jointe) $P(Species=s, Island=i)$.
    \item Calculer la \textbf{loi marginale} de \texttt{species} en sommant les probabilités jointes sur les colonnes (\texttt{axis=1}).
    \item Calculer la \textbf{loi marginale} de \texttt{island} en sommant sur les lignes (\texttt{axis=0}).
    \item Vérifier que la somme des lois marginales est bien égale à 1.
\end{enumerate}

\begin{codecell}
import pandas as pd

# 1. Tableau de contingence (effectifs)
joint_counts = pd.crosstab(penguins["species"], penguins["island"])
print("--- 1. Tableau de contingence (Effectifs) ---")
print(joint_counts)

# 2. Loi jointe (PMF)
total_penguins = len(penguins)
joint_pmf = joint_counts / total_penguins
print("\n--- 2. Loi Jointe (PMF) ---")
print(joint_pmf)

# 3. Loi marginale de 'species' (somme des lignes)
marginal_species = joint_pmf.sum(axis=1)
print("\n--- 3. Loi Marginale (Species) ---")
print(marginal_species)

# 4. Loi marginale de 'island' (somme des colonnes)
marginal_island = joint_pmf.sum(axis=0)
print("\n--- 4. Loi Marginale (Island) ---")
print(marginal_island)

# 5. Verification
print(f"\n--- 5. Verification ---")
print(f"Somme de la PMF jointe : {joint_pmf.sum().sum():.2f}")
print(f"Somme marginale (Species) : {marginal_species.sum():.2f}")
print(f"Somme marginale (Island) : {marginal_island.sum():.2f}")
\end{codecell}
\end{exercicebox}

\begin{exercicebox}[Exercice 2 : Calcul de Covariance]
Nous allons maintenant calculer la covariance entre deux variables numériques : \texttt{bill\_length\_mm} (longueur du bec) et \texttt{flipper\_length\_mm} (longueur de la nageoire).

Nous utiliserons la formule : $\text{Cov}(X,Y) = E[XY] - E[X]E[Y]$.

\textbf{Votre tâche :}
\begin{enumerate}
    \item Isoler les deux variables $X$ (\texttt{bill\_length\_mm}) et $Y$ (\texttt{flipper\_length\_mm}) dans des variables \texttt{pandas.Series}.
    \item Calculer $E[X]$ et $E[Y]$ (en utilisant \texttt{.mean()}).
    \item Calculer $E[XY]$ (Indice : \texttt{(X * Y).mean()}).
    \item Appliquer la formule pour trouver la covariance.
    \item Vérifier votre résultat en utilisant la matrice de covariance de NumPy : \texttt{np.cov(X, Y)}.
\end{enumerate}

\begin{codecell}
import numpy as np

# 1. Isoler les variables
X = penguins["bill_length_mm"]
Y = penguins["flipper_length_mm"]

# 2. Calculer E[X] et E[Y]
E_X = X.mean()
E_Y = Y.mean()

# 3. Calculer E[XY]
E_XY = (X * Y).mean()

# 4. Appliquer la formule
cov_calc = E_XY - E_X * E_Y
print(f"E[X] = {E_X:.2f}, E[Y] = {E_Y:.2f}, E[XY] = {E_XY:.2f}")
print(f"Covariance (calculee) : {cov_calc:.4f}")

# 5. Verifier avec NumPy
# np.cov retourne une matrice 2x2. La covariance est a [0,1] (ou [1,0])
# ddof=1 (par defaut) calcule la covariance d'echantillon, ce qui est standard.
cov_matrix_np = np.cov(X, Y)
print(f"Covariance (numpy)    : {cov_matrix_np[0, 1]:.4f}")
\end{codecell}
\end{exercicebox}

\begin{exercicebox}[Exercice 3 : Calcul de Corrélation]
La covariance de l'exercice 2 est positive, mais sa valeur (ex: $\approx 50.38$) est difficile à interpréter. Nous allons la normaliser pour obtenir la corrélation.

Nous utiliserons la formule : $\text{Corr}(X,Y) = \frac{\text{Cov}(X,Y)}{\sigma_X \sigma_Y}$.

\textbf{Votre tâche :}
\begin{enumerate}
    \item Récupérer la covariance calculée à l'exercice 2.
    \item Calculer les écarts-types (d'échantillon, \texttt{ddof=1}) $\sigma_X$ et $\sigma_Y$ (en utilisant \texttt{.std()}).
    \item Appliquer la formule pour trouver la corrélation.
    \item Vérifier votre résultat en utilisant \texttt{pandas.Series.corr()} (ex: \texttt{X.corr(Y)}) ou \texttt{np.corrcoef(X, Y)}.
\end{enumerate}

\begin{codecell}
import numpy as np

# X, Y proviennent de l'exercice precedent
X = penguins["bill_length_mm"]
Y = penguins["flipper_length_mm"]
cov_calc = np.cov(X, Y)[0, 1] # Utilisons la valeur numpy pour la precision

# 2. Calculer les ecarts-types (pandas .std() utilise ddof=1 par defaut)
std_X = X.std()
std_Y = Y.std()

# 3. Appliquer la formule
corr_calc = cov_calc / (std_X * std_Y)
print(f"Sigma_X = {std_X:.2f}, Sigma_Y = {std_Y:.2f}")
print(f"Correlation (calculee) : {corr_calc:.4f}")

# 4. Verifier avec Pandas
corr_pandas = X.corr(Y)
print(f"Correlation (pandas)   : {corr_pandas:.4f}")

# 4. Verifier avec NumPy
corr_matrix_np = np.corrcoef(X, Y)
print(f"Correlation (numpy)    : {corr_matrix_np[0, 1]:.4f}")
\end{codecell}
\end{exercicebox}

\begin{exercicebox}[Exercice 4 : Standardisation (Centrer-Réduire)]
Vérifions empiriquement que la standardisation $Z = \frac{X - \mu_X}{\sigma_X}$ produit une nouvelle variable avec une moyenne de 0 et un écart-type de 1.

\textbf{Votre tâche :}
\begin{enumerate}
    \item Isoler la variable $X$ = \texttt{body\_mass\_g}.
    \item Calculer sa moyenne $\mu_X$ et son écart-type $\sigma_X$. 
    (Note : pour cette transformation, on utilise l'écart-type de la population, \texttt{ddof=0}, car on transforme le set de données lui-même).
    \item Créer la variable standardisée $Z = (X - \mu_X) / \sigma_X$.
    \item Calculer la moyenne et l'écart-type de $Z$ et vérifier qu'ils sont (respectivement) très proches de 0 et 1.
\end{enumerate}

\begin{codecell}
import numpy as np

# 1. Isoler la variable
X_mass = penguins["body_mass_g"]

# 2. Calculer la moyenne et l'ecart-type (de la population, ddof=0)
mu_X = X_mass.mean()
std_X = X_mass.std(ddof=0)

print(f"--- Original ---")
print(f"Moyenne (X) : {mu_X:.4f}")
print(f"Ecart-type (X) : {std_X:.4f}")

# 3. Standardiser la variable
Z_mass = (X_mass - mu_X) / std_X

# 4. Verifier la moyenne et l'ecart-type de Z
mean_Z = Z_mass.mean()
std_Z = Z_mass.std(ddof=0) # On utilise ddof=0 ici aussi

print(f"\n--- Standardise ---")
# La moyenne sera un tres petit nombre (ex: 1.23e-16) a cause de la precision
print(f"Moyenne (Z) : {mean_Z:.2e}") 
print(f"Ecart-type (Z) : {std_Z:.4f}")
\end{codecell}
\end{exercicebox}