\newpage
\section{Espérance et Variance }

\subsection{Espérance d'une variable aléatoire discrète}

Maintenant que nous avons défini les variables aléatoires discrètes et leur distribution (PMF), l'étape suivante est de résumer ces distributions. La mesure la plus importante est leur "centre", ou leur valeur moyenne.

\begin{definitionbox}[Espérance]
L'espérance (ou valeur attendue) d'une variable aléatoire discrète $X$, qui prend les valeurs distinctes $x_1, x_2, \dots$, est définie par :
$$ E(X) = \sum_j x_j P(X=x_j) $$
\end{definitionbox}

Cette formule est une moyenne pondérée de toutes les valeurs possibles.

\begin{intuitionbox}
L'espérance représente la valeur moyenne que l'on obtiendrait si l'on répétait l'expérience un très grand nombre de fois. C'est le \textbf{centre de gravité} de la distribution de probabilité. Si les probabilités étaient des masses placées sur une tige aux positions $x_j$, l'espérance serait le point d'équilibre.
\end{intuitionbox}

L'exemple le plus simple est le lancer d'un dé.

\begin{examplebox}[Lancer d'un dé]
Soit $X$ le résultat d'un lancer de dé équilibré. Chaque face a une probabilité de $1/6$. L'espérance est :
$$ E(X) = 1\left(\frac{1}{6}\right) + 2\left(\frac{1}{6}\right) + 3\left(\frac{1}{6}\right) + 4\left(\frac{1}{6}\right) + 5\left(\frac{1}{6}\right) + 6\left(\frac{1}{6}\right) = \frac{21}{6} = 3.5 $$
Même si 3.5 n'est pas un résultat possible, c'est la valeur moyenne sur un grand nombre de lancers.
\end{examplebox}

\subsection{Linéarité de l'espérance}

Le calcul de l'espérance deviendrait très fastidieux si nous devions toujours utiliser la définition. Heureusement, l'espérance possède une propriété fondamentale qui simplifie énormément les calculs.

\begin{theorembox}[Linéarité de l'espérance]
Pour toutes variables aléatoires $X$ et $Y$, et pour toute constante $c$, on a :
\begin{align*}
E(X+Y) &= E(X) + E(Y) \\
E(cX) &= cE(X)
\end{align*}
Cette propriété est extrêmement puissante car elle ne requiert pas que $X$ et $Y$ soient indépendantes.
\end{theorembox}

La preuve de $E(cX) = cE(X)$ est directe à partir de la définition. La preuve pour la somme $E(X+Y)$ est plus complexe mais essentielle.

\begin{proofbox}
La première propriété est directe :
$$ E(cX) = \sum_x (cx) P(X=x) = c \sum_x x P(X=x) = cE(X) $$
Pour la seconde, nous devons utiliser la définition de l'espérance pour une fonction de deux variables (une extension de LOTUS). Soit $S = X+Y$. L'espérance $E(S)$ se calcule en sommant sur toutes les paires possibles $(x, y)$:
\begin{align*}
E(X+Y) &= \sum_x \sum_y (x+y) P(X=x, Y=y) \\
&= \sum_x \sum_y x P(X=x, Y=y) + \sum_x \sum_y y P(X=x, Y=y) \\
&= \sum_x x \left( \sum_y P(X=x, Y=y) \right) + \sum_y y \left( \sum_x P(X=x, Y=y) \right)
\end{align*}
Par la loi des probabilités totales (ou "marginalisation"), la somme interne $\sum_y P(X=x, Y=y)$ est simplement $P(X=x)$. De même, $\sum_x P(X=x, Y=y) = P(Y=y)$.
$$ E(X+Y) = \sum_x x P(X=x) + \sum_y y P(Y=y) = E(X) + E(Y) $$
Notez que l'indépendance n'a jamais été requise pour cette preuve.
\end{proofbox}

Cette propriété est incroyablement utile.

\begin{intuitionbox}
Cette propriété formalise une idée très simple : "la moyenne d'une somme est la somme des moyennes". Si vous jouez à deux jeux de hasard, votre gain moyen total est simplement la somme de ce que vous gagnez en moyenne à chaque jeu, que les jeux soient liés ou non.
\end{intuitionbox}

Cette propriété rend le calcul de l'espérance d'une somme trivial, comme le montre l'exemple des deux dés.

\begin{examplebox}[Somme de deux dés]
Soit $X_1$ le résultat du premier dé et $X_2$ celui du second. On sait que $E(X_1) = 3.5$ et $E(X_2) = 3.5$.
Soit $S = X_1 + X_2$ la somme des deux dés. Grâce à la linéarité, on peut calculer l'espérance de la somme sans avoir à lister les 36 résultats possibles :
$$ E(S) = E(X_1 + X_2) = E(X_1) + E(X_2) = 3.5 + 3.5 = 7 $$
\end{examplebox}

\subsection{Espérance de la loi binomiale}

Nous pouvons maintenant utiliser cette puissante propriété de linéarité pour trouver l'espérance de nos distributions de référence, en évitant des sommes complexes.

\begin{theorembox}[Espérance de la loi binomiale]
Si $X \sim \text{Bin}(n, p)$, alors son espérance est $E(X) = np$.
\end{theorembox}

Ce résultat est profondément intuitif.

\begin{intuitionbox}
Ce résultat est très naturel. Si vous lancez une pièce 100 fois ($n=100$) avec une probabilité de 50\% d'obtenir Pile ($p=0.5$), vous vous attendez en moyenne à obtenir $100 \times 0.5 = 50$ Piles. La formule $np$ généralise cette idée.
\end{intuitionbox}

La preuve formelle est un exemple parfait de l'élégance de la linéarité, utilisant les variables indicatrices.

\begin{proofbox}
Le calcul direct de l'espérance avec la PMF binomiale est possible, mais long. En utilisant la linéarité de l'espérance, on obtient une preuve beaucoup plus courte et élégante.

On peut voir une variable binomiale $X$ comme la somme de $n$ variables de Bernoulli indépendantes, $X = I_1 + I_2 + \dots + I_n$, où chaque $I_j$ représente le succès (1) ou l'échec (0) du $j$-ième essai.

Chaque $I_j$ a pour espérance $E(I_j) = 1 \cdot p + 0 \cdot (1-p) = p$.

Par linéarité de l'espérance, on a :
$$ E(X) = E(I_1) + E(I_2) + \dots + E(I_n) = \underbrace{p + p + \dots + p}_{n \text{ fois}} = np $$
\end{proofbox}

\subsection{Espérance de la loi géométrique}

Calculons maintenant l'espérance pour la loi qui modélise le temps d'attente.

\begin{theorembox}[Espérance de la loi géométrique]
L'espérance d'une variable aléatoire $X \sim \text{Geom}(p)$ (comptant le nombre d'échecs) est :
$$ E(X) = \frac{1-p}{p} = \frac{q}{p} $$
\end{theorembox}

L'intuition est aussi très forte ici :

\begin{intuitionbox}
Si un événement a 1 chance sur 10 de se produire ($p=0.1$), il est logique de penser qu'il faudra en moyenne 9 échecs ($q/p = 0.9/0.1=9$) avant qu'il ne se produise. L'espérance du nombre total d'essais (échecs + 1 succès) serait alors $1/p$.
\end{intuitionbox}

Contrairement à la loi binomiale, la preuve la plus directe ne repose pas sur la linéarité mais sur une manipulation de séries.

\begin{proofbox}[Démonstration de l'espérance géométrique via les séries entières]
Soit $X \sim \text{Geom}(p)$, où $X$ compte le nombre d'échecs avant le premier succès. La PMF est $P(X=k) = q^k p$ pour $k=0, 1, 2, \dots$, avec $q=1-p$.

Par définition, l'espérance est :
$$ E(X) = \sum_{k=0}^{\infty} k \cdot P(X=k) = \sum_{k=0}^{\infty} k q^k p $$
Le terme pour $k=0$ est nul, on peut donc commencer la somme à $k=1$ :
$$ E(X) = p \sum_{k=1}^{\infty} k q^k $$
L'astuce consiste à reconnaître que la somme ressemble à la dérivée d'une série géométrique. Rappelons la formule de la série géométrique pour $|q|<1$ :
$$ \sum_{k=0}^{\infty} q^k = \frac{1}{1-q} $$
En dérivant les deux côtés par rapport à $q$, on obtient :
$$ \frac{d}{dq} \left( \sum_{k=0}^{\infty} q^k \right) = \frac{d}{dq} \left( \frac{1}{1-q} \right) $$
$$ \sum_{k=1}^{\infty} k q^{k-1} = \frac{1}{(1-q)^2} $$
Pour faire apparaître ce terme dans notre formule d'espérance, on factorise $q$ dans la somme :
$$ E(X) = p \cdot q \sum_{k=1}^{\infty} k q^{k-1} $$
On peut maintenant remplacer la somme par son expression analytique :
$$ E(X) = p \cdot q \cdot \frac{1}{(1-q)^2} $$
Puisque $p = 1-q$, on a :
$$ E(X) = p \cdot q \cdot \frac{1}{p^2} = \frac{q}{p} $$
Ce qui démontre que l'espérance du nombre d'échecs avant le premier succès est $\frac{q}{p}$.
\end{proofbox}

\subsection{Loi du statisticien inconscient (LOTUS)}

Souvent, nous ne sommes pas intéressés par l'espérance de $X$ elle-même, mais par l'espérance d'une fonction de $X$, par exemple $E(X^2)$ ou $E(e^X)$.

\begin{theorembox}[Théorème de Transfert (LOTUS)]
Si $X$ est une variable aléatoire discrète et $g(x)$ est une fonction de $\mathbb{R}$ dans $\mathbb{R}$, alors l'espérance de la variable aléatoire $g(X)$ est donnée par :
$$ E[g(X)] = \sum_x g(x) P(X=x) $$
La somme porte sur toutes les valeurs possibles de $X$. Ce théorème est utile car il évite d'avoir à trouver la PMF de $g(X)$.
\end{theorembox}

La preuve dans le cas discret consiste simplement à regrouper les termes.

\begin{proofbox}
Soit $Y = g(X)$. Par définition, l'espérance de $Y$ est $E(Y) = \sum_y y P(Y=y)$.
L'ensemble des valeurs $y$ que $Y$ peut prendre est $\{g(x) \mid x \in \text{support de } X\}$.
Pour une valeur $y$ donnée, l'événement $\{Y=y\}$ est l'union de tous les événements $\{X=x\}$ tels que $g(x)=y$.
$$ P(Y=y) = P(g(X)=y) = \sum_{x: g(x)=y} P(X=x) $$
En substituant cela dans la définition de $E(Y)$ :
$$ E(Y) = \sum_y y \left( \sum_{x: g(x)=y} P(X=x) \right) $$
On peut réécrire $y$ comme $g(x)$ à l'intérieur de la seconde somme :
$$ E(g(X)) = \sum_y \sum_{x: g(x)=y} g(x) P(X=x) $$
Cette double somme parcourt toutes les valeurs de $y$, et pour chaque $y$, elle parcourt tous les $x$ correspondants. Cela revient à simplement sommer sur tous les $x$ possibles dès le départ :
$$ E[g(X)] = \sum_x g(x) P(X=x) $$
\end{proofbox}

Ce théorème justifie son nom : c'est ce que l'on ferait "inconsciemment".

\begin{intuitionbox}
Pour trouver la valeur moyenne d'une fonction d'une variable aléatoire (par exemple, le carré du résultat d'un dé), vous n'avez pas besoin de déterminer d'abord la distribution de ce carré. Vous pouvez simplement prendre chaque valeur possible du résultat original, lui appliquer la fonction, et pondérer ce nouveau résultat par la probabilité du résultat original.
\end{intuitionbox}

Utilisons ce théorème pour calculer $E(X^2)$ pour notre dé.

\begin{examplebox}[Calcul de $E(X^2)$ pour un dé]
Soit $X$ le résultat d'un lancer de dé. Calculons l'espérance de $Y=X^2$. La fonction est $g(x)=x^2$.
\begin{align*}
E(X^2) &= \sum_{k=1}^6 k^2 P(X=k) \\
&= 1^2\left(\frac{1}{6}\right) + 2^2\left(\frac{1}{6}\right) + 3^2\left(\frac{1}{6}\right) + 4^2\left(\frac{1}{6}\right) + 5^2\left(\frac{1}{6}\right) + 6^2\left(\frac{1}{6}\right) \\
&= \frac{1+4+9+16+25+36}{6} = \frac{91}{6} \approx 15.17
\end{align*}
\end{examplebox}

\subsection{Variance}

L'espérance nous donne le centre d'une distribution, mais elle ne dit rien sur sa "largeur" ou sa "dispersion". C'est le rôle de la variance.

\begin{definitionbox}[Variance et écart-type]
La \textbf{variance} d'une variable aléatoire $X$ mesure la dispersion de sa distribution autour de son espérance. Elle est définie par :
$$ \text{Var}(X) = E\left[ (X - E(X))^2 \right] $$
La racine carrée de la variance est appelée l' \textbf{écart-type} :
$$ \text{SD}(X) = \sqrt{\text{Var}(X)} $$
\end{definitionbox}

L'idée est de mesurer l'écart quadratique moyen à l'espérance.

\begin{intuitionbox}
La variance est la "distance carrée moyenne à la moyenne". On prend l'écart de chaque valeur par rapport à la moyenne, on le met au carré (pour que les écarts positifs et négatifs ne s'annulent pas), puis on en calcule la moyenne. L'écart-type est souvent plus interprétable car il ramène cette mesure de dispersion dans les mêmes unités que la variable aléatoire elle-même.
\end{intuitionbox}

La définition $E[(X-E(X))^2]$ est excellente pour l'interprétation, mais pénible pour le calcul. Une formule alternative est presque toujours utilisée.

\begin{theorembox}[Formule de calcul de la variance]
Pour toute variable aléatoire $X$, une formule plus pratique pour le calcul de la variance est :
$$ \text{Var}(X) = E(X^2) - [E(X)]^2 $$
\end{theorembox}

La preuve est une simple expansion algébrique utilisant la linéarité de l'espérance.

\begin{proofbox}
Soit $\mu = E(X)$. On part de la définition de la variance :
\begin{align*}
\text{Var}(X) &= E[ (X - \mu)^2 ] \\
&= E[ X^2 - 2X\mu + \mu^2 ] \quad \text{(On développe le carré)} \\
&= E(X^2) - E(2\mu X) + E(\mu^2) \quad \text{(Par linéarité de l'espérance)} \\
&= E(X^2) - 2\mu E(X) + \mu^2 \quad \text{(Car $2\mu$ et $\mu^2$ sont des constantes)} \\
&= E(X^2) - 2\mu(\mu) + \mu^2 \quad \text{(Car $E(X) = \mu$)} \\
&= E(X^2) - 2\mu^2 + \mu^2 \\
&= E(X^2) - \mu^2 = E(X^2) - [E(X)]^2
\end{align*}
\end{proofbox}

Nous pouvons maintenant calculer la variance de notre lancer de dé.

\begin{examplebox}[Variance d'un lancer de dé]
Nous avons déjà calculé pour un dé que $E(X) = 3.5$ et $E(X^2) = 91/6$. On peut maintenant trouver la variance facilement :
$$ \text{Var}(X) = E(X^2) - [E(X)]^2 = \frac{91}{6} - (3.5)^2 = \frac{91}{6} - 12.25 = 15.166... - 12.25 \approx 2.917 $$
L'écart-type est $\text{SD}(X) = \sqrt{2.917} \approx 1.708$.
\end{examplebox}

\subsection{Exercices}

% --- Espérance de base et LOTUS ---

\begin{exercicebox}[Exercice 1 : Calcul d'Espérance (PMF Simple)]
Une variable aléatoire $X$ a la distribution de probabilité suivante :
$P(X=-1) = 0.3$, $P(X=0) = 0.5$, $P(X=2) = 0.2$.
Calculez l'espérance $E(X)$.
\end{exercicebox}

\begin{exercicebox}[Exercice 2 : LOTUS (Calcul de $E(X^2)$)]
En utilisant la même variable aléatoire $X$ que dans l'exercice 1, calculez $E(X^2)$.
\end{exercicebox}

\begin{exercicebox}[Exercice 3 : Variance (Calcul de base)]
En utilisant les résultats des exercices 1 et 2, calculez la variance $\text{Var}(X)$.
\end{exercicebox}

\begin{exercicebox}[Exercice 4 : Espérance (Jeu Simple)]
Un jeu consiste à payer 2 pour lancer un dé à 6 faces. Si le dé tombe sur 6, vous gagnez 10. Sinon, vous ne gagnez rien. Soit $G$ votre gain net (gain - mise).
\begin{enumerate}
    \item Quelle est la PMF de $G$ ?
    \item Calculez $E(G)$. Le jeu est-il favorable au joueur ?
\end{enumerate}
\end{exercicebox}

\begin{exercicebox}[Exercice 5 : Variance (Jeu Simple)]
En utilisant la variable aléatoire $G$ de l'exercice 4 :
\begin{enumerate}
    \item Calculez $E(G^2)$.
    \item Calculez $\text{Var}(G)$.
\end{enumerate}
\end{exercicebox}

\begin{exercicebox}[Exercice 6 : Espérance de Bernoulli]
Soit $X$ une variable aléatoire $X \sim \text{Bern}(p)$ (variable indicatrice). En utilisant la définition de l'espérance, montrez que $E(X) = p$.
\end{exercicebox}

\begin{exercicebox}[Exercice 7 : Variance de Bernoulli]
En utilisant le résultat de l'exercice 6 et le théorème de LOTUS, montrez que $\text{Var}(X) = p(1-p)$ pour $X \sim \text{Bern}(p)$. (Indice : $X^2 = X$ pour une variable de Bernoulli).
\end{exercicebox}

% --- Linéarité de l'Espérance ---

\begin{exercicebox}[Exercice 8 : Linéarité (Simple)]
Soient $X$ et $Y$ deux variables aléatoires. On sait que $E(X) = 10$ et $E(Y) = -5$.
Calculez $E(3X - 2Y + 4)$.
\end{exercicebox}

\begin{exercicebox}[Exercice 9 : Linéarité (Trois Dés)]
On lance trois dés équilibrés à 6 faces. Soit $S$ la somme des trois résultats.
En utilisant la linéarité de l'espérance, calculez $E(S)$.
\end{exercicebox}

\begin{exercicebox}[Exercice 10 : Linéarité (Somme de Bernoulli)]
Soit $X \sim \text{Bin}(n, p)$. On rappelle que $X$ peut s'écrire comme la somme de $n$ variables de Bernoulli indépendantes $X = I_1 + \dots + I_n$, où $E(I_j) = p$.
Utilisez la linéarité de l'espérance pour prouver que $E(X) = np$.
\end{exercicebox}

% --- Espérances des Lois Classiques ---

\begin{exercicebox}[Exercice 11 : Espérance Binomiale (Application)]
Un QCM (questionnaire à choix multiples) comporte 40 questions. Chaque question a 4 options de réponse, dont une seule est correcte. Un étudiant répond à tout au hasard.
Quel est le nombre attendu (l'espérance) de bonnes réponses ?
\end{exercicebox}

\begin{exercicebox}[Exercice 12 : Espérance Géométrique (Application)]
On lance une paire de dés équilibrés. Un "succès" est d'obtenir un double-six.
\begin{enumerate}
    \item Quelle est la probabilité $p$ d'un succès ?
    \item Soit $X$ le nombre d'échecs avant le premier double-six. Quelle est l'espérance $E(X)$ ?
\end{enumerate}
\end{exercicebox}

\begin{exercicebox}[Exercice 13 : Espérance Géométrique (Attente Totale)]
En reprenant la situation de l'exercice 12 ($p=1/36$), soit $Y$ le \textit{nombre total de lancers} nécessaires pour obtenir le premier double-six ($Y = X + 1$).
Calculez $E(Y)$.
\end{exercicebox}

\begin{exercicebox}[Exercice 14 : Espérance (Loi Hypergéométrique)]
On tire 5 cartes d'un jeu de 52 cartes sans remise. Soit $X$ le nombre d'As tirés. On peut écrire $X = I_1 + I_2 + I_3 + I_4 + I_5$, où $I_j=1$ si la $j$-ème carte tirée est un As, et 0 sinon.
\begin{enumerate}
    \item Quelle est la probabilité $P(I_1 = 1)$ (que la 1ère carte soit un As) ?
    \item Quelle est la probabilité $P(I_2 = 1)$ (que la 2ème carte soit un As) ? (Indice : Pensez par symétrie ou utilisez la LTP).
    \item Calculez $E(X)$ en utilisant la linéarité.
\end{enumerate}
\end{exercicebox}

% --- Variance et E[X^2] ---

\begin{exercicebox}[Exercice 15 : Espérance et Variance (Dé à 4 faces)]
Soit $X$ le résultat d'un lancer de dé équilibré à 4 faces ($X \in \{1, 2, 3, 4\}$).
\begin{enumerate}
    \item Calculez $E(X)$.
    \item Calculez $E(X^2)$.
    \item Calculez $\text{Var}(X)$.
\end{enumerate}
\end{exercicebox}

\begin{exercicebox}[Exercice 16 : Formule de la Variance (Inverse)]
Une variable aléatoire $Y$ a une espérance $E(Y) = 5$ et une variance $\text{Var}(Y) = 4$.
Quelle est la valeur de $E(Y^2)$ ?
\end{exercicebox}

\begin{exercicebox}[Exercice 17 : Formule de la Variance (Inverse 2)]
Une variable aléatoire $W$ a $E(W^2) = 50$ et $\text{Var}(W) = 1$.
Quelles sont les deux valeurs possibles pour $E(W)$ ?
\end{exercicebox}

\begin{exercicebox}[Exercice 18 : Variance Nulle]
Une variable aléatoire $X$ a une variance $\text{Var}(X) = 0$. Que pouvez-vous conclure sur la distribution de $X$ ?
(Indice : $\text{Var}(X) = E[(X-\mu)^2]$).
\end{exercicebox}

\begin{exercicebox}[Exercice 19 : LOTUS et Linéarité]
Soit $X$ une variable aléatoire avec $E(X)=3$ et $E(X^2)=10$.
Calculez $E[(X+1)^2]$.
(Indice : Développez $(X+1)^2$ avant de prendre l'espérance).
\end{exercicebox}

\begin{exercicebox}[Exercice 20 : Synthèse (Jeu de Roulette)]
À la roulette, vous misez 1 sur "Rouge". Il y a 18 cases rouges, 18 noires, et 1 verte (le 0). Total = 37 cases.
Si "Rouge" sort, vous récupérez votre mise de 1 et gagnez 1 de plus (gain net $G=+1$).
Si "Noir" or "Vert" sort, vous perdez votre mise (gain net $G=-1$).
\begin{enumerate}
    \item Calculez $E(G)$.
    \item Calculez $E(G^2)$.
    \item Calculez $\text{Var}(G)$.
\end{enumerate}
\end{exercicebox}

\subsection{Corrections des Exercices}

% --- Corrections : Concepts de Base (PMF, CDF) ---

\begin{correctionbox}[Correction Exercice 1 : Identification de Variables Aléatoires]
1.  \textbf{Discrète}. $X$ ne peut prendre que des valeurs entières $\{0, 1, \dots, 10\}$.
2.  \textbf{Continue}. Le temps peut prendre n'importe quelle valeur dans un intervalle (par ex. $T \in [2.5, 5]$ heures).
3.  \textbf{Discrète}. $X$ ne peut prendre que des valeurs entières $\{0, 1, 2, \dots\}$.
4.  \textbf{Continue}. La température peut prendre n'importe quelle valeur dans un intervalle (par ex. $T \in [15.0, 25.0]^\circ\text{C}$).
5.  \textbf{Discrète}. $X$ ne peut prendre que des valeurs entières $\{0, 1, 2, \dots\}$ (si on compte les échecs) ou $\{1, 2, 3, \dots\}$ (si on compte les lancers).
\end{correctionbox}

\begin{correctionbox}[Correction Exercice 2 : Construction d'une PMF]
On lance un dé à 4 faces (1, 2, 3, 4). $X$ est le résultat.
1.  Valeurs possibles : $S_X = \{1, 2, 3, 4\}$.
2.  PMF : Le dé est équilibré, donc chaque face a la même probabilité $1/4$.
    $P(X=1) = 1/4$
    $P(X=2) = 1/4$
    $P(X=3) = 1/4$
    $P(X=4) = 1/4$
    Et $P(X=k) = 0$ pour tout autre $k$.
3.  Vérification : $\sum P(X=k) = 1/4 + 1/4 + 1/4 + 1/4 = 4/4 = 1$.
\end{correctionbox}

\begin{correctionbox}[Correction Exercice 3 : PMF d'une Somme]
$Y = D_1 + D_2$, où $D_1, D_2 \in \{1, 2, 3, 4\}$. Il y a $4 \times 4 = 16$ issues équiprobables (prob. 1/16 chacune).
1.  Valeurs possibles : Min = $1+1=2$. Max = $4+4=8$. $S_Y = \{2, 3, 4, 5, 6, 7, 8\}$.
2.  PMF (en comptant les issues favorables sur 16) :
    - $P(Y=2) = P(1,1) \implies 1/16$
    - $P(Y=3) = P(1,2) + P(2,1) \implies 2/16$
    - $P(Y=4) = P(1,3) + P(2,2) + P(3,1) \implies 3/16$
    - $P(Y=5) = P(1,4) + P(2,3) + P(3,2) + P(4,1) \implies 4/16$
    - $P(Y=6) = P(2,4) + P(3,3) + P(4,2) \implies 3/16$
    - $P(Y=7) = P(3,4) + P(4,3) \implies 2/16$
    - $P(Y=8) = P(4,4) \implies 1/16$
    (Vérification : $1+2+3+4+3+2+1 = 16$. La somme est $16/16 = 1$).
\end{correctionbox}

\begin{correctionbox}[Correction Exercice 4 : Construction d'une CDF]
On utilise la PMF de l'exercice 3. $F_Y(y) = P(Y \le y)$.
1.  CDF aux points de masse :
    - $F_Y(2) = P(Y \le 2) = P(Y=2) = 1/16$
    - $F_Y(3) = P(Y \le 3) = P(Y=2)+P(Y=3) = 1/16 + 2/16 = 3/16$
    - $F_Y(4) = P(Y \le 4) = 3/16 + P(Y=4) = 3/16 + 3/16 = 6/16$
    - $F_Y(5) = P(Y \le 5) = 6/16 + P(Y=5) = 6/16 + 4/16 = 10/16$
    - $F_Y(6) = P(Y \le 6) = 10/16 + P(Y=6) = 10/16 + 3/16 = 13/16$
    - $F_Y(7) = P(Y \le 7) = 13/16 + P(Y=7) = 13/16 + 2/16 = 15/16$
    - $F_Y(8) = P(Y \le 8) = 15/16 + P(Y=8) = 15/16 + 1/16 = 16/16 = 1$
2.  $F_Y(1.5) = P(Y \le 1.5) = 0$ (car la valeur minimale est 2).
3.  $F_Y(5.2) = P(Y \le 5.2) = P(Y \le 5) = F_Y(5) = 10/16$.
4.  $F_Y(10) = P(Y \le 10) = 1$ (car la valeur maximale est 8).
\end{correctionbox}

% --- Corrections : Loi de Bernoulli et Loi Binomiale ---

\begin{correctionbox}[Correction Exercice 5 : Loi de Bernoulli]
1.  $X$ suit une \textbf{loi de Bernoulli}. Le paramètre est $p=0.05$. On note $X \sim \text{Bern}(0.05)$.
2.  La PMF est :
    $P(X=1) = p = 0.05$ (succès = défectueux)
    $P(X=0) = 1-p = 0.95$ (échec = non défectueux)
\end{correctionbox}

\begin{correctionbox}[Correction Exercice 6 : Loi Binomiale (Calcul Direct)]
1.  $X$ est le nombre de succès (Pile) en $n=5$ essais indépendants avec probabilité $p=0.7$.
    $X$ suit une \textbf{loi Binomiale}. $X \sim \text{Bin}(n=5, p=0.7)$.
2.  $P(X=k) = \binom{n}{k} p^k (1-p)^{n-k}$.
    $P(X=3) = \binom{5}{3} (0.7)^3 (1-0.7)^{5-3} = 10 \times (0.343) \times (0.3)^2 = 10 \times 0.343 \times 0.09 = 0.3087$.
3.  $P(X=5) = \binom{5}{5} (0.7)^5 (0.3)^0 = 1 \times (0.7)^5 \times 1 = 0.16807$.
\end{correctionbox}

\begin{correctionbox}[Correction Exercice 7 : Loi Binomiale (Calcul Cumulé)]
On a $X \sim \text{Bin}(5, 0.7)$.
1.  $P(X=0) = \binom{5}{0} (0.7)^0 (0.3)^5 = 1 \times 1 \times (0.3)^5 = 0.00243$.
2.  L'événement "au moins 1 Pile" ($X \ge 1$) est le complémentaire de "0 Pile" ($X=0$).
    $P(X \ge 1) = 1 - P(X=0) = 1 - 0.00243 = 0.99757$.
\end{correctionbox}

\begin{correctionbox}[Correction Exercice 8 : Problème Binomial (Contrôle Qualité)]
Le tirage est \textit{avec remise}, donc les essais sont indépendants. C'est une loi binomiale.
$n = 20$ (nombre d'essais).
$p = 0.10$ (probabilité de succès = défectueux).
On cherche $P(X=2)$.
$P(X=2) = \binom{20}{2} (0.1)^2 (1-0.1)^{20-2}$
$P(X=2) = \frac{20 \times 19}{2} (0.1)^2 (0.9)^{18} = 190 \times 0.01 \times (0.9)^{18}$
$P(X=2) = 1.9 \times (0.9)^{18} \approx 1.9 \times 0.15009 \approx 0.2852$.
\end{correctionbox}

% --- Corrections : Loi Hypergéométrique ---

\begin{correctionbox}[Correction Exercice 9 : Loi Hypergéométrique (Urne)]
Le tirage est \textit{sans remise} d'une population finie.
1.  $X$ suit une \textbf{loi Hypergéométrique}.
    Paramètres : $w=7$ (blanches, succès), $b=5$ (noires, échecs), $m=4$ (nombre de tirages).
    $X \sim \text{HG}(w=7, b=5, m=4)$.
2.  On cherche $P(X=2)$.
    $P(X=k) = \frac{\binom{w}{k} \binom{b}{m-k}}{\binom{w+b}{m}}$
    $P(X=2) = \frac{\binom{7}{2} \binom{5}{4-2}}{\binom{12}{4}} = \frac{\binom{7}{2} \binom{5}{2}}{\binom{12}{4}}$
    $P(X=2) = \frac{(\frac{7 \times 6}{2}) \times (\frac{5 \times 4}{2})}{(\frac{12 \times 11 \times 10 \times 9}{4 \times 3 \times 2 \times 1})} = \frac{21 \times 10}{495} = \frac{210}{495} = \frac{14}{33} \approx 0.4242$.
\end{correctionbox}

\begin{correctionbox}[Correction Exercice 10 : Problème Hypergéométrique (Comité)]
Tirage sans remise. C'est une loi Hypergéométrique.
$w=10$ (hommes), $b=8$ (femmes), $m=6$ (taille du comité). Total $N=18$.
On cherche $P(X=3)$ (exactement 3 hommes, ce qui implique $m-k = 6-3=3$ femmes).
$P(X=3) = \frac{\binom{10}{3} \binom{8}{3}}{\binom{18}{6}}$
$P(X=3) = \frac{(\frac{10 \times 9 \times 8}{3 \times 2 \times 1}) \times (\frac{8 \times 7 \times 6}{3 \times 2 \times 1})}{(\frac{18 \times 17 \times 16 \times 15 \times 14 \times 13}{6 \times 5 \times 4 \times 3 \times 2 \times 1})} = \frac{120 \times 56}{18564} = \frac{6720}{18564} \approx 0.362$.
\end{correctionbox}

\begin{correctionbox}[Correction Exercice 11 : Binomiale vs Hypergéométrique]
Population totale $N=10000$. 10\% défectueux, donc $w=1000$ (défectueux), $b=9000$ (non défectueux).
Tirage de $m=20$ \textit{sans remise}.
1.  Loi exacte : \textbf{Loi Hypergéométrique}.
    $X \sim \text{HG}(w=1000, b=9000, m=20)$.
2.  Probabilité exacte $P(X=2)$ :
    $P(X=2) = \frac{\binom{1000}{2} \binom{9000}{18}}{\binom{10000}{20}}$
    $P(X=2) = \frac{(\frac{1000 \times 999}{2}) \times (\frac{9000 \times \dots \times 8983}{18!})}{(\frac{10000 \times \dots \times 9981}{20!})} \approx 0.2854$.
    (Le calcul est très complexe, mais on peut montrer qu'il est très proche de la binomiale).
3.  Le résultat de l'exercice 8 (Binomiale) était $\approx 0.2852$.
    L'approximation binomiale est excellente. La raison est que la taille de l'échantillon ($m=20$) est très petite par rapport à la taille de la population ($N=10000$). Le fait de ne pas remettre les 20 articles change à peine les probabilités pour les tirages suivants.
\end{correctionbox}

% --- Corrections : Loi Géométrique ---

\begin{correctionbox}[Correction Exercice 12 : Loi Géométrique (Calcul Direct)]
1.  $X$ est le nombre d'échecs avant le premier succès. $X$ suit une \textbf{loi Géométrique}.
    Le succès est "obtenir 6", donc $p = 1/6$. $X \sim \text{Geom}(p=1/6)$.
2.  "Premier 6 au 3ème lancer" signifie 2 échecs (lancers 1 et 2) puis 1 succès (lancer 3).
    C'est $P(X=2)$. $q = 1-p = 5/6$.
    $P(X=2) = q^2 p^1 = (5/6)^2 (1/6) = 25/216 \approx 0.1157$.
3.  "Premier 6 au 1er lancer" signifie 0 échec. C'est $P(X=0)$.
    $P(X=0) = q^0 p^1 = 1 \times (1/6) = 1/6$.
\end{correctionbox}

\begin{correctionbox}[Correction Exercice 13 : Loi Géométrique (Calcul Cumulé)]
$p=0.2$ (succès), $q=0.8$ (échec). $X$ compte les échecs. $X \sim \text{Geom}(0.2)$.
1.  "Exactement 4 tirs au total" signifie 3 échecs suivis d'un succès. On cherche $P(X=3)$.
    $P(X=3) = q^3 p^1 = (0.8)^3 (0.2) = 0.512 \times 0.2 = 0.1024$.
2.  "Plus de 2 tirs au total" signifie qu'il faut au moins 3 tirs. C'est l'événement "les 2 premiers tirs sont des échecs".
    La probabilité est $P(\text{Echec 1} \cap \text{Echec 2}) = q \times q = q^2$.
    $P(X \ge 2) = (0.8)^2 = 0.64$.
\end{correctionbox}

\begin{correctionbox}[Correction Exercice 14 : Variante de la Loi Géométrique]
$Y$ est le nombre total d'essais ($k=1, 2, 3, \dots$). $p$ est la prob. de succès.
1.  Pour que $Y=k$, il faut $k-1$ échecs, suivis d'un succès.
    $P(Y=k) = (1-p)^{k-1} p = q^{k-1} p$, pour $k=1, 2, \dots$
2.  Avec $p=1/6$, on cherche $P(Y=3)$.
    $P(Y=3) = (5/6)^{3-1} (1/6) = (5/6)^2 (1/6) = 25/216$.
    C'est le même résultat que $P(X=2)$ de l'exercice 12. Les deux définitions décrivent la même situation (3 lancers au total).
\end{correctionbox}

% --- Corrections : Loi de Poisson ---

\begin{correctionbox}[Correction Exercice 15 : Loi de Poisson (Calcul Direct)]
$X \sim \text{Poisson}(\lambda=5)$. PMF : $P(X=k) = \frac{e^{-\lambda} \lambda^k}{k!}$.
1.  $P(X=0) = \frac{e^{-5} 5^0}{0!} = \frac{e^{-5} \times 1}{1} = e^{-5} \approx 0.0067$.
2.  $P(X=5) = \frac{e^{-5} 5^5}{5!} = \frac{e^{-5} \times 3125}{120} = e^{-5} \times \frac{625}{24} \approx 26.04 \times e^{-5} \approx 0.1755$.
\end{correctionbox}

\begin{correctionbox}[Correction Exercice 16 : Loi de Poisson (Calcul Cumulé)]
$X \sim \text{Poisson}(\lambda=2)$. On cherche $P(X \le 2)$.
$P(X \le 2) = P(X=0) + P(X=1) + P(X=2)$
$P(X=0) = \frac{e^{-2} 2^0}{0!} = e^{-2}$
$P(X=1) = \frac{e^{-2} 2^1}{1!} = 2e^{-2}$
$P(X=2) = \frac{e^{-2} 2^2}{2!} = \frac{4e^{-2}}{2} = 2e^{-2}$
$P(X \le 2) = e^{-2} + 2e^{-2} + 2e^{-2} = 5e^{-2} \approx 5 \times 0.1353 = 0.6767$.
\end{correctionbox}

\begin{correctionbox}[Correction Exercice 17 : Loi de Poisson (Changement de $\lambda$)]
1.  Pour une page, $X \sim \text{Poisson}(\lambda=0.5)$.
    $P(X=0) = \frac{e^{-0.5} (0.5)^0}{0!} = e^{-0.5} \approx 0.6065$.
2.  Si le taux est 0.5 faute/page, le taux pour 10 pages est $\lambda_Y = 0.5 \times 10 = 5$.
    $Y \sim \text{Poisson}(\lambda_Y=5)$.
3.  On cherche $P(Y=0)$.
    $P(Y=0) = \frac{e^{-5} 5^0}{0!} = e^{-5} \approx 0.0067$.
\end{correctionbox}

\begin{correctionbox}[Correction Exercice 18 : Approximation Binomiale par Poisson]
1.  C'est un tirage de $n=10000$ clients, où chaque client est un essai de Bernoulli avec $p=0.0003$. La loi exacte est $X \sim \text{Bin}(10000, 0.0003)$.
2.  Le paramètre $\lambda$ pour l'approximation Poisson est $\lambda = np = 10000 \times 0.0003 = 3$.
3.  On utilise $Y \sim \text{Poisson}(\lambda=3)$ pour approximer $X$.
    $P(X=2) \approx P(Y=2) = \frac{e^{-3} 3^2}{2!} = \frac{9e^{-3}}{2} = 4.5 e^{-3} \approx 4.5 \times 0.04979 \approx 0.224$.
\end{correctionbox}

% --- Corrections : Synthèse et Variables Indicatrices ---

\begin{correctionbox}[Correction Exercice 19 : Choisir la Bonne Loi]
1.  Tirage sans remise d'une population finie : \textbf{Loi Hypergéométrique}.
2.  Comptage d'événements sur un intervalle de temps fixe : \textbf{Loi de Poisson}.
3.  Comptage d'essais jusqu'au premier succès : \textbf{Loi Géométrique}.
4.  Comptage de succès sur un nombre fixe d'essais indépendants : \textbf{Loi Binomiale}.
5.  Comptage d'événements rares sur un intervalle (temps/espace) : \textbf{Loi de Poisson}.
\end{correctionbox}

\begin{correctionbox}[Correction Exercice 20 : Variable Indicatrice]
$A$ = "obtenir 6". $P(A) = 1/6$.
$I_A = 1$ si $A$ se produit, $I_A = 0$ sinon.
1.  C'est une expérience avec deux issues (succès/échec). $I_A$ suit une \textbf{Loi de Bernoulli}.
    Le paramètre est $p = P(A) = 1/6$. $I_A \sim \text{Bern}(1/6)$.
2.  La PMF de $I_A$ est :
    $P(I_A = 1) = p = 1/6$
    $P(I_A = 0) = 1-p = 5/6$
\end{correctionbox}