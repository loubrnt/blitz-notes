
\subsection{Exercices}

Cette série d'exercices vise à renforcer votre compréhension des concepts fondamentaux du dénombrement et de la probabilité naïve. La difficulté augmente progressivement.

%  Concepts de Base et Probabilité Naïve 

\begin{exercicebox}[Exercice 1 : Univers et Événements]
On lance deux dés à 6 faces, un rouge et un bleu.
\begin{enumerate}
    \item Décrivez l'univers $S$ de cette expérience. Quelle est sa taille $|S|$ ?
    \item Soit $A$ l'événement "la somme des dés est égale à 7". Listez les issues appartenant à $A$. Calculez $P(A)$.
    \item Soit $B$ l'événement "le dé rouge montre un 3". Listez les issues appartenant à $B$. Calculez $P(B)$.
    \item Décrivez l'événement $A \cap B$ et calculez sa probabilité.
\end{enumerate}
\end{exercicebox}

\begin{exercicebox}[Exercice 2 : Tirage de Cartes (Prob. Naïve)]
On tire une carte au hasard d'un jeu standard de 52 cartes.
\begin{enumerate}
    \item Quelle est la probabilité de tirer un Roi ?
    \item Quelle est la probabilité de tirer une carte rouge (Cœur ou Carreau) ?
    \item Quelle est la probabilité de tirer une figure (Valet, Dame, Roi) ?
    \item Quelle est la probabilité de tirer un As rouge ?
\end{enumerate}
\end{exercicebox}

\begin{exercicebox}[Exercice 3 : Urne Simple (Prob. Naïve)]
Une urne contient 5 boules rouges, 3 boules bleues et 2 boules vertes. On tire une boule au hasard.
\begin{enumerate}
    \item Quelle est la probabilité qu'elle soit bleue ?
    \item Quelle est la probabilité qu'elle ne soit pas verte ?
\end{enumerate}
\end{exercicebox}

%  Permutations 

\begin{exercicebox}[Exercice 4 : Anagrammes (Permutation Simple)]
Combien d'anagrammes distinctes peut-on former avec les lettres du mot "MATHS" ?
\end{exercicebox}

\begin{exercicebox}[Exercice 5 : Course (Arrangement)]
Dix athlètes participent à une course. Combien y a-t-il de classements possibles pour les 3 premières places (médaille d'or, d'argent, de bronze) ?
\end{exercicebox}

\begin{exercicebox}[Exercice 6 : Anagrammes (Permutation avec Répétition)]
Combien d'anagrammes distinctes peut-on former avec les lettres du mot "PROBABILITE" ?
\end{exercicebox}

%  Combinaisons 

\begin{exercicebox}[Exercice 7 : Choix d'un Comité (Combinaison)]
Une classe compte 15 étudiants. De combien de manières peut-on choisir un comité de 4 étudiants ?
\end{exercicebox}

\begin{exercicebox}[Exercice 8 : Mains de Poker (Combinaison)]
Dans un jeu de 52 cartes, combien de "mains" de 5 cartes différentes peut-on former ?
\end{exercicebox}

\begin{exercicebox}[Exercice 9 : Comité Mixte (Combinaison)]
À partir d'un groupe de 6 hommes et 4 femmes, combien de comités de 3 personnes peut-on former contenant exactement 2 hommes et 1 femme ?
\end{exercicebox}

\begin{exercicebox}[Exercice 10 : Probabilité avec Combinaisons]
On tire simultanément 3 cartes d'un jeu de 52 cartes. Quelle est la probabilité d'obtenir exactement 2 Rois ?
\end{exercicebox}

%  Combinaisons avec Répétition (Étoiles et Bâtons) 

\begin{exercicebox}[Exercice 11 : Distribution de Bonbons (Étoiles et Bâtons)]
De combien de manières peut-on distribuer 8 bonbons identiques à 3 enfants ? (Certains enfants peuvent ne rien recevoir).
\end{exercicebox}

\begin{exercicebox}[Exercice 12 : Solutions d'Équation (Étoiles et Bâtons)]
Combien y a-t-il de solutions entières non négatives ($x_i \ge 0$) à l'équation $x_1 + x_2 + x_3 + x_4 = 10$ ?
\end{exercicebox}

\begin{exercicebox}[Exercice 13 : Distribution avec Minimum (Étoiles et Bâtons avec Contrainte)]
De combien de manières peut-on distribuer 12 pommes identiques à 4 enfants, si chaque enfant doit recevoir au moins une pomme ?
\end{exercicebox}

%  Principe d'Inclusion-Exclusion 

\begin{exercicebox}[Exercice 14 : Divisibilité (Inclusion-Exclusion 2 Ensembles)]
Parmi les entiers de 1 à 100, combien sont divisibles par 2 OU par 3 ?
\end{exercicebox}

\begin{exercicebox}[Exercice 15 : Langues (Inclusion-Exclusion 2 Ensembles)]
Dans un groupe de 50 étudiants, 30 étudient l'anglais, 25 étudient l'espagnol et 10 étudient les deux langues. Combien d'étudiants étudient au moins une de ces deux langues ? Combien n'en étudient aucune ?
\end{exercicebox}

\begin{exercicebox}[Exercice 16 : Divisibilité (Inclusion-Exclusion 3 Ensembles)]
Parmi les entiers de 1 à 100, combien sont divisibles par 2, 3 OU 5 ?
\end{exercicebox}

%  Problèmes Combinés et Plus Difficiles 

\begin{exercicebox}[Exercice 17 : Chemins sur un Grillage (Combinaison)]
Sur un grillage, combien y a-t-il de chemins pour aller du point (0,0) au point (4,3) en se déplaçant uniquement vers la droite (D) ou vers le haut (H) ?
\end{exercicebox}

\begin{exercicebox}[Exercice 18 : Probabilité Hypergéométrique]
Une urne contient 7 boules blanches et 5 boules noires. On tire successivement et sans remise 4 boules. Quelle est la probabilité d'obtenir 2 blanches et 2 noires ?
\end{exercicebox}

\begin{exercicebox}[Exercice 19 : Arrangement Circulaire]
De combien de manières 6 personnes peuvent-elles s'asseoir autour d'une table ronde ? (Deux arrangements sont considérés identiques si chaque personne a les mêmes voisins).
\end{exercicebox}

\begin{exercicebox}[Exercice 20 : Problème des Dérangements (Inclusion-Exclusion)]
Quatre lettres sont adressées à quatre personnes différentes, avec les enveloppes correspondantes. On met chaque lettre au hasard dans une enveloppe. Quelle est la probabilité qu'\textit{aucune} lettre ne soit dans la bonne enveloppe ?
\end{exercicebox}



\subsection{Corrections des Exercices}

%  Corrections : Concepts de Base et Probabilité Naïve 

\begin{correctionbox}[Correction Exercice 1 : Univers et Événements]
1) L'univers $S$ est l'ensemble de toutes les paires $(r, b)$ où $r$ est le résultat du dé rouge et $b$ celui du dé bleu. $S = \{ (1,1), (1,2), \dots, (1,6), (2,1), \dots, (6,6) \}$. La taille de l'univers est $|S| = 6 \times 6 = 36$.

2) L'événement $A$ (somme égale à 7) est $A = \{ (1,6), (2,5), (3,4), (4,3), (5,2), (6,1) \}$. Il y a $|A|=6$ issues favorables. La probabilité est $P(A) = |A|/|S| = 6/36 = 1/6$.

3) L'événement $B$ (dé rouge montre 3) est $B = \{ (3,1), (3,2), (3,3), (3,4), (3,5), (3,6) \}$. Il y a $|B|=6$ issues favorables. La probabilité est $P(B) = |B|/|S| = 6/36 = 1/6$.

4) L'événement $A \cap B$ est l'ensemble des issues où la somme est 7 ET le dé rouge est 3. La seule issue possible est $(3,4)$. Donc $A \cap B = \{ (3,4) \}$. La probabilité est $P(A \cap B) = |A \cap B|/|S| = 1/36$.
\end{correctionbox}

\begin{correctionbox}[Correction Exercice 2 : Tirage de Cartes (Prob. Naïve)]
Le nombre total d'issues est $|S| = 52$.

a) Il y a 4 Rois. $P(\text{Roi}) = 4/52 = 1/13$.

b) Il y a 26 cartes rouges (13 Cœurs + 13 Carreaux). $P(\text{Rouge}) = 26/52 = 1/2$.

c) Il y a 12 figures (4 Valets + 4 Dames + 4 Rois). $P(\text{Figure}) = 12/52 = 3/13$.

d) Il y a 2 As rouges (As de Cœur, As de Carreau). $P(\text{As Rouge}) = 2/52 = 1/26$.
\end{correctionbox}

\begin{correctionbox}[Correction Exercice 3 : Urne Simple (Prob. Naïve)]
Le nombre total de boules est $5+3+2 = 10$.

a) Il y a 3 boules bleues. $P(\text{Bleue}) = 3/10$.

b) L'événement "ne pas être verte" est le complémentaire de "être verte". Il y a 2 boules vertes, donc $P(\text{Verte}) = 2/10$. La probabilité cherchée est $P(\text{Non Verte}) = 1 - P(\text{Verte}) = 1 - 2/10 = 8/10 = 4/5$. (Alternativement, il y a $5+3=8$ boules non vertes, donc $P=8/10$).
\end{correctionbox}

%  Corrections : Permutations 

\begin{correctionbox}[Correction Exercice 4 : Anagrammes (Permutation Simple)]
Le mot "MATHS" a 5 lettres distinctes. Le nombre d'anagrammes est le nombre de permutations de ces 5 lettres, soit $5! = 5 \times 4 \times 3 \times 2 \times 1 = 120$.
\end{correctionbox}

\begin{correctionbox}[Correction Exercice 5 : Course (Arrangement)]
On cherche le nombre de façons d'ordonner 3 athlètes parmi 10. C'est un arrangement (permutation de $k$ parmi $n$) :
$P(10, 3) = \frac{10!}{(10-3)!} = \frac{10!}{7!} = 10 \times 9 \times 8 = 720$.
Il y a 720 podiums possibles.
\end{correctionbox}

\begin{correctionbox}[Correction Exercice 6 : Anagrammes (Permutation avec Répétition)]
Le mot "PROBABILITE" a 11 lettres. Les répétitions sont : B (2 fois), I (2 fois). Les autres lettres (P, R, O, A, L, T, E) apparaissent une fois.
Le nombre d'anagrammes distinctes est :
$$ \frac{11!}{2! \times 2!} = \frac{39,916,800}{2 \times 2} = \frac{39,916,800}{4} = 9,979,200 $$
\end{correctionbox}

%  Corrections : Combinaisons 

\begin{correctionbox}[Correction Exercice 7 : Choix d'un Comité (Combinaison)]
L'ordre ne compte pas, c'est donc une combinaison de 4 étudiants parmi 15 :
$$ \binom{15}{4} = \frac{15!}{4!(15-4)!} = \frac{15!}{4!11!} = \frac{15 \times 14 \times 13 \times 12}{4 \times 3 \times 2 \times 1} = 15 \times 7 \times 13 \times 1 = 1365 $$
Il y a 1365 comités possibles.
\end{correctionbox}

\begin{correctionbox}[Correction Exercice 8 : Mains de Poker (Combinaison)]
On choisit 5 cartes parmi 52, sans ordre. C'est une combinaison :
$$ \binom{52}{5} = \frac{52!}{5!(52-5)!} = \frac{52!}{5!47!} = \frac{52 \times 51 \times 50 \times 49 \times 48}{5 \times 4 \times 3 \times 2 \times 1} = 2,598,960 $$
Il y a 2,598,960 mains de poker possibles.
\end{correctionbox}

\begin{correctionbox}[Correction Exercice 9 : Comité Mixte (Combinaison)]
Il faut choisir 2 hommes parmi 6 ET 1 femme parmi 4. On multiplie les possibilités pour chaque choix :
Nombre de façons = (choix des hommes) $\times$ (choix des femmes)
$$ = \binom{6}{2} \times \binom{4}{1} = \frac{6 \times 5}{2 \times 1} \times \frac{4}{1} = 15 \times 4 = 60 $$
Il y a 60 comités possibles.
\end{correctionbox}

\begin{correctionbox}[Correction Exercice 10 : Probabilité avec Combinaisons]
L'univers $S$ est l'ensemble de toutes les mains de 3 cartes. $|S| = \binom{52}{3}$.
L'événement $A$ est "obtenir exactement 2 Rois". Pour cela, il faut choisir 2 Rois parmi les 4 Rois ET 1 carte qui n'est pas un Roi parmi les 48 autres cartes.
$|A| = \binom{4}{2} \times \binom{48}{1}$.
La probabilité est $P(A) = \frac{|A|}{|S|} = \frac{\binom{4}{2} \binom{48}{1}}{\binom{52}{3}}$.
$$ P(A) = \frac{\frac{4 \times 3}{2 \times 1} \times 48}{\frac{52 \times 51 \times 50}{3 \times 2 \times 1}} = \frac{6 \times 48}{22100} = \frac{288}{22100} \approx 0.013 $$
\end{correctionbox}

%  Corrections : Combinaisons avec Répétition (Étoiles et Bâtons) 

\begin{correctionbox}[Correction Exercice 11 : Distribution de Bonbons (Étoiles et Bâtons)]
C'est un problème de distribution de $k=8$ objets identiques (bonbons) dans $n=3$ boîtes distinctes (enfants). On utilise la formule $\binom{n+k-1}{k}$.
Nombre de manières = $\binom{3+8-1}{8} = \binom{10}{8} = \binom{10}{2} = \frac{10 \times 9}{2 \times 1} = 45$.
\end{correctionbox}

\begin{correctionbox}[Correction Exercice 12 : Solutions d'Équation (Étoiles et Bâtons)]
Cela revient à distribuer $k=10$ unités identiques dans $n=4$ variables distinctes.
Nombre de solutions = $\binom{n+k-1}{k} = \binom{4+10-1}{10} = \binom{13}{10} = \binom{13}{3} = \frac{13 \times 12 \times 11}{3 \times 2 \times 1} = 286$.
\end{correctionbox}

\begin{correctionbox}[Correction Exercice 13 : Distribution avec Minimum (Étoiles et Bâtons avec Contrainte)]
On doit distribuer $k=12$ pommes à $n=4$ enfants, avec $x_i \ge 1$.
On commence par donner une pomme à chaque enfant. Il reste $12 - 4 = 8$ pommes à distribuer sans contrainte (les $x'_i$ peuvent être nuls).
Le problème devient : distribuer $k'=8$ pommes à $n=4$ enfants.
Nombre de manières = $\binom{n+k'-1}{k'} = \binom{4+8-1}{8} = \binom{11}{8} = \binom{11}{3} = \frac{11 \times 10 \times 9}{3 \times 2 \times 1} = 165$.
\end{correctionbox}

%  Corrections : Principe d'Inclusion-Exclusion 

\begin{correctionbox}[Correction Exercice 14 : Divisibilité (Inclusion-Exclusion 2 Ensembles)]
Soit $A$ l'ensemble des entiers $\le 100$ divisibles par 2, et $B$ l'ensemble des entiers $\le 100$ divisibles par 3. On cherche $|A \cup B|$.
$|A| = \lfloor 100/2 \rfloor = 50$.
$|B| = \lfloor 100/3 \rfloor = 33$.
$|A \cap B|$ = ensemble des entiers divisibles par $2 \times 3 = 6$. $|A \cap B| = \lfloor 100/6 \rfloor = 16$.
Par inclusion-exclusion : $|A \cup B| = |A| + |B| - |A \cap B| = 50 + 33 - 16 = 67$.
\end{correctionbox}

\begin{correctionbox}[Correction Exercice 15 : Langues (Inclusion-Exclusion 2 Ensembles)]
Soit $E$ l'ensemble des étudiants étudiant l'anglais, $S$ l'ensemble de ceux étudiant l'espagnol.
$|E| = 30$, $|S| = 25$, $|E \cap S| = 10$.
Nombre d'étudiants étudiant au moins une langue : $|E \cup S| = |E| + |S| - |E \cap S| = 30 + 25 - 10 = 45$.
Nombre total d'étudiants = 50.
Nombre d'étudiants n'étudiant aucune de ces langues = Total - $|E \cup S| = 50 - 45 = 5$.
\end{correctionbox}

\begin{correctionbox}[Correction Exercice 16 : Divisibilité (Inclusion-Exclusion 3 Ensembles)]
Soit $A_2, A_3, A_5$ les ensembles des entiers $\le 100$ divisibles respectivement par 2, 3, 5. On cherche $|A_2 \cup A_3 \cup A_5|$.
$|A_2|=50$, $|A_3|=33$, $|A_5|=20$.
$|A_2 \cap A_3| = |A_6| = \lfloor 100/6 \rfloor = 16$.
$|A_2 \cap A_5| = |A_{10}| = \lfloor 100/10 \rfloor = 10$.
$|A_3 \cap A_5| = |A_{15}| = \lfloor 100/15 \rfloor = 6$.
$|A_2 \cap A_3 \cap A_5| = |A_{30}| = \lfloor 100/30 \rfloor = 3$.
Par inclusion-exclusion :
$|A_2 \cup A_3 \cup A_5| = (|A_2|+|A_3|+|A_5|) - (|A_2 \cap A_3|+|A_2 \cap A_5|+|A_3 \cap A_5|) + |A_2 \cap A_3 \cap A_5|$
$= (50+33+20) - (16+10+6) + 3 = 103 - 32 + 3 = 74$.
\end{correctionbox}

%  Corrections : Problèmes Combinés et Plus Difficiles 

\begin{correctionbox}[Correction Exercice 17 : Chemins sur un Grillage (Combinaison)]
Pour aller de (0,0) à (4,3), il faut faire un total de $4+3=7$ déplacements. Parmi ces 7 déplacements, il faut choisir les 4 moments où l'on va à droite (les 3 autres seront obligatoirement vers le haut), ou choisir les 3 moments où l'on va vers le haut.
Le nombre de chemins est $\binom{7}{4} = \binom{7}{3} = \frac{7 \times 6 \times 5}{3 \times 2 \times 1} = 35$.
\end{correctionbox}

\begin{correctionbox}[Correction Exercice 18 : Probabilité Hypergéométrique]
C'est un tirage sans remise. On peut utiliser la loi hypergéométrique ou le dénombrement.
Population totale = $7+5=12$ boules. On en tire $m=4$.
On veut $k=2$ blanches (parmi $w=7$) et $m-k=2$ noires (parmi $b=5$).
Probabilité = $\frac{\binom{w}{k} \binom{b}{m-k}}{\binom{w+b}{m}} = \frac{\binom{7}{2} \binom{5}{2}}{\binom{12}{4}}$.
$$ P = \frac{(\frac{7 \times 6}{2}) \times (\frac{5 \times 4}{2})}{(\frac{12 \times 11 \times 10 \times 9}{4 \times 3 \times 2 \times 1})} = \frac{21 \times 10}{495} = \frac{210}{495} = \frac{14}{33} \approx 0.424 $$
\end{correctionbox}

\begin{correctionbox}[Correction Exercice 19 : Arrangement Circulaire]
Pour $n$ objets distincts, le nombre d'arrangements circulaires est $(n-1)!$.
Ici, $n=6$. Le nombre de manières est $(6-1)! = 5! = 120$.
L'idée est de fixer une personne, puis d'arranger les 5 autres par rapport à elle.
\end{correctionbox}

\begin{correctionbox}[Correction Exercice 20 : Problème des Dérangements (Inclusion-Exclusion)]
On cherche le nombre de dérangements de 4 éléments, noté $D_4$ ou $!4$. La probabilité sera $D_4 / 4!$.
La formule générale des dérangements (obtenue par inclusion-exclusion) est $D_n = n! \sum_{i=0}^n \frac{(-1)^i}{i!}$.
Pour $n=4$:
$D_4 = 4! (1/0! - 1/1! + 1/2! - 1/3! + 1/4!)$
$D_4 = 24 (1 - 1 + 1/2 - 1/6 + 1/24)$
$D_4 = 24 (1/2 - 1/6 + 1/24) = 24 (12/24 - 4/24 + 1/24) = 24 (9/24) = 9$.
Il y a 9 dérangements possibles sur un total de $4! = 24$ permutations.
La probabilité est $P(\text{aucun match}) = D_4 / 4! = 9/24 = 3/8 = 0.375$.
\end{correctionbox}

\subsection{Exercices Pratiques (Python)}

Ces exercices vous aideront à appliquer les concepts de dénombrement et de probabilité naïve en utilisant les bibliothèques Python.

Pour ces exercices, vous aurez besoin de quelques bibliothèques. \texttt{math} et \texttt{itertools} sont incluses dans Python. Pour le premier exercice, vous aurez besoin de \texttt{scikit-learn} :

\begin{codecell}
pip install scikit-learn pandas
\end{codecell}



\begin{exercicebox}[Exercice 1 : Définition Naïve (Dataset Iris)]
Le dataset "Iris" contient 150 observations de fleurs, réparties équitablement en 3 espèces (50 de chaque). Nous allons l'utiliser pour appliquer la définition naïve de la probabilité.

\textbf{Code utile (chargement des données) :}
\begin{codecell}
from sklearn.datasets import load_iris
import pandas as pd

# Charger le dataset
iris = load_iris()
iris_df = pd.DataFrame(data=iris.data, columns=iris.feature_names)
iris_df["species"] = iris.target

# Le mapping des especes est : 0 = setosa, 1 = versicolor, 2 = virginica
# print(iris_df["species"].value_counts())
\end{codecell}

\textbf{Votre tâche :}
\begin{enumerate}
    \item Définissez l'univers $S$. Quelle est sa taille, $|S|$? (Le nombre total de fleurs).
    \item Définissez l'événement $A$ = "la fleur est de l'espèce 'setosa' (target == 0)". Quelle est la taille de $|A|$?
    \item Calculez $P(A)$ en utilisant la définition naïve : $P(A) = |A| / |S|$.
    \item Calculez $P(B)$ pour l'événement $B$ = "la fleur est de l'espèce 'versicolor' (target == 1)".
\end{enumerate}
\end{exercicebox}



\begin{exercicebox}[Exercice 2 : Permutations (Arrangements)]
En Python, le module \texttt{math} fournit \texttt{math.perm(n, k)} pour calculer $P(n, k)$, et le module \texttt{itertools} fournit \texttt{itertools.permutations} pour les \textit{générer}.

\textbf{Votre tâche :}
Un groupe de 5 coureurs (A, B, C, D, E) participe à une course.
\begin{enumerate}
    \item En utilisant \texttt{math.perm}, calculez combien de podiums (1er, 2e, 3e) différents sont possibles.
    \item Utilisez \texttt{itertools.permutations} pour générer la liste de tous les podiums possibles.
    \item Vérifiez que la longueur de la liste générée à l'étape 2 est égale au nombre calculé à l'étape 1.
\end{enumerate}

\begin{codecell}
import math
from itertools import permutations

coureurs = ["A", "B", "C", "D", "E"]
n = len(coureurs)
k = 3 # Podium

# 1. Calculer avec math.perm
# ... votre code ...

# 2. Generer les listes
podiums_list = list(permutations(coureurs, k))
# print(f"Quelques podiums: {podiums_list[:5]}")

# 3. Verifier
# ... votre code ...
\end{codecell}
\end{exercicebox}



\subsection{Exercices}

%  Concepts de Base et Règle du Produit 

\begin{exercicebox}[Exercice 1 : Dés et Probabilité Conditionnelle Simple]
On lance deux dés équilibrés à 6 faces.
\begin{enumerate}
    \item Quelle est la probabilité que la somme des dés soit 8 ?
    \item Sachant que le premier dé a donné un 3, quelle est la probabilité que la somme soit 8 ?
    \item Sachant que la somme est 8, quelle est la probabilité que le premier dé ait donné un 3 ?
\end{enumerate}
\end{exercicebox}

\begin{exercicebox}[Exercice 2 : Tirage de Cartes (Sans Remise)]
On tire deux cartes successivement et sans remise d'un jeu standard de 52 cartes.
\begin{enumerate}
    \item Quelle est la probabilité que la deuxième carte soit un Roi, sachant que la première était un Roi ?
    \item Quelle est la probabilité de tirer deux Rois ?
\end{enumerate}
\end{exercicebox}

\begin{exercicebox}[Exercice 3 : Urne (Règle du Produit)]
Une urne contient 7 boules rouges et 3 boules bleues. On tire deux boules successivement et sans remise.
\begin{enumerate}
    \item Quelle est la probabilité que la première boule soit rouge ?
    \item Quelle est la probabilité que la deuxième boule soit bleue, sachant que la première était rouge ?
    \item Quelle est la probabilité de tirer une boule rouge puis une boule bleue ?
\end{enumerate}
\end{exercicebox}

\begin{exercicebox}[Exercice 4 : Famille (Condition Simple)]
Une famille a deux enfants. On suppose que la probabilité d'avoir un garçon (G) ou une fille (F) est la même (0.5) et que les naissances sont indépendantes.
\begin{enumerate}
    \item Quel est l'univers $S$ des possibilités ?
    \item Sachant que l'aîné est un garçon, quelle est la probabilité que la famille ait deux garçons ?
\end{enumerate}
\end{exercicebox}

\begin{exercicebox}[Exercice 5 : Famille (Condition "Au Moins")]
En utilisant le même scénario que l'exercice 4 (famille de deux enfants) :
Sachant qu'il y a \textit{au moins un} garçon dans la famille, quelle est la probabilité que la famille ait deux garçons ?
\end{exercicebox}

%  Indépendance 

\begin{exercicebox}[Exercice 6 : Indépendance (Dés)]
On lance deux dés équilibrés.
Soit $A$ l'événement "le premier dé donne 3" et $B$ l'événement "la somme des deux dés est 7".
Les événements $A$ et $B$ sont-ils indépendants ? Justifiez par le calcul.
\end{exercicebox}

\begin{exercicebox}[Exercice 7 : Indépendance (Cartes)]
On tire une carte d'un jeu de 52 cartes.
Soit $A$ l'événement "la carte est un Roi" et $B$ l'événement "la carte est un Cœur".
Les événements $A$ et $B$ sont-ils indépendants ?
\end{exercicebox}

\begin{exercicebox}[Exercice 8 : Indépendance vs Exclusion Mutuelle]
Soient $A$ et $B$ deux événements avec $P(A)=0.5$ et $P(B)=0.3$.
\begin{enumerate}
    \item Si $A$ et $B$ sont mutuellement exclusifs (disjoints), sont-ils indépendants ?
    \item Si $A$ et $B$ sont indépendants, quelle est $P(A \cup B)$ ?
\end{enumerate}
\end{exercicebox}

%  Formule des Probabilités Totales (LTP) 

\begin{exercicebox}[Exercice 9 : LTP (Deux Urnes)]
L'urne U1 contient 2 boules noires et 3 boules blanches. L'urne U2 contient 4 boules noires et 1 boule blanche.
On choisit une urne au hasard (chaque urne a 50\% de chance d'être choisie), puis on tire une boule de cette urne.
Quelle est la probabilité de tirer une boule blanche ?
\end{exercicebox}

\begin{exercicebox}[Exercice 10 : LTP (Usine)]
Une usine utilise deux machines, M1 et M2, pour produire des pièces. M1 produit 40\% des pièces et M2 produit 60\%. 5\% des pièces de M1 sont défectueuses, et 2\% des pièces de M2 sont défectueuses.
Si l'on choisit une pièce au hasard dans la production totale, quelle est la probabilité qu'elle soit défectueuse ?
\end{exercicebox}

\begin{exercicebox}[Exercice 11 : LTP (Pièce de Monnaie Inconnue)]
On a deux pièces. La pièce A est équilibrée ($P(\text{Pile})=0.5$). La pièce B est truquée ($P(\text{Pile})=0.8$).
On choisit une pièce au hasard et on la lance. Quelle est la probabilité d'obtenir Pile ?
\end{exercicebox}

%  Règle de Bayes 

\begin{exercicebox}[Exercice 12 : Bayes (Test Médical)]
Une maladie touche 1 personne sur 1000 ($P(M)=0.001$). Un test de dépistage donne un résultat positif chez 98\% des personnes malades ($P(T|M)=0.98$). Il donne aussi un résultat positif (un "faux positif") chez 3\% des personnes non malades ($P(T|\neg M)=0.03$).
Une personne reçoit un test positif. Quelle est la probabilité qu'elle soit réellement malade ?
\end{exercicebox}

\begin{exercicebox}[Exercice 13 : Bayes (Inversion d'Urnes)]
Reprenons le scénario de l'exercice 9 (U1 avec 2N/3B, U2 avec 4N/1B).
On a tiré une boule et on constate qu'elle est blanche. Quelle est la probabilité qu'elle provienne de l'urne U1 ?
\end{exercicebox}

\begin{exercicebox}[Exercice 14 : Bayes (Spam)]
Dans une boîte de réception, 60\% des emails sont des spams. 70\% des spams contiennent le mot "gratuit". Seuls 10\% des emails légitimes contiennent le mot "gratuit".
Vous recevez un email qui contient le mot "gratuit". Quelle est la probabilité que ce soit un spam ?
\end{exercicebox}

\begin{exercicebox}[Exercice 15 : Bayes (Usine Inversée)]
Reprenons le scénario de l'exercice 10 (M1: 40\% prod, 5\% défaut; M2: 60% prod, 2% défaut).
On trouve une pièce défectueuse. Quelle est la probabilité qu'elle ait été produite par la machine M1 ?
\end{exercicebox}

%  Règle de la Chaîne et Problèmes Combinés 

\begin{exercicebox}[Exercice 16 : Règle de la Chaîne (3 Cartes)]
On tire 3 cartes successivement et sans remise d'un jeu de 52 cartes.
Quelle est la probabilité de tirer 3 Piques ?
\end{exercicebox}

\begin{exercicebox}[Exercice 17 : Problème de Monty Hall (Calcul)]
En utilisant la formalisation du problème de Monty Hall (vous choisissez la Porte 1, la voiture est en $V \in \{1, 2, 3\}$, l'animateur ouvre $H \in \{2, 3\}$) :
Calculez $P(V=1 | H=3)$ (la probabilité que la voiture soit derrière votre porte, sachant que l'animateur a ouvert la 3). Supposez que $P(V=i)=1/3$ pour $i=1,2,3$.
\end{exercicebox}

\begin{exercicebox}[Exercice 18 : Bayes avec Mise à Jour (Pièce Truquée)]
Reprenons l'exercice 11 (Pièce A équilibrée, Pièce B truquée $P(\text{Pile})=0.8$).
On choisit une pièce au hasard. On la lance deux fois et on obtient Pile, puis Pile (PP).
Quelle est la probabilité que l'on ait choisi la pièce truquée (Pièce B) ?
\end{exercicebox}

\begin{exercicebox}[Exercice 19 : Indépendance Conditionnelle (Dés)]
On lance deux dés, $D_1$ et $D_2$. Soit $S = D_1 + D_2$ leur somme.
Soit $A$ l'événement "$D_1 = 1$", $B$ l'événement "$D_2 = 1$".
$A$ et $B$ sont indépendants. Sont-ils indépendants conditionnellement à l'événement $C = \{S = 2\}$ ?
\end{exercicebox}

\begin{exercicebox}[Exercice 20 : Jeu Séquentiel]
Alice et Bob jouent à un jeu. Ils lancent un dé à tour de rôle, en commençant par Alice. Le premier qui obtient un 6 gagne.
Quelle est la probabilité qu'Alice gagne ?
\end{exercicebox}

\subsection{Corrections des Exercices}

%  Corrections : Concepts de Base et Règle du Produit 

\begin{correctionbox}[Correction Exercice 1 : Dés et Probabilité Conditionnelle Simple]
L'univers $S$ a $|S| = 6 \times 6 = 36$ issues.

1.  Soit $A$ l'événement "la somme est 8". $A = \{(2,6), (3,5), (4,4), (5,3), (6,2)\}$.
    $|A|=5$, donc $P(A) = 5/36$.

2.  Soit $B$ l'événement "le premier dé donne 3". $B = \{(3,1), (3,2), (3,3), (3,4), (3,5), (3,6)\}$.
    On cherche $P(A|B)$. Sachant $B$, l'univers est réduit à ces 6 issues. Parmi celles-ci, seule l'issue $(3,5)$ donne une somme de 8.
    Donc, $P(A|B) = 1/6$.
    *Par formule :* $A \cap B = \{(3,5)\}$, $P(A \cap B) = 1/36$. $P(B) = 6/36 = 1/6$.
    $P(A|B) = \frac{P(A \cap B)}{P(B)} = \frac{1/36}{6/36} = 1/6$.

3.  On cherche $P(B|A)$. Sachant $A$, l'univers est réduit aux 5 issues de $A$. Parmi celles-ci, seule l'issue $(3,5)$ a 3 sur le premier dé.
    Donc, $P(B|A) = 1/5$.
    *Par formule :* $P(B|A) = \frac{P(A \cap B)}{P(A)} = \frac{1/36}{5/36} = 1/5$.
\end{correctionbox}

\begin{correctionbox}[Correction Exercice 2 : Tirage de Cartes (Sans Remise)]
Soit $K_1$ l'événement "Roi au 1er tirage" et $K_2$ "Roi au 2e tirage".

1.  On cherche $P(K_2|K_1)$. Si $K_1$ s'est produit, il reste 51 cartes dans le jeu, dont $4-1=3$ Rois.
    $P(K_2|K_1) = 3/51 = 1/17$.

2.  On cherche $P(K_1 \cap K_2)$. On utilise la règle du produit :
    $P(K_1 \cap K_2) = P(K_1) \times P(K_2|K_1)$
    $P(K_1) = 4/52 = 1/13$.
    $P(K_1 \cap K_2) = (4/52) \times (3/51) = (1/13) \times (1/17) = 1/221$.
\end{correctionbox}

\begin{correctionbox}[Correction Exercice 3 : Urne (Règle du Produit)]
Urne avec 7 Rouges (R) et 3 Bleues (B). Total = 10.
Soit $R_1$ "Rouge au 1er tirage" et $B_2$ "Bleue au 2e tirage".

1.  $P(R_1) = 7/10$.
2.  On cherche $P(B_2|R_1)$. Si $R_1$ s'est produit, il reste 9 boules (6R, 3B).
    $P(B_2|R_1) = 3/9 = 1/3$.
3.  On cherche $P(R_1 \cap B_2)$.
    $P(R_1 \cap B_2) = P(R_1) \times P(B_2|R_1) = (7/10) \times (1/3) = 7/30$.
\end{correctionbox}

\begin{correctionbox}[Correction Exercice 4 : Famille (Condition Simple)]
1.  L'univers est $S = \{GG, GF, FG, FF\}$, où le premier enfant est l'aîné. $|S|=4$, chaque issue a une probabilité de 1/4.
2.  Soit $A$ l'événement "l'aîné est un garçon" : $A = \{GG, GF\}$. $P(A) = 2/4 = 1/2$.
    Soit $B$ l'événement "la famille a deux garçons" : $B = \{GG\}$. $P(B) = 1/4$.
    On cherche $P(B|A)$. L'événement $A \cap B = \{GG\}$. $P(A \cap B) = 1/4$.
    $P(B|A) = \frac{P(A \cap B)}{P(A)} = \frac{1/4}{1/2} = 1/2$.
\end{correctionbox}

\begin{correctionbox}[Correction Exercice 5 : Famille (Condition "Au Moins")]
Soit $B$ l'événement "la famille a deux garçons" : $B = \{GG\}$.
Soit $C$ l'événement "il y a au moins un garçon" : $C = \{GG, GF, FG\}$. $P(C) = 3/4$.
On cherche $P(B|C)$.
L'événement $B \cap C = \{GG\}$. $P(B \cap C) = 1/4$.
$P(B|C) = \frac{P(B \cap C)}{P(C)} = \frac{1/4}{3/4} = 1/3$.
*Intuition :* L'univers de $C$ est $\{GG, GF, FG\}$. Parmi ces 3 issues équiprobables, une seule est $GG$.
\end{correctionbox}

%  Corrections : Indépendance 

\begin{correctionbox}[Correction Exercice 6 : Indépendance (Dés)]
$A$ = "premier dé = 3". $P(A) = 6/36 = 1/6$.
$B$ = "somme = 7". $B = \{(1,6), (2,5), (3,4), (4,3), (5,2), (6,1)\}$. $P(B) = 6/36 = 1/6$.
$A \cap B$ = "premier dé = 3 ET somme = 7" = $\%(3,4)\}$. $P(A \cap B) = 1/36$.
On teste si $P(A \cap B) = P(A)P(B)$.
$P(A)P(B) = (1/6) \times (1/6) = 1/36$.
Puisque $P(A \cap B) = P(A)P(B)$, les événements $A$ et $B$ sont indépendants.
\end{correctionbox}

\begin{correctionbox}[Correction Exercice 7 : Indépendance (Cartes)]
$A$ = "Roi". $P(A) = 4/52 = 1/13$.
$B$ = "Cœur". $P(B) = 13/52 = 1/4$.
$A \cap B$ = "Roi de Cœur". $P(A \cap B) = 1/52$.
On teste si $P(A \cap B) = P(A)P(B)$.
$P(A)P(B) = (1/13) \times (1/4) = 1/52$.
Puisque $P(A \cap B) = P(A)P(B)$, les événements $A$ et $B$ sont indépendants.
\end{correctionbox}

\begin{correctionbox}[Correction Exercice 8 : Indépendance vs Exclusion Mutuelle]
$P(A)=0.5, P(B)=0.3$.

1.  Si $A$ et $B$ sont mutuellement exclusifs, $A \cap B = \emptyset$, donc $P(A \cap B) = 0$.
    Pour qu'ils soient indépendants, il faudrait $P(A \cap B) = P(A)P(B) = 0.5 \times 0.3 = 0.15$.
    Puisque $0 \neq 0.15$, ils ne sont pas indépendants. (Deux événements non impossibles ne peuvent pas être à la fois mutuellement exclusifs et indépendants).

2.  Si $A$ et $B$ sont indépendants, $P(A \cap B) = P(A)P(B) = 0.15$.
    $P(A \cup B) = P(A) + P(B) - P(A \cap B) = 0.5 + 0.3 - 0.15 = 0.65$.
\end{correctionbox}

%  Corrections : Formule des Probabilités Totales (LTP) 

\begin{correctionbox}[Correction Exercice 9 : LTP (Deux Urnes)]
Soit $U_1$ et $U_2$ les événements "choisir l'urne 1" et "choisir l'urne 2". $P(U_1)=0.5, P(U_2)=0.5$.
Soit $W$ l'événement "tirer une boule blanche".
On a $P(W|U_1) = 3 / (2+3) = 3/5 = 0.6$.
On a $P(W|U_2) = 1 / (4+1) = 1/5 = 0.2$.
Par la formule des probabilités totales :
$P(W) = P(W|U_1)P(U_1) + P(W|U_2)P(U_2)$
$P(W) = (0.6 \times 0.5) + (0.2 \times 0.5) = 0.3 + 0.1 = 0.4$.
\end{correctionbox}

\begin{correctionbox}[Correction Exercice 10 : LTP (Usine)]
Soit $M_1$ et $M_2$ les machines. $P(M_1)=0.4, P(M_2)=0.6$.
Soit $D$ l'événement "la pièce est défectueuse".
On a $P(D|M_1) = 0.05$ et $P(D|M_2) = 0.02$.
Par la formule des probabilités totales :
$P(D) = P(D|M_1)P(M_1) + P(D|M_2)P(M_2)$
$P(D) = (0.05 \times 0.4) + (0.02 \times 0.6) = 0.020 + 0.012 = 0.032$.
La probabilité est de 3.2\%.
\end{correctionbox}

\begin{correctionbox}[Correction Exercice 11 : LTP (Pièce de Monnaie Inconnue)]
Soit $A$ "choisir pièce A" et $B$ "choisir pièce B". $P(A)=0.5, P(B)=0.5$.
Soit $H$ l'événement "obtenir Pile".
On a $P(H|A) = 0.5$ et $P(H|B) = 0.8$.
Par la formule des probabilités totales :
$P(H) = P(H|A)P(A) + P(H|B)P(B)$
$P(H) = (0.5 \times 0.5) + (0.8 \times 0.5) = 0.25 + 0.40 = 0.65$.
\end{correctionbox}

%  Corrections : Règle de Bayes 

\begin{correctionbox}[Correction Exercice 12 : Bayes (Test Médical)]
Soit $M$ "Malade" et $T$ "Test Positif".
$P(M) = 0.001$, donc $P(\neg M) = 0.999$.
$P(T|M) = 0.98$.
$P(T|\neg M) = 0.03$.
On cherche $P(M|T)$. Par la règle de Bayes : $P(M|T) = \frac{P(T|M)P(M)}{P(T)}$.

1.  Calculer $P(T)$ (dénominateur) avec la LTP :
    $P(T) = P(T|M)P(M) + P(T|\neg M)P(\neg M)$
    $P(T) = (0.98 \times 0.001) + (0.03 \times 0.999) = 0.00098 + 0.02997 = 0.03095$.

2.  Appliquer la règle de Bayes :
    $P(M|T) = \frac{0.00098}{0.03095} \approx 0.03166$.
    Il n'y a que 3.17\% de chance que la personne soit malade, même avec un test positif.
\end{correctionbox}

\begin{correctionbox}[Correction Exercice 13 : Bayes (Inversion d'Urnes)]
D'après l'exercice 9, on a :
$P(W) = 0.4$ (prob. totale de tirer une blanche).
$P(W|U_1) = 0.6$.
$P(U_1) = 0.5$.
On cherche $P(U_1|W)$. Par la règle de Bayes :
$P(U_1|W) = \frac{P(W|U_1)P(U_1)}{P(W)} = \frac{0.6 \times 0.5}{0.4} = \frac{0.3}{0.4} = 0.75$.
Sachant que la boule est blanche, il y a 75\% de chance qu'elle vienne de l'urne U1.
\end{correctionbox}

\begin{correctionbox}[Correction Exercice 14 : Bayes (Spam)]
Soit $S$ "Spam" et $G$ "Contient 'gratuit'".
$P(S) = 0.6$, donc $P(\neg S) = 0.4$.
$P(G|S) = 0.7$.
$P(G|\neg S) = 0.1$.
On cherche $P(S|G)$. Par la règle de Bayes : $P(S|G) = \frac{P(G|S)P(S)}{P(G)}$.

1.  Calculer $P(G)$ (dénominateur) avec la LTP :
    $P(G) = P(G|S)P(S) + P(G|\neg S)P(\neg S)$
    $P(G) = (0.7 \times 0.6) + (0.1 \times 0.4) = 0.42 + 0.04 = 0.46$.

2.  Appliquer la règle de Bayes :
    $P(S|G) = \frac{0.42}{0.46} \approx 0.913$.
    Il y a 91.3\% de chance que l'email soit un spam.
\end{correctionbox}

\begin{correctionbox}[Correction Exercice 15 : Bayes (Usine Inversée)]
D'après l'exercice 10, on a :
$P(D) = 0.032$ (prob. totale d'être défectueux).
$P(D|M_1) = 0.05$.
$P(M_1) = 0.4$.
On cherche $P(M_1|D)$. Par la règle de Bayes :
$P(M_1|D) = \frac{P(D|M_1)P(M_1)}{P(D)} = \frac{0.05 \times 0.4}{0.032} = \frac{0.02}{0.032} = 0.625$.
Sachant que la pièce est défectueuse, il y a 62.5\% de chance qu'elle vienne de M1.
\end{correctionbox}

%  Corrections : Règle de la Chaîne et Problèmes Combinés 

\begin{correctionbox}[Correction Exercice 16 : Règle de la Chaîne (3 Cartes)]
Soit $P_i$ l'événement "tirer un Pique au $i$-ème tirage". Il y a 13 Piques sur 52 cartes.
On cherche $P(P_1 \cap P_2 \cap P_3)$. On utilise la règle de la chaîne :
$P(P_1 \cap P_2 \cap P_3) = P(P_1) \times P(P_2|P_1) \times P(P_3|P_1 \cap P_2)$
$P(P_1) = 13/52$.
$P(P_2|P_1) = 12/51$ (il reste 12 Piques sur 51 cartes).
$P(P_3|P_1 \cap P_2) = 11/50$ (il reste 11 Piques sur 50 cartes).
$P = (13/52) \times (12/51) \times (11/50) = \frac{1}{4} \times \frac{4}{17} \times \frac{11}{50} = \frac{11}{17 \times 50} = 11/850 \approx 0.0129$.
\end{correctionbox}

\begin{correctionbox}[Correction Exercice 17 : Problème de Monty Hall (Calcul)]
On cherche $P(V=1 | H=3)$. On utilise la règle de Bayes :
$P(V=1 | H=3) = \frac{P(H=3 | V=1) P(V=1)}{P(H=3)}$.

*Numérateur :* $P(V=1) = 1/3$. $P(H=3 | V=1)$ est la probabilité que l'animateur ouvre la 3, sachant que vous avez choisi la 1 et que la voiture est en 1. Il peut ouvrir la 2 ou la 3 (deux chèvres). On suppose qu'il choisit au hasard : $P(H=3 | V=1) = 1/2$.
Numérateur = $(1/2) \times (1/3) = 1/6$.

*Dénominateur $P(H=3)$ par LTP (partition sur V) :*
$P(H=3) = P(H=3|V=1)P(V=1) + P(H=3|V=2)P(V=2) + P(H=3|V=3)P(V=3)$
- $P(H=3|V=1) = 1/2$ (calculé ci-dessus).
- $P(H=3|V=2) = 1$ (l'animateur doit ouvrir la 3, car vous avez choisi 1 et la voiture est en 2).
- $P(H=3|V=3) = 0$ (l'animateur ne peut pas ouvrir la porte 3 car elle contient la voiture).
$P(H=3) = (1/2 \times 1/3) + (1 \times 1/3) + (0 \times 1/3) = 1/6 + 1/3 + 0 = 1/2$.

*Résultat :* $P(V=1 | H=3) = \frac{1/6}{1/2} = 1/3$.
(La probabilité que la voiture soit derrière votre porte reste 1/3. La probabilité qu'elle soit derrière l'autre porte fermée (la 2) est $P(V=2|H=3) = 1 - P(V=1|H=3) = 2/3$. Il faut donc changer.)
\end{correctionbox}

\begin{correctionbox}[Correction Exercice 18 : Bayes avec Mise à Jour (Pièce Truquée)]
Soit $A$ "pièce A (équil.)" et $B$ "pièce B (truquée, p=0.8)". $P(A)=P(B)=0.5$.
Soit $E$ l'événement "obtenir Pile, Pile" (PP).
On cherche $P(B|E) = \frac{P(E|B)P(B)}{P(E)}$.

1.  Probabilités conditionnelles de l'évidence $E$ :
    $P(E|A) = P(\text{PP} | A) = 0.5 \times 0.5 = 0.25$ (indépendance des lancers).
    $P(E|B) = P(\text{PP} | B) = 0.8 \times 0.8 = 0.64$.

2.  Calculer $P(E)$ (dénominateur) avec la LTP :
    $P(E) = P(E|A)P(A) + P(E|B)P(B)$
    $P(E) = (0.25 \times 0.5) + (0.64 \times 0.5) = 0.125 + 0.320 = 0.445$.

3.  Appliquer la règle de Bayes :
    $P(B|E) = \frac{P(E|B)P(B)}{P(E)} = \frac{0.64 \times 0.5}{0.445} = \frac{0.32}{0.445} \approx 0.719$.
    Après avoir observé PP, la probabilité que ce soit la pièce truquée passe de 50\% à 71.9\%.
\end{correctionbox}

\begin{correctionbox}[Correction Exercice 19 : Indépendance Conditionnelle (Dés)]
$A = \{D_1=1\}$, $B = \{D_2=1\}$, $C = \{S=2\}$.
On teste si $P(A \cap B | C) = P(A|C) P(B|C)$.

L'événement $C = \{S=2\}$ ne peut se produire que d'une seule façon : $C = \{(1,1)\}$.
Donc, $C$ est l'événement $A \cap B$. $C \subseteq A$ et $C \subseteq B$.

Calculons les termes :
- $P(A|C) = P(A \cap C) / P(C)$. Puisque $C \subseteq A$, $A \cap C = C$.
  $P(A|C) = P(C) / P(C) = 1$.
- $P(B|C) = P(B \cap C) / P(C)$. Puisque $C \subseteq B$, $B \cap C = C$.
  $P(B|C) = P(C) / P(C) = 1$.
- $P(A \cap B | C) = P((A \cap B) \cap C) / P(C)$. Puisque $A \cap B = C$, $(A \cap B) \cap C = C$.
  $P(A \cap B | C) = P(C) / P(C) = 1$.

Test d'indépendance :
$P(A \cap B | C) = 1$.
$P(A|C) P(B|C) = 1 \times 1 = 1$.
Puisque $1=1$, les événements $A$ et $B$ sont bien indépendants conditionnellement à $C$.
*Intuition :* Sachant que la somme est 2, nous savons avec certitude que $D_1=1$ et $D_2=1$. Il n'y a plus d'aléa.
\end{correctionbox}

\begin{correctionbox}[Correction Exercice 20 : Jeu Séquentiel]
Soit $p=1/6$ la probabilité de gagner (obtenir un 6) et $q=5/6$ la probabilité de rater.
Alice gagne si elle réussit au tour 1, OU si (elle rate ET Bob rate) et elle réussit au tour 3, OU si (A rate, B rate, A rate, B rate) et elle réussit au tour 5, etc.

$P(\text{A gagne}) = P(\text{A au tour 1}) + P(\text{A au tour 3}) + P(\text{A au tour 5}) + \dots$
$P(\text{A gagne}) = p + (q \times q)p + (q \times q \times q \times q)p + \dots$
$P(\text{A gagne}) = p + q^2 p + q^4 p + \dots$
$P(\text{A gagne}) = p \times (1 + q^2 + q^4 + \dots)$
$P(\text{A gagne}) = p \sum_{k=0}^{\infty} (q^2)^k$

C'est une série géométrique de premier terme $p$ et de raison $r = q^2 = (5/6)^2 = 25/36$.
La somme est $\frac{\text{premier terme}}{1 - \text{raison}} = \frac{p}{1 - q^2}$.
$P(\text{A gagne}) = \frac{1/6}{1 - 25/36} = \frac{1/6}{11/36} = \frac{1}{6} \times \frac{36}{11} = 6/11$.
\end{correctionbox}

\subsection{Exercices Pratiques (Python)}

Ces exercices vous aideront à appliquer les concepts de probabilité conditionnelle, d'indépendance et la règle de Bayes en utilisant Python, principalement avec le dataset "Titanic".

Pour ces exercices, vous aurez besoin des bibliothèques \texttt{pandas}, \texttt{seaborn} et \texttt{numpy}.

\begin{codecell}
pip install pandas seaborn numpy
\end{codecell}

\textbf{Code utile (chargement des données pour les exercices 1 à 4) :}
\begin{codecell}
import pandas as pd
import seaborn as sns
import numpy as np

# Charger le dataset Titanic
df = sns.load_dataset('titanic')

# Simplification pour les exercices:
# Remplir les ages manquants par la moyenne
df["age"] = df["age"].fillna(df["age"].mean())
# Supprimer les lignes ou 'embarked' ou 'deck' est manquant
df = df.dropna(subset=["embarked', 'deck"])

# Vous pouvez decommentez la ligne suivante pour inspecter le DataFrame
# df.info()
\end{codecell}



\begin{exercicebox}[Exercice 1 : Définition de la Probabilité Conditionnelle]
En utilisant le DataFrame \texttt{df} du Titanic, nous allons calculer $P(A|B)$.
Soit $A$ l'événement "le passager a survécu" (\texttt{survived == 1}).
Soit $B$ l'événement "le passager était un homme" (\texttt{sex == 'male'}).

\textbf{Votre tâche :}
Calculez $P(A|B) = P(\text{survived} | \text{male})$ en suivant la définition :
$$P(A|B) = \frac{P(A \cap B)}{P(B)}$$
\begin{enumerate}
    \item Calculez $P(B) = P(\text{male})$. (Nombre d'hommes / Nombre total de passagers).
    \item Calculez $P(A \cap B) = P(\text{survived} \cap \text{male})$. (Nombre d'hommes survivants / Nombre total).
    \item Divisez les deux pour trouver $P(A|B)$.
\end{enumerate}

\begin{codecell}
# Definit le nombre total de passagers (notre univers |S|)
total_passagers = len(df)

# 1. Calculer P(B) = P(male)
# ... votre code ...
p_male = 0 # A remplacer
print(f"P(male) = {p_male:.4f}")

# 2. Calculer P(A inter B) = P(survived ET male)
# ... votre code ...
p_male_survived = 0 # A remplacer
print(f"P(male ET survived) = {p_male_survived:.4f}")

# 3. Calculer P(A|B) = P(survived | male)
# ... votre code ...
p_survived_given_male = 0 # A remplacer
print(f"P(survived | male) = {p_survived_given_male:.4f}")

# Verification (methode directe, optionnelle)
# p_survived_given_male_direct = ...
# print(f"Verification directe: {p_survived_given_male_direct:.4f}")
\end{codecell}
\end{exercicebox}



\begin{exercicebox}[Exercice 2 : Indépendance]
Les événements $A$ et $B$ sont indépendants si $P(A \cap B) = P(A)P(B)$.
Soit $A$ l'événement "le passager a survécu" (\texttt{survived == 1}).
Soit $C$ l'événement "le passager était en 1ère classe" (\texttt{pclass == 1}).

\textbf{Votre tâche :}
Vérifiez si les événements $A$ et $C$ sont indépendants dans notre dataset.
\begin{enumerate}
    \item Calculez $P(A) = P(\text{survived})$.
    \item Calculez $P(C) = P(\text{pclass} == 1)$.
    \item Calculez $P(A) \times P(C)$.
    \item Calculez $P(A \cap C) = P(\text{survived} \cap \text{pclass} == 1)$.
    \item Comparez les résultats de 3 et 4. Sont-ils indépendants ?
\end{enumerate}

\begin{codecell}
total_passagers = len(df)

# 1. P(A) = P(survived)
# ... votre code ...
p_survived = 0 # A remplacer
print(f"P(survived) = {p_survived:.4f}")

# 2. P(C) = P(pclass == 1)
# ... votre code ...
p_pclass_1 = 0 # A remplacer
print(f"P(pclass == 1) = {p_pclass_1:.4f}")

# 3. P(A) * P(C)
p_a_fois_p_c = p_survived * p_pclass_1
print(f"P(A) * P(C) = {p_a_fois_p_c:.4f}")

# 4. P(A inter C) = P(survived ET pclass == 1)
# ... votre code ...
p_a_inter_c = 0 # A remplacer
print(f"P(A inter C) = {p_a_inter_c:.4f}")

# 5. Comparaison
# ... votre code ...
\end{codecell}
\end{exercicebox}



\begin{exercicebox}[Exercice 3 : Formule des Probabilités Totales]
Calculez la probabilité totale de survie $P(A) = P(\text{survived})$ en utilisant la formule des probabilités totales, partitionnée par la classe du passager (\texttt{pclass} 1, 2, et 3).
La formule est :
$$P(A) = \sum_{i=1}^{3} P(A | C_i) P(C_i)$$
où $C_i$ est l'événement "le passager est en classe $i$".

\textbf{Votre tâche :}
\begin{enumerate}
    \item Calculez $P(C_1)$, $P(C_2)$, $P(C_3)$.
    \item Calculez $P(A | C_1)$, $P(A | C_2)$, $P(A | C_3)$.
    \item Appliquez la formule pour trouver $P(A)$.
\end{enumerate}

\begin{codecell}
total_passagers = len(df)
p_total_survived = 0.0

# Boucle sur les classes 1, 2, 3
for i in [1, 2, 3]:
    # P(Ci)
    # ... votre code ...
    p_classe_i = 0 # A remplacer
    
    # P(A | Ci) = P(survived | pclass == i)
    # ... votre code ...
    p_surv_given_classe_i = 0.0 # A remplacer
        
    print(f"Classe {i}: P(C{i})={p_classe_i:.3f}, P(A|C{i})={p_surv_given_classe_i:.3f}")
    
    # Somme
    p_total_survived += p_surv_given_classe_i * p_classe_i

print(f"Probabilite totale de survie (calculee): {p_total_survived:.4f}")
\end{codecell}
\end{exercicebox}



\begin{exercicebox}[Exercice 4 : Règle de Bayes]
Utilisons la règle de Bayes pour "inverser" une probabilité. Nous voulons trouver $P(C_1 | A)$, c'est-à-dire : "sachant qu'un passager a survécu, quelle est la probabilité qu'il était en 1ère classe ?"
La formule est :
$$P(C_1 | A) = \frac{P(A | C_1) P(C_1)}{P(A)}$$

\textbf{Votre tâche :}
En utilisant les valeurs que vous avez calculées dans les exercices 2 et 3 :
\begin{enumerate}
    \item $P(A | C_1) = P(\text{survived} | \text{pclass} == 1)$
    \item $P(C_1) = P(\text{pclass} == 1)$
    \item $P(A) = P(\text{survived})$
    \item Appliquez la formule de Bayes pour trouver $P(C_1 | A)$.
\end{enumerate}

\begin{codecell}
# Recuperer les valeurs des exercices precedents
p_a_given_c1 = 0 # A remplacer
p_c1 = 0         # A remplacer
p_a = 0          # A remplacer

# 4. Appliquer Bayes
p_c1_given_a = 0 # A remplacer

print(f"P(pclass 1 | survived) = {p_c1_given_a:.4f}")

# Verification (methode directe, optionnelle)
# ... votre code ...
\end{codecell}
\end{exercicebox}



\begin{exercicebox}[Exercice 5 : Simulation du Problème de Monty Hall]
Pour prouver le résultat contre-intuitif de Monty Hall, nous allons le simuler.

\textbf{Votre tâche :}
Complétez le code ci-dessous pour simuler $N=10000$ parties.
\begin{enumerate}
    \item Simulez la stratégie "Garder" (stay) : comptez 1 victoire si \texttt{choix\_initial == voiture}.
    \item Simulez la stratégie "Changer" (switch) : comptez 1 victoire si \texttt{choix\_initial != voiture}.
    \item Calculez les taux de victoire pour les deux stratégies.
\end{enumerate}
(Note : La logique pour "Changer" est simplifiée. Si votre choix initial est faux (2/3 des cas), l'animateur ouvre l'autre mauvaise porte, et changer vous fait gagner. Si votre choix initial est bon (1/3 des cas), changer vous fait perdre.)

\begin{codecell}
import random

def simuler_monty_hall(N_simulations):
    victoires_garder = 0
    victoires_changer = 0
    
    portes = [1, 2, 3]
    
    for _ in range(N_simulations):
        # 1. Placer la voiture et faire le choix initial
        voiture = random.choice(portes)
        choix_initial = random.choice(portes)
        
        # 2. Simuler la strategie "Garder"
        # ... votre code ...
            
        # 3. Simuler la strategie "Changer"
        # ... votre code ...
            
    # 4. Calculer les taux
    taux_victoire_garder = victoires_garder / N_simulations
    taux_victoire_changer = victoires_changer / N_simulations
    
    return taux_victoire_garder, taux_victoire_changer

# Lancer la simulation
N = 100000
garder, changer = simuler_monty_hall(N)

print(f"Simulations: {N}")
print(f"Taux de victoire (Garder): {garder:.4f}")
print(f"Taux de victoire (Changer): {changer:.4f}")
\end{codecell}
\end{exercicebox}




\subsection{Exercices}

% --- Concepts de Base (PMF, CDF) ---

\begin{exercicebox}[Exercice 1 : Identification de Variables Aléatoires]
Pour chacune des situations suivantes, indiquez si la variable aléatoire $X$ est discrète ou continue.
\begin{enumerate}
    \item $X$ est le nombre de Piles obtenues en lançant 10 fois une pièce.
    \item $X$ est le temps exact nécessaire pour courir un marathon.
    \item $X$ est le nombre d'emails que vous recevez un jour donné.
    \item $X$ est la température exacte à midi à Paris.
    \item $X$ est le nombre de lancers d'un dé jusqu'à obtenir un 6.
\end{enumerate}
\end{exercicebox}

\begin{exercicebox}[Exercice 2 : Construction d'une PMF]
On lance un dé équilibré à 4 faces (un tétraèdre) numérotées de 1 à 4. Soit $X$ le résultat du lancer.
\begin{enumerate}
    \item Quelles sont les valeurs possibles pour $X$ ?
    \item Donnez la fonction de masse de probabilité (PMF) $P(X=k)$ pour chaque valeur $k$.
    \item Vérifiez que la somme des probabilités est égale à 1.
\end{enumerate}
\end{exercicebox}

\begin{exercicebox}[Exercice 3 : PMF d'une Somme]
On lance deux dés équilibrés à 4 faces (ceux de l'exercice 2). Soit $Y$ la somme des deux résultats.
\begin{enumerate}
    \item Quelles sont les valeurs possibles pour $Y$ ?
    \item Calculez la PMF $P(Y=k)$ pour chaque valeur $k$ possible (Indice : listez les 16 issues possibles).
\end{enumerate}
\end{exercicebox}

\begin{exercicebox}[Exercice 4 : Construction d'une CDF]
En utilisant la variable $Y$ et sa PMF de l'exercice 3 (somme de deux dés à 4 faces) :
\begin{enumerate}
    \item Calculez $F_Y(y) = P(Y \le y)$ pour toutes les valeurs $y$ de 2 à 8.
    \item Quelle est la valeur de $F_Y(1.5)$ ?
    \item Quelle est la valeur de $F_Y(5.2)$ ?
    \item Quelle est la valeur de $F_Y(10)$ ?
\end{enumerate}
\end{exercicebox}

% --- Loi de Bernoulli et Loi Binomiale ---

\begin{exercicebox}[Exercice 5 : Loi de Bernoulli]
Une machine produit des pièces, avec une probabilité $p=0.05$ que la pièce soit défectueuse. Soit $X$ une variable aléatoire qui vaut 1 si une pièce est défectueuse et 0 sinon.
\begin{enumerate}
    \item Quelle loi suit $X$ ? Donnez son (ou ses) paramètre(s).
    \item Quelle est la PMF de $X$ ? (Donnez $P(X=0)$ et $P(X=1)$).
\end{enumerate}
\end{exercicebox}

\begin{exercicebox}[Exercice 6 : Loi Binomiale (Calcul Direct)]
On lance une pièce truquée 5 fois ($n=5$). La probabilité d'obtenir Pile (succès) est $p=0.7$. Soit $X$ le nombre de Piles obtenus.
\begin{enumerate}
    \item Quelle loi suit $X$ ? Donnez ses paramètres.
    \item Quelle est la probabilité d'obtenir exactement 3 Piles, $P(X=3)$ ?
    \item Quelle est la probabilité d'obtenir exactement 5 Piles, $P(X=5)$ ?
\end{enumerate}
\end{exercicebox}

\begin{exercicebox}[Exercice 7 : Loi Binomiale (Calcul Cumulé)]
On reprend la situation de l'exercice 6 ($X \sim \text{Bin}(5, 0.7)$).
\begin{enumerate}
    \item Quelle est la probabilité d'obtenir 0 Pile, $P(X=0)$ ?
    \item En déduire la probabilité d'obtenir au moins 1 Pile, $P(X \ge 1)$.
\end{enumerate}
\end{exercicebox}

\begin{exercicebox}[Exercice 8 : Problème Binomial (Contrôle Qualité)]
Un lot de 10000 articles contient 10\% d'articles défectueux. On prélève un échantillon de 20 articles \textit{avec remise} pour inspection.
Quelle est la probabilité que l'échantillon contienne exactement 2 articles défectueux ?
\end{exercicebox}

% --- Loi Hypergéométrique ---

\begin{exercicebox}[Exercice 9 : Loi Hypergéométrique (Urne)]
Une urne contient 7 boules blanches et 5 boules noires (total 12). On tire $m=4$ boules \textit{sans remise}. Soit $X$ le nombre de boules blanches tirées.
\begin{enumerate}
    \item Quelle loi suit $X$ ? Donnez ses paramètres ($w, b, m$).
    \item Quelle est la probabilité d'obtenir exactement 2 boules blanches, $P(X=2)$ ?
\end{enumerate}
\end{exercicebox}

\begin{exercicebox}[Exercice 10 : Problème Hypergéométrique (Comité)]
Un département est composé de 10 hommes et 8 femmes. On choisit un comité de 6 personnes au hasard.
Quelle est la probabilité que le comité soit composé d'exactement 3 hommes et 3 femmes ?
\end{exercicebox}

\begin{exercicebox}[Exercice 11 : Binomiale vs Hypergéométrique]
Reprenons le problème de l'exercice 8 (lot de 10000 articles, 10\% défectueux), mais cette fois on prélève les 20 articles \textit{sans remise}.
\begin{enumerate}
    \item Quelle est la loi exacte du nombre $X$ d'articles défectueux ? (Donnez son nom et ses paramètres).
    \item Calculez la probabilité exacte $P(X=2)$.
    \item Comparez ce résultat à celui obtenu à l'exercice 8. L'approximation binomiale était-elle bonne ? Pourquoi ?
\end{enumerate}
\end{exercicebox}

% --- Loi Géométrique ---

\begin{exercicebox}[Exercice 12 : Loi Géométrique (Calcul Direct)]
On lance un dé équilibré à 6 faces jusqu'à obtenir un 6. Soit $X$ le nombre d'échecs \textit{avant} le premier 6.
\begin{enumerate}
    \item Quelle loi suit $X$ ? Donnez son paramètre $p$.
    \item Quelle est la probabilité que le premier 6 apparaisse au 3ème lancer ? (C'est-à-dire $P(X=2)$).
    \item Quelle est la probabilité que le premier 6 apparaisse au 1er lancer ? (C'est-à-dire $P(X=0)$).
\end{enumerate}
\end{exercicebox}

\begin{exercicebox}[Exercice 13 : Loi Géométrique (Calcul Cumulé)]
Un archer touche sa cible avec une probabilité $p=0.2$ à chaque tir. Les tirs sont indépendants. Il tire jusqu'à ce qu'il touche la cible. Soit $X$ le nombre d'échecs avant son premier succès.
\begin{enumerate}
    \item Quelle est la probabilité qu'il ait besoin d'exactement 4 tirs au total ? (C'est-à-dire $P(X=3)$).
    \item Quelle est la probabilité qu'il ait besoin de plus de 2 tirs au total ? (C'est-à-dire $P(X \ge 2)$ ou $P(\text{les 2 premiers tirs sont des échecs})$).
\end{enumerate}
\end{exercicebox}

\begin{exercicebox}[Exercice 14 : Variante de la Loi Géométrique]
Certains manuels définissent la loi géométrique $Y$ comme le \textit{nombre total d'essais} (et non le nombre d'échecs). Si $Y \sim \text{Geom}(p)$ selon cette définition :
\begin{enumerate}
    \item Quelle est la PMF $P(Y=k)$ pour $k=1, 2, 3, \dots$ ?
    \item En utilisant $p=1/6$ (lancer de dé), calculez $P(Y=3)$. Comparez avec $P(X=2)$ de l'exercice 12.
\end{enumerate}
\end{exercicebox}

% --- Loi de Poisson ---

\begin{exercicebox}[Exercice 15 : Loi de Poisson (Calcul Direct)]
Un centre d'appels reçoit en moyenne $\lambda = 5$ appels par heure. Soit $X$ le nombre d'appels reçus en une heure donnée. On suppose que $X$ suit une loi de Poisson.
\begin{enumerate}
    \item Quelle est la probabilité qu'il n'y ait aucun appel ($P(X=0)$) ?
    \item Quelle est la probabilité qu'il y ait exactement 5 appels ($P(X=5)$) ? (Laissez $e^{-5}$ dans votre réponse).
\end{enumerate}
\end{exercicebox}

\begin{exercicebox}[Exercice 16 : Loi de Poisson (Calcul Cumulé)]
Un site web reçoit en moyenne $\lambda = 2$ visiteurs par minute. Soit $X$ le nombre de visiteurs en une minute.
Calculez la probabilité de recevoir au plus 2 visiteurs, $P(X \le 2)$. (Laissez $e^{-2}$ dans votre réponse).
\end{exercicebox}

\begin{exercicebox}[Exercice 17 : Loi de Poisson (Changement de $\lambda$)]
Un livre contient en moyenne 0.5 faute de frappe par page ($\lambda=0.5$).
\begin{enumerate}
    \item Quelle est la probabilité qu'une page donnée contienne 0 faute ?
    \item Soit $Y$ le nombre de fautes dans un chapitre de 10 pages. Quel est le nouveau paramètre $\lambda_Y$ pour $Y$ ?
    \item Quelle est la probabilité que ce chapitre de 10 pages contienne 0 faute ?
\end{enumerate}
\end{exercicebox}

\begin{exercicebox}[Exercice 18 : Approximation Binomiale par Poisson]
Une compagnie d'assurance a 10000 clients ($n=10000$). La probabilité qu'un client ait un accident dans l'année est $p=0.0003$.
\begin{enumerate}
    \item Quelle est la loi exacte $X$ du nombre d'accidents ?
    \item Calculez le paramètre $\lambda = np$ pour une approximation par la loi de Poisson.
    \item En utilisant la loi de Poisson, estimez la probabilité qu'il y ait exactement 2 accidents cette année, $P(X=2)$.
\end{enumerate}
\end{exercicebox}

% --- Synthèse et Variables Indicatrices ---

\begin{exercicebox}[Exercice 19 : Choisir la Bonne Loi]
Pour chaque scénario, identifiez la loi discrète la plus appropriée (Binomiale, Hypergéométrique, Géométrique, Poisson).
\begin{enumerate}
    \item On compte le nombre de Rois en tirant 5 cartes d'un jeu, sans remise.
    \item On compte le nombre de clients arrivant à une banque entre 10h et 11h.
    \item On compte le nombre de lancers de pièce jusqu'à obtenir le premier Pile.
    \item On compte le nombre de "6" obtenus en lançant un dé 20 fois.
    \item On compte le nombre de soldats tués par ruade de cheval dans un corps d'armée en un an.
\end{enumerate}
\end{exercicebox}

\begin{exercicebox}[Exercice 20 : Variable Indicatrice]
Soit $A$ l'événement "obtenir un 6 en lançant un dé équilibré".
Soit $I_A$ la variable indicatrice de l'événement $A$.
\begin{enumerate}
    \item Quelle loi suit $I_A$ ? Donnez son nom et son paramètre.
    \item Écrivez la PMF de $I_A$.
\end{enumerate}
\end{exercicebox}

\subsection{Corrections des Exercices}

% --- Corrections : Concepts de Base (PMF, CDF) ---

\begin{correctionbox}[Correction Exercice 1 : Identification de Variables Aléatoires]
1.  \textbf{Discrète}. $X$ ne peut prendre que des valeurs entières $\{0, 1, \dots, 10\}$.
2.  \textbf{Continue}. Le temps peut prendre n'importe quelle valeur dans un intervalle (par ex. $T \in [2.5, 5]$ heures).
3.  \textbf{Discrète}. $X$ ne peut prendre que des valeurs entières $\{0, 1, 2, \dots\}$.
4.  \textbf{Continue}. La température peut prendre n'importe quelle valeur dans un intervalle (par ex. $T \in [15.0, 25.0]^\circ\text{C}$).
5.  \textbf{Discrète}. $X$ ne peut prendre que des valeurs entières $\{0, 1, 2, \dots\}$ (si on compte les échecs) ou $\{1, 2, 3, \dots\}$ (si on compte les lancers).
\end{correctionbox}

\begin{correctionbox}[Correction Exercice 2 : Construction d'une PMF]
On lance un dé à 4 faces (1, 2, 3, 4). $X$ est le résultat.
1.  Valeurs possibles : $S_X = \{1, 2, 3, 4\}$.
2.  PMF : Le dé est équilibré, donc chaque face a la même probabilité $1/4$.
    $P(X=1) = 1/4$
    $P(X=2) = 1/4$
    $P(X=3) = 1/4$
    $P(X=4) = 1/4$
    Et $P(X=k) = 0$ pour tout autre $k$.
3.  Vérification : $\sum P(X=k) = 1/4 + 1/4 + 1/4 + 1/4 = 4/4 = 1$.
\end{correctionbox}

\begin{correctionbox}[Correction Exercice 3 : PMF d'une Somme]
$Y = D_1 + D_2$, où $D_1, D_2 \in \{1, 2, 3, 4\}$. Il y a $4 \times 4 = 16$ issues équiprobables (prob. 1/16 chacune).
1.  Valeurs possibles : Min = $1+1=2$. Max = $4+4=8$. $S_Y = \{2, 3, 4, 5, 6, 7, 8\}$.
2.  PMF (en comptant les issues favorables sur 16) :
    - $P(Y=2) = P(1,1) \implies 1/16$
    - $P(Y=3) = P(1,2) + P(2,1) \implies 2/16$
    - $P(Y=4) = P(1,3) + P(2,2) + P(3,1) \implies 3/16$
    - $P(Y=5) = P(1,4) + P(2,3) + P(3,2) + P(4,1) \implies 4/16$
    - $P(Y=6) = P(2,4) + P(3,3) + P(4,2) \implies 3/16$
    - $P(Y=7) = P(3,4) + P(4,3) \implies 2/16$
    - $P(Y=8) = P(4,4) \implies 1/16$
    (Vérification : $1+2+3+4+3+2+1 = 16$. La somme est $16/16 = 1$).
\end{correctionbox}

\begin{correctionbox}[Correction Exercice 4 : Construction d'une CDF]
On utilise la PMF de l'exercice 3. $F_Y(y) = P(Y \le y)$.
1.  CDF aux points de masse :
    - $F_Y(2) = P(Y \le 2) = P(Y=2) = 1/16$
    - $F_Y(3) = P(Y \le 3) = P(Y=2)+P(Y=3) = 1/16 + 2/16 = 3/16$
    - $F_Y(4) = P(Y \le 4) = 3/16 + P(Y=4) = 3/16 + 3/16 = 6/16$
    - $F_Y(5) = P(Y \le 5) = 6/16 + P(Y=5) = 6/16 + 4/16 = 10/16$
    - $F_Y(6) = P(Y \le 6) = 10/16 + P(Y=6) = 10/16 + 3/16 = 13/16$
    - $F_Y(7) = P(Y \le 7) = 13/16 + P(Y=7) = 13/16 + 2/16 = 15/16$
    - $F_Y(8) = P(Y \le 8) = 15/16 + P(Y=8) = 15/16 + 1/16 = 16/16 = 1$
2.  $F_Y(1.5) = P(Y \le 1.5) = 0$ (car la valeur minimale est 2).
3.  $F_Y(5.2) = P(Y \le 5.2) = P(Y \le 5) = F_Y(5) = 10/16$.
4.  $F_Y(10) = P(Y \le 10) = 1$ (car la valeur maximale est 8).
\end{correctionbox}

% --- Corrections : Loi de Bernoulli et Loi Binomiale ---

\begin{correctionbox}[Correction Exercice 5 : Loi de Bernoulli]
1.  $X$ suit une \textbf{loi de Bernoulli}. Le paramètre est $p=0.05$. On note $X \sim \text{Bern}(0.05)$.
2.  La PMF est :
    $P(X=1) = p = 0.05$ (succès = défectueux)
    $P(X=0) = 1-p = 0.95$ (échec = non défectueux)
\end{correctionbox}

\begin{correctionbox}[Correction Exercice 6 : Loi Binomiale (Calcul Direct)]
1.  $X$ est le nombre de succès (Pile) en $n=5$ essais indépendants avec probabilité $p=0.7$.
    $X$ suit une \textbf{loi Binomiale}. $X \sim \text{Bin}(n=5, p=0.7)$.
2.  $P(X=k) = \binom{n}{k} p^k (1-p)^{n-k}$.
    $P(X=3) = \binom{5}{3} (0.7)^3 (1-0.7)^{5-3} = 10 \times (0.343) \times (0.3)^2 = 10 \times 0.343 \times 0.09 = 0.3087$.
3.  $P(X=5) = \binom{5}{5} (0.7)^5 (0.3)^0 = 1 \times (0.7)^5 \times 1 = 0.16807$.
\end{correctionbox}

\begin{correctionbox}[Correction Exercice 7 : Loi Binomiale (Calcul Cumulé)]
On a $X \sim \text{Bin}(5, 0.7)$.
1.  $P(X=0) = \binom{5}{0} (0.7)^0 (0.3)^5 = 1 \times 1 \times (0.3)^5 = 0.00243$.
2.  L'événement "au moins 1 Pile" ($X \ge 1$) est le complémentaire de "0 Pile" ($X=0$).
    $P(X \ge 1) = 1 - P(X=0) = 1 - 0.00243 = 0.99757$.
\end{correctionbox}

\begin{correctionbox}[Correction Exercice 8 : Problème Binomial (Contrôle Qualité)]
Le tirage est \textit{avec remise}, donc les essais sont indépendants. C'est une loi binomiale.
$n = 20$ (nombre d'essais).
$p = 0.10$ (probabilité de succès = défectueux).
On cherche $P(X=2)$.
$P(X=2) = \binom{20}{2} (0.1)^2 (1-0.1)^{20-2}$
$P(X=2) = \frac{20 \times 19}{2} (0.1)^2 (0.9)^{18} = 190 \times 0.01 \times (0.9)^{18}$
$P(X=2) = 1.9 \times (0.9)^{18} \approx 1.9 \times 0.15009 \approx 0.2852$.
\end{correctionbox}

% --- Corrections : Loi Hypergéométrique ---

\begin{correctionbox}[Correction Exercice 9 : Loi Hypergéométrique (Urne)]
Le tirage est \textit{sans remise} d'une population finie.
1.  $X$ suit une \textbf{loi Hypergéométrique}.
    Paramètres : $w=7$ (blanches, succès), $b=5$ (noires, échecs), $m=4$ (nombre de tirages).
    $X \sim \text{HG}(w=7, b=5, m=4)$.
2.  On cherche $P(X=2)$.
    $P(X=k) = \frac{\binom{w}{k} \binom{b}{m-k}}{\binom{w+b}{m}}$
    $P(X=2) = \frac{\binom{7}{2} \binom{5}{4-2}}{\binom{12}{4}} = \frac{\binom{7}{2} \binom{5}{2}}{\binom{12}{4}}$
    $P(X=2) = \frac{(\frac{7 \times 6}{2}) \times (\frac{5 \times 4}{2})}{(\frac{12 \times 11 \times 10 \times 9}{4 \times 3 \times 2 \times 1})} = \frac{21 \times 10}{495} = \frac{210}{495} = \frac{14}{33} \approx 0.4242$.
\end{correctionbox}

\begin{correctionbox}[Correction Exercice 10 : Problème Hypergéométrique (Comité)]
Tirage sans remise. C'est une loi Hypergéométrique.
$w=10$ (hommes), $b=8$ (femmes), $m=6$ (taille du comité). Total $N=18$.
On cherche $P(X=3)$ (exactement 3 hommes, ce qui implique $m-k = 6-3=3$ femmes).
$P(X=3) = \frac{\binom{10}{3} \binom{8}{3}}{\binom{18}{6}}$
$P(X=3) = \frac{(\frac{10 \times 9 \times 8}{3 \times 2 \times 1}) \times (\frac{8 \times 7 \times 6}{3 \times 2 \times 1})}{(\frac{18 \times 17 \times 16 \times 15 \times 14 \times 13}{6 \times 5 \times 4 \times 3 \times 2 \times 1})} = \frac{120 \times 56}{18564} = \frac{6720}{18564} \approx 0.362$.
\end{correctionbox}

\begin{correctionbox}[Correction Exercice 11 : Binomiale vs Hypergéométrique]
Population totale $N=10000$. 10\% défectueux, donc $w=1000$ (défectueux), $b=9000$ (non défectueux).
Tirage de $m=20$ \textit{sans remise}.
1.  Loi exacte : \textbf{Loi Hypergéométrique}.
    $X \sim \text{HG}(w=1000, b=9000, m=20)$.
2.  Probabilité exacte $P(X=2)$ :
    $P(X=2) = \frac{\binom{1000}{2} \binom{9000}{18}}{\binom{10000}{20}}$
    $P(X=2) = \frac{(\frac{1000 \times 999}{2}) \times (\frac{9000 \times \dots \times 8983}{18!})}{(\frac{10000 \times \dots \times 9981}{20!})} \approx 0.2854$.
    (Le calcul est très complexe, mais on peut montrer qu'il est très proche de la binomiale).
3.  Le résultat de l'exercice 8 (Binomiale) était $\approx 0.2852$.
    L'approximation binomiale est excellente. La raison est que la taille de l'échantillon ($m=20$) est très petite par rapport à la taille de la population ($N=10000$). Le fait de ne pas remettre les 20 articles change à peine les probabilités pour les tirages suivants.
\end{correctionbox}

% --- Corrections : Loi Géométrique ---

\begin{correctionbox}[Correction Exercice 12 : Loi Géométrique (Calcul Direct)]
1.  $X$ est le nombre d'échecs avant le premier succès. $X$ suit une \textbf{loi Géométrique}.
    Le succès est "obtenir 6", donc $p = 1/6$. $X \sim \text{Geom}(p=1/6)$.
2.  "Premier 6 au 3ème lancer" signifie 2 échecs (lancers 1 et 2) puis 1 succès (lancer 3).
    C'est $P(X=2)$. $q = 1-p = 5/6$.
    $P(X=2) = q^2 p^1 = (5/6)^2 (1/6) = 25/216 \approx 0.1157$.
3.  "Premier 6 au 1er lancer" signifie 0 échec. C'est $P(X=0)$.
    $P(X=0) = q^0 p^1 = 1 \times (1/6) = 1/6$.
\end{correctionbox}

\begin{correctionbox}[Correction Exercice 13 : Loi Géométrique (Calcul Cumulé)]
$p=0.2$ (succès), $q=0.8$ (échec). $X$ compte les échecs. $X \sim \text{Geom}(0.2)$.
1.  "Exactement 4 tirs au total" signifie 3 échecs suivis d'un succès. On cherche $P(X=3)$.
    $P(X=3) = q^3 p^1 = (0.8)^3 (0.2) = 0.512 \times 0.2 = 0.1024$.
2.  "Plus de 2 tirs au total" signifie qu'il faut au moins 3 tirs. C'est l'événement "les 2 premiers tirs sont des échecs".
    La probabilité est $P(\text{Echec 1} \cap \text{Echec 2}) = q \times q = q^2$.
    $P(X \ge 2) = (0.8)^2 = 0.64$.
\end{correctionbox}

\begin{correctionbox}[Correction Exercice 14 : Variante de la Loi Géométrique]
$Y$ est le nombre total d'essais ($k=1, 2, 3, \dots$). $p$ est la prob. de succès.
1.  Pour que $Y=k$, il faut $k-1$ échecs, suivis d'un succès.
    $P(Y=k) = (1-p)^{k-1} p = q^{k-1} p$, pour $k=1, 2, \dots$
2.  Avec $p=1/6$, on cherche $P(Y=3)$.
    $P(Y=3) = (5/6)^{3-1} (1/6) = (5/6)^2 (1/6) = 25/216$.
    C'est le même résultat que $P(X=2)$ de l'exercice 12. Les deux définitions décrivent la même situation (3 lancers au total).
\end{correctionbox}

% --- Corrections : Loi de Poisson ---

\begin{correctionbox}[Correction Exercice 15 : Loi de Poisson (Calcul Direct)]
$X \sim \text{Poisson}(\lambda=5)$. PMF : $P(X=k) = \frac{e^{-\lambda} \lambda^k}{k!}$.
1.  $P(X=0) = \frac{e^{-5} 5^0}{0!} = \frac{e^{-5} \times 1}{1} = e^{-5} \approx 0.0067$.
2.  $P(X=5) = \frac{e^{-5} 5^5}{5!} = \frac{e^{-5} \times 3125}{120} = e^{-5} \times \frac{625}{24} \approx 26.04 \times e^{-5} \approx 0.1755$.
\end{correctionbox}

\begin{correctionbox}[Correction Exercice 16 : Loi de Poisson (Calcul Cumulé)]
$X \sim \text{Poisson}(\lambda=2)$. On cherche $P(X \le 2)$.
$P(X \le 2) = P(X=0) + P(X=1) + P(X=2)$
$P(X=0) = \frac{e^{-2} 2^0}{0!} = e^{-2}$
$P(X=1) = \frac{e^{-2} 2^1}{1!} = 2e^{-2}$
$P(X=2) = \frac{e^{-2} 2^2}{2!} = \frac{4e^{-2}}{2} = 2e^{-2}$
$P(X \le 2) = e^{-2} + 2e^{-2} + 2e^{-2} = 5e^{-2} \approx 5 \times 0.1353 = 0.6767$.
\end{correctionbox}

\begin{correctionbox}[Correction Exercice 17 : Loi de Poisson (Changement de $\lambda$)]
1.  Pour une page, $X \sim \text{Poisson}(\lambda=0.5)$.
    $P(X=0) = \frac{e^{-0.5} (0.5)^0}{0!} = e^{-0.5} \approx 0.6065$.
2.  Si le taux est 0.5 faute/page, le taux pour 10 pages est $\lambda_Y = 0.5 \times 10 = 5$.
    $Y \sim \text{Poisson}(\lambda_Y=5)$.
3.  On cherche $P(Y=0)$.
    $P(Y=0) = \frac{e^{-5} 5^0}{0!} = e^{-5} \approx 0.0067$.
\end{correctionbox}

\begin{correctionbox}[Correction Exercice 18 : Approximation Binomiale par Poisson]
1.  C'est un tirage de $n=10000$ clients, où chaque client est un essai de Bernoulli avec $p=0.0003$. La loi exacte est $X \sim \text{Bin}(10000, 0.0003)$.
2.  Le paramètre $\lambda$ pour l'approximation Poisson est $\lambda = np = 10000 \times 0.0003 = 3$.
3.  On utilise $Y \sim \text{Poisson}(\lambda=3)$ pour approximer $X$.
    $P(X=2) \approx P(Y=2) = \frac{e^{-3} 3^2}{2!} = \frac{9e^{-3}}{2} = 4.5 e^{-3} \approx 4.5 \times 0.04979 \approx 0.224$.
\end{correctionbox}

% --- Corrections : Synthèse et Variables Indicatrices ---

\begin{correctionbox}[Correction Exercice 19 : Choisir la Bonne Loi]
1.  Tirage sans remise d'une population finie : \textbf{Loi Hypergéométrique}.
2.  Comptage d'événements sur un intervalle de temps fixe : \textbf{Loi de Poisson}.
3.  Comptage d'essais jusqu'au premier succès : \textbf{Loi Géométrique}.
4.  Comptage de succès sur un nombre fixe d'essais indépendants : \textbf{Loi Binomiale}.
5.  Comptage d'événements rares sur un intervalle (temps/espace) : \textbf{Loi de Poisson}.
\end{correctionbox}

\begin{correctionbox}[Correction Exercice 20 : Variable Indicatrice]
$A$ = "obtenir 6". $P(A) = 1/6$.
$I_A = 1$ si $A$ se produit, $I_A = 0$ sinon.
1.  C'est une expérience avec deux issues (succès/échec). $I_A$ suit une \textbf{Loi de Bernoulli}.
    Le paramètre est $p = P(A) = 1/6$. $I_A \sim \text{Bern}(1/6)$.
2.  La PMF de $I_A$ est :
    $P(I_A = 1) = p = 1/6$
    $P(I_A = 0) = 1-p = 5/6$
\end{correctionbox}

\subsection{Exercices Pratiques (Python)}

Ces exercices vous aideront à implémenter et visualiser les fonctions de masse (PMF) des principales lois discrètes vues dans ce chapitre.

Pour ces exercices, vous aurez besoin des bibliothèques \texttt{math} (incluse), \texttt{random} (incluse), \texttt{pandas} et \texttt{matplotlib}.

\begin{codecell}
pip install pandas matplotlib
\end{codecell}

\begin{exercicebox}[Exercice 1 : PMF Empirique (Somme de deux dés)]
Nous allons simuler le lancer de deux dés à 6 faces et tracer la PMF empirique de leur somme, $Y = D_1 + D_2$.

\textbf{Votre tâche :}
\begin{enumerate}
    \item Créez une fonction \texttt{lancer\_deux\_des()} qui simule le lancer de deux dés (valeurs de 1 à 6) et retourne leur somme.
    \item Appelez cette fonction 10000 fois et stockez les résultats dans une liste.
    \item Convertissez cette liste en \texttt{pandas.Series}.
    \item Utilisez la méthode \texttt{.value\_counts(normalize=True).sort\_index()} sur la Series pour obtenir la PMF empirique (les probabilités de chaque somme).
    \item Utilisez \texttt{.plot(kind='bar')} pour visualiser cette PMF.
\end{enumerate}

\begin{codecell}
import random
import pandas as pd
import matplotlib.pyplot as plt

def lancer_deux_des():
    # ... votre code ...
    pass

N_simulations = 10000
resultats = []

# 2. Boucle de simulation
# ... votre code ...

# 3. Conversion en Series
# ... votre code ...

# 4. Calcul de la PMF empirique
# ... votre code ...

# 5. Affichage
# ... votre code ...
# plt.title("PMF Empirique de la Somme de Deux Des")
# plt.xlabel("Somme Y")
# plt.ylabel("Probabilite P(Y=k)")
# plt.show()
\end{codecell}
\end{exercicebox}

\begin{exercicebox}[Exercice 2 : Loi Binomiale (Implémentation de la PMF)]
Nous allons implémenter la formule de la PMF Binomiale $P(X=k) = \binom{n}{k} p^k (1-p)^{n-k}$ en utilisant le module \texttt{math}.

\textbf{Votre tâche :}
\begin{enumerate}
    \item Importez \texttt{math}.
    \item Écrivez une fonction \texttt{pmf\_binomiale(k, n, p)} qui prend $k$, $n$, et $p$ et retourne la probabilité $P(X=k)$.
    \item Testez votre fonction en calculant la probabilité d'obtenir exactement 6 Piles ($k=6$) en 10 lancers ($n=10$) d'une pièce équilibrée ($p=0.5$).
\end{enumerate}

\begin{codecell}
import math

def pmf_binomiale(k, n, p):
    # ... votre code ...
    pass

# 4. Test
k, n, p = 6, 10, 0.5
# probabilite = ...
# print(f"P(X={k}) pour Bin({n}, {p}) = {probabilite:.5f}")
\end{codecell}
\end{exercicebox}

\begin{exercicebox}[Exercice 3 : Loi Hypergéométrique (Implémentation de la PMF)]
Implémentons la PMF de la loi Hypergéométrique $P(X=k) = \frac{\binom{w}{k} \binom{b}{m-k}}{\binom{w+b}{m}}$.

\textbf{Votre tâche :}
\begin{enumerate}
    \item Écrivez une fonction \texttt{pmf\_hypergeometrique(k, w, b, m)} qui calcule $P(X=k)$.
    \item Testez votre fonction avec l'exemple du cours : un comité de $m=5$ personnes choisi parmi $w=8$ hommes et $b=10$ femmes. Quelle est la probabilité d'avoir $k=2$ hommes ?
\end{enumerate}

\begin{codecell}
import math

def pmf_hypergeometrique(k, w, b, m):
    # k = succes desires (parmi w)
    # w = nombre total de succes dans la population
    # b = nombre total d'echecs dans la population
    # m = taille de l'echantillon
    
    # ... votre code ...
    pass

# 3. Test
w, b, m, k = 8, 10, 5, 2
# probabilite = ...
# print(f"Probabilite d'avoir {k} hommes : {probabilite:.5f}")
\end{codecell}
\end{exercicebox}

\begin{exercicebox}[Exercice 4 : Loi Géométrique (Simulation)]
Nous allons simuler la loi Géométrique $X \sim \text{Geom}(p)$ qui compte le nombre d'échecs ($k$) avant le premier succès.

\textbf{Votre tâche :}
\begin{enumerate}
    \item Écrivez une fonction \texttt{simuler\_geometrique(p)} qui simule des essais de Bernoulli (avec probabilité $p$) jusqu'à obtenir un succès, et qui retourne le nombre d'échecs.
    \item Appelez cette fonction 10000 fois pour $p=1/6$ (lancer de dé) et stockez les $k$ (nombre d'échecs) dans une liste.
    \item Calculez et affichez la moyenne du nombre d'échecs dans vos simulations.
\end{enumerate}

\begin{codecell}
import random

def simuler_geometrique(p):
    # ... votre code ...
    pass

N_simulations = 10000
p = 1/6.0
resultats_geom = []

# 3. Boucle de simulation
# ... votre code ...

# 4. Calculer la moyenne
# ... votre code ...
# print(f"Moyenne d'echecs avant succes (p={p:.3f}): {moyenne_echecs:.3f}")
\end{codecell}
\end{exercicebox}

\begin{exercicebox}[Exercice 5 : Loi de Poisson (Implémentation de la PMF)]
Implémentons la PMF de la loi de Poisson $P(X=k) = \frac{e^{-\lambda} \lambda^k}{k!}$.

\textbf{Votre tâche :}
\begin{enumerate}
    \item Importez \texttt{math}.
    \item Écrivez une fonction \texttt{pmf\_poisson(k, lmbda)} qui calcule $P(X=k)$.
    \item Testez votre fonction en calculant les probabilités pour l'exemple de Bortkiewicz ($\lambda=0.61$) : $P(X=0)$, $P(X=1)$, et $P(X=2)$.
\end{enumerate}

\begin{codecell}
import math

def pmf_poisson(k, lmbda):
    # ... votre code ...
    pass

# 4. Test avec l'exemple de Bortkiewicz
lmbda = 0.61

# p_k0 = ...
# p_k1 = ...
# p_k2 = ...

# print(f"Parametre lambda = {lmbda}")
# print(f"P(X=0) = {p_k0:.5f}")
# print(f"P(X=1) = {p_k1:.5f}")
# print(f"P(X=2) = {p_k2:.5f}")
\end{codecell}
\end{exercicebox}

\begin{exercicebox}[Exercice 6 : Approximation Binomiale par Poisson]
Nous avons vu que la loi de Poisson $\text{Poisson}(\lambda)$ est une excellente approximation de la loi Binomiale $\text{Bin}(n, p)$ lorsque $n$ est grand, $p$ est petit, et $\lambda = np$.

\textbf{Votre tâche :}
En utilisant vos fonctions \texttt{pmf\_binomiale} (Ex 2) et \texttt{pmf\_poisson} (Ex 5) :
\begin{enumerate}
    \item Choisissez des paramètres : $n=1000$ et $p=0.002$.
    \item Calculez le paramètre $\lambda = np$ correspondant pour la loi de Poisson.
    \item Calculez $P(X=k)$ pour $k=0, 1, 2, 3$ en utilisant la PMF \textbf{Binomiale}.
    \item Calculez $P(Y=k)$ pour $k=0, 1, 2, 3$ en utilisant la PMF \textbf{Poisson}.
    \item Affichez les résultats côte à côte pour comparer la précision de l'approximation.
\end{enumerate}

\begin{codecell}
import math

# Collez vos fonctions pmf_binomiale et pmf_poisson ici
def pmf_binomiale(k, n, p):
    # ... (code de l'exercice 2) ...
    pass

def pmf_poisson(k, lmbda):
    # ... (code de l'exercice 5) ...
    pass

# 1. Parametres
n = 1000
p = 0.002

# 2. Calculer lambda
lmbda = n * p
print(f"n={n}, p={p}, lambda={lmbda}")
print("-" * 30)
print(f"k \t Binomiale \t Poisson")
print("-" * 30)

# 3. & 4. Boucle de calcul et comparaison
for k_val in [0, 1, 2, 3, 4]:
    # prob_bin = ...
    # prob_poi = ...
    # print(f"{k_val} \t {prob_bin:.6f} \t {prob_poi:.6f}")
    pass
\end{codecell}
\end{exercicebox}


\subsection{Exercices}

% --- Espérance de base et LOTUS ---

\begin{exercicebox}[Exercice 1 : Calcul d'Espérance (PMF Simple)]
Une variable aléatoire $X$ a la distribution de probabilité suivante :
$P(X=-1) = 0.3$, $P(X=0) = 0.5$, $P(X=2) = 0.2$.
Calculez l'espérance $E(X)$.
\end{exercicebox}

\begin{exercicebox}[Exercice 2 : LOTUS (Calcul de $E(X^2)$)]
En utilisant la même variable aléatoire $X$ que dans l'exercice 1, calculez $E(X^2)$.
\end{exercicebox}

\begin{exercicebox}[Exercice 3 : Variance (Calcul de base)]
En utilisant les résultats des exercices 1 et 2, calculez la variance $\text{Var}(X)$.
\end{exercicebox}

\begin{exercicebox}[Exercice 4 : Espérance (Jeu Simple)]
Un jeu consiste à payer 2 pour lancer un dé à 6 faces. Si le dé tombe sur 6, vous gagnez 10. Sinon, vous ne gagnez rien. Soit $G$ votre gain net (gain - mise).
\begin{enumerate}
    \item Quelle est la PMF de $G$ ?
    \item Calculez $E(G)$. Le jeu est-il favorable au joueur ?
\end{enumerate}
\end{exercicebox}

\begin{exercicebox}[Exercice 5 : Variance (Jeu Simple)]
En utilisant la variable aléatoire $G$ de l'exercice 4 :
\begin{enumerate}
    \item Calculez $E(G^2)$.
    \item Calculez $\text{Var}(G)$.
\end{enumerate}
\end{exercicebox}

\begin{exercicebox}[Exercice 6 : Espérance de Bernoulli]
Soit $X$ une variable aléatoire $X \sim \text{Bern}(p)$ (variable indicatrice). En utilisant la définition de l'espérance, montrez que $E(X) = p$.
\end{exercicebox}

\begin{exercicebox}[Exercice 7 : Variance de Bernoulli]
En utilisant le résultat de l'exercice 6 et le théorème de LOTUS, montrez que $\text{Var}(X) = p(1-p)$ pour $X \sim \text{Bern}(p)$. (Indice : $X^2 = X$ pour une variable de Bernoulli).
\end{exercicebox}

% --- Linéarité de l'Espérance ---

\begin{exercicebox}[Exercice 8 : Linéarité (Simple)]
Soient $X$ et $Y$ deux variables aléatoires. On sait que $E(X) = 10$ et $E(Y) = -5$.
Calculez $E(3X - 2Y + 4)$.
\end{exercicebox}

\begin{exercicebox}[Exercice 9 : Linéarité (Trois Dés)]
On lance trois dés équilibrés à 6 faces. Soit $S$ la somme des trois résultats.
En utilisant la linéarité de l'espérance, calculez $E(S)$.
\end{exercicebox}

\begin{exercicebox}[Exercice 10 : Linéarité (Somme de Bernoulli)]
Soit $X \sim \text{Bin}(n, p)$. On rappelle que $X$ peut s'écrire comme la somme de $n$ variables de Bernoulli indépendantes $X = I_1 + \dots + I_n$, où $E(I_j) = p$.
Utilisez la linéarité de l'espérance pour prouver que $E(X) = np$.
\end{exercicebox}

% --- Espérances des Lois Classiques ---

\begin{exercicebox}[Exercice 11 : Espérance Binomiale (Application)]
Un QCM (questionnaire à choix multiples) comporte 40 questions. Chaque question a 4 options de réponse, dont une seule est correcte. Un étudiant répond à tout au hasard.
Quel est le nombre attendu (l'espérance) de bonnes réponses ?
\end{exercicebox}

\begin{exercicebox}[Exercice 12 : Espérance Géométrique (Application)]
On lance une paire de dés équilibrés. Un "succès" est d'obtenir un double-six.
\begin{enumerate}
    \item Quelle est la probabilité $p$ d'un succès ?
    \item Soit $X$ le nombre d'échecs avant le premier double-six. Quelle est l'espérance $E(X)$ ?
\end{enumerate}
\end{exercicebox}

\begin{exercicebox}[Exercice 13 : Espérance Géométrique (Attente Totale)]
En reprenant la situation de l'exercice 12 ($p=1/36$), soit $Y$ le \textit{nombre total de lancers} nécessaires pour obtenir le premier double-six ($Y = X + 1$).
Calculez $E(Y)$.
\end{exercicebox}

\begin{exercicebox}[Exercice 14 : Espérance (Loi Hypergéométrique)]
On tire 5 cartes d'un jeu de 52 cartes sans remise. Soit $X$ le nombre d'As tirés. On peut écrire $X = I_1 + I_2 + I_3 + I_4 + I_5$, où $I_j=1$ si la $j$-ème carte tirée est un As, et 0 sinon.
\begin{enumerate}
    \item Quelle est la probabilité $P(I_1 = 1)$ (que la 1ère carte soit un As) ?
    \item Quelle est la probabilité $P(I_2 = 1)$ (que la 2ème carte soit un As) ? (Indice : Pensez par symétrie ou utilisez la LTP).
    \item Calculez $E(X)$ en utilisant la linéarité.
\end{enumerate}
\end{exercicebox}

% --- Variance et E[X^2] ---

\begin{exercicebox}[Exercice 15 : Espérance et Variance (Dé à 4 faces)]
Soit $X$ le résultat d'un lancer de dé équilibré à 4 faces ($X \in \{1, 2, 3, 4\}$).
\begin{enumerate}
    \item Calculez $E(X)$.
    \item Calculez $E(X^2)$.
    \item Calculez $\text{Var}(X)$.
\end{enumerate}
\end{exercicebox}

\begin{exercicebox}[Exercice 16 : Formule de la Variance (Inverse)]
Une variable aléatoire $Y$ a une espérance $E(Y) = 5$ et une variance $\text{Var}(Y) = 4$.
Quelle est la valeur de $E(Y^2)$ ?
\end{exercicebox}

\begin{exercicebox}[Exercice 17 : Formule de la Variance (Inverse 2)]
Une variable aléatoire $W$ a $E(W^2) = 50$ et $\text{Var}(W) = 1$.
Quelles sont les deux valeurs possibles pour $E(W)$ ?
\end{exercicebox}

\begin{exercicebox}[Exercice 18 : Variance Nulle]
Une variable aléatoire $X$ a une variance $\text{Var}(X) = 0$. Que pouvez-vous conclure sur la distribution de $X$ ?
(Indice : $\text{Var}(X) = E[(X-\mu)^2]$).
\end{exercicebox}

\begin{exercicebox}[Exercice 19 : LOTUS et Linéarité]
Soit $X$ une variable aléatoire avec $E(X)=3$ et $E(X^2)=10$.
Calculez $E[(X+1)^2]$.
(Indice : Développez $(X+1)^2$ avant de prendre l'espérance).
\end{exercicebox}

\begin{exercicebox}[Exercice 20 : Synthèse (Jeu de Roulette)]
À la roulette, vous misez 1 sur "Rouge". Il y a 18 cases rouges, 18 noires, et 1 verte (le 0). Total = 37 cases.
Si "Rouge" sort, vous récupérez votre mise de 1 et gagnez 1 de plus (gain net $G=+1$).
Si "Noir" or "Vert" sort, vous perdez votre mise (gain net $G=-1$).
\begin{enumerate}
    \item Calculez $E(G)$.
    \item Calculez $E(G^2)$.
    \item Calculez $\text{Var}(G)$.
\end{enumerate}
\end{exercicebox}

\subsection{Corrections des Exercices}

% --- Corrections : Concepts de Base (PMF, CDF) ---

\begin{correctionbox}[Correction Exercice 1 : Identification de Variables Aléatoires]
1.  \textbf{Discrète}. $X$ ne peut prendre que des valeurs entières $\{0, 1, \dots, 10\}$.
2.  \textbf{Continue}. Le temps peut prendre n'importe quelle valeur dans un intervalle (par ex. $T \in [2.5, 5]$ heures).
3.  \textbf{Discrète}. $X$ ne peut prendre que des valeurs entières $\{0, 1, 2, \dots\}$.
4.  \textbf{Continue}. La température peut prendre n'importe quelle valeur dans un intervalle (par ex. $T \in [15.0, 25.0]^\circ\text{C}$).
5.  \textbf{Discrète}. $X$ ne peut prendre que des valeurs entières $\{0, 1, 2, \dots\}$ (si on compte les échecs) ou $\{1, 2, 3, \dots\}$ (si on compte les lancers).
\end{correctionbox}

\begin{correctionbox}[Correction Exercice 2 : Construction d'une PMF]
On lance un dé à 4 faces (1, 2, 3, 4). $X$ est le résultat.
1.  Valeurs possibles : $S_X = \{1, 2, 3, 4\}$.
2.  PMF : Le dé est équilibré, donc chaque face a la même probabilité $1/4$.
    $P(X=1) = 1/4$
    $P(X=2) = 1/4$
    $P(X=3) = 1/4$
    $P(X=4) = 1/4$
    Et $P(X=k) = 0$ pour tout autre $k$.
3.  Vérification : $\sum P(X=k) = 1/4 + 1/4 + 1/4 + 1/4 = 4/4 = 1$.
\end{correctionbox}

\begin{correctionbox}[Correction Exercice 3 : PMF d'une Somme]
$Y = D_1 + D_2$, où $D_1, D_2 \in \{1, 2, 3, 4\}$. Il y a $4 \times 4 = 16$ issues équiprobables (prob. 1/16 chacune).
1.  Valeurs possibles : Min = $1+1=2$. Max = $4+4=8$. $S_Y = \{2, 3, 4, 5, 6, 7, 8\}$.
2.  PMF (en comptant les issues favorables sur 16) :
    - $P(Y=2) = P(1,1) \implies 1/16$
    - $P(Y=3) = P(1,2) + P(2,1) \implies 2/16$
    - $P(Y=4) = P(1,3) + P(2,2) + P(3,1) \implies 3/16$
    - $P(Y=5) = P(1,4) + P(2,3) + P(3,2) + P(4,1) \implies 4/16$
    - $P(Y=6) = P(2,4) + P(3,3) + P(4,2) \implies 3/16$
    - $P(Y=7) = P(3,4) + P(4,3) \implies 2/16$
    - $P(Y=8) = P(4,4) \implies 1/16$
    (Vérification : $1+2+3+4+3+2+1 = 16$. La somme est $16/16 = 1$).
\end{correctionbox}

\begin{correctionbox}[Correction Exercice 4 : Construction d'une CDF]
On utilise la PMF de l'exercice 3. $F_Y(y) = P(Y \le y)$.
1.  CDF aux points de masse :
    - $F_Y(2) = P(Y \le 2) = P(Y=2) = 1/16$
    - $F_Y(3) = P(Y \le 3) = P(Y=2)+P(Y=3) = 1/16 + 2/16 = 3/16$
    - $F_Y(4) = P(Y \le 4) = 3/16 + P(Y=4) = 3/16 + 3/16 = 6/16$
    - $F_Y(5) = P(Y \le 5) = 6/16 + P(Y=5) = 6/16 + 4/16 = 10/16$
    - $F_Y(6) = P(Y \le 6) = 10/16 + P(Y=6) = 10/16 + 3/16 = 13/16$
    - $F_Y(7) = P(Y \le 7) = 13/16 + P(Y=7) = 13/16 + 2/16 = 15/16$
    - $F_Y(8) = P(Y \le 8) = 15/16 + P(Y=8) = 15/16 + 1/16 = 16/16 = 1$
2.  $F_Y(1.5) = P(Y \le 1.5) = 0$ (car la valeur minimale est 2).
3.  $F_Y(5.2) = P(Y \le 5.2) = P(Y \le 5) = F_Y(5) = 10/16$.
4.  $F_Y(10) = P(Y \le 10) = 1$ (car la valeur maximale est 8).
\end{correctionbox}

% --- Corrections : Loi de Bernoulli et Loi Binomiale ---

\begin{correctionbox}[Correction Exercice 5 : Loi de Bernoulli]
1.  $X$ suit une \textbf{loi de Bernoulli}. Le paramètre est $p=0.05$. On note $X \sim \text{Bern}(0.05)$.
2.  La PMF est :
    $P(X=1) = p = 0.05$ (succès = défectueux)
    $P(X=0) = 1-p = 0.95$ (échec = non défectueux)
\end{correctionbox}

\begin{correctionbox}[Correction Exercice 6 : Loi Binomiale (Calcul Direct)]
1.  $X$ est le nombre de succès (Pile) en $n=5$ essais indépendants avec probabilité $p=0.7$.
    $X$ suit une \textbf{loi Binomiale}. $X \sim \text{Bin}(n=5, p=0.7)$.
2.  $P(X=k) = \binom{n}{k} p^k (1-p)^{n-k}$.
    $P(X=3) = \binom{5}{3} (0.7)^3 (1-0.7)^{5-3} = 10 \times (0.343) \times (0.3)^2 = 10 \times 0.343 \times 0.09 = 0.3087$.
3.  $P(X=5) = \binom{5}{5} (0.7)^5 (0.3)^0 = 1 \times (0.7)^5 \times 1 = 0.16807$.
\end{correctionbox}

\begin{correctionbox}[Correction Exercice 7 : Loi Binomiale (Calcul Cumulé)]
On a $X \sim \text{Bin}(5, 0.7)$.
1.  $P(X=0) = \binom{5}{0} (0.7)^0 (0.3)^5 = 1 \times 1 \times (0.3)^5 = 0.00243$.
2.  L'événement "au moins 1 Pile" ($X \ge 1$) est le complémentaire de "0 Pile" ($X=0$).
    $P(X \ge 1) = 1 - P(X=0) = 1 - 0.00243 = 0.99757$.
\end{correctionbox}

\begin{correctionbox}[Correction Exercice 8 : Problème Binomial (Contrôle Qualité)]
Le tirage est \textit{avec remise}, donc les essais sont indépendants. C'est une loi binomiale.
$n = 20$ (nombre d'essais).
$p = 0.10$ (probabilité de succès = défectueux).
On cherche $P(X=2)$.
$P(X=2) = \binom{20}{2} (0.1)^2 (1-0.1)^{20-2}$
$P(X=2) = \frac{20 \times 19}{2} (0.1)^2 (0.9)^{18} = 190 \times 0.01 \times (0.9)^{18}$
$P(X=2) = 1.9 \times (0.9)^{18} \approx 1.9 \times 0.15009 \approx 0.2852$.
\end{correctionbox}

% --- Corrections : Loi Hypergéométrique ---

\begin{correctionbox}[Correction Exercice 9 : Loi Hypergéométrique (Urne)]
Le tirage est \textit{sans remise} d'une population finie.
1.  $X$ suit une \textbf{loi Hypergéométrique}.
    Paramètres : $w=7$ (blanches, succès), $b=5$ (noires, échecs), $m=4$ (nombre de tirages).
    $X \sim \text{HG}(w=7, b=5, m=4)$.
2.  On cherche $P(X=2)$.
    $P(X=k) = \frac{\binom{w}{k} \binom{b}{m-k}}{\binom{w+b}{m}}$
    $P(X=2) = \frac{\binom{7}{2} \binom{5}{4-2}}{\binom{12}{4}} = \frac{\binom{7}{2} \binom{5}{2}}{\binom{12}{4}}$
    $P(X=2) = \frac{(\frac{7 \times 6}{2}) \times (\frac{5 \times 4}{2})}{(\frac{12 \times 11 \times 10 \times 9}{4 \times 3 \times 2 \times 1})} = \frac{21 \times 10}{495} = \frac{210}{495} = \frac{14}{33} \approx 0.4242$.
\end{correctionbox}

\begin{correctionbox}[Correction Exercice 10 : Problème Hypergéométrique (Comité)]
Tirage sans remise. C'est une loi Hypergéométrique.
$w=10$ (hommes), $b=8$ (femmes), $m=6$ (taille du comité). Total $N=18$.
On cherche $P(X=3)$ (exactement 3 hommes, ce qui implique $m-k = 6-3=3$ femmes).
$P(X=3) = \frac{\binom{10}{3} \binom{8}{3}}{\binom{18}{6}}$
$P(X=3) = \frac{(\frac{10 \times 9 \times 8}{3 \times 2 \times 1}) \times (\frac{8 \times 7 \times 6}{3 \times 2 \times 1})}{(\frac{18 \times 17 \times 16 \times 15 \times 14 \times 13}{6 \times 5 \times 4 \times 3 \times 2 \times 1})} = \frac{120 \times 56}{18564} = \frac{6720}{18564} \approx 0.362$.
\end{correctionbox}

\begin{correctionbox}[Correction Exercice 11 : Binomiale vs Hypergéométrique]
Population totale $N=10000$. 10\% défectueux, donc $w=1000$ (défectueux), $b=9000$ (non défectueux).
Tirage de $m=20$ \textit{sans remise}.
1.  Loi exacte : \textbf{Loi Hypergéométrique}.
    $X \sim \text{HG}(w=1000, b=9000, m=20)$.
2.  Probabilité exacte $P(X=2)$ :
    $P(X=2) = \frac{\binom{1000}{2} \binom{9000}{18}}{\binom{10000}{20}}$
    $P(X=2) = \frac{(\frac{1000 \times 999}{2}) \times (\frac{9000 \times \dots \times 8983}{18!})}{(\frac{10000 \times \dots \times 9981}{20!})} \approx 0.2854$.
    (Le calcul est très complexe, mais on peut montrer qu'il est très proche de la binomiale).
3.  Le résultat de l'exercice 8 (Binomiale) était $\approx 0.2852$.
    L'approximation binomiale est excellente. La raison est que la taille de l'échantillon ($m=20$) est très petite par rapport à la taille de la population ($N=10000$). Le fait de ne pas remettre les 20 articles change à peine les probabilités pour les tirages suivants.
\end{correctionbox}

% --- Corrections : Loi Géométrique ---

\begin{correctionbox}[Correction Exercice 12 : Loi Géométrique (Calcul Direct)]
1.  $X$ est le nombre d'échecs avant le premier succès. $X$ suit une \textbf{loi Géométrique}.
    Le succès est "obtenir 6", donc $p = 1/6$. $X \sim \text{Geom}(p=1/6)$.
2.  "Premier 6 au 3ème lancer" signifie 2 échecs (lancers 1 et 2) puis 1 succès (lancer 3).
    C'est $P(X=2)$. $q = 1-p = 5/6$.
    $P(X=2) = q^2 p^1 = (5/6)^2 (1/6) = 25/216 \approx 0.1157$.
3.  "Premier 6 au 1er lancer" signifie 0 échec. C'est $P(X=0)$.
    $P(X=0) = q^0 p^1 = 1 \times (1/6) = 1/6$.
\end{correctionbox}

\begin{correctionbox}[Correction Exercice 13 : Loi Géométrique (Calcul Cumulé)]
$p=0.2$ (succès), $q=0.8$ (échec). $X$ compte les échecs. $X \sim \text{Geom}(0.2)$.
1.  "Exactement 4 tirs au total" signifie 3 échecs suivis d'un succès. On cherche $P(X=3)$.
    $P(X=3) = q^3 p^1 = (0.8)^3 (0.2) = 0.512 \times 0.2 = 0.1024$.
2.  "Plus de 2 tirs au total" signifie qu'il faut au moins 3 tirs. C'est l'événement "les 2 premiers tirs sont des échecs".
    La probabilité est $P(\text{Echec 1} \cap \text{Echec 2}) = q \times q = q^2$.
    $P(X \ge 2) = (0.8)^2 = 0.64$.
\end{correctionbox}

\begin{correctionbox}[Correction Exercice 14 : Variante de la Loi Géométrique]
$Y$ est le nombre total d'essais ($k=1, 2, 3, \dots$). $p$ est la prob. de succès.
1.  Pour que $Y=k$, il faut $k-1$ échecs, suivis d'un succès.
    $P(Y=k) = (1-p)^{k-1} p = q^{k-1} p$, pour $k=1, 2, \dots$
2.  Avec $p=1/6$, on cherche $P(Y=3)$.
    $P(Y=3) = (5/6)^{3-1} (1/6) = (5/6)^2 (1/6) = 25/216$.
    C'est le même résultat que $P(X=2)$ de l'exercice 12. Les deux définitions décrivent la même situation (3 lancers au total).
\end{correctionbox}

% --- Corrections : Loi de Poisson ---

\begin{correctionbox}[Correction Exercice 15 : Loi de Poisson (Calcul Direct)]
$X \sim \text{Poisson}(\lambda=5)$. PMF : $P(X=k) = \frac{e^{-\lambda} \lambda^k}{k!}$.
1.  $P(X=0) = \frac{e^{-5} 5^0}{0!} = \frac{e^{-5} \times 1}{1} = e^{-5} \approx 0.0067$.
2.  $P(X=5) = \frac{e^{-5} 5^5}{5!} = \frac{e^{-5} \times 3125}{120} = e^{-5} \times \frac{625}{24} \approx 26.04 \times e^{-5} \approx 0.1755$.
\end{correctionbox}

\begin{correctionbox}[Correction Exercice 16 : Loi de Poisson (Calcul Cumulé)]
$X \sim \text{Poisson}(\lambda=2)$. On cherche $P(X \le 2)$.
$P(X \le 2) = P(X=0) + P(X=1) + P(X=2)$
$P(X=0) = \frac{e^{-2} 2^0}{0!} = e^{-2}$
$P(X=1) = \frac{e^{-2} 2^1}{1!} = 2e^{-2}$
$P(X=2) = \frac{e^{-2} 2^2}{2!} = \frac{4e^{-2}}{2} = 2e^{-2}$
$P(X \le 2) = e^{-2} + 2e^{-2} + 2e^{-2} = 5e^{-2} \approx 5 \times 0.1353 = 0.6767$.
\end{correctionbox}

\begin{correctionbox}[Correction Exercice 17 : Loi de Poisson (Changement de $\lambda$)]
1.  Pour une page, $X \sim \text{Poisson}(\lambda=0.5)$.
    $P(X=0) = \frac{e^{-0.5} (0.5)^0}{0!} = e^{-0.5} \approx 0.6065$.
2.  Si le taux est 0.5 faute/page, le taux pour 10 pages est $\lambda_Y = 0.5 \times 10 = 5$.
    $Y \sim \text{Poisson}(\lambda_Y=5)$.
3.  On cherche $P(Y=0)$.
    $P(Y=0) = \frac{e^{-5} 5^0}{0!} = e^{-5} \approx 0.0067$.
\end{correctionbox}

\begin{correctionbox}[Correction Exercice 18 : Approximation Binomiale par Poisson]
1.  C'est un tirage de $n=10000$ clients, où chaque client est un essai de Bernoulli avec $p=0.0003$. La loi exacte est $X \sim \text{Bin}(10000, 0.0003)$.
2.  Le paramètre $\lambda$ pour l'approximation Poisson est $\lambda = np = 10000 \times 0.0003 = 3$.
3.  On utilise $Y \sim \text{Poisson}(\lambda=3)$ pour approximer $X$.
    $P(X=2) \approx P(Y=2) = \frac{e^{-3} 3^2}{2!} = \frac{9e^{-3}}{2} = 4.5 e^{-3} \approx 4.5 \times 0.04979 \approx 0.224$.
\end{correctionbox}

% --- Corrections : Synthèse et Variables Indicatrices ---

\begin{correctionbox}[Correction Exercice 19 : Choisir la Bonne Loi]
1.  Tirage sans remise d'une population finie : \textbf{Loi Hypergéométrique}.
2.  Comptage d'événements sur un intervalle de temps fixe : \textbf{Loi de Poisson}.
3.  Comptage d'essais jusqu'au premier succès : \textbf{Loi Géométrique}.
4.  Comptage de succès sur un nombre fixe d'essais indépendants : \textbf{Loi Binomiale}.
5.  Comptage d'événements rares sur un intervalle (temps/espace) : \textbf{Loi de Poisson}.
\end{correctionbox}

\begin{correctionbox}[Correction Exercice 20 : Variable Indicatrice]
$A$ = "obtenir 6". $P(A) = 1/6$.
$I_A = 1$ si $A$ se produit, $I_A = 0$ sinon.
1.  C'est une expérience avec deux issues (succès/échec). $I_A$ suit une \textbf{Loi de Bernoulli}.
    Le paramètre est $p = P(A) = 1/6$. $I_A \sim \text{Bern}(1/6)$.
2.  La PMF de $I_A$ est :
    $P(I_A = 1) = p = 1/6$
    $P(I_A = 0) = 1-p = 5/6$
\end{correctionbox}

\subsection{Exercices Pratiques (Python)}

Ces exercices vous aideront à calculer et à vérifier empiriquement les concepts d'espérance et de variance en utilisant des simulations.

Pour ces exercices, vous aurez besoin de la bibliothèque \texttt{numpy}.

\begin{codecell}
pip install numpy
\end{codecell}

\begin{exercicebox}[Exercice 1 : $E(X)$ $E(X^2)$ et Variance (Dé)]
Nous allons simuler $N$ lancers d'un dé à 6 faces pour vérifier empiriquement la définition de l'espérance, le théorème LOTUS, et la formule de calcul de la variance.

\textbf{Votre tâche :}
\begin{enumerate}
    \item Simulez 100 000 lancers d'un dé équilibré (valeurs de 1 à 6) et stockez les résultats dans un tableau NumPy.
    \item Calculez l'espérance empirique $E(X)$ en prenant la moyenne du tableau.
    \item En utilisant LOTUS, calculez l'espérance empirique $E(X^2)$ (en créant un nouveau tableau des carrés, puis en prenant sa moyenne).
    \item Calculez la variance empirique en utilisant la formule : $\text{Var}(X) = E(X^2) - [E(X)]^2$.
    \item Comparez votre résultat à la variance calculée directement avec \texttt{numpy.var()}.
\end{enumerate}

\begin{codecell}
import numpy as np

N_simulations = 100000

# 1. Simuler N lancers d'un de a 6 faces
# lancers = ...

# 2. Calculer E(X) (moyenne empirique)
# E_X = ...
# print(f"E(X) empirique: {E_X:.4f} (Theorique: 3.5)")

# 3. Calculer E(X^2) (LOTUS)
# lancers_carres = ...
# E_X2 = ...
# print(f"E(X^2) empirique: {E_X2:.4f} (Theorique: 91/6 = 15.1667)")

# 4. Calculer Var(X) avec la formule
# var_calc = ...
# print(f"Variance (calculee): {var_calc:.4f}")

# 5. Calculer Var(X) avec la fonction numpy
# var_np = ...
# print(f"Variance (numpy.var): {var_np:.4f}")
# print(f"Difference: {np.abs(var_calc - var_np):.6f}")
\end{codecell}
\end{exercicebox}

\begin{exercicebox}[Exercice 2 : Linearite de l'Esperance]
Vérifions empiriquement que $E(X+Y) = E(X) + E(Y)$. Nous allons simuler deux variables aléatoires différentes : $X$ (un dé à 4 faces) et $Y$ (un dé à 6 faces).

\textbf{Votre tâche :}
\begin{enumerate}
    \item Simulez $N=100000$ lancers d'un dé à 4 faces ($X$).
    \item Simulez $N=100000$ lancers d'un dé à 6 faces ($Y$).
    \item Créez la variable aléatoire $Z = X + Y$.
    \item Calculez les moyennes empiriques $E(X)$, $E(Y)$, et $E(Z)$.
    \item Vérifiez que $E(Z)$ est très proche de $E(X) + E(Y)$.
\end{enumerate}

\begin{codecell}
import numpy as np

N_simulations = 100000

# 1. Simuler X (de a 4 faces) et Y (de a 6 faces)
# X = ...
# Y = ...

# 2. Creer Z = X + Y
# Z = ...

# 3. Calculer les moyennes empiriques
# E_X = ...
# E_Y = ...
# E_Z = ...

# 4. Verifier la linearite
# print(f"E(X) = {E_X:.4f}")
# print(f"E(Y) = {E_Y:.4f}")
# print(f"E(X) + E(Y) = {E_X + E_Y:.4f}")
# print(f"E(Z) = E(X+Y) = {E_Z:.4f}")
\end{codecell}
\end{exercicebox}

\begin{exercicebox}[Exercice 3 : Esperance Binomiale (Simulation)]
La théorie nous dit que pour $X \sim \text{Bin}(n, p)$, $E(X) = np$. Nous allons vérifier cela par simulation.

\textbf{Votre tâche :}
\begin{enumerate}
    \item Définissez les paramètres $n=20$ et $p=0.4$.
    \item Simulez 100 000 réalisations d'une variable aléatoire $X \sim \text{Bin}(n, p)$ en utilisant \texttt{numpy.random.binomial()}.
    \item Calculez la moyenne empirique de vos simulations.
    \item Comparez la moyenne empirique à l'espérance théorique $np$.
\end{enumerate}

\begin{codecell}
import numpy as np

n, p = 20, 0.4
N_simulations = 100000

# 1. Simuler N fois une loi Bin(n, p)
# resultats_bin = ...

# 2. Calculer la moyenne empirique
# moyenne_empirique = ...

# 3. Calculer la moyenne theorique
# moyenne_theorique = ...

# 4. Afficher
# print(f"Moyenne empirique: {moyenne_empirique:.4f}")
# print(f"Esperance theorique (np): {moyenne_theorique:.4f}")
\end{codecell}
\end{exercicebox}

\begin{exercicebox}[Exercice 4 : Esperance Geometrique (Simulation)]
Pour $X \sim \text{Geom}(p)$ (comptant les échecs), $E(X) = q/p$. Vérifions cela.

\textbf{Votre tâche :}
\begin{enumerate}
    \item Définissez $p=0.2$ (et $q=1-p$).
    \item Simulez 100 000 réalisations d'une variable $Y \sim \text{Geom}(p)$ en utilisant \texttt{numpy.random.geometric()}.
    \item \textbf{Attention :} \texttt{numpy.random.geometric} compte le nombre d'essais ($k=1, 2, \dots$). Pour obtenir $X$ (le nombre d'échecs, $k=0, 1, \dots$), vous devez soustraire 1 de chaque résultat.
    \item Calculez la moyenne empirique de $X$ (le nombre d'échecs).
    \item Comparez cette moyenne à l'espérance théorique $q/p$.
\end{enumerate}

\begin{codecell}
import numpy as np

p = 0.2
q = 1 - p
N_simulations = 100000

# 1. Simuler N fois une loi Geom(p) (nb d'essais)
# resultats_geom_essais = ...

# 3. Convertir en nombre d'echecs
# resultats_geom_echecs = ...

# 4. Calculer la moyenne empirique des echecs
# moyenne_empirique = ...

# 5. Calculer la moyenne theorique des echecs
# moyenne_theorique = ...

# 6. Afficher
# print(f"Moyenne empirique (echecs): {moyenne_empirique:.4f}")
# print(f"Esperance theorique (q/p): {moyenne_theorique:.4f}")
\end{codecell}
\end{exercicebox}

\begin{exercicebox}[Exercice 5 : Esperance et Variance de Bernoulli]
La variable aléatoire de Bernoulli $X \sim \text{Bern}(p)$ est la brique de base. Théoriquement, $E(X) = p$ et $\text{Var}(X) = p(1-p)$.

\textbf{Votre tâche :}
\begin{enumerate}
    \item Définissez $p=0.8$.
    \item Simulez 100 000 essais de Bernoulli (résultats 0 ou 1) avec probabilité $p$. (Indice : \texttt{numpy.random.choice} ou \texttt{numpy.random.binomial} avec $n=1$).
    \item Calculez l'espérance empirique (la moyenne) et la variance empirique (\texttt{numpy.var}).
    \item Comparez-les aux valeurs théoriques $p$ et $p(1-p)$.
\end{enumerate}

\begin{codecell}
import numpy as np

p = 0.8
N_simulations = 100000

# 1. Simuler N essais de Bernoulli
# essais = ...

# 2. Calculer l'esperance et la variance empiriques
# E_empirique = ...
# Var_empirique = ...

# 3. Calculer les valeurs theoriques
# E_theorique = ...
# Var_theorique = ...

# 4. Afficher
# print(f"Esperance: Empirique={E_empirique:.4f}, Theorique={E_theorique:.4f}")
# print(f"Variance:  Empirique={Var_empirique:.4f}, Theorique={Var_theorique:.4f}")
\end{codecell}
\end{exercicebox}



\subsection{Exercices}

% --- PDF, CDF, et Espérance de Base ---

\begin{exercicebox}[Exercice 1 : Validation d'une PDF]
Soit $X$ une variable aléatoire continue. On propose la fonction $f(x) = c x^2$ pour $x \in [0, 1]$ et $f(x) = 0$ sinon.
\begin{enumerate}
    \item Trouvez la constante $c$ pour que $f(x)$ soit une fonction de densité de probabilité (PDF) valide.
    \item En utilisant la valeur de $c$ trouvée, calculez $P(X \le 0.5)$.
\end{enumerate}
\end{exercicebox}

\begin{exercicebox}[Exercice 2 : PDF à partir d'une CDF]
La fonction de répartition (CDF) d'une variable aléatoire $Y$ est donnée par :
$$ F(y) = \begin{cases} 0 & \text{si } y < 0 \\ y^3 & \text{si } 0 \le y \le 1 \\ 1 & \text{si } y > 1 \end{cases} $$
\begin{enumerate}
    \item Trouvez la fonction de densité de probabilité (PDF) $f(y)$ de $Y$.
    \item Calculez $P(0.1 \le Y \le 0.5)$ en utilisant la CDF.
\end{enumerate}
\end{exercicebox}

\begin{exercicebox}[Exercice 3 : Calcul d'Espérance et Variance (PDF Simple)]
Utilisez la PDF $f(x)$ trouvée dans l'exercice 1 ($f(x) = 3x^2$ pour $x \in [0, 1]$).
\begin{enumerate}
    \item Calculez l'espérance $E[X]$.
    \item Calculez $E[X^2]$ en utilisant le théorème de transfert (LOTUS).
    \item Déduisez-en la variance $\text{Var}(X)$.
\end{enumerate}
\end{exercicebox}

\begin{exercicebox}[Exercice 4 : PDF Triangulaire]
Soit $X$ une v.a. avec la PDF $f(x) = 1 - |x|$ pour $x \in [-1, 1]$ et $f(x)=0$ sinon.
\begin{enumerate}
    \item Vérifiez que l'aire totale sous la courbe est 1. (Indice : c'est un triangle).
    \item Calculez $E[X]$. (Indice : utilisez la symétrie).
    \item Calculez $P(X > 0.5)$.
\end{enumerate}
\end{exercicebox}

% --- Loi Uniforme ---

\begin{exercicebox}[Exercice 5 : Loi Uniforme (Bus)]
Un bus arrive à un arrêt toutes les 15 minutes. Vous arrivez à l'arrêt à un moment aléatoire. Soit $X$ votre temps d'attente. On suppose $X \sim \text{Unif}(0, 15)$.
\begin{enumerate}
    \item Quelle est la PDF $f(x)$ de $X$ ?
    \item Quelle est la probabilité que vous attendiez moins de 5 minutes ?
    \item Quelle est l'espérance de votre temps d'attente $E[X]$ ?
\end{enumerate}
\end{exercicebox}

\begin{exercicebox}[Exercice 6 : Loi Uniforme (Variance)]
Soit $X \sim \text{Unif}(a, b)$. On rappelle que $\text{Var}(X) = \frac{(b-a)^2}{12}$.
Un signal est uniformément distribué entre 5 Volts et 10 Volts.
\begin{enumerate}
    \item Quelle est l'espérance du signal ?
    \item Quelle est la variance du signal ?
\end{enumerate}
\end{exercicebox}

\begin{exercicebox}[Exercice 7 : Loi Uniforme (CDF)]
Soit $X \sim \text{Unif}(-1, 3)$.
\begin{enumerate}
    \item Calculez la CDF $F(x) = P(X \le x)$. (N'oubliez pas les 3 parties : $x < -1$, $-1 \le x \le 3$, $x > 3$).
    \item Utilisez la CDF pour calculer $P(0 \le X \le 2)$.
\end{enumerate}
\end{exercicebox}

% --- Loi Exponentielle ---

\begin{exercicebox}[Exercice 8 : Loi Exponentielle (Durée de vie)]
La durée de vie (en heures) d'un composant électronique suit une loi exponentielle avec $\lambda = 0.01$. $X \sim \text{Exp}(0.01)$.
\begin{enumerate}
    \item Quelle est l'espérance de la durée de vie $E[X]$ ?
    \item Quelle est la probabilité que le composant dure plus de 100 heures ? (Rappel : $P(X > t) = e^{-\lambda t}$).
\end{enumerate}
\end{exercicebox}

\begin{exercicebox}[Exercice 9 : Loi Exponentielle (Propriété de non-mémoire)]
On reprend le composant de l'exercice 8 ($X \sim \text{Exp}(0.01)$).
\begin{enumerate}
    \item Calculez la probabilité que le composant dure plus de 50 heures, $P(X > 50)$.
    \item Calculez la probabilité qu'il dure plus de 150 heures, \textit{sachant qu'il a déjà duré 100 heures} : $P(X > 150 | X > 100)$.
    \item Comparez vos résultats de (1) et (2) et commentez.
\end{enumerate}
\end{exercicebox}

\begin{exercicebox}[Exercice 10 : Loi Exponentielle (CDF)]
Le temps d'attente $T$ (en minutes) à un guichet suit $T \sim \text{Exp}(\lambda=0.5)$.
\begin{enumerate}
    \item Quelle est la CDF $F(t)$ de $T$ ?
    \item Quelle est la probabilité d'attendre entre 1 et 3 minutes, $P(1 \le T \le 3)$ ?
\end{enumerate}
\end{exercicebox}

\begin{exercicebox}[Exercice 11 : Trouver $\lambda$]
La durée de vie moyenne d'une ampoule suit une loi exponentielle avec une espérance de 800 heures.
\begin{enumerate}
    \item Trouvez le paramètre $\lambda$.
    \item Quelle est la probabilité qu'une ampoule meure avant 500 heures ?
\end{enumerate}
\end{exercicebox}

% --- Lois Conjointes (Continu) ---

\begin{exercicebox}[Exercice 12 : Validation d'une PDF Conjointe]
On considère la fonction $f(x, y) = c(x+y)$ pour $0 \le x \le 1$ et $0 \le y \le 1$, et $f(x,y)=0$ sinon.
\begin{enumerate}
    \item Mettez en place la double intégrale pour $\int_0^1 \int_0^1 c(x+y) \, dx \, dy$.
    \item Calculez l'intégrale et trouvez la valeur de $c$ qui en fait une PDF valide.
\end{enumerate}
\end{exercicebox}

\begin{exercicebox}[Exercice 13 : Calcul de Densités Marginales]
En utilisant la PDF $f(x,y)$ et la constante $c$ de l'exercice 12 :
\begin{enumerate}
    \item Calculez la densité marginale $f_X(x) = \int_{-\infty}^{\infty} f(x, y) \, dy$. (Définie pour $x \in [0, 1]$).
    \item Calculez la densité marginale $f_Y(y) = \int_{-\infty}^{\infty} f(x, y) \, dx$. (Définie pour $y \in [0, 1]$).
\end{enumerate}
\end{exercicebox}

\begin{exercicebox}[Exercice 14 : Indépendance (Continu)]
En utilisant les résultats des exercices 12 et 13 :
\begin{enumerate}
    \item Calculez le produit $f_X(x) f_Y(y)$.
    \item Comparez $f(x,y)$ et $f_X(x) f_Y(y)$. $X$ et $Y$ sont-elles indépendantes ?
\end{enumerate}
\end{exercicebox}

\begin{exercicebox}[Exercice 15 : Calcul de Probabilité Conjointe]
On utilise toujours $f(x,y) = x+y$ pour $x, y \in [0, 1]$ (on a trouvé $c=1$).
Calculez $P(X \le 0.5 \text{ et } Y \le 0.5)$.
\end{exercicebox}

\begin{exercicebox}[Exercice 16 : PDF Conjointe Uniforme]
Soit $(X, Y)$ un couple uniformément distribué sur le carré $[0, 2] \times [0, 2]$.
\begin{enumerate}
    \item Quelle est la surface du carré ?
    \item Quelle est la valeur de la PDF $f(x, y)$ à l'intérieur de ce carré ?
    \item Calculez $P(0 \le X \le 1 \text{ et } 1 \le Y \le 2)$. (Indice : c'est le volume d'un pavé).
\end{enumerate}
\end{exercicebox}

\begin{exercicebox}[Exercice 17 : Indépendance (Uniforme)]
Pour la PDF de l'exercice 16 ($f(x,y) = 1/4$ sur le carré $[0,2]\times[0,2]$) :
\begin{enumerate}
    \item Calculez les marginales $f_X(x)$ et $f_Y(y)$.
    \item $X$ et $Y$ sont-elles indépendantes ?
\end{enumerate}
\end{exercicebox}

% --- Espérance, Covariance (Continu) ---

\begin{exercicebox}[Exercice 18 : Espérances Marginales]
En utilisant les lois marginales $f_X(x)$ et $f_Y(y)$ de l'exercice 13 :
\begin{enumerate}
    \item Calculez $E[X]$.
    \item Calculez $E[Y]$.
\end{enumerate}
\end{exercicebox}

\begin{exercicebox}[Exercice 19 : LOTUS Conjoint et Covariance]
En utilisant la PDF $f(x,y) = x+y$ sur le carré $[0,1]\times[0,1]$ :
\begin{enumerate}
    \item Calculez $E[XY] = \int_0^1 \int_0^1 (xy)(x+y) \, dx \, dy$.
    \item En utilisant $E[X]$ et $E[Y]$ de l'exercice 18, calculez $\text{Cov}(X, Y)$.
\end{enumerate}
\end{exercicebox}

\begin{exercicebox}[Exercice 20 : Covariance et Indépendance]
Soient $X$ et $Y$ les variables de l'exercice 16 (uniformes sur $[0, 2] \times [0, 2]$).
\begin{enumerate}
    \item $X$ et $Y$ sont-elles indépendantes (d'après Ex. 17) ?
    \item Que vaut $\text{Cov}(X, Y)$ ? (Sans calcul).
\end{enumerate}
\end{exercicebox}

\subsection{Corrections des Exercices}

% --- Corrections : PDF, CDF, et Espérance de Base ---

\begin{correctionbox}[Correction Exercice 1 : Validation d'une PDF]
$f(x) = c x^2$ pour $x \in [0, 1]$.
1.  On doit avoir $\int_{-\infty}^{\infty} f(x) \, dx = 1$.
    $$ \int_0^1 c x^2 \, dx = c \left[ \frac{x^3}{3} \right]_0^1 = c \left( \frac{1^3}{3} - 0 \right) = \frac{c}{3} $$
    Pour que l'intégrale vaille 1, il faut $c/3 = 1$, donc $\mathbf{c=3}$.
2.  On utilise $f(x) = 3x^2$.
    $$ P(X \le 0.5) = \int_0^{0.5} 3x^2 \, dx = \left[ x^3 \right]_0^{0.5} = (0.5)^3 - 0 = 0.125 $$
\end{correctionbox}

\begin{correctionbox}[Correction Exercice 2 : PDF à partir d'une CDF]
1.  La PDF $f(y)$ est la dérivée de la CDF $F(y)$.
    - Si $y < 0$, $f(y) = \frac{d}{dy}(0) = 0$.
    - Si $0 \le y \le 1$, $f(y) = \frac{d}{dy}(y^3) = 3y^2$.
    - Si $y > 1$, $f(y) = \frac{d}{dy}(1) = 0$.
    Donc, $\mathbf{f(y) = 3y^2}$ pour $y \in [0, 1]$ et 0 sinon.
2.  On utilise la propriété $P(a \le Y \le b) = F(b) - F(a)$.
    $$ P(0.1 \le Y \le 0.5) = F(0.5) - F(0.1) = (0.5)^3 - (0.1)^3 $$
    $$ = 0.125 - 0.001 = 0.124 $$
\end{correctionbox}

\begin{correctionbox}[Correction Exercice 3 : Calcul d'Espérance et Variance]
On utilise $f(x) = 3x^2$ pour $x \in [0, 1]$.
1.  $E[X] = \int_{-\infty}^{\infty} x f(x) \, dx = \int_0^1 x (3x^2) \, dx = \int_0^1 3x^3 \, dx$
    $E[X] = \left[ \frac{3x^4}{4} \right]_0^1 = \mathbf{3/4}$.
2.  $E[X^2] = \int_{-\infty}^{\infty} x^2 f(x) \, dx = \int_0^1 x^2 (3x^2) \, dx = \int_0^1 3x^4 \, dx$
    $E[X^2] = \left[ \frac{3x^5}{5} \right]_0^1 = \mathbf{3/5}$.
3.  $\text{Var}(X) = E[X^2] - (E[X])^2 = \frac{3}{5} - \left(\frac{3}{4}\right)^2 = \frac{3}{5} - \frac{9}{16}$
    $\text{Var}(X) = \frac{3 \times 16 - 9 \times 5}{80} = \frac{48 - 45}{80} = \mathbf{3/80}$.
\end{correctionbox}

\begin{correctionbox}[Correction Exercice 4 : PDF Triangulaire]
$f(x) = 1 - |x|$ pour $x \in [-1, 1]$.
1.  La PDF est un triangle de base 2 (de -1 à 1) et de hauteur $f(0)=1$.
    Aire = $\frac{1}{2} \times \text{base} \times \text{hauteur} = \frac{1}{2} \times 2 \times 1 = 1$. C'est bien une PDF valide.
2.  La fonction $f(x)$ est symétrique autour de $x=0$. L'espérance est donc le centre de symétrie.
    $E[X] = \mathbf{0}$.
    (Calcul : $E[X] = \int_{-1}^1 x(1-|x|)dx = 0$ car l'intégrande est une fonction impaire sur un intervalle symétrique).
3.  $P(X > 0.5) = \int_{0.5}^1 f(x) \, dx$. Pour $x>0$, $f(x) = 1-x$.
    $P(X > 0.5) = \int_{0.5}^1 (1-x) \, dx = \left[ x - \frac{x^2}{2} \right]_{0.5}^1$
    $= \left(1 - \frac{1}{2}\right) - \left(0.5 - \frac{(0.5)^2}{2}\right) = 0.5 - (0.5 - 0.125) = \mathbf{0.125}$ (ou $1/8$).
\end{correctionbox}

% --- Corrections : Loi Uniforme ---

\begin{correctionbox}[Correction Exercice 5 : Loi Uniforme (Bus)]
$X \sim \text{Unif}(a=0, b=15)$.
1.  $f(x) = \frac{1}{b-a} = \mathbf{\frac{1}{15}}$ pour $x \in [0, 15]$, et 0 sinon.
2.  $P(X < 5) = \int_0^5 \frac{1}{15} \, dx = \frac{1}{15} [x]_0^5 = \frac{5-0}{15} = \mathbf{1/3}$.
    (C'est la longueur de l'intervalle [0, 5] divisée par la longueur totale [0, 15]).
3.  $E[X] = \frac{a+b}{2} = \frac{0+15}{2} = \mathbf{7.5}$ minutes.
\end{correctionbox}

\begin{correctionbox}[Correction Exercice 6 : Loi Uniforme (Variance)]
Signal $X \sim \text{Unif}(a=5, b=10)$.
1.  $E[X] = \frac{a+b}{2} = \frac{5+10}{2} = \mathbf{7.5}$ Volts.
2.  $\text{Var}(X) = \frac{(b-a)^2}{12} = \frac{(10-5)^2}{12} = \frac{5^2}{12} = \mathbf{25/12}$ Volts$^2$.
\end{correctionbox}

\begin{correctionbox}[Correction Exercice 7 : Loi Uniforme (CDF)]
$X \sim \text{Unif}(a=-1, b=3)$. $f(x) = \frac{1}{3 - (-1)} = 1/4$ sur $[-1, 3]$.
1.  CDF $F(x) = \int_{-\infty}^x f(t) dt$.
    - Si $x < -1$ : $F(x) = \int_{-\infty}^x 0 \, dt = \mathbf{0}$.
    - Si $-1 \le x \le 3$ : $F(x) = \int_{-1}^x \frac{1}{4} \, dt = \frac{1}{4}[t]_{-1}^x = \mathbf{\frac{x - (-1)}{4} = \frac{x+1}{4}}$.
    - Si $x > 3$ : $F(x) = P(X \le 3) = \mathbf{1}$.
2.  $P(0 \le X \le 2) = F(2) - F(0) = \frac{2+1}{4} - \frac{0+1}{4} = \frac{3}{4} - \frac{1}{4} = \frac{2}{4} = \mathbf{0.5}$.
\end{correctionbox}

% --- Corrections : Loi Exponentielle ---

\begin{correctionbox}[Correction Exercice 8 : Loi Exponentielle (Durée de vie)]
$X \sim \text{Exp}(\lambda = 0.01)$.
1.  $E[X] = \frac{1}{\lambda} = \frac{1}{0.01} = \mathbf{100}$ heures.
2.  $P(X > t) = e^{-\lambda t}$.
    $P(X > 100) = e^{-0.01 \times 100} = e^{-1} \approx \mathbf{0.3679}$.
\end{correctionbox}

\begin{correctionbox}[Correction Exercice 9 : Loi Exponentielle (Propriété de non-mémoire)]
$X \sim \text{Exp}(\lambda = 0.01)$.
1.  $P(X > 50) = e^{-0.01 \times 50} = e^{-0.5} \approx \mathbf{0.6065}$.
2.  Par la propriété de non-mémoire, le fait d'avoir déjà duré 100 heures est "oublié".
    $P(X > 150 | X > 100) = P(X > 100 + 50 | X > 100) = P(X > 50)$.
    Résultat = $e^{-0.5} \approx \mathbf{0.6065}$.
3.  Les probabilités sont identiques. Le fait que le composant ait survécu 100 heures ne donne aucune information sur sa probabilité de survivre 50 heures de plus.
\end{correctionbox}

\begin{correctionbox}[Correction Exercice 10 : Loi Exponentielle (CDF)]
$T \sim \text{Exp}(\lambda=0.5)$.
1.  $F(t) = P(T \le t) = 1 - e^{-\lambda t} = \mathbf{1 - e^{-0.5 t}}$ (pour $t \ge 0$).
2.  $P(1 \le T \le 3) = F(3) - F(1)$
    $= (1 - e^{-0.5 \times 3}) - (1 - e^{-0.5 \times 1})$
    $= 1 - e^{-1.5} - 1 + e^{-0.5} = e^{-0.5} - e^{-1.5}$
    $\approx 0.6065 - 0.2231 = \mathbf{0.3834}$.
\end{correctionbox}

\begin{correctionbox}[Correction Exercice 11 : Trouver $\lambda$]
1.  On sait que $E[X] = 1/\lambda$.
    $800 = 1/\lambda \implies \lambda = 1/800 = \mathbf{0.00125}$.
2.  On cherche $P(X < 500) = F(500)$.
    $P(X < 500) = 1 - e^{-\lambda \times 500} = 1 - e^{-0.00125 \times 500} = 1 - e^{-0.625}$
    $\approx 1 - 0.5353 = \mathbf{0.4647}$.
\end{correctionbox}

% --- Corrections : Lois Conjointes (Continu) ---

\begin{correctionbox}[Correction Exercice 12 : Validation d'une PDF Conjointe]
1.  $\int_{0}^1 \int_{0}^1 c(x+y) \, dx \, dy = 1$.
2.  Calcul de l'intégrale interne (par rapport à $x$) :
    $\int_0^1 c(x+y) \, dx = c \left[ \frac{x^2}{2} + xy \right]_0^1 = c \left( (\frac{1}{2} + y) - (0) \right) = c(\frac{1}{2} + y)$.
    Calcul de l'intégrale externe (par rapport à $y$) :
    $\int_0^1 c(\frac{1}{2} + y) \, dy = c \left[ \frac{y}{2} + \frac{y^2}{2} \right]_0^1 = c \left( (\frac{1}{2} + \frac{1}{2}) - (0) \right) = c(1)$.
    On doit avoir $c(1) = 1$, donc $\mathbf{c=1}$.
\end{correctionbox}

\begin{correctionbox}[Correction Exercice 13 : Calcul de Densités Marginales]
On utilise $f(x,y) = x+y$ pour $x,y \in [0,1]$.
1.  Marginale $f_X(x)$ : On intègre par rapport à $y$.
    $f_X(x) = \int_0^1 (x+y) \, dy = \left[ xy + \frac{y^2}{2} \right]_0^1 = (x + \frac{1}{2}) - (0) = \mathbf{x + 0.5}$ (pour $x \in [0,1]$).
2.  Marginale $f_Y(y)$ : On intègre par rapport à $x$.
    $f_Y(y) = \int_0^1 (x+y) \, dx = \left[ \frac{x^2}{2} + xy \right]_0^1 = (\frac{1}{2} + y) - (0) = \mathbf{y + 0.5}$ (pour $y \in [0,1]$).
\end{correctionbox}

\begin{correctionbox}[Correction Exercice 14 : Indépendance (Continu)]
1.  $f_X(x) f_Y(y) = (x + 0.5)(y + 0.5) = xy + 0.5x + 0.5y + 0.25$.
2.  La PDF conjointe est $f(x,y) = x+y$.
    Puisque $f(x,y) \neq f_X(x) f_Y(y)$, les variables $X$ et $Y$ \textbf{ne sont pas indépendantes}.
\end{correctionbox}

\begin{correctionbox}[Correction Exercice 15 : Calcul de Probabilité Conjointe]
$f(x,y) = x+y$. On cherche $P(X \le 0.5, Y \le 0.5)$.
$$ \int_0^{0.5} \int_0^{0.5} (x+y) \, dx \, dy $$
Intégrale interne (sur $x$) : $\int_0^{0.5} (x+y) \, dx = \left[ \frac{x^2}{2} + xy \right]_0^{0.5} = \frac{(0.5)^2}{2} + 0.5y = \frac{0.25}{2} + 0.5y = 0.125 + 0.5y$.
Intégrale externe (sur $y$) : $\int_0^{0.5} (0.125 + 0.5y) \, dy = \left[ 0.125y + \frac{0.5y^2}{2} \right]_0^{0.5}$
$= [ 0.125y + 0.25y^2 ]_0^{0.5} = (0.125)(0.5) + (0.25)(0.5)^2 = 0.0625 + (0.25)(0.25)$
$= 0.0625 + 0.0625 = \mathbf{0.125}$ (ou $1/8$).
\end{correctionbox}

\begin{correctionbox}[Correction Exercice 16 : PDF Conjointe Uniforme]
$(X, Y)$ uniforme sur le carré $[0, 2] \times [0, 2]$.
1.  Surface = $2 \times 2 = \mathbf{4}$.
2.  La densité est $f(x,y) = 1 / \text{Surface} = \mathbf{1/4}$ pour $x,y \in [0, 2]$, et 0 sinon.
3.  On cherche $P(0 \le X \le 1, 1 \le Y \le 2)$. C'est le volume sous $f(x,y)=1/4$ au-dessus du rectangle $[0,1]\times[1,2]$.
    Volume = $\text{Hauteur} \times \text{Surface de la base}$
    Surface de la base = $(1-0) \times (2-1) = 1$.
    Volume = $(1/4) \times 1 = \mathbf{1/4}$.
\end{correctionbox}

\begin{correctionbox}[Correction Exercice 17 : Indépendance (Uniforme)]
$f(x,y) = 1/4$ sur $[0, 2] \times [0, 2]$.
1.  Marginale $f_X(x)$ (pour $x \in [0, 2]$) :
    $f_X(x) = \int_0^2 (1/4) \, dy = (1/4) [y]_0^2 = (1/4)(2) = \mathbf{1/2}$.
    Marginale $f_Y(y)$ (pour $y \in [0, 2]$) :
    $f_Y(y) = \int_0^2 (1/4) \, dx = (1/4) [x]_0^2 = (1/4)(2) = \mathbf{1/2}$.
    (Ce sont des lois $\text{Unif}(0, 2)$).
2.  On vérifie $f(x,y) \stackrel{?}{=} f_X(x) f_Y(y)$.
    $f_X(x) f_Y(y) = (1/2) \times (1/2) = 1/4$.
    C'est égal à $f(x,y) = 1/4$.
    Oui, $X$ et $Y$ \textbf{sont indépendantes}. (C'est toujours le cas pour une densité uniforme sur un rectangle aligné sur les axes).
\end{correctionbox}

% --- Corrections : Espérance, Covariance (Continu) ---

\begin{correctionbox}[Correction Exercice 18 : Espérances Marginales]
On utilise $f_X(x) = x+0.5$ et $f_Y(y) = y+0.5$ (pour $x,y \in [0,1]$).
1.  $E[X] = \int_0^1 x f_X(x) \, dx = \int_0^1 x(x+0.5) \, dx = \int_0^1 (x^2 + 0.5x) \, dx$
    $= \left[ \frac{x^3}{3} + \frac{0.5x^2}{2} \right]_0^1 = \left[ \frac{x^3}{3} + \frac{x^2}{4} \right]_0^1 = \frac{1}{3} + \frac{1}{4} = \mathbf{7/12}$.
2.  Par symétrie, $f_Y(y)$ a la même forme que $f_X(x)$, donc $E[Y] = \mathbf{7/12}$.
\end{correctionbox}

\begin{correctionbox}[Correction Exercice 19 : LOTUS Conjoint et Covariance]
1.  $E[XY] = \int_0^1 \int_0^1 (xy)(x+y) \, dx \, dy = \int_0^1 \int_0^1 (x^2y + xy^2) \, dx \, dy$
    Intégrale interne (sur $x$) : $\int_0^1 (x^2y + xy^2) \, dx = \left[ \frac{x^3y}{3} + \frac{x^2y^2}{2} \right]_0^1 = \frac{y}{3} + \frac{y^2}{2}$.
    Intégrale externe (sur $y$) : $\int_0^1 (\frac{y}{3} + \frac{y^2}{2}) \, dy = \left[ \frac{y^2}{6} + \frac{y^3}{6} \right]_0^1 = \frac{1}{6} + \frac{1}{6} = \frac{2}{6} = \mathbf{1/3}$.
2.  $\text{Cov}(X, Y) = E[XY] - E[X]E[Y] = \frac{1}{3} - \left(\frac{7}{12}\right)\left(\frac{7}{12}\right)$
    $= \frac{1}{3} - \frac{49}{144} = \frac{48}{144} - \frac{49}{144} = \mathbf{-1/144}$.
\end{correctionbox}

\begin{correctionbox}[Correction Exercice 20 : Covariance et Indépendance]
1.  D'après l'exercice 17, $X$ et $Y$ \textbf{sont indépendantes}.
2.  Puisque $X$ et $Y$ sont indépendantes, leur covariance est nulle.
    $\text{Cov}(X, Y) = \mathbf{0}$.
\end{correctionbox}

\subsection{Exercices Pratiques (Python)}

Ces exercices vous aideront à visualiser les concepts de PDF, CDF, espérance et variance pour les lois Uniforme et Exponentielle en utilisant des simulations avec \texttt{numpy} et \texttt{matplotlib}.

\begin{codecell}
pip install numpy matplotlib scipy
\end{codecell}
(\texttt{scipy} n'est pas strictement nécessaire ici, mais souvent utile pour les stats)

\begin{exercicebox}[Exercice 1 : PDF vs Histogramme (Loi Uniforme)]
Nous allons simuler une loi uniforme $X \sim \text{Unif}(a, b)$ et comparer l'histogramme des données simulées à la PDF théorique.

\textbf{Votre tâche :}
\begin{enumerate}
    \item Définissez $a=2$, $b=8$. Calculez la hauteur de la PDF théorique ($1 / (b-a)$).
    \item Simulez 100 000 échantillons d'une loi uniforme $\text{Unif}(a, b)$ avec \texttt{numpy.random.uniform(a, b, N)}.
    \item Tracez l'histogramme de ces échantillons. Utilisez \texttt{bins=50} et \textbf{\texttt{density=True}}.
    \item Sur le même graphique, tracez une ligne horizontale représentant la PDF théorique (de $a$ à $b$, à la hauteur calculée).
\end{enumerate}

\begin{codecell}
import numpy as np
import matplotlib.pyplot as plt

N_simulations = 100000
a, b = 2, 8

# 1. Calculer la hauteur de la PDF
# ... votre code ...

# 2. Simuler les echantillons
# ... votre code ...

# 3. Tracer l'histogramme normalise
# ... votre code ...

# 4. Tracer la PDF theorique
# ... votre code ...
# plt.legend()
# plt.title("Loi Uniforme: Histogramme vs PDF")
# plt.show()
\end{codecell}
\end{exercicebox}

\begin{exercicebox}[Exercice 2 : CDF Empirique vs CDF Théorique (Loi Uniforme)]
Comparons la fonction de répartition (CDF) empirique des données simulées à la CDF théorique $F(x) = (x-a)/(b-a)$.

\textbf{Votre tâche :}
\begin{enumerate}
    \item Reprenez les échantillons simulés de l'exercice 1 ($a=2, b=8$).
    \item Tracez la CDF empirique des échantillons. Utilisez \texttt{plt.hist(echantillons, bins=1000, density=True, cumulative=True, histtype='step')}.
    \item Sur le même graphique, tracez la CDF théorique. (Indice : Créez un \texttt{np.linspace} de $a$ à $b$, puis calculez $y = (x-a)/(b-a)$). N'oubliez pas que la CDF vaut 0 avant $a$ et 1 après $b$.
\end{enumerate}

\begin{codecell}
import numpy as np
import matplotlib.pyplot as plt

# (Regenerez les echantillons si necessaire)
# N_simulations = 100000
# a, b = 2, 8
# echantillons = ...

# 2. Tracer la CDF empirique
# ... votre code ...

# 3. Tracer la CDF theorique
# x_theorique = np.linspace(a - 1, b + 1, 400) # Un peu avant et apres
# y_cdf_theorique = ... # Calculer la CDF theorique
# (Attention aux 3 parties : < a, entre a et b, > b)
# ... votre code ...

# plt.title("Loi Uniforme: CDF Empirique vs Theorique")
# plt.xlabel("Valeur x")
# plt.ylabel("Probabilite Cumulative P(X <= x)")
# plt.legend()
# plt.grid(True)
# plt.show()
\end{codecell}
\end{exercicebox}

\begin{exercicebox}[Exercice 3 : Espérance et Variance (Loi Uniforme)]
Vérifions les formules $E[X] = (a+b)/2$ et $\text{Var}(X) = (b-a)^2/12$.

\textbf{Votre tâche :}
\begin{enumerate}
    \item Reprenez les échantillons simulés de l'exercice 1 ($a=2, b=8$).
    \item Calculez les valeurs théoriques pour $E[X]$ et $\text{Var}(X)$.
    \item Calculez les valeurs empiriques en utilisant \texttt{.mean()} et \texttt{.var(ddof=0)} sur vos échantillons.
    \item Comparez les résultats théoriques et empiriques.
\end{enumerate}

\begin{codecell}
import numpy as np

# (Regenerez les echantillons si necessaire)
# N_simulations = 100000
# a, b = 2, 8
# echantillons = ...

# 2. Calculs theoriques
# ... votre code ...
# print(f"Theorique: E(X)=..., Var(X)=...")

# 3. Calculs empiriques
# ... votre code ...
# print(f"Empirique: E(X)=..., Var(X)=...")
\end{codecell}
\end{exercicebox}

\begin{exercicebox}[Exercice 4 : PDF et CDF (Loi Exponentielle)]
Nous allons simuler une loi exponentielle $X \sim \text{Exp}(\lambda)$ et comparer les histogrammes empiriques de la PDF et de la CDF à leurs formes théoriques.

\textbf{Votre tâche :}
\begin{enumerate}
    \item Définissez $\lambda = 0.5$.
    \item Simulez 100 000 échantillons d'une loi $\text{Exp}(\lambda)$ avec \texttt{numpy.random.exponential(scale=1/lmbda, size=N)}.
    \item Tracez l'histogramme PDF (\texttt{density=True}).
    \item Sur le même graphique, tracez la PDF théorique $f(x) = \lambda e^{-\lambda x}$.
    \item Tracez l'histogramme CDF (\texttt{density=True, cumulative=True}).
    \item Sur ce second graphique, tracez la CDF théorique $F(x) = 1 - e^{-\lambda x}$.
\end{enumerate}

\begin{codecell}
import numpy as np
import matplotlib.pyplot as plt

N_simulations = 100000
lmbda = 0.5
scale = 1 / lmbda

# 2. Simuler les echantillons
# ... votre code ...

# 3. & 4. PDF (Graphique 1)
# plt.figure(figsize=(10, 4))
# plt.subplot(1, 2, 1)
# ... votre code histogramme ...
# x_theorique = np.linspace(0, np.max(echantillons), 200)
# y_pdf_theorique = ... # Calculer la PDF theorique
# ... votre code plot ...
# plt.title("PDF Exponentielle")

# 5. & 6. CDF (Graphique 2)
# plt.subplot(1, 2, 2)
# ... votre code histogramme cumulatif ...
# y_cdf_theorique = ... # Calculer la CDF theorique
# ... votre code plot ...
# plt.title("CDF Exponentielle")
# plt.show()
\end{codecell}
\end{exercicebox}

\begin{exercicebox}[Exercice 5 : Espérance et Variance (Loi Exponentielle)]
Vérifions empiriquement les formules $E[X] = 1/\lambda$ et $\text{Var}(X) = 1/\lambda^2$ pour la loi exponentielle.

\textbf{Votre tâche :}
\begin{enumerate}
    \item Reprenez les échantillons simulés de l'exercice 4 ($\lambda=0.5$).
    \item Calculez les valeurs théoriques pour $E[X]$ et $\text{Var}(X)$.
    \item Calculez les valeurs empiriques en utilisant \texttt{.mean()} et \texttt{.var(ddof=0)}.
    \item Comparez les résultats théoriques et empiriques.
\end{enumerate}

\begin{codecell}
import numpy as np

# (Regenerez les echantillons si necessaire)
# N_simulations = 100000
# lmbda = 0.5
# echantillons = ...

# 2. Calculs theoriques
# E_theorique = ...
# Var_theorique = ...
# print(f"Theorique: E(X)=..., Var(X)=...")

# 3. Calculs empiriques
# E_empirique = ...
# Var_empirique = ...
# print(f"Empirique: E(X)=..., Var(X)=...")
\end{codecell}
\end{exercicebox}


\subsection{Exercices}

% --- Lois Jointes et Marginales (Discret) ---

\begin{exercicebox}[Exercice 1 : Loi Jointe et Marginales]
Soit le tableau suivant représentant la loi de probabilité jointe $P(X=x, Y=y)$ d'un couple de variables aléatoires $(X, Y)$.

\begin{center}
\begin{tabular}{|c|ccc|}
\hline
\diagbox{$X$}{$Y$} & 0 & 1 & 2 \\ \hline
0 & 0.1 & 0.2 & 0.1 \\
1 & 0.3 & 0.1 & 0.2 \\ \hline
\end{tabular}
\end{center}

\begin{enumerate}
    \item Vérifiez qu'il s'agit bien d'une loi de probabilité.
    \item Calculez la loi marginale de $X$, $P(X=x)$.
    \item Calculez la loi marginale de $Y$, $P(Y=y)$.
\end{enumerate}
\end{exercicebox}

\begin{exercicebox}[Exercice 2 : Calcul de Probabilité Jointe]
En utilisant la loi jointe de l'exercice 1 :
\begin{enumerate}
    \item Calculez $P(X=0, Y \le 1)$.
    \item Calculez $P(X=Y)$.
    \item Calculez $P(X > Y)$.
\end{enumerate}
\end{exercicebox}

\begin{exercicebox}[Exercice 3 : Indépendance (Loi Jointe)]
En utilisant la loi jointe de l'exercice 1 :
\begin{enumerate}
    \item Calculez $P(X=0) \times P(Y=0)$.
    \item Comparez ce résultat à $P(X=0, Y=0)$.
    \item Les variables $X$ et $Y$ sont-elles indépendantes ? Justifiez.
\end{enumerate}
\end{exercicebox}

% --- Espérance, Covariance et Corrélation ---

\begin{exercicebox}[Exercice 4 : Espérances Marginales]
En utilisant les lois marginales calculées à l'exercice 1 :
\begin{enumerate}
    \item Calculez l'espérance $E[X]$.
    \item Calculez l'espérance $E[Y]$.
\end{enumerate}
\end{exercicebox}

\begin{exercicebox}[Exercice 5 : Espérance d'une Fonction (LOTUS)]
En utilisant la loi jointe de l'exercice 1, calculez $E[XY]$.
(Indice : $E[XY] = \sum_x \sum_y (xy) P(X=x, Y=y)$).
\end{exercicebox}

\begin{exercicebox}[Exercice 6 : Covariance (Calcul)]
En utilisant les résultats des exercices 4 et 5, calculez la covariance $\text{Cov}(X,Y)$.
\end{exercicebox}

\begin{exercicebox}[Exercice 7 : Variances Marginales]
En utilisant les lois marginales de l'exercice 1 et les espérances de l'exercice 4 :
\begin{enumerate}
    \item Calculez $E[X^2]$ et $\text{Var}(X)$.
    \item Calculez $E[Y^2]$ et $\text{Var}(Y)$.
\end{enumerate}
\end{exercicebox}

\begin{exercicebox}[Exercice 8 : Corrélation (Calcul)]
En utilisant les résultats des exercices 6 et 7, calculez le coefficient de corrélation $\text{Corr}(X,Y)$.
\end{exercicebox}

% --- Propriétés de la Variance et de la Covariance ---

\begin{exercicebox}[Exercice 9 : Variance d'une Somme (Non Indépendant)]
Soient $X$ et $Y$ deux variables aléatoires telles que $\text{Var}(X) = 10$, $\text{Var}(Y) = 5$ et $\text{Cov}(X,Y) = 2$.
Calculez $\text{Var}(X+Y)$.
\end{exercicebox}

\begin{exercicebox}[Exercice 10 : Variance d'une Différence (Indépendant)]
Soient $X$ et $Y$ deux variables aléatoires \textbf{indépendantes} telles que $\text{Var}(X) = 16$ et $\text{Var}(Y) = 9$.
\begin{enumerate}
    \item Que vaut $\text{Cov}(X,Y)$ ?
    \item Calculez $\text{Var}(X-Y)$. (Rappel : $\text{Var}(X-Y) = \text{Var}(X) + \text{Var}(Y) - 2\text{Cov}(X,Y)$).
\end{enumerate}
\end{exercicebox}

\begin{exercicebox}[Exercice 11 : Bilinéarité de la Covariance]
Soient $X, Y, Z$ trois variables aléatoires. Exprimez $\text{Cov}(X+Y, Z)$ en fonction des covariances des variables individuelles.
\end{exercicebox}

\begin{exercicebox}[Exercice 12 : Variance d'une Combinaison Linéaire]
Soient $X$ et $Y$ deux variables aléatoires indépendantes avec $\text{Var}(X) = 4$ et $\text{Var}(Y) = 2$.
Calculez $\text{Var}(3X - 5Y + 1)$.
\end{exercicebox}

\begin{exercicebox}[Exercice 13 : Variance d'une Somme (Dés)]
On lance deux dés équilibrés $D_1$ et $D_2$. Soit $S = D_1 + D_2$.
On rappelle que pour un dé, $\text{Var}(D_i) = 35/12$.
\begin{enumerate}
    \item Les variables $D_1$ et $D_2$ sont-elles indépendantes ?
    \item Calculez $\text{Var}(S)$.
\end{enumerate}
\end{exercicebox}

\begin{exercicebox}[Exercice 14 : Covariance et Variance]
Soit $X$ une variable aléatoire. En utilisant la bilinéarité de la covariance, montrez que $\text{Cov}(X, X) = \text{Var}(X)$.
\end{exercicebox}

\begin{exercicebox}[Exercice 15 : Covariance avec une Constante]
Soit $X$ une variable aléatoire et $c$ une constante.
Montrez que $\text{Cov}(X, c) = 0$. (Indice : $E[c]=c$ et $E[Xc] = cE[X]$).
\end{exercicebox}

% --- Standardisation et Somme de Poissons ---

\begin{exercicebox}[Exercice 16 : Standardisation (Centrer-Réduire)]
Soit $X$ une variable aléatoire avec $E[X] = 10$ et $\text{Var}(X) = 4$.
Soit $Z = \frac{X - E[X]}{\sqrt{\text{Var}(X)}} = \frac{X - 10}{2}$ la variable standardisée.
\begin{enumerate}
    \item Calculez $E[Z]$.
    \item Calculez $\text{Var}(Z)$.
\end{enumerate}
\end{exercicebox}

\begin{exercicebox}[Exercice 17 : Corrélation et Standardisation]
Soit $\text{Corr}(X,Y) = 0.5$. Soient $Z_X$ et $Z_Y$ les versions standardisées de $X$ et $Y$.
Que vaut $\text{Cov}(Z_X, Z_Y)$ ? (Indice : regardez l'intuition de la corrélation).
\end{exercicebox}

\begin{exercicebox}[Exercice 18 : Somme de Lois de Poisson]
Un magasin reçoit des clients au comptoir A selon $X \sim \text{Poisson}(\lambda_1=5 \text{ clients/heure})$ et au comptoir B selon $Y \sim \text{Poisson}(\lambda_2=3 \text{ clients/heure})$. On suppose que $X$ et $Y$ sont indépendantes.
Soit $S = X+Y$ le nombre total de clients arrivant au magasin en une heure.
\begin{enumerate}
    \item Quelle est la loi de $S$ ? Donnez son nom et son paramètre.
    \item Quelle est la probabilité qu'exactement 6 clients au total arrivent en une heure, $P(S=6)$ ?
\end{enumerate}
\end{exercicebox}

\begin{exercicebox}[Exercice 19 : Corrélation Nulle mais Dépendance]
Soit $X$ une variable aléatoire $X \in \{-1, 0, 1\}$, avec $P(X=-1)=1/3$, $P(X=0)=1/3$, $P(X=1)=1/3$.
Soit $Y = X^2$.
\begin{enumerate}
    \item Calculez $E[X]$.
    \item Calculez $E[XY]$. (Indice : $E[XY] = E[X^3]$).
    \item Calculez $\text{Cov}(X,Y)$.
    \item Les variables $X$ et $Y$ sont-elles indépendantes ?
\end{enumerate}
\end{exercicebox}

\begin{exercicebox}[Exercice 20 : Bornes de la Corrélation]
Soit $X$ une variable aléatoire et $Y = -3X + 5$.
Sans faire de calcul, que vaut $\text{Corr}(X,Y)$ ? Justifiez.
\end{exercicebox}

\subsection{Corrections des Exercices}

% --- Corrections : Lois Jointes et Marginales (Discret) ---

\begin{correctionbox}[Correction Exercice 1 : Loi Jointe et Marginales]
1.  On somme toutes les probabilités du tableau :
    $0.1 + 0.2 + 0.1 + 0.3 + 0.1 + 0.2 = 1.0$.
    Puisque la somme est 1 et toutes les probabilités sont non négatives, c'est une loi valide.

2.  Loi marginale de $X$ (somme des lignes) :
    $P(X=0) = P(X=0, Y=0) + P(X=0, Y=1) + P(X=0, Y=2) = 0.1 + 0.2 + 0.1 = 0.4$.
    $P(X=1) = P(X=1, Y=0) + P(X=1, Y=1) + P(X=1, Y=2) = 0.3 + 0.1 + 0.2 = 0.6$.

3.  Loi marginale de $Y$ (somme des colonnes) :
    $P(Y=0) = P(X=0, Y=0) + P(X=1, Y=0) = 0.1 + 0.3 = 0.4$.
    $P(Y=1) = P(X=0, Y=1) + P(X=1, Y=1) = 0.2 + 0.1 = 0.3$.
    $P(Y=2) = P(X=0, Y=2) + P(X=1, Y=2) = 0.1 + 0.2 = 0.3$.
\end{correctionbox}

\begin{correctionbox}[Correction Exercice 2 : Calcul de Probabilité Jointe]
1.  $P(X=0, Y \le 1) = P(X=0, Y=0) + P(X=0, Y=1) = 0.1 + 0.2 = 0.3$.
2.  $P(X=Y) = P(X=0, Y=0) + P(X=1, Y=1) = 0.1 + 0.1 = 0.2$.
3.  $P(X > Y) = P(X=1, Y=0) = 0.3$. (C'est la seule case où $x > y$).
\end{correctionbox}

\begin{correctionbox}[Correction Exercice 3 : Indépendance (Loi Jointe)]
On utilise les lois marginales de l'exercice 1 : $P(X=0)=0.4$ et $P(Y=0)=0.4$.
1.  $P(X=0) \times P(Y=0) = 0.4 \times 0.4 = 0.16$.
2.  Dans le tableau joint, $P(X=0, Y=0) = 0.1$.
3.  Puisque $P(X=0, Y=0) \neq P(X=0) \times P(Y=0)$ (car $0.1 \neq 0.16$), les variables $X$ et $Y$ \textbf{ne sont pas indépendantes}. (Un seul contre-exemple suffit).
\end{correctionbox}

% --- Corrections : Espérance, Covariance et Corrélation ---

\begin{correctionbox}[Correction Exercice 4 : Espérances Marginales]
1.  $E[X] = \sum_x x P(X=x) = (0)(P(X=0)) + (1)(P(X=1))$
    $E[X] = (0)(0.4) + (1)(0.6) = 0.6$.
2.  $E[Y] = \sum_y y P(Y=y) = (0)(P(Y=0)) + (1)(P(Y=1)) + (2)(P(Y=2))$
    $E[Y] = (0)(0.4) + (1)(0.3) + (2)(0.3) = 0 + 0.3 + 0.6 = 0.9$.
\end{correctionbox}

\begin{correctionbox}[Correction Exercice 5 : Espérance d'une Fonction (LOTUS)]
On somme $(xy)P(X=x, Y=y)$ sur les 6 cases. Les termes où $x=0$ ou $y=0$ sont nuls.
$E[XY] = (0 \cdot 0)(0.1) + (0 \cdot 1)(0.2) + (0 \cdot 2)(0.1) + (1 \cdot 0)(0.3) + (1 \cdot 1)(0.1) + (1 \cdot 2)(0.2)$
$E[XY] = 0 + 0 + 0 + 0 + (1)(0.1) + (2)(0.2) = 0.1 + 0.4 = 0.5$.
\end{correctionbox}

\begin{correctionbox}[Correction Exercice 6 : Covariance (Calcul)]
On utilise la formule $\text{Cov}(X,Y) = E[XY] - E[X]E[Y]$.
$$ \text{Cov}(X,Y) = 0.5 - (0.6)(0.9) = 0.5 - 0.54 = -0.04 $$
\end{correctionbox}

\begin{correctionbox}[Correction Exercice 7 : Variances Marginales]
1.  Pour $X$:
    $E[X^2] = (0^2)(0.4) + (1^2)(0.6) = 0.6$.
    $\text{Var}(X) = E[X^2] - (E[X])^2 = 0.6 - (0.6)^2 = 0.6 - 0.36 = 0.24$.
2.  Pour $Y$:
    $E[Y^2] = (0^2)(0.4) + (1^2)(0.3) + (2^2)(0.3) = 0 + 0.3 + (4)(0.3) = 0.3 + 1.2 = 1.5$.
    $\text{Var}(Y) = E[Y^2] - (E[Y])^2 = 1.5 - (0.9)^2 = 1.5 - 0.81 = 0.69$.
\end{correctionbox}

\begin{correctionbox}[Correction Exercice 8 : Corrélation (Calcul)]
On utilise la formule $\text{Corr}(X,Y) = \frac{\text{Cov}(X,Y)}{\sqrt{\text{Var}(X)\text{Var}(Y)}}$.
$$ \text{Corr}(X,Y) = \frac{-0.04}{\sqrt{0.24 \times 0.69}} = \frac{-0.04}{\sqrt{0.1656}} \approx \frac{-0.04}{0.4069} \approx -0.098 $$
La corrélation est très faible et négative.
\end{correctionbox}

% --- Corrections : Propriétés de la Variance et de la Covariance ---

\begin{correctionbox}[Correction Exercice 9 : Variance d'une Somme (Non Indépendant)]
On utilise la formule générale :
$$ \text{Var}(X+Y) = \text{Var}(X) + \text{Var}(Y) + 2\text{Cov}(X,Y) $$
$$ \text{Var}(X+Y) = 10 + 5 + 2(2) = 15 + 4 = 19 $$
\end{correctionbox}

\begin{correctionbox}[Correction Exercice 10 : Variance d'une Différence (Indépendant)]
1.  Puisque $X$ et $Y$ sont indépendantes, leur covariance est nulle : $\text{Cov}(X,Y) = 0$.
2.  On utilise la formule générale :
    $$ \text{Var}(X-Y) = \text{Var}(X + (-1)Y) = \text{Var}(X) + \text{Var}(-1 \cdot Y) + 2\text{Cov}(X, -Y) $$
    $$ = \text{Var}(X) + (-1)^2 \text{Var}(Y) - 2\text{Cov}(X, Y) $$
    $$ \text{Var}(X-Y) = \text{Var}(X) + \text{Var}(Y) - 2(0) $$
    $$ \text{Var}(X-Y) = 16 + 9 = 25 $$
\end{correctionbox}

\begin{correctionbox}[Correction Exercice 11 : Bilinéarité de la Covariance]
La covariance est linéaire sur son premier argument :
$$ \text{Cov}(X+Y, Z) = \text{Cov}(X, Z) + \text{Cov}(Y, Z) $$
\end{correctionbox}

\begin{correctionbox}[Correction Exercice 12 : Variance d'une Combinaison Linéaire]
On utilise $\text{Var}(aX + bY + c) = a^2 \text{Var}(X) + b^2 \text{Var}(Y) + 2ab\text{Cov}(X,Y)$.
Ici $a=3$, $b=-5$, $c=1$. $X$ et $Y$ sont indépendantes, donc $\text{Cov}(X,Y)=0$.
$$ \text{Var}(3X - 5Y + 1) = (3)^2 \text{Var}(X) + (-5)^2 \text{Var}(Y) + 0 $$
$$ = 9 \times (4) + 25 \times (2) = 36 + 50 = 86 $$
(Note : la constante additive $c=1$ ne change pas la variance).
\end{correctionbox}

\begin{correctionbox}[Correction Exercice 13 : Variance d'une Somme (Dés)]
1.  Oui, les lancers de deux dés standards sont des événements physiquement indépendants.
2.  Puisqu'ils sont indépendants, $\text{Cov}(D_1, D_2) = 0$.
    $$ \text{Var}(S) = \text{Var}(D_1 + D_2) = \text{Var}(D_1) + \text{Var}(D_2) $$
    $$ \text{Var}(S) = \frac{35}{12} + \frac{35}{12} = \frac{70}{12} = \frac{35}{6} $$
\end{correctionbox}

\begin{correctionbox}[Correction Exercice 14 : Covariance et Variance]
Par définition, $\text{Cov}(A, B) = E[(A-\mu_A)(B-\mu_B)]$.
Posons $A=X$ et $B=X$. Alors $\mu_A = \mu_X$ et $\mu_B = \mu_X$.
$$ \text{Cov}(X, X) = E[(X-\mu_X)(X-\mu_X)] = E[(X-\mu_X)^2] $$
C'est la définition de $\text{Var}(X)$.
\end{correctionbox}

\begin{correctionbox}[Correction Exercice 15 : Covariance avec une Constante]
On utilise la formule de calcul $\text{Cov}(X,c) = E[Xc] - E[X]E[c]$.
Par linéarité, $E[Xc] = cE[X]$.
L'espérance d'une constante est la constante elle-même : $E[c] = c$.
$$ \text{Cov}(X, c) = cE[X] - E[X]c = 0 $$
\end{correctionbox}

% --- Corrections : Standardisation et Somme de Poissons ---

\begin{correctionbox}[Correction Exercice 16 : Standardisation (Centrer-Réduire)]
$Z = \frac{X - 10}{2} = \frac{1}{2}X - 5$.
1.  Calcul de $E[Z]$ par linéarité :
    $$ E[Z] = E\left[ \frac{1}{2}X - 5 \right] = \frac{1}{2}E[X] - 5 = \frac{1}{2}(10) - 5 = 5 - 5 = 0 $$
2.  Calcul de $\text{Var}(Z)$ par les propriétés de la variance :
    $$ \text{Var}(Z) = \text{Var}\left( \frac{1}{2}X - 5 \right) = \left(\frac{1}{2}\right)^2 \text{Var}(X) = \frac{1}{4} \text{Var}(X) $$
    $$ \text{Var}(Z) = \frac{1}{4}(4) = 1 $$
    Par définition, une variable standardisée a une moyenne de 0 et une variance de 1.
\end{correctionbox}

\begin{correctionbox}[Correction Exercice 17 : Corrélation et Standardisation]
La corrélation $\text{Corr}(X,Y)$ EST, par définition, la covariance des versions standardisées $Z_X$ et $Z_Y$.
$$ \text{Cov}(Z_X, Z_Y) = \text{Corr}(X,Y) = 0.5 $$
\end{correctionbox}

\begin{correctionbox}[Correction Exercice 18 : Somme de Lois de Poisson]
1.  Puisque $X$ et $Y$ sont des v.a. de Poisson \textbf{indépendantes}, leur somme $S=X+Y$ suit aussi une \textbf{loi de Poisson}.
    Le nouveau paramètre est la somme des paramètres : $\lambda_S = \lambda_1 + \lambda_2 = 5 + 3 = 8$.
    Donc, $S \sim \text{Poisson}(\lambda=8)$.
2.  On cherche $P(S=6)$ pour $S \sim \text{Poisson}(8)$.
    $$ P(S=6) = \frac{e^{-8} 8^6}{6!} = \frac{e^{-8} \times 262144}{720} = 364.08 \times e^{-8} \approx 0.122 $$
\end{correctionbox}

\begin{correctionbox}[Correction Exercice 19 : Corrélation Nulle mais Dépendance]
1.  $E[X] = (-1)(1/3) + (0)(1/3) + (1)(1/3) = -1/3 + 0 + 1/3 = 0$.
2.  $E[XY] = E[X(X^2)] = E[X^3]$.
    $E[X^3] = (-1)^3(1/3) + (0)^3(1/3) + (1)^3(1/3) = (-1)(1/3) + 0 + (1)(1/3) = 0$.
3.  $\text{Cov}(X,Y) = E[XY] - E[X]E[Y] = 0 - (0)E[Y] = 0$.
    Les variables sont \textbf{non corrélées}.
4.  Les variables $X$ et $Y$ sont-elles indépendantes ? Non.
    Test : $P(X=1, Y=1) \stackrel{?}{=} P(X=1)P(Y=1)$.
    - $P(X=1, Y=1) = P(X=1, X^2=1) = P(X=1) = 1/3$.
    - $P(X=1) = 1/3$.
    - $P(Y=1) = P(X^2=1) = P(X=1) + P(X=-1) = 1/3 + 1/3 = 2/3$.
    - $P(X=1)P(Y=1) = (1/3)(2/3) = 2/9$.
    Puisque $1/3 \neq 2/9$, elles \textbf{ne sont pas indépendantes}.
    C'est un exemple classique de dépendance non linéaire avec covariance nulle.
\end{correctionbox}

\begin{correctionbox}[Correction Exercice 20 : Bornes de la Corrélation]
$Y$ est une fonction linéaire parfaite de $X$ : $Y = aX + b$ avec $a=-3$ et $b=5$.
La corrélation $\text{Corr}(X,Y)$ mesure la force de la relation \textit{linéaire}. Puisqu'elle est parfaite, la corrélation doit être $\pm 1$.
Le coefficient $a = -3$ est négatif, donc la relation est décroissante.
Par conséquent, $\text{Corr}(X,Y) = -1$.
\end{correctionbox}

\subsection{Exercices Pratiques (Python)}

Nous allons maintenant utiliser Python et les bibliothèques \texttt{pandas}, \texttt{numpy} et \texttt{seaborn} pour explorer les concepts de distributions multivariées sur un jeu de données réel : le dataset "penguins".

Ces exercices vont nous permettre de calculer empiriquement :
\begin{itemize}
    \item Les distributions jointes (discrètes)
    \item Les distributions marginales
    \item La covariance
    \item La corrélation
    \item L'effet de la standardisation
\end{itemize}

\begin{codecell}
# Cellule d'installation et d'importation
pip install numpy pandas seaborn
\end{codecell}

\begin{codecell}
import numpy as np
import pandas as pd
import seaborn as sns

# Charger le jeu de donnees et supprimer les lignes avec des valeurs manquantes
# pour simplifier les calculs.
penguins = sns.load_dataset('penguins').dropna()

print("Apercu des donnees (penguins) :")
print(penguins.head())
\end{codecell}

\begin{exercicebox}[Exercice 1 : Lois Jointe et Marginale (Données Discrètes)]
Nous allons analyser la relation entre deux variables catégorielles : \texttt{species} (l'espèce) et \texttt{island} (l'île où vit le manchot).

\textbf{Votre tâche :}
\begin{enumerate}
    \item Utiliser \texttt{pandas.crosstab} pour créer un tableau de contingence (les effectifs) de \texttt{species} et \texttt{island}.
    \item Normaliser ce tableau par l'effectif total pour obtenir la \textbf{loi jointe} (PMF jointe) $P(Species=s, Island=i)$.
    \item Calculer la \textbf{loi marginale} de \texttt{species} en sommant les probabilités jointes sur les colonnes (\texttt{axis=1}).
    \item Calculer la \textbf{loi marginale} de \texttt{island} en sommant sur les lignes (\texttt{axis=0}).
    \item Vérifier que la somme des lois marginales est bien égale à 1.
\end{enumerate}

\begin{codecell}
import pandas as pd

# 1. Tableau de contingence (effectifs)
joint_counts = pd.crosstab(penguins["species"], penguins["island"])
print("--- 1. Tableau de contingence (Effectifs) ---")
print(joint_counts)

# 2. Loi jointe (PMF)
total_penguins = len(penguins)
joint_pmf = joint_counts / total_penguins
print("\n--- 2. Loi Jointe (PMF) ---")
print(joint_pmf)

# 3. Loi marginale de 'species' (somme des lignes)
marginal_species = joint_pmf.sum(axis=1)
print("\n--- 3. Loi Marginale (Species) ---")
print(marginal_species)

# 4. Loi marginale de 'island' (somme des colonnes)
marginal_island = joint_pmf.sum(axis=0)
print("\n--- 4. Loi Marginale (Island) ---")
print(marginal_island)

# 5. Verification
print(f"\n--- 5. Verification ---")
print(f"Somme de la PMF jointe : {joint_pmf.sum().sum():.2f}")
print(f"Somme marginale (Species) : {marginal_species.sum():.2f}")
print(f"Somme marginale (Island) : {marginal_island.sum():.2f}")
\end{codecell}
\end{exercicebox}

\begin{exercicebox}[Exercice 2 : Calcul de Covariance]
Nous allons maintenant calculer la covariance entre deux variables numériques : \texttt{bill\_length\_mm} (longueur du bec) et \texttt{flipper\_length\_mm} (longueur de la nageoire).

Nous utiliserons la formule : $\text{Cov}(X,Y) = E[XY] - E[X]E[Y]$.

\textbf{Votre tâche :}
\begin{enumerate}
    \item Isoler les deux variables $X$ (\texttt{bill\_length\_mm}) et $Y$ (\texttt{flipper\_length\_mm}) dans des variables \texttt{pandas.Series}.
    \item Calculer $E[X]$ et $E[Y]$ (en utilisant \texttt{.mean()}).
    \item Calculer $E[XY]$ (Indice : \texttt{(X * Y).mean()}).
    \item Appliquer la formule pour trouver la covariance.
    \item Vérifier votre résultat en utilisant la matrice de covariance de NumPy : \texttt{np.cov(X, Y)}.
\end{enumerate}

\begin{codecell}
import numpy as np

# 1. Isoler les variables
X = penguins["bill_length_mm"]
Y = penguins["flipper_length_mm"]

# 2. Calculer E[X] et E[Y]
E_X = X.mean()
E_Y = Y.mean()

# 3. Calculer E[XY]
E_XY = (X * Y).mean()

# 4. Appliquer la formule
cov_calc = E_XY - E_X * E_Y
print(f"E[X] = {E_X:.2f}, E[Y] = {E_Y:.2f}, E[XY] = {E_XY:.2f}")
print(f"Covariance (calculee) : {cov_calc:.4f}")

# 5. Verifier avec NumPy
# np.cov retourne une matrice 2x2. La covariance est a [0,1] (ou [1,0])
# ddof=1 (par defaut) calcule la covariance d'echantillon, ce qui est standard.
cov_matrix_np = np.cov(X, Y)
print(f"Covariance (numpy)    : {cov_matrix_np[0, 1]:.4f}")
\end{codecell}
\end{exercicebox}

\begin{exercicebox}[Exercice 3 : Calcul de Corrélation]
La covariance de l'exercice 2 est positive, mais sa valeur (ex: $\approx 50.38$) est difficile à interpréter. Nous allons la normaliser pour obtenir la corrélation.

Nous utiliserons la formule : $\text{Corr}(X,Y) = \frac{\text{Cov}(X,Y)}{\sigma_X \sigma_Y}$.

\textbf{Votre tâche :}
\begin{enumerate}
    \item Récupérer la covariance calculée à l'exercice 2.
    \item Calculer les écarts-types (d'échantillon, \texttt{ddof=1}) $\sigma_X$ et $\sigma_Y$ (en utilisant \texttt{.std()}).
    \item Appliquer la formule pour trouver la corrélation.
    \item Vérifier votre résultat en utilisant \texttt{pandas.Series.corr()} (ex: \texttt{X.corr(Y)}) ou \texttt{np.corrcoef(X, Y)}.
\end{enumerate}

\begin{codecell}
import numpy as np

# X, Y proviennent de l'exercice precedent
X = penguins["bill_length_mm"]
Y = penguins["flipper_length_mm"]
cov_calc = np.cov(X, Y)[0, 1] # Utilisons la valeur numpy pour la precision

# 2. Calculer les ecarts-types (pandas .std() utilise ddof=1 par defaut)
std_X = X.std()
std_Y = Y.std()

# 3. Appliquer la formule
corr_calc = cov_calc / (std_X * std_Y)
print(f"Sigma_X = {std_X:.2f}, Sigma_Y = {std_Y:.2f}")
print(f"Correlation (calculee) : {corr_calc:.4f}")

# 4. Verifier avec Pandas
corr_pandas = X.corr(Y)
print(f"Correlation (pandas)   : {corr_pandas:.4f}")

# 4. Verifier avec NumPy
corr_matrix_np = np.corrcoef(X, Y)
print(f"Correlation (numpy)    : {corr_matrix_np[0, 1]:.4f}")
\end{codecell}
\end{exercicebox}

\begin{exercicebox}[Exercice 4 : Standardisation (Centrer-Réduire)]
Vérifions empiriquement que la standardisation $Z = \frac{X - \mu_X}{\sigma_X}$ produit une nouvelle variable avec une moyenne de 0 et un écart-type de 1.

\textbf{Votre tâche :}
\begin{enumerate}
    \item Isoler la variable $X$ = \texttt{body\_mass\_g}.
    \item Calculer sa moyenne $\mu_X$ et son écart-type $\sigma_X$. 
    (Note : pour cette transformation, on utilise l'écart-type de la population, \texttt{ddof=0}, car on transforme le set de données lui-même).
    \item Créer la variable standardisée $Z = (X - \mu_X) / \sigma_X$.
    \item Calculer la moyenne et l'écart-type de $Z$ et vérifier qu'ils sont (respectivement) très proches de 0 et 1.
\end{enumerate}

\begin{codecell}
import numpy as np

# 1. Isoler la variable
X_mass = penguins["body_mass_g"]

# 2. Calculer la moyenne et l'ecart-type (de la population, ddof=0)
mu_X = X_mass.mean()
std_X = X_mass.std(ddof=0)

print(f"--- Original ---")
print(f"Moyenne (X) : {mu_X:.4f}")
print(f"Ecart-type (X) : {std_X:.4f}")

# 3. Standardiser la variable
Z_mass = (X_mass - mu_X) / std_X

# 4. Verifier la moyenne et l'ecart-type de Z
mean_Z = Z_mass.mean()
std_Z = Z_mass.std(ddof=0) # On utilise ddof=0 ici aussi

print(f"\n--- Standardise ---")
# La moyenne sera un tres petit nombre (ex: 1.23e-16) a cause de la precision
print(f"Moyenne (Z) : {mean_Z:.2e}") 
print(f"Ecart-type (Z) : {std_Z:.4f}")
\end{codecell}
\end{exercicebox}



\subsection{Exercices}

% --- PDF, CDF et Loi Normale Standard ---

\begin{exercicebox}[Exercice 1 : Concepts de Base $\Phi(z)$]
Soit $Z \sim \mathcal{N}(0, 1)$ la loi normale standard. Sa CDF est $\Phi(z)$.
Exprimez les probabilités suivantes en termes de $\Phi(z)$ :
\begin{enumerate}
    \item $P(Z \le 1.5)$
    \item $P(Z > 1)$
    \item $P(Z \le -1.5)$ (Indice : utilisez la symétrie $\Phi(-z) = 1 - \Phi(z)$)
    \item $P(-1.5 \le Z \le 1.5)$
\end{enumerate}
\end{exercicebox}

\begin{exercicebox}[Exercice 2 : Utilisation d'une Table $\Phi(z)$]
En utilisant une table ou une calculatrice pour la loi $\mathcal{N}(0, 1)$, on sait que $\Phi(1) \approx 0.8413$, $\Phi(1.96) \approx 0.975$ et $\Phi(2) \approx 0.9772$.
Calculez :
\begin{enumerate}
    \item $P(Z > 1)$
    \item $P(Z \le -2)$
    \item $P(-1.96 \le Z \le 1.96)$
\end{enumerate}
\end{exercicebox}

\begin{exercicebox}[Exercice 3 : Propriétés de la PDF $\phi(z)$]
Soit $\phi(z)$ la PDF de la loi $\mathcal{N}(0, 1)$.
\begin{enumerate}
    \item Quelle est la valeur de $\phi(0)$ ? (Le pic de la courbe).
    \item Que vaut $\phi(z)$ par rapport à $\phi(-z)$ ?
    \item Que vaut $\int_{-\infty}^{\infty} \phi(z) \, dz$ ?
\end{enumerate}
\end{exercicebox}

% --- Standardisation (Z-score) et Calcul de Probabilités ---

\begin{exercicebox}[Exercice 4 : Calcul de Z-scores]
Une variable aléatoire $X$ suit une loi normale $\mathcal{N}(\mu=50, \sigma^2=100)$. Notez que $\sigma=10$.
Calculez le Z-score pour les valeurs suivantes de $X$ :
\begin{enumerate}
    \item $x = 60$
    \item $x = 50$
    \item $x = 35$
\end{enumerate}
\end{exercicebox}

\begin{exercicebox}[Exercice 5 : Calcul de Probabilité (Général)]
La taille des hommes adultes dans un pays suit une loi normale $\mathcal{N}(175 \text{ cm}, 7^2 \text{ cm}^2)$.
Soit $X$ la taille d'un homme choisi au hasard. Calculez :
\begin{enumerate}
    \item $P(X \le 182 \text{ cm})$ (Indice : Standardisez $x=182$ et utilisez $\Phi(1) \approx 0.8413$)
    \item $P(X > 168 \text{ cm})$
\end{enumerate}
\end{exercicebox}

\begin{exercicebox}[Exercice 6 : Calcul de Probabilité (Intervalle)]
Les scores à un test de QI suivent une loi normale $\mathcal{N}(100, 15^2)$.
Quelle est la probabilité qu'une personne choisie au hasard ait un QI compris entre 85 et 115 ?
(Indice : Standardisez les deux bornes).
\end{exercicebox}

\begin{exercicebox}[Exercice 7 : Calcul de Probabilité (Queue Extrême)]
En utilisant la même loi $\mathcal{N}(100, 15^2)$ pour le QI :
Quelle est la probabilité qu'une personne ait un QI supérieur à 130 ?
(Indice : Utilisez $\Phi(2) \approx 0.9772$).
\end{exercicebox}

% --- Problèmes Inverses (Trouver x) ---

\begin{exercicebox}[Exercice 8 : Problème Inverse (Z-score)]
Soit $Z \sim \mathcal{N}(0, 1)$. Trouvez la valeur $z$ telle que :
(Utilisez $\Phi(1.28) \approx 0.90$ et $\Phi(1.645) \approx 0.95$)
\begin{enumerate}
    \item $P(Z \le z) = 0.90$
    \item $P(Z > z) = 0.05$ (Indice : si $P(Z>z)=0.05$, que vaut $P(Z \le z)$ ?)
    \item $P(Z \le z) = 0.10$ (Indice : utilisez la symétrie)
\end{enumerate}
\end{exercicebox}

\begin{exercicebox}[Exercice 9 : Problème Inverse (Général)]
Les scores au test $\mathcal{N}(100, 15^2)$ sont utilisés pour sélectionner des candidats. Seul le top 5\% des scores est accepté.
Quel est le score minimum requis pour être accepté ?
(Indice : Utilisez $z \approx 1.645$ pour le top 5\%).
\end{exercicebox}

\begin{exercicebox}[Exercice 10 : Problème Inverse (Intervalle Central)]
Soit $Z \sim \mathcal{N}(0, 1)$. Trouvez la valeur $z$ telle que $P(-z \le Z \le z) = 0.95$.
(Indice : si 95\% est au centre, combien reste-t-il dans chaque queue ? Utilisez $\Phi(1.96) \approx 0.975$).
\end{exercicebox}

\begin{exercicebox}[Exercice 11 : Problème Inverse (Général)]
La durée de vie d'une batterie suit $\mathcal{N}(500 \text{ heures}, 50^2 \text{ heures}^2)$.
Le fabricant veut offrir une garantie. Il ne veut remplacer que 2.5\% des batteries.
Quelle durée de garantie (en heures) doit-il proposer ?
(Indice : $P(Z \le -1.96) \approx 0.025$).
\end{exercicebox}

% --- Règle Empirique (68-95-99.7) ---

\begin{exercicebox}[Exercice 12 : Règle Empirique (Application)]
Le poids de paquets de café suit $\mathcal{N}(250g, 5^2g^2)$.
En utilisant la règle empirique (68-95-99.7), donnez un intervalle qui contient :
\begin{enumerate}
    \item Environ 68\% des poids.
    \item Environ 95\% des poids.
    \item Environ 99.7\% des poids.
\end{enumerate}
\end{exercicebox}

\begin{exercicebox}[Exercice 13 : Règle Empirique (Probabilité)]
En utilisant la situation de l'exercice 12 ($\mathcal{N}(250, 5^2)$) et la règle empirique :
\begin{enumerate}
    \item Estimez $P(245 \le X \le 255)$.
    \item Estimez $P(X \le 240)$. (Indice : L'intervalle $\mu \pm 2\sigma$ est [240, 260] et contient 95\%. Utilisez la symétrie).
\end{enumerate}
\end{exercicebox}

% --- Propriétés (Transformations Linéaires et Sommes) ---

\begin{exercicebox}[Exercice 14 : Transformation Linéaire (Celsius -> Fahrenheit)]
La température $T_C$ à midi en été dans une ville suit $\mathcal{N}(25, 3^2)$ (en degrés Celsius).
On convertit la température en Fahrenheit : $T_F = 1.8 \times T_C + 32$.
Quelle est la loi de $T_F$ ? (Donnez sa moyenne et sa variance).
\end{exercicebox}

\begin{exercicebox}[Exercice 15 : Transformation Linéaire (Z-score)]
Soit $X \sim \mathcal{N}(\mu, \sigma^2)$. Soit $Y = aX+b$.
Trouvez $a$ et $b$ (en fonction de $\mu$ et $\sigma$) tels que $Y \sim \mathcal{N}(0, 1)$.
\end{exercicebox}

\begin{exercicebox}[Exercice 16 : Somme de Normales Indépendantes]
Soit $X \sim \mathcal{N}(10, 3^2)$ et $Y \sim \mathcal{N}(20, 4^2)$. $X$ et $Y$ sont indépendantes.
Soit $S = X + Y$.
\begin{enumerate}
    \item Quelle est la loi de $S$ ? (Donnez sa moyenne et sa variance).
    \item Quel est l'écart-type de $S$ ?
\end{enumerate}
\end{exercicebox}

\begin{exercicebox}[Exercice 17 : Différence de Normales Indépendantes]
En utilisant $X$ et $Y$ de l'exercice 16, soit $D = Y - X$.
\begin{enumerate}
    \item Quelle est la loi de $D$ ? (Donnez sa moyenne et sa variance).
    \item Quel est l'écart-type de $D$ ? (Comparez-le à celui de $S$).
\end{enumerate}
\end{exercicebox}

\begin{exercicebox}[Exercice 18 : Application (Somme)]
Le poids d'une boîte vide $B$ suit $\mathcal{N}(100g, 5^2)$. Le poids du contenu $C$ suit $\mathcal{N}(800g, 10^2)$. $B$ et $C$ sont indépendants.
Soit $T = B+C$ le poids total.
\begin{enumerate}
    \item Quelle est la loi de $T$ ?
    \item Calculez $P(T > 925g)$. (Utilisez $\Phi(2) \approx 0.9772$).
\end{enumerate}
\end{exercicebox}

\begin{exercicebox}[Exercice 19 : Moyenne d'un Échantillon (Avancé)]
Soient $X_1, X_2, X_3, X_4$ quatre observations indépendantes de la loi $\mathcal{N}(10, 4^2)$.
Soit $\bar{X} = \frac{X_1 + X_2 + X_3 + X_4}{4}$ la moyenne de l'échantillon.
\begin{enumerate}
    \item Soit $S = X_1+X_2+X_3+X_4$. Quelle est la loi de $S$ ?
    \item En utilisant la transformation linéaire $\bar{X} = \frac{1}{4}S$, quelle est la loi de $\bar{X}$ ?
\end{enumerate}
\end{exercicebox}

\begin{exercicebox}[Exercice 20 : Comparaison (Différence)]
Alice et Bob passent un examen. Les notes d'Alice $A$ suivent $\mathcal{N}(80, 5^2)$. Les notes de Bob $B$ suivent $\mathcal{N}(78, 3^2)$. On suppose leurs notes indépendantes.
Quelle est la probabilité que Bob ait une meilleure note qu'Alice ?
(Indice : Calculez $P(B > A)$, ce qui est équivalent à $P(B - A > 0)$).
\end{exercicebox}

\subsection{Corrections des Exercices}

% --- Corrections : PDF, CDF et Loi Normale Standard ---

\begin{correctionbox}[Correction Exercice 1 : Concepts de Base $\Phi(z)$]
1.  $P(Z \le 1.5) = \Phi(1.5)$.
2.  $P(Z > 1) = 1 - P(Z \le 1) = 1 - \Phi(1)$.
3.  $P(Z \le -1.5) = 1 - P(Z \le 1.5) = 1 - \Phi(1.5)$.
4.  $P(-1.5 \le Z \le 1.5) = P(Z \le 1.5) - P(Z \le -1.5) = \Phi(1.5) - (1 - \Phi(1.5)) = 2\Phi(1.5) - 1$.
\end{correctionbox}

\begin{correctionbox}[Correction Exercice 2 : Utilisation d'une Table $\Phi(z)$]
Données : $\Phi(1) \approx 0.8413$, $\Phi(1.96) \approx 0.975$, $\Phi(2) \approx 0.9772$.
1.  $P(Z > 1) = 1 - \Phi(1) \approx 1 - 0.8413 = 0.1587$.
2.  $P(Z \le -2) = 1 - \Phi(2) \approx 1 - 0.9772 = 0.0228$.
3.  $P(-1.96 \le Z \le 1.96) = \Phi(1.96) - \Phi(-1.96) = \Phi(1.96) - (1 - \Phi(1.96))$
    $= 2\Phi(1.96) - 1 \approx 2(0.975) - 1 = 1.95 - 1 = 0.95$.
    (C'est l'intervalle de confiance à 95\%).
\end{correctionbox}

\begin{correctionbox}[Correction Exercice 3 : Propriétés de la PDF $\phi(z)$]
$\phi(z) = \frac{1}{\sqrt{2\pi}} e^{-z^2/2}$.
1.  $\phi(0) = \frac{1}{\sqrt{2\pi}} e^{0} = \frac{1}{\sqrt{2\pi}} \approx 0.3989$.
2.  Puisque $z^2 = (-z)^2$, on a $\phi(z) = \phi(-z)$. La fonction est paire (symétrique par rapport à l'axe y).
3.  Par définition d'une PDF, l'aire totale sous la courbe doit être 1. $\int_{-\infty}^{\infty} \phi(z) \, dz = 1$.
\end{correctionbox}

% --- Corrections : Standardisation (Z-score) et Calcul de Probabilités ---

\begin{correctionbox}[Correction Exercice 4 : Calcul de Z-scores]
$X \sim \mathcal{N}(\mu=50, \sigma^2=100) \implies \sigma=10$.
$Z = \frac{X - \mu}{\sigma}$.
1.  $x = 60 \implies z = (60 - 50) / 10 = 10 / 10 = 1$.
2.  $x = 50 \implies z = (50 - 50) / 10 = 0 / 10 = 0$.
3.  $x = 35 \implies z = (35 - 50) / 10 = -15 / 10 = -1.5$.
\end{correctionbox}

\begin{correctionbox}[Correction Exercice 5 : Calcul de Probabilité (Général)]
$X \sim \mathcal{N}(175, 7^2)$. $\mu=175, \sigma=7$.
1.  $P(X \le 182) = P\left(Z \le \frac{182 - 175}{7}\right) = P(Z \le \frac{7}{7}) = P(Z \le 1) = \Phi(1) \approx 0.8413$.
2.  $P(X > 168) = P\left(Z > \frac{168 - 175}{7}\right) = P(Z > \frac{-7}{7}) = P(Z > -1)$.
    Par symétrie, $P(Z > -1) = P(Z < 1) = \Phi(1) \approx 0.8413$.
\end{correctionbox}

\begin{correctionbox}[Correction Exercice 6 : Calcul de Probabilité (Intervalle)]
$X \sim \mathcal{N}(100, 15^2)$. $\mu=100, \sigma=15$.
On cherche $P(85 \le X \le 115)$.
$z_1 = (85 - 100) / 15 = -15 / 15 = -1$.
$z_2 = (115 - 100) / 15 = 15 / 15 = 1$.
$P(-1 \le Z \le 1) = \Phi(1) - \Phi(-1) = \Phi(1) - (1 - \Phi(1)) = 2\Phi(1) - 1$.
En utilisant $\Phi(1) \approx 0.8413$, $P \approx 2(0.8413) - 1 = 1.6826 - 1 = 0.6826$.
(On retrouve la règle des 68\%).
\end{correctionbox}

\begin{correctionbox}[Correction Exercice 7 : Calcul de Probabilité (Queue Extrême)]
$X \sim \mathcal{N}(100, 15^2)$.
On cherche $P(X > 130)$.
$z = (130 - 100) / 15 = 30 / 15 = 2$.
$P(X > 130) = P(Z > 2) = 1 - P(Z \le 2) = 1 - \Phi(2)$.
En utilisant $\Phi(2) \approx 0.9772$, $P \approx 1 - 0.9772 = 0.0228$.
\end{correctionbox}

% --- Corrections : Problèmes Inverses (Trouver x) ---

\begin{correctionbox}[Correction Exercice 8 : Problème Inverse (Z-score)]
1.  $P(Z \le z) = 0.90 \implies z = \Phi^{-1}(0.90) \approx 1.28$.
2.  $P(Z > z) = 0.05 \implies P(Z \le z) = 1 - 0.05 = 0.95$.
    $z = \Phi^{-1}(0.95) \approx 1.645$.
3.  $P(Z \le z) = 0.10$. C'est dans la queue gauche. Par symétrie, $z = - \Phi^{-1}(1 - 0.10) = - \Phi^{-1}(0.90)$.
    $z \approx -1.28$.
\end{correctionbox}

\begin{correctionbox}[Correction Exercice 9 : Problème Inverse (Général)]
$X \sim \mathcal{N}(100, 15^2)$. On cherche $x$ tel que $P(X > x) = 0.05$.
1.  Trouver le Z-score : $P(Z > z) = 0.05 \implies P(Z \le z) = 0.95 \implies z \approx 1.645$.
2.  Convertir en $x$ : $z = (x-\mu)/\sigma \implies x = \mu + z\sigma$.
    $x = 100 + (1.645)(15) = 100 + 24.675 = 124.675$.
    Le score minimum est d'environ 125.
\end{correctionbox}

\begin{correctionbox}[Correction Exercice 10 : Problème Inverse (Intervalle Central)]
$P(-z \le Z \le z) = 0.95$.
Si 95\% est au centre, il reste $1 - 0.95 = 0.05$ (ou 5\%) dans les deux queues.
Par symétrie, chaque queue a $0.05 / 2 = 0.025$.
La probabilité à gauche de $z$ est $P(Z \le z) = 0.95 + 0.025 = 0.975$.
On cherche $z = \Phi^{-1}(0.975)$.
En utilisant l'indice, $z \approx 1.96$.
\end{correctionbox}

\begin{correctionbox}[Correction Exercice 11 : Problème Inverse (Général)]
$X \sim \mathcal{N}(500, 50^2)$. On cherche $x$ tel que $P(X \le x) = 0.025$.
1.  Trouver le Z-score : $P(Z \le z) = 0.025$. C'est la queue gauche.
    En utilisant l'indice $P(Z \le -1.96) \approx 0.025$, on a $z \approx -1.96$.
2.  Convertir en $x$ : $x = \mu + z\sigma$.
    $x = 500 + (-1.96)(50) = 500 - 98 = 402$.
    Le fabricant doit proposer une garantie de 402 heures.
\end{correctionbox}

% --- Corrections : Règle Empirique (68-95-99.7) ---

\begin{correctionbox}[Correction Exercice 12 : Règle Empirique (Application)]
$X \sim \mathcal{N}(\mu=250, \sigma=5)$.
1.  68\% $\implies \mu \pm 1\sigma = 250 \pm 5 \implies [245, 255]$.
2.  95\% $\implies \mu \pm 2\sigma = 250 \pm 2(5) = 250 \pm 10 \implies [240, 260]$.
3.  99.7\% $\implies \mu \pm 3\sigma = 250 \pm 3(5) = 250 \pm 15 \implies [235, 265]$.
\end{correctionbox}

\begin{correctionbox}[Correction Exercice 13 : Règle Empirique (Probabilité)]
1.  $P(245 \le X \le 255)$ est l'intervalle $\mu \pm 1\sigma$.
    La probabilité est d'environ 68\% ou 0.68.
2.  L'intervalle $\mu \pm 2\sigma$ est $[240, 260]$ et contient 95\% des données.
    Il reste $100\% - 95\% = 5\%$ dans les deux queues (i.e., $P(X < 240) + P(X > 260) = 0.05$).
    Par symétrie, la queue gauche $P(X < 240)$ est $0.05 / 2 = 0.025$.
    La probabilité est d'environ 2.5\% ou 0.025.
\end{correctionbox}

% --- Corrections : Propriétés (Transformations Linéaires et Sommes) ---

\begin{correctionbox}[Correction Exercice 14 : Transformation Linéaire]
$T_C \sim \mathcal{N}(25, 3^2)$. $T_F = a T_C + b$ avec $a=1.8$ et $b=32$.
Loi de $T_F$ : $T_F \sim \mathcal{N}(a\mu + b, (a\sigma)^2)$.
Moyenne : $E[T_F] = 1.8(25) + 32 = 45 + 32 = 77$.
Variance : $\text{Var}(T_F) = (1.8)^2 \text{Var}(T_C) = (1.8)^2 (3^2) = (1.8 \times 3)^2 = (5.4)^2 = 29.16$.
Donc, $T_F \sim \mathcal{N}(77, 29.16)$.
\end{correctionbox}

\begin{correctionbox}[Correction Exercice 15 : Transformation Linéaire (Z-score)]
On veut $Y = aX+b \sim \mathcal{N}(0, 1)$.
$E[Y] = aE[X] + b = a\mu + b$. On veut $a\mu + b = 0$.
$\text{Var}(Y) = a^2 \text{Var}(X) = a^2 \sigma^2$. On veut $a^2 \sigma^2 = 1$.
De $\text{Var}(Y)=1 \implies a^2 = 1/\sigma^2 \implies a = 1/\sigma$ (en supposant $a>0$).
De $E[Y]=0 \implies (1/\sigma)\mu + b = 0 \implies b = -\mu/\sigma$.
Les constantes sont $a = 1/\sigma$ et $b = -\mu/\sigma$. (C'est la définition de la standardisation).
\end{correctionbox}

\begin{correctionbox}[Correction Exercice 16 : Somme de Normales Indépendantes]
$X \sim \mathcal{N}(10, 9)$ et $Y \sim \mathcal{N}(20, 16)$. $S = X+Y$.
1.  La somme de normales indépendantes est une normale.
    $E[S] = E[X] + E[Y] = 10 + 20 = 30$.
    $\text{Var}(S) = \text{Var}(X) + \text{Var}(Y) = 9 + 16 = 25$.
    Donc, $S \sim \mathcal{N}(30, 25)$.
2.  $\text{Var}(S) = 25 \implies \sigma_S = \sqrt{25} = 5$.
\end{correctionbox}

\begin{correctionbox}[Correction Exercice 17 : Différence de Normales Indépendantes]
$D = Y - X$.
1.  La différence est aussi une normale.
    $E[D] = E[Y] - E[X] = 20 - 10 = 10$.
    $\text{Var}(D) = \text{Var}(Y + (-1)X) = \text{Var}(Y) + (-1)^2 \text{Var}(X) = \text{Var}(Y) + \text{Var}(X)$.
    $\text{Var}(D) = 16 + 9 = 25$.
    Donc, $D \sim \mathcal{N}(10, 25)$.
2.  $\sigma_D = \sqrt{25} = 5$. (Identique à $\sigma_S$. La variance s'additionne toujours).
\end{correctionbox}

\begin{correctionbox}[Correction Exercice 18 : Application (Somme)]
$B \sim \mathcal{N}(100, 25)$, $C \sim \mathcal{N}(800, 100)$. $T = B+C$.
1.  $E[T] = E[B] + E[C] = 100 + 800 = 900$.
    $\text{Var}(T) = \text{Var}(B) + \text{Var}(C) = 25 + 100 = 125$.
    $T \sim \mathcal{N}(900, 125)$.
2.  $P(T > 925)$. $\sigma_T = \sqrt{125} = \sqrt{25 \times 5} = 5\sqrt{5} \approx 11.18$.
    $z = (925 - 900) / \sqrt{125} = 25 / (5\sqrt{5}) = 5/\sqrt{5} = \sqrt{5} \approx 2.236$.
    $P(T > 925) = P(Z > 2.236) = 1 - \Phi(2.236) \approx 1 - 0.9873 = 0.0127$.
    (Note : L'indice $\Phi(2) \approx 0.9772$ semble être une approximation pour un $z$ de 2, qui n'est pas le bon $z$ ici).
\end{correctionbox}

\begin{correctionbox}[Correction Exercice 19 : Moyenne d'un Échantillon (Avancé)]
$X_i \sim \mathcal{N}(10, 16)$ (indép.). $\bar{X} = \frac{1}{4} S$ où $S = X_1+X_2+X_3+X_4$.
1.  $S$ est une somme de normales indépendantes.
    $E[S] = E[X_1] + \dots + E[X_4] = 4 \times 10 = 40$.
    $\text{Var}(S) = \text{Var}(X_1) + \dots + \text{Var}(X_4) = 4 \times 16 = 64$.
    $S \sim \mathcal{N}(40, 64)$.
2.  $\bar{X}$ est une transformation linéaire de $S$.
    $E[\bar{X}] = E[\frac{1}{4}S] = \frac{1}{4}E[S] = \frac{1}{4}(40) = 10$.
    $\text{Var}(\bar{X}) = \text{Var}(\frac{1}{4}S) = (\frac{1}{4})^2 \text{Var}(S) = \frac{1}{16}(64) = 4$.
    $\bar{X} \sim \mathcal{N}(10, 4)$.
\end{correctionbox}

\begin{correctionbox}[Correction Exercice 20 : Comparaison (Différence)]
$A \sim \mathcal{N}(80, 25)$, $B \sim \mathcal{N}(78, 9)$. Indép.
On cherche $P(B > A)$, ce qui est $P(B - A > 0)$.
Soit $D = B - A$. $D$ suit une loi normale.
$E[D] = E[B] - E[A] = 78 - 80 = -2$.
$\text{Var}(D) = \text{Var}(B) + \text{Var}(A) = 9 + 25 = 34$.
Donc $D \sim \mathcal{N}(-2, 34)$. $\sigma_D = \sqrt{34} \approx 5.83$.
On cherche $P(D > 0)$.
$z = (0 - (-2)) / \sqrt{34} = 2 / \sqrt{34} \approx 0.342$.
$P(D > 0) = P(Z > 0.342) = 1 - \Phi(0.342) \approx 1 - 0.6338 = 0.3662$.
Il y a environ 36.6\% de chance que Bob ait une meilleure note.
\end{correctionbox}

\subsection{Exercices Pratiques (Python)}

L'une des applications les plus célèbres de la loi normale est la modélisation des rendements financiers. Bien que ce modèle ne soit pas parfait (les krachs boursiers sont plus fréquents que ne le prédit la loi normale), il constitue la pierre angulaire de la finance moderne.

Nous allons supposer que les \textbf{rendements logarithmiques} quotidiens d'un actif financier (comme l'indice S\&P 500) suivent une loi normale $X \sim \mathcal{N}(\mu, \sigma^2)$.

\begin{itemize}
    \item $\mu$ est le rendement moyen quotidien (souvent proche de zéro).
    \item $\sigma$ est la volatilité quotidienne (l'écart-type du rendement).
\end{itemize}

Nous utiliserons \texttt{yfinance} pour obtenir des données réelles, \texttt{numpy} pour les calculs, et \texttt{scipy.stats} pour les fonctions $\Phi$ (CDF) et $\Phi^{-1}$ (PPF).

\begin{codecell}
# Cellule d'installation et d'importation
pip install numpy pandas yfinance scipy
\end{codecell}

\begin{codecell}
import numpy as np
import pandas as pd
import yfinance as yf
from scipy.stats import norm
\end{codecell}

\begin{exercicebox}[Exercice 1 : Modélisation des Rendements du S\&P 500]
Notre première étape est d'obtenir les données de l'indice S\&P 500 (ticker : GSPC) et d'estimer les paramètres $\mu$ et $\sigma$ de notre modèle normal.

\textbf{Votre tâche :}
\begin{enumerate}
    \item Télécharger les données du GSPC des 5 dernières années.
    \item Calculer les rendements logarithmiques quotidiens. La formule est $R = \log(P_t / P_{t-1})$. (Indice : utilisez \texttt{np.log(data["Close"] / data["Close"].shift(1))}).
    \item Estimer $\mu$ (la moyenne) et $\sigma$ (l'écart-type) de ces rendements.
    \item Afficher $\mu$ et $\sigma$. Vous avez maintenant votre modèle $X \sim \mathcal{N}(\mu, \sigma^2)$ !
\end{enumerate}


\begin{codecell}
import numpy as np
import yfinance as yf

ticker = "GSPC"
data = yf.download(ticker, period='5y')

# 1. Calculer les rendements log (log(P_t / P_{t-1}))
# Indice : utilisez .shift(1) pour P_{t-1}
# log_returns = ...
log_returns = log_returns.dropna() # On enleve la premiere valeur (NaN)

# 2. Estimer mu (moyenne) et sigma (ecart-type)
# mu = ...
# sigma = ...

# 3. Afficher les parametres
# print(f"Modele N(mu={mu:.6f}, sigma={sigma:.6f})")
\end{codecell}
\end{exercicebox}

\begin{exercicebox}[Exercice 2 : Calcul de Probabilité (Z-score)]
Utilisons notre modèle $\mathcal{N}(\mu, \sigma^2)$ de l'exercice 1. Un "krach" pourrait être défini comme une baisse de plus de 3\% en une seule journée.

Quelle est la probabilité que cela se produise, selon notre modèle ?

\textbf{Votre tâche :}
\begin{enumerate}
    \item Définir la valeur $x$ d'une baisse de 3\% (en rendement log) : $x = \log(0.97)$.
    \item Standardiser $x$ pour obtenir le Z-score : $z = (x - \mu) / \sigma$.
    \item Utiliser \texttt{scipy.stats.norm.cdf(z)} (qui est $\Phi(z)$) pour trouver la probabilité $P(X \le x)$.
\end{enumerate}

\begin{codecell}
from scipy.stats import norm

# Supposons que mu et sigma sont definis (de l'Ex 1)
# mu = ... (copiez votre valeur de l'Ex 1)
# sigma = ... (copiez votre valeur de l'Ex 1)

# 1. Definir x pour une baisse de 3%
# x = np.log(...)

# 2. Standardiser x pour obtenir le Z-score
# z_score = ...

# 3. Utiliser norm.cdf(z) pour trouver P(X <= x)
# probability = ...

# print(f"Probabilite d'une baisse > 3% : {probability:.8f}")
\end{codecell}
\end{exercicebox}

\begin{exercicebox}[Exercice 3 : Problème Inverse (Value at Risk - VaR)]
Le "Value at Risk" (VaR) est un concept financier qui répond à la question : "Quel est le montant minimum que je peux m'attendre à perdre avec une probabilité $p$ ?"

Calculons le 5\% VaR quotidien. C'est la valeur $x$ (rendement) telle que $P(X \le x) = 0.05$.

\textbf{Votre tâche :}
\begin{enumerate}
    \item Trouver le Z-score $z$ qui correspond au 5ème percentile (probabilité 0.05).
    (Indice : utilisez \texttt{scipy.stats.norm.ppf(0.05)}, qui est $\Phi^{-1}(0.05)$).
    \item "Dé-standardiser" ce Z-score pour trouver la valeur $x$ : $x = \mu + z \cdot \sigma$.
    \item Interpréter le résultat (convertir $x$ en pourcentage : $np.exp(x) - 1$).
\end{enumerate}

\begin{codecell}
from scipy.stats import norm

# Supposons que mu et sigma sont definis (de l'Ex 1)
# mu = ...
# sigma = ...

probabilite = 0.05

# 1. Trouver le Z-score pour 5% (Indice: norm.ppf)
# z_score_var = ...

# 2. De-standardiser pour trouver x (x = mu + z*sigma)
# x_var = ...

# 3. Convertir en pourcentage (Indice: np.exp(x_var) - 1)
# percent_loss = ...

# print(f"Le 5% VaR est une perte de {abs(percent_loss):.2f}%")
\end{codecell}
\end{exercicebox}

\begin{exercicebox}[Exercice 4 : Règle Empirique (68-95-99.7)]
Vérifions à quel point la règle empirique (68-95-99.7) s'applique à nos données réelles de \texttt{log\_returns}.

\textbf{Votre tâche :}
\begin{enumerate}
    \item Définir les intervalles $1\sigma$, $2\sigma$, et $3\sigma$ autour de la moyenne $\mu$.
    \item Calculer le pourcentage réel de rendements (dans \texttt{log\_returns}) qui tombent dans chacun de ces trois intervalles.
    \item Comparer ces pourcentages empiriques aux valeurs théoriques (68.3\%, 95.4\%, 99.7\%).
\end{enumerate}

\begin{codecell}
# Supposons que mu, sigma, et log_returns sont definis

# 1. Definir les bornes
borne_1s_inf = mu - 1 * sigma
borne_1s_sup = mu + 1 * sigma
# ... faire de meme pour 2s et 3s ...
borne_2s_inf = ...
borne_2s_sup = ...
borne_3s_inf = ...
borne_3s_sup = ...

# 2. Compter le pourcentage de 'log_returns' dans chaque intervalle
# Indice: ((log_returns > borne_inf) & (log_returns < borne_sup)).mean()
# within_1s = ...
# within_2s = ...
# within_3s = ...

# print(f"Empirique 1-sigma: {within_1s:.4f} (Theorique: 0.6827)")
# print(f"Empirique 2-sigma: {within_2s:.4f} (Theorique: 0.9545)")
# print(f"Empirique 3-sigma: {within_3s:.4f} (Theorique: 0.9973)")
\end{codecell}
\end{exercicebox}

\begin{exercicebox}[Exercice 5 : Stabilité par Addition (Portfolio Simple)]
Un portfolio est composé de 50\% de S\&P 500 (GSPC) et 50\% d'Or (GC=F).
Nous allons modéliser la loi du rendement de ce portfolio, $P = 0.5 X_S + 0.5 X_G$.

Nous utiliserons les théorèmes $E[aX+bY] = aE[X]+bE[Y]$ et, en supposant (pour cet exercice) l'indépendance : $\text{Var}(aX+bY) = a^2\text{Var}(X) + b^2\text{Var}(Y)$.

\textbf{Votre tâche :}
\begin{enumerate}
    \item Télécharger les données de l'Or (\texttt{`GC=F`}) et calculer $\mu_G$ et $\sigma_G^2$ (variance) de ses rendements log.
    \item Récupérer $\mu_S$ et $\sigma_S^2$ (variance) du S\&P 500 de l'exercice 1.
    \item Calculer la moyenne du portfolio $\mu_P = 0.5\mu_S + 0.5\mu_G$.
    \item Calculer la variance du portfolio $\sigma_P^2 = (0.5)^2\sigma_S^2 + (0.5)^2\sigma_G^2$.
    \item Afficher l'écart-type $\sigma_P$ et le comparer à $\sigma_S$ et $\sigma_G$.
\end{enumerate}

\begin{codecell}
# mu_S et sigma_S de l'Ex 1
# mu_S = ...
# var_S = sigma_S**2

# 1. Obtenir les donnees pour l'Or ('GC=F') et calculer mu_G, var_G
# gold_data = ...
# gold_returns = ...
# mu_G = ...
# var_G = ...

# 2. Poids
w_S = 0.5
w_G = 0.5

# 3. Calculer mu_P (moyenne du portfolio)
# mu_P = ...

# 4. Calculer var_P (en supposant l'independance)
# var_P = ...
# sigma_P = np.sqrt(var_P)

# print(f"Volatilite S&P 500: {sigma_S:.6f}")
# print(f"Volatilite Or: {np.sqrt(var_G):.6f}")
# print(f"Volatilite Portfolio (indep): {sigma_P:.6f}")
\end{codecell}
\end{exercicebox}