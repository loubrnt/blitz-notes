\subsection{Exercices}

% --- Section 1 : Définitions, Skewness et Kurtosis ---

\begin{exercicebox}[Exercice 1 : Concepts (Moments)]
\begin{enumerate}
    \item À quoi correspond le 1er moment non centré, $E[X^1]$ ?
    \item À quoi correspond le 2ème moment centré, $E[(X-\mu)^2]$ ?
\end{enumerate}
\end{exercicebox}

\begin{exercicebox}[Exercice 2 : Interprétation (Skewness)]
Une distribution des salaires dans une entreprise a un skewness de +2.5. Qu'est-ce que cela signifie concrètement sur la répartition des salaires ?
\end{exercicebox}

\begin{exercicebox}[Exercice 3 : Interprétation (Skewness)]
Si une distribution est parfaitement symétrique, que vaut son skewness ?
\end{exercicebox}

\begin{exercicebox}[Exercice 4 : Interprétation (Kurtosis)]
Une distribution des rendements d'un actif financier a un "excess kurtosis" de +5.0.
\begin{enumerate}
    \item Comment appelle-t-on ce type de distribution ?
    \item Qu'est-ce que cela implique sur la probabilité des "krachs" (événements extrêmes) par rapport à une loi normale ?
\end{enumerate}
\end{exercicebox}

\begin{exercicebox}[Exercice 5 : Interprétation (Kurtosis)]
Une distribution a un "excess kurtosis" de -1.0.
\begin{enumerate}
    \item Comment appelle-t-on ce type de distribution ?
    \item Comment décririez-vous visuellement son "pic" et ses "queues" par rapport à une loi normale ?
\end{enumerate}
\end{exercicebox}

\begin{exercicebox}[Exercice 6 : Comparaison (Skewness)]
La distribution exponentielle est-elle symétrique, asymétrique à gauche ou asymétrique à droite ? Quel est le signe de son skewness ?
\end{exercicebox}

\begin{exercicebox}[Exercice 7 : Comparaison (Kurtosis)]
La distribution uniforme est-elle mésokurtique, leptokurtique ou platykurtique ? Son "excess kurtosis" est-il positif, négatif ou nul ?
\end{exercicebox}

% --- Section 2 : Symétrie ---

\begin{exercicebox}[Exercice 8 : Propriété de Symétrie]
Soit $X$ une variable aléatoire symétrique autour de sa moyenne $\mu$.
Montrez que son 3ème moment centré, $E[(X-\mu)^3]$, est nul (et donc que son skewness est nul).
\end{exercicebox}

\begin{exercicebox}[Exercice 9 : Vérification de Symétrie (PDF)]
La PDF d'une loi $X$ est $f(x) = \frac{1}{2}e^{-|x-3|}$.
Cette distribution est-elle symétrique ? Si oui, autour de quel point $\mu$ ?
\end{exercicebox}

\begin{exercicebox}[Exercice 10 : Symétrie et Moments]
Si une distribution a un skewness de 0, est-elle forcément symétrique ? (Indice : Pensez à une distribution qui aurait des queues asymétriques mais qui s'annuleraient pour le moment d'ordre 3).
\end{exercicebox}

% --- Section 3 : Moments d'Échantillon vs Population ---

\begin{exercicebox}[Exercice 11 : Définitions (Échantillon vs Pop.)]
Quelle est la différence conceptuelle entre $\sigma^2$ et $s^2$ ?
\end{exercicebox}

\begin{exercicebox}[Exercice 12 : Correction de Bessel ($n-1$)]
Pourquoi utilise-t-on $n-1$ au dénominateur pour $s^2$ ? Que se passerait-il si nous avions un échantillon de $n=1$ et que nous divisions par $n$ ?
\end{exercicebox}

\begin{exercicebox}[Exercice 13 : Calcul (Moments d'Échantillon)]
On observe l'échantillon de données suivant : $\{ 2, 3, 10 \}$.
\begin{enumerate}
    \item Calculez la moyenne d'échantillon $\bar{X}$.
    \item Calculez la variance d'échantillon non biaisée $s^2$.
\end{enumerate}
\end{exercicebox}

% --- Section 4 : Fonctions Génératrices (MGF) - Calculs ---

\begin{exercicebox}[Exercice 14 : MGF (Propriété de base)]
Soit $M_X(t)$ la MGF d'une variable $X$. Que vaut $M_X(0)$ ?
\end{exercicebox}

\begin{exercicebox}[Exercice 15 : MGF (Série de Taylor)]
Le développement en série de Taylor d'une MGF est $M_X(t) = 1 + 5t + 14t^2 + O(t^3)$.
(Rappel : $M_X(t) = 1 + E[X]t + E[X^2]\frac{t^2}{2!} + \dots$)
\begin{enumerate}
    \item Trouvez $E[X]$.
    \item Trouvez $E[X^2]$.
    \item Calculez $\text{Var}(X)$.
\end{enumerate}
\end{exercicebox}

\begin{exercicebox}[Exercice 16 : MGF (Calcul de moments)]
La MGF de la loi Exponentielle de paramètre $\lambda$ est $M_X(t) = \frac{\lambda}{\lambda - t}$.
\begin{enumerate}
    \item Calculez $M_X'(t)$ et trouvez $E[X] = M_X'(0)$.
    \item Calculez $M_X''(t)$ et trouvez $E[X^2] = M_X''(0)$.
\end{enumerate}
\end{exercicebox}

\begin{exercicebox}[Exercice 17 : MGF (Variance de l'Exponentielle)]
En utilisant les résultats de l'exercice 16, déduisez $\text{Var}(X)$ pour une loi Exponentielle($\lambda$).
\end{exercicebox}

\begin{exercicebox}[Exercice 18 : MGF (Loi Normale)]
La MGF de $X \sim \mathcal{N}(\mu, \sigma^2)$ est $M_X(t) = \exp(\mu t + \frac{1}{2}\sigma^2 t^2)$.
Calculez $M_X'(t)$ et vérifiez que $M_X'(0) = \mu$.
\end{exercicebox}

\begin{exercicebox}[Exercice 19 : MGF (Loi Normale)]
En utilisant la MGF de l'exercice 18, calculez $M_X''(t)$ et montrez que $E[X^2] = \mu^2 + \sigma^2$.
\end{exercicebox}

\begin{exercicebox}[Exercice 20 : MGF (Loi Normale)]
En utilisant les résultats des exercices 18 et 19, retrouvez la formule de la variance $\text{Var}(X)$.
\end{exercicebox}

% --- Section 5 : MGF - Sommes de Variables Indépendantes ---

\begin{exercicebox}[Exercice 21 : Propriété des Sommes]
Soient $X$ et $Y$ deux variables aléatoires \textbf{indépendantes}. Soit $S = X+Y$.
Comment la MGF de $S$, $M_S(t)$, est-elle liée à $M_X(t)$ et $M_Y(t)$ ?
\end{exercicebox}

\begin{exercicebox}[Exercice 22 : Application (Somme de Poissons)]
$X \sim \text{Poisson}(\lambda_1)$ a pour MGF $M_X(t) = e^{\lambda_1(e^t - 1)}$.
$Y \sim \text{Poisson}(\lambda_2)$ a pour MGF $M_Y(t) = e^{\lambda_2(e^t - 1)}$.
$X$ et $Y$ sont indépendantes.
\begin{enumerate}
    \item Trouvez la MGF de $S = X+Y$.
    \item En reconnaissant la forme de la MGF de $S$, quelle est la loi de $S$ ?
\end{enumerate}
\end{exercicebox}

\begin{exercicebox}[Exercice 23 : Application (Somme de Binomiales)]
La MGF de $X \sim \text{Bin}(n, p)$ est $M_X(t) = (p e^t + (1-p))^n$.
Soit $Y \sim \text{Bin}(m, p)$ (même $p$), indépendante de $X$.
Quelle est la loi de $S = X+Y$ ? Justifiez avec les MGF.
\end{exercicebox}

\begin{exercicebox}[Exercice 24 : Application (Transformation Linéaire)]
Soit $M_X(t)$ la MGF de $X$. Soit $Y = aX + b$.
Exprimez $M_Y(t)$ en fonction de $M_X(t)$.
(Indice : $M_Y(t) = E[e^{t(aX+b)}]$).
\end{exercicebox}

\begin{exercicebox}[Exercice 25 : Application (Moyenne d'Échantillon)]
Soient $X_1, \dots, X_n$ des v.a. i.i.d. (indépendantes et identiquement distribuées) avec la MGF $M_X(t)$.
Soit $\bar{X} = \frac{1}{n} \sum X_i$ la moyenne d'échantillon.
Exprimez la MGF de $\bar{X}$ en fonction de $M_X(t)$.
(Indice : utilisez les résultats des exercices 21 et 24).
\end{exercicebox}


\subsection{Corrections des Exercices}

\begin{correctionbox}[Correction Exercice 1 : Concepts (Moments)]
1.  Le 1er moment non centré $E[X^1]$ est l'espérance $\mu$.
2.  Le 2ème moment centré $E[(X-\mu)^2]$ est la variance $\sigma^2$.
\end{correctionbox}

\begin{correctionbox}[Correction Exercice 2 : Interprétation (Skewness)]
Un skewness de +2.5 est fortement positif. Cela signifie que la distribution des salaires est \textbf{asymétrique à droite}.
Concrètement : la grande majorité des employés a un salaire regroupé (autour de la médiane), mais il existe une "longue queue" de quelques individus avec des salaires très élevés. Ces valeurs extrêmes "tirent" la moyenne vers la droite (Moyenne > Médiane).
\end{correctionbox}

\begin{correctionbox}[Correction Exercice 3 : Interprétation (Skewness)]
Si une distribution est parfaitement symétrique (comme la loi normale ou la loi uniforme), les écarts positifs et négatifs à la moyenne s'annulent parfaitement. Le skewness est \textbf{nul (0)}.
\end{correctionbox}

\begin{correctionbox}[Correction Exercice 4 : Interprétation (Kurtosis)]
1.  Un excess kurtosis > 0 est dit \textbf{leptokurtique}.
2.  Cela signifie que la distribution a des "queues plus épaisses" que la loi normale. La probabilité d'événements extrêmes (très grands gains ou très grandes pertes, comme un "krach") est \textbf{plus élevée} que ce qu'un modèle normal ne prédirait.
\end{correctionbox}

\begin{correctionbox}[Correction Exercice 5 : Interprétation (Kurtosis)]
1.  Un excess kurtosis < 0 est dit \textbf{platykurtique}.
2.  Visuellement, par rapport à une loi normale, la distribution est plus "aplatie" ou "carrée". Elle a un pic central moins prononcé et des queues plus "légères" (les événements extrêmes sont moins probables). La distribution uniforme est un exemple classique.
\end{correctionbox}

\begin{correctionbox}[Correction Exercice 6 : Comparaison (Skewness)]
La distribution exponentielle (ex: temps d'attente) a un pic à 0 et une longue queue vers les grandes valeurs. Elle est \textbf{asymétrique à droite}, et son skewness est \textbf{positif} (Skew = 2).
\end{correctionbox}

\begin{correctionbox}[Correction Exercice 7 : Comparaison (Kurtosis)]
La distribution uniforme est "plate" et n'a pas de pic central ni de queues épaisses. Elle est \textbf{platykurtique}. Son "excess kurtosis" est \textbf{négatif} (Kurtosis Excessif = -1.2).
\end{correctionbox}

\begin{correctionbox}[Correction Exercice 8 : Propriété de Symétrie]
Soit $Y = X - \mu$. La symétrie implique que $Y$ et $-Y$ (qui est $\mu - X$) ont la même distribution.
$E[(X-\mu)^3] = E[Y^3]$.
Puisque $Y$ et $-Y$ ont la même distribution, ils ont les mêmes moments : $E[Y^k] = E[(-Y)^k]$ pour tout $k$.
Pour $k=3$ :
$E[Y^3] = E[(-Y)^3] = E[(-1)^3 Y^3] = E[-Y^3] = -E[Y^3]$.
La seule valeur qui est égale à son opposé est 0.
$E[Y^3] = -E[Y^3] \implies 2 E[Y^3] = 0 \implies E[Y^3] = 0$.
Donc, $E[(X-\mu)^3] = 0$.
\end{correctionbox}

\begin{correctionbox}[Correction Exercice 9 : Vérification de Symétrie (PDF)]
On teste la condition $f(x) = f(2\mu - x)$. Le candidat pour $\mu$ est 3.
$f(2\mu - x) = f(2(3) - x) = f(6 - x)$.
$f(6 - x) = \frac{1}{2}e^{-|(6-x)-3|} = \frac{1}{2}e^{-|3-x|}$.
Puisque $|3-x| = |-(x-3)| = |x-3|$, on a $f(6-x) = \frac{1}{2}e^{-|x-3|} = f(x)$.
Oui, la distribution est symétrique autour de $\mu=3$.
\end{correctionbox}

\begin{correctionbox}[Correction Exercice 10 : Symétrie et Moments]
Non. Le Skewness nul est une condition nécessaire pour la symétrie, but pas suffisante.
On peut construire des distributions (assez exotiques) qui sont non symétriques, mais où les asymétries se compensent d'une manière qui annule le 3ème moment, résultant en un skewness de 0. Cependant, dans la plupart des cas pratiques, Skewness = 0 est un très bon indicateur de symétrie.
\end{correctionbox}

\begin{correctionbox}[Correction Exercice 11 : Définitions (Échantillon vs Pop.)]
$\sigma^2$ (variance de population) est un \textbf{paramètre} théorique. C'est la "vraie" variance de l'ensemble de la population (souvent inconnue).
$s^2$ (variance d'échantillon) est une \textbf{statistique}. C'est une \textit{estimation} de $\sigma^2$ calculée à partir d'un sous-ensemble de données (l'échantillon).
\end{correctionbox}

\begin{correctionbox}[Correction Exercice 12 : Correction de Bessel ($n-1$)]
Si $n=1$ (un seul échantillon $X_1$), la moyenne d'échantillon est $\bar{X} = X_1$.
La somme des carrés des écarts est $\sum (X_i - \bar{X})^2 = (X_1 - X_1)^2 = 0$.
Si on divisait par $n=1$, on obtiendrait $s^2 = 0/1 = 0$. Cela estimerait faussement que la population n'a pas de variance, ce qui est absurde.
Diviser par $n-1$ (donc $1-1=0$) donne $0/0$, une forme indéfinie, ce qui nous dit correctement qu'on ne peut pas estimer une dispersion à partir d'un seul point.
\end{correctionbox}

\begin{correctionbox}[Correction Exercice 13 : Calcul (Moments d'Échantillon)]
Échantillon : $\{ 2, 3, 10 \}$. $n=3$.
1.  $\bar{X} = \frac{1}{3} (2 + 3 + 10) = \frac{15}{3} = 5$.
2.  $s^2 = \frac{1}{n-1} \sum (X_i - \bar{X})^2 = \frac{1}{2} \left[ (2-5)^2 + (3-5)^2 + (10-5)^2 \right]$
    $s^2 = \frac{1}{2} \left[ (-3)^2 + (-2)^2 + (5)^2 \right]$
    $s^2 = \frac{1}{2} [ 9 + 4 + 25 ] = \frac{38}{2} = 19$.
\end{correctionbox}

\begin{correctionbox}[Correction Exercice 14 : MGF (Propriété de base)]
$M_X(t) = E[e^{tX}]$.
En $t=0$, $M_X(0) = E[e^{0 \cdot X}] = E[e^0] = E[1] = 1$.
Toute MGF valide doit valoir 1 en $t=0$.
\end{correctionbox}

\begin{correctionbox}[Correction Exercice 15 : MGF (Série de Taylor)]
On identifie les coefficients de la série $M_X(t) = 1 + \frac{E[X]}{1!}t + \frac{E[X^2]}{2!}t^2 + \dots$
1.  Le coefficient de $t$ est $E[X]$. On lit $5t$. Donc $E[X] = 5$.
2.  Le coefficient de $t^2$ est $E[X^2]/2!$. On lit $14t^2$.
    $E[X^2] / 2 = 14 \implies E[X^2] = 28$.
3.  $\text{Var}(X) = E[X^2] - (E[X])^2 = 28 - (5)^2 = 28 - 25 = 3$.
\end{correctionbox}

\begin{correctionbox}[Correction Exercice 16 : MGF (Calcul de moments)]
$M_X(t) = \lambda (\lambda - t)^{-1}$.
1.  $M_X'(t) = \lambda \cdot (-1) (\lambda - t)^{-2} \cdot (-1) = \lambda (\lambda - t)^{-2}$.
    $E[X] = M_X'(0) = \lambda (\lambda - 0)^{-2} = \lambda (\lambda^{-2}) = 1/\lambda$.
2.  $M_X''(t) = \lambda \cdot (-2) (\lambda - t)^{-3} \cdot (-1) = 2\lambda (\lambda - t)^{-3}$.
    $E[X^2] = M_X''(0) = 2\lambda (\lambda - 0)^{-3} = 2\lambda (\lambda^{-3}) = 2/\lambda^2$.
\end{correctionbox}

\begin{correctionbox}[Correction Exercice 17 : MGF (Variance de l'Exponentielle)]
$\text{Var}(X) = E[X^2] - (E[X])^2$
$\text{Var}(X) = (2/\lambda^2) - (1/\lambda)^2 = 2/\lambda^2 - 1/\lambda^2 = 1/\lambda^2$.
\end{correctionbox}

\begin{correctionbox}[Correction Exercice 18 : MGF (Loi Normale)]
$M_X(t) = \exp(\mu t + \frac{1}{2}\sigma^2 t^2)$.
On utilise la règle de la chaîne : $(\exp(u))' = \exp(u) \cdot u'$.
$M_X'(t) = \exp(\mu t + \frac{1}{2}\sigma^2 t^2) \cdot \frac{d}{dt}(\mu t + \frac{1}{2}\sigma^2 t^2)$
$M_X'(t) = \exp(\mu t + \frac{1}{2}\sigma^2 t^2) \cdot (\mu + \sigma^2 t)$.
$E[X] = M_X'(0) = \exp(0) \cdot (\mu + 0) = 1 \cdot \mu = \mu$.
\end{correctionbox}

\begin{correctionbox}[Correction Exercice 19 : MGF (Loi Normale)]
On dérive $M_X'(t)$ (règle du produit $u \cdot v$) :
$u = \exp(\mu t + \frac{1}{2}\sigma^2 t^2) \implies u' = u \cdot (\mu + \sigma^2 t)$
$v = (\mu + \sigma^2 t) \implies v' = \sigma^2$
$M_X''(t) = u'v + uv'$
$M_X''(t) = [\exp(\dots)(\mu + \sigma^2 t)] \cdot (\mu + \sigma^2 t) + [\exp(\dots)] \cdot (\sigma^2)$
On évalue en $t=0$ :
$E[X^2] = M_X''(0) = [\exp(0)(\mu)] \cdot (\mu) + [\exp(0)] \cdot (\sigma^2)$
$E[X^2] = (1 \cdot \mu) \cdot \mu + 1 \cdot \sigma^2 = \mu^2 + \sigma^2$.
\end{correctionbox}

\begin{correctionbox}[Correction Exercice 20 : MGF (Loi Normale)]
$\text{Var}(X) = E[X^2] - (E[X])^2$
$\text{Var}(X) = (\mu^2 + \sigma^2) - (\mu)^2 = \sigma^2$.
La MGF confirme bien que le paramètre $\sigma^2$ est la variance.
\end{correctionbox}

\begin{correctionbox}[Correction Exercice 21 : Propriété des Sommes]
Si $X$ et $Y$ sont indépendantes, la MGF de la somme est le \textbf{produit} des MGF :
$M_{X+Y}(t) = M_X(t) \cdot M_Y(t)$.
\end{correctionbox}

\begin{correctionbox}[Correction Exercice 22 : Application (Somme de Poissons)]
1.  $M_S(t) = M_X(t) \cdot M_Y(t) = e^{\lambda_1(e^t - 1)} \cdot e^{\lambda_2(e^t - 1)}$
    $M_S(t) = e^{\lambda_1(e^t - 1) + \lambda_2(e^t - 1)} = e^{(\lambda_1 + \lambda_2)(e^t - 1)}$.
2.  On reconnaît la MGF d'une loi de Poisson. Le paramètre est ce qui multiplie $(e^t - 1)$.
    Donc $S$ suit une loi de Poisson de paramètre $(\lambda_1 + \lambda_2)$.
    $S \sim \text{Poisson}(\lambda_1 + \lambda_2)$.
\end{correctionbox}

\begin{correctionbox}[Correction Exercice 23 : Application (Somme de Binomiales)]
$M_S(t) = M_X(t) \cdot M_Y(t) = (p e^t + (1-p))^n \cdot (p e^t + (1-p))^m$
$M_S(t) = (p e^t + (1-p))^{n+m}$.
On reconnaît la MGF d'une loi Binomiale avec $n+m$ essais et une probabilité de succès $p$.
$S \sim \text{Bin}(n+m, p)$.
\end{correctionbox}

\begin{correctionbox}[Correction Exercice 24 : Application (Transformation Linéaire)]
$M_Y(t) = E[e^{tY}] = E[e^{t(aX + b)}] = E[e^{taX + tb}]$
$M_Y(t) = E[e^{taX} \cdot e^{tb}]$.
Puisque $e^{tb}$ est une constante :
$M_Y(t) = e^{tb} E[e^{(ta)X}]$.
Par définition, $E[e^{(ta)X}]$ est la MGF de $X$ évaluée au point $(ta)$.
$M_Y(t) = e^{tb} M_X(ta)$.
\end{correctionbox}

\begin{correctionbox}[Correction Exercice 25 : Application (Moyenne d'Échantillon)]
Soit $S = \sum X_i$. $\bar{X} = S / n = (1/n)S$.
1.  D'abord, la MGF de la somme $S$ (Exercice 21) :
    $M_S(t) = M_{X_1}(t) \cdot \dots \cdot M_{X_n}(t) = [M_X(t)]^n$ (car i.i.d.)
2.  Ensuite, on utilise la transformation linéaire $\bar{X} = aS + b$ avec $a=1/n$ et $b=0$ (Exercice 24) :
    $M_{\bar{X}}(t) = e^{b t} M_S(a t) = e^0 M_S(t/n)$
    $M_{\bar{X}}(t) = M_S(t/n)$.
3.  On combine les deux :
    $M_{\bar{X}}(t) = [M_X(t/n)]^n$.
\end{correctionbox}

\subsection{Exercices Python}

La loi log-normale est fondamentale en finance. Elle repose sur l'idée que si les \textbf{log-rendements} d'une action $X_i = \ln(P_i / P_{i-1})$ sont (approximativement) normaux, alors le prix futur $P_t$, qui est un \textbf{produit} de ces rendements ($P_t = P_0 \times e^{X_1} \times \dots \times e^{X_t}$), suivra une loi log-normale.

Nous allons estimer les paramètres $\mu$ et $\sigma^2$ de la loi normale sous-jacente à partir des log-rendements journaliers de Microsoft (MSFT) et Google (GOOG), puis utiliser la théorie log-normale pour modéliser les prix.

\begin{codecell}
!pip install yfinance
import yfinance as yf
import pandas as pd
import numpy as np
from scipy.stats import norm # Moteur pour les calculs de CDF/PDF
import matplotlib.pyplot as plt

# Definir les tickers et la periode
tickers = ["MSFT", "GOOG"]
start_date = "2020-01-01"
end_date = "2024-12-31"

# Telecharger les prix de cloture ajustes
data = yf.download(tickers, start=start_date, end=end_date)["Adj Close"]

# Calculer les LOG-RENDEMENTS journaliers
log_returns = np.log(data / data.shift(1)).dropna()

# Renommer les colonnes
log_returns.columns = ["MSFT_LogReturn", "GOOG_LogReturn"]

# 'log_returns' est notre DataFrame.
# X_msft = log_returns["MSFT_LogReturn"]
# X_goog = log_returns["GOOG_LogReturn"]
\end{codecell}

\begin{exercicebox}[Exercice 1 : Estimer les Paramètres $\mu$ et $\sigma^2$]
Soit $P_t$ le prix de MSFT. Le modèle suppose que $X = \ln(P_t/P_{t-1}) \sim \mathcal{N}(\mu, \sigma^2)$. Les paramètres $\mu$ et $\sigma^2$ sont les paramètres "log-normaux".

\textbf{Votre tâche :}
\begin{enumerate}
    \item Estimer $\mu$ (l'espérance du log-rendement journalier) pour MSFT.
    \item Estimer $\sigma^2$ (la variance du log-rendement journalier) pour MSFT.
    \item Estimer $\sigma$ (l'écart-type du log-rendement journalier) pour MSFT.
\end{enumerate}
\end{exercicebox}

\begin{exercicebox}[Exercice 2 : Test de Normalité (Graphique)]
La théorie log-normale repose sur la normalité des log-rendements $X$. Vérifions-le visuellement.

\textbf{Votre tâche :}
\begin{enumerate}
    \item Utiliser $\mu$ et $\sigma$ (pour MSFT) de l'Exercice 1.
    \item \textbf{(Plot)} Tracer l'histogramme des log-rendements \textbf{empiriques} de MSFT (Indice : \texttt{plt.hist(..., density=True, bins=50)}).
    \item \textbf{(Plot)} Superposer la PDF \textbf{théorique} de la loi normale $\mathcal{N}(\mu, \sigma^2)$ sur cet histogramme.
    \item (Indice : Créez un \texttt{np.linspace}, calculez la PDF avec \texttt{norm.pdf(x, loc=mu, scale=sigma)}, puis \texttt{plt.plot()}).
    \item (Conclusion) La cloche théorique s'ajuste-t-elle bien aux données réelles ?
\end{enumerate}
\end{exercicebox}

\begin{exercicebox}[Exercice 3 : Asymétrie (Prix vs Log-Rendements)]
La théorie dit que les log-rendements $X$ sont symétriques (Normaux), mais que les prix $P_t$ sont asymétriques à droite (Log-Normaux).

\textbf{Votre tâche :}
\begin{enumerate}
    \item Calculer la moyenne et la médiane de la série des \textbf{log-rendements} de MSFT.
    \item Calculer la moyenne et la médiane de la série des \textbf{prix} de MSFT (la colonne \texttt{data['MSFT']}).
    \item Comparer les deux paires. Les log-rendements sont-ils symétriques (moyenne $\approx$ médiane) ? Les prix sont-ils asymétriques (moyenne $>$ médiane) ?
    \item \textbf{(Plot)} Créer deux histogrammes côte à côte (\texttt{plt.subplot}) pour visualiser la distribution des log-rendements et celle des prix.
\end{enumerate}
\end{exercicebox}

\begin{exercicebox}[Exercice 4 : Espérance vs Médiane (Théorique)]
Soit $Y = P_t/P_{t-1} = e^X$ la variable "ratio de prix journalier". $Y \sim \text{Log-}\mathcal{N}(\mu, \sigma^2)$.
Théorie : $\text{Med}(Y) = e^{\mu}$ et $E[Y] = e^{\mu + \sigma^2/2}$.

\textbf{Votre tâche :}
\begin{enumerate}
    \item Utiliser $\mu$ et $\sigma^2$ (pour MSFT) de l'Exercice 1.
    \item Calculer la médiane \textbf{théorique} $\text{Med}(Y)$.
    \item Calculer l'espérance \textbf{théorique} $E[Y]$.
    \item Vérifier que $E[Y] > \text{Med}(Y)$, confirmant l'asymétrie.
\end{enumerate}
\end{exercicebox}

\begin{exercicebox}[Exercice 5 : Espérance Théorique vs Empirique]
Vérifions le calcul de $E[Y]$ de l'exercice 4 de manière empirique.

\textbf{Votre tâche :}
\begin{enumerate}
    \item Créer la série $Y$ (ratio de prix journalier) : $Y = \exp(X_{\text{msft}})$.
    \item Calculer l'espérance \textbf{empirique} de $Y$ (la moyenne de cette série $Y$).
    \item Comparer cette valeur empirique à l'espérance \textbf{théorique} $e^{\mu + \sigma^2/2}$ calculée à l'exercice 4.
\end{enumerate}
\end{exercicebox}

\begin{exercicebox}[Exercice 6 : Variance Théorique vs Empirique]
Théorie : $\text{Var}(Y) = (e^{\sigma^2} - 1) \cdot e^{2\mu + \sigma^2}$.

\textbf{Votre tâche :}
\begin{enumerate}
    \item Utiliser $\mu$ et $\sigma^2$ (pour MSFT) de l'Exercice 1.
    \item Calculer la variance \textbf{théorique} $\text{Var}(Y)$ en utilisant la formule ci-dessus.
    \item Calculer la variance \textbf{empirique} de la série $Y$ (créée à l'Ex 5).
    \item Comparer les deux résultats.
\end{enumerate}
\end{exercicebox}

\begin{exercicebox}[Exercice 7 : Modélisation du Prix Futur (Paramètres)]
Modélisons le prix de GOOG dans $t=20$ jours ouvrés (environ 1 mois).
Le prix $P_{20}$ est log-normal si l'on suppose $P_{20} = P_0 \cdot e^{X_{20}}$, où $P_0$ est le prix actuel.
Le log-rendement total $X_{20} = \ln(P_{20}/P_0)$ suit $X_{20} \sim \mathcal{N}(t\mu, t\sigma^2)$.

\textbf{Votre tâche :}
\begin{enumerate}
    \item Estimer $\mu_G$ et $\sigma_G^2$ (journaliers) pour GOOG (similaire à l'Ex 1).
    \item Définir $t=20$.
    \item Calculer $\mu_{20} = t\mu_G$ (l'espérance du log-rendement sur 20 jours).
    \item Calculer $\sigma_{20}^2 = t\sigma_G^2$ (la variance du log-rendement sur 20 jours).
\end{enumerate}
\end{exercicebox}

\begin{exercicebox}[Exercice 8 : Calcul de Probabilité (Prix Futur)]
En utilisant les paramètres $\mu_{20}$ et $\sigma_{20} = \sqrt{\sigma_{20}^2}$ de l'exercice 7 pour GOOG :

\textbf{Votre tâche :}
\begin{enumerate}
    \item Calculer la probabilité que GOOG ait un rendement positif sur 20 jours.
    \item On cherche $P(P_{20} > P_0) \implies P(P_{20}/P_0 > 1) \implies P(\ln(P_{20}/P_0) > \ln(1))$.
    \item Calculer $P(X_{20} > 0)$.
    \item (Indice : Standardiser 0 avec $\mu_{20}$ et $\sigma_{20}$, puis utiliser $1 - \Phi(z)$).
    \item \textbf{(Plot)} Tracer la PDF de $X_{20} \sim \mathcal{N}(\mu_{20}, \sigma_{20}^2)$ et hachurer la zone $x > 0$.
\end{enumerate}
\end{exercicebox}

\begin{exercicebox}[Exercice 9 : Calcul de Probabilité (Perte > 5\%)]
En utilisant les paramètres $\mu_{20}$ et $\sigma_{20}$ de l'exercice 7 pour GOOG :

\textbf{Votre tâche :}
\begin{enumerate}
    \item Calculer la probabilité que GOOG perde plus de 5\% sur 20 jours.
    \item On cherche $P(P_{20} < 0.95 \times P_0) \implies P(P_{20}/P_0 < 0.95)$.
    \item Calculer $P(X_{20} < \ln(0.95))$.
    \item (Indice : Standardiser $\ln(0.95)$ avec $\mu_{20}$ et $\sigma_{20}$, puis utiliser $\Phi(z)$).
    \item \textbf{(Plot)} Tracer la PDF de $X_{20}$ et hachurer la zone $x < \ln(0.95)$.
\end{enumerate}
\end{exercicebox}

\begin{exercicebox}[Exercice 10 : Problème Inverse (Intervalle de Confiance)]
Trouvons l'intervalle de 95\% pour le prix de GOOG dans 20 jours.
Nous cherchons les bornes $y_1, y_2$ telles que $P(y_1 \le P_{20} \le y_2) = 0.95$.
On suppose un intervalle centré sur la loi normale sous-jacente (entre $z=-1.96$ et $z=+1.96$).

\textbf{Votre tâche :}
\begin{enumerate}
    \item Trouver $z_{inf} = -1.96$ et $z_{sup} = +1.96$.
    \item "Dé-standardiser" ces Z-scores pour trouver les log-rendements $x_1$ et $x_2$ :
        $x = \mu_t + z \sigma_t$ (en utilisant $\mu_{20}$ et $\sigma_{20}$ de l'Ex 7).
    \item Convertir ces log-rendements en ratios de prix $y = e^x$.
    \item (Conclusion) L'intervalle de 95\% pour le ratio de prix est $[y_1, y_2]$.
\end{enumerate}
\end{exercicebox}

\begin{exercicebox}[Exercice 11 : Calcul de la Médiane vs Espérance (Prix Futur)]
Pour le prix de GOOG dans 20 jours, $P_{20} = P_0 \cdot Y_{20}$, où $Y_{20} \sim \text{Log-}\mathcal{N}(\mu_{20}, \sigma_{20}^2)$.

\textbf{Votre tâche :}
\begin{enumerate}
    \item Calculer le ratio de prix \textbf{médian} attendu : $\text{Med}(Y_{20}) = e^{\mu_{20}}$.
    \item Calculer le ratio de prix \textbf{moyen} (espérance) attendu : $E[Y_{20}] = e^{\mu_{20} + \sigma_{20}^2 / 2}$.
    \item (Conclusion) Lequel est le plus élevé ? Pourquoi est-ce important pour un investisseur ?
\end{enumerate}
\end{exercicebox}