


\subsection{Exercices}

%  Concepts de Base et Règle du Produit 

\begin{exercicebox}[Exercice 1 : Dés et Probabilité Conditionnelle Simple]
On lance deux dés équilibrés à 6 faces.
\begin{enumerate}
    \item Quelle est la probabilité que la somme des dés soit 8 ?
    \item Sachant que le premier dé a donné un 3, quelle est la probabilité que la somme soit 8 ?
    \item Sachant que la somme est 8, quelle est la probabilité que le premier dé ait donné un 3 ?
\end{enumerate}
\end{exercicebox}

\begin{exercicebox}[Exercice 2 : Tirage de Cartes (Sans Remise)]
On tire deux cartes successivement et sans remise d'un jeu standard de 52 cartes.
\begin{enumerate}
    \item Quelle est la probabilité que la deuxième carte soit un Roi, sachant que la première était un Roi ?
    \item Quelle est la probabilité de tirer deux Rois ?
\end{enumerate}
\end{exercicebox}

\begin{exercicebox}[Exercice 3 : Urne (Règle du Produit)]
Une urne contient 7 boules rouges et 3 boules bleues. On tire deux boules successivement et sans remise.
\begin{enumerate}
    \item Quelle est la probabilité que la première boule soit rouge ?
    \item Quelle est la probabilité que la deuxième boule soit bleue, sachant que la première était rouge ?
    \item Quelle est la probabilité de tirer une boule rouge puis une boule bleue ?
\end{enumerate}
\end{exercicebox}

\begin{exercicebox}[Exercice 4 : Famille (Condition Simple)]
Une famille a deux enfants. On suppose que la probabilité d'avoir un garçon (G) ou une fille (F) est la même (0.5) et que les naissances sont indépendantes.
\begin{enumerate}
    \item Quel est l'univers $S$ des possibilités ?
    \item Sachant que l'aîné est un garçon, quelle est la probabilité que la famille ait deux garçons ?
\end{enumerate}
\end{exercicebox}

\begin{exercicebox}[Exercice 5 : Famille (Condition "Au Moins")]
En utilisant le même scénario que l'exercice 4 (famille de deux enfants) :
Sachant qu'il y a \textit{au moins un} garçon dans la famille, quelle est la probabilité que la famille ait deux garçons ?
\end{exercicebox}

%  Indépendance 

\begin{exercicebox}[Exercice 6 : Indépendance (Dés)]
On lance deux dés équilibrés.
Soit $A$ l'événement "le premier dé donne 3" et $B$ l'événement "la somme des deux dés est 7".
Les événements $A$ et $B$ sont-ils indépendants ? Justifiez par le calcul.
\end{exercicebox}

\begin{exercicebox}[Exercice 7 : Indépendance (Cartes)]
On tire une carte d'un jeu de 52 cartes.
Soit $A$ l'événement "la carte est un Roi" et $B$ l'événement "la carte est un Cœur".
Les événements $A$ et $B$ sont-ils indépendants ?
\end{exercicebox}

\begin{exercicebox}[Exercice 8 : Indépendance vs Exclusion Mutuelle]
Soient $A$ et $B$ deux événements avec $P(A)=0.5$ et $P(B)=0.3$.
\begin{enumerate}
    \item Si $A$ et $B$ sont mutuellement exclusifs (disjoints), sont-ils indépendants ?
    \item Si $A$ et $B$ sont indépendants, quelle est $P(A \cup B)$ ?
\end{enumerate}
\end{exercicebox}

%  Formule des Probabilités Totales (LTP) 

\begin{exercicebox}[Exercice 9 : LTP (Deux Urnes)]
L'urne U1 contient 2 boules noires et 3 boules blanches. L'urne U2 contient 4 boules noires et 1 boule blanche.
On choisit une urne au hasard (chaque urne a 50\% de chance d'être choisie), puis on tire une boule de cette urne.
Quelle est la probabilité de tirer une boule blanche ?
\end{exercicebox}

\begin{exercicebox}[Exercice 10 : LTP (Usine)]
Une usine utilise deux machines, M1 et M2, pour produire des pièces. M1 produit 40\% des pièces et M2 produit 60\%. 5\% des pièces de M1 sont défectueuses, et 2\% des pièces de M2 sont défectueuses.
Si l'on choisit une pièce au hasard dans la production totale, quelle est la probabilité qu'elle soit défectueuse ?
\end{exercicebox}

\begin{exercicebox}[Exercice 11 : LTP (Pièce de Monnaie Inconnue)]
On a deux pièces. La pièce A est équilibrée ($P(\text{Pile})=0.5$). La pièce B est truquée ($P(\text{Pile})=0.8$).
On choisit une pièce au hasard et on la lance. Quelle est la probabilité d'obtenir Pile ?
\end{exercicebox}

%  Règle de Bayes 

\begin{exercicebox}[Exercice 12 : Bayes (Test Médical)]
Une maladie touche 1 personne sur 1000 ($P(M)=0.001$). Un test de dépistage donne un résultat positif chez 98\% des personnes malades ($P(T|M)=0.98$). Il donne aussi un résultat positif (un "faux positif") chez 3\% des personnes non malades ($P(T|\neg M)=0.03$).
Une personne reçoit un test positif. Quelle est la probabilité qu'elle soit réellement malade ?
\end{exercicebox}

\begin{exercicebox}[Exercice 13 : Bayes (Inversion d'Urnes)]
Reprenons le scénario de l'exercice 9 (U1 avec 2N/3B, U2 avec 4N/1B).
On a tiré une boule et on constate qu'elle est blanche. Quelle est la probabilité qu'elle provienne de l'urne U1 ?
\end{exercicebox}

\begin{exercicebox}[Exercice 14 : Bayes (Spam)]
Dans une boîte de réception, 60\% des emails sont des spams. 70\% des spams contiennent le mot "gratuit". Seuls 10\% des emails légitimes contiennent le mot "gratuit".
Vous recevez un email qui contient le mot "gratuit". Quelle est la probabilité que ce soit un spam ?
\end{exercicebox}

\begin{exercicebox}[Exercice 15 : Bayes (Usine Inversée)]
Reprenons le scénario de l'exercice 10 (M1: 40\% prod, 5\% défaut; M2: 60% prod, 2% défaut).
On trouve une pièce défectueuse. Quelle est la probabilité qu'elle ait été produite par la machine M1 ?
\end{exercicebox}

%  Règle de la Chaîne et Problèmes Combinés 

\begin{exercicebox}[Exercice 16 : Règle de la Chaîne (3 Cartes)]
On tire 3 cartes successivement et sans remise d'un jeu de 52 cartes.
Quelle est la probabilité de tirer 3 Piques ?
\end{exercicebox}

\begin{exercicebox}[Exercice 17 : Problème de Monty Hall (Calcul)]
En utilisant la formalisation du problème de Monty Hall (vous choisissez la Porte 1, la voiture est en $V \in \{1, 2, 3\}$, l'animateur ouvre $H \in \{2, 3\}$) :
Calculez $P(V=1 | H=3)$ (la probabilité que la voiture soit derrière votre porte, sachant que l'animateur a ouvert la 3). Supposez que $P(V=i)=1/3$ pour $i=1,2,3$.
\end{exercicebox}

\begin{exercicebox}[Exercice 18 : Bayes avec Mise à Jour (Pièce Truquée)]
Reprenons l'exercice 11 (Pièce A équilibrée, Pièce B truquée $P(\text{Pile})=0.8$).
On choisit une pièce au hasard. On la lance deux fois et on obtient Pile, puis Pile (PP).
Quelle est la probabilité que l'on ait choisi la pièce truquée (Pièce B) ?
\end{exercicebox}

\begin{exercicebox}[Exercice 19 : Indépendance Conditionnelle (Dés)]
On lance deux dés, $D_1$ et $D_2$. Soit $S = D_1 + D_2$ leur somme.
Soit $A$ l'événement "$D_1 = 1$", $B$ l'événement "$D_2 = 1$".
$A$ et $B$ sont indépendants. Sont-ils indépendants conditionnellement à l'événement $C = \{S = 2\}$ ?
\end{exercicebox}

\begin{exercicebox}[Exercice 20 : Jeu Séquentiel]
Alice et Bob jouent à un jeu. Ils lancent un dé à tour de rôle, en commençant par Alice. Le premier qui obtient un 6 gagne.
Quelle est la probabilité qu'Alice gagne ?
\end{exercicebox}

\subsection{Corrections des Exercices}

%  Corrections : Concepts de Base et Règle du Produit 

\begin{correctionbox}[Correction Exercice 1 : Dés et Probabilité Conditionnelle Simple]
L'univers $S$ a $|S| = 6 \times 6 = 36$ issues.

1.  Soit $A$ l'événement "la somme est 8". $A = \{(2,6), (3,5), (4,4), (5,3), (6,2)\}$.
    $|A|=5$, donc $P(A) = 5/36$.

2.  Soit $B$ l'événement "le premier dé donne 3". $B = \{(3,1), (3,2), (3,3), (3,4), (3,5), (3,6)\}$.
    On cherche $P(A|B)$. Sachant $B$, l'univers est réduit à ces 6 issues. Parmi celles-ci, seule l'issue $(3,5)$ donne une somme de 8.
    Donc, $P(A|B) = 1/6$.
    *Par formule :* $A \cap B = \{(3,5)\}$, $P(A \cap B) = 1/36$. $P(B) = 6/36 = 1/6$.
    $P(A|B) = \frac{P(A \cap B)}{P(B)} = \frac{1/36}{6/36} = 1/6$.

3.  On cherche $P(B|A)$. Sachant $A$, l'univers est réduit aux 5 issues de $A$. Parmi celles-ci, seule l'issue $(3,5)$ a 3 sur le premier dé.
    Donc, $P(B|A) = 1/5$.
    *Par formule :* $P(B|A) = \frac{P(A \cap B)}{P(A)} = \frac{1/36}{5/36} = 1/5$.
\end{correctionbox}

\begin{correctionbox}[Correction Exercice 2 : Tirage de Cartes (Sans Remise)]
Soit $K_1$ l'événement "Roi au 1er tirage" et $K_2$ "Roi au 2e tirage".

1.  On cherche $P(K_2|K_1)$. Si $K_1$ s'est produit, il reste 51 cartes dans le jeu, dont $4-1=3$ Rois.
    $P(K_2|K_1) = 3/51 = 1/17$.

2.  On cherche $P(K_1 \cap K_2)$. On utilise la règle du produit :
    $P(K_1 \cap K_2) = P(K_1) \times P(K_2|K_1)$
    $P(K_1) = 4/52 = 1/13$.
    $P(K_1 \cap K_2) = (4/52) \times (3/51) = (1/13) \times (1/17) = 1/221$.
\end{correctionbox}

\begin{correctionbox}[Correction Exercice 3 : Urne (Règle du Produit)]
Urne avec 7 Rouges (R) et 3 Bleues (B). Total = 10.
Soit $R_1$ "Rouge au 1er tirage" et $B_2$ "Bleue au 2e tirage".

1.  $P(R_1) = 7/10$.
2.  On cherche $P(B_2|R_1)$. Si $R_1$ s'est produit, il reste 9 boules (6R, 3B).
    $P(B_2|R_1) = 3/9 = 1/3$.
3.  On cherche $P(R_1 \cap B_2)$.
    $P(R_1 \cap B_2) = P(R_1) \times P(B_2|R_1) = (7/10) \times (1/3) = 7/30$.
\end{correctionbox}

\begin{correctionbox}[Correction Exercice 4 : Famille (Condition Simple)]
1.  L'univers est $S = \{GG, GF, FG, FF\}$, où le premier enfant est l'aîné. $|S|=4$, chaque issue a une probabilité de 1/4.
2.  Soit $A$ l'événement "l'aîné est un garçon" : $A = \{GG, GF\}$. $P(A) = 2/4 = 1/2$.
    Soit $B$ l'événement "la famille a deux garçons" : $B = \{GG\}$. $P(B) = 1/4$.
    On cherche $P(B|A)$. L'événement $A \cap B = \{GG\}$. $P(A \cap B) = 1/4$.
    $P(B|A) = \frac{P(A \cap B)}{P(A)} = \frac{1/4}{1/2} = 1/2$.
\end{correctionbox}

\begin{correctionbox}[Correction Exercice 5 : Famille (Condition "Au Moins")]
Soit $B$ l'événement "la famille a deux garçons" : $B = \{GG\}$.
Soit $C$ l'événement "il y a au moins un garçon" : $C = \{GG, GF, FG\}$. $P(C) = 3/4$.
On cherche $P(B|C)$.
L'événement $B \cap C = \{GG\}$. $P(B \cap C) = 1/4$.
$P(B|C) = \frac{P(B \cap C)}{P(C)} = \frac{1/4}{3/4} = 1/3$.
*Intuition :* L'univers de $C$ est $\{GG, GF, FG\}$. Parmi ces 3 issues équiprobables, une seule est $GG$.
\end{correctionbox}

%  Corrections : Indépendance 

\begin{correctionbox}[Correction Exercice 6 : Indépendance (Dés)]
$A$ = "premier dé = 3". $P(A) = 6/36 = 1/6$.
$B$ = "somme = 7". $B = \{(1,6), (2,5), (3,4), (4,3), (5,2), (6,1)\}$. $P(B) = 6/36 = 1/6$.
$A \cap B$ = "premier dé = 3 ET somme = 7" = $\%(3,4)\}$. $P(A \cap B) = 1/36$.
On teste si $P(A \cap B) = P(A)P(B)$.
$P(A)P(B) = (1/6) \times (1/6) = 1/36$.
Puisque $P(A \cap B) = P(A)P(B)$, les événements $A$ et $B$ sont indépendants.
\end{correctionbox}

\begin{correctionbox}[Correction Exercice 7 : Indépendance (Cartes)]
$A$ = "Roi". $P(A) = 4/52 = 1/13$.
$B$ = "Cœur". $P(B) = 13/52 = 1/4$.
$A \cap B$ = "Roi de Cœur". $P(A \cap B) = 1/52$.
On teste si $P(A \cap B) = P(A)P(B)$.
$P(A)P(B) = (1/13) \times (1/4) = 1/52$.
Puisque $P(A \cap B) = P(A)P(B)$, les événements $A$ et $B$ sont indépendants.
\end{correctionbox}

\begin{correctionbox}[Correction Exercice 8 : Indépendance vs Exclusion Mutuelle]
$P(A)=0.5, P(B)=0.3$.

1.  Si $A$ et $B$ sont mutuellement exclusifs, $A \cap B = \emptyset$, donc $P(A \cap B) = 0$.
    Pour qu'ils soient indépendants, il faudrait $P(A \cap B) = P(A)P(B) = 0.5 \times 0.3 = 0.15$.
    Puisque $0 \neq 0.15$, ils ne sont pas indépendants. (Deux événements non impossibles ne peuvent pas être à la fois mutuellement exclusifs et indépendants).

2.  Si $A$ et $B$ sont indépendants, $P(A \cap B) = P(A)P(B) = 0.15$.
    $P(A \cup B) = P(A) + P(B) - P(A \cap B) = 0.5 + 0.3 - 0.15 = 0.65$.
\end{correctionbox}

%  Corrections : Formule des Probabilités Totales (LTP) 

\begin{correctionbox}[Correction Exercice 9 : LTP (Deux Urnes)]
Soit $U_1$ et $U_2$ les événements "choisir l'urne 1" et "choisir l'urne 2". $P(U_1)=0.5, P(U_2)=0.5$.
Soit $W$ l'événement "tirer une boule blanche".
On a $P(W|U_1) = 3 / (2+3) = 3/5 = 0.6$.
On a $P(W|U_2) = 1 / (4+1) = 1/5 = 0.2$.
Par la formule des probabilités totales :
$P(W) = P(W|U_1)P(U_1) + P(W|U_2)P(U_2)$
$P(W) = (0.6 \times 0.5) + (0.2 \times 0.5) = 0.3 + 0.1 = 0.4$.
\end{correctionbox}

\begin{correctionbox}[Correction Exercice 10 : LTP (Usine)]
Soit $M_1$ et $M_2$ les machines. $P(M_1)=0.4, P(M_2)=0.6$.
Soit $D$ l'événement "la pièce est défectueuse".
On a $P(D|M_1) = 0.05$ et $P(D|M_2) = 0.02$.
Par la formule des probabilités totales :
$P(D) = P(D|M_1)P(M_1) + P(D|M_2)P(M_2)$
$P(D) = (0.05 \times 0.4) + (0.02 \times 0.6) = 0.020 + 0.012 = 0.032$.
La probabilité est de 3.2\%.
\end{correctionbox}

\begin{correctionbox}[Correction Exercice 11 : LTP (Pièce de Monnaie Inconnue)]
Soit $A$ "choisir pièce A" et $B$ "choisir pièce B". $P(A)=0.5, P(B)=0.5$.
Soit $H$ l'événement "obtenir Pile".
On a $P(H|A) = 0.5$ et $P(H|B) = 0.8$.
Par la formule des probabilités totales :
$P(H) = P(H|A)P(A) + P(H|B)P(B)$
$P(H) = (0.5 \times 0.5) + (0.8 \times 0.5) = 0.25 + 0.40 = 0.65$.
\end{correctionbox}

%  Corrections : Règle de Bayes 

\begin{correctionbox}[Correction Exercice 12 : Bayes (Test Médical)]
Soit $M$ "Malade" et $T$ "Test Positif".
$P(M) = 0.001$, donc $P(\neg M) = 0.999$.
$P(T|M) = 0.98$.
$P(T|\neg M) = 0.03$.
On cherche $P(M|T)$. Par la règle de Bayes : $P(M|T) = \frac{P(T|M)P(M)}{P(T)}$.

1.  Calculer $P(T)$ (dénominateur) avec la LTP :
    $P(T) = P(T|M)P(M) + P(T|\neg M)P(\neg M)$
    $P(T) = (0.98 \times 0.001) + (0.03 \times 0.999) = 0.00098 + 0.02997 = 0.03095$.

2.  Appliquer la règle de Bayes :
    $P(M|T) = \frac{0.00098}{0.03095} \approx 0.03166$.
    Il n'y a que 3.17\% de chance que la personne soit malade, même avec un test positif.
\end{correctionbox}

\begin{correctionbox}[Correction Exercice 13 : Bayes (Inversion d'Urnes)]
D'après l'exercice 9, on a :
$P(W) = 0.4$ (prob. totale de tirer une blanche).
$P(W|U_1) = 0.6$.
$P(U_1) = 0.5$.
On cherche $P(U_1|W)$. Par la règle de Bayes :
$P(U_1|W) = \frac{P(W|U_1)P(U_1)}{P(W)} = \frac{0.6 \times 0.5}{0.4} = \frac{0.3}{0.4} = 0.75$.
Sachant que la boule est blanche, il y a 75\% de chance qu'elle vienne de l'urne U1.
\end{correctionbox}

\begin{correctionbox}[Correction Exercice 14 : Bayes (Spam)]
Soit $S$ "Spam" et $G$ "Contient 'gratuit'".
$P(S) = 0.6$, donc $P(\neg S) = 0.4$.
$P(G|S) = 0.7$.
$P(G|\neg S) = 0.1$.
On cherche $P(S|G)$. Par la règle de Bayes : $P(S|G) = \frac{P(G|S)P(S)}{P(G)}$.

1.  Calculer $P(G)$ (dénominateur) avec la LTP :
    $P(G) = P(G|S)P(S) + P(G|\neg S)P(\neg S)$
    $P(G) = (0.7 \times 0.6) + (0.1 \times 0.4) = 0.42 + 0.04 = 0.46$.

2.  Appliquer la règle de Bayes :
    $P(S|G) = \frac{0.42}{0.46} \approx 0.913$.
    Il y a 91.3\% de chance que l'email soit un spam.
\end{correctionbox}

\begin{correctionbox}[Correction Exercice 15 : Bayes (Usine Inversée)]
D'après l'exercice 10, on a :
$P(D) = 0.032$ (prob. totale d'être défectueux).
$P(D|M_1) = 0.05$.
$P(M_1) = 0.4$.
On cherche $P(M_1|D)$. Par la règle de Bayes :
$P(M_1|D) = \frac{P(D|M_1)P(M_1)}{P(D)} = \frac{0.05 \times 0.4}{0.032} = \frac{0.02}{0.032} = 0.625$.
Sachant que la pièce est défectueuse, il y a 62.5\% de chance qu'elle vienne de M1.
\end{correctionbox}

%  Corrections : Règle de la Chaîne et Problèmes Combinés 

\begin{correctionbox}[Correction Exercice 16 : Règle de la Chaîne (3 Cartes)]
Soit $P_i$ l'événement "tirer un Pique au $i$-ème tirage". Il y a 13 Piques sur 52 cartes.
On cherche $P(P_1 \cap P_2 \cap P_3)$. On utilise la règle de la chaîne :
$P(P_1 \cap P_2 \cap P_3) = P(P_1) \times P(P_2|P_1) \times P(P_3|P_1 \cap P_2)$
$P(P_1) = 13/52$.
$P(P_2|P_1) = 12/51$ (il reste 12 Piques sur 51 cartes).
$P(P_3|P_1 \cap P_2) = 11/50$ (il reste 11 Piques sur 50 cartes).
$P = (13/52) \times (12/51) \times (11/50) = \frac{1}{4} \times \frac{4}{17} \times \frac{11}{50} = \frac{11}{17 \times 50} = 11/850 \approx 0.0129$.
\end{correctionbox}

\begin{correctionbox}[Correction Exercice 17 : Problème de Monty Hall (Calcul)]
On cherche $P(V=1 | H=3)$. On utilise la règle de Bayes :
$P(V=1 | H=3) = \frac{P(H=3 | V=1) P(V=1)}{P(H=3)}$.

*Numérateur :* $P(V=1) = 1/3$. $P(H=3 | V=1)$ est la probabilité que l'animateur ouvre la 3, sachant que vous avez choisi la 1 et que la voiture est en 1. Il peut ouvrir la 2 ou la 3 (deux chèvres). On suppose qu'il choisit au hasard : $P(H=3 | V=1) = 1/2$.
Numérateur = $(1/2) \times (1/3) = 1/6$.

*Dénominateur $P(H=3)$ par LTP (partition sur V) :*
$P(H=3) = P(H=3|V=1)P(V=1) + P(H=3|V=2)P(V=2) + P(H=3|V=3)P(V=3)$
- $P(H=3|V=1) = 1/2$ (calculé ci-dessus).
- $P(H=3|V=2) = 1$ (l'animateur doit ouvrir la 3, car vous avez choisi 1 et la voiture est en 2).
- $P(H=3|V=3) = 0$ (l'animateur ne peut pas ouvrir la porte 3 car elle contient la voiture).
$P(H=3) = (1/2 \times 1/3) + (1 \times 1/3) + (0 \times 1/3) = 1/6 + 1/3 + 0 = 1/2$.

*Résultat :* $P(V=1 | H=3) = \frac{1/6}{1/2} = 1/3$.
(La probabilité que la voiture soit derrière votre porte reste 1/3. La probabilité qu'elle soit derrière l'autre porte fermée (la 2) est $P(V=2|H=3) = 1 - P(V=1|H=3) = 2/3$. Il faut donc changer.)
\end{correctionbox}

\begin{correctionbox}[Correction Exercice 18 : Bayes avec Mise à Jour (Pièce Truquée)]
Soit $A$ "pièce A (équil.)" et $B$ "pièce B (truquée, p=0.8)". $P(A)=P(B)=0.5$.
Soit $E$ l'événement "obtenir Pile, Pile" (PP).
On cherche $P(B|E) = \frac{P(E|B)P(B)}{P(E)}$.

1.  Probabilités conditionnelles de l'évidence $E$ :
    $P(E|A) = P(\text{PP} | A) = 0.5 \times 0.5 = 0.25$ (indépendance des lancers).
    $P(E|B) = P(\text{PP} | B) = 0.8 \times 0.8 = 0.64$.

2.  Calculer $P(E)$ (dénominateur) avec la LTP :
    $P(E) = P(E|A)P(A) + P(E|B)P(B)$
    $P(E) = (0.25 \times 0.5) + (0.64 \times 0.5) = 0.125 + 0.320 = 0.445$.

3.  Appliquer la règle de Bayes :
    $P(B|E) = \frac{P(E|B)P(B)}{P(E)} = \frac{0.64 \times 0.5}{0.445} = \frac{0.32}{0.445} \approx 0.719$.
    Après avoir observé PP, la probabilité que ce soit la pièce truquée passe de 50\% à 71.9\%.
\end{correctionbox}

\begin{correctionbox}[Correction Exercice 19 : Indépendance Conditionnelle (Dés)]
$A = \{D_1=1\}$, $B = \{D_2=1\}$, $C = \{S=2\}$.
On teste si $P(A \cap B | C) = P(A|C) P(B|C)$.

L'événement $C = \{S=2\}$ ne peut se produire que d'une seule façon : $C = \{(1,1)\}$.
Donc, $C$ est l'événement $A \cap B$. $C \subseteq A$ et $C \subseteq B$.

Calculons les termes :
- $P(A|C) = P(A \cap C) / P(C)$. Puisque $C \subseteq A$, $A \cap C = C$.
  $P(A|C) = P(C) / P(C) = 1$.
- $P(B|C) = P(B \cap C) / P(C)$. Puisque $C \subseteq B$, $B \cap C = C$.
  $P(B|C) = P(C) / P(C) = 1$.
- $P(A \cap B | C) = P((A \cap B) \cap C) / P(C)$. Puisque $A \cap B = C$, $(A \cap B) \cap C = C$.
  $P(A \cap B | C) = P(C) / P(C) = 1$.

Test d'indépendance :
$P(A \cap B | C) = 1$.
$P(A|C) P(B|C) = 1 \times 1 = 1$.
Puisque $1=1$, les événements $A$ et $B$ sont bien indépendants conditionnellement à $C$.
*Intuition :* Sachant que la somme est 2, nous savons avec certitude que $D_1=1$ et $D_2=1$. Il n'y a plus d'aléa.
\end{correctionbox}

\begin{correctionbox}[Correction Exercice 20 : Jeu Séquentiel]
Soit $p=1/6$ la probabilité de gagner (obtenir un 6) et $q=5/6$ la probabilité de rater.
Alice gagne si elle réussit au tour 1, OU si (elle rate ET Bob rate) et elle réussit au tour 3, OU si (A rate, B rate, A rate, B rate) et elle réussit au tour 5, etc.

$P(\text{A gagne}) = P(\text{A au tour 1}) + P(\text{A au tour 3}) + P(\text{A au tour 5}) + \dots$
$P(\text{A gagne}) = p + (q \times q)p + (q \times q \times q \times q)p + \dots$
$P(\text{A gagne}) = p + q^2 p + q^4 p + \dots$
$P(\text{A gagne}) = p \times (1 + q^2 + q^4 + \dots)$
$P(\text{A gagne}) = p \sum_{k=0}^{\infty} (q^2)^k$

C'est une série géométrique de premier terme $p$ et de raison $r = q^2 = (5/6)^2 = 25/36$.
La somme est $\frac{\text{premier terme}}{1 - \text{raison}} = \frac{p}{1 - q^2}$.
$P(\text{A gagne}) = \frac{1/6}{1 - 25/36} = \frac{1/6}{11/36} = \frac{1}{6} \times \frac{36}{11} = 6/11$.
\end{correctionbox}

\subsection{Exercices Python}

Ces exercices appliquent les concepts de probabilité conditionnelle, de la règle de Bayes et de la formule des probabilités totales au jeu de données "Titanic".

\begin{codecell}
import pandas as pd
import seaborn as sns
import math

# Charger le dataset Titanic
df = sns.load_dataset("titanic")

# On retire les lignes ou l'age est inconnu pour simplifier les calculs
# C'est notre Univers S.
df = df.dropna(subset=["age"]) 

\end{codecell}

\begin{exercicebox}[Exercice 1 : Définition de $P(A|B)$]
Calculez la probabilité qu'un passager ait survécu ($A$), \textbf{sachant que} ce passager était en première classe ($B$).

\textbf{Votre tâche :}
\begin{enumerate}
    \item Soit $A$ = "le passager a survécu" (\texttt{survived == 1}).
    \item Soit $B$ = "le passager est en première classe" (\texttt{pclass == 1}).
    \item Trouver $|B|$ (le nombre de passagers en 1ère classe).
    \item Trouver $|A \cap B|$ (le nombre de survivants de 1ère classe).
    \item Calculer $P(A|B) = \frac{|A \cap B|}{|B|}$.
\end{enumerate}
\end{exercicebox}

\begin{exercicebox}[Exercice 2 : Règle du Produit (Tirage sans remise)]
On tire au hasard 2 passagers de l'univers \texttt{df} sans remise. Calculez la probabilité que le premier passager soit un survivant ($A_1$) ET que le second passager soit aussi un survivant ($A_2$).

Utilisez la Règle du Produit : $P(A_1 \cap A_2) = P(A_1)P(A_2|A_1)$.

\textbf{Votre tâche :}
\begin{enumerate}
    \item Trouver $|S|$ (total passagers) et $|A_1|$ (total survivants).
    \item Calculer $P(A_1) = |A_1| / |S|$.
    \item Calculer $P(A_2|A_1)$. (Indice : après avoir tiré un survivant, combien de passagers restent ? Combien de survivants restent ?).
    \item Calculer le produit $P(A_1) \times P(A_2|A_1)$.
\end{enumerate}
\end{exercicebox}

\begin{exercicebox}[Exercice 3 : Formule des Probabilités Totales]
Calculez la probabilité totale qu'un passager ait survécu ($B$) en utilisant la formule des probabilités totales. Utilisez la partition des trois classes de passagers ($A_1$=1ère, $A_2$=2e, $A_3$=3e classe).

La formule est : $P(B) = \sum_{i=1}^{3} P(B|A_i)P(A_i)$.

\textbf{Votre tâche :}
\begin{enumerate}
    \item Pour $i=1, 2, 3$, calculer $P(A_i)$, la probabilité d'appartenir à chaque classe (ex: $P(A_1) = |\text{pclass 1}| / |S|$).
    \item Pour $i=1, 2, 3$, calculer $P(B|A_i)$, la probabilité de survie sachant la classe (cf. Exercice 1).
    \item Appliquer la formule : $P(B) = P(B|A_1)P(A_1) + P(B|A_2)P(A_2) + P(B|A_3)P(A_3)$.
    \item (Vérification) Comparez votre résultat au calcul direct $P(B) = |\text{survivants}| / |S|$.
\end{enumerate}
\end{exercicebox}

\begin{exercicebox}[Exercice 4 : Règle de Bayes]
En utilisant les résultats de l'exercice précédent, appliquez la Règle de Bayes.

On observe qu'un passager a survécu ($B$). Quelle est la probabilité qu'il s'agisse d'un passager de première classe ($A_1$) ?

\textbf{Votre tâche :}
\begin{enumerate}
    \item On cherche $P(A_1|B) = \frac{P(B|A_1)P(A_1)}{P(B)}$.
    \item Récupérer $P(B|A_1)$ (la probabilité de survie en 1ère classe) de l'exercice 3.
    \item Récupérer $P(A_1)$ (la probabilité d'être en 1ère classe) de l'exercice 3.
    \item Récupérer $P(B)$ (la probabilité totale de survie) de l'exercice 3.
    \item Effectuer le calcul.
\end{enumerate}
\end{exercicebox}

\begin{exercicebox}[Exercice 5 : Indépendance de deux événements]
Les événements $A$ = "être une femme" (\texttt{sex == 'female'}) et $B$ = "avoir survécu" (\texttt{survived == 1}) sont-ils indépendants ?

\textbf{Votre tâche :}
\begin{enumerate}
    \item Prouver ou réfuter l'indépendance en vérifiant si $P(A \cap B) = P(A)P(B)$.
    \item Calculer $P(A) = |\text{femmes}| / |S|$.
    \item Calculer $P(B) = |\text{survivants}| / |S|$.
    \item Calculer $P(A \cap B) = |\text{femmes survivantes}| / |S|$.
    \item Comparer $P(A \cap B)$ au produit $P(A) \times P(B)$.
    \item (Alternative) Comparer $P(B|A)$ à $P(B)$. L'information "être une femme" change-t-elle la probabilité de survie ?
\end{enumerate}
\end{exercicebox}